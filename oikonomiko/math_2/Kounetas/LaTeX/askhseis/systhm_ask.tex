\input{preamble_ask.tex}
\input{definitions_ask.tex}

\pagestyle{vangelis}



\begin{document}

\begin{center}
  \textcolor{Col1}{\minibox{\large\bfseries{Ασκήσεις στους Πίνακες και τα Συστήματα}}}
\end{center}

\vspace{\baselineskip}

\begin{enumerate}
  \item Οι συναρτήσεις προσφοράς και ζήτησης για δύο αλληλοεξαρτώμενα αγαθά δίνονται από 
    \[
    \sysdelim.\}
    \systeme{
      Q_{D_1} = 50-2P_{1} + P_{2}, 
      Q_{D_2} = 10 + P_{1} - 4P_{2},
      Q_{S_1} = -20 + P_{1},
      Q_{S_2} = -10+5P_{2}
    }
  \] 
  Να διατυπωθεί το σύστημα με αγνώστους τις τιμές ισορροπίας $ P_{1} $ 
  και $ P_{2} $ και στη συνέχεια να λυθεί με τη μέθοδο \textlatin{Crammer} και τη 
  μέθοδο αντίστροφου πίνακα.

  \hfill Απ: $ {P_{1}=25, \; P_{2}=5} $


\item Οι συναρτήσεις προσφοράς και ζήτησης για δύο αλληλοεξαρτώμενα αγαθά δίνονται από 
  \[
  \sysdelim.\}
  \systeme{
    Q_{D_1} = 400-5P_{1} - 3P_{2}, 
    Q_{D_2} = 300 - 2P_{1} - 3P_{2},
    Q_{S_1} = -60 + 3P_{1},
    Q_{S_2} = -100+2P_{2}
  }
\]
\begin{enumerate}[i)]
  \item Δείξτε ότι οι τιμές ισορροπίας αποτελούν λύση του συστήματος
    \[
      \begin{pmatrix}
        8 & 3 \\
        2 & 5
      \end{pmatrix}\cdot 
      \begin{pmatrix}
        P_{1}\\
        P_{2}
      \end{pmatrix}= 
      \begin{pmatrix}
        460 \\
        400
      \end{pmatrix}
    \] 
  \item Χρησιμοποιείστε τον κανόνα του \textlatin{Crammer} για να βρείτε την 
    τιμή ισορροπίας για το αγαθό 1.

    \hfill Απ: $ P_{1}=32 \frac{6}{17} $ 
\end{enumerate}


\item Οι εξισώσεις που προσδιορίζουν ένα υπόδειγμα δύο χωρών που έχουν εμπορικές
  συναλλαγές δίνονται από τις παρακάτω εξισώσεις
  \[
  \sysdelim.\}
  \systeme{
    Y_{1}=C_{1}+I_{1}^*+X_{1}-M_{1}, 
    C_{1}=0.7Y_{1}+50,
    I_{1}^{*}=200,
    M_{1}=0.3Y_{1}
  } \quad \text{και} \quad 
\sysdelim.\}
\systeme{
  Y_{2}=C_{2}+I_{2}^*+X_{2}-M_{2}, 
  C_{2}=0.8Y_{1}+100,
  I_{2}^{*}=300,
  M_{2}=0.1Y_{2}
}
\] 
Εκφράστε αυτό το σύστημα σε μορφή πίνακα και στη συνέχεια βρείτε τις τιμές των 
$ Y_{1} $ και $ Y_{2} $. Υπολογίστε το υπόλοιπο των πληρωμών σε αυτές τις χώρες.    

\hfill \textcolor{Col1}{Υπόδειξη:} Υπόλοιπο πληρωμών $ = M_{1}-X_{1}$ ή $ M_{1}-M_{2} $

\hfill Απ: $ 
\begin{pmatrix*}[r]
  0.6 & -0.1 \\
  -0.3 & 0.6
\end{pmatrix*}\cdot 
\begin{pmatrix}
  Y_{1} \\
  Y_{2}
\end{pmatrix} = 
\begin{pmatrix}
  250 \\
  400
\end{pmatrix}, \; Y_{1}=766.67, \; Y_{2}=2100, \; \text{υπόλοιπο: } 20
$  

\item Οι εξισώσεις που προσδιορίζουν ένα υπόδειγμα δύο χωρών που έχουν εμπορικές
  συναλλαγές δίνονται από τις παρακάτω εξισώσεις
  \[
  \sysdelim.\}
  \systeme{
    Y_{1}=C_{1}+I_{1}^*+X_{1}-M_{1}, 
    C_{1}=0.6Y_{1}+50,
    M_{1}=0.2Y_{1}
  } \quad \text{και} \quad 
\sysdelim.\}
\systeme{
  Y_{2}=C_{2}+I_{2}^*+X_{2}-M_{2}, 
  C_{2}=0.8Y_{1}+80,
  M_{2}=0.1Y_{2}
}
\] 
Αν $ I_{2}^{*} = 70 $ βρείτε την τιμή του $ I_{1}^{*} $ αν το υπόλοιπο πληρωμών 
είναι 0.

\hfill \textcolor{Col1}{Υπόδειξη:} Υπόλοιπο πληρωμών = 0 
$ \Leftrightarrow M_{1}=X_{1}$ ή $ M_{1}=M_{2} $

\hfill \textcolor{Col1}{Υπόδειξη:} κατασκευάστε σύστημα $ 3\times 3 $ με αγνώστους 
$ Y_{1}, Y_{2}, I_{1}^{*} $ 

\sysdelim.\}
\hfill Απ: $ \systeme{
  0.6Y_{1}-0.1Y_{2}-I_{1}^{*}=50,
  -0.2Y_{1}+0.3Y_{2}=150,
  0.2Y_{1}-0.1Y_{2}=0
} $, $ I_{1}^{*}=100 $
\end{enumerate}


\end{document}
