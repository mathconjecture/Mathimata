\documentclass[a4paper,12pt]{article}
\usepackage{etex}
%%%%%%%%%%%%%%%%%%%%%%%%%%%%%%%%%%%%%%
% Babel language package
\usepackage[english,greek]{babel}
% Inputenc font encoding
\usepackage[utf8]{inputenc}
%%%%%%%%%%%%%%%%%%%%%%%%%%%%%%%%%%%%%%

%%%%% math packages %%%%%%%%%%%%%%%%%%
\usepackage{amsmath}
\usepackage{amssymb}
\usepackage{amsfonts}
\usepackage{amsthm}
\usepackage{proof}

\usepackage{physics}

%%%%%%% symbols packages %%%%%%%%%%%%%%
\usepackage{bm} %for use \bm instead \boldsymbol in math mode 
\usepackage{dsfont}
\usepackage{stmaryrd}
%%%%%%%%%%%%%%%%%%%%%%%%%%%%%%%%%%%%%%%


%%%%%% graphicx %%%%%%%%%%%%%%%%%%%%%%%
\usepackage{graphicx}
\usepackage{color}
%\usepackage{xypic}
\usepackage[all]{xy}
\usepackage{calc}
\usepackage{booktabs}
\usepackage{minibox}
%%%%%%%%%%%%%%%%%%%%%%%%%%%%%%%%%%%%%%%

\usepackage{enumerate}

\usepackage{fancyhdr}
%%%%% header and footer rule %%%%%%%%%
\setlength{\headheight}{14pt}
\renewcommand{\headrulewidth}{0pt}
\renewcommand{\footrulewidth}{0pt}
\fancypagestyle{plain}{\fancyhf{}
\fancyhead{}
\lfoot{}
\rfoot{\small \thepage}}
\fancypagestyle{vangelis}{\fancyhf{}
\rhead{\small \leftmark}
\lhead{\small }
\lfoot{}
\rfoot{\small \thepage}}
%%%%%%%%%%%%%%%%%%%%%%%%%%%%%%%%%%%%%%%

\usepackage{hyperref}
\usepackage{url}
%%%%%%% hyperref settings %%%%%%%%%%%%
\hypersetup{pdfpagemode=UseOutlines,hidelinks,
bookmarksopen=true,
pdfdisplaydoctitle=true,
pdfstartview=Fit,
unicode=true,
pdfpagelayout=OneColumn,
}
%%%%%%%%%%%%%%%%%%%%%%%%%%%%%%%%%%%%%%

\usepackage[space]{grffile}

\usepackage{geometry}
\geometry{left=25.63mm,right=25.63mm,top=36.25mm,bottom=36.25mm,footskip=24.16mm,headsep=24.16mm}

%\usepackage[explicit]{titlesec}
%%%%%% titlesec settings %%%%%%%%%%%%%
%\titleformat{\chapter}[block]{\LARGE\sc\bfseries}{\thechapter.}{1ex}{#1}
%\titlespacing*{\chapter}{0cm}{0cm}{36pt}[0ex]
%\titleformat{\section}[block]{\Large\bfseries}{\thesection.}{1ex}{#1}
%\titlespacing*{\section}{0cm}{34.56pt}{17.28pt}[0ex]
%\titleformat{\subsection}[block]{\large\bfseries{\thesubsection.}{1ex}{#1}
%\titlespacing*{\subsection}{0pt}{28.80pt}{14.40pt}[0ex]
%%%%%%%%%%%%%%%%%%%%%%%%%%%%%%%%%%%%%%

%%%%%%%%% My Theorems %%%%%%%%%%%%%%%%%%
\newtheorem{thm}{Θεώρημα}[section]
\newtheorem{cor}[thm]{Πόρισμα}
\newtheorem{lem}[thm]{λήμμα}
\theoremstyle{definition}
\newtheorem{dfn}{Ορισμός}[section]
\newtheorem{dfns}[dfn]{Ορισμοί}
\theoremstyle{remark}
\newtheorem{remark}{Παρατήρηση}[section]
\newtheorem{remarks}[remark]{Παρατηρήσεις}
%%%%%%%%%%%%%%%%%%%%%%%%%%%%%%%%%%%%%%%




\newcommand{\vect}[2]{(#1_1,\ldots, #1_#2)}
%%%%%%% nesting newcommands $$$$$$$$$$$$$$$$$$$
\newcommand{\function}[1]{\newcommand{\nvec}[2]{#1(##1_1,\ldots, ##1_##2)}}

\newcommand{\linode}[2]{#1_n(x)#2^{(n)}+#1_{n-1}(x)#2^{(n-1)}+\cdots +#1_0(x)#2=g(x)}

\newcommand{\vecoffun}[3]{#1_0(#2),\ldots ,#1_#3(#2)}

\newcommand{\mysum}[1]{\sum_{n=#1}^{\infty}


\everymath{\displaystyle}

\thispagestyle{askhseis}

\begin{document}
\begin{center}
  \minibox{\large\bfseries \textcolor{Col1}{Πολικές Καμπύλες}}
\end{center}

\vspace{\baselineskip}


\section*{Εμβαδό Επίπεδων Χωρίων}

\begin{enumerate}
  \item Να υπολογιστεί το εμβαδό του χωρίου που ορίζεται από την καρδιοειδή καμπύλη
    $ r = a(1  + \cos{\theta}) $.

    \hfill Απ: $ \frac{3\pi a^{2}}{2} $

  \item Να υπολογιστεί το εμβαδό του χωρίου που περικλείεται από την καμπύλη $
    r = a \cos{3\theta} $.

    \hfill Απ: $ \frac{\pi a^{2}}{4} $

  \item Να υπολογιστεί το εμβαδό του χωρίου που ορίζεται από την καμπύλη 
    $ r = 4 \sin{3\theta}  $. 

    \hfill Απ: $ 4 \pi $  

  \item Να υπολογιστεί το εμβαδό του χωρίου που περικλείεται από την καμπύλη $
    r = a \sin{2\theta} $.

    \hfill Απ: $ \frac{\pi a^{2}}{2} $

\end{enumerate}



\section*{Μήκος Καμπύλης}


\begin{enumerate}
  \item Να υπολογιστεί το μήκος της καρδιοειδούς καμπύλης $ r = a(1 + \cos{\theta}), 
    \; a>0 $.

    \hfill Απ: $ 8 a $

  \item Να υπολογιστεί το μήκος της καμπύλης 
    $ r = a\sin^{3}{\left(\frac{\theta}{3}\right)}, \; a>0 $.

    \hfill Απ: $ \frac{3 \pi a}{2} $ 
\end{enumerate}




\section*{Εμβαδό Επιφάνειας απο Περιστροφή}

\begin{enumerate}
  \item Να υπολογιστεί το εμβαδο της επιφάνειας του στερεού που προκύπτει από την
    περιστροφή της καμπύλης $ r = a(1 - \cos{\theta}) $ γύρω από τον πολικό
    άξονα. 

    \hfill Απ: $ \frac{32 \pi  a^{2}}{5} $
\end{enumerate}

\end{document}
