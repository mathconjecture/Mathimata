\input{preamble/preamble.tex}
\newcommand{\vect}[2]{(#1_1,\ldots, #1_#2)}
%%%%%%% nesting newcommands $$$$$$$$$$$$$$$$$$$
\newcommand{\function}[1]{\newcommand{\nvec}[2]{#1(##1_1,\ldots, ##1_##2)}}

\newcommand{\linode}[2]{#1_n(x)#2^{(n)}+#1_{n-1}(x)#2^{(n-1)}+\cdots +#1_0(x)#2=g(x)}

\newcommand{\vecoffun}[3]{#1_0(#2),\ldots ,#1_#3(#2)}

\newcommand{\suma}{\sum_{n=0}^{\infty}a_n x^n}

\newcommand{\sumb}{\sum_{n=1}^{\infty}a_n n x^{n-1}}

\newcommand{\sumc}{\sum_{n=2}^{\infty}a_n n (n-1) x^{n-2}}

\newcommand{\varsum}[2]{\sum_{n=#1}^{#2}}

\thispagestyle{empty}

\begin{document}

\begin{center}
    \minibox[frame]{\large\bfseries Ασκήσεις στο Άθροισμα Διανυσματικών Χώρων} 
\end{center}

\vspace{\baselineskip}

\begin{enumerate}
    \item Έστω $ V = \mathbb{R}^{3} $ και $ W_{1} = \{(x_{1},0,x_{2})\in 
        \mathbb{R}^{3} \mid x_{1}, x_{2} \in \mathbb{R}\}  $ και 
        $ W_{2} = \{(0,y_{1},y_{2}) \in \mathbb{R}^{3} \mid y_{1}, y_{2} \in
        \mathbb{R}  \}  $. Να δείξετε ότι 
        \begin{enumerate}[i)]
            \item $ W_{1} \leq \mathbb{R}^{3} $
            \item $ W_{2} \leq \mathbb{R}^{3} $
            \item $ \mathbb{R}^{3} = W_{1}+W_{2} $
        \end{enumerate}

    \item Έστω $ V = \mathbb{R}^{3} $ και οι υπόχωροί του $ W_{1} = 
        \{ (x,0,0) \in \mathbb{R}^{3} \mid x \in \mathbb{R}\} $, $ W_{2} = 
        \{ (0,0,z) \mid z \in \mathbb{R} \} $ και 
        $ W_{3} = \{ (x,y,x) \mid x,y \in \mathbb{R} \} $. Να δείξετε ότι 
        \begin{enumerate}[i)]
            \item $ W_{1} \cap W_{2} = W_{2} \cap W_{3} = W_{3} \cap W_{1} = 
                \{ \mathbf{0} \} $
            \item $ \mathbb{R}^{3} = W_{1}+W_{2}+W_{3} $
            \item Να εξετάσετε αν $ \mathbb{R}^{3} = W_{1} \oplus W_{2} \oplus W_{3} $
        \end{enumerate}

    \item Έστω διανυσματικός χώρος $V$ και $ W_{1}, W_{2} $ υπόχωροί του, τέτοιοι ώστε 
        $ V = W_{1} \oplus W_{2} $. Αν $ W_{1} \leq W \leq V $, να δείξετε ότι 
        $ W = W_{1} \oplus (W_{2} \cap W_{3}) $.

        \hfill \textbf{Υπόδειξη:} Για να δείξετε ότι $ W = W_{1} + (W_{2}\cap W_{3}) $, 
        δείξτε ότι $ W \leq W_{1} + (W_{2} \cap W_{3}) $ και 
        $ W_{1} + (W_{2} \cap W_{3}) \leq W $
\end{enumerate}

\end{document}


