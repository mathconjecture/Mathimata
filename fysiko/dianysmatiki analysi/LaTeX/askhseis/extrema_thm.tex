\input{preamble.tex}
\input{definitions_ask.tex}

\pagestyle{askhseis}

\renewcommand{\vec}{\mathbf}

\begin{document}

\begin{center}
  \minibox{\large \bfseries \textcolor{Col1}{θέματα στα Ακρότατα Συναρτήσεων πολλών
  Μεταβλητών}}
\end{center}

\vspace{\baselineskip}

\section*{Δυο Μεταβλητών}

\begin{enumerate}
  \item \textbf{(Ιούν. 2019)} Να βρεθούν τα τοπικά ακρότατα της συνάρτησης 
    $ f(x,y) = x^{4}+y^{4}-4xy+1 $.

    \hfill Απ: $ f_{\rm{suddle}}(0,0) = 1, \; f_{\min}(1,1)=-1, \; f_{\min}(-1,-1)=-1 $

  \item \textbf{(Σεπ 2014)} Χαρακτηρίστε τα κρίσιμα σημεία της συνάρτησης 
    $ f(x,y) = x^{3}+y^{2} -6xy+6x+3y $.

    \hfill Απ: $ f_{\rm{suddle}}(1,3/2) = 19/4, \; f_{\min}(5,27/2) = -109/4 $ 
\end{enumerate}

\section*{Ακρότατα σε κλειστή και φραγμένη περιοχή (Απόλυτα Ακρότατα)}



\end{document}
