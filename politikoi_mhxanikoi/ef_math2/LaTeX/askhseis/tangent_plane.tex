\input{preamble_ask.tex}
\input{definitions_ask.tex}

\geometry{top=2cm}
\pagestyle{askhseis}
\everymath{\displaystyle}


\begin{document}

\begin{center}
  {\color{Col1}\bfseries\large Ασκήσεις}
\end{center} 


 

\section*{Εφαπτόμενο Επίπεδο - Κάθετη Ευθεία}

\begin{enumerate}
  \item Να υπολογίσετε το εφαπτόμενο επίπεδο και την κάθετη ευθεία της επιφάνειας 
    $ x^{2}+2y^{2}-z = 0 $ στο σημείο $ (1,1,3) $.

    \hfill Απ: \begin{tabular}{l}
      $ \Pi : 2x+4y-z -3 = 0 $ \\
      $ K : \frac{x-1}{2} = \frac{y-1}{4} = \frac{z-3}{-1} $
    \end{tabular} 

  \item Να υπολογίσετε το εφαπτόμενο επίπεδο και την κάθετη ευθεία της επιφάνειας 
    $ z = xy $ στο σημείο $ (2,3,6) $.

    \hfill Απ: \begin{tabular}{l}
      $ \Pi : 3x+2y-z-6 = 0 $ \\
      $ K : \frac{x-2}{3} = \frac{y-3}{2} = \frac{z-6}{-1} $
    \end{tabular} 

  \item Να υπολογίσετε τις εξισώσεις του εφαπτόμενου επιπέδου και την κάθετης ευθείας 
    της επιφάνειας $ x^{2} + y^{2} + z = 9 $ στο σημείο $ P(1,2,4) $.

    \hfill Απ: \begin{tabular}{l}
      $ \Pi : 2x+4y+z-14=0 $ \\
      $ K : \frac{x-1}{2} = \frac{y-2}{4} = z-4 $
    \end{tabular} 

  \item Να υπολογίσετε τις εξισώσεις του εφαπτόμενου επιπέδου και την κάθετης ευθείας 
    της επιφάνειας $ 2x^{2} + 2y^{2} - z^{2} +12=0 $ στο σημείο $ P(1,-1,4) $.

    \hfill Απ: \begin{tabular}{l}
      $ \Pi : x-y-2z+6=0 $ \\
      $ K : \frac{x-1}{-4} = \frac{y+1}{4} = \frac{z-4}{8} $
    \end{tabular} 

  \item Να υπολογίσετε το εφαπτόμενο επίπεδο και την κάθετη ευθεία της επιφάνειας 
    $ \frac{x^{2}}{a^{2}} + \frac{y^{2}}{b^{2}} + \frac{z^{2}}{c^{2}} = 1  $ στο σημείο 
    $ (1,1,2) $.

    \hfill Απ: \begin{tabular}{l}
      $ \Pi : \frac{2}{a^{2}} x+ \frac{2}{b^{2}} y+ \frac{4}{c^{2}} z- 
      \left(\frac{2}{a^{2}}+ \frac{2}{b^{2}}+ \frac{8}{c^{2}}\right) = 0 $ \\
      $ K : \frac{x-1}{2/a^{2}} = \frac{y-1}{2/b^{2}} = \frac{z-2}{4/c^{2}} $
    \end{tabular} 

  \item Να προσδιορίσετε τα σημεία της επιφάνειας $ x^{3}+y^{3}+z-3xy=0 $ στα οποία 
    το εφαπτόμενο επίπεδο είναι οριζόντιο.
    \hfill (\textbf{υπόδειξη:} $\Pi$ οριζόντιο $ \Leftrightarrow \grad f = \mathbf{0} $)

    \hfill Απ: $ (0,0,0) $ και $ (1,1,1) $ 
\end{enumerate}

\end{document}
