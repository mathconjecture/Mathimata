\documentclass[a4paper,12pt]{article}


\usepackage[english,greek]{babel}
\usepackage[utf8]{inputenc}

\usepackage{amsmath}
\usepackage{amssymb}
\usepackage{amsthm}
\usepackage{amsfonts}

\usepackage[margin=2cm]{geometry}
\usepackage{outlines}

\usepackage[margin=2cm]{geometry}


\begin{document}


\begin{center}
\fbox{\Large\bfseries{Ασκήσεις στον Απλό Τόκο}}
\end{center}

\vspace{1cm}

\begin{enumerate}
\item Αρχικό κεφάλαιο 2500\euro\ τοκίζεται σήμερα στην Τράπεζα Δ με απλό τόκο για 3 έτη και επιτόκιο 2\%. Ζητείται ο τόκος και η μέλλουσα αξία του αρχικού κεφαλαίου.

\hspace{.15\textwidth}Λύση

\vspace{3cm}

\item Αρχικό κεφάλαιο 4500\euro\ τοκίζεται σήμερα στην Τράπεζα Η με απλό τόκο για 9 μήνες και επιτόκιο 2\%. Ζητείται ο τόκος του αρχικού κεφαλαίου.

\hspace{.15\textwidth}Λύση

\vspace{3cm}

\item Αρχικό κεφάλαιο 15000\euro\ τοκίζεται σήμερα στην Τράπεζα Ω για 6 μήνες και 20 ημέρες. Το επιτόκιο υπολογίζεται στη βάση εμπορικού έτους και είναι 1,5\%. Ζητείται ο τόκος του αρχικού κεφαλαίου. 

\hspace{.15\textwidth}Λύση

\vspace{3cm}

\item Αρχικό κεφάλαιο 15000\euro\ τοκίζεται την 10/02/2015 στην τράπεζα Η απλό τόκο και λήγει την 22/12/2015. Το επιτόκιο υπολογίζεται στη βάση μικτού έτους και είναι 1,5\%. Ζητείται ο τόκος του αρχικού κεφαλαίου. 

\hspace{.15\textwidth}Λύση

\vspace{3cm}

\newpage

\item Αρχικό κεφάλαιο 200000\euro\ τοκίζεται την 06/04/2015 στην τράπεζα Β με απλό τόκο και λήγει την 19/08/2015. Το επιτόκιο υπολογίζεται στη βάση μικτού έτους και είναι 2,5\%. Ζητείται ο τόκος του αρχικού κεφαλαίου μετά την επιβολή φόρου 15\%.

\hspace{.15\textwidth}Λύση

\vspace{3cm}

\item Ένας καταθέτης διαπραγματεύεται με την τράπεζα Ε την κατάθεση ποσού 55000\euro\ για 2 χρόνια. Η τράπεζα του προσφέρει το πρόγραμμα απλού τόκου ως εξής. Για τα πρώτα 10000\euro\ επιτόκιο 2\%, για το ποσό από 10000 έως 29000 επιτόκιο 3\% και για το υπόλοιπο ποσό επιτόκιο 3,5\%. Ζητείται το καθαρό ποσό των τόκων που θα εισπράξει ο καταθέτης μετά την παρακράτηση φόρου καταθέσεων 15\%. 

\hspace{.15\textwidth}Λύση

\vspace{4cm}

\item Αρχικό κεφάλαιο 40000\euro\ τοκίζεται σήμερα στην τράπεζα Η με απλό τόκο για 18 μήνες. Το επιτόκιο για τον πρώτο χρόνο είναι 1,5\% και για το υπόλοιπο μέχρι τη λήξη 2\%. Ζητείται ο τόκος του αρχικού κεφαλαίου. 

\hspace{.15\textwidth}Λύση

\vspace{3cm}

\item Κεφάλαιο 140000\euro\ τοκίζεται με απλό τόκο και ετήσιο επιτόκιο 1,75\% για διάστημα 18 μηνών. Να υπολογιστεί ο τόκος του κεφαλαίου στη βάση μικτού έτους με τη μέθοδο του τοκαρίθμου.

\hspace{.15\textwidth}Λύση

\vspace{3cm}

\newpage

\item Ο πελάτης Α έχει προυπολογίσει τις επόμενες καταθέσεις στη διάρκεια του δίσεκτου έτους 201Χ στην τράπεζα Ω. Την 10/03 θα καταθέσει 60000\euro\ την 25/05 80000\euro\ και την 18/08 50000\euro. Το τραπεζικό επιτόκιο που πέτυχε στη βάση πολιτικού έτους ήταν 8\%. Στις 24/12 θα αποσύρει το ποσό και τους τόκους προκειμένου ν' αγοράσει μια κατοικία αξίας 198000\euro\ και γι' αυτό απευθύνεται στο γραφείο σας να τον συμβουλέψετε αν το κεφάλαιο μαζί με τους τόκους μετά τη φορολόγησή τους (15\%) επαρκούν για την αγορά που έχει προγραμματίσει. Συμβουλέψτε τον πελάτη σας εφαρμόζοντας τη μέθοδο του τοκαρίθμου.

\hspace{.15\textwidth}Λύση

\vspace{5cm}

\begin{center}
\fbox{\bfseries Μέσο Επιτόκιο}
\end{center}

\item Η τράπεζα Ε δέχεται από τον πελάτη της Νικολάου κατάθεση όψης 50000\euro\ τοκοφόρα για 180 ημέρες, απλού τόκου, με επιτόκιο χορήγησης 2,2\% και κατάθεση προθεσμιακού λογαριασμού 40000\euro\, τοκοφόρα για 630 ημέρες, με απλό τόκο και προνομιακό επιτόκιο 3,6\%. Ποιο είναι το μέσο επιτόκιο χορηγήσεων για τον πελάτη της Νικολάου. (έτος μικτό)

\hspace{.15\textwidth}Λύση

\vspace{3cm}

\item Η επιχείρηση ΗΛΙΟΣ υπογράφει δανειακή σύμβαση με την τράπεζα Ζ για το ποσό των 260000\euro\ με τη συμφωνία να προκαταβληθούν οι τόκοι του δανείου. Το επιτόκιο της δανειακής σύμβασης είναι 5,8\% με απλό τόκο και η διάρκεια αποπληρωμής 2,6 έτη. Τα έξοδα φακέλου είναι 1\textperthousand. Ποιο ποσό θα εισπράξει η επιχείρηση ΗΛΙΟΣ με την υπογραφή της δανειακής σύμβασης; (έτος εμπορικό)

\hspace{.15\textwidth}Λύση

\vspace{3cm}

\newpage

\item Η επιχείρηση ΕΡΜΗΣ υπογράφει δανειακή σύμβαση με την τράπεζα Κ για ποσό 500000\euro\ με τη συμφωνία να προκαταβληθούν οι τόκοι του δανείου. Το επιτόκιο της δανειακής σύμβασης είναι 6,4\% με απλό τόκο και η διάρκεια αποπληρωμής 1 έτος και 45 ημέρες. Τα έξοδα φακέλου είναι 1\textperthousand. Ποιο ποσό θα εισπράξει η επιχείρηση με την υπογραφή της δανειακής σύμβασης?

\hspace{.15\textwidth}Λύση

\vspace{4cm}

\begin{center}
\fbox{\bfseries Προεξόφληση}
\end{center}

\item Συναλλαγματική ονομαστικής αξίας 2400\euro\ προεξοφλείται 55 ημέρες πριν τη λήξη της. Το προεξοφλητικό επιτόκιο είναι 4,8\%. Ποιο είναι το προεξόφλημα και ποια η πραγματική αξία της συναλλαγματικής? (έτος μικτό)

\hspace{.15\textwidth}Λύση

\vspace{3cm}

\item Συναλλαγματική ονομαστικής αξίας 200\euro\ προεξοφλείται 50 ημέρες πριν τη λήξη της με εξωτερική προεξόφληση αποδίδοντας πραγματική αξία 197\euro. Ποιο είναι το προεξοφλητικό επιτόκιο και το προεξόφλημα? (έτος εμπορικό)

\hspace{.15\textwidth}Λύση

\vspace{3cm}

\item Μια συναλλαγματική έκδοσης 12/02/2015 και λήξης 30/08/2015 ονομαστικής αξίας 600\euro\ ζητείται να προεξοφληθεί εσωτερικά στις 12/06/2015. Το προεξοφλητικό επιτόκιο είναι 8\%. Ποια είναι η πραγματική αξία της συναλλαγματικής και πόσο είναι το προεξόφλημα? (έτος μικτό)

\hspace{.15\textwidth}Λύση

\vspace{3cm}

\item Μια συναλλαγματική προεξοφλείται δύο μήνες πριν τη λήξη της εσωτερικά με προεξοφλητικό επιτόκιο 6,5\% και αποφέρει πραγματική αξία 1000\euro. Ποια είναι η ονομαστική αξία της συναλλαγματικής και πόσο είναι το προεξόφλημα; (έτος μικτό)

\hspace{.15\textwidth}Λύση

\vspace{3cm}

\item Ο επιχειρηματίας Ανδρέου έχει στο χαρτοφυλάκιό του συναλλαγματική που υπέγραψε ο χρεώστης Νιάρος την 10/02/2015 ονομαστικής αξίας 13200\euro\ και λήξης την 15/09/2015. Ο Ανδρέου, στις 15/05/2015 έχει ανάγκη 12861,75\euro\ σε μετρητά για να εξοφλήσει οφειλή του προς το Δημόσιο και επιθυμεί να προεξοφλήσει τη συναλλαγματική στην τράπεζα Δ η οποία καθορίζει προεξοφλητικό επιτόκιο 7,5\%. Επαρκούν τα χρήματα που θα λάβει ο επιχειρηματίας Ανδρέου από την εξωτερική προεξόφληση της συναλλαγματικής? Θα του περισσέψουν χρήματα ή όχι και ποσα?

\hspace{.15\textwidth}Λύση

\vspace{4cm}

\item Ο Ανδρέου πρέπει να πληρώσει μια συναλλαγματική ονομαστικής αξίας 4700\euro\ που λήγει σε 4 μήνες. Το ποσό θα το δανειστεί από την τράπεζα Ω σήμερα με επιτόκιο 6,5\%, αλλά η τράπεζα ζητά προκαταβολικά τους τόκους του δανείου. Ποιο είναι το ποσό του δανείου?

\hspace{.15\textwidth}Λύση

\vspace{3cm}

\item Ο επιχειρηματίας Σπύρου έχει μεταφέρει μέρος του λογαριασμού όψης που διατηρεί στην τράπεζα Β προκειμένου να επωφεληθεί από το πρόγραμμα καταθέσεων ΔΗΛΟΣ της τράπεζας. Το πρόγραμμα ΔΗΛΟΣ είναι απλού τόκου και αποδίδει επιτόκιο 3\% για τρίμηνη διάρκεια, 4\% για διάστημα 4 έως 8 μήνες και 4,5\% για διάστημα από 9 έως 16 μήνες. Σήμερα, μετά από 16 μήνες συμμετοχής του στο πρόγραμμα ΔΗΛΟΣ εισπράττει από το ταμείο της τράπεζας τόκους και κεφάλαιο 6200\euro. Πόσο κεφάλαιο επένδυσε?

\hspace{.15\textwidth}Λύση

\vspace{3cm}

\item Μια επιχείρηση προχώρησε σε σύναψη μικρού δανείου απλού τόκου σε ένα υπάλληλό της. Για το 75\% του ποσού που δάνεισε ζήτησε επιτόκιο 5\% και ήταν διάρκειας 12 μηνών. Το υπόλοιπο του δανείου σημφωνήθηκε να είναι διάρκειας 18 μηνών με επιτόκιο 4,5\%. Μετά από τη λήξη της περιόδου τοκοφορίας η επιχείρηση εισέπραξε τόκους 1200\euro. Ποιο είναι το κεφάλαιο που δάνεισε η επιχείρηση?

\hspace{.15\textwidth}Λύση

\vspace{3cm}

\item Δυο διαφορετικά ποσά κατατέθηκαν την ίδια ημέρα στις τράπεζες Β και Δ με απλό τόκο και τοκοφόρα περίοδο 6 μηνών. Η τράπεζα Β έδωσε επιτόκιο 4\% και η Δ 5\%, ενώ ο συνολικός τόκος που απέδωσαν και τα δύο ποσά ήταν 600\euro. Η Δ απέδωσε τόκο 40\euro\ λιγότερο από τη Β. Ποια ποσά κατατέθηκαν στις δύο τράπεζες?

\hspace{.15\textwidth}Λύση

\vspace{3cm}

\item Μια συναλλαγματική προεξοφλείται 50 ημέρες πριν ληξει. Η διαφορά της εσωτερικής και εξωτερικής προεξόφλησης είναι 2\euro. Το επιτόκιο προεξόφλησης είναι 9\%. Ποια είναι η ονομαστική αξία της συναλλαγματικής?

\hspace{.15\textwidth}Λύση

\vspace{3cm}


\item Συναλλαγματική αξίας 2500\euro\ προεξοφλείται εξωτερικά 65 ημέρες πριν λήξει. Το προεξοφλητικό επιτόκιο είναι 8\%. Η τράπεζα κράτησε προμήθεια 0,4\% για κάθε μήνα και ολόκληρο μήνα, έξοδα 0,15\% και ΕΦΤΕ 1,5\euro\ για κάθε χιλιάδα και ολόκληρη χιλιάδα. Τι ποσό θα αποδοθεί στον δικαιούχο?

\hspace{.15\textwidth}Λύση

\vspace{3cm}

\newpage

\item Ο έμπορος Ανδρέου εξοφλεί τον προμηθευτή Ιωάννου αποδεχόμενος συναλλαγματική λήξης 22/07 για αγορά εμπορευμάτων. Ο Ιωάννου προεξοφλεί εξωτερικά τη συναλλαγματική στην τράπεζα Η στις 12/06 με επιτόκιο 10\% και εισπράττει 12500\euro. Πόσο κοστίζουν τα εμπορεύματα του Ανδρέου? (έτος πολιτικό)

\hspace{.15\textwidth}Λύση

\vspace{3cm}

\end{enumerate}

\end{document}