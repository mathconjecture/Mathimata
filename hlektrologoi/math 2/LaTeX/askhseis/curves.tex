\documentclass[a4paper,12pt]{article}
\usepackage{etex}
%%%%%%%%%%%%%%%%%%%%%%%%%%%%%%%%%%%%%%
% Babel language package
\usepackage[english,greek]{babel}
% Inputenc font encoding
\usepackage[utf8]{inputenc}
%%%%%%%%%%%%%%%%%%%%%%%%%%%%%%%%%%%%%%

%%%%% math packages %%%%%%%%%%%%%%%%%%
\usepackage{amsmath}
\usepackage{amssymb}
\usepackage{amsfonts}
\usepackage{amsthm}
\usepackage{proof}

\usepackage{physics}

%%%%%%% symbols packages %%%%%%%%%%%%%%
\usepackage{bm} %for use \bm instead \boldsymbol in math mode 
\usepackage{dsfont}
\usepackage{stmaryrd}
%%%%%%%%%%%%%%%%%%%%%%%%%%%%%%%%%%%%%%%


%%%%%% graphicx %%%%%%%%%%%%%%%%%%%%%%%
\usepackage{graphicx}
\usepackage{color}
%\usepackage{xypic}
\usepackage[all]{xy}
\usepackage{calc}
\usepackage{booktabs}
\usepackage{minibox}
%%%%%%%%%%%%%%%%%%%%%%%%%%%%%%%%%%%%%%%

\usepackage{enumerate}

\usepackage{fancyhdr}
%%%%% header and footer rule %%%%%%%%%
\setlength{\headheight}{14pt}
\renewcommand{\headrulewidth}{0pt}
\renewcommand{\footrulewidth}{0pt}
\fancypagestyle{plain}{\fancyhf{}
\fancyhead{}
\lfoot{}
\rfoot{\small \thepage}}
\fancypagestyle{vangelis}{\fancyhf{}
\rhead{\small \leftmark}
\lhead{\small }
\lfoot{}
\rfoot{\small \thepage}}
%%%%%%%%%%%%%%%%%%%%%%%%%%%%%%%%%%%%%%%

\usepackage{hyperref}
\usepackage{url}
%%%%%%% hyperref settings %%%%%%%%%%%%
\hypersetup{pdfpagemode=UseOutlines,hidelinks,
bookmarksopen=true,
pdfdisplaydoctitle=true,
pdfstartview=Fit,
unicode=true,
pdfpagelayout=OneColumn,
}
%%%%%%%%%%%%%%%%%%%%%%%%%%%%%%%%%%%%%%

\usepackage[space]{grffile}

\usepackage{geometry}
\geometry{left=25.63mm,right=25.63mm,top=36.25mm,bottom=36.25mm,footskip=24.16mm,headsep=24.16mm}

%\usepackage[explicit]{titlesec}
%%%%%% titlesec settings %%%%%%%%%%%%%
%\titleformat{\chapter}[block]{\LARGE\sc\bfseries}{\thechapter.}{1ex}{#1}
%\titlespacing*{\chapter}{0cm}{0cm}{36pt}[0ex]
%\titleformat{\section}[block]{\Large\bfseries}{\thesection.}{1ex}{#1}
%\titlespacing*{\section}{0cm}{34.56pt}{17.28pt}[0ex]
%\titleformat{\subsection}[block]{\large\bfseries{\thesubsection.}{1ex}{#1}
%\titlespacing*{\subsection}{0pt}{28.80pt}{14.40pt}[0ex]
%%%%%%%%%%%%%%%%%%%%%%%%%%%%%%%%%%%%%%

%%%%%%%%% My Theorems %%%%%%%%%%%%%%%%%%
\newtheorem{thm}{Θεώρημα}[section]
\newtheorem{cor}[thm]{Πόρισμα}
\newtheorem{lem}[thm]{λήμμα}
\theoremstyle{definition}
\newtheorem{dfn}{Ορισμός}[section]
\newtheorem{dfns}[dfn]{Ορισμοί}
\theoremstyle{remark}
\newtheorem{remark}{Παρατήρηση}[section]
\newtheorem{remarks}[remark]{Παρατηρήσεις}
%%%%%%%%%%%%%%%%%%%%%%%%%%%%%%%%%%%%%%%




\input{definitions_ask.tex}


\usepackage{array}
\pagestyle{askhseis}
\everymath{\displaystyle}

\begin{document}

\begin{center}
  \minibox{\large \bfseries \textcolor{Col1}{Καμπύλες (Ασκήσεις)}}
\end{center}

\vspace{\baselineskip}

\begin{enumerate}
  \item Έστω ένα σώμα διαγράφει ελικοειδή τροχιά προς τα πάνω έχοντας διάνυσμα θέσεως 
    $\mathbf{r}(t) = 3\cos t \mathbf{i} + 3\sin t \mathbf{j} + t^2 \mathbf{k}$. 
    Να βρεθούν:
    \begin{enumerate}[i)]
      \item Τα διανύσματα $\vb{v}, \vb{a}$.
      \item Το μέτρο $\abs{\vb{v}}$ την τυχαία χρονική στιγμή $t$.
      \item Τη χρονική στιγμή κατά την οποία $\vb{v}\perp\vb{a}$
    \end{enumerate}
    \hfill Απ: \begin{tabular}{l}
      $ \mathrm{i)} \quad 
      \mathbf{v}(t) = -3\sin{t} \mathbf{i} + 3\cos t \mathbf{j} + 2t \mathbf{k} $ \\
      $\phantom{\mathrm{i)}} \quad 
      \mathbf{a}(t) = -3\cos t \mathbf{i} + (-3)\sin t \mathbf{j} + 2 \mathbf{k} $ \\
      $ \mathrm{ii)} \quad \norm{\mathbf{v}(t)} = \sqrt{9+4t^{2}} $ \\
      $ \mathrm{iii)} \quad t=0$
    \end{tabular}

  \item Θεωρούμε την καμπύλη $ \mathbf{r}(t) = t \mathbf{i} + 2 \mathbf{j} + (t^{2}-3)
    \mathbf{k}, \; t \in \mathbb{R} $. Να βρεθεί η εξίσωση της εφαπτομένης της καμπύλης 
    στο σημείο για $ t = 2 $.
    \hfill Απ: $ \mathbf{R}(t) = (2+t) \mathbf{i} + 2 \mathbf{j} + (1+4t) \mathbf{k}$ 

  \item Θεωρούμε την καμπύλη $ \mathbf{r}(t) = t \mathbf{i} + t \mathbf{j} + t^{2}
    \mathbf{k}, \; t \in \mathbb{R} $. Να βρεθεί η εξίσωση της εφαπτομένης της καμπύλης 
    στο σημείο $ (1,1,1) $.
    \hfill Απ: $ \mathbf{R}(t) = (1+t) \mathbf{i} + (1+t) \mathbf{j} + (1+2t) 
    \mathbf{k} $ 

  \item Ένα σώμα κινείται επί της ελικοειδούς καμπύλης 
    $\mathbf{r}(t) = \cos t \mathbf{i} + \sin t \mathbf{j} + t \mathbf{k}$. Πόσο 
    είναι το μήκος της καμπύλης από $t=0$ έως $t=2\pi$.  
    \hfill Απ: $s=2\sqrt{2}\pi$

  \item Αν η επιτάχυνση ενός σώματος δίνεται από την εξίσωση 
    $\mathbf{a}(t) = 2 \mathbf{j} + 6t \mathbf{k}$ τότε να βρείτε την 
    εξίσωση της ταχύτητας και της θέσης του σώματος αν είναι γνωστή η αρχική ταχύτητα 
    $\vb{v}(0)=\vb{j}-\vb{k}$ και η αρχική θέση $\vb{r}(0)=\vb{i}-2\,\vb{j}+3\,\vb{k}$.

    \hfill Απ: \begin{tabular}{l}
      $ \mathbf{v}(t)=t\,\mathbf{i}+(2t+1)\,\mathbf{j}+(3t^{2}-1)\,\mathbf{k} $ \\
      $ \mathbf{r}(t)=\left(\frac{1}{2}t^{2}+1\right)\,\mathbf{i}+(t^{2}+t-2)\,
      \mathbf{j}+(t^{3}-t+3)\,\mathbf{k} $
    \end{tabular}

  \item Ένα σωμάτιο κινείται σε επίπεδη τροχιά με παραμετρικές εξισώσεις 
    $x(t)=t\sin t + \cos t$ και $y(t)=-t\cos t+\sin t$, όπου $t$ είναι ο χρόνος.
    \begin{enumerate}[i)]
      \item Να βρεθεί το μήκος $s$ της τροχιάς στο διάστημα $t=0$ έως $t=2$.
      \item Να βρεθούν η εφαπτόμενη (επιτρόχιος) και η κάθετη (κεντρομόλος) 
        συνιστώσα της επιτάχυνσης όταν $t=2$.
      \item Να βρεθούν οι συντεταγμένες του κέντρου καμπυλότητας 
        του κέντρου καμπυλότητας όταν $t=2$.
    \end{enumerate}

    \hfill Απ: $\begin{tabular}{ll}
      $\rm{i)}$ & $ s=2 $ \\
      $ \rm{ii)} $ & $\vb{\alpha}=\vb{T}+t\vb{N} $ \\
      $ \rm{iii)} $ & $ K(\cos t, \sin t),\; t=2 $
    \end{tabular}$
\end{enumerate}



\end{document}
