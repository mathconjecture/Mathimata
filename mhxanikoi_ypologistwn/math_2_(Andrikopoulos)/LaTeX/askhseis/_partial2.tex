\input{preamble_ask.tex}
\input{definitions_ask.tex}


\pagestyle{askhseis}
\everymath{\displaystyle}

\begin{document}

\begin{center}
  \minibox{\large\bfseries \textcolor{Col1}{Ασκήσεις στις Μερικές Παραγώγους 2}}
\end{center}

\section*{Ορισμός - Διαφορισιμότητα}

\begin{enumerate}
  \item Έστω η συνάρτηση $ f(x,y) = xy $. Να υπολογιστούν με τοv ορισμό οι 
    $ f_{x}(1,1) $ και η $ f_{y}(1,1) $. 
    \hfill Απ: 
    \begin{tabular}{l}
      $f_{x}(1,1) = 1$ \\
      $f_{y}(1,1) = 1$
    \end{tabular} 


  \item Να εξετάσετε αν η συνάρτηση 
    $
    f(x,y) = 
    \begin{cases} 
      xy \frac{x^{2}-y^{2}}{x^{2}+y^{2}}, & (x,y) 
      \neq (0,0) \\ 
      0, & (x,y) = (0,0)
    \end{cases}
    $ 
    είναι διαφορίσιμη στο $ (0,0) $. 
    \hfill Απ: ναι 
\end{enumerate}

\section*{Κλίση - Λαπλασιανή - Αρμονικές Συναρτήσεις}

\begin{enumerate}
  \item Να υπολογίσετε την κλίση των παρακάτω συναρτήσεων.
    \begin{enumerate}[i)]
      % \item $ f(x,y) = x^{2} \mathrm{e}^{y+x^{3}} $ 
      %   \hfill Απ: $ \grad(f) = (2x\mathrm{e}^{y+x^{3}}+3x^{4}\mathrm{e}^{y+x^{3}}, 
      %   x^{2} \mathrm{e}^{y+x^{3}}) $ 
      \item $ f(x,y) = \frac{1}{y+x^{3}} $ 
        \hfill Απ: $\grad(f) =\left(\frac{-3x^{2}}{(y+x^{3})^{2}},
        \frac{-1}{(y+x^{3})^{2}}\right)$ 
      \item $ f(x,y) = x^{2}y^{3}z $   
        \hfill Απ: $ \grad(f) = (2xy^{3}z,3xy^{2}z,x^{2}y^{3}) $ 
      \item $ f(x,y) = x^{2} y \mathrm{e}^{zy}  $ 
        \hfill Απ: $ \grad(f) = (2xy\mathrm{e}^{zy},
        x^{2}\mathrm{e}^{zy}+x^{2}yz\mathrm{e}^{zy},x^{2}y^{2} \mathrm{e}^{zy})$ 
    \end{enumerate}

  \item Αν $ d\mathbf{r} = dx \mathbf{i}+ dy \mathbf{j}$, τότε να αποδείξετε ότι 
    $ \grad{f} \cdot d \mathbf{r} = df $

  \item Να υπολογίσετε τη Λαπλασιανή των παρακάτω συναρτήσεων.
    \begin{enumerate}[i)]
      \item $ f(x,y) = x^{2} \mathrm{e}^{y+x^{3}} $ 
        \hfill Απ: $ \grad^{2} f = 2 \mathrm{e}^{y+x^{3}} + 6x^{3} \mathrm{e}^{y+x^{3}} 
        + 12x^{3} \mathrm{e}^{y+x^{3}} + 9x^{6} \mathrm{e}^{y+x^{3}} + x^{2}
        \mathrm{e}^{y+x^{3}} $ 
      \item $ f(x,y) = \frac{1}{y+x^{3}} $
        \hfill Απ: $ \grad^{2}f = \frac{-6(y+x^{3})+18x^{4}+6x^{2}}{(y+x^{3})^{3}} $ 
    \end{enumerate}

  \item Να δείξετε ότι οι παρακάτω συναρτήσεις είναι αρμονικές:
    \begin{enumerate}[(i)]
      \item $ f(x,y) = x^{3}-3xy^{2} $
      \item $ f(x,y) = \ln(x^{2} + y^{2}) $
    \end{enumerate}
\end{enumerate}


\section*{Ομογενείς Συναρτήσεις}

\begin{enumerate}
  \item Αν $ u = u(x,y) $ και $ v=v(x,y) $ ομογενείς βαθμού $ \rho $, 
    τότε να δείξετε ότι $ \forall f(u,v) $ με συνεχείς μερικές παραγώγους 1ης τάξης
    ισχύει 
    \[
      xf_{x}+yf_{y}= \rho (u f_{u}+vf_{v}) 
    \] 
  \item Να αποδείξετε ότι αν $f(x,y)$ είναι ομογενής συνάρτηση, βαθμού $ \rho $, τότε 
    οι συναρτήσεις $ f_{x}, f_{y} $ είναι επίσης ομογενείς, βαθμού $ \rho -1 $.
\end{enumerate}


\section*{Πολυώνυμο Taylor}

\begin{enumerate}
  \item Να βρεθούν τα αναπτύγματα Taylor, μέχρι και όρους 
    \textbf{2ης τάξης}, των συναρτήσεων:

    \begin{enumerate}[i)]
      \item  $f(x,y)=y\cos{xy} $, γύρω από το σημείο 
        $ \left(1, \frac{ \pi }{ 2 }\right) $.

        \hfill Απ: $f(x,y)=-\frac{\pi^{2}}{4}(x-1) - \frac{ \pi }{ 2 } 
        \left(y - \frac{ \pi }{2 }\right) - \pi(x-1)
        \left(y-\frac{\pi}{2}\right)- \left(y- \frac{ \pi }{ 2} \right)^{2} $

      \item $ f(x,y)=e^{x}\tan{y} $ σε δυνάμεις των $ (x-1) $ και 
        $ \left(y - \frac{ \pi }{ 4 }\right) $

        \hfill Απ: $ f(x,y) = e + e(x-1) + 2e\left(y- \frac{ \pi }{ 4 }\right)
        + \frac{1}{ 2 } \left(e(x-1)^{2}+4e(x-1)\left(y- \frac{ \pi }{ 4 }
        \right) + 4e\left(y- \frac{ \pi }{ 4 } \right)^{2}\right) $
    \end{enumerate}

  \item Να βρεθεί το ανάπτυγμα Maclaurin, μέχρι όρους 2ης τάξης, 
    της συνάρτησης $ f(x,y) = e^{x}\ln(1+y)$.

    \hfill Απ: $ f(x,y)=y + xy - \frac{1}{ 2 } y^{2} + \frac{1}{ 2 } x^{2}y - 
    \frac{1}{ 2 } xy^{2} + \frac{1}{ 3 } y^{3} $
\end{enumerate}



\end{document}
