\documentclass[a4paper,table]{report}
\documentclass[a4paper,12pt]{article}
\usepackage{etex}
%%%%%%%%%%%%%%%%%%%%%%%%%%%%%%%%%%%%%%
% Babel language package
\usepackage[english,greek]{babel}
% Inputenc font encoding
\usepackage[utf8]{inputenc}
%%%%%%%%%%%%%%%%%%%%%%%%%%%%%%%%%%%%%%

%%%%% math packages %%%%%%%%%%%%%%%%%%
\usepackage{amsmath}
\usepackage{amssymb}
\usepackage{amsfonts}
\usepackage{amsthm}
\usepackage{proof}

\usepackage{physics}

%%%%%%% symbols packages %%%%%%%%%%%%%%
\usepackage{bm} %for use \bm instead \boldsymbol in math mode 
\usepackage{dsfont}
\usepackage{stmaryrd}
%%%%%%%%%%%%%%%%%%%%%%%%%%%%%%%%%%%%%%%


%%%%%% graphicx %%%%%%%%%%%%%%%%%%%%%%%
\usepackage{graphicx}
\usepackage{color}
%\usepackage{xypic}
\usepackage[all]{xy}
\usepackage{calc}
\usepackage{booktabs}
\usepackage{minibox}
%%%%%%%%%%%%%%%%%%%%%%%%%%%%%%%%%%%%%%%

\usepackage{enumerate}

\usepackage{fancyhdr}
%%%%% header and footer rule %%%%%%%%%
\setlength{\headheight}{14pt}
\renewcommand{\headrulewidth}{0pt}
\renewcommand{\footrulewidth}{0pt}
\fancypagestyle{plain}{\fancyhf{}
\fancyhead{}
\lfoot{}
\rfoot{\small \thepage}}
\fancypagestyle{vangelis}{\fancyhf{}
\rhead{\small \leftmark}
\lhead{\small }
\lfoot{}
\rfoot{\small \thepage}}
%%%%%%%%%%%%%%%%%%%%%%%%%%%%%%%%%%%%%%%

\usepackage{hyperref}
\usepackage{url}
%%%%%%% hyperref settings %%%%%%%%%%%%
\hypersetup{pdfpagemode=UseOutlines,hidelinks,
bookmarksopen=true,
pdfdisplaydoctitle=true,
pdfstartview=Fit,
unicode=true,
pdfpagelayout=OneColumn,
}
%%%%%%%%%%%%%%%%%%%%%%%%%%%%%%%%%%%%%%

\usepackage[space]{grffile}

\usepackage{geometry}
\geometry{left=25.63mm,right=25.63mm,top=36.25mm,bottom=36.25mm,footskip=24.16mm,headsep=24.16mm}

%\usepackage[explicit]{titlesec}
%%%%%% titlesec settings %%%%%%%%%%%%%
%\titleformat{\chapter}[block]{\LARGE\sc\bfseries}{\thechapter.}{1ex}{#1}
%\titlespacing*{\chapter}{0cm}{0cm}{36pt}[0ex]
%\titleformat{\section}[block]{\Large\bfseries}{\thesection.}{1ex}{#1}
%\titlespacing*{\section}{0cm}{34.56pt}{17.28pt}[0ex]
%\titleformat{\subsection}[block]{\large\bfseries{\thesubsection.}{1ex}{#1}
%\titlespacing*{\subsection}{0pt}{28.80pt}{14.40pt}[0ex]
%%%%%%%%%%%%%%%%%%%%%%%%%%%%%%%%%%%%%%

%%%%%%%%% My Theorems %%%%%%%%%%%%%%%%%%
\newtheorem{thm}{Θεώρημα}[section]
\newtheorem{cor}[thm]{Πόρισμα}
\newtheorem{lem}[thm]{λήμμα}
\theoremstyle{definition}
\newtheorem{dfn}{Ορισμός}[section]
\newtheorem{dfns}[dfn]{Ορισμοί}
\theoremstyle{remark}
\newtheorem{remark}{Παρατήρηση}[section]
\newtheorem{remarks}[remark]{Παρατηρήσεις}
%%%%%%%%%%%%%%%%%%%%%%%%%%%%%%%%%%%%%%%




\input{definitions_ask.tex}


\pagestyle{askhseis}


\begin{document}

\begin{center}
  \minibox{\large\bfseries \textcolor{Col1}{Ασκήσεις στις Ιδιοτιμές - Ιδιοδιανύσματα
  Πίνακα}}
\end{center}

\vspace{\baselineskip}

\begin{enumerate}

\item Να βρεθούν οι \textbf{ιδιοτιμές} και τα \textbf{ιδιοδιανύσματα} των παρακάτω 
  πινάκων:

\begin{enumerate}[i)]

\item $\begin{pmatrix}
3 & 2 \\
2 & 3
\end{pmatrix}$\hfill Απ: \begin{tabular}{l}
$\lambda_1=1$, $\lambda_2=5$ \\
$X_1=(-1,1)^T$ \\
$X_2=(1,1)^T$
\end{tabular}

\item $\begin{pmatrix}
1 & 2 \\
4 & 3
\end{pmatrix}$\hfill Απ: \begin{tabular}{l}
$\lambda_1=-1$, $\lambda_2=5$ \\
$X_1=(-1,1)^T$ \\
$X_2=(1,2)^T$
\end{tabular}

\item $\begin{pmatrix}
2 & 1 \\
1 & 2
\end{pmatrix}$\hfill Απ: \begin{tabular}{l}
$\lambda_1=-3$, $\lambda_2=1$ \\
$X_1=(1,1)^T$ \\
$X_2=(-1,1)^T$
\end{tabular}

\item $\begin{pmatrix}
2 & 1 \\
0 & 2
\end{pmatrix}$\hfill Απ: \begin{tabular}{l}
$\lambda_{1,2}=2 \; (\text{διπλή})$ \\
$X_1=(1,0)^T$ \\
\end{tabular}

\item $\begin{pmatrix}
2 & -3 \\
3 & \phantom{-}2
\end{pmatrix}$\hfill Απ: \begin{tabular}{l}
$\lambda_{1,2}=2\pm 3i$ \\
$X_1=(i,1)^T$ \\
$X_2=(-i,1)^T$
\end{tabular}

\end{enumerate}


\end{enumerate}



\end{document}

