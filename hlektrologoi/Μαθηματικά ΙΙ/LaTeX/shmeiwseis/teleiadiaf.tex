\documentclass[a4paper,12pt]{article}
\usepackage{etex}
%%%%%%%%%%%%%%%%%%%%%%%%%%%%%%%%%%%%%%
% Babel language package
\usepackage[english,greek]{babel}
% Inputenc font encoding
\usepackage[utf8]{inputenc}
%%%%%%%%%%%%%%%%%%%%%%%%%%%%%%%%%%%%%%

%%%%% math packages %%%%%%%%%%%%%%%%%%
\usepackage{amsmath}
\usepackage{amssymb}
\usepackage{amsfonts}
\usepackage{amsthm}
\usepackage{proof}

\usepackage{physics}

%%%%%%% symbols packages %%%%%%%%%%%%%%
\usepackage{bm} %for use \bm instead \boldsymbol in math mode 
\usepackage{dsfont}
\usepackage{stmaryrd}
%%%%%%%%%%%%%%%%%%%%%%%%%%%%%%%%%%%%%%%


%%%%%% graphicx %%%%%%%%%%%%%%%%%%%%%%%
\usepackage{graphicx}
\usepackage{color}
%\usepackage{xypic}
\usepackage[all]{xy}
\usepackage{calc}
\usepackage{booktabs}
\usepackage{minibox}
%%%%%%%%%%%%%%%%%%%%%%%%%%%%%%%%%%%%%%%

\usepackage{enumerate}

\usepackage{fancyhdr}
%%%%% header and footer rule %%%%%%%%%
\setlength{\headheight}{14pt}
\renewcommand{\headrulewidth}{0pt}
\renewcommand{\footrulewidth}{0pt}
\fancypagestyle{plain}{\fancyhf{}
\fancyhead{}
\lfoot{}
\rfoot{\small \thepage}}
\fancypagestyle{vangelis}{\fancyhf{}
\rhead{\small \leftmark}
\lhead{\small }
\lfoot{}
\rfoot{\small \thepage}}
%%%%%%%%%%%%%%%%%%%%%%%%%%%%%%%%%%%%%%%

\usepackage{hyperref}
\usepackage{url}
%%%%%%% hyperref settings %%%%%%%%%%%%
\hypersetup{pdfpagemode=UseOutlines,hidelinks,
bookmarksopen=true,
pdfdisplaydoctitle=true,
pdfstartview=Fit,
unicode=true,
pdfpagelayout=OneColumn,
}
%%%%%%%%%%%%%%%%%%%%%%%%%%%%%%%%%%%%%%

\usepackage[space]{grffile}

\usepackage{geometry}
\geometry{left=25.63mm,right=25.63mm,top=36.25mm,bottom=36.25mm,footskip=24.16mm,headsep=24.16mm}

%\usepackage[explicit]{titlesec}
%%%%%% titlesec settings %%%%%%%%%%%%%
%\titleformat{\chapter}[block]{\LARGE\sc\bfseries}{\thechapter.}{1ex}{#1}
%\titlespacing*{\chapter}{0cm}{0cm}{36pt}[0ex]
%\titleformat{\section}[block]{\Large\bfseries}{\thesection.}{1ex}{#1}
%\titlespacing*{\section}{0cm}{34.56pt}{17.28pt}[0ex]
%\titleformat{\subsection}[block]{\large\bfseries{\thesubsection.}{1ex}{#1}
%\titlespacing*{\subsection}{0pt}{28.80pt}{14.40pt}[0ex]
%%%%%%%%%%%%%%%%%%%%%%%%%%%%%%%%%%%%%%

%%%%%%%%% My Theorems %%%%%%%%%%%%%%%%%%
\newtheorem{thm}{Θεώρημα}[section]
\newtheorem{cor}[thm]{Πόρισμα}
\newtheorem{lem}[thm]{λήμμα}
\theoremstyle{definition}
\newtheorem{dfn}{Ορισμός}[section]
\newtheorem{dfns}[dfn]{Ορισμοί}
\theoremstyle{remark}
\newtheorem{remark}{Παρατήρηση}[section]
\newtheorem{remarks}[remark]{Παρατηρήσεις}
%%%%%%%%%%%%%%%%%%%%%%%%%%%%%%%%%%%%%%%




\newcommand{\vect}[2]{(#1_1,\ldots, #1_#2)}
%%%%%%% nesting newcommands $$$$$$$$$$$$$$$$$$$
\newcommand{\function}[1]{\newcommand{\nvec}[2]{#1(##1_1,\ldots, ##1_##2)}}

\newcommand{\linode}[2]{#1_n(x)#2^{(n)}+#1_{n-1}(x)#2^{(n-1)}+\cdots +#1_0(x)#2=g(x)}

\newcommand{\vecoffun}[3]{#1_0(#2),\ldots ,#1_#3(#2)}

\newcommand{\mysum}[1]{\sum_{n=#1}^{\infty}





\begin{document}

\chapter{Μερική Παράγωγος}

\section{Ολικά Διαφορικά}

\begin{dfn}
	Η παράσταση  $ P(x,y)dx + Q(x,y)dy $ είναι ολικό (ή τέλειο) διαφορικό αν υπάρχει συνάρτηση  $
	f(x,y) $ τέτοια ώστε $ df = P(x,y)dx + Q(x,y)dy $. Τότε ισχύουν οι παρακάτω σχέσεις:
	 \[
		 \pdv{f}{x} = P(x,y) \quad \text{και} \quad \pdv{f}{y} = Q(x,y)
	 \] 
\end{dfn}

\begin{prop}
	Αν οι  $ P(x,y) $  και  $ Q(x,y) $  είναι συνεχείς συναρτήσεις και έχουν συνεχείς παραγώγους
	πρώτης τάξης τότε η  παράσταση  $ P(x,y)dx + Q(x,y)dy $ είναι τέλειο διαφορικό αν 
	\[ 
		\pdv{P}{y} = \pdv{Q}{x} 
\]
\end{prop}

\begin{rem}
    Η παραπάνω ισότητα μας εξασφαλίζει ότι $ f_{xy} = f_{yx} $. 
\end{rem}

\begin{dfn}
	Η παράσταση  $ P(x,y,z)dx + Q(x,y,z)dy + R(x,y,z)dz $ είναι τέλειο διαφορικό αν υπάρχει
	συνάρτηση  $ f(x,y,z) $  τέτοια ώστε  $ df = P(x,y,z)dx + Q(x,y,z)dy + R(x,y,z)dz $.  Τότε
	ισχύουν οι παρακάτω σχέσεις:
	 \[
		 \pdv{f}{x} = P(x,y,z) \quad \text{και} \quad \pdv{f}{y} = Q(x,y,z) \quad \text{και} \quad
		 \pdv{f}{z} = R(x,y,z) 
	 \] 
\end{dfn}

\begin{prop}
	
	Αν οι  $ P(x,y,z) $, $ Q(x,y,z) $  και  $ R(x,y,z) $ είναι συνεχείς συναρτήσεις και έχουν
	συνεχείς παραγώγους πρώτης τάξης τότε η  παράσταση $ P(x,y,z)dx + Q(x,y,z)dy + R(x,y,z)dz $   είναι τέλειο διαφορικό αν 
	  \[
		  \pdv{P}{y} = \pdv{Q}{x} \quad \text{και} \quad \pdv{Q}{z} = \pdv{R}{y} \quad \text{και}
		  \quad  \pdv{P}{z} = \pdv{R}{x}
	  \] 
\end{prop}

\begin{rem}
    Η παραπάνω ισότητες μας εξασφαλίζουν την ισότητα των μικτών παραγώγων.
\end{rem}

\begin{rem}
	Οι συναρτήσεις  $ f(x,y) $  και  $ f(x,y,z) $ υπολογίζονται επίσης από τις παρακάτω σχέσεις:
	\begin{align*} 
	f(x,y) &= \int_{x_{0}}^{x} P(t,y) \,{dt} + \int_{y_{0}}^{y} Q(x_{0},t) \,{dt} \\
	f(x,y,z) &= \int_{x_{0}}^{x} P(t,y,z) \,{dt} + \int_{y_{0}}^{y} Q(x_{0},t,z) \,{dt} + \int
		 _{z_{0}}^{z} R(x_{0},y_{0},t) \,{dt}  
\end{align*} 
όπου τα   $ x_{0} $, $ y_{0} $  και  $ z_{0} $  επιλέγονται αυθαίρετα στο πεδίο ορισμού των  $ P
$, $ Q $  και  $ R $.
\end{rem}



\end{document}
	
