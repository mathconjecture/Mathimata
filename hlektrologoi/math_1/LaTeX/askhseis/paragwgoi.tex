\documentclass[a4paper,table]{report}
\input{preamble_ask.tex}
\input{definitions_ask.tex}

\newcommand{\twocolumnsidesss}[2]{\begin{minipage}[t]{0.52\linewidth}\raggedright
        #1
        \end{minipage}\hfill\begin{minipage}[t]{0.47\linewidth}\raggedright
        #2
    \end{minipage}
}

\geometry{top=2cm,left=1.5cm,right=1.5cm}

\pagestyle{askhseis}

\begin{document}


\begin{center}
  \minibox{\large \bfseries \textcolor{Col1}{Ασκήσεις στις Παραγώγους}}
\end{center}

\vspace{\baselineskip}

\begin{enumerate}

  \item Να υπολογιστούν οι \textbf{παράγωγοι} των παρακάτω συναρτήσεων
    \textcolor{Col1}{(Κανόνας Αλυσίδας})
    \begin{enumerate}[(i)]
      \item $ f(x) = \ln{(\sqrt[5]{1+3x^{2}})} $ \quad
        \textcolor{Col1}{Υπόδειξη:} με δύο τρόπους
        \hfill Απ: $ \frac{6x}{5(1+3x^{2})} $
      \item $ f(x) = \ln({\sin({\cos{x}})}) $ \quad 
        \textcolor{Col1}{Υπόδειξη:} με δύο τρόπους \hfill Απ: $
        \frac{1}{\sin{(\cos{x})}} [\cos{(\cos{x})}] (- \sin{x}) $ 
      \item $ f(x) = \arctan{\left(\frac{x}{\sqrt{1 + x^{2}}}\right)} $ \hfill Απ: $
        \frac{1}{(1+2x^{2})\sqrt{1 + x^{2}}} $
      \item $ f(x) = \ln{(e^{\sin{x}})} + \sqrt{x^{2} - 25x} $ \hfill Απ: $
        \cos{x} + \frac{2x - 25}{2 \sqrt{x^{2} - 25x}}  $  
    \end{enumerate}

  \item  Να υπολογιστούν οι \textbf{παράγωγοι} των παρακάτω συναρτήσεων
    \textcolor{Col1}{(Λογαριθμική Παραγώγιση})

    \begin{enumerate}[(i)]
      \item $ f(x) = (\cos{x})^{\sin{2x}} $ \hfill Απ: $
        (\cos{x})^{\sin{2x}} 2(\cos{2x} \ln{(\cos{x})} - \sin^{2}{x}) $
      \item $ f(x) = \left(1 + \frac{1}{x} \right)^{x} $ \hfill Απ: $
        \left(1 + \frac{1}{x}\right)^{x}\left[\ln{(1 + \frac{1}{x})} -
        \frac{1}{x+1}\right] $
      \item $ f(x)=(\sin{x})^{x} $ \hfill Απ: $ (\sin{x})^{x}[\ln{(\sin{x}
        )} + x \cot{x}] $ 
      \item $ f(x)=\cos{x}^{x} $ \hfill Απ: $ (- \sin{x^{x}})x^{x} (1 +
        \ln{x}) $
    \end{enumerate}

  \item Να βρεθούν οι \textbf{παράγωγοι} των \textcolor{Col1}{αντίστροφων,} των παρακάτω 
    συναρτήσεων.

    \textcolor{Col1}{Υπόδειξη:} 
      $ \cosh^{2}{x} - \sinh^{2}{x} = 1 $, \;
      $ \tanh{x} = \frac{\sinh{x}}{\cosh{x}}$, \; 
      $ \frac{1}{\cos^{2}{x}} = 1+ \tan^{2}{x} $, \; 
      \begin{enumerate}[(i)]
        \twocolumnside{
          \item $ y = \cos{x} $ \hfill Απ: $ \frac{-1}{\sqrt{1 - y^{2}}} $
          \item $ y = \tan{x} $ \hfill Απ: $ \frac{1}{1 + y^{2}} $
          }{
          \item $ y = \cosh{x} $  \hfill Απ: $ \frac{1}{\sqrt{y^{2} - 1}} $
          \item $ y = \tanh{x} $ \hfill Απ: $ \frac{1}{x^{2} - 1} $
          }
      \end{enumerate}

  \item Δίνεται η σχέση $ x^{2} - xy + y^{2} = 3 $, $ y=y(x) $. Να βρεθεί η 1η
    και η 2η \textbf{παράγωγος} της $y$ ως προς $x$ στο σημείο $ (1,-1) $.
    \hfill Απ: $ y' = 1$, $ y'' = \frac{2}{3} $

  \item Δίνεται η σχέση $ 4x^{3} - 3xy^{2} + 6x^{2} - 5xy - 8 y^{2} + 9x + 14
    = 0$. Να βρείτε τις εξισώσεις της \textbf{εφαπτομένης} και της \textbf{κάθετης
    ευθείας} της καμπύλης στο σημείο $ (-2,3) $.

    \textcolor{Col1}{Υπόδειξη:} 
    $ \varepsilon: y-y(x_{0}) = y'(x_{0})(x- x_{0}) $, \;
    $ \kappa: y-y(x_{0}) = -\frac{1}{y'(x_{0})}(x- x_{0}) $, \;

    \hfill Απ: $\varepsilon\colon y = -\frac{9}{2} x - 6 $, 
    $\kappa\colon y = \frac{2}{9} x + \frac{31}{9} $.

  \item Να υπολογιστεί η 1η \textbf{παράγωγος} $ \dv{y}{x} $ της συνάρτησης $ y=y(x) $ 
    που ορίζεται μέσω των \textcolor{Col1}{παραμετρικών εξισώσεων}:
    \begin{enumerate}[i)]
      \item $ x = 3t^{2}, \; y=2t^{2}-1 $ \hfill Απ: $ \dv{y}{x} = \frac{4}{9t} $ 
      \item $ x = \cos{t} + t \sin{t}, \; y= \sin{t} - t \cos{t} $ 
        \hfill Απ: $ \dv{y}{x} = \tan{t} $ 
    \end{enumerate}

  \item Να υπολογιστεί η 2η \textbf{παράγωγος} $ \dv[2]{y}{x}$  της συνάρτησης 
    $ y=y(x) $ που ορίζεται μέσω των \textcolor{Col1}{παραμετρικών εξισώσεων}:
    \begin{enumerate}[i)]
      \item $ x=2 + \sqrt{t}, \; y=t^{2}-1 $ \hfill Απ: $ \dv[2]{y}{x} = 12 t $ 
      \item $ x=2 \cos{t}, \; y= 3 \sin{t}  $ \hfill Απ: $ \dv[2]{y}{x} = -
        \frac{3}{4} \frac{1}{\sin^{3}{t}} $  
    \end{enumerate}

  \item Να υπολογιστεί η εξίσωση της \textbf{εφαπτομένης} της καμπύλης 
    $ x=2 \sqrt{2} \cos{t} $ και $ y= \sqrt{2} \sin{t} $ στο σημείο $ (2,1) $.

    \hfill Απ: $ \dv{y}{x} \eval_{t= \pi /4} = - \frac{1}{2}, \; \varepsilon: 2y+x=4 $ 

  % \item {\bfseries (Σεπ 2017)}
  %   \begin{enumerate}[i)]
  %     \item Να δοθεί ο ορισμός καθώς και η γεωμετρική
  %       ερμηνεία του \textbf{διαφορικού} πρώτης τάξης της συνάρτησης $ y = g(x) $ στο
  %       τυχαίο σημείο $x$. 
  %     \item Να βρεθεί το διαφορικό δεύτερης τάξης της \textbf{σύνθετης} συνάρτησης 
  %       $ z(x) = f(u(x))	$.
  %   \end{enumerate}

  \item Να υπολογιστούν κατά προσέγγιση, με τη βοήθεια του \textbf{διαφορικού} οι τιμές 
    των παρακάτω παραστάσεων :
    \begin{enumerate}[i)]
      \item $\sqrt{104}$ \hfill Απ: $10,2$
      \item $\sqrt[4]{17}$ \hfill Απ: $\frac{1}{4}17^{-\frac{3}{4}}+2$
    \end{enumerate}

  \item Να υπολογιστούν τα παρακάτω \textbf{όρια}, με τη βοήθεια του κανόνα
    \textcolor{Col1}{L' H\^{o}spital}.
    \begin{enumerate}[(i)]
      \twocolumnside{
        \item $ \lim_{x\to 1} \left(\frac{1}{\ln{x}} - \frac{1}{x-1}\right) $ \hfill
          Απ: $ \frac{1}{2} $
        \item $ \lim_{x\to 1} \left[(1-x) \tan{\frac{\pi x}{2}}\right] $ \hfill Απ: $
          \frac{2}{\pi} $
        \item $ \lim_{x\to \frac{\pi}{4}} \frac{\sqrt{2} - \sin{x} -
          \cos{x}}{\ln{(\sin{2x})}} $ \hfill Απ: $ - \frac{\sqrt{2}}{4} $
          }{
        \item $ \lim_{x\to +\infty} \left(1 + \frac{1}{x} +
          \frac{2}{x^{2}}\right)^{x} $ \hfill Απ: $ e $ 
        \item $ \lim_{x\to 0^{+}} \left(\frac{1}{x}\right)^{\sin{x}} $ \hfill $ 1 $
        \item $ \lim_{x\to 0} \left(\cos{2x}\right)^{\frac{3}{x^{2}}}  $ \hfill Απ:
          $ e^{-6} $
        }
    \end{enumerate}

    %A Desmi p.179
  \item Να δείξετε ότι η εξίσωση $ 3x^{5}-x^{3}=5 $ έχει \textbf{ακριβώς μία} 
    ρίζα στο διάστημα $ (1,2) $.
    %A Desmi p.176
  \item Να δείξετε ότι η εξίσωση $ \cos{x} + 1 = x $ έχει \textbf{ακριβώς μία} 
    ρίζα στο διάστημα $ (0, \pi/2) $. 

  \item {\bfseries (Ιαν 2018)} Να αποδείξετε ότι η εξίσωση 
    $ x^{2} = x \sin{x} + \cos{x} $ έχει \textbf{ακριβώς δύο} πραγματικές ρίζες 
    $ x_{1} $, $ x_{2} $, με $ x_{1} \in (-\pi, 0) $, και $x_{2} \in (0, \pi) $.

  \item Να αποδείξετε τις παρακάτω \textbf{ανισότητες}
    \begin{enumerate}[i)]
      \item $ 2 \sqrt{x} \geq 3 - \frac{1}{x} $, για κάθε $ x>0 $.
      \item $ \cos{x} > 1 - \frac{x^{2}}{2} $, για κάθε $ x>0 $.
      \item $ 1- \frac{1}{x} \leq \ln{x} \leq x-1 $, για κάθε $ x>0 $.
    \end{enumerate}

  \item Να αποδείξετε ότι για κάθε $x,y \in \mathbb{R}$ με $ x \neq y $, ισχύουν 
    \begin{enumerate}[i)]
      \item $ \abs{ \cos^{2}{x} - \cos^{2}{y}} \leq \abs{x-y} $ 
      \item $ \mathrm{e}^{x} < \frac{\mathrm{e}^{x} - \mathrm{e}^{y}}{x-y} <
        \mathrm{e}^{y} $, αν $ x<y $.
    \end{enumerate}

  \item Να αποδείξετε την παρακάτω \textbf{ανισότητα} με τη βοήθεια του Θεωρήματος
    Μέσης Τιμής.   
    \[
      \frac{a - b}{\cos^{2}{b}} \leq \tan{a} - \tan{b}\leq \frac{a -
      b}{\cos^{2}{a}}, \qq{με}  0 < b \leq a < \frac{\pi}{2}
    \]

  \item Να αποδείξετε την \textbf{ανισότητα} με τη βοήθεια του Θεωρήματος
    Μέσης Τιμής.   
    \[
      \frac{a-b}{a} \leq \ln{\frac{a}{b}} \leq \frac{a-b}{b}, \qq{με}  0<b\leq a 
    \]

  \item Να αποδείξετε την \textbf{ανισότητα} με τη βοήθεια του Θεωρήματος
    Μέσης Τιμής.   
    \[
      n b^{n-1} (a-b) < a^{n}-b^{n}< n a^{n-1} (a-b), \qq{με} 0<b<a, \; n \in
      \mathbb{N}-\{1\}
    \]
    %%A Desmi p.178
  %\item Να αποδείξετε την παρακάτω \textbf{ανισότητα} 
    %$ x+1 \leq \mathrm{e}^{x} \leq x \mathrm{e}^{x} + 1 $ για κάθε $ x \in \mathbb{R} $ 

    % \item Να δειχθεί ότι η συνάρτηση $ f(x) = \frac{x^{2} + 6x + 12}{x^{2} - 6x
    %   + 12} $ είναι καλή προσέγγιση της συνάρτησης $ e^{x} $ για μικρές τιμές
    %   του $x$ διότι τα αναπτύγματα των δύο συναρτήσεων συμπίπτουν στους 5
    %   πρώτους όρους. 
    %   \hfill Απ: $ f(x) = e^{x} \cong 1 + x + \frac{x^{2}}{2} +
    %   \frac{x^{3}}{6} + \frac{x^{4}}{24} $

  \item Να υπολογιστεί το προσεγγιστικό πολυώνυμο Maclaurin 3ης τάξης, των 
    παρακάτω συναρτήσεων.

    \twocolumnsidesss{
      \begin{enumerate}[i)]
        \item $ y= \mathrm{e}^{-x} $ 
          \hfill Απ: $ \mathrm{e}^{-x} \approx 1-x+ \frac{1}{2} x^{2} - \frac{1}{6} x^{3} $ 
        \item $ y= \ln{(x+2)} $ \;
          \hfill Απ: $ \ln{(x+2)} \approx \ln{2} + \frac{1}{2} x - \frac{1}{8} x^{2} +
          \frac{1}{24} x^{3} $ 
      \end{enumerate}
    }{
      \begin{enumerate}[i),start=3]
        \item $ y= \frac{1}{x-1} $ \hfill Απ: $ \frac{1}{x-1} \approx -1 -x -x^{2} - x^{3} $ 
        \item $ y= \sqrt{x+1} $ \hfill Απ: $ \sqrt{x+1} \approx 1 + \frac{1}{2} x -
          \frac{1}{8} x^{2} + \frac{1}{16} x^{3} $ 
      \end{enumerate}
    }

    \item Έστω $ f(x) = \ln{(1+x)} $, $ x>-1 $. Να υπολογιστεί το ανάπτυγμα
      \textbf{Maclaurin} 4ης τάξης και στη συνέχεια να υπολογιστεί το αντίστοιχο 
      \textbf{σφάλμα} για $ x = 0,1 $.

      \hfill Απ: \begin{tabular}{l}
        $ \ln(1+x) \cong x - \frac{1}{2} x^{2} + \frac{1}{3}x^{3} - 
        \frac{1}{4} x^{4} $ \\ 
        $ \abs{R_{4}(0,1)} < 0,000002$	
      \end{tabular}

  \item{\bfseries (Ιαν 2016)} Να βρείτε μια \textbf{πολυωνυμική προσέγγιση} 3ης τάξης 
    της συνάρτησης που ορίζεται πεπλεγμένα από την εξίσωση 
    $ x^{2} - xy + y^{2} = 3$ στο σημείο $ (1,-1) $.

    \hfill Απ: $f(x) \cong -1 + (x-1) + \frac{(x-1)^{2}}{3} +
    \frac{(x-1){3}}{9}$

  \item{\bfseries (Σεπ 2023)} Να δείξετε ότι 
    \[
      \sin{x} \cong \sin{a} + \cos{a} (x-a) - \frac{\sin{a}}{2!} (x-a)^{2} -
      \frac{\cos{\xi} (x-a)^{3}}{3!}
    \]
    όπου $\xi$ μεταξύ $a$ και $x$. Στη συνέχεια να υπολογίσετε το $
    \sin{\ang{51}}$ καθώς και το διαπραττόμενο \textbf{σφάλμα}.

    \hfill Απ: $ \abs{R_{3}(\ang{51})} < 0,00019 $

    % \item Να βρεθεί η ακτίνα καμπυλότητας στο τυχαίο σημείο $\theta$ της
    %   καμπύλης που περιγράφεται από τις εξισώσεις 
    %   \begin{align*}
    %     x &= a(\theta - \sin{\theta}) \\
    %     y &= a(1 - \cos{\theta})
    %   \end{align*}		
    %   \hfill Απ: $ \rho = 4a \abs{\sin{\frac{\theta}{2}}} $

\end{enumerate}


\end{document}
