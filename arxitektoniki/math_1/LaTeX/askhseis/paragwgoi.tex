\input{preamble_ask.tex}
\input{definitions_ask.tex}
\input{tikz.tex}

\geometry{top=2.5cm}

\newcommand{\twocolumnsidescc}[2]{\begin{minipage}[t]{0.72\linewidth}
        #1
        \end{minipage}\hfill\begin{minipage}[t]{0.23\linewidth}
        #2
    \end{minipage}
}


\pagestyle{askhseis}

\begin{document}


\begin{center}
  \minibox{\large \bfseries \textcolor{Col1}{Ασκήσεις στις Παραγώγους}}
\end{center}

\vspace{\baselineskip}

\begin{enumerate}

  % \item  Να υπολογιστούν οι \textbf{παράγωγοι} των παρακάτω συναρτήσεων

  %   \begin{enumerate}[(i)]
  %     \item $ f(x) = (\cos{x})^{\sin{2x}} $ \hfill Απ: $
  %       (\cos{x})^{\sin{2x}} 2(\cos{2x} \ln{(\cos{x})} - \sin^{2}{x}) $
  %     \item $ f(x) = \left(1 + \frac{1}{x} \right)^{x} $ \hfill Απ: $
  %       \left(1 + \frac{1}{x}\right)^{x}\left[\ln{(1 + \frac{1}{x})} -
  %       \frac{1}{x+1}\right] $
  %     \item $ f(x)=(\sin{x})^{x} $ \hfill Απ: $ (\sin{x})^{x}[\ln{(\sin{x}
  %       )} + x \cot{x}] $ 
  %     \item $ f(x)=\cos{x}^{x} $ \hfill Απ: $ (- \sin{x^{x}})x^{x} (1 +
  %       \ln{x}) $
  %   \end{enumerate}

  % \item Να βρεθούν οι παράγωγοι των αντίστροφων, των παρακάτω συναρτήσεων.

  %   \begin{enumerate}[(i)]
  %     \item $ y = \cos{x} $ \hfill Απ: $ \frac{-1}{\sqrt{1 - y^{2}}} $
  %     \item $ y = \tan{x} $ \hfill Απ: $ \frac{1}{1 + y^{2}} $
  %   \end{enumerate}

  \item Δίνεται η σχέση $ x^{2} - xy + y^{2} = 3 $, $ y=y(x) $. Να βρεθεί η 1η
    και η 2η παράγωγος της $y$ ως προς $x$ στο σημείο $ (1,-1) $.

    \hfill Απ: $ y' = 1$, $ y'' = \frac{2}{3} $

  \item Να βρεθεί, αν υπάρχει, σημείο (ή σημεία) στο οποίο η καμπύλη 
    $ y= x+ \sin{x}, \;  0 \leq x \leq 2 \pi $ έχει οριζόντια εφαπτομένη. Αν 
    υπάρχει, ποια είναι η εξίσωση της εφαπτομένης σε αυτό το σημείο? 
    \hfill Απ: $ (\pi , \pi), \; \varepsilon: y= \pi $  

  \item Να βρεθεί η κλίση της εφαπτομένης της καμπύλης $ (x^{2}+y^{2})^{2} = 4x^{2}y $, 
    στο σημείο $ (-1,1) $.
    \hfill Απ: 0 

  \item Να βρεθεί η εξίσωση της εφαπτομένης της καμπύλης $x^{3} + y^{3}=9xy$, στο 
    σημείο $ (2,4) $.
    \hfill Απ: $ \varepsilon: 4x-5y=-12 $

  \item Δίνεται η σχέση $ 4x^{3} - 3xy^{2} + 6x^{2} - 5xy - 8 y^{2} + 9x + 14
    = 0$. Να βρείτε τις εξισώσεις της \textbf{εφαπτομένης (\varepsilon)} και της
    \textbf{κάθετης (\kappa)} 
    ευθείας της καμπύλης στο σημείο $ (-2,3) $.

    \textcolor{Col1}{Υπόδειξη:} 
    $ \varepsilon: y-y(x_{0}) = y'(x_{0})(x- x_{0}) $, \;
    $ \kappa: y-y(x_{0}) = -\frac{1}{y'(x_{0})}(x- x_{0}) $, \;

    \hfill Απ: $\varepsilon\colon y = \frac{9}{2} x - 6 $, 
    $\kappa\colon y = \frac{2}{9} x + \frac{31}{9} $.

  \item Να υπολογισθεί η κλίση της καμπύλης $ x= 2 \sqrt{2} \cos{t} $ και $ y=
    \sqrt{2} \sin{t} $ στο σημείο $ (2,1) $ και να βρεθεί η εξίσωση της εφαπτομένης 
    σε αυτό το σημείο.

    \hfill Απ: $ \eval{\dv{y}{x}} _{t= \frac{\pi}{4}} = - \frac{1}{2}, \; 
    \varepsilon: 2y+x=4 $  

  \item Να υπολογιστούν τα τοπικά ακρότατα των παρακάτω συναρτήσεων.

    \begin{enumerate}[i)]
      \renewcommand{\itemsep}{15pt}
      \item $ f(x) = 3x^{2}-2x+1 $ 
        \hfill Απ: $ f_{\rm_{min}}\left(\frac{1}{3} \right) = \frac{2}{3} $ 
      \item $ f(x) = x^{3}+3x^{2}-5 $ 
        \hfill Απ: $ f_{\rm min}(0) = -5 $, $ f_{\rm max}(-2) = -1 $ 
      \item $ f(x) = x^{3} - x $ 
        \hfill Απ: $ f_{\rm min}\left(\frac{1}{\sqrt{3}} \right) 
        = - \frac{2}{3 \sqrt{3}} $, $ f_{\rm max} \left(-\frac{1}{\sqrt{3}} 
        \right) = \frac{2}{3 \sqrt{3}} $
      % \item $ f(x) = 3x^{4} -16x^{3}+18x^{2} $ 
      %   \hfill Απ: $ f_{\rm min}(0)=0 $, $ f_{\rm min}(3)
      %   =-27 $, $ f_{\rm max}(1) = 5 $ 
      \item $ f(x) = \frac{x^{2}}{e^{x}} $
        \hfill Απ: $ f_{\rm min}(0) = 0 $, $f_{\rm max}(2) = \frac{4}{e^{2}} $ 
    \end{enumerate}

  \item Να βρεθούν τα ακρότατα της συνάρτησης $ x^{2}-xy+y^{2}=4 $, όπου $ y=y(x) $.

    \hfill Απ: \textlatin{max}: $ y(1,2) $ και \textlatin{min}: $ y(-1,-2) $

  \item  \twocolumnsidescc{
  Σε μία άκρη του κήπου μας θέλουμε να φτιάξουμε ένα παρτέρι για λουλούδια 
    σχήματος ορθογωνίου παραλληλεπιπέδου, το οποίο να εφάπτεται στον τοίχο. 
    Χρειάζεται να επενδυθεί με ασφαλτόπανο για μόνωση. Διαθέτουμε ένα ασφαλτόπανο 
    μήκους $ s=\SI{10}{m} $ και πλάτους $ s/2 \; \SI{}{m} $. Θα κόψουμε το ασφαλτόπανο
    στις δύο γωνίες του κατά μήκος $x$, όπως φαίνεται στο σχήμα, έτσι ώστε λυγίζοντας
    τις πλευρές προς τα πάνω να δημιουργηθεί ένα παραλληλεπίπεδο κουτί για να βάλουμε
    μέσα το χώμα. Πόσο πρέπει να είναι το μήκος $x$ του τετραγώνου που θα κόψουμε 
    από κάθε γωνία, ώστε το παρτέρι να δέχεται τον μεγαλύτερο όγκο χώματος;

    \hfill Απ: $ x= 5/3, \; V_{\text{\textlatin{max}}} = 1000/27 $ 
    }{
\begin{tikzpicture}[scale=0.6,baseline={100pt}]
		\node  (0) at (0, -0.5) {};
		\node  (1) at (0, 0) {};
		\node  (2) at (0, 6) {};
		\node  (3) at (0, 6.5) {};
		\node  (4) at (3, 6) {};
		\node  (5) at (3, 0) {};
		\node  (6) at (3, 1) {};
		\node  (7) at (4, 1) {};
		\node  (8) at (3, 5) {};
		\node  (9) at (4, 5) {};
    \coordinate (0c) at ($(0)-(0.5,0)$) ;
    \coordinate (3c) at ($(3)-(0.5,0)$) ;
    \draw[pattern=north west lines] (0.center) -- (1.center) -- (2.center) -- 
      (3.center) -- (3c.center) -- (0c.center) -- (0.center) ;
    \fill[Col2!10] (1.center) -- (2.center) -- (4.center) -- (8.center) --
      (9.center) -- (7.center) -- (6.center) -- (5.center) -- (1.center) ;
    \begin{scope}[graph,Col2]
		\draw (0.center) to (1.center);
		\draw (1.center) to (2.center);
		\draw (2.center) to (3.center);
		\draw (2.center) to (4.center);
		\draw (4.center) to (8.center);
		\draw (8.center) to (9.center);
		\draw (9.center) to (7.center);
		\draw (7.center) to (6.center);
		\draw (6.center) to (5.center);
		\draw (5.center) to (1.center);
  \end{scope}
    \draw[dotted] (5) -| (7) ;
    \draw[dotted] (4) -| (9) ;
    \coordinate (1a) at ($(1)-(0,0.9)$) ;
    \coordinate (5a) at ($(5)+(1,0)-(0,0.9)$) ;
    \coordinate (5b) at ($(5-|7)+(0.8,0)$) ;
    \coordinate (4b) at ($(4-|9)+(0.8,0)$) ;
    \node at (4) [above right,xshift=4pt] {$x$} ;
    \node at (9) [above right,yshift=4pt] {$x$} ;
    \draw[{Stealth}-{Stealth}] (1a) node {$|$} -- node[fill=white]{$ s/2 $}
      (5a) node {$|$} ;
    \draw[{Stealth}-{Stealth}] (4b) node[rotate=90] {$|$} -- node[fill=white]{$ s $}
      (5b) node[rotate=90] {$|$} ;
\end{tikzpicture}
    }

  \item Έστω $ f(x) = \ln{(1+x)} $, $ x>-1 $. Να υπολογιστεί το ανάπτυγμα
    \textlatin{\textbf{Maclaurin}} 4ης τάξης και στη συνέχεια να υπολογιστεί το 
    αντίστοιχο σφάλμα για $ x = 0,1 $.

    \hfill Απ: \begin{tabular}{l}
      $ \ln(1+x) \cong x - \frac{1}{2} x^{2} + \frac{1}{3}x^{3} - 
      \frac{1}{4} x^{4} $ \\ 
      $ \abs{R_{4}(0,1)} < 0,000002$	
    \end{tabular}

  \item Να βρείτε μια \textbf{πολυωνυμική προσέγγιση} 3ης τάξης 
    της συνάρτησης που ορίζεται πεπλεγμένα από την εξίσωση 
    $ x^{2} - xy + y^{2} = 3$ στο σημείο $ (1,-1) $.

    \hfill Απ: $f(x) \cong -1 + (x-1) + \frac{(x-1)^{2}}{3} +
    \frac{(x-1){3}}{9}$

\end{enumerate}



\end{document}

