\documentclass[a4paper,12pt]{article}

\usepackage[english,greek]{babel}
\usepackage[utf8]{inputenc}

\usepackage{amsmath}
\usepackage{amssymb}
\usepackage{amsfonts}
\usepackage{amsthm}
\usepackage[top=2cm,bottom=2cm,left=2cm,right=2cm]{geometry}
\usepackage{outlines}

\DeclareMathOperator{\rank}{rank}

\begin{document}
\thispagestyle{empty}
\begin{center}
    {\large\bfseries \textsc{Τμημα Διοικησης Επιχειρησεων \\
        Πανεπιστημιο Πατρων}} \\[0,5cm]
        \textbf{Μαθηματικός Λογισμός} \\[0,5cm]
        Εξεταστική Σεπτεμβρίου $2015$
    \end{center}
    \vspace{0,5cm}
    \begin{description}

        \item [{\bfseries Θέμα $1$ο:}] Έστω το γραμμικό υπόδειγμα:
            \begin{align*}
                (1+\lambda)x + y + z &= 1 \\
                x + (1 + \lambda)y + z &=\lambda\\
                x+y+(1+\lambda)z&=\lambda^2
            \end{align*}
    \end{description}

    \begin{enumerate}

        \item Να υπολογίσετε την τάξη του πίνακα των συντελεστών των αγνώστων. ($1,5$ μον.)
        \item Για ποιες τιμές του $\lambda$ το γραμμικό σύστημα έχει μια λύση, για ποιες άπειρες και για ποιες είναι αδύνατο. ($1,5$ μον.)
        \item Όταν το παραπάνω σύστημα έχει μια λύση, να την βρείτε με τη μέθοδο της αντιστροφής των πινάκων. ($1$ μον.)
        \item Έχουμε ένα γραμμικό σύστημα $m$ εξισώσεων με $n$ αγνώστους. Αναλύστε τα είδη των λύσεων που προκύπτουν αναφορικά με την τάξη του πίνακα των συντελεστών των αγνώστων και τη σχέση μεταξύ $m$ και $n$. ($2$ μον.)

    \end{enumerate}



    \end{document}
