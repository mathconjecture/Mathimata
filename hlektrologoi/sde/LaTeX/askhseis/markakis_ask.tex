\input{preamble_ask.tex}
\input{definitions_ask.tex}
\input{myboxes.tex}

\pagestyle{askhseis}

\begin{document}



\begin{center}
  \minibox{\large \bfseries \color{Col1} Άσκηση (Μαρκάκης)}
\end{center}

\vspace{\baselineskip}

\begin{mybox2}
\begin{exercise}
  Αν ο ρυθμός απορρόφησης της ενέργειας $E$, ακτινοβολίας που πέφτει κάθετα σε 
  επιφάνεια υγρού (π.χ. θάλασσα) είναι $ \frac{\Delta E}{\Delta s} = -a E(s) $, 
  όπου $s$ το βάθος και $a$ ο συντελεστής απορρόφησης (π.χ. $ a=0.1 $ : απορρόφηση 
  $ 10\% ανά μονάδα μήκους $), 
  \begin{enumerate}
    \item Βρείτε την ενέργεια ως συνάρτηση του βάθους, $ E(s) $ (Νόμος απορρόφησης
      Lambert)
    \item Για $ a=0.5 $ υπολογίστε το βάθος όπου έχει απορροφηθεί το $ 90\% $ της 
      προσπίπτουσας ακτινοβολίας $ E_{0} $, $(E(0)=E_{0}) $
    \item Αφού βρείτε μια μαθηματική έκφραση του $a$, υπολογίστε το ποσοστό $ \% $
      απορρόφησης ανά μονάδα μήκους, για $ a=0.5 $.
  \end{enumerate}
\end{exercise}
\end{mybox2}
\begin{solution}
  \begin{enumerate}
    \item Έχουμε ότι  
      \[ 
        \frac{\Delta E}{\Delta s} = - a E(s) \xrightarrow[\Delta t \to 0]{\lim} 
        \boxed{\dv{E}{s} = -a E(s)} \quad \text{χωριζομένων μεταβλητών} 
      \]
      Επομένως
      \[
        \frac{dE}{E} = -a ds \Rightarrow \int \frac{1}{E} \,{dE} = -a \int \,{ds} 
        \Rightarrow \ln{\abs{E}} = -as + c \Rightarrow \abs{E} = \mathrm{e}^{-as+c} 
        \Rightarrow E = \pm \mathrm{e}^{c} \cdot \mathrm{e}^{-as} 
      \] 
      Άρα, αν θέσουμε $ k = \pm \mathrm{e}^{c} $, έχουμε ότι 
      \[
        E(s) = k \mathrm{e}^{-as}  
      \] 
      Για $ s=0 $, έχουμε ότι 
      $ E(0)= k \mathrm{e}^{-a\cdot 0} \Rightarrow E_{0} = k $. Άρα 
      \[
        E(s) = E_{0} \mathrm{e}^{-as} 
      \] 
    \item Αφού έχει απορροφηθεί το $ 90\% $ της προσπίπτουσας ενέργειας, 
      αυτό σημαίνει ότι ζητάμε το βάθος όπου η ενέργεια που φτάνει είναι μόλις 
      $ 10\% $ της αρχικής ενέργειας. Άρα
      \[ 
        E(s) = 10\% E_{0} \Leftrightarrow E(s) = 0.1E_{0} \Leftrightarrow E_{0}
        \mathrm{e}^{-0.5s} = 0.1E_{0} \Leftrightarrow \mathrm{e}^{-0.5s} = 0.1
        \Leftrightarrow -0.5s = \ln{0.1} 
      \]
      άρα
      \[
        s = - \frac{\ln{0.1}}{0.5} \approx 4,64 \quad\text{μονάδες μήκους}
      \] 
    \item Έχουμε 
      \[
        E(s) = E_{0} \mathrm{e}^{-as} \Rightarrow \frac{E(s)}{E_{0}} =
        \mathrm{e}^{-as} \Rightarrow \ln{\frac{E(s)}{E_{0}}} = -as \Rightarrow 
        a = - \frac{\ln{\frac{E(s)}{E_{0}}}}{s}
      \] 
  \end{enumerate}
\end{solution}

\end{document}
