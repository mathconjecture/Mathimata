\documentclass[a4paper,12pt]{article}


\usepackage[english,greek]{babel}
\usepackage[utf8]{inputenc}

\usepackage{amsmath}
\usepackage{amssymb}
\usepackage{amsfonts}
\usepackage[top=2cm,bottom=2cm,left=2cm,right=2cm]{geometry}
\usepackage{graphicx}

\begin{document}

\begin{center}
\fbox{\Large\bfseries{Συναρτήσεις Πολλών Μεταβλητών}}
\end{center}

\vspace{2\baselineskip}

\begin{enumerate}
\item Να βρεθούν και να χαρακτηριστούν τα ακρότατα των παρακάτω συναρτήσεων.
\begin{enumerate}
\item $z=x^2+xy+2y^2+3$
\item $z=-x^2-y^2+6x+2y$
\item $z=e^{2x}-2x+2y^2=3$
\end{enumerate}

\item Έστω η συνάρτηση $z=(x-2)^4+(y-3)^4$
\begin{enumerate}

\item Δείξτε, διαισθητικά, ότι η $z$ έχει ένα ελάχιστο $z^*=0$ στο $(x^*,y^*)=(2,3)$.
\item Ικανοποιείται για αυτήν την τιμή η αναγκαία συνθήκη $1$ης τάξης?
\item Ικανοποιείται η αναγκαία συνθήκη $2$ης τάξης?
\item Βρείτε την τιμή του $d^2z$. Ικανοποιεί την αναγκαία συνθήκη δεύτερης τάξης για ελάχιστο?
\end{enumerate}

\item Εκφράστε κάθε τετραγωνική μορφή παρακάτω σαν ένα γινόμενο πινάκων το οποίο περιέχει έναν \underline{συμμετρικό} πίνακα.

\begin{enumerate}
\item $q=3u^2-4uv+7v^2$
\item $q=u^2+7uv+3v^2$
\item $q=8uv-u^2-31v^2$
\item $q=6xy-5y^2-2x^2$
\item $q=3u_1^2-2u_1u_2+4u_1u_3+5u_2^2+4u_3^2-2u_2u_3$
\item $q=-u^2+4uv-6uw-4v^2-7w^2$
\end{enumerate}

\item Από τις διακρίνουσες των συμμετρικών πινάκων των συντελεστών της προηγούμενης άσκησης να εξακριβώσετε από το κριτήριο ορίζουσας ποιες τετραγωνικές μορφές είναι θετικές και ποιες αρνητικά ορισμένες.

\item Βρείτε τις χαρακτηριστικές ρίζες των επόμενων πινάκων. Τι συμπεραίνετε για τα πρόσημα των αντίστοιχων τετραγωνικών μορφών?

\begin{enumerate}
\item $D=
\begin{pmatrix}
4 & 2\\
2 & 3
\end{pmatrix}$
\item $E=
\begin{pmatrix}
-2 & \phantom{-}2\\
\phantom{-}2 & -4
\end{pmatrix}$
\item $F=
\begin{pmatrix}
5 & 3\\
3 & 0
\end{pmatrix}$

\end{enumerate}

\item Βρείτε τα χαρακτηριστικά διανύσματα του πίνακα $\begin{pmatrix}
4 & 2\\
2 & 1
\end{pmatrix}$

\item Βρείτε τις ακρότατες τιμές, αν υπάρχουν, των παρακάτω συναρτήσεων. Να χαρακτηρίστε τα ακρότατα με τη μέθοδο των οριζουσών.

\begin{enumerate}
\item $z=x_1^2+3x_2^2-3x_1x_2+4x_2x_3+6x_3^2$
\item $z=29 - (x_1^2+x_2^2+x_3^2)$
\item $z=x_1x_3 + x_1^2-x_2 + x_2x_3+x_2^2+3x_3^2$
\item $e^{2x}+e^{-y}+e^{w^2}-(2x+2e^w-y+)$
\end{enumerate}

\item Απαντήστε στις παρακάτω ερωτήσεις σε σχέση με τους Εσσιανόυς πίνακες και τις χαρακτηριστικές ρίζες για τις συναρτήσεις της προηγούμενης άσκησης.

\begin{enumerate}
\item Ποια από τα παραπάνω προβλήματα δίνουν διαγώνιους Εσσιανούς πίνακες. Έχουν τα διαγώνια στοιχεία το ίδιο πρόσημο?
\item Τι συμπεραίνετε για την οριστικότητα του προσήμου του $d^2z$ από τις χαρακτηριστικές ρίζες?
\item  Είναι τα αποτελέσματα του κριτηρίου της χαρακτηριστικής ρίζας τα ίδια με αυτά του κριτηρίου ορίζουσας?
\end{enumerate}

\item Χρησιμοποιείστε τον αλγεβρικό ορισμό της κοιλότητας και της κυρτότητας για να χαρακτηρίσετε τις επόμενες συναρτήσεις.

\begin{enumerate}
\item $z=x^2$
\item $z=x_1^2+2x_2^2$
\item $z=2x^2-xy+y^2$
\end{enumerate}



\item {\bfseries Παρατήρηση:} Αν μια συνάρτηση είναι διαφορίσιμη (υπάρχουν οι παράγωγοι όλων των τάξεων και είναι συνεχείς) τότε ο αλγεβρικός ορισμός της κοιλότητας και της κυρτότητας γίνεται:

Μια διαφορίσιμη συνάρτηση $f(x)$ είναι κοίλη (κυρτή) αν και μόνον αν, για κάθε δεδομένο σημείο $u$ και κάθε άλλο σημείο $v$ στο Π.Ο.
\[
f(v)\leq (\geq) f(u)+f'(u)(v-u)\quad \text{για μια μεταβλητή} 
\]

Μια διαφορίσιμη συνάρτηση $f(x_1,\ldots,x_n)$ είναι κοίλη (κυρτή) αν και μόνον αν, για κάθε σημείο $u=(u_1,\ldots,u_n)$ και κάθε άλλο σημείο $v=(v_1,\ldots,v_n)$  στο Π.Ο.
\[
f(v)\leq (\geq) f(u)+\sum\limits_{j=1}^nf_j(u)(v_j-u_j)
\]
όπου $f_j(u)=\frac{\partial f}{\partial x_j}$ υπολογίζεται στο $u=(u_1,\ldots,u_n)$

\item Χρησιμοποιώντας τον παραπάνω ορισμό ελέξτε ως προς την κοιλότητα, κυρτότητα τις παρακάτω συναρτήσεις.

\begin{enumerate}
\item $z=-x^2$
\item $z=(x_1+x_2)^2$
\item $z=-xy$
\end{enumerate}

\item Χρησιμοποιείστε τη μέθοδο \textlatin{Lagrange} για να βρείτε τα ακρότατα και στη συνέχεια την πλαισιωμένη Εσσιανή για να τα χαρακτηρίσετε. Πως μεταβάλλεται η τιμή του ακρότατου, αν ((χαλαρώσουμε)) τον περιορισμό?

\begin{enumerate}
\item $z=xy, x+2y=2$
\item $z=x(y+4), x+y=8$
\item $z=x-3y-xy, x+y=6$
\item $z=7-y+x^2, x+y=0$
\end{enumerate}

\end{enumerate}


\end{document}