\input{preamble_ask.tex}
\input{definitions_ask.tex}

\pagestyle{askhseis}

% \linespread{1.3}
\renewcommand{\vec}{\mathbf}

\begin{document}

\begin{center}
  \minibox{\large \bfseries \textcolor{Col1}{θέματα στo Διπλό Ολοκλήρωμα}}
\end{center}

\vspace{\baselineskip}

\section*{Εμβαδό}

\begin{enumerate}
  \item \textbf{(Σεπ. 2019)} Να υπολογίσετε το εμβαδό του χωρίου που περικλείεται από 
    τις καμπύλες $ xy=2 $, $ 4y=x^2 $ και $ y=4 $.

    \hfill Απ: $ \frac{28}{3} - 2 \ln{4} $

  \item \textbf{(Ιουν. 2019)} Να υπολογίσετε το εμβαδό του χωρίου που περικλείεται από 
    τις καμπύλες $ y= \mathrm{e}^{x} $, $ y=x $, $ y=4 $ και $ x=0 $ με $ y<
    \mathrm{e}^{x} $. 

    % \hfill Απ:  

  \item \textbf{(Ιουν. 2017)} Να υπολογίσετε το εμβαδό της επίπεδης περιοχής που 
    βρίσκεται μεταξύ του κύκλου με εξίσωση $ x^{2}+y^{2}=16 $ και της έλλειψης 
    $ \frac{x^{2}}{9} + \frac{y^{2}}{4} =1 $ με χρήση κατάλληλου διπλού ολοκληρώματος.

    % \hfill Απ:  
\end{enumerate}


\section*{Διπλά Ολοκληρώματα}

\begin{enumerate}
  \item \textbf{(Σεπ. 2017)} 
    \begin{enumerate}[i)]
      \item Να σχεδιαστεί η περιοχή $T$ του επιπέδου $ Oxy $ στην 
        οποία αναφέρεται το διπλό ολοκλήρωμα 
        \[ 
          \int_{0} ^{\ln{2}} \,{dx} \int _{0}^{\mathrm{e}^{x}} \mathrm{e}^{-x} \,{dy} 
        \]
      \item Να γίνει αντιστροφή των μεταβλητών και να υπολογιστεί ακολούθως το 
        ολοκλήρωμα με τα νέα ορία.
    \end{enumerate}

  \item \textbf{(Σεπ. 2016)} 
    \begin{enumerate}[i)]
      \item Έστω το διπλό ολοκλήρωμα $ \iint_{T} \,{dy}{dx} = \int _{0}^{\sqrt{3}}
        \,{dx} \int _{\frac{x}{\sqrt{3}}}^{\sqrt{4-x^{2}}} \,{dy}$.
        \begin{enumerate}[i)]
          \item Σχεδιάστε την περιοχή ολοκλήρωσης $T$.
          \item Ξαναγράψτε το ολοκλήρωμα με την αντίστροφη σειρά ολοκλήρωσης.
          \item Ξαναγράψτε το ολοκλήρωμα χρησιμοποιώντας πολικές συντεταγμένες.
          \item Υπολογίστε το ολοκλήρωμα $ \iint_{T} \,{dy}{dx} $.
        \end{enumerate}

      \item Να γίνει αντιστροφή των μεταβλητών και να υπολογιστεί ακολούθως το 
        ολοκλήρωμα με τα νέα ορία.
    \end{enumerate}
\end{enumerate}

\end{document}

