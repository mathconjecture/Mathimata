\documentclass[a4paper,12pt]{article}
\usepackage{etex}
%%%%%%%%%%%%%%%%%%%%%%%%%%%%%%%%%%%%%%
% Babel language package
\usepackage[english,greek]{babel}
% Inputenc font encoding
\usepackage[utf8]{inputenc}
%%%%%%%%%%%%%%%%%%%%%%%%%%%%%%%%%%%%%%

%%%%% math packages %%%%%%%%%%%%%%%%%%
\usepackage{amsmath}
\usepackage{amssymb}
\usepackage{amsfonts}
\usepackage{amsthm}
\usepackage{proof}

\usepackage{physics}

%%%%%%% symbols packages %%%%%%%%%%%%%%
\usepackage{bm} %for use \bm instead \boldsymbol in math mode 
\usepackage{dsfont}
\usepackage{stmaryrd}
%%%%%%%%%%%%%%%%%%%%%%%%%%%%%%%%%%%%%%%


%%%%%% graphicx %%%%%%%%%%%%%%%%%%%%%%%
\usepackage{graphicx}
\usepackage{color}
%\usepackage{xypic}
\usepackage[all]{xy}
\usepackage{calc}
\usepackage{booktabs}
\usepackage{minibox}
%%%%%%%%%%%%%%%%%%%%%%%%%%%%%%%%%%%%%%%

\usepackage{enumerate}

\usepackage{fancyhdr}
%%%%% header and footer rule %%%%%%%%%
\setlength{\headheight}{14pt}
\renewcommand{\headrulewidth}{0pt}
\renewcommand{\footrulewidth}{0pt}
\fancypagestyle{plain}{\fancyhf{}
\fancyhead{}
\lfoot{}
\rfoot{\small \thepage}}
\fancypagestyle{vangelis}{\fancyhf{}
\rhead{\small \leftmark}
\lhead{\small }
\lfoot{}
\rfoot{\small \thepage}}
%%%%%%%%%%%%%%%%%%%%%%%%%%%%%%%%%%%%%%%

\usepackage{hyperref}
\usepackage{url}
%%%%%%% hyperref settings %%%%%%%%%%%%
\hypersetup{pdfpagemode=UseOutlines,hidelinks,
bookmarksopen=true,
pdfdisplaydoctitle=true,
pdfstartview=Fit,
unicode=true,
pdfpagelayout=OneColumn,
}
%%%%%%%%%%%%%%%%%%%%%%%%%%%%%%%%%%%%%%

\usepackage[space]{grffile}

\usepackage{geometry}
\geometry{left=25.63mm,right=25.63mm,top=36.25mm,bottom=36.25mm,footskip=24.16mm,headsep=24.16mm}

%\usepackage[explicit]{titlesec}
%%%%%% titlesec settings %%%%%%%%%%%%%
%\titleformat{\chapter}[block]{\LARGE\sc\bfseries}{\thechapter.}{1ex}{#1}
%\titlespacing*{\chapter}{0cm}{0cm}{36pt}[0ex]
%\titleformat{\section}[block]{\Large\bfseries}{\thesection.}{1ex}{#1}
%\titlespacing*{\section}{0cm}{34.56pt}{17.28pt}[0ex]
%\titleformat{\subsection}[block]{\large\bfseries{\thesubsection.}{1ex}{#1}
%\titlespacing*{\subsection}{0pt}{28.80pt}{14.40pt}[0ex]
%%%%%%%%%%%%%%%%%%%%%%%%%%%%%%%%%%%%%%

%%%%%%%%% My Theorems %%%%%%%%%%%%%%%%%%
\newtheorem{thm}{Θεώρημα}[section]
\newtheorem{cor}[thm]{Πόρισμα}
\newtheorem{lem}[thm]{λήμμα}
\theoremstyle{definition}
\newtheorem{dfn}{Ορισμός}[section]
\newtheorem{dfns}[dfn]{Ορισμοί}
\theoremstyle{remark}
\newtheorem{remark}{Παρατήρηση}[section]
\newtheorem{remarks}[remark]{Παρατηρήσεις}
%%%%%%%%%%%%%%%%%%%%%%%%%%%%%%%%%%%%%%%




\newcommand{\vect}[2]{(#1_1,\ldots, #1_#2)}
%%%%%%% nesting newcommands $$$$$$$$$$$$$$$$$$$
\newcommand{\function}[1]{\newcommand{\nvec}[2]{#1(##1_1,\ldots, ##1_##2)}}

\newcommand{\linode}[2]{#1_n(x)#2^{(n)}+#1_{n-1}(x)#2^{(n-1)}+\cdots +#1_0(x)#2=g(x)}

\newcommand{\vecoffun}[3]{#1_0(#2),\ldots ,#1_#3(#2)}

\newcommand{\mysum}[1]{\sum_{n=#1}^{\infty}


\thispagestyle{empty}

\begin{document}

\begin{center}
    \minibox[frame]{\large\bfseries Ασκήσεις στο Άθροισμα Διανυσματικών Χώρων} 
\end{center}

\vspace{\baselineskip}

\begin{enumerate}
    \item Έστω $ V = \mathbb{R}^{3} $ και $ W_{1} = \{(x_{1},0,x_{2})\in 
        \mathbb{R}^{3} \mid x_{1}, x_{2} \in \mathbb{R}\}  $ και 
        $ W_{2} = \{(0,y_{1},y_{2}) \in \mathbb{R}^{3} \mid y_{1}, y_{2} \in
        \mathbb{R}  \}  $. Να δείξετε ότι 
        \begin{enumerate}[i)]
            \item $ W_{1} \leq \mathbb{R}^{3} $
            \item $ W_{2} \leq \mathbb{R}^{3} $
            \item $ \mathbb{R}^{3} = W_{1}+W_{2} $
        \end{enumerate}

    \item Έστω $ V = \mathbb{R}^{3} $ και οι υπόχωροί του $ W_{1} = 
        \{ (x,0,0) \in \mathbb{R}^{3} \mid x \in \mathbb{R}\} $, $ W_{2} = 
        \{ (0,0,z) \mid z \in \mathbb{R} \} $ και 
        $ W_{3} = \{ (x,y,x) \mid x,y \in \mathbb{R} \} $. Να δείξετε ότι 
        \begin{enumerate}[i)]
            \item $ W_{1} \cap W_{2} = W_{2} \cap W_{3} = W_{3} \cap W_{1} = 
                \{ \mathbf{0} \} $
            \item $ \mathbb{R}^{3} = W_{1}+W_{2}+W_{3} $
            \item Να εξετάσετε αν $ \mathbb{R}^{3} = W_{1} \oplus W_{2} \oplus W_{3} $
        \end{enumerate}

    \item Έστω διανυσματικός χώρος $V$ και $ W_{1}, W_{2} $ υπόχωροί του, τέτοιοι ώστε 
        $ V = W_{1} \oplus W_{2} $. Αν $ W_{1} \leq W \leq V $, να δείξετε ότι 
        $ W = W_{1} \oplus (W_{2} \cap W_{3}) $.

        \hfill \textbf{Υπόδειξη:} Για να δείξετε ότι $ W = W_{1} + (W_{2}\cap W_{3}) $, 
        δείξτε ότι $ W \leq W_{1} + (W_{2} \cap W_{3}) $ και 
        $ W_{1} + (W_{2} \cap W_{3}) \leq W $
\end{enumerate}

\end{document}


