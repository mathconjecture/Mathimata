\documentclass[a4paper,table]{report}
\input{preamble_ask.tex}
\input{definitions_ask.tex}

\geometry{top=1.5cm}
\pagestyle{askhseis}

\begin{document}

\begin{center}
  \minibox{\bfseries\large\color{Col1} Μερική Παράγωγος}
\end{center} 

\vspace{\baselineskip} 

\section*{Ορισμός}

\begin{enumerate}
  \item Έστω η συνάρτηση $ f(x,y) = xy $. Να υπολογιστούν με τη 
    βοήθεια του ορισμού οι $ f_{x}(1,1) $ και η $ f_{y}(1,1) $. 

    \hfill Απ: 
    \begin{tabular}{l}
      $f_{x}(1,1) = 1$ \\
      $f_{y}(1,1) = 1$
    \end{tabular} 

  \item Να υπολογιστούν οι μερικές παράγωγοι 1ης τάξης της συνάρτησης
    $
    f(x,y) = 
    \begin{cases}
      x^{2} \sin{\frac{y}{x}}, & x \neq 0 \\
      0, & x = 0 
    \end{cases}
    $ 

    \hfill Απ: 
    $  f_{x} = 
    \begin{cases}
      2x \sin{\frac{y}{x}} - y \cos{\frac{y}{x}}, & 
      x \neq 0  \\ 
      0, & x = 0 
    \end{cases} $ \quad 
    και \quad
    $ f_{y} = 
    \begin{cases}
      x \cos{\frac{y}{x}}, & x \neq 0 \\
      0, & x = 0 
    \end{cases} $    

\end{enumerate}


\section*{Διαφορισιμότητα}

\begin{enumerate}
  \item Να εξεταστεί αν οι παρακάτω συναρτήσεις είναι διαφορίσιμες στο σημείο 
    $ (0,0) $.

    \begin{enumerate*}[i),itemjoin=\hspace{1.5cm}]
      \item  $ f(x,y) = 
        \begin{cases} 
          \frac{xy}{\sqrt{x^{2}+y^{2}}}, & (x,y) \neq (0,0) \\
          0, & (x,y) = (0,0)
        \end{cases} $ 
      \item $ g(x,y) = 
        \begin{cases} 
          \frac{(x+y)^{2}}{x^{2}+y^{2}}, & (x,y) \neq (0,0) \\
          0, & (x,y) = (0,0)
        \end{cases} $
    \end{enumerate*}
    \hfill Απ: 
    \begin{enumerate*}[i),itemjoin=\hspace{10pt}]
      \item όχι
      \item όχι
    \end{enumerate*}

  \item Να εξετάσετε αν η συνάρτηση 
    $
    f(x,y) = 
    \begin{cases} 
      xy \frac{x^{2}-y^{2}}{x^{2}+y^{2}}, & (x,y) 
      \neq (0,0) \\ 
      0, & (x,y) = (0,0)
    \end{cases}
    $ 
    είναι διαφορίσιμη στο σημείο $ (0,0) $. 
    \hfill Απ: ναι 
\end{enumerate}


\end{document}
