\input{preamble_ask.tex}
\input{definitions_ask.tex}
\input{myboxes}


\pagestyle{askhseis}
\everymath{\displaystyle}

\begin{document}

\begin{center}
\minibox{\large\bfseries \textcolor{Col1}{Γενικευμένα Ολοκληρώματα (λύσεις)}}
\end{center}

\vspace{\baselineskip}

\begin{enumerate}[i)]

  \item $ \boxed{\int _{0}^{+\infty} \frac{1}{1+x^2}  \,{dx}} $
    \begin{solution}
      \[
        \int _{0}^{+\infty} \frac{1}{1+x^2}  \,{dx} = \bigl[\arctan x\bigr]_{0}^{\infty} 
        = \lim_{x \to \infty} \arctan x - \arctan{0} = \frac{\pi}{2} - 0 = \frac{\pi}{2}
      \] 
    \end{solution}

  \item $ \boxed{\int _{1}^{+\infty} \frac{1}{\sqrt{x}}\,{dx}} $
    \begin{solution}
      \[
        \int _{1}^{+\infty} \frac{1}{\sqrt{x}}\,{dx} = 
        \bigl[2 \sqrt{x}\bigr]_{1}^{\infty} = 
        2 \bigl[\sqrt{x}\bigr]_{1}^{\infty} = 
        2 \bigl(\lim_{x \to \infty} \sqrt{x} - \sqrt{1}\bigr) = 2 (\infty - 1) = 2 \cdot
        \infty = \infty 
      \]
    \end{solution}

  \item $ \boxed{\int _{0}^{4} \frac{1}{\sqrt{4-x}} \,{dx}} = \int _{0}^{4^{-}}
    \frac{1}{\sqrt{4-x}} \,{dx} $
    \begin{solution}
      \[
        \int _{0}^{4^{-}} \frac{1}{\sqrt{4-x}} \,{dx} = \Bigl[
        \frac{2\sqrt{4-x}}{-1} \Bigr]_{0}^{4^{-}} = -2 [\sqrt{4-x}]_{0}^{4^-}
        = -2 \bigl(\lim_{x \to 4^{-}} \sqrt{4-x} - \sqrt{4}\bigr) = -2 (0 - 2) = 4
      \]
    \end{solution}

  \item $ \boxed{\int _{0}^{1} \frac{1}{\sqrt{1-x^2}} \,{dx}} = \int _{0}^{1^{-}} 
    \frac{1}{\sqrt{1-x^2}} \,{dx} $
    \begin{solution}
      \[
        \int _{0}^{1^{-}} \frac{1}{\sqrt{1-x^2}} \,{dx} = 
        \bigl[\arcsin x\bigr]_{0}^{1^-} = 
        \lim_{x \to 1^{-}} \arcsin{x} - \arcsin{0} = \arcsin{1} - 0 = \frac{\pi}{2}
      \]
    \end{solution}

  \item $ \boxed{\int _{-1}^{1} \frac{1}{x^{2/3}} \,{dx}} = 
    \int _{-1}^{0^-} \frac{1}{x^{2/3}} \,{dx} + \int _{0^{+}}^{1} 
    \frac{1}{x^{2/3}} \,{dx} $
    \begin{solution}
      Για το αόριστο, έχουμε: $ \int \frac{1}{x^{2/3}} \,{dx} = \int x^{-2/3} \,{dx} =
      \frac{x^{- \frac{2}{3} +1}}{- \frac{2}{3}+1} = \frac{x^{1/3}}{1/3} = 3
      x^{1/3} $
      Άρα
      \begin{align*}
        \int _{-1}^{0^-} \frac{1}{x^{2/3}} \,{dx} + \int _{0^{+}}^{1} 
        \frac{1}{x^{2/3}} \,{dx} 
        &= \bigl[3 x^{1/3}\bigr]_{-1}^{0^-} + \bigl[3 x^{1/3}\bigr]_{0^+}^{1} =  
        3\bigl[x^{1/3}\bigr]_{-1}^{0^-} + 3\bigl[x^{1/3}\bigr]_{0^+}^{1} \\ 
        &=  3 (\lim_{x \to 0^-} \sqrt[3]{x} - \sqrt[3]{-1}) + 
        ( \sqrt[3]{1} - \lim_{x \to 0^+} \sqrt[3]{x}) = 3 (0-(-1)) + 3(1-0) = 6 
      \end{align*}
    \end{solution}

  \item $ \boxed{\int _{- \infty}^{-2} \frac{2}{x^{2}-1} \,{dx}} $ 
    \begin{solution}
      Για το αόριστο, έχουμε: $ \int \frac{2}{x^{2}-1}  \,{dx} $, είναι ολοκλήρωμα Ρητής.
      \[
        \frac{2}{x^{2}-1} = \frac{2}{(x-1)(x+1)} = \frac{A}{x-1} + \frac{B}{x+1} =
        \frac{1}{x-1} + \frac{-1}{x+1}
      \] 
      Άρα
      \[
        \int \frac{2}{x^2-1} \,{dx} = \int \Bigl(\frac{1}{x-1} - \frac{1}{x+1}\Bigr) 
        \,{dx} = \ln{\abs{x-1} - \ln{\abs{x+1}}} = \ln{\abs{\frac{x-1}{x+1}}}  
      \] 
      Οπότε
      \[
        \int _{- \infty}^{-2} \frac{2}{x^{2}-1} \,{dx} =
        \Bigl[\ln{\abs{\frac{x-1}{x+1}}}\Bigr]_{- \infty}^{-2} =
        \Bigl(\ln{\abs{\frac{-3}{-1}}} -
        \lim_{x \to - \infty} \ln{\abs{\frac{x-1}{x+1}}}\Bigr) = ( \ln{3} - \ln{1}) =
        \ln{3}
      \] 
    \end{solution}

  \item $ \boxed{\int _{2}^{\infty} \frac{2}{x^{2}-1} \,{dx}} $
    \begin{solution}
      \[
        \int _{2}^{\infty} \frac{2}{x^{2}-1} \,{dx} = \Bigl[\ln{\abs{\frac{x-1}{x+1}}}
        \Bigr]_{2}^{\infty} = \Bigl(\lim_{x \to \infty} 
        \ln{\abs{\frac{x-1}{x+1}} - \ln{3}}\Bigr) = \Bigl(
        \ln{1} - \ln{\frac{1}{3}}\Bigr) = (\ln{1} - \ln{1} + \ln{3}) = \ln{3}
      \] 
    \end{solution}

  \item $ \boxed{\int _{- \infty}^{2} \frac{2}{x^{2}+4}\,{dx}} $
    \begin{solution}
      Για το αόριστο έχουμε $ \int \frac{2}{x^{2}+4} \,{dx} = 2\int \frac{1}{2^{2}+x^{2}}
      \,{dx} = 2 \Bigl(\frac{1}{2} \arctan{\frac{x}{2}}\Bigr) =
      \arctan{\frac{x}{2}} $
      Άρα
      \[
        \int _{- \infty}^{2} \frac{2}{x^{2}+4}\,{dx} = 
        \bigl[\arctan{\frac{x}{2}}\bigr]_{- \infty}^{2} = 
        \arctan{1} - \lim_{x \to - \infty} \arctan{\frac{x}{2}} =
        \frac{\pi}{4} - \bigl(- \frac{\pi}{2}\bigr) = \frac{3 \pi}{4} 
      \]
    \end{solution}

  \item $ \boxed{\int _{- \infty}^{+ \infty} \frac{2x}{(x^{2}+1)^{2}} \,{dx}} = 
    \int _{- \infty}^{0} \frac{2x}{(x^{2}+1)^{2}} \,{dx} + 
    \int _{0}^{+ \infty} \frac{2x}{(x^{2}+1)^{2}} \,{dx} $
    \begin{solution}
      Για το αόριστο, θέτουμε $ u = x^{2}+1 \Rightarrow du = 2x dx $ και με 
      αντικατάσταση, έχουμε:
      \[
        \int \frac{2x}{(x^{2}+1)^{2}} \,{dx} = \int \frac{1}{u^{2}} \,{du} = -
        \frac{1}{u} = - \frac{1}{x^{2}+1}
      \] 
      Άρα
      \begin{align*}
        \int _{- \infty}^{0} \frac{2x}{(x^{2}+1)^{2}} \,{dx} + 
        \int _{0}^{+ \infty} \frac{2x}{(x^{2}+1)^{2}} \,{dx} 
    &= 
    \Bigl[- \frac{1}{x^{2}+1} \Bigr]_{- \infty}^{0} + \Bigl[- \frac{1}{x^{2}+1}
    \Bigr]_{0}^{+ \infty} = 
    -\Bigl[\frac{1}{x^{2}+1} \Bigr]_{- \infty}^{0} - \Bigl[\frac{1}{x^{2}+1}
    \Bigr]_{0}^{+ \infty} \\ 
    &= 
    -\Bigl(1 - \lim_{x \to - \infty} \frac{1}{x^{2}+1}\Bigr) - 
    \Bigl( \lim_{x \to \infty} \frac{1}{x^{2}+1} - 1\Bigr) = -(1-0)-(0-1) = 0
      \end{align*} 
    \end{solution}

  \item $ \boxed{\int _{- \infty}^{+ \infty} \frac{x}{(x^{2}+4)^{3/2}} \,{dx}} = \int _{- \infty
    }^{0} \frac{x}{(x^{2}+4)^{3/2}} \,{dx} + \int _{0}^{+ \infty} 
    \frac{x}{(x^{2}+4)^{3/2}} \,{dx} $
    \begin{solution}
      Για το αόριστο, θέτουμε $ u = x^{2}+4 \Rightarrow du = 2x dx \Rightarrow x dx = du
      /2 $ και με 
      αντικατάσταση, έχουμε:
      \[
        \int \frac{x}{(x^{2}+4)^{3/2}} \,{dx} = \int \frac{du/2}{u^{3/2}} = 
        \frac{1}{2} \int u^{-3/2} \,{du} = \frac{1}{2} \frac{u^{- \frac{3}{2}
        + 1}}{- \frac{3}{2} + 1} = \frac{1}{2} \frac{u^{- \frac{1}{2}}}{-
        \frac{1}{2}} = - u^{-1/2} = - \frac{1}{\sqrt{u}} = - \frac{1}{\sqrt{x^{2}+4}} 
      \] 
      Άρα
      \begin{align*}
        \int _{- \infty}^{0} \frac{x}{(x^{2}+1)^{3/2}} \,{dx} + 
        \int _{0}^{+ \infty} \frac{x}{(x^{2}+1)^{3/2}} \,{dx} 
    &= 
    \Bigl[- \frac{1}{\sqrt{x^{2}+4}} \Bigr]_{- \infty}^{0} + 
    \Bigl[- \frac{1}{\sqrt{x^{2}+4}} \Bigr]_{0}^{+ \infty} = 
    -\Bigl[\frac{1}{\sqrt{x^{2}+4}} \Bigr]_{- \infty}^{0} - 
    \Bigl[\frac{1}{\sqrt{x^{2}+4}} \Bigr]_{0}^{+ \infty} \\ 
    &= -\Bigl(\frac{1}{2} - \lim_{x \to - \infty} \frac{1}{\sqrt{x^{2}+4}}\Bigr) - 
    \Bigl( \lim_{x \to \infty} \frac{1}{\sqrt{x^{2}+4}} - \frac{1}{2}\Bigr) \\ 
    &= -\Bigl(\frac{1}{2} -0\Bigr)-\Bigl(0- \frac{1}{2}\Bigr) = 0
      \end{align*}
    \end{solution}

  \item $ \boxed{\int _{0}^{2} \frac{x+1}{\sqrt{4-x^2}} \,{dx}} = \int _{0}^{2^-}
    \frac{x+1}{\sqrt{4-x^{2}}} \,{dx} $ 
    \begin{solution}
      Για το αόριστο, έχουμε $ \int \frac{x+1}{\sqrt{4- x^{2}}} \,{dx} = 
      \int \frac{x}{\sqrt{4-x^{2}}} \,{dx} + \int \frac{1}{\sqrt{4-x^{2}}} \,{dx} 
      = - \sqrt{4-x^{2}} + \arcsin{\frac{x}{2}} $ 

      όπου για το ολοκλήρωμα $ \int \frac{x}{\sqrt{4-x^{2}}} \,{dx}  $ θέτουμε $ u =
      \sqrt{4-x^{2}} \Rightarrow du = \frac{1}{2 \sqrt{4-x^{2}}} (-2x)dx \Rightarrow
      \frac{x}{\sqrt{4-x^{2}}} dx = - du $, 

      οπότε με αντικατάσταση έχουμε: 
    $\int \frac{x}{\sqrt{4-x^{2}}} \,{dx} = \int - \,{du} = -u = - \sqrt{4-x^{2}}$ 

    και το ολοκλήρωμα $ \int \frac{1}{\sqrt{4-x^{2}}} \,{dx} = \int
  \frac{1}{\sqrt{2^{2}-x^{2}}} \,{dx} = \arcsin{\frac{x}{2}} $

  Άρα 
  \begin{align*}
    \int _{0}^{2^-} \frac{x+1}{\sqrt{4-x^{2}}}  \,{dx} 
    &= \Bigl[- \sqrt{4-x^{2}} +
     \arcsin{\frac{x}{2}} \Bigr]_{0}^{2^-} = - \lim_{x \to 2^-} \sqrt{4-x^2} +
     \lim_{x \to 2^-} \arcsin{\frac{x}{2}} + \sqrt{4} - \arcsin{0} \\
    &= 0 + \arcsin{1} + 2 = \frac{\pi}{2} + 2 = \frac{4 + \pi}{2}
     \end{align*}
  \end{solution}


  \item $ \boxed{\int_0^{\infty} \frac{1}{\sqrt{x} (1+x)} \,{dx}} $
    \begin{solution}
      Για το αόριστο θέτουμε $ u= \sqrt{x} \Rightarrow u^{2}=x \Rightarrow dx = 2udu $ 
      και με αντικατάσταση, έχουμε: 
      \[ 
        \int \frac{1}{\sqrt{x} (1+x)} \,{dx} = \int \frac{1}{u(1+u^{2})} \cdot 2u \,{du}
        = 2\int \frac{1}{1+u^{2}} \,{du} = \arctan{u} = \arctan{(\sqrt{x})}
      \]
      Άρα 
      \[ 
        \int _{0}^{\infty} \frac{1}{\sqrt{x} (1+x)} \,{dx} = \left[2
        \arctan{\sqrt{x}}\right]_{0}^{\infty} = 2( \lim_{x \to \infty}
        \arctan{\sqrt{x}} - \arctan{0}) = 2\left( \frac{\pi}{2} - 0\right) = \pi
      \]
    \end{solution}
  \item $ \boxed{\int _{- \infty}^{0} x \mathrm{e}^{x} \,{dx}} $
    \begin{solution}
      Για το αόριστο με τη μέθοδο <<κατά παράγοντες>> βρίσκουμε
      \[
        \int x \mathrm{e}^{x} \,{dx} = \mathrm{e}^{x} (x-1) 
      \]
      Άρα 
      \[
        \int _{-\infty}^{0} x \mathrm{e}^{x} \,{dx} =
        \left[\mathrm{e}^{x}(x-1)\right]_{-\infty}^{0} = (-1 - \lim_{x \to - \infty}
        \mathrm{e}^{x}(x-1)) = (-1 - 0) = -1 
      \]
Για το όριο $ \lim_{x \to - \infty} \mathrm{e}^{x} (x-1) $ έχουμε:
\[
  \lim_{x \to -\infty} \mathrm{e}^{x} (x-1) \overset{0 \cdot (-\infty)}{=}  
   \lim_{x \to - \infty} \frac{x-1}{\mathrm{e}^{-x}}
  \overset{(\frac{- \infty}{\infty})}{=} \lim_{x \to - \infty} \frac{1}{-
    \mathrm{e}^{-x}} \overset{(\frac{1}{\infty})}{=} 0   
\] 
    \end{solution}

  \item $ \boxed{\int _{- \infty }^{\infty } 2x \mathrm{e}^{-x^{2}} \,{dx}} = \int _{- \infty
    }^{0} 2x \mathrm{e}^{-x^{2}} \,{dx} + \int _{0}^{\infty} 2x \mathrm{e}^{-x^{2}} \,{dx} 
    $
    \begin{solution}
      Για το αόριστο θέτουμε $ u = - x^{2} \Rightarrow du = -2xdx $ και με αντικατάσταση, 
      έχουμε:
      \[
        \int - \mathrm{e}^{u} \,{du} = - \mathrm{e}^{u} = - \mathrm{e}^{-x^{2}}
      \] 
      Άρα 
      \[ 
        \int _{- \infty }^{\infty } 2x \mathrm{e}^{-x^{2}} \,{dx} = 
        [- \mathrm{e}^{-x^{2}} ]_{- \infty}^{0} + [- \mathrm{e}^{-x^{2}} ]_{0}^{\infty} 
      = - (1 - \lim_{x \to - \infty} \mathrm{e}^{-x^{2}} ) - ( \lim_{x \to \infty}
      \mathrm{e}^{-x^{2}} - 1) = - (1 - 0) - (0-1) = 0
      \]
    \end{solution}

  \item $ \boxed{\int _{0}^{1} x \ln{x} \,{dx}} = \int _{0^{+}}^{1} x \ln{x} \,{dx}  $
    \begin{solution}
      Για το αόριστο, με μέθοδο "κατά παράγοντες" βρίσκουμε
      \[
        \int x \ln{x} \,{dx} = \frac{x^{2} \ln{x}}{2} - \frac{x^{2}}{4}
      \] 
      Άρα
      \[
        \int _{0^{+}}^{1} x \ln{x} \,{dx} = \left[\frac{x^{2} \ln{x}}{2} -
          \frac{x^{2}}{4}\right]_{0^{+}}^{1} = -\frac{1}{4} - \lim_{x \to 0^{+}} \left(
        \frac{x^{2} \ln{x}}{2} - \frac{x^{2}}{4}\right) = - \frac{1}{4} - 0 = -
        \frac{1}{4}
      \] 
      Για το όριο $ \lim_{x \to 0^{+}} x^{2} \ln{x} $ έχουμε
      \[
        \lim_{x \to 0^{+}} x^{2} \ln{x} \overset{0 (- \infty)}{=} 
        \lim_{x \to 0^{+}} \frac{\ln{x}}{\frac{1}{x^{2}}} 
        \overset{(\frac{- \infty}{\infty})}{=}  
        \lim_{x \to 0^{+}} \frac{\frac{1}{x}}{-\frac{2}{x^{3}}} = \lim_{x \to 0^{+}} 
        - \frac{x^{2}}{2} = 0
      \] 
    \end{solution}

  \item $ \boxed{\int _{0}^{2} \frac{1}{\sqrt{4-x^{2}}} \,{dx}} = 
    \int _{0}^{2^{-}} \frac{1}{\sqrt{4-x^{2}}} \,{dx} $
    \begin{solution}
      \[
        \int _{0}^{2^{-}} \frac{1}{\sqrt{4-x^{2}}} \,{dx} = \int _{0}^{2^{-}}
        \frac{1}{\sqrt{2^{2}-x^{2}}}  \,{dx} = \left[\arcsin{\frac{x}{2}
      }\right]_{0}^{2^{-}} = \arcsin{1} - \arcsin{0} = \frac{\pi}{2} - 0 =
      \frac{\pi}{2}
      \]
    \end{solution}


  \item $ \boxed{\int _{0}^{1} \frac{4x}{\sqrt{1 - x^{4}}} \,{dx}} =  \int _{0}^{1^{-}}
    \frac{4x}{\sqrt{1 - x^{4}}} \,{dx} = \int _{0}^{1^{-}} \frac{4x}{\sqrt{1 -
      (x^{2})^{2}}} \,{dx} $ 
      \begin{solution}
        Για το αόριστο, θέτουμε $ u = x^{2} \Rightarrow du = 2x dx $
        και με αντικατάσταση έχουμε:
        \[
          \int \frac{4x}{\sqrt{1 -(x^{2})^{2}} } \,{dx} = \int \frac{2}{ \sqrt{1 -u^{2}}}
          \,{du} = 2 \arcsin{u} = 2 \arcsin{x^{2}}
        \] 
        Άρα
        \[
          \int _{0}^{1} \frac{4x}{\sqrt{1 - x^{4}}} \,{dx} = \bigl[2
          \arcsin{x^{2}}\bigr]_{0}^{1^{-}} = 2\bigl(\lim_{x \to 1^{-}} \arcsin{x^{2}} - \arcsin{0}\bigr)
          = 2\left( \frac{\pi}{2} - 0\right) = \pi
        \]
      \end{solution}
\end{enumerate}


\end{document}

