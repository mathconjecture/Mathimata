\input{preamble_ask.tex}
\input{definitions_ask.tex}


\pagestyle{askhseis}
\everymath{\displaystyle}

\begin{document}

\begin{center}
  {\color{Col1}\minibox{\large\bfseries Ασκήσεις στις 
  Μερικές Παραγώγους}}
\end{center}

\section*{Κανόνες Παραγώγισης}

\begin{enumerate}
  \item Με τη βοήθεια των κανόνων παραγώγισης να υπολογιστούν οι μερικές 
    παράγωγοι των συναρτήσεων:

    \begin{enumerate}[i)]
      \item $f(x,y)=y\sin (xy)$ \hfill Απ: \begin{tabular}{l}
          $f_x=y^2\cos(xy)$ \\ 
          $f_y=\sin(xy)+yx\cos(xy)$
        \end{tabular}

      \item $f(x,y)=y\ln(x+y)$\hfill Απ: \begin{tabular}{l}
          $f_x=\frac{y}{x+y}$ \\ 
          $f_y=\ln(x+y)+\frac{y}{x+y}$
        \end{tabular}

      \item $ f(x,y,z) = (x+y^{2}) \sin{(xz)} $ \hfill Απ: \begin{tabular}{l}
          $ f_{x} = \sin{(xz)} + z(x+y^{2}) \sin{(xz)} $ \\
          $ f_{y} = 2y \sin{(xz)} $ \\
          $ f_{z} = x(x+y^{2}) \sin{(xz)} $
        \end{tabular} 
    \end{enumerate}

  \item Έστω η συνάρτηση $ f(x,y) = x^{2} \sin{(x+y)} $. Να υπολογίσετε τις μερικές 
    παραγώγους 1ης και 2ης τάξης.

    \hfill Απ: \begin{tabular}{l}
      $ f_{x} = 2x \sin{(x+y)} + x^{2} \cos{(x+y)} $ \\
      $ f_{y} = x^{2} \cos{(x+y)} $ \\
      $ f_{xx} =  -x^{2} \sin{(x+y)} $ \\
      $ f_{yy} = 2 \sin{(x+y)} + 4x \cos{(x+y)} -x^{2} \sin{(x+y)} $ \\
      $ f_{xy}=f_{yx} =  2x \cos{(x+y)} -x^{2} \sin{(x+y)} $
    \end{tabular}

  \item Έστω η συνάρτηση $f(x,y)=\ln\left(\cos y+x\cos x\right)$.  Να υπολογισθεί 
    η $ f_{xy} $ στο σημείο $(\pi,-\pi/2)$.  \hfill Απ: $\frac{1}{\pi^2}$

    % \item Έστω η συνάρτηση $ f(x,y) = x^{y} $. Να δείξετε ότι ισχύει ότι το θεώρημα 
    %   Schwartz, δηλαδή ότι $ f_{xy} = f_{yx} $.
\end{enumerate}


\section*{Διαφορικό}

\begin{enumerate}
  \item Να υπολογίσετε το ολικό διαφορικό 1ης τάξης, της συνάρτησης 
    $f(x,y)=\ln(xy)+\cos(y^2)$ 

    \hfill Απ: $df=\frac{dx}{x}+\left(\frac{1}{y}-2y\sin(y^2)\right)dy$

  \item Να υπολογίσετε τη διαφορά $ \Delta f $ και το ολικό διαφορικό $ df $ της 
    συνάρτησης $ f(x,y) = \sin{(x+y)} $, στο σημείο $ (0,0) $, όταν $ \Delta x = 0,1 $ 
    και $ \Delta y = 0,2 $.

    \hfill Απ: $ \Delta f = 0.29552, \; df = 0.3 $
\end{enumerate}


\section*{Σύνθετη Παραγώγιση}

\begin{enumerate}

  \item Να βρεθεί η παράγωγος $\dv{f}{t}$ της συνάρτησης 
    $f(x,y)=\ln(y^2-x^2)$, όταν $x=\sin t, y=\cos t$, για $t=\frac{\pi}{8}$.

    \hfill Απ: $-2$

  \item Να βρεθεί η παράγωγος $\dv{w}{t}$ της συνάρτησης 
    $ w = xy+z $, όταν $ x = \cos{t}, y = \sin{t}$ και $ z = t $, για 
    $ t = \frac{ \pi }{ 4 } $.

    \hfill Απ: 1

  \item Δίνεται η συνάρτηση $ z=f(x,y) $, όπου $ x=u+v $ και $ y = u-v $. 
    Να αποδείξετε ότι $ z_{xx}-z_{yy} = z_{u}\cdot z_{v} $ 

  \item Έστω η συνάρτηση  $ f(x,y) = x^{2} + xy $, όπου $ x=r \cos{\theta} $ και 
    $ y= r \sin{\theta} $. 
    Να υπολογίσετε τις μερικές παραγώγους $ \pdv{f}{r} $ και $ \pdv{f}{\theta} $.  

    \hfill Απ: 
      $ \pdv{f}{r} = 2r(\cos{\theta} )(\cos{\theta} + \sin{\theta}) $ \; και \; 
      $ \pdv{f}{\theta}=r^{2}(\cos{2\theta} - \sin{2 \theta}) $

  \item Δίνεται η συνάρτηση $ f = f(x,y) $, όπου $ x=r \cos{\theta} $ και 
    $ y= r \sin{\theta} $. Να δείξετε ότι 
    \[
      \left(\pdv[2]{f}{x}\right)^{2} + \left(\pdv[2]{f}{y}\right)^{2} = 
      \left(\pdv[2]{f}{r}\right)^{2} + \frac{1}{r^{2}} 
      \left(\pdv[2]{f}{\theta}\right)^{2}
    \] 

  \item Να δείξετε ότι η συνάρτηση $ y(x,t) = f(x+at)+g(x-at) $ ικανοποιεί την 
    εξίσωση $ y_{tt} = a^{2} y_{xx}, \; a \in \mathbb{R} $ αν $ f,g $ είναι συνεχείς 
    με συνεχείς μερικές παραγώγους 2ης τάξης.

  \item Αν $ z=z(x,y) $ με $ x+y= \ln{(u+v)} $ και $ x-y= \ln{(u-v)} $, τότε να δείξετε 
    ότι $ z_{xx}-z_{yy}=(u^{2}-v^{2}) (z_{uu} - z_{vv}) $.

  \item Αν $ z=f(u)+g(v) $ με $ u=ax+by $ και $ v=ax-by $, με $ a,b \in \mathbb{R} $, 
    τότε να δείξετε ότι $ a^{2} z_{yy} - b^{2} z_{xx} = 0 $.
\end{enumerate}


\section*{Τέλεια Διαφορικά}

\begin{enumerate}
  \item Για τις παρακάτω παραστάσεις, να αποδείξετε ότι είναι \textbf{τέλεια
    διαφορικά} και να υπολογίσετε τη συνάρτηση.
    \begin{enumerate}[i)]
      \item $ \left(x+e^{x/y}\right)dx + e^{x/y}\left(1- \frac{x}{y}\right)dy $
        \hfill Απ: $ f(x,y) = \frac{x^{2}}{2} +y e^{x/y} + c $ 

      \item $\left(2e^{x}+\frac{1}{x}-3\sin y\right)dx+3(y^2-x\cos y)dy$ 
        \hfill  Απ: $ f(x,y,z) = y^{3}-3x \sin{y} + 2e^{x} + \ln{x} +c $.
        %spand (114)

      \item $(2xy+z)dx+(x^{2}+z^{2})dy+(x+2yz)dz$ 
        \hfill  Απ: $ f(x,y,z) = x^{2}y+z^{2}y+xz +c $.

        % \item $ (3x^{2}+3y-1)dx + (z^{2}+3x)dy+(2yz+1)dz $
        % \hfill Απ: $ f(x,y,z) = x^{3}+3xy-x+yz^{2}+z+c $

      \item $ \cos(x+yz)dx + z\cos(x+yz)dy+y\cos(x+yz)dz $
        \hfill Απ: $ f(x,y,z) = \sin(x+yz) + c $
    \end{enumerate}
\end{enumerate}


\section*{Τοπικά Ακρότατα}

\begin{enumerate}
  \item Να βρεθούν και να χαρακτηριστούν τα κρίσιμα σημεία  των παρακάτω συναρτήσεων:
    \begin{enumerate}[i)]
      \item $ f(x,y) = x^{3} + y^{3} + 3xy $ 
        \hfill Απ: max: $(-1,-1)  $, σάγμα: $ (0,0) $
      \item $ f(x,y) = x^{2}+y^{4} $ 
        \hfill Απ: min: $ (0,0) $ 
      \item $ f(x,y) = x^{3} + y^{3} - 3x -12y + 50 $ 
        \hfill Απ: max: $ (-1,-2)$, min: $ (1,2) $, 
        σάγμα: $ (1,-2), (-1,2) $
      \item $ f(x,y) = x^{3} + y^{3} -3x -3y + 1 $ 
        \hfill Απ: max: $(-1,-1)  $, min: $ (1,1) $,
        σάγμα: $ (1,-1), (-1,1) $
      \item $ f(x,y) = x^{3} + 4xy -4y^{2} $ 
        \hfill Απ: max: $ (-2/3, -1/3)  $, σάγμα: $ (0,0) $
      % \item $ f(x,y) = x^{4} + y^{4} -2(x-y)^{2}$  
      %   \hfill Απ: min: $ (\sqrt{2} , -\sqrt{2}), (-\sqrt{2} , \sqrt{2}) $, 
      %   σάγμα: $ (0,0) $
      \item $ f(x,y) = \mathrm{e}^{2x} - 2x + 2y^{2} +3 $ \hfill Απ: max: $ (0,0) $  
      \item $ f(x,y) = (x^{2}-3y^{2})e^{1-x^{2}-y^{2}} $ 
        \hfill Απ: max: $ (1,0), (-1,0) $, min: $ (0,1), (0,-1) $, 
        σάγμα: $ (0,0) $
      \item $ f(x,y,z) = 2-x^{2}+2xy-3y^{2}-2z^{2} $ \hfill Απ:  max $ (0,0,0) $ 
      \item $ f(x,y,z) = x^{2}+y^{2}+z^{2}-2x-1 $ \hfill Απ:  min: $ (1,0,0) $ 
      \item $ f(x,y,z) = 2x^{2}+xy+4y^{2}+xz+z^{2}+2 $ \hfill Απ: min: $ (0,0,0) $ 
    \end{enumerate}

  \item Να βρεθούν και να χαρακτηριστούν τα κρίσιμα σημεία της συνάρτησης 
    $ f(x,y,z) = x^{3} + y^{3}+z^{3} + 3xy +3yz + 3xz $

    \hfill Απ: σάγμα: $(0,0,0)$, max: $(-2,-2,-2)$  
\end{enumerate}


\section*{Ακρότατα Υπό Συνθήκη}

\begin{enumerate}

  \item 
    \begin{enumerate}[i)]
      \item $ z=xy $ με περιορισμό $ x+2y=2 $ \hfill Απ: max: $ (1,1/2) $ 
      \item $ z=7-y+x^{2} $ με περιορισμό $ x+y=0 $ \hfill Απ: min: $ (-1/2,1/2) $ 
      \item $ z=x-3y-xy $ με περιορισμό $ x+y=6 $ \hfill Απ: min: $ (1,5) $ 
    \end{enumerate}

  \item \textbf{(θέμα 2021)} 
    Να υπολογιστούν τα τοπικά ακρότατα της συνάρτησης $ f(x,y,z) = x^{2}+y^{2}+z^{2}
    $ που ικανοποιούν τον περιορισμό $ x+y+z+1=0 $.
    \hfill Απ: min: $ (-1/3,-1/3,-1/3) $ 

  \item Να υπολογιστούν τα τοπικά ακρότατα της συνάρτησης 
    $ f(x,y,z) = x^{2}+y^{2}+z^{2}-2x-2y-z+ \frac{5}{4} $ που ικανοποιούν τον 
    περιορισμό $ x^{2}+y^{2}-z=0  $.
    \hfill Απ: min: $ (1/ \sqrt[3]{4} , 1/ \sqrt[3]{4}) $ 

  \item Να υπολογιστούν τα τοπικά ακρότατα της συνάρτησης 
    $f(x,y,z)=xyz$ που ικανοποιούν την εξίσωση $x+y+z-1=0$.  
    \hfill Απ: max: $ (1/3,1/3,1/3) $ 

    %Thomas 12th 14.8 ex.1 
  \item Να βρείτε τα ακρότατα της συνάρτησης $ f(x,y) = xy $ πάνω στην έλλειψη 
    $ x^{2}+2y^{2}=1 $.

    \hfill Απ: 
    \begin{tabular}{l}
      max: $ f(\sqrt{2} /2, 1/2) = f(- \sqrt{2} /2, -1/2) = \frac{\sqrt{2}}{2} $ \\
      min $ f(\sqrt{2} /2, -1/2) = f(- \sqrt{2} /2, 1/2) = -\frac{\sqrt{2}}{2} $ \\
    \end{tabular}

    %Thomas 12th 14.8 ex.14 
  \item Να βρείτε τα ακρότατα της συνάρτησης $ f(x,y) = 3x-y+6 $ υπό τον περιορισμό 
    $ x^{2}+y^{2}=4 $.

    \hfill Απ:  
    \begin{tabular}{l}
      max: $ f(\frac{6}{\sqrt{10}} , - \frac{2}{\sqrt{10}}) = 2 \sqrt{10} +6 $ \\
      min $ f(-\frac{6}{\sqrt{10}} , + \frac{2}{\sqrt{10}}) = -2 \sqrt{10} +6 $ \\
    \end{tabular}
\end{enumerate}



\end{document}
