\documentclass[a4paper,12pt]{article}
\usepackage{etex}
%%%%%%%%%%%%%%%%%%%%%%%%%%%%%%%%%%%%%%
% Babel language package
\usepackage[english,greek]{babel}
% Inputenc font encoding
\usepackage[utf8]{inputenc}
%%%%%%%%%%%%%%%%%%%%%%%%%%%%%%%%%%%%%%

%%%%% math packages %%%%%%%%%%%%%%%%%%
\usepackage{amsmath}
\usepackage{amssymb}
\usepackage{amsfonts}
\usepackage{amsthm}
\usepackage{proof}

\usepackage{physics}

%%%%%%% symbols packages %%%%%%%%%%%%%%
\usepackage{bm} %for use \bm instead \boldsymbol in math mode 
\usepackage{dsfont}
\usepackage{stmaryrd}
%%%%%%%%%%%%%%%%%%%%%%%%%%%%%%%%%%%%%%%


%%%%%% graphicx %%%%%%%%%%%%%%%%%%%%%%%
\usepackage{graphicx}
\usepackage{color}
%\usepackage{xypic}
\usepackage[all]{xy}
\usepackage{calc}
\usepackage{booktabs}
\usepackage{minibox}
%%%%%%%%%%%%%%%%%%%%%%%%%%%%%%%%%%%%%%%

\usepackage{enumerate}

\usepackage{fancyhdr}
%%%%% header and footer rule %%%%%%%%%
\setlength{\headheight}{14pt}
\renewcommand{\headrulewidth}{0pt}
\renewcommand{\footrulewidth}{0pt}
\fancypagestyle{plain}{\fancyhf{}
\fancyhead{}
\lfoot{}
\rfoot{\small \thepage}}
\fancypagestyle{vangelis}{\fancyhf{}
\rhead{\small \leftmark}
\lhead{\small }
\lfoot{}
\rfoot{\small \thepage}}
%%%%%%%%%%%%%%%%%%%%%%%%%%%%%%%%%%%%%%%

\usepackage{hyperref}
\usepackage{url}
%%%%%%% hyperref settings %%%%%%%%%%%%
\hypersetup{pdfpagemode=UseOutlines,hidelinks,
bookmarksopen=true,
pdfdisplaydoctitle=true,
pdfstartview=Fit,
unicode=true,
pdfpagelayout=OneColumn,
}
%%%%%%%%%%%%%%%%%%%%%%%%%%%%%%%%%%%%%%

\usepackage[space]{grffile}

\usepackage{geometry}
\geometry{left=25.63mm,right=25.63mm,top=36.25mm,bottom=36.25mm,footskip=24.16mm,headsep=24.16mm}

%\usepackage[explicit]{titlesec}
%%%%%% titlesec settings %%%%%%%%%%%%%
%\titleformat{\chapter}[block]{\LARGE\sc\bfseries}{\thechapter.}{1ex}{#1}
%\titlespacing*{\chapter}{0cm}{0cm}{36pt}[0ex]
%\titleformat{\section}[block]{\Large\bfseries}{\thesection.}{1ex}{#1}
%\titlespacing*{\section}{0cm}{34.56pt}{17.28pt}[0ex]
%\titleformat{\subsection}[block]{\large\bfseries{\thesubsection.}{1ex}{#1}
%\titlespacing*{\subsection}{0pt}{28.80pt}{14.40pt}[0ex]
%%%%%%%%%%%%%%%%%%%%%%%%%%%%%%%%%%%%%%

%%%%%%%%% My Theorems %%%%%%%%%%%%%%%%%%
\newtheorem{thm}{Θεώρημα}[section]
\newtheorem{cor}[thm]{Πόρισμα}
\newtheorem{lem}[thm]{λήμμα}
\theoremstyle{definition}
\newtheorem{dfn}{Ορισμός}[section]
\newtheorem{dfns}[dfn]{Ορισμοί}
\theoremstyle{remark}
\newtheorem{remark}{Παρατήρηση}[section]
\newtheorem{remarks}[remark]{Παρατηρήσεις}
%%%%%%%%%%%%%%%%%%%%%%%%%%%%%%%%%%%%%%%




\input{definitions_ask.tex}

\pagestyle{vangelis}
\everymath{\displaystyle}


\begin{document}


\section*{Εφαρμογές Διπλού Ολοκληρώματος}

Έστω ότι το χωρίο $D$ είναι ένα \textbf{υλικό φύλλο} και ότι η πυκνότητα της 
μάζας σε κάθε σημείο του $(x,y)$ δίνεται από την συνεχή συνάρτηση $\delta=\delta(x,y)$.

Η \textcolor{Col1}{συνολική μάζα} $M$ που κατανέμεται στο χωρίο $D$ δίνεται από τον τύπο:
\[
   \text{\bfseries Μάζα:} \quad M=\iint_{D}\delta(x,y)\,dxdy
\]

Οι \textcolor{Col1}{στατικές (πρώτες) ροπές} ως προς τους άξονες $x$ και $y$, 
δίνονται αντίστοιχα από τους τύπους:
\[
  \begin{tabular}{>{\bfseries}l<{}>{$}l<{$}}
    Στατικές Ροπές: & M_{x}=\iint_{D}y\delta(x,y)\,dxdy \\
      &  M_{y}=\iint_{D}x\delta(x,y)\,dxdy \\
  \end{tabular}
\]

Οι συντεταγμένες $(\overline{x},\overline{y})$ του \textcolor{Col1}{κέντρου μάζας} 
$K$ του χωρίου $D$ δίνονται από τους τύπους:
\[
  \text{\bfseries Κέντρο Μάζας:} \quad \overline{x}=\frac{M_{y}}{M},\; 
  \overline{y}=\frac{M_{x}}{M}
\]

Οι \textcolor{Col1}{ροπές αδράνειας (δεύτερες ροπές)} ως προς τους άξονες $x$ και $y$, 
η ροπή αδράνειας ως προς τυχαία ευθεία $L$ με $r(x,y)$ να είναι η απόσταση του $(x,y)$ 
από την $L$ και η πολική ροπή ως προς την αρχή των αξόνων, δίνονται αντίστοιχα από 
τους τύπους:
\[
  \begin{tabular}{>{\bfseries}l<{}>{$}l<{$}}
    Ροπές Αδράνειας: 
     & I_{x}=\iint_{D}y^{2}\delta(x,y)\,dxdy \\
     &  I_{y}=\iint_{D}x^{2}\delta(x,y)\,dxdy \\
     &  I_{L}=\iint_{D}r^{2}(x,y)\delta(x,y)\,dxdy \\
     &  I_{o}=\iint_{D}(x^{2}+y^{2})\delta(x,y)\,dxdy =  I_{x}+I_{y}
  \end{tabular}
\]



\section*{Εφαρμογές Τριπλού Ολοκληρώματος}

Έστω ότι το στερεό χωρίο $V$ είναι ένα \textbf{υλικό σώμα} και ότι η πυκνότητα της 
μάζας σε κάθε σημείο του $(x,y,z)$ δίνεται από την συνεχή συνάρτηση 
$\delta=\delta(x,y,z)$.

Η \textcolor{Col1}{συνολική μάζα} $M$ που κατανέμεται στο χωρίο $V$ δίνεται από τον τύπο:
\[
   \text{\bfseries Μάζα:} \quad M=\iiint_{V}\delta(x,y,z)\,dxdydz
\]

Οι \textcolor{Col1}{στατικές (πρώτες) ροπές} ως προς τα επίπεδα συντεταγμένων $Oxy$, 
$Oxz$ και $Oyz$ δίνονται αντίστοιχα από τους τύπους:
\[
  \begin{tabular}{>{\bfseries}l<{}>{$}l<{$}}
    Στατικές Ροπές: & M_{xy}=\iiint_{V}z\delta(x,y,z)\,dxdydz \\
      &  M_{xz}=\iiint_{V}y\delta(x,y,z)\,dxdydz \\
      & M_{yz}=\iiint_{V}x\delta(x,y,z)\,dxdydz
  \end{tabular}
\]

Οι συντεταγμένες $(\overline{x},\overline{y},\overline{z})$ του 
\textcolor{Col1}{κέντρου μάζας} $K$ του χωρίου $V$ δίνονται από τους τύπους:
\[
  \text{\bfseries Κέντρο Μάζας:} \quad \overline{x}=\frac{M_{yz}}{M},\; 
  \overline{y}=\frac{M_{xz}}{M},\; \overline{z}=\frac{M_{xy}}{M}
\]

Οι \textcolor{Col1}{ροπές αδράνειας (δεύτερες ροπές)} ως προς τα επίπεδα συντεταγμένων 
$Oxy$, $Oxz$ και $Oyz$, τους άξονες $x$, $y$ και $z$ και η ροπή αδράνειας ως 
προς την αρχή των αξόνων, δίνονται αντίστοιχα από τους τύπους:
\[
  \begin{tabular}{>{\bfseries}l<{}>{$}l<{$}}
     Ροπές Αδράνειας: & I_{xy}=\iiint_{V}z^{2}\delta(x,y,z)\,dxdydz \\
      &  I_{xz}=\iiint_{V}y^{2}\delta(x,y,z)\,dxdydz \\
      &  I_{yz}=\iiint_{V}x^{2}\delta(x,y,z)\,dxdydz \\
      &  I_{x}=\iiint_{V}(y^{2}+z^{2})\delta(x,y,z)\,dxdydz = I_{xy}+I_{xz} \\
      &  I_{y}=\iiint_{V}(x^{2}+z^{2})\delta(x,y,z)\,dxdydz = I_{xy}+I{yz} \\
      &  I_{z}=\iiint_{V}(x^{2}+y^{2})\delta(x,y,z)\,dxdydz = I_{xz}+I_{yz} \\
      &  I_{o}=\iiint_{V}(x^{2}+y^{2}+z^{2})\delta(x,y,z)\,dxdydz =  I_{xy}+I_{xz}+I_{yz}
  \end{tabular}
\]


\section*{Εφαρμογές Επικαμπύλιου Ολοκληρώματος}

Έστω ότι $c$ είναι μια \textbf{υλική καμπύλη} και ότι η πυκνότητα της μάζας σε 
κάθε σημείο της $(x,y,z)$ δίνεται από την συνεχή συνάρτηση $\delta=\delta(x,y,z)$.

Η \textcolor{Col1}{συνολική μάζα} $M$ που κατανέμεται στην καμπύλη $c$ δίνεται από 
τον τύπο:
\[
   \text{\bfseries Μάζα:} \quad M=\int_{c}\delta(x,y,z)\,ds
\]

Οι \textcolor{Col1}{(πρώτες) ροπές} ως προς τα επίπεδα συντεταγμένων $Oxy$, $Oxz$ και 
$Oyz$, δίνονται αντίστοιχα από τους τύπους:
\[
  \begin{tabular}{>{\bfseries}l<{}>{$}l<{$}}
    Πρώτες Ροπές: & M_{xy}=\int_{c}z\delta(x,y,z)\,ds \\
      &  M_{xz}=\int_{c}y\delta(x,y,z)\,ds \\
      & M_{yz}=\int_{c}x\delta(x,y,z)\,ds
  \end{tabular}
\]

Οι συντεταγμένες $(\overline{x},\overline{y}, \overline{z})$ του \textcolor{Col1}{κέντρου
μάζας} $K$ της καμπύλης $c$ δίνονται από τους τύπους:
\[
  \text{\bfseries Κέντρο Μάζας:} \quad \overline{x}=\frac{M_{yz}}{M},\; 
  \overline{y}=\frac{M_{xz}}{M},\; \overline{z}=\frac{M_{xy}}{M}
\]

Οι \textcolor{Col1}{ροπές αδράνειας (δεύτερες ροπές)} ως προς τους άξονες $x$, $y$ και 
$z$, δίνονται αντίστοιχα από τους τύπους:
\[
  \begin{tabular}{>{\bfseries}l<{}>{$}l<{$}}
     Ροπές Αδράνειας: & I_{x}=\int_{c}(y^{2}+z^{2})\delta(x,y,z)\,ds \\
      &  I_{y}=\int_{c}(x^{2}+z^{2})\delta(x,y,z)\,ds \\
      &  I_{z}=\int_{c}(x^{2}+y^{2})\delta(x,y,z)\,ds \\
  \end{tabular}
\]

\end{document}
