\documentclass[a4paper,12pt]{article}
\usepackage{etex}
%%%%%%%%%%%%%%%%%%%%%%%%%%%%%%%%%%%%%%
% Babel language package
\usepackage[english,greek]{babel}
% Inputenc font encoding
\usepackage[utf8]{inputenc}
%%%%%%%%%%%%%%%%%%%%%%%%%%%%%%%%%%%%%%

%%%%% math packages %%%%%%%%%%%%%%%%%%
\usepackage{amsmath}
\usepackage{amssymb}
\usepackage{amsfonts}
\usepackage{amsthm}
\usepackage{proof}

\usepackage{physics}

%%%%%%% symbols packages %%%%%%%%%%%%%%
\usepackage{bm} %for use \bm instead \boldsymbol in math mode 
\usepackage{dsfont}
\usepackage{stmaryrd}
%%%%%%%%%%%%%%%%%%%%%%%%%%%%%%%%%%%%%%%


%%%%%% graphicx %%%%%%%%%%%%%%%%%%%%%%%
\usepackage{graphicx}
\usepackage{color}
%\usepackage{xypic}
\usepackage[all]{xy}
\usepackage{calc}
\usepackage{booktabs}
\usepackage{minibox}
%%%%%%%%%%%%%%%%%%%%%%%%%%%%%%%%%%%%%%%

\usepackage{enumerate}

\usepackage{fancyhdr}
%%%%% header and footer rule %%%%%%%%%
\setlength{\headheight}{14pt}
\renewcommand{\headrulewidth}{0pt}
\renewcommand{\footrulewidth}{0pt}
\fancypagestyle{plain}{\fancyhf{}
\fancyhead{}
\lfoot{}
\rfoot{\small \thepage}}
\fancypagestyle{vangelis}{\fancyhf{}
\rhead{\small \leftmark}
\lhead{\small }
\lfoot{}
\rfoot{\small \thepage}}
%%%%%%%%%%%%%%%%%%%%%%%%%%%%%%%%%%%%%%%

\usepackage{hyperref}
\usepackage{url}
%%%%%%% hyperref settings %%%%%%%%%%%%
\hypersetup{pdfpagemode=UseOutlines,hidelinks,
bookmarksopen=true,
pdfdisplaydoctitle=true,
pdfstartview=Fit,
unicode=true,
pdfpagelayout=OneColumn,
}
%%%%%%%%%%%%%%%%%%%%%%%%%%%%%%%%%%%%%%

\usepackage[space]{grffile}

\usepackage{geometry}
\geometry{left=25.63mm,right=25.63mm,top=36.25mm,bottom=36.25mm,footskip=24.16mm,headsep=24.16mm}

%\usepackage[explicit]{titlesec}
%%%%%% titlesec settings %%%%%%%%%%%%%
%\titleformat{\chapter}[block]{\LARGE\sc\bfseries}{\thechapter.}{1ex}{#1}
%\titlespacing*{\chapter}{0cm}{0cm}{36pt}[0ex]
%\titleformat{\section}[block]{\Large\bfseries}{\thesection.}{1ex}{#1}
%\titlespacing*{\section}{0cm}{34.56pt}{17.28pt}[0ex]
%\titleformat{\subsection}[block]{\large\bfseries{\thesubsection.}{1ex}{#1}
%\titlespacing*{\subsection}{0pt}{28.80pt}{14.40pt}[0ex]
%%%%%%%%%%%%%%%%%%%%%%%%%%%%%%%%%%%%%%

%%%%%%%%% My Theorems %%%%%%%%%%%%%%%%%%
\newtheorem{thm}{Θεώρημα}[section]
\newtheorem{cor}[thm]{Πόρισμα}
\newtheorem{lem}[thm]{λήμμα}
\theoremstyle{definition}
\newtheorem{dfn}{Ορισμός}[section]
\newtheorem{dfns}[dfn]{Ορισμοί}
\theoremstyle{remark}
\newtheorem{remark}{Παρατήρηση}[section]
\newtheorem{remarks}[remark]{Παρατηρήσεις}
%%%%%%%%%%%%%%%%%%%%%%%%%%%%%%%%%%%%%%%




\newcommand{\vect}[2]{(#1_1,\ldots, #1_#2)}
%%%%%%% nesting newcommands $$$$$$$$$$$$$$$$$$$
\newcommand{\function}[1]{\newcommand{\nvec}[2]{#1(##1_1,\ldots, ##1_##2)}}

\newcommand{\linode}[2]{#1_n(x)#2^{(n)}+#1_{n-1}(x)#2^{(n-1)}+\cdots +#1_0(x)#2=g(x)}

\newcommand{\vecoffun}[3]{#1_0(#2),\ldots ,#1_#3(#2)}

\newcommand{\mysum}[1]{\sum_{n=#1}^{\infty}




\pagestyle{askhseis}
\begin{document}


\begin{center}
  \minibox{\bfseries\large \textcolor{Col1}{Ασκήσεις στα Υλικά Χωρία}}
\end{center}

\vspace{\baselineskip}

\begin{enumerate}
  \item Να βρεθεί το κέντρο μάζας του υλικού χωρίου $D$ που ορίζεται από τις καμπύλες 
    $y^{2}=x$ και $y=x^{3}$, όταν η συνάρτηση πυκνότητας μάζας, είναι $\delta(x,y)=y$

  \hfill Απ: $\overline{x}=\frac{7}{12}$, $\overline{y}=\frac{14}{25}$

  \item Να βρεθούν οι συντεταγμένες του κέντρου βάρους του ομογενούς $(\delta(x,y)=1)$ 
    χωρίου που περικλείεται από τις καμπύλες $y^{2}=2x$, $x=2$ και $y=0$.

  \hfill Απ: $\overline{x}=\frac{6}{5}$, $\overline{y}=\frac{3}{4}$

  \item  Να υπολογιστούν το κέντρο μάζας και η ροπή αδράνειας ως προς την αρχή των 
    αξόνων της επιφάνειας που περικλείεται από τις καμπύλες $y=x^{3}$ και  $y=4x$ με 
    $x,y\geq 0$, όταν $\delta(x,y)=1$.

  \hfill Απ: $\overline{x}=\frac{16}{15}$, $\overline{y}=\frac{64}{21}$, 
  $I_{o}=\frac{848}{15}$

  \item Να βρεθούν οι συντεταγμένες του κέντρου μάζας του στερεού χωρίου $V$ που 
    είναι το πρώτο όγδοο σφαίρας ακτίνας $R=1$ όταν η κατανομή μάζας στο χωρίο, 
    δίνεται από τη συνάρτηση πυκνότητας $\delta(x,y)=1$.

  \hfill Απ: $\overline{x}=\frac{3}{8}$, $\overline{y}=\frac{3}{8}$, 
  $\overline{z}=\frac{3}{8}$

  \item Να υπολογιστεί η συνολική μάζα του στερεού χωρίου $V$ που περικλείεται 
    από τις επιφάνειες $z^{2}=x^{2}+y^{2}$, $z=0$ και $z=1$, όταν η 
    συνάρτηση πυκνότητας μάζας είναι $\delta(x,y)=kz$, $k\in\mathbb{R}$.

  \hfill Απ: $M=\frac{k\pi}{4}$

  \item Να βρεθούν οι συντεταγμένες του κέντρου βάρους του στερεού $V$ που 
    περικλείεται από τις επιφάνειες $z=x^{2}+y^{2}$, $x^{2}+y^{2}=2x$ και το επίπεδο 
    $z=0$ όταν η συνάρτηση πυκνότητας μάζας είναι $\delta(x,y,z)=1$.

  \hfill Απ: $\overline{x}=\frac{4}{3}$, $\overline{y}=0$, $\overline{z}=\frac{10}{9}$
\end{enumerate}


\end{document}
