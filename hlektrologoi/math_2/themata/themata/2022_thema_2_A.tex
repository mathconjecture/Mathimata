\documentclass[a4paper,table]{report}
\input{preamble_themata}
\newcommand{\vect}[2]{(#1_1,\ldots, #1_#2)}
%%%%%%% nesting newcommands $$$$$$$$$$$$$$$$$$$
\newcommand{\function}[1]{\newcommand{\nvec}[2]{#1(##1_1,\ldots, ##1_##2)}}

\newcommand{\linode}[2]{#1_n(x)#2^{(n)}+#1_{n-1}(x)#2^{(n-1)}+\cdots +#1_0(x)#2=g(x)}

\newcommand{\vecoffun}[3]{#1_0(#2),\ldots ,#1_#3(#2)}

\newcommand{\suma}{\sum_{n=0}^{\infty}a_n x^n}

\newcommand{\sumb}{\sum_{n=1}^{\infty}a_n n x^{n-1}}

\newcommand{\sumc}{\sum_{n=2}^{\infty}a_n n (n-1) x^{n-2}}

\newcommand{\varsum}[2]{\sum_{n=#1}^{#2}}
\input{myboxes}

\everymath{\displaystyle}
\pagestyle{vangelis}
\geometry{top=2cm}


\begin{document}

\begin{mybox3}
  \begin{thema}
    Ένα σωμάτιο κινείται σε τροχιά με παραμετρικές εξισώσεις:
    \[
      x(t)=t \sin{t} + \cos{t}, \quad y(t)= -t \cos{t} + \sin{t}, \quad t \in \mathbb{R} 
    \] 
    \begin{enumerate}[i)]
      \item Να βρεθεί το μήκος της τροχιάς που διανύει το κινητό για $t \in [0, \pi] $.
      \item Να βρεθεί η εφαπτομενική και η κάθετη συνιστώσα της επιτάχυνσης, καθώς και ο 
        κύκλος καμπυλότητας στη θέση που αντιστοιχεί στη χρονική στιγμή $ t= \pi $.

        \hfill \textcolor{Col1}{\textbf{Δίνεται ο τύπος:}} $ \mathbf{a}(t) = \dv[2]{s}{t}
        \mathbf{T}+\kappa\Bigl(\dv{s}{t}\Bigr)^{2} \mathbf{N} $
    \end{enumerate}
  \end{thema}
\end{mybox3}
\begin{solution}
\item {}
  \begin{enumerate}[i)]
    \item Σχηματίζουμε την διανυσματική παραμετρική εξίσωση της καμπύλης, από τις 
      δοθείσες παραμετρικές εξισώσεις:
      \[
        \mathbf{r}(t) = (t \sin{t} + \cos{t}) \, \mathbf{i} + (-t \cos{t} + \sin{t} )
        \, \mathbf{j} 
      \] 
      Οπότε, έχουμε:
      \begin{align*}
        \mathbf{r'}(t) &= (\cancel{\sin{t}} + t \cos{t} - \cancel{\sin{t}\,}) 
        \mathbf{i} + (- \cancel{\cos{t}} + t \sin{t} + \cancel{\cos{t})\,} \mathbf{j} = 
        t \cos{t} \, \mathbf{i} + t \sin{t} \, \mathbf{j} \\
        \norm{\mathbf{r'}(t)} &= \sqrt{t^2 \cos^{2}{t} + t^2 \sin^{2}{t}} = \sqrt{t^2}
        \overset{t>0}{=} t
      \end{align*} 
      Άρα το μήκος της καμπύλης, είναι:
      \[
        \int _{0}^{ \pi} t \,{dt} = \left[ \frac{t^2}{2} \right]_{0}^{\pi} = \frac{\pi
        ^{2}}{2}
      \]
    \item Για να βρούμε την εφαπτομενική (επιτρόχιο) και κάθετη (κεντρομόλο) συνιστώσα 
      της επιτάχυνσης, βρίσκουμε:
      \begin{align*}
        \dv{s}{t} &= \norm{\mathbf{r'}(t)} = t \\
        \dv[2]{s}{t} &= 1
      \end{align*}
      καθώς και τα μοναδιαία διανύσματα $ \mathbf{T} $ και $ \mathbf{N} $
      \begin{align*}
        \mathbf{T} &= \frac{\mathbf{r'}(t)}{\norm{\mathbf{r'}(t)}} = \frac{t \cos{t}}{t} \,
        \mathbf{i} + \frac{t \sin{t}}{t} \, \mathbf{j} = \cos{t} \, \mathbf{i} +
        \sin{t} \, \mathbf{j} \\
        \dv{T}{t} &= - \sin{t} \, \mathbf{i} + \cos{t} \, \mathbf{j} \\
        \norm{\dv{T}{t}} &= \sqrt{\sin^{2}{t} + \cos^{2}{t}} = 1 \\
        \mathbf{N} &= \frac{\dv{T}{t}}{\norm{\dv{T}{t}}} = - \sin{t} \, \mathbf{i} + \cos{t}
        \, \mathbf{j} 
      \end{align*}
      Επίσης, βρίσκουμε και την καμπυλότητα
      \[
        \kappa = \frac{\norm{\dv{T}{t}}}{\norm{\mathbf{r'}(t)}} = \frac{1}{t} 
      \] 
      Άρα, σύμφωνα με τον τύπο της επιτάχυνσης, η εφαπτομενική συνιστώσα της
      (\textbf{επιτρόχιος} επιτάχυνση), στο τυχαίο σημείο $t$ θα είναι:
      \[
        \mathbf{a_{\varepsilon}} = \Bigl(\dv[2]{s}{t}\Bigr)T = 1 (\cos{t} \, \mathbf{i}+
        \sin{t} \, \mathbf{j}) = \cos{t} \, \mathbf{i} + \sin{t} \, \mathbf{j}
      \] 
      και η κάθετη συνιστώσα της (\textbf{κεντρομόλος} επιτάχυνση), στο τυχαίο σημείο 
      $t$, θα είναι:
      \[
        \mathbf{a_{\kappa}} = \kappa \Bigl(\dv{s}{t} \Bigr)^2 N = \frac{1}{\cancel{t}}
        t^{\cancel{2}}
        (- \sin{t} \, \mathbf{i} + \cos{t} \, \mathbf{j}) = t (- \sin{t} \, \mathbf{i} + \cos{t} \, \mathbf{j})
      \] 
      Στο σημείο $ t= \pi $ οι συνιστώσες αυτές, γίνονται:
      \[
        \mathbf{a_{\varepsilon}(\pi)} = \cos{\pi} \, \mathbf{i} + \sin{\pi} \, \mathbf{j}
        = - \, \mathbf{i}
      \] 
      και 
      \[
        \mathbf{a_{\kappa}(\pi)} = \pi (- \sin{\pi} \, \mathbf{i} + \cos{\pi} \,
        \mathbf{j}) = - \pi \, \mathbf{j}
      \] 
      και γενική η επιτάχυνση στο σημείο $ t = \pi $, γίνεται:
      \[
        \mathbf{a}(t) = - \, \mathbf{i} - \pi \, \mathbf{j}  
      \] 
  \end{enumerate}
\end{solution}

\end{document}
