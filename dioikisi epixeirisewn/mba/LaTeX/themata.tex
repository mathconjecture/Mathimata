\input{$HOME/Desktop/preamble/preamble.tex}
\newcommand{\vect}[2]{(#1_1,\ldots, #1_#2)}
%%%%%%% nesting newcommands $$$$$$$$$$$$$$$$$$$
\newcommand{\function}[1]{\newcommand{\nvec}[2]{#1(##1_1,\ldots, ##1_##2)}}

\newcommand{\linode}[2]{#1_n(x)#2^{(n)}+#1_{n-1}(x)#2^{(n-1)}+\cdots +#1_0(x)#2=g(x)}

\newcommand{\vecoffun}[3]{#1_0(#2),\ldots ,#1_#3(#2)}

\newcommand{\suma}{\sum_{n=0}^{\infty}a_n x^n}

\newcommand{\sumb}{\sum_{n=1}^{\infty}a_n n x^{n-1}}

\newcommand{\sumc}{\sum_{n=2}^{\infty}a_n n (n-1) x^{n-2}}

\newcommand{\varsum}[2]{\sum_{n=#1}^{#2}}

\everymath{\displaystyle}

\thispagestyle{empty}



\begin{document}

\begin{center}
    \fbox{\large\bfseries Θέματα}
\end{center}

\vspace{\baselineskip}

\begin{enumerate}
    \item Έστω το γραμμικό υπόδειγμα ενός αγαθού σε μια μεμονωμένη αγορά της μορφής
        \[
        \sysdelim.\}\systeme*{Q_{d}= \alpha - \beta P,Q_{s}=- \gamma + \delta P} 
        \] 
        Ποια από τις παρακάτω εκφράσεις είναι σωστή;
        \begin{enumerate}[i)]
            \item Η μείωση του $\alpha$ προκαλεί μείωση της τιμής ισορροπίας επειδή $\beta$,
                $\delta > 0$ 
            \item Η μείωση του $\alpha$ προκαλεί αύξηση της τιμής ισορροπίας επειδή $\beta$,
                $\delta > 0$
        \end{enumerate}

    \item Έστω το γραμμικό υπόδειγμα ενός αγαθού σε μια μεμονωμένη αγορά της μορφής
        \[
        \sysdelim.\}\systeme*{Q_{d}= \alpha - \beta P,Q_{s}=- \gamma + \delta P} 
        \] 
        Ποια από τις παρακάτω εκφράσεις είναι σωστή;
        \begin{enumerate}[i)]
            \item Η αύξηση  του $\beta$ προκαλεί αύξηση της τιμής ισορροπίας επειδή $\alpha$,
                $\gamma > 0$ 
            \item Η αύξηση  του $\beta$ προκαλεί μείωση της τιμής ισορροπίας επειδή $\alpha$,
                $\gamma > 0$
        \end{enumerate}

    \item Έστω η συνάρτηση $ f(x) = (x-3)^{3} $. Στο σημείο $ x_{0} = 3 $ η συνάρτηση παρουσιάζει:

        \begin{inparaenum}[i)]
        \item τοπικό μέγιστο \quad
        \item τοπικό ελάχιστο \quad \item σημείο καμπής
        \end{inparaenum}

    \item Έστω η συνάρτηση $ f(x) = x^{3} - 9x^{2}+27x-27 $. Στο σημείο $ x_{0} = 3 $ η συνάρτηση
        παρουσιάζει:

        \begin{inparaenum}[i)]
        \item τοπικό μέγιστο \quad
        \item τοπικό ελάχιστο \quad \item σημείο καμπής
        \end{inparaenum}

    \item Έστω η συνάρτηση $ f(x) = x^{3}+3px+5 $. Ποια από τις παρακάτω προτάσεις είναι λάθος;

        \begin{enumerate}[i)]
            \item Αν $ p>0 $ η συνάρτηση δεν παρουσιάζει τοπικά ακρότατα
            \item Αν $ p>0 $ η συνάρτηση παρουσιάζει τοπικά ακρότατα
        \end{enumerate}

    \item Έστω η συνάρτηση $ f(x) = x^{n}-nx+k $, όπου $ n>0 $ και $ n \neq 1 $. Ποια από τις
        παρακάτω προτάσεις είναι σωστή;

        \begin{enumerate}[i)]
            \item Αν $ n>1 $ τότε η συνάρτηση έχει τοπικό ελάχιστο στο 1
            \item Αν $ n>1 $ τότε η συνάρτηση έχει τοπικό ελάχιστο στο 1
        \end{enumerate}


\end{enumerate}



\end{document}
