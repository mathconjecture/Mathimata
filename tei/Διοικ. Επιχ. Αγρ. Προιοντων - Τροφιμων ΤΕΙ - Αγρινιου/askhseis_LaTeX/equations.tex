\documentclass[a4paper,12pt]{article}
\usepackage{etex}
%%%%%%%%%%%%%%%%%%%%%%%%%%%%%%%%%%%%%%
% Babel language package
\usepackage[english,greek]{babel}
% Inputenc font encoding
\usepackage[utf8]{inputenc}
%%%%%%%%%%%%%%%%%%%%%%%%%%%%%%%%%%%%%%

%%%%% math packages %%%%%%%%%%%%%%%%%%
\usepackage{amsmath}
\usepackage{amssymb}
\usepackage{amsfonts}
\usepackage{amsthm}
\usepackage{proof}

\usepackage{physics}

%%%%%%% symbols packages %%%%%%%%%%%%%%
\usepackage{bm} %for use \bm instead \boldsymbol in math mode 
\usepackage{dsfont}
\usepackage{stmaryrd}
%%%%%%%%%%%%%%%%%%%%%%%%%%%%%%%%%%%%%%%


%%%%%% graphicx %%%%%%%%%%%%%%%%%%%%%%%
\usepackage{graphicx}
\usepackage{color}
%\usepackage{xypic}
\usepackage[all]{xy}
\usepackage{calc}
\usepackage{booktabs}
\usepackage{minibox}
%%%%%%%%%%%%%%%%%%%%%%%%%%%%%%%%%%%%%%%

\usepackage{enumerate}

\usepackage{fancyhdr}
%%%%% header and footer rule %%%%%%%%%
\setlength{\headheight}{14pt}
\renewcommand{\headrulewidth}{0pt}
\renewcommand{\footrulewidth}{0pt}
\fancypagestyle{plain}{\fancyhf{}
\fancyhead{}
\lfoot{}
\rfoot{\small \thepage}}
\fancypagestyle{vangelis}{\fancyhf{}
\rhead{\small \leftmark}
\lhead{\small }
\lfoot{}
\rfoot{\small \thepage}}
%%%%%%%%%%%%%%%%%%%%%%%%%%%%%%%%%%%%%%%

\usepackage{hyperref}
\usepackage{url}
%%%%%%% hyperref settings %%%%%%%%%%%%
\hypersetup{pdfpagemode=UseOutlines,hidelinks,
bookmarksopen=true,
pdfdisplaydoctitle=true,
pdfstartview=Fit,
unicode=true,
pdfpagelayout=OneColumn,
}
%%%%%%%%%%%%%%%%%%%%%%%%%%%%%%%%%%%%%%

\usepackage[space]{grffile}

\usepackage{geometry}
\geometry{left=25.63mm,right=25.63mm,top=36.25mm,bottom=36.25mm,footskip=24.16mm,headsep=24.16mm}

%\usepackage[explicit]{titlesec}
%%%%%% titlesec settings %%%%%%%%%%%%%
%\titleformat{\chapter}[block]{\LARGE\sc\bfseries}{\thechapter.}{1ex}{#1}
%\titlespacing*{\chapter}{0cm}{0cm}{36pt}[0ex]
%\titleformat{\section}[block]{\Large\bfseries}{\thesection.}{1ex}{#1}
%\titlespacing*{\section}{0cm}{34.56pt}{17.28pt}[0ex]
%\titleformat{\subsection}[block]{\large\bfseries{\thesubsection.}{1ex}{#1}
%\titlespacing*{\subsection}{0pt}{28.80pt}{14.40pt}[0ex]
%%%%%%%%%%%%%%%%%%%%%%%%%%%%%%%%%%%%%%

%%%%%%%%% My Theorems %%%%%%%%%%%%%%%%%%
\newtheorem{thm}{Θεώρημα}[section]
\newtheorem{cor}[thm]{Πόρισμα}
\newtheorem{lem}[thm]{λήμμα}
\theoremstyle{definition}
\newtheorem{dfn}{Ορισμός}[section]
\newtheorem{dfns}[dfn]{Ορισμοί}
\theoremstyle{remark}
\newtheorem{remark}{Παρατήρηση}[section]
\newtheorem{remarks}[remark]{Παρατηρήσεις}
%%%%%%%%%%%%%%%%%%%%%%%%%%%%%%%%%%%%%%%




\newcommand{\vect}[2]{(#1_1,\ldots, #1_#2)}
%%%%%%% nesting newcommands $$$$$$$$$$$$$$$$$$$
\newcommand{\function}[1]{\newcommand{\nvec}[2]{#1(##1_1,\ldots, ##1_##2)}}

\newcommand{\linode}[2]{#1_n(x)#2^{(n)}+#1_{n-1}(x)#2^{(n-1)}+\cdots +#1_0(x)#2=g(x)}

\newcommand{\vecoffun}[3]{#1_0(#2),\ldots ,#1_#3(#2)}

\newcommand{\mysum}[1]{\sum_{n=#1}^{\infty}


\pagestyle{askhseis}
\geometry{top=2cm,left=1.5cm,right=1.5cm}

\begin{document}

\setcounter{chapter}{1}

\begin{center}
  \minibox{\large\bfseries \textcolor{Col1}{Ευθείες}}
\end{center}

\vspace{\baselineskip}

\begin{enumerate}
  \item  Δίνεται η ευθεία ε:  $ 3x - 4y = 12 $.  Να βρείτε:
    \begin{enumerate}[i)]
      \item Να εξετάσετε αν η ευθεία διέρχεται από τα σημεία $\left(2,- \frac{3}{2}
        \right)$, $ (2,-2) $ \hfill Απ:  ναι, όχι 
      \item  Τα σημεία τομής της ευθείας με τους άξονες  $ xx' $  και  $ yy' $ \hfill Απ:
        $ xx': 4, yy':-3 $ 
      \item Να γραφεί στη μορφή $ y=ax+b $ και να προσδιορίσετε την κλίση της. \hfill
        Απ: $ \lambda = \frac{3}{4} $ 
    \end{enumerate}

  \item  Δίνονται οι ευθείες 
    $ \varepsilon_1: y = 3x-6 $, $ \varepsilon_2: y = -x+2 $  και 
    $ \varepsilon_3:   y = -4x+8 $
    \begin{enumerate}[i)]
      \item  Να βρείτε το σημείο τομής των ευθειών.  \hfill Απ: $ (2,0) $
      \item  Να δείξετε ότι οι τρεις ευθείες διέρχονται από το ίδιο σημείο.
    \end{enumerate}

  \item  Να βρείτε την εξίσωση της ευθείας  $ \varepsilon $,  η οποία διέρχεται 
    από το σημείο  $ A(1,2) $ και έχει συντελεστή διεύθυνσης  $ \frac{1}{3} $.

    \hfill Απ: $ y = \frac{1}{3} x - \frac{7}{3} $ 

  \item  Να βρείτε την εξίσωση της ευθείας  $\varepsilon$, η οποία διέρχεται 
    από τα σημεία  $ A(1,1) $  και  $ B(3,5) $. 
    \hfill Απ: $ y=2x-1 $ 

  \item  Να βρείτε τα  σημεία τομής των ευθειών:
    \begin{enumerate}[i)]
      \item $ \varepsilon_{1}: x + 2y = 5 $   και  $\varepsilon_{2}: 4x + y = 6 $.
        \hfill Απ: $ (1,2) $ 
      \item $ \varepsilon_{1}: 4x+3y=11 $ και  $ \varepsilon_{2}: 5x+7y=17 $ \hfill Απ: $
        (2,1) $ 
    \end{enumerate}


    \begin{center}
      \minibox{\large\bfseries  \textcolor{Col1}{Εξισώσεις}}
    \end{center}

  \item  Να λυθούν οι παρακάτω εξισώσεις.

    \twocolumnside{
      \begin{enumerate}[i)]
        \item $ 2x^{2} + 5x - 3 = 0 $ \hfill  Απ: $ \frac{1}{2}, -3 $ 
        \item $ x^{2} + 5x + 4 = 0 $ \hfill Απ:  $ -4, -1 $
        \item $ x^{2} - 10x + 25 = 0 $ \hfill Απ: $ 5 $
        \item $ x^{2} - 7x + 6 = 0 $ \hfill Απ: $ 6, 1 $
      \end{enumerate}
      }{
      \begin{enumerate}[i),start=5]
        \item $ x^{2} - 4x + 5 = 0 $ \hfill Απ:  αδύνατη
        \item $ -x^{2} - 8x + 9 = 0 $ \hfill Απ: $ -9, 1 $
        \item $ -x^{2} + x + 42 = 0 $ \hfill Απ: $ \frac{1}{3}, -1 $  
        \item $ 4x^{2} + 20x + 25 = 0 $ \hfill Απ: $ - \frac{5}{2} $ 
      \end{enumerate}
    }

    \vspace{0.5\baselineskip}

    \twocolumnside{
    \item Να λυθούν οι παρακάτω εξισώσεις.
      \begin{enumerate}[i)]
        \item $ 3x^{2} - 2x + 1 = 2 $ \hfill Απ: $ 1, - \frac{1}{3} $  
        \item $ 3x^{2} + 2x - 4 = 3x - 7 $ \hfill Απ:  αδύνατη
        \item $ 4x^{2} - 6x + 5 = 4 - 5x^{2} $ \hfill Απ: $ \frac{1}{3} $
        \item $ 5x^{2} - 3x = 2x + x^{2} - 1 $ \hfill Απ: $ \frac{1}{4}, 1 $ 
      \end{enumerate}
      }{
    \item Να λυθούν οι παρακάτω εξισώσεις.
      \begin{enumerate}[i)]
        \item $ (x^{2} - 4)(x^{2} - 5x + 6) = 0 $ \hfill Απ: $ -2, 3, 2 $ 
        \item $ (x^{2} + 5x)(3x^{2} - 2x - 8) = 0 $ \hfill Απ: $ 0, -5, 2, -\frac{4}{3} $
        \item $ (x^{2} - 7x + 6)(4x^{2}- 8x + 3) = 0 $ \hfill Απ: $ 6, 1, \frac{1}{2},
          \frac{3}{2} $
        \item $ (9x^{2} - 6x + 1)(x^{2} - x - 2) = 0 $ 
          \hfill Απ: $ \frac{1}{3}, 2, -1  $ 
      \end{enumerate}
    }

    \vspace{0.5\baselineskip}

  \item Να λυθούν οι παρακάτω εξισώσεις.
    \begin{enumerate}[i)]
      \item $ (x-1)(x-2) = x(x-4)+(x+2)^{2} $ \hfill Απ: $ -1, -2 $
      \item $ (2x-1)^{2} - (x-3)^{2} = (x+1)(2x-1) - 1 $ \hfill Απ: $ 2, -3 $
      \item $ (2-x)^{2} - (x-1)(x+1) = (x+3)^{2} - 6x $ \hfill Απ: $ -2 $
      \item $ 2(x-2)(x+3) + (-x-1)^{2} = (x-1)^{2} - (x-2)^{2} $ \hfill Απ: $ -2,
        \frac{4}{3} $ 
    \end{enumerate}

    \vspace{0.5\baselineskip}

    \twocolumnside{
    \item Να λυθούν οι παρακάτω εξισώσεις.
      \begin{enumerate}[i)]
        \item $ \frac{x^{2}}{6} - \frac{2x}{3} = \frac{3x-10}{4} $ \hfill Απ: $
          \frac{5}{2}, 6 $
        \item $ \frac{x-2}{2} = \frac{(x-1)(x+1)}{3} - \frac{2x+1}{6} $ \hfill Απ: $
          \frac{3}{2}, 1 $ 
        \item $ \left(\frac{x-1}{2}\right)^{2} - \frac{2x}{5} = 
          \frac{2x+3}{2} - \frac{(5x-1)(2x+5)}{20} $ \hfill Απ: $ 2, -1 $ 
      \end{enumerate}
      }{
    \item Να λυθούν οι παρακάτω εξισώσεις.
      \begin{enumerate}[i)]
        \item $ 1 - \frac{3}{x} = \frac{10}{x^{2}} $ \hfill Απ: $ -2, 5 $
        \item $ 2 - \frac{x+4}{x} = \frac{21}{x^{2}} $ \hfill Απ: $ 7, -3 $
        \item $ \frac{7}{6x} = \frac{1}{2x^{2}} - 1 $ \hfill Απ: $ \frac{1}{3},
          -\frac{3}{2} $ 
        \item $ 1 - \frac{2}{3x} = \frac{2}{x} - \frac{2-x}{x^{2}} $ \hfill Απ: $ 3,
          \frac{2}{3} $  
        \item $ \frac{240}{x+2} = \frac{240}{x} - 10 $ \hfill Απ: $ 6, -8 $
      \end{enumerate}
    }

\end{enumerate}

\end{document}
