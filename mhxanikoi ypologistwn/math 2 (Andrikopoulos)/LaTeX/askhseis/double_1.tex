\input{preamble_ask.tex}
\newcommand{\vect}[2]{(#1_1,\ldots, #1_#2)}
%%%%%%% nesting newcommands $$$$$$$$$$$$$$$$$$$
\newcommand{\function}[1]{\newcommand{\nvec}[2]{#1(##1_1,\ldots, ##1_##2)}}

\newcommand{\linode}[2]{#1_n(x)#2^{(n)}+#1_{n-1}(x)#2^{(n-1)}+\cdots +#1_0(x)#2=g(x)}

\newcommand{\vecoffun}[3]{#1_0(#2),\ldots ,#1_#3(#2)}

\newcommand{\mysum}[1]{\sum_{n=#1}^{\infty}



\everymath{\displaystyle}


\begin{document}

\begin{center}
\fbox{\large\bf Ασκήσεις Διπλό Ολοκλήρωμα}
\end{center}

\vspace{\baselineskip}

\begin{enumerate}

\item Να υπολογιστούν τα παρακάτω διπλά ολοκληρώματα (ορθογώνια χωρία).

\begin{enumerate}[i)]

\item $\int\limits_1^2\!\!\!\int\limits_0^ 3xy\,dxdy$ \hfill Απ: $\frac{27}{4}$

\item $\iint\limits_{R}(x^2+y^2)\,dxdy,\quad R=[-1,1]\times[0,1]$ \hfill Απ: $\frac{4}{3}$

\item $\iint\limits_{R}y(x^3-12x)\,dxdy,\quad R=[-2,1]\times[0,1]$ \hfill Απ: $\frac{57}{8}$

\item $\iint\limits_{R}\cos x\sin y\,dxdy,\quad R=[0,\frac{\pi}{2}]\times[0,\frac{\pi}{2}]$ \hfill Απ: $1$

\item $\iint\limits_{R}\sin(x+y)\,dxdy, \quad R=[0,\frac{\pi}{2}]\times[0,\frac{\pi}{2}]$ \hfill Απ: $2$
\end{enumerate}

\vspace{\baselineskip}

\item Να υπολογιστούν τα παρακάτω διπλά ολοκληρώματα. Οπου ειναι δυνατον να γινει ο υπολογισμος με δύο τρόπους (γενικότερα χωρία).

\begin{enumerate}[i)]

\item $\int\limits_1^2\!\!\int\limits_0^{4x} xy\,dxdy$ \hfill Απ: $30$

\item $\int\limits_0^1\!\!\int\limits_{x^2}^{x}(x^2+3y+2)\,dxdy$ \hfill Απ: $\frac{7}{12}$

\item $\iint\limits_{D}xy\,dxdy,\quad D=\left\{\,(x,y)\in\mathbb{R}^2\mid 0\leq x\leq 2,\; 0\leq y\leq \sqrt{x}\,\right\}$ \hfill Απ: $\frac{4}{3}$

\item $\iint\limits_{D}(x^2-y^2)\,dxdy,\quad D=\left\{\,(x,y)\in\mathbb{R}^2 \mid -1\leq x\leq 1,\;-x^2\leq y\leq x^2\,\right\}$  

\hfill Απ: $\frac{64}{105}$
\item $\iint\limits_{D}x\,dxdy, \quad D$ περικλείεται από τις καμπύλες $y=x$ και $x^2+y^2=4$ με $x,y\geq 0$.

\hfill Απ: $\frac{4\sqrt{2}}{3}$

\item $\iint\limits_{D}(x-1)\,dxdy,\quad D$ περικλείεται από τις καμπύλες $y=x$ και $y=x^3$. 

\hfill Απ: $-\frac{1}{2}$

\item $\int\limits_0^1\int\limits_0^{x^2}e^{\frac{y}{x}}\,dydx$\hfill Απ: $\frac{1}{2}$


\end{enumerate}





\item Να υπολογιστούν τα παρακάτω διπλά ολοκληρώματα, επιλέγοντας την κατάλληλη σειρά ολοκλήρωσης.

\begin{enumerate}[i)]

\item $\iint\limits_{D}\frac{\sin x}{x}\,dxdy,\quad D$ περικλείεται από τις ευθείες $y=x, y=0$ και $x=\pi$.

\hfill Απ: $2$

\item $\iint\limits_{D}\frac{x}{\sqrt{1+x^2+y^2}}\,dxdy,\quad D=\left\{\,(x,y)\in\mathbb{R}^2\mid 0\leq x\leq 2,\; 0\leq y\leq \frac{x^2}{2}\,\right\}$

\hfill Απ: $-1+\frac{5}{4}\ln 5$

\end{enumerate}


\vspace{\baselineskip}


\item Να βρείτε το \textbf{εμβαδόν} του χωρίου $D$ που περικλείεται από τις καμπύλες: 

\begin{enumerate}[i)]

\item $y=x$ και $y=x^2$ στο $1$ο τεταρτημόριο. \hfill Απ: $\frac{1}{6}$
\item $y=x+2$ και $y=x^2$ \hfill Απ: $\frac{9}{2}$
\item $y=3-2x^2$ και $y=x^4$ \hfill Απ: $\frac{64}{15}$
\item $x=\frac{1}{4}$ και $y^2=4x$ \hfill Απ: $\frac{1}{3}$
\item $y^2=x$ και $y=x^2$ \hfill Απ: $\frac{1}{3}$
\item $xy=2$, $4y=x^2$ και $y=4$ \hfill Απ: $\frac{28}{3}-2\ln 4$

\end{enumerate}

\vspace{\baselineskip}

\item Να υπολογιστεί ο \textbf{όγκος} του στερεού που περικλείεται κάτω από την επιφάνεια $z=x^2+y^2+1$ και πάνω από το τετραγωνικό χωρίο $D$ πλευράς $1$ του επιπέδου $Oxy$.

\hfill Απ: $\frac{5}{3}$

\vspace{\baselineskip}



\item Να υπολογιστεί ο \textbf{όγκος} του τετραέδρου που περικλείεται από το επίπεδο $x+y+z=1$ και τα επίπεδα των αξόνων. 

\hfill Απ: $\frac{1}{6}$ 



\end{enumerate}

\pagebreak

\begin{center}
\fbox{\large\bf Υποδείξεις}
\end{center}

\vspace{\baselineskip}

\begin{enumerate}

\item Ο τύπος του εμβαδού επίπεδου χωρίου $D$ είναι: 
\[
E_{D}=\iint\limits_{D}\,dxdy
\]

\vspace{\baselineskip}

\item Ο τύπος του όγκου του στερεού που περικλείεται κάτω από τη γραφική παράσταση επιφάνειας $z=f(x,y)$ και ενός επίπεδου χωρίου $D$ του επιπέδου $Oxy$ είναι: 
\[
V=\iint\limits_{D}f(x,y)\,dxdy
\]

\vspace{\baselineskip}

\item Η εξίσωση του επιπέδου της άσκησης $6$ είναι $x+y+z=1$, άρα η ζητούμενη επιφάνεια του επιπέδου θα είναι $z=\underbrace{1-x-y}_{f(x,y)}$.

\vspace{\baselineskip}

\item Οι εξισώσεις των συντεταγμενων επιπέδων είναι:
\begin{itemize}
\item $Oxy: \, z=0$
\item $Oxz: \, y=0$
\item $Oyz: \, x=0$
\end{itemize}



\end{enumerate}



\end{document}
