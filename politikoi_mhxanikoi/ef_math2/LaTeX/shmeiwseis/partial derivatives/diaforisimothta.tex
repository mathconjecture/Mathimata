\chapter{Διαφορισιμότητα}

\section{Ορισμός}
\begin{dfn}
\item {}
  Μια συνάρτηση $ f(x,y) $ λέγεται \textcolor{Col1}{διαφορίσιμη} στο σημείο 
  $ P_{0}(x_{0}, y_{0}) $ αν ισχύει η παρακάτω \textbf{συνθήκη διαφορισιμότητας}: 
  \[
    \eval{\Delta f(x,y)}_{P_{0}} \!\!= f(x,y)-f(x_{0}, y_{0}) 
    \xlongequal[k=\underbrace{y- y_{0}}_{\Delta y=dy}]{h=\overbrace{x-
    x_{0}}^{\Delta x=dx}}  f(x_{0}+h, y_{0}+k)-f(x_{0}, y_{0}) =
    \!\!\! \underbrace{\eval{\pdv{f}{x}} _{P_{0}}\!\!\!\cdot h + 
    \eval{\pdv{f}{y} }_{P_{0}}\!\!\!\cdot k}_
    {\minibox[c]{$\eval{df(x,y)}_{P_{0}}$ \\ Πρωτεύον
    μέρος: \\ Γραμμικό ως προς $h,k$}} + \!\!\!\!  
    \underbrace{\varepsilon _{1}(h,k)h+ 
    \varepsilon _{2}(h,k)k}_{\minibox[c]{$G(h,k)$ : (πολύ μικρό) 
  \\ Δευτερεύον μέρος: \\ Μη γραμμικό ως προς $h,k$}}                 
\]
\end{dfn}

\begin{rem}
\item {}
  Η συνάρτηση $ G(h,k) = \eval{\Delta f(x,y)}_{P_{0}} - \eval{df(x,y)}_{P_{0}} $
\end{rem}

\begin{dfn}[Επαναδιατύπωση]
\item {}
  Μια συνάρτηση $ f(x,y) $ είναι \textcolor{Col1}{διαφορίσιμη} σε ένα σημείο 
  $ P_{0}(x_{0}, y_{0}) $ αν και μόνον αν
  \begin{enumerate}[i)]
    \item Υπάρχουν οι μερικές παράγωγοι $ \eval{\pdv{f}{x}}_{P_{0}},
      \eval{\pdv{f}{y}}_{P_{0}} $
    \item $ \lim\limits_{\substack{h\to 0 \\k \to 0}} \varepsilon _{1}(h,k) = 
      0 \quad \text{και} \quad \lim\limits_{\substack{h\to 0 \\k \to 0}} 
      \varepsilon _{2}(h,k)=0 \overset{\text{ή}}{\Leftrightarrow} 
      \lim\limits_{\substack{h\to 0 \\k \to 0}} 
      \frac{G(h,k)}{\sqrt{h^{2}+k^{2}}} = 0 $.
  \end{enumerate}
\end{dfn}

\begin{prop}
  Κάθε διαφορίσιμη συνάρτηση είναι και συνεχής.
\end{prop}

\begin{rem}
\item {}
  \begin{enumerate}[i)]
    \item Το αντίστροφο της παραπάνω πρότασης δεν ισχύει.
    \item Το αντιθετοαντίστροφο της πρότασης μας λέει, ότι αν μια συνάρτηση, 
      δεν είναι συνεχής, τότε δεν είναι διαφορίσιμη.
  \end{enumerate}
\end{rem}

\begin{thm}
\item {}
  Μια συνάρτηση $ f(x,y) $ είναι διαφορίσιμη στο $ P_{0}(x_{0}, y_{0}) $ 
  αν υπάρχουν οι μερικές παράγωγοι και είναι συνεχείς στο $ P_{0} $.
\end{thm}

\begin{example}
  Να εξεταστεί ως προς τη διαφορισιμότητα στο σημείο $(0,0)$ 
  η συνάρτηση 
  \[ 
    f(x,y) = 
    \begin{cases} 
      \frac{xy}{x^{2}+y^{2}}, &(x,y) \neq (0,0) \\ 0, & (x,y) = (0,0) 
    \end{cases} 
  \]
\end{example}
\begin{solution}
\item {}
  Εξετάζουμε τη συνέχεια της συνάρτησης στο $ (0,0) $. Έχουμε
  \[ \lim\limits_{\substack{x\to 0 \\y \to 0}} f(x,y) =
    \lim\limits_{\substack{x\to 0 \\y \to 0}} \frac{xy}{x^{2}+y^{2}} =
    \lim_{r \to 0} \frac{r^{2} \cos{\theta} \sin{\theta}}{r^{2}} =
  \cos{\theta} \sin{\theta} \] άρα δεν υπάρχει το όριο, γιατί εξαρτάται από το 
  $\theta$, επομένως η συνάρτηση δεν είναι συνεχής, άρα ούτε κ παραγωγίσιμη στο 
  $ (0,0) $.
\end{solution}

\begin{example}
\item Να εξεταστεί ως προς τη διαφορισιμότητα στο σημείο $(0,0)$ 
  η συνάρτηση 
  \[
    f(x,y) = 
    \begin{cases} 
      x \frac{x^{2}-y^{2}}{x^{2}+y^{2}}, &(x,y) \neq (0,0) \\ 0, & (x,y) = (0,0)  
    \end{cases} 
  \] 
\end{example}
\begin{solution}
  \begin{description}
    \item [Α΄ Τρόπος: (Με το θεώρημα)]
    \item {}
      Εξετάζουμε την ύπαρξη των μερικών παραγώγων της $f$ στο $ (0,0) $.
      \begin{myitemize}
        \item Αν $ (x,y) \neq (0,0) $ έχουμε:
          \[
            f_{x} = \left(x \frac{x^{2}-y^{2}}{x^{2}+y^{2}}\right)_
            {x} = \cdots = \frac{x^{4}+4x^{2}y^{2}-y^{4}}{(x^{2}+
            y^{2})^{2}} \quad \text{και} \quad f_{y} = 
            \left(x\frac{x^{2}-y^{2}}{x^{2}+y^{2}}\right)_{y} = 
            \cdots = - \frac{4x^{3}y}{(x^{2}+y^{2})^{2}} 
          \] 

        \item Αν $ (x,y) = (0,0) $ έχουμε:
          \begin{align*}
            f_{x}(0,0) = \lim_{x \to 0} \frac{f(x,0)-f(0,0)}{x-0} = 
            \lim_{x \to 0} = \lim_{x \to 0} \frac{x}{x} = 1
            \quad \text{και} \quad
            f_{y}(0,0) = \lim_{y \to 0} \frac{f(0,y)-f(0,0)}{y-0} = 
            \lim_{y \to 0} 0 = 0
          \end{align*} 
          Επομένως υπάρχουν οι μερικές παράγωγοι της $f$ στο $ (0,0) $
          και έχουμε:
          \[
            f_{x}(x,y) = 
            \begin{cases}
              \frac{x^{4}+4x^{2}y^{2}-y^{4}}{(x^{2}+y^{2})^{2}}, 
                                    & (x,y) \neq (0,0) \\ 1, & (x,y) = (0,0) 
            \end{cases} \quad \text{και} \quad
            f_{y}(x,y) = \begin{cases} - 
              \frac{4x^{3}y}{(x^{2}+y^{2})^{2}}, &(x,y)
            \neq (0,0) \\ 0, & (x,y) = (0,0) \end{cases} 
          \]
      \end{myitemize}
      Εξετάζουμε ως προς τη συνέχεια τις μερικές παραγώγους.
      \begin{myitemize}
        \item Για την $ f_{x} $
          \begin{align*}
            L_{1} = \lim_{x \to 0} 
            \left(\lim_{y \to 0}
              \frac{x^{4}+4x^{2}y^{2}-y^{4}}{(x^{2}+y^{2})^{2}}
            \right) = 
            \lim_{x \to 0} 1 = 1
            \intertext{και}
            L_{2}= \lim_{y \to 0} \left(\lim_{x \to 0}
            \frac{x^{4}+4x^{2}y^{2}-y^{4}}{(x^{2}+y^{2})^{2}}\right) = 
            \lim_{y \to 0} (-1) = -1
          \end{align*}
        \item Για την $ f_{y} $
          \begin{align*}
            \lim\limits_{\substack{x\to 0 \\y \to 0}} -
            \frac{4x^{3}y}{(x^{2}+y^{2})^{2}} = \lim_{r \to 0} -
            \frac{4r^{4} \cos^{3}{\theta} \sin{\theta}}{(r^{2})^{2}} = 
            \lim_{r \to 0} (-4 \cos^{3}{\theta} \sin{\theta}) = 
            -4 \cos^{3}{\theta} \sin{\theta} \quad \text{(δεν υπάρχει)}
          \end{align*}
      \end{myitemize}

      Επομένως καμία από τις μερικές παραγώγους της $f$ δεν είναι συνεχής στο
      $(0,0)$, άρα η $f$ δεν είναι διαφορίσιμη σύμφωνα με το θεώρημα.
    \item [Β᾽ Τρόπος: (Με ορισμό)]
    \item {}
      Εξετάζουμε με τον ορισμός την ύπαρξη των μερικών παραγώγων της $f$
      και αν η $ G(h,k) $ είναι πολύ μικρή.
      \begin{enumerate}[i)]
        \item Υπάρχουν οι μερικές παράγωγοι της $f$ στο $ (0,0) $.
        \item Εύρεση της $ G(h,k) $.
          \[
            G(h,k) = \Delta f(0,0) - df(0,0) 
          \] 
      \end{enumerate}
      \begin{myitemize}
        \item $ \Delta f(0,0) = f(0+h,0+k) - f(0,0) = 
          h \frac{h^{2}-k^{2}}{h^{2}+k^{2}} - 0 = h
          \frac{h^{2}-k^{2}}{h^{2}+k^{2}} $
        \item $ df(0,0) = f_{x}(0,0)h+f_{y}(0,0)k = 1h+0k= h $
      \end{myitemize}
      Επομένως $ G(h,k) = h \frac{h^{2}-k^{2}}{h^{2}+k^{2}} - h =
      - \frac{2hk^{2}}{h^{2}+k^{2}}$
      \[
        \lim\limits_{\substack{h\to 0 \\k \to 0}} 
        \frac{G(h,k)}{\sqrt{h^{2}+k^{2}}} =
        \lim\limits_{\substack{h\to 0 \\k \to 0}} 
        \frac{-\frac{2hk^{2}}{h^{2}+k^{2}}}{\sqrt{h^{2}+k^{2}}}
        = \lim\limits_{\substack{h\to 0 \\k \to 0}}-
        \frac{2hk^{2}}{(h^{2}+k^{2})^{\frac{3}{2}}} =
        -2 \lim_{r \to 0} \frac{4r^{3} \cos{\theta}
        \sin^{2}{\theta}}{(r^{2})^{\frac{3}{2}}} =
        -2 \cos{\theta} \sin^{2}{\theta}  
      \]
      Επομένως η $f$ δεν είναι διαφορίσιμη στο $ (0,0) $.  
  \end{description}
\end{solution}

\begin{example}
  Να εξετάσετε ως προς τη διαφορισιμότητα τη συνάρτηση:
  \[
    f(x,y) = 
    \begin{cases}
      y^{2} \sin{\frac{x}{y}}, &y \neq 0 \\0, &y=0 
    \end{cases}
  \]
\end{example}
\begin{solution}
  Ελέγχουμε την ύπαρξη των μερικών παραγώγων της $f$.
  \begin{myitemize}
    \item Για $ y \neq 0 $ η $f$ είναι διαφορίσιμη ως σύνθεση διαφορίσιμων
      συναρτήσεων και έχουμε: 
      \[
        f_{x} = y \cos{\frac{x}{y}}  \quad \text{και} \quad  f_{y} = 2y
        \sin{\frac{x}{y}} - x \cos{\frac{x}{y}} 
      \]
    \item Για $ y = 0 $, έχουμε
      \begin{align*}
        f_{x}(x,0) = \lim_{h \to 0} \frac{f(x+h,0)-f(x,0)}{h} = 
        \lim_{h \to 0} \frac{0 - 0}{h} = \lim_{h \to 0} 0 = 0
        \intertext{και}
        f_{y}(x,0) = \lim_{k \to 0} \frac{f(x,0+k)-f(x,0)}{k} = \lim_{k \to
        0} \frac{k^{2} \sin{\frac{x}{k}} - 0}{k} = \lim_{k \to 0}
        k \sin{\frac{x}{k}} = 0
      \end{align*}
  \end{myitemize}
  Επομένως υπάρχουν οι μερικές παράγωγοι της $ f $.  

  Ελέγχουμε τη συνέχεια της $ f_{x} $ στο $ y=0 $. Παρατηρούμε:
  \[
    f_{x}(x,y) = 
    \begin{cases}
      y \cos{\frac{x}{y}}, & y \neq 0 \\ 0, & y=0 
    \end{cases}
  \] 
  είναι συνεχής στο $ y=0 $, γιατί 
  $ \lim_{y \to 0} y \cos{\frac{x}{y}} = 0 = f_{x}(0,0) $. 
  Άρα η $f$ είναι διαφορίσιμη.
\end{solution}



