\documentclass[a4paper,table]{report}
\input{preamble_ask.tex}
\input{definitions_ask.tex}
\input{myboxes.tex}

\geometry{top=1cm}

\pagestyle{vangelis}
\everymath{\displaystyle}
\setcounter{chapter}{1}

\begin{document}

\chapter*{Προβλήματα Συνοριακών Τιμών}

Θυμόμαστε από το μάθημα των Συνήθων Διαφορικών Εξισώσεων, ότι ένα πρόβλημα αρχικών τιμών,
αποτελείται από μια συνήθη διαφορική εξίσωση, μαζί με κάποιες αρχικές συνθήκες,
που αφορούν την άγνωστη συνάρτηση και τις παραγώγους της και είναι υπολογισμένες σε 
κάποια συνήθως αρχική τιμή του Πεδίου Ορισμού της ανεξάρτητης μεταβλητής.

\section*{Προβλήματα Συνοριακών Τιμών}

\section*{Sturm Liouville}

Έστω η σδε 2ης τάξης
\begin{equation}\label{eq:sl}
  [f(x)y']' + [h(x)+ \lambda w(x)]y = 0, \quad x \in [a,b]  
\end{equation}
η οποία ικανοποιεί τις συνοριακές συνθήκες 
\begin{gather}\label{eq:cond}
  a_{1}y(a)+ a_{2}y'(a)=0 \\
  b_{1}y(b)+ b_{2}y'(b)=0 \notag, 
\end{gather} 
όπου $\lambda$ είναι παράμετρος και $ a_{1}, a_{2}, b_{1}, b_{2} $ πραγματικές σταθερές,
για τις οποίες ισχύει ότι σε κάθε συνθήκη δεν μπορεί να είναι ταυτόχρονα και οι δύο
σταθερές 0.

\begin{dfn}
  Η εξίσωση~\eqref{eq:sl} ονομάζεται εξίσωση Sturm Liouville και αν συνοδεύεται από 
  συνοριακές συνθήκες της μορφής~\eqref{eq:cond} ονομάζεται Πρόβλημα Συνοριακών Τιμών 
  Sturm Liouville ή απλώς Πρόβλημα Sturm Liouville.
\end{dfn}



\subsection*{Ιδιότητες Ιδιοτιμών και Ιδιοσυναρτήσεων}

\begin{enumerate}
  \item Οι ιδιοτιμές του ΠΣΤ είναι πραγματικές, απλές, αριθμήσιμες, διατεταγμένες και 
    υπάρχει ελάχιστη ιδιοτιμή, δηλαδή ισχύει:
    \[
      \lambda _{1} < \lambda _{2} < \lambda _{3} \cdots 
     \]
     και επιπλέον $ \lim_{n \to \infty} \lambda _{n} = \infty $

   \item Έστω $ \lambda _{n} $ ιδιοτιμή με αντίστοιχη ιδιοσυνάρτηση $ \phi _{n}(x) $. 
     Η $ \phi _{n}(x) $ έχει ακριβώς $ n-1 $ σημεία μηδενισμού στο $ (a,b) $.

   \item Το σύνολο των ιδιοσυναρτήσεων είναι πλήρες, δηλαδή κάθε $ f \in L^{2}(a,b) $ 
     μπορεί να αναπαρασταθεί ως γενικευμένη σειρά Fourier 
     \[
       f(x) = \sum_{n=1}^{\infty} c_{n} \phi _{n}(x) 
      \] 

%todo diaforikes να συνεχίσω τη θεωρια Sturm Liouville


     
\end{enumerate}

\end{document}
