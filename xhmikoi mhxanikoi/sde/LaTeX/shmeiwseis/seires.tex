\input{preamble_ask.tex}
\input{definitions_ask.tex}


\pagestyle{vangelis}
\geometry{left=15.63mm,right=15.63mm,top=30.25mm,bottom=34.25mm,
footskip=24.16mm,headsep=24.16mm}


\begin{document}

\begin{center}
  \minibox{\large\bfseries \textcolor{Col1}{Δυναμοσειρές}}
\end{center}

\vspace{\baselineskip}

\begin{dfn}
  Μια σειρά συναρτήσεων της μορφής $ \sum_{n=0}^{\infty} a_{n}(x- x_{0})^{n} $ 
  όπου $ a_{n} $ και $ x_{0} $ είναι σταθεροί αριθμοί, ονομάζεται 
  \textcolor{Col1}{δυναμοσειρά} με κέντρο το $ x= x_{0} $. 
  Αν $ x_{0}=0 $ τότε έχουμε δυναμοσειρά 
  $ \sum_{n=0}^{\infty} a_{n} x^{n} $ με κέντρο το 0 .
\end{dfn}

\begin{rem}
  Στη συνέχεια θα αναφερθούμε κυρίως σε δυναμοσειρές με κέντρο το 0, αφού η 
  αντικατάσταση $ t=x- x_{0} $ μετασχηματίζει οποιαδήποτε δυναμοσειρά της μορφής 
  $ \sum_{n=0}^{\infty} a_{n}(x- x_{0})^{n} $ σε δυναμοσειρά της μορφής $
  \sum_{n=0}^{\infty} a_{n} x^{n} $. 
\end{rem}

\section*{Ακτίνα και Διάστημα Σύγκλισης}

\begin{dfn}
  Ο αριθμός $ R = \lim_{n \to \infty} \abs{\frac{a_{n}}{a_{n+1}}} $ ονομάζεται 
  \textcolor{Col1}{ακτίνα σύγκλισης} της δυναμοσειράς και ισχύει:
  \begin{myitemize}
    \item Αν $ R=0 $ τότε η σειρά συγκλίνει \textbf{μόνο για $ \mathbf{x=0} $}.
    \item Αν $0 < R < \infty $ τότε η σειρά συγκλίνει για $ \mathbf{\abs{x} < R} $ και 
      αποκλίνει για $ \abs{x} > R $.
    \item Αν $ R= \infty $ τότε η σειρά συγκλίνει \textbf{για κάθε $ \mathbf{x \in
      \mathbb{R}} $}.
  \end{myitemize}
\end{dfn}

\begin{rem}
    Μια δυναμοσειρά συγκλίνει πάντα στο κέντρο της $ x_{0} $.
    Αν το διάστημα σύγκλισης είναι $ \abs{x} < R \Leftrightarrow -R < x < R $, 
      τότε για $ x=R $ και $ x=-R $ η σειρά μπορεί να συγκλίνει ή όχι.
\end{rem}

\begin{example}
  Να βρεθεί το διάστημα και η ακτίνα σύγκλισης της δυναμοσειράς $ \sum_{n=0}^{\infty}
  \frac{n}{2^{n}} x^{n} $.
  \begin{solution}
    \[
      R= \lim_{n \to \infty} \abs{\frac{a_{n}}{a_{n+1}}} = 
      \lim_{n \to \infty} \abs{\frac{\frac{n}{2^{n}}}{\frac{n+1}{2^{n+1}}}} = 
      \lim_{n \to \infty} \left(\frac{n}{n+1} \cdot \frac{2^{n+1}}{2^{n}}\right) = 
      \lim_{n \to \infty} \left(\frac{n}{n+1} \cdot \frac{\cancel{2^{n}} \cdot
      2}{\cancel{2^{n}}} \right) = 
      \lim_{n \to \infty} \left(\frac{n}{n+1} \cdot 2 \right) = 2 \cdot 1 = 2
    \] 
    Επομένως η ακτίνα σύγκλισης της δυναμοσειράς είναι $ R=2 $ και το 
    διάστημα σύγκλισης είναι $ \abs{x} < 2 $.
  \end{solution}
\end{example}

\begin{example}
  Να βρεθεί η ακτίνα σύγκλισης της δυναμοσειράς 
  $ \sum_{n=0}^{\infty} \frac{1}{3^{n}} (x-3)^{2n} $ 
\end{example}
\begin{solution}
  Θέτω $ k=2n$, άρα $ n= k/2 $. Οπότε αν $ n=0 \Rightarrow k=0 $ και αν 
  $ n \to \infty \Rightarrow k \to \infty $. Οπότε η σειρά γίνεται:
  \[
    \sum_{k=0}^{\infty} \frac{1}{3^{k/2}} (x-3)^{k} = \sum_{k=0}^{\infty} 
    \frac{1}{(\sqrt{3}) ^{k}} (x-3)^{k}
  \] 
  \[
    R= \lim_{k \to \infty} \abs{\frac{a_{k}}{a_{k+1}}} = 
    \lim_{k \to \infty} \abs{\frac{\frac{1}{(\sqrt{3} )^{k}}}{\frac{1}{(\sqrt{3})^
    {k+1}}}} = 
    \lim_{k \to \infty} \left(\frac{(\sqrt{3})^{k+1}}{(\sqrt{3})^{k}}\right) = 
    \lim_{k \to \infty} \left(\frac{\cancel{(\sqrt{3})^{k}} \cdot 
      \sqrt{3}}{\cancel{(\sqrt{3})^{k}}} 
    \right) = \sqrt{3}
  \]
  Επομένως η ακτίνα σύγκλισης της δυναμοσειράς είναι $ R= \sqrt{3} $.
\end{solution}

\section*{Πράξεις Δυναμοσειρών}

Έστω $ \sum_{n=0}^{\infty} a_{n}x^{n} $ και $ \sum_{n=0}^{\infty} b_{n}x^{n} $ 
δυναμοσειρές με ακτίνες σύγκλισης $ R_{1} $ και $ R_{2} $ αντίστοιχα και έστω 
$ R = \min \{R_{1}, R_{2}\} $. Για $ \abs{x} < R $ ορίζουμε 
\[
  f(x) = \sum_{n=0}^{\infty} a_{n} x^{n} \quad \text{και} \quad g(x) =
  \sum_{n=0}^{\infty} b_{n} x^{n}
\] 
Τότε για $ \mathbf{\abs{x} < R} $ ισχύει:
\begin{myitemize}
  \item $ f(x)\pm g(x)= \sum_{n=0}^{\infty} (a_{n}\pm b_{n}) x^{n} $
  \item $ \lambda f(x) = \sum_{n=0}^{\infty} (\lambda a_{n}) x^{n}, \quad \lambda \in
    \mathbb{R} $
  \item $ f(x) \cdot g(x) = \sum_{n=0}^{\infty} (a_{0}b_{n} + a_{1}b_{n-1} + \cdots +
    a_{n} b_{0}) x^{n} = \sum_{n=0}^{\infty} \bigl(\sum_{k=0}^{n} a_{k} b_{n-k}\bigr) 
    x^{n} $
\end{myitemize}

\begin{prop}
  Αν για κάθε $ \abs{x} < R $ ισχύει $ \sum_{n=0}^{\infty} a_{n}x^{n} =
  \sum_{n=0}^{\infty} b_{n} x^{n} $ τότε $ a_{n}=b_{n}, \quad \forall n \geq 0 $.
\end{prop}

\begin{prop}
  Ειδικότερα, αν $ \sum_{n=0}^{\infty} a_{n}x^{n} = 0 $ τότε $ a_{n}=0, \quad \forall n
  \geq 0 $.
\end{prop}

\section*{Παραγώγιση Δυναμοσειρών}

Έστω $ f(x) = \sum_{n=0}^{\infty} a_{n}x^{n}, \quad \abs{x} < R $. Τότε η $f$ είναι 
άπειρες φορές παραγωγίσιμη και ισχύει:
\begin{myitemize}
  \item $ f'(x) = \sum_{n=\textcolor{Col1}{1}}^{\infty} n a_{n} x^{n-1} = 
    \sum_{n=\textcolor{Col1}{0}}^{\infty} n a_{n}x^{n-1} $
  \item $ f''(x) = \sum_{n=\textcolor{Col1}{2}}^{\infty} n(n-1) a_{n} x^{n-2} = 
    \sum_{n=\textcolor{Col1}{0}}^{\infty} n(n-1) a_{n}x^{n-2} $
\end{myitemize}

\section*{Αναλυτικές Συναρτήσεις}

Μια συνάρτηση $f$ ονομάζεται \textcolor{Col1}{αναλυτική στο $ x_{0} $}, 
αν μπορεί να γραφεί σε μορφή δυναμοσειράς με κέντρο το $ x_{0} $. 
Όπως είναι γνωστό, από την Ανάλυση, αν η $f$ είναι αναλυτική στο $ x_{0} $ τότε το 
ανάπτυγμά της με μορφή δυναμοσειράς είναι \textbf{μοναδικό} και ισχύει
\[
  f(x) = \sum_{n=0}^{\infty} \frac{f^{(n)}(x_{0})}{n!} (x- x_{0})^{n} 
  \quad \text{\color{Col1}Τύπος Taylor}
\] 
Αν $ x_{0}=0 $ τότε ισχύει 
\[
  f(x) = \sum_{n=0}^{\infty} \frac{f^{(n)}(0)}{n!} x^{n} 
  \quad \text{\color{Col1} Τύπος Maclaurin}
\] 

\section*{Αναπτύγματα βασικών συναρτήσεων}

\setlength{\jot}{10pt}
\begin{align*}
  \color{Col1}{\frac{1}{1-x}}&= 1+x+x^2+\cdots +x^n+\cdots = \sum_{n=0}^{\infty}x^n, 
  \quad \text{για}\; \abs{x}<1 \\
  \color{Col1}\frac{1}{1+x}&=1-x+x^2-\cdots+(-1)^nx^n+\cdots = 
  \sum_{n=0}^{\infty}(-1)^n x^n, \quad \text{για}\; \abs{x}<1 \\
  \color{Col1}e^x&= 1+x+\frac{x^2}{2!}+\cdots+\frac{x^n}{n!}+\cdots = 
  \sum_{n=0}^{\infty}\frac{x^n}{n!}, \quad \text{για}\; \abs{x}<\infty \\
  \color{Col1}\sin x&=x-\frac{x^3}{3!}+\frac{x^5}{5!}-\cdots+\frac{(-1)^n 
  x^{2n+1}}{(2n+1)!}+\cdots = \sum_{n=0}^{\infty}\frac{(-1)^nx^{2n+1}}{(2n+1)!},
  \quad \text{για}\; \abs{x}<\infty \\
    \color{Col1}\cos x&=1-\frac{x^2}{2!}+\frac{x^4}{4!}-\cdots 
    \frac{(-1)^nx^{2n}}{(2n)!}+ \cdots = \sum_{n=0}^{\infty}\frac{(-1)^nx^{2n}}{(2n)!}, 
    \quad \text{για}\; \abs{x}<\infty \\
    \color{Col1}\sinh x&= x +\frac{x^3}{3!}+\frac{x^5}{5!}+\cdots+
    \frac{x^{2n+1}}{(2n+1)!}+\cdots = \sum_{n=0}^{\infty}\frac{x^{2n+1}}{(2n+1)!},
    \quad \text{για}\; \abs{x}<\infty \\
    \color{Col1}\cosh x&= 1+\frac{x^2}{2!}+\frac{x^4}{4!}+\cdots +\frac{x^{2n}}{(2n)!}+
    \cdots = \sum_{n=0}^{\infty}\frac{x^{2n}}{(2n)!}, 
    \quad \text{για}\; \abs{x}<\infty \\
  \end{align*}
  \end{document}
