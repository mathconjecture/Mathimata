\input{preamble_ask.tex}
\input{definitions_ask.tex}


\everymath{\displaystyle}
\pagestyle{askhseis}


\begin{document}

\begin{center}
  \minibox{\large\bfseries \textcolor{Col1}{Ασκήσεις Επανάληψης}}
\end{center}

\vspace{\baselineskip}


\begin{enumerate}

  \item Να λυθούν οι παρακάτω εξισώσεις.

    \begin{enumerate}[i)]
      \item $ (1-x)^{\frac{1}{2}} = 4x-1 $ \hfill Απ: $ x=7/16 $
      \item $ (1-x)^{\frac{1}{2}} = 4x $ \hfill Απ: $ x = \frac{-1+ \sqrt{65}}{32} $ 
    \end{enumerate}

  \item Να βρεθεί η τιμή και η ποσότητα ισορροπίας για τα παρακάτω υποδείγματα:

    \begin{enumerate}[i)]
      \item 
        $ \left.
          \begin{matrix*}[l]
            Q_{d} = 100 - 2P \\
            Q_{s} = 10P-15 \\
            Q_{d}=Q_{s}
          \end{matrix*} 
        \right\}$ 
        \hfill Απ: $ \bar{P} = \frac{115}{12} $, $ \bar{Q} = \frac{485}{6} $

      \item  
        $ 
        \left.
          \begin{matrix*}[l]
            Q_{d} = 50 - 3P \\
            Q_{s}=7P+5 \\
            Q_{d}=Q_{s}
          \end{matrix*} 
        \right\}$
        \hfill Απ: $ \bar{P} = \frac{45}{10} $, $ \bar{Q} = \frac{365}{10} $ 

      \item 
        $ 
        \left.
          \begin{matrix*}[l]
            Q_{d}=10-P^{2} \\
            Q_{s}=8P-4 \\
            Q_{d}=Q_{s}
          \end{matrix*} 
        \right\}$
        \hfill Απ: $ \bar{P}=-4\pm \sqrt{30} $   

      % \item 
      %   $ 
      %   \left.
      %     \begin{matrix*}[l]
      %       Q_{d_{1}}=28-3P_{1}+P_{2} \\
      %       Q_{s_{1}}=-4+4P_{1} \\
      %       Q_{d_{1}}=Q_{s_{1}} \\
      %       Q_{d_{2}}=10+P_{1}-P_{2} \\
      %       Q_{s_{2}}=-5+9P_{2} \\
      %       Q_{d_{2}}=Q_{s_{2}}
      %     \end{matrix*} 
      %   \right\}$
      %   \hfill Απ: $ \bar{P}_{1} = \frac{335}{69} $, $ \bar{P}_{2} = \frac{137}{69} $  
    \end{enumerate}

  \item Να λυθεί το παρακάτω σύστημα με τη μέθοδο Crammer.
    \[
      \left.
        \begin{matrix*}[l]
          Y=C+I_{0}+G_{0} \\
          C=a+bY
        \end{matrix*} 
      \right\}
    \]
    όπου $ a>0,\; 0<b<1 $ και $ I_{0}, \; G_{0} $ σταθερές.

    \hfill Απ: $ \bar{Y} = \frac{a + I_{0}+G_{0}}{1-b} $, 
    $ \bar{C} = \frac{a + b(I_{0}+G_{0})}{1-b} $  

  \item Να υπολογιστούν τα παρακάτω όρια.

    \begin{enumerate}[i)]
      % \item $ \lim_{x\to 4^{+}} \frac{\abs{x-4}}{x^{2}-16} $ \hfill Απ: $ \frac{1}{8} $
      % \item $ \lim_{x\to 8^{+}} \frac{3x+2}{9x^{2}-69x-24} $ \hfill Απ: $ +\infty $
      % \item $ \lim_{x\to +\infty} (\sqrt{x^{2}-4x+1} - x) $ \hfill Απ: $ -2 $ 
      \item $ \lim_{x\to 4} \frac{x^{2}+13x+36}{x+4} $ \hfill Απ: $ 13 $
      \item $ \lim_{x\to 1} \frac{2x^{2}+3x-4}{x^{2}+5x-3} $ \hfill Απ: $ \frac{1}{3} $
      \item $ \lim_{x\to 2} (16+ \sqrt{x+2}) $ \hfill Απ: $ 18 $
      \item $ \lim_{x\to 1} \frac{2x^{2}+3x-5}{x^{2}+5x-6} $ \hfill Απ: $ 1 $
      \item $ \lim_{x\to 2} \frac{\sqrt{2x+5} - 3}{x-2} $ \hfill Απ: $ \frac{1}{3} $
      \item $ \lim_{x\to 4} \frac{x^{3}-7x^{2}+17x-20}{x^{2}-5x+4} $ \hfill Απ: $ 3 $ 
    \end{enumerate}

  \item Να βρεθεί η ολική παράγωγος των ακόλουθων συναρτήσεων

    \begin{enumerate}[i)]
      \item $ z = f(x,y) = 2x+xy-y^{2} $, όπου $ x = g(y) = 3y^{2} 
        $ \hfill Απ: $ \dv{f}{y} = 9y^{2}+10y $ 
      \item $ z = f(x,y) = 6x^{2}-3xy+2y^{2} $, όπου $ x = g(y) = \frac{1}{y} $
        \hfill Απ: $ \dv{f}{y} = - \frac{12}{y^{3}} + \frac{4}{y} $
      \item $ z = f(x,y) = (x+y)(x-2y) $, όπου $ x = g(y) = 2-7y $
        \hfill Απ: $ \dv{f}{y} = -30 + 108y $ 
    \end{enumerate}

  % \item Δίνεται η συνάρτηση κατανάλωσης $ C=a+by $ με $ a>0 $ και $ 0<b<1 $.

  %   \begin{enumerate}[i)]
  %     \item Να βρεθεί η οριακή και η μέση συνάρτηση κατανάλωσης. 
  %     \item Να βρεθεί η εισοδηματική ελαστικότητα της κατανάλωσης και να προσδιοριστεί το
  %       πρόσημο υποθέτοντας ότι $ y>0 $. 
  %     \item Να δείξετε ότι αυτή η συνάρτηση κατανάλωσης είναι ανελαστική σε όλα τα θετικά
  %       επίπεδα εισοδήματος. 

  %       \hfill Απ:  \begin{enumerate*}[i)]
  %         \item  $ AC = \frac{a}{y} +b $, $ MC =b $  \;
  %         \item $ \varepsilon_{C,y} = \frac{by}{a+by} $ \;
  %         \item $ \abs{ \varepsilon_{C,y}}<1,\; \forall y>0 $ \;
  %       \end{enumerate*}
  %   \end{enumerate}

  \item Να βρεθούν τα στάσιμα σημεία των παρακάτω συναρτήσεων αν υποθέσουμε ότι το 
    πεδίο ορισμού τους είναι το σύνολο των πραγματικών αριθμών.

    \begin{enumerate}[i)]
      \item $ f(x) = -2x^{2}+4x+9 $ \hfill Απ: $ (1,11) $
      \item $ f(x) = 5x^{2}+x $ \hfill Απ: $ \left(-\frac{1}{10}, -\frac{1}{20}\right) $ 
    \end{enumerate}

  \item Να δείξετε ότι η συνάρτηση $ f(x) = x + \frac{1}{x} $ με $ x \neq 0 $ έχει δύο 
    σχετικά ακρότατα, ένα μέγιστο και ένα ελάχιστο. 

    \hfill Απ: $ x_{min} = 1$, \; $ x_{max}=-1  $ 

  \item Να βρεθούν η 2η και η 3η παράγωγος των παρακάτω συναρτήσεων.

    \begin{enumerate}[i)]
      \item $ f(x) = ax^{2}+bx+c $ \hfill Απ: $ f''(x)=2a $
      \item $ f(x) = 6x^{4}-3x-4 $ \hfill Απ: $ f''(x)=72x^{2} $ 
    \end{enumerate}

  \item Να βρεθούν τα μέγιστα και τα ελάχιστα των παρακάτω συναρτήσεων με το 
    κριτήριο της 2ης παραγώγου.

    \begin{enumerate}[i)]
      \item $ y = -2x^{2}+8x+25 $ \hfill Απ: $ f_{max}(2)=33 $
      \item $ y = x^{3}+6x^{2}+7 $ \hfill Απ: $ f_{max}(0)=7 $, $ f_{min}(-4)=39 $ 
    \end{enumerate}

  \item Να βρεθούν οι τιμές των παρακάτω παραγοντικών εκφράσεων.

    \begin{enumerate}[i)]
      \item $ 5! $ \hfill Απ: $ 120 $
      \item $ \frac{(n+2)!}{n!} $ \hfill Απ: $ (n+1)(n+2) $
    \end{enumerate}

  \item Θεωρώντας ως δεδομένο ότι η $ e^{t} $ είναι ίση με την παράγωγό της, 
    χρησιμοποιείστε τον κανόνα αλυσίδας για να βρείτε την $ \dv{y}{t} $ για τις 
    επόμενες συναρτήσεις.

    \begin{enumerate}[i)]
      \item $ y = e^{5t} $ \hfill Απ: $ \dv{y}{t} = 5e^{5t} $
      \item $ y = 4e^{3t} $ \hfill Απ: $ \dv{y}{t} = 12e^{3t} $ 
    \end{enumerate}

  \item Χρησιμοποιώντας την άπειρη σειρά του $ e^{x} $ να βρεθεί μια προσεγγιστική τιμή 
    (εως 3 δεκαδικά ψηφία) των 

    \begin{enumerate}[i)]
      \item $ y=e^{2} $ \hfill Απ: $ e^{2} \approx 7,388 $ 
      \item $ y=e^{\frac{1}{2}}= \sqrt{e} $ \hfill Απ: $ \sqrt{e} \approx 1,649 $ 
    \end{enumerate}

  \item Να υπολογιστούν με τη βοήθεια των ιδιοτήτων των λογαριθμικών συναρτήσεων οι 
    παρακάτω παραστάσεις.

    \begin{enumerate}[i)]
      \item $ \ln{e^{2}} $ \hfill Απ: $ 2 $ 
      \item $ \ln{e^{x}} - e^{\ln{x}} $ \hfill Απ: $ 0  $ 
      \item $ \ln{\frac{1}{e^{3}}} $ \hfill Απ: $ -3 $ 
    \end{enumerate}

  % \item Να βρεθεί η αντίστροφη της συνάρτησης $ y = ab^{ct}$ \hfill Απ: $ t = \frac{\log_{b}{y} -
  %   \log_{b}{a}}{c} $ 

  \item Να βρεθούν οι παράγωγοι των παρακάτω συναρτήσεων.

    \begin{enumerate}[i)]
      \item $ y = 5^{t} $ \hfill Απ: $ y'=5^{t} \ln{5} $
      \item $ y = 13^{2t+3} $ \hfill Απ: $ y' = 2\cdot 13^{2t+3} \ln{13}  $ 
      \item $ y = \ln{(t+9)} $ \hfill Απ: $y'= \frac{1}{t+9} $
      \item $ y = xe^{x} $ \hfill Απ: $ y'=e^{x}(1+x) $ 
    \end{enumerate}
\end{enumerate}


  \end{document}
