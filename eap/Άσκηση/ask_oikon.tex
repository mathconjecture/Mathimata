\input{C:/Users/Vang/Desktop/MyNotesLaTeX/preamble.tex}
\newcommand{\vect}[2]{(#1_1,\ldots, #1_#2)}
%%%%%%% nesting newcommands $$$$$$$$$$$$$$$$$$$
\newcommand{\function}[1]{\newcommand{\nvec}[2]{#1(##1_1,\ldots, ##1_##2)}}

\newcommand{\linode}[2]{#1_n(x)#2^{(n)}+#1_{n-1}(x)#2^{(n-1)}+\cdots +#1_0(x)#2=g(x)}

\newcommand{\vecoffun}[3]{#1_0(#2),\ldots ,#1_#3(#2)}

\newcommand{\suma}{\sum_{n=0}^{\infty}a_n x^n}

\newcommand{\sumb}{\sum_{n=1}^{\infty}a_n n x^{n-1}}

\newcommand{\sumc}{\sum_{n=2}^{\infty}a_n n (n-1) x^{n-2}}

\newcommand{\varsum}[2]{\sum_{n=#1}^{#2}}



\begin{document}

\begin{center}
\fbox{\large \bfseries Ασκήσεις προς επίλυση με τη βοήθεια του \textlatin{Maxima}}
\end{center}

\vspace{2\baselineskip}

\begin{enumerate}[\bfseries Άσκηση 1\ : ]
\item Υποθέτουμε ότι έχουμε μια επιχείρηση της οποίας η συνάρτηση προσφοράς ενός προϊόντος της είναι η: $10P=Q_s+5000$, και ότι η ζήτηση για το προϊόν αυτό διαμορφώνεται στο ύψος των $550$ μονάδων την εβδομάδα, όταν η τιμή του είναι $400$ ν.μ. και $760$ μονάδες, όταν η τιμή του έιναι $300$ ν.μ.
\begin{enumerate}[a)]
\item Να προσδιοριστεί η συνάρτηση ζήτησης με την προυπόθεση ότι είναι γραμμική.
\item Να βρεθει η τιμή και η ποσότητα ισορροπίας.
\item Να γίνει γραφική επίλυση - απεικόνιση της αγοράς.
\end{enumerate}

\underline{\bfseries Λύση}

\begin{enumerate}[a)]

\item Σύμφωνα με την υπόθεση, η συνάρτηση ζήτησης είναι γραμμική συνάρτηση της τιμής. Άρα:
\begin{equation}\label{eq:demand}
Q_d=aP+b
\end{equation}

Επίσης όταν η τιμή είναι $400$ ν.μ. τότε η ζήτηση είναι $550$ μονάδες προϊόντος, άρα ισχύει:
\begin{equation}\label{eq:dem1}
550=a400+b
\end{equation}
και όταν η τιμή είναι $300$ ν.μ. τότε η ζήτηση διαμορφώνεται στις $760$ μονάδες προϊόντος, άρα:
\[
760=a300+b
\]

Επομένως:

\[
\left.
\begin{align*}
550&=a400+b \\
760&=a300+b
\end{align*}
\right\}\Leftrightarrow
\left.
\begin{align*}
550&=\phantom{-}a400+b \\
-760&=-a300-b
\end{align*}
\right\}\Leftrightarrow
\left.
\begin{align*}
550&=\phantom{-}a400+b \\
-760&=-a300-b
\end{align*}
\right\}\overset{+}{\Leftrightarrow}
\]
\[
-210=100a \Leftrightarrow a=-\frac{210}{100} \Leftrightarrow a=-2,1
\]

Οπότε με αντικατάσταση της τιμής του $a$ στην \eqref{eq:dem1} βρίσκουμε ότι $b=1390$.

Άρα η ζητούμενη συνάρτηση ζήτησης είναι:
\[
\boldsymbol{Q_d=-2,1P+1390}
\]

\item Για να βρούμε την τιμή και την ποσότητα ισορροπίας, λύνουμε το σύστημα στην κατάσταση ισορροπίας:
\[
\left.
\begin{align*}
Q_d&=-2,1P+1390 \\
Q_s&=\phantom{-}10P-5000
\end{align*}
\right\}
\]

Όπότε, έχουμε:
\[
Q_s=Q_d
\]
\[
10P-5000=-2,1P+1390\Leftrightarrow 12,1P=6390\Leftrightarrow P=\frac{6390}{12,1} \Leftrightarrow
\]
\[
 \boldsymbol{P^*\approx 528}
\]
Τέλος με αντικατάσταση της τιμής ισορροπίας σε μία από τις δύο εξισώσεις του συστήματος, έστω στην δεύτερη, έχουμε:
\[
Q=10\cdot 528-5000\Leftrightarrow 
\]
\[
\boldsymbol{Q^*=280}
\]

\item 

\begin{figure}[h]
\begin{center}
\includegraphics[scale=.22]{geogebra-export}
\end{center}
\end{figure}

\end{enumerate}


\item Να βρεθούν οι τιμές των $m, n$ έτσι ώστε ο πίνακας 
\(
L=\begin{pmatrix}
m & \phantom{-}1 \\
n & -1
\end{pmatrix}
\)
να έχει αντίστροφο τον ίδιο τον $L$.

Ο πίνακας $L$ θα έχει αντίστροφο τον εαυτό του αν και μόνον αν: $L\cdot L=I$. Οπότε:
\[
\begin{pmatrix}
m & \phantom{-}1\\
n & -1
\end{pmatrix}\cdot 
\begin{pmatrix}
m & \phantom{-}1 \\
n & -1
\end{pmatrix}=
\begin{pmatrix}
1 & 0 \\
0 & 1
\end{pmatrix}\Leftrightarrow
\]
\[
\begin{pmatrix}
m^2+n & m-1 \\
mn-n & n+1
\end{pmatrix}=
\begin{pmatrix}
1 & 0 \\
0 & 1
\end{pmatrix}\Leftrightarrow
\]

Από την ισότητα των πινάκων, προκύπτει το παρακάτω σύστημα:

\[
\left.
\begin{align*}
m^2+n=1 \\
m-1=0 \\
mn-n=0 \\
n+1=1
\end{align*}
\right\}\Leftrightarrow \boldsymbol{n=0} \quad\text{και}\quad \boldsymbol{m=1}
\]
Άρα ο ζητούμενος πίνακας είναι ο:
\[
L=\begin{pmatrix}
1 & \phantom{-}1 \\
0 & -1
\end{pmatrix}
\]

\end{enumerate}


\end{document}