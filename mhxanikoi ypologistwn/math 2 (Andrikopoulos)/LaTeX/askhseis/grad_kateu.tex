\input{preamble_ask.tex}
\input{definitions_ask.tex}


\pagestyle{askhseis}
\everymath{\displaystyle}


\begin{document}

\begin{center}
  {\color{Col1}\minibox{\bfseries\large Ασκήσεις: Κλίση, 
      Παράγωγος κατά Κατεύθυνση }}
\end{center} 

\vspace{\baselineskip}

\section*{Κλίση - Λαπλασιανή - Αρμονικές Συναρτήσεις}

\begin{enumerate}
  \item Να υπολογίσετε την κλίση των παρακάτω συναρτήσεων.
    \begin{enumerate}[i)]
      \item $ f(x,y) = x^{2} \mathrm{e}^{y+x^{3}} $ 
        \hfill Απ: $ \grad(f) = (2x\mathrm{e}^{y+x^{3}}+3x^{4}\mathrm{e}^{y+x^{3}}, 
        x^{2} \mathrm{e}^{y+x^{3}}) $ 
      \item $ f(x,y) = \frac{1}{y+x^{3}} $ 
        \hfill Απ: $\grad(f) =\left(\frac{-3x^{2}}{(y+x^{3})^{2}},
        \frac{-1}{(y+x^{3})^{2}}\right)$ 
      \item $ f(x,y) = x^{2}y^{3}z $   
        \hfill Απ: $ \grad(f) = (2xy^{3}z,3xy^{2}z,x^{2}y^{3}) $ 
      \item $ f(x,y) = x^{2} y \mathrm{e}^{zy}  $ 
        \hfill Απ: $ \grad(f) = (2xy\mathrm{e}^{zy},
        x^{2}\mathrm{e}^{zy}+x^{2}yz\mathrm{e}^{zy},x^{2}y^{2} \mathrm{e}^{zy})$ 
    \end{enumerate}

  \item Αν $ d\mathbf{r} = dx \mathbf{i}+ dy \mathbf{j}$, τότε να αποδείξετε ότι 
    $ \grad{f} \cdot d \mathbf{r} = df $

  \item Να υπολογίσετε τη Λαπλασιανή των παρακάτω συναρτήσεων.
    \begin{enumerate}[i)]
      \item $ f(x,y) = x^{2} \mathrm{e}^{y+x^{3}} $ 
        \hfill Απ: $ \grad^{2} f = 2 \mathrm{e}^{y+x^{3}} + 6x^{3} \mathrm{e}^{y+x^{3}} 
        + 12x^{3} \mathrm{e}^{y+x^{3}} + 9x^{6} \mathrm{e}^{y+x^{3}} + x^{2}
        \mathrm{e}^{y+x^{3}} $ 
      \item $ f(x,y) = \frac{1}{y+x^{3}} $
        \hfill Απ: $ \grad^{2}f = \frac{-6(y+x^{3})+18x^{4}+6x^{2}}{(y+x^{3})^{3}} $ 
    \end{enumerate}

  \item Να δείξετε ότι οι παρακάτω συναρτήσεις είναι αρμονικές:
    \begin{enumerate}[(i)]
      \item $ f(x,y) = x^{3}-3xy^{2} $
      \item $ f(x,y) = \ln(x^{2} + y^{2}) $
    \end{enumerate}
\end{enumerate}

\section*{Παράγωγος κατά κατεύθυνση}

\begin{enumerate}
  \item Έστω η πραγματική συνάρτηση $ f(x,y) = xy$.  Να υπολογίσετε την παράγωγο 
    κατά κατεύθυνση της $ f(x,y) $ στο σημείο $ P(1,2) $ και προς την κατεύθυνση 
    του διανύσματος $ \mathbf{n} = 3 \mathbf{i} + 4 \mathbf{j} $

    \hfill Απ: $ 2 \sqrt{2} $ 

    % \item Να βρεθεί η παράγωγος κατά κατεύθυνση της συνάρτησης 
    % $ f(x,y) = x^{2}y^{3}-4y $ 
    %   στο σημείο $ P(2,-1) $ κατά τη διεύθυνση του διανύσματος 
    %   $ \mathbf{u}=2 \mathbf{i}+5 \mathbf{j} $.	

    % \hfill Απ: $ \frac{32}{\sqrt{ 29 }} $ 

  \item Έστω η πραγματική συνάρτηση $ f(x,y,z) = x^{2}y^{2} + \cos{z} $.  
    Να υπολογίσετε την παράγωγο κατά κατεύθυνση της $ f(x,y,z) $ στο σημείο 
    $ P(1,1,0) $ και προς την κατεύθυνση 
    του διανύσματος $ \mathbf{n} = 2 \mathbf{i} + 2 \mathbf{j} + 0 \mathbf{k} $

    \hfill Απ: $ 2 $ 


  \item Να βρεθεί η παράγωγος κατά κατεύθυνση της συνάρτησης 
    $ f(x,y,z) = x \mathrm{e}^{y^{2}-z^{2}} $ στο σημείο $ P(1,2,-2) $ κατά 
    την κατεύθυνση του διανύσματος $ \mathbf{u} = \mathbf{i}-2 \mathbf{k} $.

    \hfill Απ: $ - 7/\sqrt{5} $ 

  \item Έστω η συνάρτηση $ f(x,y,z) = \frac{1}{x^{2}+y^{2}+z^{2}} = 
    \frac{1}{\rho ^{2}} $. Να υπολογίσετε την παράγωγο κατά κατεύθυνση της 
    συνάρτησης $ f(x,y) $, στο τυχαίο σημείο $ (x,y) $ και προς την κατεύθυνση 
    του διανύσματος $ \grad f $. Τι παρατηρείτε?

    \hfill Απ: $ D_{\grad f}f = \norm{\grad f} = 2/\rho ^{3} $ 

  \item Δίνεται η πραγματική συνάρτηση $ f(x,y) = xe^{y} $.
    \begin{enumerate}[i)]
      \item Να υπολογίσετε την παράγωγο κατά κατεύθυνση της $ f(x,y) $ στο σημείο 
        $ P(2,0) $ και προς την κατεύθυνση του σημείου $ Q(1/2, 2) $.
      \item Να προσδιορίσετε την κατεύθυνση προς την οποία η $ f(x,y) $ παρουσιάζει το
        μεγαλύτερο ρυθμό αύξησης στο σημείο $ P $.
      \item Να υπολογίσετε τη μέγιστη τιμή της παραγώγου κατά κατεύθυνση της 
        $ f(x,y) $ στο σημείο $ P(2,0) $.

        \hfill Απ: $ \rm{i}) 1, \; iii) \sqrt{ 5 } $ 
    \end{enumerate}
\end{enumerate}

\section*{Εφαπτόμενο Επίπεδο - Κάθετη Ευθεία}

\begin{enumerate}
  \item Να υπολογίσετε το εφαπτόμενο επίπεδο και την κάθετη ευθεία της επιφάνειας 
    $ x^{2}+2y^{2}-z = 0 $ στο σημείο $ (1,1,3) $.

    \hfill Απ: \begin{tabular}{l}
      $ \Pi : 2x+4y-z -3 = 0 $ \\
      $ K : \frac{x-1}{2} = \frac{y-1}{4} = \frac{z-3}{-1} $
    \end{tabular} 

  \item Να υπολογίσετε το εφαπτόμενο επίπεδο και την κάθετη ευθεία της επιφάνειας 
    $ z = xy $ στο σημείο $ (2,3,6) $.

    \hfill Απ: \begin{tabular}{l}
      $ \Pi : 3x+2y-z-6 = 0 $ \\
      $ K : \frac{x-2}{3} = \frac{y-3}{2} = \frac{z-6}{-1} $
    \end{tabular} 

  \item Να υπολογίσετε τις εξισώσεις του εφαπτόμενου επιπέδου και την κάθετης ευθείας 
    της επιφάνειας $ x^{2} + y^{2} + z = 9 $ στο σημείο $ P(1,2,4) $.

    \hfill Απ: \begin{tabular}{l}
      $ \Pi : 2x+4y+z-14=0 $ \\
      $ K : \frac{x-1}{2} = \frac{y-2}{4} = z-4 $
    \end{tabular} 

  \item Να υπολογίσετε τις εξισώσεις του εφαπτόμενου επιπέδου και την κάθετης ευθείας 
    της επιφάνειας $ 2x^{2} + 2y^{2} - z^{2} +12=0 $ στο σημείο $ P(1,-1,4) $.

    \hfill Απ: \begin{tabular}{l}
      $ \Pi : x-y-2z+6=0 $ \\
      $ K : \frac{x-1}{-4} = \frac{y+1}{4} = \frac{z-4}{8} $
    \end{tabular} 

  \item Να υπολογίσετε το εφαπτόμενο επίπεδο και την κάθετη ευθεία της επιφάνειας 
    $ \frac{x^{2}}{a^{2}} + \frac{y^{2}}{b^{2}} + \frac{z^{2}}{c^{2}} = 1  $ στο σημείο 
    $ (1,1,2) $.

    \hfill Απ: \begin{tabular}{l}
      $ \Pi : \frac{2}{a^{2}} x+ \frac{2}{b^{2}} y+ \frac{4}{c^{2}} z- 
      \left(\frac{2}{a^{2}}+ \frac{2}{b^{2}}+ \frac{8}{c^{2}}\right) = 0 $ \\
      $ K : \frac{x-1}{2/a^{2}} = \frac{y-1}{2/b^{2}} = \frac{z-2}{4/c^{2}} $
    \end{tabular} 

  \item Να προσδιορίσετε τα σημεία της επιφάνειας $ x^{3}+y^{3}+z-3xy=0 $ στα οποία 
    το εφαπτόμενο επίπεδο είναι οριζόντιο.
    \hfill (\textbf{υπόδειξη:} $\Pi$ οριζόντιο $ \Leftrightarrow \grad f = \mathbf{0} $)

    \hfill Απ: $ (0,0,0) $ και $ (1,1,1) $ 
\end{enumerate}

\end{document}
