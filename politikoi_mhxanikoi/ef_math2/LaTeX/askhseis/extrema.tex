\documentclass[a4paper,table]{report}
\input{preamble.tex}
\input{definitions_ask.tex}

\geometry{left=20.63mm,right=20.63mm,top=20.25mm,bottom=30.25mm,footskip=24.16mm,headsep=24.16mm}

\pagestyle{askhseis}

\renewcommand{\vec}{\mathbf}

\begin{document}

\begin{center}
  \minibox{\large \bfseries \textcolor{Col1}{Ασκήσεις στα Ακρότατα}}
\end{center}


\section*{Τοπικά Ακρότατα}

\begin{enumerate}
  \item Να βρεθούν και να χαρακτηριστούν τα κρίσιμα σημεία  των παρακάτω συναρτήσεων:
    \begin{enumerate}[i)]
      \item $ f(x,y) = x^{3} + y^{3} + 3xy $ 
        \hfill Απ: max: $(-1,-1)  $, σάγμα: $ (0,0) $
      \item $ f(x,y) = x^{2}+y^{4} $ 
        \hfill Απ: min: $ (0,0) $ 
      \item $ f(x,y) = x^{3} + y^{3} - 3x -12y + 50 $ 
        \hfill Απ: max: $ (-1,-2)$, min: $ (1,2) $, 
        σάγμα: $ (1,-2), (-1,2) $
      \item $ f(x,y) = x^{3} + y^{3} -3x -3y + 1 $ 
        \hfill Απ: max: $(-1,-1)  $, min: $ (1,1) $,
        σάγμα: $ (1,-1), (-1,1) $
      \item $ f(x,y) = x^{3} + 4xy -4y^{2} $ 
        \hfill Απ: max: $ (-2/3, -1/3)  $, σάγμα: $ (0,0) $
      \item $ f(x,y) = x^{4} + y^{4} -2(x-y)^{2}$  
        \hfill Απ: min: $ (\sqrt{2} , -\sqrt{2}), (-\sqrt{2} , \sqrt{2}) $, 
        σάγμα: $ (0,0) $
      \item $ f(x,y) = (x^{2}-3y^{2})e^{1-x^{2}-y^{2}} $ 
        \hfill Απ: max: $ (1,0), (-1,0) $, min: $ (0,1), (0,-1) $, 
        σάγμα: $ (0,0) $
      \item $ f(x,y,z) = 2x^{2} + xy + 4y^{2} + xz + z^{2} + 2  $ \hfill Απ: 
        $ f_{\text{min}}(0,0,0) = 2 $
    \end{enumerate}

  \item Να βρεθεί η ελάχιστη απόσταση του επιπέδου με εξίσωση $ x+y+z=4 $, από την 
    αρχή των αξόνων.

    \hfill Απ: $ d_{\min}(4/3,4/3) = 4\frac{\sqrt{3}}{3} $  

  \item Να βρεθεί η ελάχιστη απόσταση του επιπέδου με εξίσωση $ 3x+2y+z=6 $, από την 
    αρχή των αξόνων.

    \hfill Απ: $ d_{\min}(9/7,6/7) = 3\frac{\sqrt{14}}{7} $  

  \item Να βρεθεί η ελάχιστη απόσταση του σημείου $ P(2,-1,1) $ από το επίπεδο με 
    εξίσωση $ x+y-z=2 $. 
    
    \hfill Απ: $ d_{min}(8/3,-1/3) = 2 /\sqrt{3} $ 

  \item Να βρεθεί η ελάχιστη απόσταση του σημείου $ P(-6,4,0) $ από τον κώνο με 
    εξίσωση $ z = \sqrt{x^{2}+y^{2}} $. 
    
    \hfill Απ: $ d_{min}(-3,2) = \sqrt{26} $ 

%   \item Δίνεται τρίγωνο ΑΒΓ. Να βρεθεί σημείο P, στο επίπεδο του τριγώνου, ώστε 
%     το άθροισμα των τετραγώνων των αποστάσεών του από τις κορυφές του τριγώνου 
%     να είναι ελάχιστο.
\end{enumerate}


\section*{Ακρότατα Υπό Συνθήκη}

\begin{enumerate}

    %Thomas 12th 14.8 ex.1 
  \item Να βρείτε τα ακρότατα της συνάρτησης $ f(x,y) = xy $ πάνω στην έλλειψη 
    $ x^{2}+2y^{2}=1 $.

    \hfill Απ: 
    \begin{tabular}{l}
      max: $ f(\sqrt{2} /2, 1/2) = f(- \sqrt{2} /2, -1/2) = \frac{\sqrt{2}}{2} $ \\
      min $ f(\sqrt{2} /2, -1/2) = f(- \sqrt{2} /2, 1/2) = -\frac{\sqrt{2}}{2} $ \\
    \end{tabular}

    %Thomas 12th 14.8 ex.14 
  \item Να βρείτε τα ακρότατα της συνάρτησης $ f(x,y) = 3x-y+6 $ υπό τον περιορισμό 
    $ x^{2}+y^{2}=4 $.

    \hfill Απ:  
    \begin{tabular}{l}
      max: $ f(\frac{6}{\sqrt{10}} , - \frac{2}{\sqrt{10}}) = 2 \sqrt{10} +6 $ \\
      min $ f(-\frac{6}{\sqrt{10}} , + \frac{2}{\sqrt{10}}) = -2 \sqrt{10} +6 $ \\
    \end{tabular}

    %%Thomas 12th 14.8 ex.15 
  %\item Η θερμοκρασία σε κάθε σημείο μιας μεταλλικής πλάκας δίνεται από τη σχέση
    %$ T(x,y) = 4x^{2}-4xy+y^2 $.
    %Ένα μυρμήγκι, που βρίσκεται πάνω στην πλάκα, περπατά στην περιφέρεια κύκλου, 
    %με κέντρο την αρχή των αξόνων και ακτίνας 5. Να υπολογίσετε την μέγιστη και την 
    %ελάχιστη θερμοκρασία που θα συναντήσει το μυρμήγκι κατά τη διαδρομή του.

    % \hfill Απ:  
    % \begin{tabular}{l}
    %   max $ f(2 \sqrt{5} , - \sqrt{5}) = f(-2 \sqrt{5} , \sqrt{5}) = 125 $ \\
    %   min $ f(\sqrt{5} , 2 \sqrt{5}) = f(- \sqrt{5} , - 2 \sqrt{5}) = 0 $ 
    % \end{tabular}

  \item Να υπολογιστούν τα τοπικά ακρότατα της συνάρτησης $ f(x,y,z) = x^{2}+y^{2}+z^{2}
    $ που ικανοποιούν τον περιορισμό $ x+y+z+1=0 $.
    \hfill Απ: min: $ (-1/3,-1/3,-1/3) $ 

  \item Να υπολογιστούν τα τοπικά ακρότατα της συνάρτησης 
    $ f(x,y,z) = x^{2}+y^{2}+z^{2}-2x-2y-z+ \frac{5}{4} $ που ικανοποιούν τον 
    περιορισμό $ x^{2}+y^{2}-z=0  $.
    \hfill Απ: min: $ (1/ \sqrt[3]{4} , 1/ \sqrt[3]{4}) $ 

  \item Να υπολογιστούν τα τοπικά ακρότατα της συνάρτησης 
    $f(x,y,z)=xyz$ που ικανοποιούν την εξίσωση $x+y+z-1=0$.  
    \hfill Απ: max: $ (1/3,1/3,1/3) $ 

  \item Να υπολογιστούν τα τοπικά ακρότατα της συνάρτησης 
    $ f(x,y,z) = x^{2}+y^{2}+z^{2} $, που ικανοποιούν τους περιορισμούς 
    $ zx+zy=-2 $ και $ xy=1 $.
    \hfill Απ: min: $ (-1,-1,1) $, max: $ (1,1,-1) $ 

  \item \textbf Θεωρούμε έναν μεταλλικό κυκλικό δίσκο με εξίσωση 
    $ x^{2}+y^{2}=4 $. Ο δίσκος θερμαίνεται έτσι ώστε η θερμοκρασία $ T $ σε κάθε 
    σημείο του δίσκο $ (x,y) $ να δίνεται από τη σχέση
    \[
      T(x,y) = 2x^{2}+y^{2}-y.
    \] 
    Βρείτε τα πιο θερμά και τα πιο ψυχρά σημεία του μεταλλικού δίσκου καθώς και τη
    θερμοκρασία τους.

    \hfill Απ: min: (0,1/2), max: ($ \sqrt{15}/{2} , -1/2 $), ($ -\sqrt{15}/{2} , -1/2 $)
\end{enumerate}


\end{document}
