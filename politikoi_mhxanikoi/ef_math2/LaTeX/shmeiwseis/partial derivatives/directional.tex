\chapter{Παράγωγος Κατά Κατεύθυνση}


\section{Κλίση - Λαπλασιανή}

\begin{dfn}[Τελεστής Hamilton ή Ανάδελτα]
  \[ \grad = \pdv{}{x} \mathbf{i} + \pdv{}{y} \mathbf{j} + \pdv{}{z} \mathbf{k} \quad
  \text{ή} \quad \grad = \left(\pdv{}{x} , \pdv{}{y} , \pdv{}{z}\right)  \]
\end{dfn}

\begin{dfn}
  Έστω $ f(x,y,z) $ συνάρτηση. Τότε η διανυσματική συνάρτηση 
  \[ \grad f = \pdv{f}{x} \mathbf{i}+ \pdv{f}{y} \mathbf{j} + \pdv{f}{z} \mathbf{k} 
  \quad \text{ή} \quad \grad f = \left(\pdv{f}{x} , \pdv{f}{y} , \pdv{f}{z}\right)\]
  ονομάζεται \textcolor{Col1}{κλίση} της συνάρτησης $ f(x,y,z) $.
\end{dfn}

\begin{dfn}[Τελεστής Laplace]
  \[ 
    \Delta \overset{\text{ή}}{=} \laplacian = \pdv[2]{}{x} + \pdv[2]{}{y} 
    \quad \text{ή} \quad 
    \Delta \overset{\text{ή}}{=} \laplacian = \pdv[2]{}{x} + \pdv[2]{}{y} + \pdv[2]{}{z} 
  \]
\end{dfn}

\begin{dfn}
  Έστω $ f(x,y,z) $ συνάρτηση. Τότε η πραγματική συνάρτηση 
  \[ 
    \Delta f \overset{\text{ή}}{=} \laplacian f = \pdv[2]{f}{x} + \pdv[2]{f}{y} + \pdv[2]{f}{z} 
  \]
  ονομάζεται \textcolor{Col1}{Λαπλασιανή} της συνάρτησης $ f(x,y,z) $.
\end{dfn}


\section{Αρμονικές Συναρτήσεις}

\begin{dfn}
  Η συνάρτηση $ f(x,y) $ λέγεται \textcolor{Col1}{αρμονική} αν ικανοποιεί την
  εξίσωση $ \Delta f = 0 $ ή ισοδύναμα $ \laplacian f = 0 $.  Δηλαδή 
  \[
    \pdv[2]{f}{x} + \pdv[2]{f}{y} = 0  
  \]
\end{dfn}

\begin{rem}
\item {}
  \begin{enumerate}
    \item Η παραπάνω εξίσωση, ονομάζεται \textcolor{Col1}{εξίσωση Laplace}.
    \item Η εξίσωση Laplace για συναρτήσεις πολλών μεταβλητών παίρνει τη 
      μορφή
      \[
        \pdv[2]{f}{x_{1}} + \pdv[2]{f}{x_{2}} + \cdots + 
        \pdv[2]{f}{x_{n}}=0 
      \] 
  \end{enumerate}
\end{rem}

\begin{prop}
\item {}
  Αν $ f(x,y) $ συνεχής, έχει συνεχείς μερικές παραγώγους 1ης τάξης 
  και είναι αρμονική, τότε και οι συναρτήσεις $ f_{x} $ και $ f_{y} $
  είναι αρμονικές. 
\end{prop}

\begin{proof}
\item {}
  $f$ αρμονική $ \Rightarrow f_{xx}+f_{yy}=0 $.

  Θέτουμε $ g(x,y)=f_{x}(x,y) $. Για να δείξουμε ότι $ f_{x} $ αρμονική, 
  αρκεί να δείξουμε ότι $ g(x,y) $ είναι αρμονική. Πράγματι:
  \begin{align*}
    g_{x} &= f_{xx} \quad \text{και} \quad g_{xx} = (f_{xx})_{x} \\ 
    g_{y} &= f_{xy} \quad \text{και} \quad g_{yy} = (f_{xy})_{y} =
    (f_{yx})_{y} = f_{yxy} = f_{yyx} = (f_{yy})_{x}
  \end{align*}
  Άρα 
  \[
    g_{xx}+g_{yy} = (f_{xx})_{x} + (f_{yy})_{x} = 
    (f_{xx}+f_{yy})_{x}= (0)_{x} =0
  \] 
\end{proof}

\begin{example}
  Να εξεταστεί αν η συνάρτηση $ f(x,y)= \ln{\sqrt{x^{2}+y^{2}}} $ είναι αρμονική.
\end{example}
\begin{solution}
  \begin{align*}
    \pdv{f}{x} &= f_{x} = \left(\ln{\sqrt{x^{2}+y^{2}}} \right)_{x} = 
    \frac{1}{\sqrt{x^{2}+y^{2}}} \cdot 
    \left(\sqrt{x^{2}+y^{2}}\right)_{x} = 
    \frac{1}{\sqrt{x^{2}+y^{2}}} \cdot \frac{\cancel{2}x}{\cancel{2} 
    \sqrt{x^{2}+y^{2}}} = \frac{x}{x^{2}+y^{2}} \\
    \pdv[2]{f}{x} &= f_{xx} = \left(\frac{x}{x^{2}+y^{2}}\right)_{x} = 
    \frac{x^{2}+y^{2}-2x^{2}}{(x^{2}+y^{2})^{2}} = 
    \frac{y^{2}-x^{2}}{(x^{2}+y^{2})^{2}} 
    \intertext{και}
    \pdv{f}{y} &= f_{y} = \left(\ln{\sqrt{x^{2}+y^{2}}} \right)_{y} = 
    \frac{1}{\sqrt{x^{2}+y^{2}}} \cdot 
    \left(\sqrt{x^{2}+y^{2}}\right)_{y} = 
    \frac{1}{\sqrt{x^{2}+y^{2}}} \cdot \frac{\cancel{2}y}{\cancel{2} 
    \sqrt{x^{2}+y^{2}}} = \frac{y}{x^{2}+y^{2}} \\
    \pdv[2]{f}{y} &= f_{yy} = \left(\frac{y}{x^{2}+y^{2}}\right)_{y} = 
    \frac{x^{2}+y^{2}-2y^{2}}{(x^{2}+y^{2})^{2}} = 
    \frac{x^{2}-y^{2}}{(x^{2}+y^{2})^{2}} 
  \end{align*}  
  Προφανώς έχουμε ότι
  \[
    \pdv[2]{f}{x} + \pdv[2]{f}{y} = f_{xx}+f_{yy} =
    \frac{y^{2}-x^{2}}{(x^{2}+y^{2})^{2}} + 
    \frac{x^{2}-y^{2}}{(x^{2}+y^{2})^{2}} = 0  
  \] 
\end{solution}

\section{Παράγωγος κατά Κατεύθυνση}

\begin{dfn}
  Η \textcolor{Col1}{παράγωγος κατά κατεύθυνση} της \textbf{διαφορίσιμης} συνάρτησης 
  $ f(x,y) $ στο σημείο $ P(x_{0}, y_{0}) $ και προς την κατεύθυνση του διανύσματος 
  $ \mathbf{u} $ δίνεται από τον τύπο
  \[
    \dv{f}{\mathbf{u}} \overset{\text{ή}}{=} D_{\mathbf{u}}(f)(P) = \grad f(P) 
    \cdot \mathbf{\widehat{u}} 
  \] 
  όπου $ \grad f (P) $ είναι η κλίση της συνάρτησης $f$ στο σημείο $P$ και 
  $ \mathbf{\widehat{u}} $ είναι το \textbf{μοναδιαίο} διάνυσμα $ \mathbf{u} $.
\end{dfn}
\begin{example}
  Έστω η συνάρτηση $ f(x,y) = 100-x^{2}-y^{2} $. 
  \begin{enumerate}[i)]
    \item Να βρείτε την παράγωγο της $f$ στο σημείο $ P_{0}(3,4) $ και προς την 
      κατεύθυνση του \textbf{διανύσματος} $ \mathbf{u} = 3 \mathbf{i}- 4 \mathbf{j} $.
    \item Να βρείτε την παράγωγο της $f$ στο σημείο $ P_{0}(3,4) $ και προς την 
      κατεύθυνση του \textbf{σημείου} $P(1,2)$.
    \item Προς ποια κατεύθυνση η $f$ αυξάνει με το \textbf{μεγαλύτερο ρυθμό}, 
      στο σημείο $ P_{0} $; Ποιος είναι ο μέγιστος ρυθμός μεταβολής της $f$;
    \item Προς ποια κατεύθυνση η παράγωγος κατά κατεύθυνση γίνεται \textbf{μηδέν}; στο 
      σημείο $ P_{0} $;
  \end{enumerate}
\end{example}
\begin{solution}
  \begin{enumerate}[i)]
    \item Έχουμε ότι $ D_{\mathbf{u}}f(P_{0}) = \grad f(P_{0}) \cdot 
      \mathbf{\widehat{u}} $
      \begin{align*}
        \grad f = (f_{x}, f_{y}) = (-2x, -2y) \Rightarrow \grad f(P_{0}) = (-6,-8) \\
        \mathbf{\widehat{u}} = \frac{\mathbf{u}}{\norm{\mathbf{u}}} =
        \frac{1}{\sqrt{3^{2}+(-4)^{2}}} (3,-4) = \left(\frac{3}{5} , - 
        \frac{4}{5}\right)
      \end{align*} 
      Επομένως
      \[
        D_{\mathbf{u}}f(P_{0}) = (-6,-8) \cdot \left(\frac{3}{5} , - \frac{4}{5}\right) 
        = - \frac{18}{5} + \frac{32}{5} = \frac{14}{5} > 0
      \] 
    \item Βρίσκουμε το διάνυσμα $ \vec{P_{0}P} = (1-3,2-4) = (-2,-2) $. Τότε το 
      μοναδιαίο $ \vec{\widehat{P_{0}P}} =
      \frac{\vec{P_{0}P}}{\norm{\vec{P_{0}P}}} = \frac{1}{\sqrt{8}} (-2,-2) = \left(-
      \frac{1}{\sqrt{2}} , - \frac{1}{\sqrt{2}}\right) $
      Άρα 
      \[
        D_{\mathbf{P_{0}P}}f(P_{0}) = \grad f(P_{0}) \cdot \vec{P_{0}P} = (-6,-8) 
        \cdot \left(- \frac{1}{\sqrt{2}} , - \frac{1}{\sqrt{2}}\right) = 
        \frac{6}{\sqrt{2}} + \frac{8}{\sqrt{2}} = \frac{14}{\sqrt{2}} > 0
      \] 
    \item Ζητάμε την κατεύθυνση $ \mathbf{u} $ προς την οποία η $f$ αυξάνει με το 
      μεγαλύτερο ρυθμό, δηλαδή η παράγωγος κατά κατεύθυνση γίνεται μέγιστη. Έχουμε:
      \[
        D_{\mathbf{u}}f(P_{0}) \max \Leftrightarrow \norm{\grad f (P_{0})}
        \norm{\mathbf{\widehat{u}}} \cos{\theta} \max \Leftrightarrow 
        \norm{\grad f(P_{0})} \cos{\theta} \max \Leftrightarrow \cos{\theta} \max
        \Leftrightarrow \theta = 0
      \]
      Δηλαδή $ \mathbf{u} $ παράλληλο με $ \grad f(P_{0}) $. Οπότε, η $f$ αυξάνει με 
      το μεγαλύτερο ρυθμό, προς την κατεύθυνση της κλίσης της. Μάλιστα, ισχύει:
      \[
        - \norm{\grad f(P_{0})} \leq D_{\mathbf{u}} f(P_{0}) \leq \norm{\grad f(P_{0})}
      \]
      για κάθε κατεύθυνση $ \mathbf{u} $ στο σημείο $ P_{0} $. 
      Δηλαδή, ο \textbf{μέγιστος ρυθμός} μεταβολής της $f$ στο σημείο $ P_{0} $ είναι $
      D_{\mathbf{u}}f(P_{0})_{\max} = \norm{\grad f(P_{0})} $ και παρατηρείται προς 
      την κατεύθυνση που δείχνει η κλίση, και ο \textbf{ελάχιστος ρυθμός} μεταβολής
      αντίστοιχα, είναι $ D_{\mathbf{u}}f(P_{0})_{\min} = -\norm{\grad f(P_{0})} $ 
      και παρατηρείται προς την αντίθετη κατεύθυνση από αυτή της κλίσης.
    \item Έχουμε ότι 
      \[ 
        D_{\mathbf{u}}f(P_{0}) = 0 \Leftrightarrow \norm{\grad f (P_{0})}
        \norm{\mathbf{\widehat{u}}} \cos{\theta} = 0 \Leftrightarrow 
        \cos{\theta} = 0 \Leftrightarrow \theta = \pi/2 
      \]
      Δηλαδή, προς κατεύθυνση \textbf{κάθετη} προς την κατεύθυνση που μας δείχνει η 
      κλίση της $ f $ στο σημείο $ P_{0} $, ο ρυθμός μεταβολής της συνάρτησης είναι 
      μηδέν, και επομένως η $f$ είναι σταθερή. Για να προσδιορίσουμε το διάνυσμα $
      \mathbf{u} $, έχουμε: Έστω $ \mathbf{u} = (u_{1}, u_{2}) $. 
      \[
        \mathbf{u} \perp \grad f(P_{0}) \Leftrightarrow \mathbf{u} \cdot 
        \grad f (P_{0}) = 0 \Leftrightarrow (u_1,u_{2}) \cdot (-6,-8) =
        0 \Leftrightarrow \inlineequation[eq:perp]{-6u_{1}-8u_{2}=0}
      \]
      Επίσης, επειδή $ \mathbf{u} $ είναι μοναδιαίο, έχουμε: 
      \[
        \norm{\mathbf{u}} =1 \Leftrightarrow \sqrt{u_{1}^{2}+u_{2}^{2}} = 1
        \Leftrightarrow \inlineequation[eq:unit]{u_{1}^{2}+u_{2}^{2}=1}
      \] 
      Από τις παραπάνω σχέσεις, για τα $ u_{1} $ και $ u_{2} $, εύκολα τα υπολογίζουμε,
      και βρίσκουμε ότι $ \mathbf{u} = \left(- \frac{4}{5}
      , \frac{3}{5}\right) $ ή $ \mathbf{u} = 
      \left( \frac{4}{5} , - \frac{3}{5}\right) $. 
  \end{enumerate}
\end{solution}


\section{Τύπος Taylor και Maclaurin}

\begin{example}
  Να υπολογιστεί το ανάπτυγμα της συνάρτησης $f(x,y)=x^3+y^3+xy^2$ γύρω από το 
  σημείο $ (1,2) $ (ή ισοδύναμα σε δυνάμεις του $(x-1)$ και $(y-2)$).
\end{example}
\begin{solution}
  Παρατηρούμε ότι δεν αναφέρεται στην εκφώνηση μέχρι τους όρους ποιας τάξης 
  χρειάζεται να βρω το ανάπτυγμα.  Γι' αυτό, μιας και η συνάρτηση είναι πολυωνυμική,
  βρίσκω μέχρι την τάξη όπου μηδενίζονται οι μερικές παράγωγοι: 

  (Δηλαδή στο συγκεκριμένο παράδειγμα μέχρι $3$ης τάξης, αφού οι παράγωγοι 
  $4$ης και ανώτερης τάξης, θα είναι όλες μηδέν)

  \vspace{\baselineskip}

  \twocolumnsides{\begin{myitemize}
      \item $f_x=3x^2+y^2\Rightarrow f_x(1,2)=7$
      \item $f_y=3y^2+2xy\Rightarrow f_y(1,2)=16$
      \item $f_{xx}=6x\Rightarrow f_{xx}(1,2)=6$
      \item $f_{xy}=2y\Rightarrow f_{xy}(1,2)=4$
      \item $f_{yy}=6y+2x\Rightarrow f_{yy}(1,2)=14$
      \end{myitemize}}{\begin{myitemize}
      \item $f_{xxx}=6$
      \item $f_{xxy}=f_{xyx}=0$
      \item $f_{xyy}=f_{yxy}=2$
      \item $f_{yyy}=6$ 
  \end{myitemize}}

  \vspace{\baselineskip}

  Με αντικατάσταση των μερικών παραγώγων στον τύπο Taylor, έχουμε:
  \begin{align*}
    f(x,y)&=13+\Bigl(7(x-1)+16(y-2)\Bigr)+ \\ 
          &\quad +\frac{1}{2!}\Bigl(6(x-1)^2 +2\cdot 4(x-1)(y-2)+14(y-2)^2\Bigr)+ \\
          &\quad +\frac{1}{3!}\Bigl(6(x-1)^3+3\cdot 0(x-1)^2(y-2)+3
          \cdot 2(x-1)(y-2)^2+6(y-2)^3\Bigr).
  \end{align*}
  Και μετά τις πράξεις, έχουμε: 
  \begin{align*}
    f(x,y)&=13+7(x-1)(y-2)+16(y-2)+3(x-1)^2+4(x-1)(y-2) \\
          &\quad +7(y-2)^2+(x-1)^3+(x-1)(y-2)^2+(y-2)^3.
  \end{align*}
  Δεν κάνουμε άλλες πράξεις.
  Έχουμε το ανάπτυγμα της $f(x,y)$ σε δυνάμεις του $(x-1)$ και $(y-2)$ όπως ζητήθηκε.
\end{solution}



