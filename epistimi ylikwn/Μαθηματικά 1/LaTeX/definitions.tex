%%%%%%% nesting newcommands $$$$$$$$$$$$$$$$$$$
\newcommand{\function}[1]{\newcommand{\nvec}[2]{#1(##1_1,\ldots, ##1_##2)}}

\newcommand{\linode}[2]{#1_n(x)#2^{(n)}+#1_{n-1}(x)#2^{(n-1)}+\cdots +#1_0(x)#2=g(x)} 
\newcommand{\vecoffun}[3]{#1_0(#2),\ldots,#1_#3(#2)}

\newcommand{\mysum}[1]{\sum_{n=#1}^{\infty}}


\renewcommand{\vector}[1]{(x_1,x_2,\ldots,x_{#1})}
\newcommand{\avector}[2]{(#1_1,#1_2,\ldots,#1_{#2})}
\newcommand{\aDEFvector}[2][a]{(#1_1,#1_2,\ldots,#1_{#2})}

\newcommand{\rt}[3]{\vb{r}(t)=#1\,\vb{i}+#2\,\vb{j}+#3\,\vb{k}}
\newcommand{\rtt}[2]{\vb{r}(t)=#1\,\vb{i}+#2\,\vb{j}}
\newcommand{\rs}[3]{\vb{r}(s)=#1\,\vb{i}+#2\,\vb{j}+#3\,\vb{k}}
\newcommand{\rss}[2]{\vb{r}(s)=#1\,\vb{i}+#2\,\vb{j}}
\newcommand{\vect}[4]{\vb{#1}=#2\,\vb{i}+#3\,\vb{j}+#4\,\vb{k}}
\newcommand{\vectt}[3]{\vb{#1}=#2\,\vb{i}+#3\,\vb{j}}
\newcommand{\vectsur}[3]{\vb{r}(u,v)=#1\,\vb{i}+#2\,\vb{j}+#3\,\vb{k}}

\DeclareMathOperator{\Arg}{Arg}

% --- Macro \xvec
\makeatletter
\newlength\xvec@height%
\newlength\xvec@depth%
\newlength\xvec@width%
\newcommand{\xvec}[2][]{%
  \ifmmode%
    \settoheight{\xvec@height}{$#2$}%
    \settodepth{\xvec@depth}{$#2$}%
    \settowidth{\xvec@width}{$#2$}%
  \else%
    \settoheight{\xvec@height}{#2}%
    \settodepth{\xvec@depth}{#2}%
    \settowidth{\xvec@width}{#2}%
  \fi%
  \def\xvec@arg{#1}%
  \def\xvec@dd{:}%
  \def\xvec@d{.}%
  \raisebox{.2ex}{\raisebox{\xvec@height}{\rlap{%
    \kern.05em%  (Because left edge of drawing is at .05em)
    \begin{tikzpicture}[scale=1]
    \pgfsetroundcap
    \draw (.05em,0)--(\xvec@width-.05em,0);
    \draw (\xvec@width-.05em,0)--(\xvec@width-.15em, .075em);
    \draw (\xvec@width-.05em,0)--(\xvec@width-.15em,-.075em);
    \ifx\xvec@arg\xvec@d%
      \fill(\xvec@width*.45,.5ex) circle (.5pt);%
    \else\ifx\xvec@arg\xvec@dd%
      \fill(\xvec@width*.30,.5ex) circle (.5pt);%
      \fill(\xvec@width*.65,.5ex) circle (.5pt);%
    \fi\fi%
    \end{tikzpicture}%
  }}}%
  #2%
}
\makeatother

% --- Override \vec with an invocation of \xvec.
\let\stdvec\vec
\renewcommand{\vec}[1]{\xvec[]{#1}}
% --- Define \dvec and \ddvec for dotted and double-dotted vectors.
\newcommand{\dvec}[1]{\xvec[.]{#1}}
\newcommand{\ddvec}[1]{\xvec[:]{#1}}


