\documentclass[a4paper,12pt]{article}
\usepackage{etex}
%%%%%%%%%%%%%%%%%%%%%%%%%%%%%%%%%%%%%%
% Babel language package
\usepackage[english,greek]{babel}
% Inputenc font encoding
\usepackage[utf8]{inputenc}
%%%%%%%%%%%%%%%%%%%%%%%%%%%%%%%%%%%%%%

%%%%% math packages %%%%%%%%%%%%%%%%%%
\usepackage{amsmath}
\usepackage{amssymb}
\usepackage{amsfonts}
\usepackage{amsthm}
\usepackage{proof}

\usepackage{physics}

%%%%%%% symbols packages %%%%%%%%%%%%%%
\usepackage{bm} %for use \bm instead \boldsymbol in math mode 
\usepackage{dsfont}
\usepackage{stmaryrd}
%%%%%%%%%%%%%%%%%%%%%%%%%%%%%%%%%%%%%%%


%%%%%% graphicx %%%%%%%%%%%%%%%%%%%%%%%
\usepackage{graphicx}
\usepackage{color}
%\usepackage{xypic}
\usepackage[all]{xy}
\usepackage{calc}
\usepackage{booktabs}
\usepackage{minibox}
%%%%%%%%%%%%%%%%%%%%%%%%%%%%%%%%%%%%%%%

\usepackage{enumerate}

\usepackage{fancyhdr}
%%%%% header and footer rule %%%%%%%%%
\setlength{\headheight}{14pt}
\renewcommand{\headrulewidth}{0pt}
\renewcommand{\footrulewidth}{0pt}
\fancypagestyle{plain}{\fancyhf{}
\fancyhead{}
\lfoot{}
\rfoot{\small \thepage}}
\fancypagestyle{vangelis}{\fancyhf{}
\rhead{\small \leftmark}
\lhead{\small }
\lfoot{}
\rfoot{\small \thepage}}
%%%%%%%%%%%%%%%%%%%%%%%%%%%%%%%%%%%%%%%

\usepackage{hyperref}
\usepackage{url}
%%%%%%% hyperref settings %%%%%%%%%%%%
\hypersetup{pdfpagemode=UseOutlines,hidelinks,
bookmarksopen=true,
pdfdisplaydoctitle=true,
pdfstartview=Fit,
unicode=true,
pdfpagelayout=OneColumn,
}
%%%%%%%%%%%%%%%%%%%%%%%%%%%%%%%%%%%%%%

\usepackage[space]{grffile}

\usepackage{geometry}
\geometry{left=25.63mm,right=25.63mm,top=36.25mm,bottom=36.25mm,footskip=24.16mm,headsep=24.16mm}

%\usepackage[explicit]{titlesec}
%%%%%% titlesec settings %%%%%%%%%%%%%
%\titleformat{\chapter}[block]{\LARGE\sc\bfseries}{\thechapter.}{1ex}{#1}
%\titlespacing*{\chapter}{0cm}{0cm}{36pt}[0ex]
%\titleformat{\section}[block]{\Large\bfseries}{\thesection.}{1ex}{#1}
%\titlespacing*{\section}{0cm}{34.56pt}{17.28pt}[0ex]
%\titleformat{\subsection}[block]{\large\bfseries{\thesubsection.}{1ex}{#1}
%\titlespacing*{\subsection}{0pt}{28.80pt}{14.40pt}[0ex]
%%%%%%%%%%%%%%%%%%%%%%%%%%%%%%%%%%%%%%

%%%%%%%%% My Theorems %%%%%%%%%%%%%%%%%%
\newtheorem{thm}{Θεώρημα}[section]
\newtheorem{cor}[thm]{Πόρισμα}
\newtheorem{lem}[thm]{λήμμα}
\theoremstyle{definition}
\newtheorem{dfn}{Ορισμός}[section]
\newtheorem{dfns}[dfn]{Ορισμοί}
\theoremstyle{remark}
\newtheorem{remark}{Παρατήρηση}[section]
\newtheorem{remarks}[remark]{Παρατηρήσεις}
%%%%%%%%%%%%%%%%%%%%%%%%%%%%%%%%%%%%%%%




\newcommand{\vect}[2]{(#1_1,\ldots, #1_#2)}
%%%%%%% nesting newcommands $$$$$$$$$$$$$$$$$$$
\newcommand{\function}[1]{\newcommand{\nvec}[2]{#1(##1_1,\ldots, ##1_##2)}}

\newcommand{\linode}[2]{#1_n(x)#2^{(n)}+#1_{n-1}(x)#2^{(n-1)}+\cdots +#1_0(x)#2=g(x)}

\newcommand{\vecoffun}[3]{#1_0(#2),\ldots ,#1_#3(#2)}

\newcommand{\mysum}[1]{\sum_{n=#1}^{\infty}


\geometry{top=1.9cm}

\pagestyle{askhseis}
\everymath{\displaystyle}

\begin{document}

\begin{center}
  \minibox[c]{\large\bf \textcolor{Col1}{Ασκήσεις Τριπλό Ολοκλήρωμα} \\ 
\textcolor{Col1}{(Αλλαγη μεταβλητων)}}
\end{center}

% \vspace{\baselineskip}

\section*{Κυλινδρικές Συντεταγμένες}

\begin{enumerate}
  \item Να υπολογιστουν τα παρακατω τριπλα ολοκληρωματα με τη χρήση 
    κυλινδρικών συντεταγμένων.

    \begin{enumerate}[i)]
      %spandagos p.262
      \item $ I=\iiint_{\Omega} x\,dV $, \quad $ \Omega = 
        \{(x,y,z)\in \mathbb{R}^{3} \mid x^{2}+y^{2} \leq a^{2},\; x 
        \geq 0,\; y \geq 0,\; 0 \leq z \leq h \}  $ 
        \hfill Απ: $ \frac{ha^{3}}{3} $  
        %spandagos p.293
      \item $ I=\iiint_{\Omega} z(x^{2}+y^{2})\,dV $, \quad $ \Omega = \{(x,y,z)\in
        \mathbb{R}^{3} \mid x^{2}+y^{2} \leq 1,\; 2 \leq z \leq 3 \} $ 
        \hfill Απ: $ \frac{5 \pi }{4} $ 
        %spandagos p.295 (παραλλαγή για b=2)
      \item $ I=\iiint_{\Omega}xyz\,dV $, \quad $ \Omega = \{(x,y,z)\in \mathbb{R}^{3} 
        \mid x^{2}+y^{2} \leq a^{2}, x \geq 0,\; y \geq 0,\; 0 \leq z \leq 2 \} $ 
        \hfill Απ: $ \frac{a^{4}}{4} $ 
        %spandagos p.297 (παραλλαγή για x^{2}+y^{2}<4)
      \item $ I=\iiint_{\Omega} \sqrt{ x^{2}+y^{2} }\, dV $, \quad $ \Omega = 
        \{(x,y,z)\in \mathbb{R}^{3} \mid x^{2}+y^{2} \leq 4,\; y+z \leq 4,\; z \geq 0\}$ 
        \hfill Απ: $ \frac{64 \pi}{3} $ 
    \end{enumerate}
\end{enumerate}


\section*{Σφαιρικές Συντεταγμένες}


\begin{enumerate}
  \item Να υπολογιστούν τα παρακάτω τριπλά ολοκληρώματα με τη χρήση 
    σφαιρικών συντεταγμένων.

    \begin{enumerate}[i)]
        %spandagos p.290
      \item $ I = \iiint_{\Omega} y^{2} \,dV $, \quad $ \Omega = \{(x,y,z)\in 
        \mathbb{R}^{3} \mid x^{2}+y^{2}+z^{2} \leq a^{2}, y \geq 0 \} $ 
        \hfill Απ: $ \frac{2 \pi a^{5}}{15} $ 
        %spandagos p.267
      \item $ I= \iiint_{\Omega} (x^{2}+y^{2}+z^{2})\,dV $, \quad $ \Omega = 
        \{(x,y,z)\in \mathbb{R}^{3} \mid x^{2}+y^{2}+z^{2} \leq 1 \}$ 
        \hfill Απ: $ \frac{4 \pi}{5} $ 
        %spandagos p.280
      \item $ I=\iiint_{\Omega} x\,dV $, \quad $ \Omega = \{(x,y,z)\in \mathbb{R}^{3} 
        \mid x^{2}+y^{2}+z^{2} \leq a^{2},\; x \geq 0, y \geq 0,\; z \geq 0\} $ 
        \hfill Απ: $ \frac{\pi a^{4}}{16} $ 
        %spandagos p.267
      \item $ I = \iiint_{\Omega} \frac{1}{(x^{2}+y^{2}+z^{2})^{3/2}} \,dV$, \quad 
        $\Omega = \{ (x,y,z) \in \mathbb{R}^{3} \mid a \leq x^{2}+y^{2}+z^{2} \leq b \}$,
        με $ a,b > 0 $. \hfill Απ: $ 4 \pi \ln{\frac{b}{a}} $ 
    \end{enumerate}
\end{enumerate}


    % \item $ I=\iiint_{\Omega} (x^{2}+y^{2}+z^{2}) \,dV $, \quad $ \Omega $ 
    %   περικλείεται από το παραβολοειδές $ z = x^{2}+y^{2} $, τον κύλινδρο 
    %   $ x^{2}+y^{2}=1 $ και τα επίπεδα συντεταγμένων.  
    %   \hfill Απ: $ \frac{\pi}{8} $ 

\section*{Εφαρμογές Τριπλού Ολοκληρώματος}


\begin{enumerate}
  \item Να υπολογιστεί ο \textbf{όγκος} του ορθού κυλίνδρου $ x^{2}+y^{2} = a^{2},
  \; 0 \leq z \leq h \}  $.  \hfill Απ:  $ \pi a^{2} h $

\item Να υπολογιστεί ο \textbf{όγκος} της σφαίρας $ x^{2}+y^{2}+z^{2} = R^{2} $
  \hfill Απ: $ \frac{4}{3} \pi R^{3} $ 

\item Να υπολογιστεί ο \textbf{όγκος} του ελλειψοειδούς $ \frac{x^{2}}{a^{2}} +
  \frac{y^{2}}{b^{2}} + \frac{z^{2}}{c^{2}} =1 $ 
  \hfill Απ: $ \frac{4}{3} \pi abc $ 

\item Να υπολογιστεί ο \textbf{όγκος} του στερεού που περικλείεται από το 
  ελλειπτικό παραβολοειδές $ z=x^{2}+4y^{2} $ και το επίπεδο $ z=1 $. 
  \hfill Απ: $ \pi /4 $ 

  %spandagos p.268
\item Να υπολογιστεί ο \textbf{όγκος} του στερεού, που περικλείεται από 
  το παραβολοειδές $ z = x^{2}+y^{2} $ τον κυλίνδρο $ x^{2}+y^{2}=a^{2} $ και 
  το επιπέδο $ z = 0$.  
  \hfill Απ: $ a^{4} {\pi}/{2} $  

\item Να υπολογιστεί ο \textbf{όγκος} του στερεού που περικλείεται από τα παραβολοειδή 
  $ z=x^{2}+y^{2} $ και $ z = 4-y^{2} $ 
  \hfill Απ: $ 4 \pi $ 

  %%spandagos p.282p
%\item Να υπολογιστεί ο \textbf{όγκος} του στερεού που περικλείεται από τον κύλινδρο 
  %$ x^{2}+z^{2}=a^{2} $ και τα επίπεδα $ y=0 $, $ x+y=a $ με $ a>0 $.
  %\hfill Απ: $ \pi a^{3} $ 

\item Να υπολογιστεί ο \textbf{όγκος} του στερεού που περικλείεται από τις επιφάνειες 
  $ x^{2}+y^{2}=z $, $ x^{2}+y^{2}=2x $ και το επίπεδο $ z=0 $. 
  \hfill Απ: $ {6 \pi}/{4} $ 

  %spandagos p.268
\item Να υπολογιστει ο \textbf{ογκος} του στερεου που βρισκεται εντος της σφαιρας 
  $x^2+y^2+z^2=az$ και εντός του κωνου $z^2=x^2+y^2$ με $ z>0 $. 
  \hfill Απ: $\pi{a^3}/{8}$

\item Να υπολογιστεί ο \textbf{όγκος} του στερεού που περικλείεται από τις επιφάνειες 
  $ z^{2}\!=x^{2}+y^{2} $ και $ x^{2}+y^{2}\!+(z-2)^{2}=4 $.  

  \hfill Απ: $ {8 \pi}/{3} $ 
\end{enumerate}

\enlargethispage{2\baselineskip}

\section*{Υποδείξεις}
 
\begin{myitemize}
  \item Για τον υπολογισμό του όγκου του ελλειψοειδούς, να κάνετε αλλαγή μεταβλητών,  
    θέτωντας $ x/a=u, y/b=v $ και $ z/c=w $ και στη συνέχεια να εφαρμόσετε σφαιρικές 
    συντεταγμένες.
\end{myitemize}


\end{document}


