\input{preamble_ask.tex}
\input{definitions_ask.tex}


\pagestyle{askhseis}
\everymath{\displaystyle}

\begin{document}

\begin{center}
  \minibox{\large \bfseries \textcolor{Col1}{Δυναμοσειρές -- Σειρές Taylor και Maclaurin}}
\end{center}

\vspace{\baselineskip}


\section*{Διάστηματα Σύγκλισης}

\begin{enumerate}
  \item Να βρεθεί το \textbf{πλήρες} διάστημα σύγκλισης των παρακάτω δυναμοσειρών.

    \twocolumnside{
      \begin{enumerate}[i)]
        %logou
        \item $\sum_{n=1}^{\infty} \frac{x^{n}}{n}$ \hfill Απ: $-1\leq x <1$
          %logou
        \item $\sum_{n=0}^{\infty} n!x^{n}$ \hfill Απ: $x=0$
        \item $\sum_{n=1}^{\infty} (-1)^{n}\frac{x^{n}}{n}$ \hfill Απ: $-1<x\leq 1$
        \item $\sum_{n=1}^{\infty} \frac{x^{n}}{2^{n}n^{2}}$ \hfill Απ: $-2\leq x \leq 2$
        \item $\sum_{n=0}^{\infty} \frac{x^{n}}{n!}$ \hfill Απ: $x\in \mathbb{R}$
      \end{enumerate}
      }{
      \begin{enumerate}
        \item $\sum_{n=1}^{\infty} \frac{x^{n}}{n^{n+1}}$ \hfill Απ: $x\in \mathbb{R}$
        \item $\sum_{n=1}^{\infty} \frac{(x+1)^{n}}{\sqrt{n}}$ \hfill Απ: $-2\leq x <0$
        \item $\sum_{n=0}^{\infty} \frac{n^{2}+1}{(n+1)!}x^{n}$ 
          \hfill Απ: $x\in\mathbb{R}$
        \item $\sum_{n=1}^{\infty} \frac{(2x-1)^{n}}{nx^{n}}$ 
          \hfill Απ: $\frac{1}{3}\leq x<1$
        \item $\sum_{n=0}^{\infty} (-1)^{n}\frac{x^{2n+1}}{2n+1}$ 
          \hfill Απ: $-1\leq x\leq 1$
      \end{enumerate}
    }
\end{enumerate}


\section*{Taylor -- Maclaurin}

\begin{enumerate}
  \item Να υπολογιστεί το ανάπτυγμα Maclaurin, των παρακάτω συναρτήσεων.
    \begin{enumerate}[i)]
      \item $ y= \mathrm{e}^{-2x} $ 
        \hfill Απ: $\mathrm{e}^{-2x} = 1-2x+ 2x^{2} - \frac{4}{3} x^{3} +
        \frac{2}{3} x^{4} + \cdots $ 
      \item $ y= \ln{(x+2)} $ 
        \hfill Απ: $ \ln{(x+2)} = \ln{2} + \frac{1}{2} x - \frac{1}{8} x^{2} +
        \frac{1}{24} x^{3} + \cdots $ 
      \item $ y= \frac{1}{x-1} $ \hfill Απ:$\frac{1}{x-1} = -1 -x -x^{2} - x^{3} +
        \cdots $ 
    \end{enumerate}

  \item Να υπολογιστεί το ανάπτυγμα Taylor, γύρω από το σημείο $ x_{0}=1 $, 
    των παρακάτω συναρτήσεων.
    \begin{enumerate}[i)]
      \item $ y= \ln{x} $ 
        \hfill Απ: $ \ln{x} = (x-1) - \frac{1}{2} (x-1)^{2} + \frac{1}{3} (x-1)^{3} - 
        \frac{1}{4} (x-1)^4 + \cdots $ 
      \item $ y= \frac{1}{x+1} $ \hfill Απ: $ \frac{1}{x+1} = \frac{1}{2} -
        \frac{1}{4} (x-1) + \frac{1}{3} (x-1)^{2} - \frac{1}{16} (x-1)^{3} $ 
    \end{enumerate}


  \item Να υπολογιστεί το $ \lim_{x \to 0} \frac{\mathrm{e}^{x} -
    \mathrm{e}^{-x}}{\sin{x}} $ \hfill Απ: 2 

  \item Να υπολογιστεί το $ \lim_{x \to 0} \frac{x - \arctan(x)}{x^{3}} $ \hfill Απ: 1/3 

  \item Να βρείτε μια \textbf{πολυωνυμική προσέγγιση} 3ης τάξης 
    της συνάρτησης που ορίζεται πεπλεγμένα από την εξίσωση 
    $ x^{2} - xy + y^{2} = 3$ στο σημείο $ (1,-1) $.

    \hfill Απ: $f(x) \cong -1 + (x-1) + \frac{(x-1)^{2}}{3} +
    \frac{(x-1){3}}{9}$
\end{enumerate}

\end{document}
