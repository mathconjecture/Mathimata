\input{preamble_ask.tex}
\input{definitions_ask.tex}
\input{myboxes.tex}

\begin{document}


\section*{Προβλήματα Συνοριακών Τιμών}

Μία διαφορική εξίσωση, η οποία συνοδεύεται από \textcolor{Col1}{συνοριακές συνθήκες}, 
δηλαδή σχέσεις που αναφέρονται στα συνοριακά σημεία $ x=a $ και $ x=b $ κάποιου 
διαστήματος $x \in [a,b]$, λέγεται \textbf{Πρόβλημα Συνοριακών Τιμών}. 
Εμείς θα ασχοληθούμε με \textbf{γραμμικές} εξισώσεις της μορφής
\begin{equation}\label{eq:lin1}
  a(x)y'' + b(x) y'+ c(x)=d(x), \quad \text{όπου } y=y(x), \; \text{με } x \in [a,b] 
\end{equation}
οι οποίες συνοδεύονται από συνοριακές συνθήκες, που εν γένει έχουν τη μορφή
\begin{gather}\label{eq:bound1}
  \begin{rcases}
    a_{1} y(a) + a_{2} y'(a) = d_{1} \\
    b_{1} y(b) + b_{2} y'(b) = d_{2}
  \end{rcases}
\end{gather}
με τους συντελεστές σε κάθε συνθήκη, να μην είναι ταυτόχρονα μηδέν.
Αν $ d(x)=0 $, τότε η εξίσωση~\eqref{eq:lin1} λέγεται \textcolor{Col1}{ομογενής} και αν 
$ d_{1}= d_{2}= 0 $, τότε οι συνοριακές συνθήκες~\eqref{eq:bound1} λέγονται
\textcolor{Col1}{ομογενείς}. Μία \textbf{ομογενής} εξίσωση, η οποία συνοδεύεται από 
\textbf{ομογενείς} συνοριακές συνθήκες, λεγεται \textcolor{Col1}{ομογενές πρόβλημα 
συνοριακών τιμών}.

\begin{rem}
 Οι συνθήκες της μορφής~\eqref{eq:bound1} λέγονται \textbf{αμιγείς}, γιατί κάθε συνθήκη 
 περιέχει μόνο τιμές των συναρτήσεων $ y $ και $ y' $ στο \textbf{ίδιο} άκρο του 
 διαστήματος, δηλαδή το $ x=a $ ή το $ x=b $.
\end{rem}


\section*{Προβλήματα Ιδιοτιμών}

Προβλήματα της μορφής $ \mathcal{L} y= \lambda y $, όπου $ y=y(x) $ κάποια συνάρτηση 
και $ \mathcal{L} $ ένας \textbf{γραμμικός} διαφορικός τελεστής, ονομάζονται 
\textcolor{Col1}{προβλήματα ιδιοτιμών}. Οι τιμές του $ \lambda $ για τις οποίες κάποια 
συνάρτηση $ y $ ικανοποιεί την εξίσωση $ \mathcal{L} y= \lambda y $ ονομάζονται 
\textcolor{Col1}{ιδιοτιμές} και οι αντίστοιχες συναρτήσεις, ονομάζονται 
\textcolor{Col1}{ιδιοσυναρτήσεις}. Το σύνολο των ιδιοτιμών, ονομάζεται
\textcolor{Col1}{φάσμα}. Για παράδειγμα, αν
\[
  \mathcal{L} = a(x) \dv[2]{}{x} + b(x) \dv{}{x} + c(x ) 
\] 
τότε η εξίσωση $ Ly= \lambda y $, όπου $ y=y(x) $ γίνεται:
\[
  a(x) \dv[2]{y}{x} + b(x) \dv{y}{x} + c(x) y = \lambda (x) y 
\]
η οποία γράφεται
\[
  a(x) \dv[2]{y}{x} + b(x) \dv{y}{x} + (c(x)- \lambda )y = 0
\] 
ή ισοδύναμα
\begin{equation}\label{eq:lin2}
  a(x) y''(x) + b(x) y'(x) + (c(x)- \lambda )y = 0 
\end{equation} 

Σκοπός μας είναι η μελέτη του προβλήματος ιδιοτιμών~\eqref{eq:lin2} όταν συνοδεύεται από 
τα παρακάτω σετ συνοριακών συνθηκών στο διάστημα $ [0,L] $.
\begin{myitemize}
  \item Αμιγείς συνθήκες \textbf{Robin} 
    $ 
    \begin{cases}
      y'(0)=h\, y(0) \\
      y'(L)=H\, y(L) 
    \end{cases}
    $ 
    όπου $ h,H $ σταθερές που μπορεί να πάρουν και τις τιμές 0 ή $ \infty $.
    \begin{myitemize}
      \item Αν $ h,H \to \infty $ τότε προκύπτουν οι συνθήκες \textbf{Dirichlet}  
        $ 
        \begin{cases}
          y(0)=0 \\
          y(L)=0
        \end{cases}
        $
        προκειμένου η $ y' $ να είναι πεπερασμένη.
      \item Αν $ h=H=0 $ τότε προκύπτουν οι συνθήκες \textbf{Newman} 
        $ 
        \begin{cases}
          y'(0)=0 \\
          y'(L)=0
        \end{cases}
        $
    \end{myitemize}
  \item Περιοδικές συνθήκες (\textbf{μικτές})
    $ \begin{cases}
      y(0)=y(L) \\
      y'(0)=y'(L)
    \end{cases}$
\end{myitemize}

\begin{prop}
  Το πρόβλημα ιδιοτιμών~\eqref{eq:lin2} με ομογενείς συνοριακές συνθήκες (αμιγείς ή 
  περιοδικές) έχει μη τετριμμένη λύση $ y \neq 0 $ μόνο όταν η παράμετρος $ \lambda $ 
  παίρνει μια διακριτή ακολουθία τιμών $ \lambda_{1}, \lambda_{2}, \ldots,
  \lambda_{n}, \ldots $ («κβάντωση» ιδιοτιμών).
\end{prop}



\section*{Προβλήματα Συνοριακών Τιμών Sturm-Liouville}

Η εξίσωση~\eqref{eq:lin2} μπορεί πάντοτε να γραφεί στη μορφή 
\begin{equation}
  p(x)y'' + p'(x)y' + (\lambda w(x)- v(x))y = 0
\end{equation}
Πράγματι, πολλαπλασιάζουμε την εξίσωση~\eqref{eq:lin2} με τον 
\textbf{ολοκληρωτικό παράγοντα} 
\[
  \mu (x) = \frac{1}{a(x)} \mathrm{e}^{\int \frac{b(x)}{a(x)}} \,{dx}
\] 
και η εξίσωση γίνεται
\[
  \mu(x) a(x)(x) y'' + \mu(x) b(x)(x) y' + (\mu(x) c(x)(x) - \mu(x) \lambda)y = 0
\]
Θέτουμε $ \mu(x) a(x) = p(x) $, και παρατηρούμε ότι $ \mu(x) b(x) = p'(x) $, 
οπότε αν θέσουμε επίσης $ \mu (x) c(x) = - u(x) $ (\textbf{δυναμικό}) και 
$ - \mu (x) = w(x) $ (\textbf{βάρος}), η εξίσωση γίνεται
\[
  p(x) y'' + p'(x)y' + (\lambda w(x) - u(x))y = 0 \quad \text{(\textbf{Μορφή Liouville})}
\] 
ή ισοδύναμα
\[
  [p(x)y']' +  (\lambda w(x) - u(x))y = 0
\] 


\begin{prop}
  Αν $ y_{1}, y 2 $ είναι δύο ιδιοσυναρτήσεις του προβλήματος Sturm-Liouville 
  \[
    py'' + p'y' + (\lambda w-u) y = 0 
  \]
  με αντίστοιχες ιδιοτιμές $ \lambda_{1}, \lambda_{2} $ τότε ικανοποιούν την 
  ταυτότητα του Green 
  \[
    (pW)' + (\lambda_{2} - \lambda_{1}) w y_{1} y_{2} = 0
  \] 
  όπου $ W = 
  \begin{vmatrix*}[r]
    y_{1} & y_{2} \\
    y_{1}' & y_{2}'
  \end{vmatrix*} = y_{1} y_{2}' - y_{1}' y_{2} $, η ορίζουσα Wronski των συναρτήσεων 
  $ y_{1}, y_{2} $.
\end{prop}


\begin{thm}
  Ένα πρόβλημα ιδιοτιμών Sturm-Liouville $ (py')' + (\lambda w-u)y=0 $ με $ 0 \leq x \leq
  L$ όπου $ p(x), w(x), u(x) \in \mathbb{R} $ με $w(x)>0$, έχει πραγματικές ιδιοτιμές 
  αν συνοδεύεται από συνοριακές συνθήκες της μορφής 
  \begin{myitemize}
    \item Αμιγείς, δηλ $ y'(0) = h y(0) $ και $ y'(L) = Hy(L) $, συμπεριλαμβανομένων 
      των οριακών περιπτώσεων $ h,H \to \infty $ ή $ 0 $, που αντιστοιχούν σε μηδενισμό 
      της συναϱτησης ή της παραγώγου της στα άκρα του διαστήματος.
    \item Οι ομογενείς συνοριακές συνθήκες του προβλήματος είναι περιοδικές 
      \[
        y(0)=y(L) \quad \text{και} \quad y'(0) = y'(L)
      \] 
      και επιπλέον $ p(0)=p(L) $ όπου $ p(x) $ είναι ο συντελεστής του $ y'' $ 
      στη μορφή Liouville.
  \end{myitemize}
\end{thm}

\begin{rem}
  Οι συνοριακές συνθήκες που διασφαλίζουν ότι οι ιδιοτιμές είναι πραγματικές ονομάζονται 
  αυτοσυζυγείς συνοριακές συνθήκες. Επομένως
  \begin{myitemize}
    \item Οι \textbf{αμιγείς} συνθήκες είναι πάντοτε αυτοσυζυγείς
    \item Οι \textbf{περιοδικές} συνθήκες είναι αυτοσυζυγείς αν $ p(0)=p(L) $
  \end{myitemize}
\end{rem}

\begin{prop}
  Τα δύο βασικά θεωρήματα του προβλήματος ιδιοτιμών Sturm-Liouville (πραγματικότητα
  ιδιοτιμών και ορθογωνιότητα ιδιοσυναρτησέων) ισχύουν και στην περίπτωση όπου το 
  ένα ή και τα δύο άκρα του διαστήματος $ [a,b] $ στο οποίο ορίζεται το πρόβλημα, 
  είναι ιδιόμορφα, π.χ. $ p(a)=0 $ ή και $ p(b)=0 $ αν αντικαταστήσουμε την εκεί 
  συνοριακή συνθήκη με την απαίτηση να είναι η συνάρτηση πεπερασμένη στο ιδιόμορφο άκρο.
\end{prop}

\section*{Ορθογωνιότητα Συναρτήσεων}

Έστω $ f(x), g(x) $ δύο συναρτήσεις μιας πραγματικής μεταβλητής $ x $, με διάστημα 
ορισμού $ [a,b] $. Θεωρούμε ότι είναι στοιχεία ενός διανυσματικού χώρου συναρτήσεων, και 
επιτρέπουμε να έχουν και μιγαδικές τιμές. Δύο ορισμοί εσωτερικού γινομένου στο 
διανυσματικό χώρο των συναρτήσεων, είναι:
\begin{gather}
  \langle f, g\rangle = \int _{a}^{b} f^{*}(x)g(x) \,{dx} \label{eq:inner1} \\
  \langle f, g\rangle = \int _{a}^{b} w(x) f^{*}(x)g(x) \,{dx} \label{eq:inner2} 
\end{gather}
όπου στο δεύτερο ορισμό η συνάρτηση βάρους $ w(x) \in \mathbb{R} $ και επιπλέον 
$ w(x) >0, \; \forall x \in [a,b] $. 

\begin{rem}
  Σε ένα χώρο συναρτήσεων, μπορούμε μέσω του εσωτερικού γινόμενου να ορίσουμε 
  το «μήκος» μιας συνάρτησης $f$, ως 
  \[
    \norm{f} = \sqrt{\langle f,f \rangle}  
  \] 
  όπου χρησιμοποιώντας τον ορισμό~\eqref{eq:inner1} παίρνουμε
  \[
    \norm{f} = \sqrt{\int _{a}^{b}f^{*}(x)f(x) \,{dx}} = \sqrt{\int _{a}^{b}
    \abs{f(x)}^{2} \,{dx} }  
  \]
  ενώ χρησιμοποιώντας τον ορισμό~\eqref{eq:inner2} παίρνουμε
  \[
    \norm{f} = \sqrt{\int _{a}^{b}w(x)f^{*}(x)f(x) \,{dx}} = \sqrt{\int _{a}^{b}w(x)
    \abs{f(x)}^{2} \,{dx}} 
  \] 
  Η απαίτηση $ w(x)>0 $ εξασφαλίζει ότι 
  \[
    \int _{a}^{b}w(x) \abs{f(x)} ^{2} \,{dx} \geq 0
  \] 
\end{rem}


\end{document}
