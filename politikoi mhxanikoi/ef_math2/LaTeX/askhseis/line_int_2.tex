\documentclass[a4paper,12pt]{article}
\usepackage{etex}
%%%%%%%%%%%%%%%%%%%%%%%%%%%%%%%%%%%%%%
% Babel language package
\usepackage[english,greek]{babel}
% Inputenc font encoding
\usepackage[utf8]{inputenc}
%%%%%%%%%%%%%%%%%%%%%%%%%%%%%%%%%%%%%%

%%%%% math packages %%%%%%%%%%%%%%%%%%
\usepackage{amsmath}
\usepackage{amssymb}
\usepackage{amsfonts}
\usepackage{amsthm}
\usepackage{proof}

\usepackage{physics}

%%%%%%% symbols packages %%%%%%%%%%%%%%
\usepackage{bm} %for use \bm instead \boldsymbol in math mode 
\usepackage{dsfont}
\usepackage{stmaryrd}
%%%%%%%%%%%%%%%%%%%%%%%%%%%%%%%%%%%%%%%


%%%%%% graphicx %%%%%%%%%%%%%%%%%%%%%%%
\usepackage{graphicx}
\usepackage{color}
%\usepackage{xypic}
\usepackage[all]{xy}
\usepackage{calc}
\usepackage{booktabs}
\usepackage{minibox}
%%%%%%%%%%%%%%%%%%%%%%%%%%%%%%%%%%%%%%%

\usepackage{enumerate}

\usepackage{fancyhdr}
%%%%% header and footer rule %%%%%%%%%
\setlength{\headheight}{14pt}
\renewcommand{\headrulewidth}{0pt}
\renewcommand{\footrulewidth}{0pt}
\fancypagestyle{plain}{\fancyhf{}
\fancyhead{}
\lfoot{}
\rfoot{\small \thepage}}
\fancypagestyle{vangelis}{\fancyhf{}
\rhead{\small \leftmark}
\lhead{\small }
\lfoot{}
\rfoot{\small \thepage}}
%%%%%%%%%%%%%%%%%%%%%%%%%%%%%%%%%%%%%%%

\usepackage{hyperref}
\usepackage{url}
%%%%%%% hyperref settings %%%%%%%%%%%%
\hypersetup{pdfpagemode=UseOutlines,hidelinks,
bookmarksopen=true,
pdfdisplaydoctitle=true,
pdfstartview=Fit,
unicode=true,
pdfpagelayout=OneColumn,
}
%%%%%%%%%%%%%%%%%%%%%%%%%%%%%%%%%%%%%%

\usepackage[space]{grffile}

\usepackage{geometry}
\geometry{left=25.63mm,right=25.63mm,top=36.25mm,bottom=36.25mm,footskip=24.16mm,headsep=24.16mm}

%\usepackage[explicit]{titlesec}
%%%%%% titlesec settings %%%%%%%%%%%%%
%\titleformat{\chapter}[block]{\LARGE\sc\bfseries}{\thechapter.}{1ex}{#1}
%\titlespacing*{\chapter}{0cm}{0cm}{36pt}[0ex]
%\titleformat{\section}[block]{\Large\bfseries}{\thesection.}{1ex}{#1}
%\titlespacing*{\section}{0cm}{34.56pt}{17.28pt}[0ex]
%\titleformat{\subsection}[block]{\large\bfseries{\thesubsection.}{1ex}{#1}
%\titlespacing*{\subsection}{0pt}{28.80pt}{14.40pt}[0ex]
%%%%%%%%%%%%%%%%%%%%%%%%%%%%%%%%%%%%%%

%%%%%%%%% My Theorems %%%%%%%%%%%%%%%%%%
\newtheorem{thm}{Θεώρημα}[section]
\newtheorem{cor}[thm]{Πόρισμα}
\newtheorem{lem}[thm]{λήμμα}
\theoremstyle{definition}
\newtheorem{dfn}{Ορισμός}[section]
\newtheorem{dfns}[dfn]{Ορισμοί}
\theoremstyle{remark}
\newtheorem{remark}{Παρατήρηση}[section]
\newtheorem{remarks}[remark]{Παρατηρήσεις}
%%%%%%%%%%%%%%%%%%%%%%%%%%%%%%%%%%%%%%%




\newcommand{\vect}[2]{(#1_1,\ldots, #1_#2)}
%%%%%%% nesting newcommands $$$$$$$$$$$$$$$$$$$
\newcommand{\function}[1]{\newcommand{\nvec}[2]{#1(##1_1,\ldots, ##1_##2)}}

\newcommand{\linode}[2]{#1_n(x)#2^{(n)}+#1_{n-1}(x)#2^{(n-1)}+\cdots +#1_0(x)#2=g(x)}

\newcommand{\vecoffun}[3]{#1_0(#2),\ldots ,#1_#3(#2)}

\newcommand{\mysum}[1]{\sum_{n=#1}^{\infty}


\thispagestyle{empty}
\everymath{\displaystyle}

\begin{document}

\begin{center}
\fbox{\large\bf Ασκήσεις Επικαμπυλιο Ολοκλήρωμα (IΙου Ειδους)}
\end{center}

\vspace{\baselineskip}

\begin{enumerate} 

\item Να υπολογιστουν τα παρακατω επικαμπυλια ολοκληρωματα:

\begin{enumerate}[i)]

\item $\int\limits_C \vec{F}\,d\vec{r}$, οπου $\vec{F}(x,y,z)=(3x^2-6yz)\vec{i}+(2y+3xz)\vec{j}+(1-4xyz^2)\vec{k}$ και $C$ η καμπυλη με παραμετρικες εξισωσεις $x(t)=t, y(t)=t^2, z(t)=t^3$, με $t\in [0,1]$

\hfill Απ: $2$

\end{enumerate}

\item Να υπολογιστει το επικαμπυλιο ολοκληρωμα $I=\int\limits_C xdy-ydx$ απο το σημειο $A(0,0)$ εως το σημειο $B(1,1)$ κατα μηκος των παρακατω καμπυλων. (Τι παρατηρειτε?)

\begin{enumerate}[i)]
\item της ευθειας $y=x$
\item των ευθειων $y=0$ και $x=1$
\item των ευθειων $x=0$ και $y=1$
\item της παραβολης $y=x^2$

\hfill Απ: \begin{tabular}{l}
$I=0$ \\
$I=1$ \\
$I=-1$ \\
$I=\sfrac{1}{3}$
\end{tabular}
\end{enumerate}

\item Να υπολογιστει το επικαμπυλιο ολοκληρωμα $\int\limits_C (y+z)dx$, οπου $C$ ειναι η καμπυλη με διανυσματικη παραμετρικη εξισωση $\vec{r}(t)=\cos t\vec{i}+\sin t\vec{k}$, με $t\in [0,2\pi]$.

\hfill Απ: $-\pi$

\item Να υπολογιστει το επικαμπυλιο ολοκληρωμα $I=\int\limits_C (xy+z)dx+(xy+z)dy+(xy+z)dz$, οπου $C$ η καμπυλη $y^2+z^2=a^2$, $x=0$ με αρχη το σημειο $A(0,0,0$ και περας το σημειο $B(0,a,0)$

\hfill Απ: $-\frac{a^2}{2}+\frac{\pi a^2}{4}$

\underline{\textbf{Υποδειξη:}}
Οταν η καμπυλη δινεται ως συναρτηση της μορφης $y=f(x)$, τοτε για την παραμετρικοποιηση της, θετω: $x=t$ και $y=f(t)$. 

πχ. Αν $C:y=3x^2$, με $x\in [2,3]$ τοτε θετω: $x(t)=t$ και $y(t)=3t^2$ με $t\in [2,3]$.

\end{enumerate}


\end{document}