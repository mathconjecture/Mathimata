\input{preamble_ask.tex}
\input{definitions_ask.tex}
\input{tikz}

\pagestyle{askhseis}


\begin{document}

\begin{center}
  \minibox{\large\bfseries \textcolor{Col2}{Πίνακες}}
\end{center}

% \section*{Ακολουθίες}

\vspace{\baselineskip}

\hfill \textcolor{Col1}{\textbf{Σωστό}} \quad \textcolor{Col1}{\textbf{Λάθος}}
\begin{enumerate}[itemsep=.5\baselineskip]
  \item \textcolor{Col1}{Αν $ A \in \textbf{M}_{n}(\mathbb{R}) $ και $
      A^{2}=I_{n} $, τότε $ A=I_{n} $ ή $ A=-I_{n} $}. 
    \hfill $\textcolor{Col1}{\square} \qquad \qquad \textcolor{Col1}{\blacksquare}$

    Απ: Λάθος. Αν $ A= 
    \begin{pmatrix}
      0 & 1 \\
      1 & 0
    \end{pmatrix} $, τότε $ A^{2}=I_{2} $, όμως $ A \neq I_{2} $ και $ A \neq -I_{2} $.

  \item \textcolor{Col1}{Αν $ A \in \textbf{M}_{n}(\mathbb{R}) $ και $
      A^{2}=I_{n}
    $, τότε $ A^{k}=I_{n}, \; \forall k \geq 2 $}.
    \hfill $\textcolor{Col1}{\square} \qquad \qquad \textcolor{Col1}{\blacksquare}$

    Απ: Λάθος. Αν $ A= 
    \begin{pmatrix}
      0 & 1 \\
      1 & 0
    \end{pmatrix} $, τότε $ A^{3}= A^{2}\cdot A = I_{2} \cdot A = A $, 
    όμως $ A \neq I_{2} $.

  \item \textcolor{Col1}{Αν $ A \in \textbf{M}_{n}(\mathbb{R}) $ και $ A^{2}=I_{n} $, 
    τότε $ A^{k}=I_{n} $ για όλους τους άρτιους φυσικούς αριθμούς $k$}.
    \hfill $\textcolor{Col1}{\blacksquare} \qquad \qquad \textcolor{Col1}{\square}$

    Απ: Σωστό. Αποδεικνύεται με Μαθηματική Επαγωγή
    %todo να γράψω την απόδειξη οπως και την άσκηση 3 με τις ταυτότητες πινάκων
       

\end{enumerate}



\end{document}


