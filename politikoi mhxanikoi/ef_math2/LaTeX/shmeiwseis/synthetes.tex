\documentclass[a4paper,12pt]{article}
\usepackage{etex}
%%%%%%%%%%%%%%%%%%%%%%%%%%%%%%%%%%%%%%
% Babel language package
\usepackage[english,greek]{babel}
% Inputenc font encoding
\usepackage[utf8]{inputenc}
%%%%%%%%%%%%%%%%%%%%%%%%%%%%%%%%%%%%%%

%%%%% math packages %%%%%%%%%%%%%%%%%%
\usepackage{amsmath}
\usepackage{amssymb}
\usepackage{amsfonts}
\usepackage{amsthm}
\usepackage{proof}

\usepackage{physics}

%%%%%%% symbols packages %%%%%%%%%%%%%%
\usepackage{bm} %for use \bm instead \boldsymbol in math mode 
\usepackage{dsfont}
\usepackage{stmaryrd}
%%%%%%%%%%%%%%%%%%%%%%%%%%%%%%%%%%%%%%%


%%%%%% graphicx %%%%%%%%%%%%%%%%%%%%%%%
\usepackage{graphicx}
\usepackage{color}
%\usepackage{xypic}
\usepackage[all]{xy}
\usepackage{calc}
\usepackage{booktabs}
\usepackage{minibox}
%%%%%%%%%%%%%%%%%%%%%%%%%%%%%%%%%%%%%%%

\usepackage{enumerate}

\usepackage{fancyhdr}
%%%%% header and footer rule %%%%%%%%%
\setlength{\headheight}{14pt}
\renewcommand{\headrulewidth}{0pt}
\renewcommand{\footrulewidth}{0pt}
\fancypagestyle{plain}{\fancyhf{}
\fancyhead{}
\lfoot{}
\rfoot{\small \thepage}}
\fancypagestyle{vangelis}{\fancyhf{}
\rhead{\small \leftmark}
\lhead{\small }
\lfoot{}
\rfoot{\small \thepage}}
%%%%%%%%%%%%%%%%%%%%%%%%%%%%%%%%%%%%%%%

\usepackage{hyperref}
\usepackage{url}
%%%%%%% hyperref settings %%%%%%%%%%%%
\hypersetup{pdfpagemode=UseOutlines,hidelinks,
bookmarksopen=true,
pdfdisplaydoctitle=true,
pdfstartview=Fit,
unicode=true,
pdfpagelayout=OneColumn,
}
%%%%%%%%%%%%%%%%%%%%%%%%%%%%%%%%%%%%%%

\usepackage[space]{grffile}

\usepackage{geometry}
\geometry{left=25.63mm,right=25.63mm,top=36.25mm,bottom=36.25mm,footskip=24.16mm,headsep=24.16mm}

%\usepackage[explicit]{titlesec}
%%%%%% titlesec settings %%%%%%%%%%%%%
%\titleformat{\chapter}[block]{\LARGE\sc\bfseries}{\thechapter.}{1ex}{#1}
%\titlespacing*{\chapter}{0cm}{0cm}{36pt}[0ex]
%\titleformat{\section}[block]{\Large\bfseries}{\thesection.}{1ex}{#1}
%\titlespacing*{\section}{0cm}{34.56pt}{17.28pt}[0ex]
%\titleformat{\subsection}[block]{\large\bfseries{\thesubsection.}{1ex}{#1}
%\titlespacing*{\subsection}{0pt}{28.80pt}{14.40pt}[0ex]
%%%%%%%%%%%%%%%%%%%%%%%%%%%%%%%%%%%%%%

%%%%%%%%% My Theorems %%%%%%%%%%%%%%%%%%
\newtheorem{thm}{Θεώρημα}[section]
\newtheorem{cor}[thm]{Πόρισμα}
\newtheorem{lem}[thm]{λήμμα}
\theoremstyle{definition}
\newtheorem{dfn}{Ορισμός}[section]
\newtheorem{dfns}[dfn]{Ορισμοί}
\theoremstyle{remark}
\newtheorem{remark}{Παρατήρηση}[section]
\newtheorem{remarks}[remark]{Παρατηρήσεις}
%%%%%%%%%%%%%%%%%%%%%%%%%%%%%%%%%%%%%%%




\newcommand{\vect}[2]{(#1_1,\ldots, #1_#2)}
%%%%%%% nesting newcommands $$$$$$$$$$$$$$$$$$$
\newcommand{\function}[1]{\newcommand{\nvec}[2]{#1(##1_1,\ldots, ##1_##2)}}

\newcommand{\linode}[2]{#1_n(x)#2^{(n)}+#1_{n-1}(x)#2^{(n-1)}+\cdots +#1_0(x)#2=g(x)}

\newcommand{\vecoffun}[3]{#1_0(#2),\ldots ,#1_#3(#2)}

\newcommand{\mysum}[1]{\sum_{n=#1}^{\infty}


\everymath{\displaystyle}
\pagestyle{vangelis}

\begin{document}


\chapter{Παράγωγος Σύνθετων Συναρτήσεων}

\section{1η Περίπτωση: \ensuremath{z=f(x,y),  x=x(t),  y=y(t)}} 

\begin{thm}
  Αν η συνάρτηση $ f(x,y) $ είναι ορισμένη στο ανοιχτό σύνολο 
  $ A \subseteq \mathbb{R}^{2} $ και $ x = x(t) $, $ y=y(t) $, με 
  $ t \in [a,b] $ και η $f$ έχει συνεχείς μερικές 
  παραγώγους στο $A$ και οι $ x(t) $ και $ y(t) $ είναι παραγωγίσιμες στο $ [a,b] $, 
  τότε η παράγωγος της σύνθετης συνάρτησης $f$ ως προς $t$ δίνεται από τον τύπο:

  \twocolumnsidel{
    \[\xymatrix{ & f \ar@{-}[dl]_{\mathlarger{\pdv{f}{x}}}
        \ar@{-}[dr]^{\mathlarger{\pdv{f}{y}}} &  \\
        x \ar@{-}[dr]_{\mathlarger{\dv{x}{t}}} & & y 
    \ar@{-}[dl]^{\mathlarger{\dv{y}{t}}} \\ & t & }\]
    }{
    \begin{equation}\label{eq:deriv1}
      \dv{f}{t} = \pdv{f}{x} \dv{x}{t} + \pdv{f}{y} \dv{y}{t} 
    \end{equation}
    Ενώ η 2η παράγωγος από τον τύπο:
    \[
      \dv[2]{f}{t} =  \pdv[2]{f}{x} \left(\dv{x}{t}\right)^{2} + 
      2 \pdv[2]{f}{x}{y} \dv{x}{t} \dv{y}{t} + \pdv[2]{f}{y} 
      \left(\dv{y}{t}\right)^{2} + \pdv{f}{x} \dv[2]{x}{t} + \pdv{f}{y} \dv[2]{y}{t}
    \]
  }
\end{thm}

\section{2η Περίπτωση: \ensuremath{z=f(x,y),  x=x(u,v),  y=y(u,v)}} 

\begin{thm}
  Αν η συνάρτηση $ f(x,y) $ είναι ορισμένη στο ανοιχτό σύνολο 
  $ A \subseteq \mathbb{R}^{2} $ και $ x = x(u,v) $, $ y=y(u,v) $, με 
  και η $f$ έχει συνεχείς μερικές παραγώγους στο $A$ και οι $ x $ και $ y $, έχουν 
  συνεχείς μερικές παραγώγους στο $ E \subseteq \mathbb{R}^{2} $,
  τότε οι μερικές παράγωγοι της $f$, υπάρχουν και δίνονται από τους τύπους:
\end{thm}

\twocolumnsidel{
  \[
    \xymatrix@C-6pt{ & & f \ar@{-}[dl]_{\mathlarger{\pdv{f}{x}}}^{\mathlarger{f_{x}}} 
      \ar@{-}[dr]^{\mathlarger{\pdv{f}{y}}}_{\mathlarger{f_{y}}}
                     & &  \\
                     & x \ar@{-}[dl]_{\mathlarger{\pdv{x}{u}}}
      \ar@{-}[d]^{\mathlarger{\pdv{x}{v}}} &  
                                           & y \ar@{-}[d]_{\mathlarger{\pdv{y}{u}}} 
      \ar@{-}[dr]^{\mathlarger{\pdv{y}{v}}} & \\
    u & v & & u & v \\ } 
  \]
  }{
  \begin{equation}\label{eq:deriv2}
    \pdv{f}{u} = \pdv{f}{x} \pdv{x}{u} + \pdv{f}{y} \pdv{y}{u} 
    \quad \text{και} \quad
    \pdv{f}{v} = \pdv{f}{x} \pdv{x}{v} + \pdv{f}{y} \pdv{y}{v} 
  \end{equation}
  Ενώ οι μερικές παράγωγοι 2ης τάξης, δίνονται από τους τύπους:
  \[
    \pdv[2]{f}{u} =  \pdv[2]{f}{x} \left(\pdv{x}{u}\right)^{2} + 
    2 \pdv[2]{f}{x}{y} \pdv{x}{u} \pdv{y}{u} + \pdv[2]{f}{y} 
    \left(\pdv{y}{u}\right)^{2} + \pdv{f}{x} \pdv[2]{x}{u} + \pdv{f}{y} \pdv[2]{y}{u}
  \]
  \[
    \pdv[2]{f}{v} =  \pdv[2]{f}{x} \left(\pdv{x}{v}\right)^{2} + 
    2 \pdv[2]{f}{x}{y} \pdv{x}{u} \pdv{y}{v} + \pdv[2]{f}{y} 
    \left(\pdv{y}{v}\right)^{2} + \pdv{f}{x} \pdv[2]{x}{v} + \pdv{f}{y} \pdv[2]{y}{v}
  \]
}

\begin{rem}
  Οι τύποι~\eqref{eq:deriv1} και~\eqref{eq:deriv2} προέκυψαν αθροίζοντας κάθε φορά, 
  τα μονοπάτια που ξεκινούν από τη μεταβλητή $f$ και καταλήγουν στη μεταβλητή ως προς 
  την οποία παραγωγίζουμε, όπου κάθε 
  μονοπάτι αποτελείται από το γινόμενο των παραγώγων που συναντούμε "διασχίζοντάς" το.
\end{rem}


\section{Αποδείξεις των τύπων των μερικών Παραγώγων 2ης τάξης}

\subsection{Απόδειξη με το συμβολισμό του Leibnitz}

  Αποδεικνύουμε τον τύπο για την $ \pdv[2]{f}{u} $ και ομοίως προκύπτουν και οι τύποι 
  για τις $ \pdv[2]{f}{v}  $ και $ \pdv[2]{f}{u}{v} $.

  \vspace{\baselineskip}

\twocolumnsidel{
  \[ 
    \xymatrix@C-6pt{ & & \pdv{f}{x} 
      \ar@{-}[dl]_{\mathlarger{\pdv[2]{f}{x}}} 
      \ar@{-}[dr]^{\mathlarger{\pdv[2]{f}{x}{y}}} & &  \\
                     & x \ar@{-}[dl]_{\mathlarger{\pdv{x}{u}}}
      \ar@{-}[d]^{\mathlarger{\pdv{x}{v}}} &  
                                           & y \ar@{-}[d]_{\mathlarger{\pdv{y}{u}}} 
      \ar@{-}[dr]^{\mathlarger{\pdv{y}{v}}} & \\
      u & v & & u & v \\ 
    } 
  \] 
  \[\xymatrix@C-6pt{ & & \pdv{f}{y} 
      \ar@{-}[dl]_{\mathlarger{\pdv[2]{f}{y}{x}}} 
      \ar@{-}[dr]^{\mathlarger{\pdv[2]{f}{y}}} & &  \\
             & x \ar@{-}[dl]_{\mathlarger{\pdv{x}{u}}}
      \ar@{-}[d]^{\mathlarger{\pdv{x}{v}}} &  
                                           & y \ar@{-}[d]_{\mathlarger{\pdv{y}{u}}} 
      \ar@{-}[dr]^{\mathlarger{\pdv{y}{v}}} & \\
    u & v & & u & v \\ }
  \] 
  }{
  \begin{proof}
  \[\begin{aligned}
    \pdv[2]{f}{u} 
  &= \pdv{}{u}\left(\pdv{f}{u}\right) = \pdv{}{u} 
  \left( \pdv{f}{x} \pdv{x}{u} + \pdv{f}{y} \pdv{y}{u}\right) = 
  \pdv{}{u} \left(\pdv{f}{x} \pdv{x}{u}\right) + \pdv{}{u} 
  \left(\pdv{f}{y} \pdv{y}{u}\right) \\
  &= \pdv{}{u} \left( \pdv{f}{x}\right) \pdv{x}{u} + \pdv{f}{x} \pdv{}{u} \left(
  \pdv{x}{u} \right) + \pdv{}{u} \left(\pdv{f}{y} \right) \pdv{y}{u} + \pdv{f}{y} 
  \pdv{}{u} \left(\pdv{y}{u}\right) \\
  &=\left[ \pdv[2]{f}{x} \pdv{x}{u} + \pdv[2]{f}{x}{y} \pdv{y}{u} \right] \pdv{x}{u} +
  \pdv{f}{x} \pdv[2]{x}{u} + 
  \left[ \pdv[2]{f}{y}{x} \pdv{x}{u} + \pdv[2]{f}{y} \pdv{y}{u} \right] \pdv{y}{u} +
  \pdv{f}{y} \pdv[2]{y}{u} \\
  &= \pdv[2]{f}{x} \left(\pdv{x}{u}\right)^{2} + \pdv[2]{f}{x}{y} \pdv{y}{u}
  \pdv{x}{u} + \pdv{f}{x} \pdv[2]{x}{u} + \pdv[2]{f}{y}{x} \pdv{x}{u} \pdv{y}{u} + 
  \pdv[2]{f}{y} \left(\pdv{y}{u}\right)^{2} +
  \pdv{f}{y} \pdv[2]{y}{u} \\
  &= \pdv[2]{f}{x} \left(\pdv{x}{u}\right)^{2} + 2\pdv[2]{f}{x}{y} \pdv{x}{u}
  \pdv{y}{u} + \pdv[2]{f}{y} \left(\pdv{y}{u}\right)^{2} + \pdv{f}{x} \pdv[2]{x}{u} +
  \pdv{f}{y} \pdv[2]{y}{u} \\
  \end{aligned}
\]
\end{proof}
}


\subsection{Απόδειξη με το συμβολισμό των δεικτών}

Αποδεικνύουμε τον τύπο για την $ f_{uu} $ και ομοίως προκύπτουν και οι τύποι για τις  
$ f_{vv} $ και $ f_{uv} $.

\vspace{\baselineskip}

\twocolumnsidel{
{
  \[ 
    \xymatrix@C-6pt{ & & f_{x}
      \ar@{-}[dl]_{\mathlarger{f_{xx}}} 
      \ar@{-}[dr]^{\mathlarger{f_{xy}}} & &  \\
                                        & x \ar@{-}[dl]_{\mathlarger{x_{u}}}
      \ar@{-}[d]^{\mathlarger{x_{v}}} &  
                                      & y \ar@{-}[d]_{\mathlarger{y_{u}}} 
      \ar@{-}[dr]^{\mathlarger{y_{v}}} & \\
      u & v & & u & v \\ 
    } 
  \] 
  \[\xymatrix@C-6pt{ & & f_{y}
      \ar@{-}[dl]_{\mathlarger{f_{yx}}} 
      \ar@{-}[dr]^{\mathlarger{f_{yy}}} & &  \\
                                        & x \ar@{-}[dl]_{\mathlarger{x_{u}}}
      \ar@{-}[d]^{\mathlarger{x_{v}}} &  
                                      & y \ar@{-}[d]_{\mathlarger{y_{u}}} 
      \ar@{-}[dr]^{\mathlarger{y_{v}}} & \\
    u & v & & u & v \\ }
  \] 
  }
}{
\begin{proof}
\[
   \begin{aligned}
     f_{uu} &= (f_{u})_{u} = (f_{x}x_{u}+f_{y}y_{u})_{u} \\
            &=(f_{x}x_{u})_{u}+ (f_{y}y_{u})_{u} \\
            &=(f_{x})_{u}x_{u} + f_{x}(x_{u})_{u} + (f_{y})_{u}y_{u}+ f_{y}(y_{u})_{u} \\
            &= (f_{xx}x_{u}+f_{xy}{y_{u}})x_{u} + f_{x} x_{uu} + 
            (f_{yx}x_{u}+f_{yy}y_{u})y_{u} + f_{y}y_{uu} \\
            &= f_{xx}(x_{u})^{2} + f_{xy}y_{u}x_{u}+ f_{x}x_{uu} + 
            f_{yx}x_{u}y_{u}+f_{yy}(y_{u})^{2}+ f_{y}y_{uu} \\
            &= f_{xx}(x_{u})^{2}+ 2f_{xy}x_{u}y_{u} + f_{yy}(y_{u})^{2} + f_{x}x_{uu} + 
            f_{y}y_{uu}
   \end{aligned}
 \] 
\end{proof}
}

\end{document}
