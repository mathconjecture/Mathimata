\input{preamble.tex}
\newcommand{\vect}[2]{(#1_1,\ldots, #1_#2)}
%%%%%%% nesting newcommands $$$$$$$$$$$$$$$$$$$
\newcommand{\function}[1]{\newcommand{\nvec}[2]{#1(##1_1,\ldots, ##1_##2)}}

\newcommand{\linode}[2]{#1_n(x)#2^{(n)}+#1_{n-1}(x)#2^{(n-1)}+\cdots +#1_0(x)#2=g(x)}

\newcommand{\vecoffun}[3]{#1_0(#2),\ldots ,#1_#3(#2)}

\newcommand{\suma}{\sum_{n=0}^{\infty}a_n x^n}

\newcommand{\sumb}{\sum_{n=1}^{\infty}a_n n x^{n-1}}

\newcommand{\sumc}{\sum_{n=2}^{\infty}a_n n (n-1) x^{n-2}}

\newcommand{\varsum}[2]{\sum_{n=#1}^{#2}}

\everymath{\displaystyle}

\thispagestyle{askhseis}

\begin{document}
\begin{center}
  \minibox{\large\bfseries \textcolor{Col1}{Πολικές Καμπύλες}}
\end{center}

\vspace{\baselineskip}


\section*{Εμβαδό Επίπεδων Χωρίων}

\begin{enumerate}
  \item Να υπολογιστεί το εμβαδό του χωρίου που ορίζεται από την καρδιοειδή καμπύλη
    $ r = a(1  + \cos{\theta}) $.

    \hfill Απ: $ \frac{3\pi a^{2}}{2} $

  \item Να υπολογιστεί το εμβαδό του χωρίου που περικλείεται από την καμπύλη $
    r = a \cos{3\theta} $.

    \hfill Απ: $ \frac{\pi a^{2}}{4} $

  \item Να υπολογιστεί το εμβαδό του χωρίου που ορίζεται από την καμπύλη 
    $ r = 4 \sin{3\theta}  $. 

    \hfill Απ: $ 4 \pi $  

  \item Να υπολογιστεί το εμβαδό του χωρίου που περικλείεται από την καμπύλη $
    r = a \sin{2\theta} $.

    \hfill Απ: $ \frac{\pi a^{2}}{2} $

\end{enumerate}



\section*{Μήκος Καμπύλης}


\begin{enumerate}
  \item Να υπολογιστεί το μήκος της καρδιοειδούς καμπύλης $ r = a(1 + \cos{\theta}), 
    \; a>0 $.

    \hfill Απ: $ 8 a $

  \item Να υπολογιστεί το μήκος της καμπύλης 
    $ r = a\sin^{3}{\left(\frac{\theta}{3}\right)}, \; a>0 $.

    \hfill Απ: $ \frac{3 \pi a}{2} $ 
\end{enumerate}




\section*{Εμβαδό Επιφάνειας απο Περιστροφή}

\begin{enumerate}
  \item Να υπολογιστεί το εμβαδο της επιφάνειας του στερεού που προκύπτει από την
    περιστροφή της καμπύλης $ r = a(1 - \cos{\theta}) $ γύρω από τον πολικό
    άξονα. 

    \hfill Απ: $ \frac{32 \pi  a^{2}}{5} $
\end{enumerate}

\end{document}
