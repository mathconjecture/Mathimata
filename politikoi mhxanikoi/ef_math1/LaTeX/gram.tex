\documentclass[a4paper,12pt]{article}

\usepackage[english,greek]{babel}
\usepackage[utf8]{inputenc}

\usepackage{amsmath}
\usepackage{amssymb}
\usepackage{amsfonts}
\usepackage{amsthm}
\usepackage[top=2cm,bottom=2cm,left=2cm,right=2cm]{geometry}

\usepackage{systeme} 


\begin{document}
\thispagestyle{empty}
\begin{center}
{\bfseries Ασκήσεις στα Γραμμικά Συστήματα (Με παράμετρο)}
\end{center}

\vspace{\baselineskip}


\begin{enumerate}
	\item Για ποιες τιμες του $\lambda$ το παρακάτω σύστημα έχει και μη μηδενικές λύσεις?

		\systeme{
			(2\-\ \lambda)x +y = 0, 
			-x - \lambda y +z = 0, 
			x + 3y + (1\-\ \lambda)z = 0
		} \hfill Απ: $ \lambda = 1 $ και $ \lambda = -2 $


	\item Να λυθεί και να διερευνηθεί για κάθε τιμή του πραγματικού $\lambda$, το παρακάτω σύστημα:

		\systeme{
			(\lambda\-\ 2)x + y =0, 
			-x-2y+z =0, 
			x+3y+(1\-\ \lambda)z=0
		} \hfill Απ: $ \lambda =2 $, $ \lambda =3 $



\item Για ποιες τιμές του $\lambda\in \mathbb{R}$ το παρακάτω σύστημα είναι αδύνατο?


	\systeme{
	x -\lambda y +z =1, 
x-y+z =-1, 
\lambda x+ \lambda^2 y -z= \lambda^2 
	} \hfill Απ: $ \lambda = 1 $, $ \lambda = -1 $


\item Να διερευνηθεί για κάθε τιμή του $\lambda$ το παρακάτω σύστημα και να δοθούν οι λύσεις του.

	\systeme {
	2\lambda x + y +\lambda z =1, 
x +\lambda y+z =\lambda, 
\lambda x+  y +\lambda z= -\lambda
	} \hfill Απ: $ \lambda =0 $, $ \lambda = 1 $, $ \lambda = -1 $
\end{enumerate}



\end{document}
