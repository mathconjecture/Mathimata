\documentclass[a4paper]{report}
\input{preamble.tex}
\input{definitions_ask.tex}


\everymath{\displaystyle}

\pagestyle{askhseis}

\begin{document}

\begin{center}
  \minibox{\large\bfseries \textcolor{Col1}{Aσκήσεις στις Εφαρμογές του
  Ολοκληρώματος}}
\end{center}

\vspace{\baselineskip}

\begin{exercise}
  \begin{enumerate}[i)]
    \item Να υπολογιστεί το εμβαδό του επίπεδου χωρίου που περικλείεται 
      από τις καμπύλες $ f(x) = 3x^2-x+5 $ και $ g(x) = 2x^2-2x+7 $.

  \hfill Απ: $ E = \frac{9}{2} $ 

    \item Να υπολογιστεί το εμβαδό του επίπεδου χωρίου που περικλείεται 
      από τις καμπύλες $ f(x) = 2x^2+6x+5 $ και $ g(x) = x^2+2x+2 $.

  \hfill Απ: $ E = \frac{4}{3} $ 
  \end{enumerate}
\end{exercise}

\begin{exercise}
  Να υπολογιστεί η μέση τιμή $ av(f) $ και η R.M.S. τιμή των παρακάτω 
  συναρτήσεων:

  \begin{enumerate}[i)]
    \item $ f(x) = 2 \cos{x} + 3 \sin{x} $ στο διάστημα $ [0, \pi] $.

      \quad \textcolor{Col1}{Υπόδειξη:} 
      $ \sin^{2}{x} = \frac{1- \cos{2x}}{2}$ και $ \cos^{2}{x}
      = \frac{1+ \cos{2x}}{2} $ και $ \sin{2x} = 2 \sin{x} \cos{x} $ 

      \hfill Απ: $ av(f) = \frac{6}{\pi} $, $ RMS= \frac{13 \pi}{2} $ 
  \end{enumerate}
\end{exercise}










\end{document}

