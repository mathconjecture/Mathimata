\documentclass[a4paper,12pt]{article}

\usepackage[english,greek]{babel}
\usepackage[utf8]{inputenc}
\usepackage{amsmath}
\usepackage{amssymb}
\usepackage{amsfonts}

\usepackage[margin=1in]{geometry}

\begin{document}

\thispagestyle{empty}

\begin{center}
\fbox{\Large\bfseries Ασκήσεις στη Μαθηματική Επαγωγή}
\end{center}

\vspace{1cm}


\textbf{Παράδειγμα:} Να αποδείξετε ότι 

\begin{equation}\label{eq:prop}
\frac{1}{1\cdot 2}+\frac{1}{2\cdot 3}+\cdots + \frac{1}{n(n+1)} = \frac{n}{n+1}, \quad \text{ισχύει} \quad \forall n\in \mathbb{N}.
\end{equation}

\textbf{Απόδειξη:}

\begin{itemize}
\item Αρχικά αποδεικνύουμε ότι η πρόταση ισχύει για $n=1$. (Δηλ. αποδεικνύουμε ότι $P(1)$).

Πράγματι για $n=1$ (δηλ. με αντικατάσταση στη σχέση (\ref{eq:prop}) )έχω: 
\[
\frac{1}{1\cdot 2} = \frac{1}{1+1} \Leftrightarrow \frac{1}{2}=\frac{1}{2}.
\]
\item Έστω ότι η πρόταση ισχύει για $n=k$, $k\in \mathbb{N}$ τυχαίος φυσικός αριθμός (Δηλ. έστω ότι $P(k)$).
Δηλαδή, έστω ότι:  

\begin{equation}\label{eq:ye}
\frac{1}{1\cdot 2}+\frac{1}{2\cdot 3}+\cdots + \frac{1}{k(k+1)} = \frac{k}{k+1}.
\end{equation}

\item Τελικά αποδεικνύουμε ότι η πρόταση ισχύει για $n=k+1$ (Δηλ. $P(k+1)$). Πράγματι: 
\begin{align*}
\underbrace{\frac{1}{1\cdot 2}+\frac{1}{2\cdot 3}+\cdots +\frac{1}{k(k+1)}}+\frac{1}{(k+1)(k+1+1)} &= \frac{k}{k+1}+\frac{1}{(k+1)(k+1+1)} \\
&\overset{(\ref{eq:ye})}{=} \frac{k(k+2)+1}{(k+1)(k+2)}  \\
&=  \frac{k^2+2k+1}{(k+1)(k+2)}  \\ 
&=  \frac{(k+1)^2}{(k+1)(k+2)}  \\
&=  \frac{k+1}{k+2}  \\ 
&=  \frac{k+1}{k+1+1}
\end{align*}
\end{itemize}

\vspace{1cm}

\textbf{Ασκήσεις:} Να αποδειχθούν οι παρακάτω προτάσεις.
\begin{enumerate}
\item $2+4+6+\cdots +2n = n(n+1)$, για κάθε $n\in \mathbb{N}$.
\item $1^2+2^2+\cdots +n^2 = \frac{n(n+1)(n+2)}{6}$, για κάθε $n\in \mathbb{N}$.
\item $1\cdot 2 + 2\cdot 3 +\cdots + n(n+1) = \frac{n(n+1)(n+2)}{3}$, για κάθε $n\geq 1$.
\end{enumerate}

\end{document}