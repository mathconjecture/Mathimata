\documentclass[a4paper,12pt]{article}
\usepackage{etex}
%%%%%%%%%%%%%%%%%%%%%%%%%%%%%%%%%%%%%%
% Babel language package
\usepackage[english,greek]{babel}
% Inputenc font encoding
\usepackage[utf8]{inputenc}
%%%%%%%%%%%%%%%%%%%%%%%%%%%%%%%%%%%%%%

%%%%% math packages %%%%%%%%%%%%%%%%%%
\usepackage{amsmath}
\usepackage{amssymb}
\usepackage{amsfonts}
\usepackage{amsthm}
\usepackage{proof}

\usepackage{physics}

%%%%%%% symbols packages %%%%%%%%%%%%%%
\usepackage{bm} %for use \bm instead \boldsymbol in math mode 
\usepackage{dsfont}
\usepackage{stmaryrd}
%%%%%%%%%%%%%%%%%%%%%%%%%%%%%%%%%%%%%%%


%%%%%% graphicx %%%%%%%%%%%%%%%%%%%%%%%
\usepackage{graphicx}
\usepackage{color}
%\usepackage{xypic}
\usepackage[all]{xy}
\usepackage{calc}
\usepackage{booktabs}
\usepackage{minibox}
%%%%%%%%%%%%%%%%%%%%%%%%%%%%%%%%%%%%%%%

\usepackage{enumerate}

\usepackage{fancyhdr}
%%%%% header and footer rule %%%%%%%%%
\setlength{\headheight}{14pt}
\renewcommand{\headrulewidth}{0pt}
\renewcommand{\footrulewidth}{0pt}
\fancypagestyle{plain}{\fancyhf{}
\fancyhead{}
\lfoot{}
\rfoot{\small \thepage}}
\fancypagestyle{vangelis}{\fancyhf{}
\rhead{\small \leftmark}
\lhead{\small }
\lfoot{}
\rfoot{\small \thepage}}
%%%%%%%%%%%%%%%%%%%%%%%%%%%%%%%%%%%%%%%

\usepackage{hyperref}
\usepackage{url}
%%%%%%% hyperref settings %%%%%%%%%%%%
\hypersetup{pdfpagemode=UseOutlines,hidelinks,
bookmarksopen=true,
pdfdisplaydoctitle=true,
pdfstartview=Fit,
unicode=true,
pdfpagelayout=OneColumn,
}
%%%%%%%%%%%%%%%%%%%%%%%%%%%%%%%%%%%%%%

\usepackage[space]{grffile}

\usepackage{geometry}
\geometry{left=25.63mm,right=25.63mm,top=36.25mm,bottom=36.25mm,footskip=24.16mm,headsep=24.16mm}

%\usepackage[explicit]{titlesec}
%%%%%% titlesec settings %%%%%%%%%%%%%
%\titleformat{\chapter}[block]{\LARGE\sc\bfseries}{\thechapter.}{1ex}{#1}
%\titlespacing*{\chapter}{0cm}{0cm}{36pt}[0ex]
%\titleformat{\section}[block]{\Large\bfseries}{\thesection.}{1ex}{#1}
%\titlespacing*{\section}{0cm}{34.56pt}{17.28pt}[0ex]
%\titleformat{\subsection}[block]{\large\bfseries{\thesubsection.}{1ex}{#1}
%\titlespacing*{\subsection}{0pt}{28.80pt}{14.40pt}[0ex]
%%%%%%%%%%%%%%%%%%%%%%%%%%%%%%%%%%%%%%

%%%%%%%%% My Theorems %%%%%%%%%%%%%%%%%%
\newtheorem{thm}{Θεώρημα}[section]
\newtheorem{cor}[thm]{Πόρισμα}
\newtheorem{lem}[thm]{λήμμα}
\theoremstyle{definition}
\newtheorem{dfn}{Ορισμός}[section]
\newtheorem{dfns}[dfn]{Ορισμοί}
\theoremstyle{remark}
\newtheorem{remark}{Παρατήρηση}[section]
\newtheorem{remarks}[remark]{Παρατηρήσεις}
%%%%%%%%%%%%%%%%%%%%%%%%%%%%%%%%%%%%%%%




\newcommand{\vect}[2]{(#1_1,\ldots, #1_#2)}
%%%%%%% nesting newcommands $$$$$$$$$$$$$$$$$$$
\newcommand{\function}[1]{\newcommand{\nvec}[2]{#1(##1_1,\ldots, ##1_##2)}}

\newcommand{\linode}[2]{#1_n(x)#2^{(n)}+#1_{n-1}(x)#2^{(n-1)}+\cdots +#1_0(x)#2=g(x)}

\newcommand{\vecoffun}[3]{#1_0(#2),\ldots ,#1_#3(#2)}

\newcommand{\mysum}[1]{\sum_{n=#1}^{\infty}


\everymath{\displaystyle}
\pagestyle{vangelis}

\begin{document}

\begin{center}
  \minibox[c]{\large \bfseries \textcolor{Col1}{Ακολουθίες}\\ \large 
  \textcolor{Col1}{Ασκήσεις}}
\end{center}

\vspace{\baselineskip}


\setcounter{chapter}{1}
\section*{Φραγμένες Ακολουθίες}

\begin{enumerate}
  \item Να δείξετε ότι η ακολουθία $ a_{n} = (-1)^{n}\frac{1}{2n} $ είναι 
    φραγμένη.
    \hfill Απ: $ \abs{a_{n}} \leq \frac{1}{2} $ 
  \item Να δείξετε ότι η ακολουθία $ a_{n} = \frac{5 \cos^{3}{n}}{n+2} $ 
    είναι φραγμένη.
    \hfill Απ: $ \abs{a_{n}} < \frac{5}{2}  $ 
  \item Να δείξετε ότι η ακολουθία $ a_{n} = \frac{\cos{n} + n \sin{n}}{n^{2}} $ 
    είναι φραγμένη. 
    \hfill Απ: $ \abs{a_{n}} \leq 2 $ 
  \item Να δείξετε ότι η ακολουθία $ a_{n} = \frac{n}{3^{n}} $ είναι 
    φραγμένη. 
    \hfill Απ: $ \abs{a_{n}}< \frac{1}{2} $
  \item Να δείξετε ότι η ακολουθία $ a_{n} = 2^{n} $ δεν είναι άνω 
    φραγμένη.
    % \item Να δείξετε ότι η ακολουθία $ a_{n} = \frac{n^{3} + \sin{5n}}{n} $ δεν είναι 
    %     φραγμένη.
  \item Να δείξετε ότι η ακολουθία $ a_{n} = \frac{n^{2}}{3n+ \sin^{2}{n}} $ δεν 
    είναι άνω φραγμένη.
\end{enumerate}

\section*{Μονότονες Ακολουθίες}

\begin{enumerate}
  \item Να δείξετε ότι ακολουθία $ a_{n} = \frac{n}{5n-1} $ είναι 
    γνησίως φθίνουσα.
  \item Να δείξετε ότι ακολουθία $ a_{n} = \frac{n}{3^{n}} $ είναι 
    γνησίως φθίνουσα.
  \item Να δείξετε ότι ακολουθία $ a_{n} = \frac{2^{n}}{n!} $ είναι 
    γνησίως φθίνουσα.
  \item Να δείξετε ότι ακολουθία $ a_{n} = \frac{2n^{2}-1}{n} $ είναι γνησίως 
    αύξουσα.
  \item Να δείξετε ότι ακολουθία $ a_{n} =  \frac{(-1)^{n}}{n^{2}+2} $ 
    δεν είναι μονότονη.
\end{enumerate}



\section*{Άλγεβρα και θεωρήματα Ορίων}

\begin{enumerate}

  \item Να υπολογιστούν τα παρακάτω όρια.
    \begin{enumerate}[i)]
      \item $ \lim_{n \to \infty} \frac{n^{2}+3n}{n^{2}+2n+1} $ \hfill Απ: 1 
      \item $ \lim_{n \to \infty} \sqrt[3]{\frac{n^{3}+n}{n^{3}+2n}} $ 
        \hfill Απ: 1 
    \end{enumerate}

  \item Να υπολογιστούν τα παρακάτω όρια με τη βοήθεια του Κριτηρίου 
    Παρεμβολής.

    \begin{enumerate}[i)]
      \item $ \lim_{n \to \infty} \frac{(-1)^{n}}{n^{2}+2n}  $ \hfill Απ: 0  
      \item $ \lim_{n \to \infty} (\sqrt{n+2} - \sqrt{n})  $ \hfill Απ:0
      \item $ \lim_{n \to \infty} \frac{4 \sin^{3}{n} + 3 \cos^{2}{n}}{n^{2}} $ 
        \hfill Απ:0
      \item $ \lim_{n \to \infty} \frac{\cos{n} + 3 \sin{4n}}{ 2
        \sqrt{n} -1} $ \hfill Απ: 0  
    \end{enumerate}
\end{enumerate}


\end{document}

