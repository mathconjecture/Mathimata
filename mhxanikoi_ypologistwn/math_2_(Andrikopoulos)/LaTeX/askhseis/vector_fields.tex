\input{preamble_ask.tex}
\input{definitions_ask.tex}

\pagestyle{askhseis}
\everymath{\displaystyle}
\renewcommand{\vec}{\mathbf}

\begin{document}

\begin{center}
  \minibox{\large\bf \textcolor{Col1}{Κλίση, Απόκλιση, Στροβιλισμός}}
\end{center}

\vspace{\baselineskip}

\begin{enumerate}

  \item Να αποδείξετε τις παρακάτω ταυτότητες για τους διαφορικούς τελεστές.

    \begin{enumerate}[i)]
      \item $\curl(\grad f)=0, $ για κάθε βαθμωτό πεδίο $f$.
      \item $\div(\curl F)=0$ για κάθε διανυσματικό πεδίο $ \mathbf{F} $.
    \end{enumerate}

  \item Να βρεθεί η κλίση της συνάρτησης $ f(x,y)=3y^2e^x+2x^3y $ στο σημείο $P(1,2)$.

    \hfill Απ: $\grad f(1,2)=12(1+e)\vec{i}+2(1+6e)\vec{j}$

  \item Να βρεθεί η κλίση της συνάρτησης $ f(x,y, z)=xy\ln(x-z) $ στο σημείο $P(1,1,-1)$

    \hfill Απ: $\grad f(1,-1,1)=\Bigl(\frac{1}{2}+\ln2\Bigr)\vec{i}+\ln 2\vec{j}- 
    \frac{1}{2}\vec{k} $


  \item Να υπολογιστεί η απόκλιση του διανυσματικού πεδίου
    $ \boldsymbol{F}(x,y,z)=(2x^3-z)\vec{i}+y\cos x\vec{j}+xe^z\vec{k} $

    \hfill Απ: $\div\boldsymbol{F}=6x^2+\cos x+xe^z$

  \item Να βρεθεί η απόκλιση του διανυσματικού πεδίου 
    $ \boldsymbol{F}=(2x^2y^3)\vec{i}-(3y^2z)\vec{j}+(z^2y)\vec{k} $ στο σημείο $P(2,-1,1)$

    \hfill Απ: $\div F(2,-1,1)=-4$

  \item Να υπολογιστεί ο στροβιλισμός του διανυσματικού πεδίου
    $ \boldsymbol{F}(x,y,z)=(x+y)\vec{i}+(4z\sin x)\vec{j}+x^3\vec{k} $

    \hfill Απ: $\curl F=-4\sin x\vec{i}-3x^2\vec{j}+(4z\cos x-1)\vec{k}$

  \item Να βρεθεί ο στροβιλισμός του διανυσματικού πεδίου 
    $ x^2y\vec{i}+(x^2-yz)\vec{j}+3xz\vec{k} $

    \hfill Απ: $\curl F=y\vec{i}-3z\vec{j}+x(2-x)\vec{k}$

  \item Δίνεται το βαθμωτό πεδίο 
    $f(x,y,z)=3y^4z^2\vec{i}+4x^3z^2\vec{j}-3x^2y^2\vec{k}$. Να δείξετε ότι το 
    πεδίο είναι σωληνοειδές.

  \item Δίνεται το διανυσματικό πεδίο 
    $\boldsymbol{F}(x,y,z)=(x^4+yz^3)\vec{i}+x^3\vec{j}+(-1+x^2+y^2)\vec{k}$. 
    Να υπολογιστούν:
    \begin{enumerate}[i)]
      \item $\div F$ \hfill Απ: $\mathrm{i)}\; \div F=4x^3$ 
      \item $\grad(\div F)$ \hfill Απ:  
        $\mathrm{ii)}\; \grad(\div F)=12x^2\vec{i}+0\vec{j}+0\vec{k}$ 
      \item $\curl(\curl F)$
        \hfill Απ: 
        $\mathrm{iii)}\; \curl(\curl F)=(-6yz)\vec{i}+(-6x)\vec{j}+(-2y^2-2x^2)\vec{k}$

    \end{enumerate}

  \item Δίνεται το διανυσματικό πεδίο $\boldsymbol{F}=x\vec{i}+y\vec{j}+z\vec{k}$. 
    \begin{enumerate}[i)]
      \item Να δείξετε ότι το πεδίο είναι αστρόβιλο.
      \item Να βρειτε το $\div F$. \hfill Απ: $\mathrm{ii)}\; \div F=3$ 
      \item Να βρειτε το $\grad(\div F)$.
        \hfill Απ: $\mathrm{iii)}\; \grad(\div F)=0$
    \end{enumerate}



  \item Θεωρούμε το διανυσματικό πεδίο 
    $\boldsymbol{F}(x,y,z)=e^{2x}\vec{i}+f(x,z)\vec{j}+e^z\vec{k}$ 
    \begin{enumerate}[i)]
      \item Να βρεθεί η συνάρτηση $f(x,z)$ ώστε το πεδίο $ \bm{F} $ 
        να είναι συντηρητικό \hfill Απ:  $\mathrm{i)}\; f(x,z)=c$ (σταθ.)
      \item Να βρείτε τη συνάρτηση δυναμικού.
        \hfill Απ:  $\mathrm{ii)}\;f(x,y,z)=\frac{e^{2x}}{2}+cy+e^z-\frac{3}{2}$
    \end{enumerate}
\end{enumerate}

\end{document}

