\input{preamble.tex}
\input{definitions_ask.tex}

\pagestyle{askhseis}

\renewcommand{\vec}{\mathbf}

\begin{document}

\begin{center}
  \minibox{\large \bfseries \textcolor{Col1}{Ασκήσεις στα Ακρότατα}}
\end{center}

\vspace{\baselineskip}

\section*{Τοπικά Ακρότατα}

\begin{enumerate}
  \item Να βρεθούν και να χαρακτηριστούν τα κρίσιμα σημεία  των παρακάτω συναρτήσεων:
    \begin{enumerate}[i)]
      \item $ f(x,y) = x^{3} + y^{3} + 3xy $ 
        \hfill Απ: max: $(-1,-1)  $, σάγμα: $ (0,0) $
      \item $ f(x,y) = x^{2}+y^{4} $ 
        \hfill Απ: min: $ (0,0) $ 
      \item $ f(x,y) = x^{3} + y^{3} - 3x -12y + 50 $ 
        \hfill Απ: max: $ (-1,-2)$, min: $ (1,2) $, 
        σάγμα: $ (1,-2), (-1,2) $
      \item $ f(x,y) = x^{3} + y^{3} -3x -3y + 1 $ 
        \hfill Απ: max: $(-1,-1)  $, min: $ (1,1) $,
        σάγμα: $ (1,-1), (-1,1) $
      \item $ f(x,y) = x^{3} + 4xy -4y^{2} $ 
        \hfill Απ: max: $ (-2/3, -1/3)  $, σάγμα: $ (0,0) $
      \item $ f(x,y) = x^{4} + y^{4} -2(x-y)^{2}$  
        \hfill Απ: min: $ (\sqrt{2} , -\sqrt{2}), (-\sqrt{2} , \sqrt{2}) $, 
        σάγμα: $ (0,0) $
      \item $ f(x,y) = (x^{2}-3y^{2})e^{1-x^{2}-y^{2}} $ 
        \hfill Απ: max: $ (1,0), (-1,0) $, min: $ (0,1), (0,-1) $, 
        σάγμα: $ (0,0) $
    \end{enumerate}

  \item Να βρεθούν και να χαρακτηριστούν τα κρίσιμα σημεία της συνάρτησης 
    $ f(x,y,z) = x^{3} + y^{3}+z^{3} + 3xy +3yz + 3xz $

    \hfill Απ: σάγμα: $(0,0,0)$, max: $(-2,-2,-2)$  

  \item Να βρεθεί η ελάχιστη απόσταση του επιπέδου με εξίσωση $ x+y+z=4 $, από την 
    αρχή των αξόνων.

    \hfill Απ: $ d_{\min}(4/3,4/3) = 4\frac{\sqrt{3}}{3} $  

  \item Να βρεθεί η ελάχιστη απόσταση του επιπέδου με εξίσωση $ 3x+2y+z=6 $, από την 
    αρχή των αξόνων.

    \hfill Απ: $ d_{\min}(9/7,6/7) = 3\frac{\sqrt{14}}{7} $  

  \item Να βρεθεί η ελάχιστη απόσταση του σημείου $ P(2,-1,1) $ από το επίπεδο με 
    εξίσωση $ x+y-z=2 $. 
    
    \hfill Απ: $ d_{min}(8/3,-1/3) = 2 /\sqrt{3} $ 

  \item Να βρεθεί η ελάχιστη απόσταση του σημείου $ P(-6,4,0) $ από τον κώνο με 
    εξίσωση $ z = \sqrt{x^{2}+y^{2}} $. 
    
    \hfill Απ: $ d_{min}(-3,2) = \sqrt{26} $ 

  % \item Δίνεται τρίγωνο ΑΒΓ. Να βρεθεί σημείο P, στο επίπεδο του τριγώνου, ώστε 
  %   το άθροισμα των τετραγώνων των αποστάσεών του από τις κορυφές του τριγώνου 
  %   να είναι ελάχιστο.
\end{enumerate}


\section*{Απόλυτα Ακρότατα (Σε κλειστή και φραγμένη περιοχή)}

\begin{enumerate}
    %Thomas 12th 14.7 ex.41 
  \item Μια λεπτή, επίπεδη, μεταλλική πλάκα, σε σχήμα κυκλικού δίσκου 
    $ x^{2}+y^2 \leq 1 $, θερμαίνεται, έτσι ώστε η θερμοκρασία σε κάθε σημείο της να 
    δίνεται από τη συνάρτηση 
    $ T(x,y) = 2y^{2}+x^{2}-x $.
    Να βρείτε τη θερμοκρασία στα πιο θερμά και πιο ψυχρά σημεία αυτής της πλάκας.

    \hfill Απ:  
    \begin{tabular}{l}
      max $ T(-1/2, \sqrt{3} /2) = T(-1/2, - \sqrt{3} /2) = 9/4 $ \\
      min $ f(1/2,0) = -1/4 $ 
    \end{tabular}

  \item Να υπολογιστούν τα ολικά ακρότατα της συνάρτησης $ f(x,y) = \sqrt{x^{2}+y^{2}} +
    y^{2}-1 $, στον κυκλικό δίσκο $ x^{2}+y^{2} \leq 9 $. 

    \hfill Απ: 
    \begin{tabular}{l}
      max: $ (0,3), (0,-3) $ \\
      min: $ (3,0), (-3,0) $, 
    \end{tabular}

    %Thomas 12th 14.7 ex.32 
  \item Να υπολογιστούν τα ακρότατα της συνάρτησης $ f(x,y) = x^{2}-xy+y^{2}+1 $ 
    στην κλειστή, τριγωνική περιοχή του επιπέδου, που περικλείεται από τις καμπύλες 
    $ x=0 $, $ y=4 $ και $ y=x $.

    \hfill Απ:  
    \begin{tabular}{l}
      max $ f(0,4) f(4,4) = 17 $ \\
      min $ f(0,0) = 1 $ 
    \end{tabular}

    %Thomas 12th 14.7 ex.31 
  \item Να υπολογιστούν τα ακρότατα της συνάρτησης $ f(x,y) = 2x^{2}-4x+y^{2}-4y+1 $ 
    στην κλειστή, τριγωνική περιοχή του επιπέδου, που περικλείεται από τις καμπύλες 
    $ x=0 $, $ y=2 $ και $ y=2x $.

    \hfill Απ:  
    \begin{tabular}{l}
      max $ f(0,0) = 1 $ \\
      min $ f(1,2) = -5 $ 
    \end{tabular}
\end{enumerate}



\end{document}


