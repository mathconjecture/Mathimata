\documentclass[a4paper,table]{report}
\documentclass[a4paper,12pt]{article}
\usepackage{etex}
%%%%%%%%%%%%%%%%%%%%%%%%%%%%%%%%%%%%%%
% Babel language package
\usepackage[english,greek]{babel}
% Inputenc font encoding
\usepackage[utf8]{inputenc}
%%%%%%%%%%%%%%%%%%%%%%%%%%%%%%%%%%%%%%

%%%%% math packages %%%%%%%%%%%%%%%%%%
\usepackage{amsmath}
\usepackage{amssymb}
\usepackage{amsfonts}
\usepackage{amsthm}
\usepackage{proof}

\usepackage{physics}

%%%%%%% symbols packages %%%%%%%%%%%%%%
\usepackage{bm} %for use \bm instead \boldsymbol in math mode 
\usepackage{dsfont}
\usepackage{stmaryrd}
%%%%%%%%%%%%%%%%%%%%%%%%%%%%%%%%%%%%%%%


%%%%%% graphicx %%%%%%%%%%%%%%%%%%%%%%%
\usepackage{graphicx}
\usepackage{color}
%\usepackage{xypic}
\usepackage[all]{xy}
\usepackage{calc}
\usepackage{booktabs}
\usepackage{minibox}
%%%%%%%%%%%%%%%%%%%%%%%%%%%%%%%%%%%%%%%

\usepackage{enumerate}

\usepackage{fancyhdr}
%%%%% header and footer rule %%%%%%%%%
\setlength{\headheight}{14pt}
\renewcommand{\headrulewidth}{0pt}
\renewcommand{\footrulewidth}{0pt}
\fancypagestyle{plain}{\fancyhf{}
\fancyhead{}
\lfoot{}
\rfoot{\small \thepage}}
\fancypagestyle{vangelis}{\fancyhf{}
\rhead{\small \leftmark}
\lhead{\small }
\lfoot{}
\rfoot{\small \thepage}}
%%%%%%%%%%%%%%%%%%%%%%%%%%%%%%%%%%%%%%%

\usepackage{hyperref}
\usepackage{url}
%%%%%%% hyperref settings %%%%%%%%%%%%
\hypersetup{pdfpagemode=UseOutlines,hidelinks,
bookmarksopen=true,
pdfdisplaydoctitle=true,
pdfstartview=Fit,
unicode=true,
pdfpagelayout=OneColumn,
}
%%%%%%%%%%%%%%%%%%%%%%%%%%%%%%%%%%%%%%

\usepackage[space]{grffile}

\usepackage{geometry}
\geometry{left=25.63mm,right=25.63mm,top=36.25mm,bottom=36.25mm,footskip=24.16mm,headsep=24.16mm}

%\usepackage[explicit]{titlesec}
%%%%%% titlesec settings %%%%%%%%%%%%%
%\titleformat{\chapter}[block]{\LARGE\sc\bfseries}{\thechapter.}{1ex}{#1}
%\titlespacing*{\chapter}{0cm}{0cm}{36pt}[0ex]
%\titleformat{\section}[block]{\Large\bfseries}{\thesection.}{1ex}{#1}
%\titlespacing*{\section}{0cm}{34.56pt}{17.28pt}[0ex]
%\titleformat{\subsection}[block]{\large\bfseries{\thesubsection.}{1ex}{#1}
%\titlespacing*{\subsection}{0pt}{28.80pt}{14.40pt}[0ex]
%%%%%%%%%%%%%%%%%%%%%%%%%%%%%%%%%%%%%%

%%%%%%%%% My Theorems %%%%%%%%%%%%%%%%%%
\newtheorem{thm}{Θεώρημα}[section]
\newtheorem{cor}[thm]{Πόρισμα}
\newtheorem{lem}[thm]{λήμμα}
\theoremstyle{definition}
\newtheorem{dfn}{Ορισμός}[section]
\newtheorem{dfns}[dfn]{Ορισμοί}
\theoremstyle{remark}
\newtheorem{remark}{Παρατήρηση}[section]
\newtheorem{remarks}[remark]{Παρατηρήσεις}
%%%%%%%%%%%%%%%%%%%%%%%%%%%%%%%%%%%%%%%




\input{definitions2.tex}
\input{tikz.tex}
\input{myboxes.tex}

\usepackage[cmtip,all]{xy}
\usepackage{silence}
\WarningsOff[catoptions]
\usepackage{extarrows}

\newcommand{\twocolumnsidelcc}[2]{\begin{minipage}[c]{0.26\linewidth}\raggedright
        #1
        \end{minipage}\hfill\begin{minipage}[c]{0.69\linewidth}\raggedright
        #2
    \end{minipage}
}

\geometry{left=10mm,right=10mm,top=30.00mm,bottom=32.00mm,footskip=24.16mm,headsep=24.16mm}
\everymath{\displaystyle}
\pagestyle{vangelis}

\setcounter{chapter}{1}


\begin{document}



\chapter*{Παράγωγος Σύνθετων Συναρτήσεων}

\section{1η Περίπτωση: \ensuremath{z=f(x,y),  x=x(t),  y=y(t)}} 

\begin{thm}
  Αν η συνάρτηση $ f(x,y) $ είναι ορισμένη στο ανοιχτό σύνολο 
  $ A \subseteq \mathbb{R}^{2} $ και $ x = x(t) $, $ y=y(t) $, με 
  $ t \in [a,b] $ και η $f$ έχει συνεχείς μερικές 
  παραγώγους στο $A$ και οι $ x(t) $ και $ y(t) $ είναι παραγωγίσιμες στο 
  $ [a,b] $, τότε η παράγωγος της σύνθετης συνάρτησης $f$ ως προς $t$ δίνεται από 
  τον τύπο:

  \vspace{\baselineskip}

  \twocolumnsideslc{
    \begin{center}
      \begin{tikzpicture}
        \node (f) at (0,0) {$f$};  
        \node (x) at (-1.5,-1.5) {$x$};  
        \node (y) at (1.5,-1.5) {$y$};  
        \node (t) at (0,-3) {$t$};  
        \draw (f) edge node[midway,above left] {$\pdv{f}{x}$} (x) ;
        \draw (f) edge node[midway,above right] {$\pdv{f}{y}$} (y) ;
        \draw (x) edge node[midway,below left] {$\dv{x}{t}$} (t) ;
        \draw (y) edge node[midway,below right] {$\dv{y}{t}$} (t) ;
      \end{tikzpicture}
    \end{center}
  }{
    \begin{equation}\label{eq:deriv1}
      \dv{f}{t} = \pdv{f}{x} \dv{x}{t} + \pdv{f}{y} \dv{y}{t} 
    \end{equation}
    Ενώ η 2η παράγωγος από τον τύπο:
    \[
      \dv[2]{f}{t} =  \pdv[2]{f}{x} \left(\dv{x}{t}\right)^{2} + 
      2 \pdv[2]{f}{x}{y} \dv{x}{t} \dv{y}{t} + \pdv[2]{f}{y} 
      \left(\dv{y}{t}\right)^{2} + \pdv{f}{x} \dv[2]{x}{t} + \pdv{f}{y} \dv[2]{y}{t}
    \]
  }
\end{thm}

\section{2η Περίπτωση: \ensuremath{z=f(x,y),  x=x(u,v),  y=y(u,v)}} 

\begin{thm}
  Αν η συνάρτηση $ f(x,y) $ είναι ορισμένη στο ανοιχτό σύνολο 
  $ A \subseteq \mathbb{R}^{2} $ και $ x = x(u,v) $, $ y=y(u,v) $, με 
  και η $f$ έχει συνεχείς μερικές παραγώγους στο $A$ και οι $ x $ και $ y $, έχουν 
  συνεχείς μερικές παραγώγους στο $ E \subseteq \mathbb{R}^{2} $,
  τότε οι μερικές παράγωγοι της $f$, υπάρχουν και δίνονται από τους τύπους:
\end{thm}

\twocolumnsideslc{ 
  \begin{center}
    \begin{tikzpicture}
      \node (f) at (0,0) {$f$};  
      \node (x) at (-1,-1.5) {$x$};  
      \node (y) at (1,-1.5) {$y$};  
      \node (ux) at (-2,-3) {$u$};  
      \node (vx) at (-0.5,-3) {$v$};  
      \node (uy) at (0.5,-3) {$u$};  
      \node (vy) at (2,-3) {$v$};  
      \draw (f) edge node[midway,above left] {$ \pdv{f}{x}$} (x) ;
      \draw (f) edge node[midway,above right] {$ \pdv{f}{y}$} (y) ;
      \draw (x) edge node[midway,above left] {$ \pdv{x}{u}$} (ux) ;
      \draw (x) edge node[midway,above right] {$ \pdv{x}{v}$} (vx) ;
      \draw (y) edge node[midway,above left] {$ \pdv{y}{u}$} (uy) ;
      \draw (y) edge node[midway,above right] {$ \pdv{y}{v}$} (vy) ;
    \end{tikzpicture}  
  \end{center}
}{
  \begin{equation}
    \label{eq:deriv2}
    \pdv{f}{u} = \pdv{f}{x} \pdv{x}{u} + \pdv{f}{y} \pdv{y}{u} 
    \quad \text{και} \quad
    \pdv{f}{v} = \pdv{f}{x} \pdv{x}{v} + \pdv{f}{y} \pdv{y}{v} 
  \end{equation}
  Ενώ οι μερικές παράγωγοι 2ης τάξης, δίνονται από τους τύπους:
  \[
    \pdv[2]{f}{u} =  \pdv[2]{f}{x} \left(\pdv{x}{u}\right)^{2} + 
    2 \pdv[2]{f}{x}{y} \pdv{x}{u} \pdv{y}{u} + \pdv[2]{f}{y} 
    \left(\pdv{y}{u}\right)^{2} + \pdv{f}{x} \pdv[2]{x}{u} + \pdv{f}{y} 
    \pdv[2]{y}{u}
  \]
  \[
    \pdv[2]{f}{v} =  \pdv[2]{f}{x} \left(\pdv{x}{v}\right)^{2} + 
    2 \pdv[2]{f}{x}{y} \pdv{x}{u} \pdv{y}{v} + \pdv[2]{f}{y} 
    \left(\pdv{y}{v}\right)^{2} + \pdv{f}{x} \pdv[2]{x}{v} + \pdv{f}{y} 
    \pdv[2]{y}{v}
  \]
  \[
    \pdv[2]{f}{v}{u} = \pdv[2]{f}{x} \pdv{x}{u} \pdv{x}{v} + \pdv[2]{f}{x}{y}
    \left(\pdv{x}{u} \pdv{y}{v}+ \pdv{x}{v} \pdv{y}{u} \right) + \pdv[2]{f}{y} 
    \pdv{y} {u} \pdv{y}{v} + \pdv{f}{x} \pdv[2]{x}{u}{v} + \pdv{f}{y} 
    \pdv[2]{y}{u}{v} 
  \]
}


\enlargethispage{\baselineskip}

\begin{rem}
  Οι τύποι~\eqref{eq:deriv1} και~\eqref{eq:deriv2} προέκυψαν αθροίζοντας κάθε φορά, 
  τα μονοπάτια που ξεκινούν από τη μεταβλητή $f$ και καταλήγουν στη μεταβλητή ως 
  προς την οποία παραγωγίζουμε, όπου κάθε μονοπάτι αποτελείται από το γινόμενο των 
  παραγώγων που συναντούμε "διασχίζοντάς" το.
\end{rem}

\pagebreak

\section{Αποδείξεις των τύπων των μερικών Παραγώγων 2ης τάξης}

\subsection{Απόδειξη με το συμβολισμό του Leibnitz}

Αποδεικνύουμε τον τύπο για την $ \pdv[2]{f}{u} $ και ομοίως προκύπτουν και οι τύποι 
για τις $ \pdv[2]{f}{v}  $ και $ \pdv[2]{f}{u}{v} $.

\vspace{\baselineskip}

\twocolumnsideslc
{
  \begin{center}
    \begin{tikzpicture}
      \node (f) at (0,0) {$ \pdv{f}{x} $};  
      \node (x) at (-1,-1.5) {$x$};  
      \node (y) at (1,-1.5) {$y$};  
      \node (ux) at (-2,-3) {$u$};  
      \node (vx) at (-0.5,-3) {$v$};  
      \node (uy) at (0.5,-3) {$u$};  
      \node (vy) at (2,-3) {$v$};  
      \draw (f) edge node[midway,above left] {$ \pdv[2]{f}{x}$} (x) ;
      \draw (f) edge node[midway,above right] {$ \pdv[2]{f}{y}{x}$} (y) ;
      \draw (x) edge node[midway,above left] {$ \pdv{x}{u}$} (ux) ;
      \draw (x) edge node[midway,above right] {$ \pdv{x}{v}$} (vx) ;
      \draw (y) edge node[midway,above left] {$ \pdv{y}{u}$} (uy) ;
      \draw (y) edge node[midway,above right] {$ \pdv{y}{v}$} (vy) ;
    \end{tikzpicture}  
  \end{center}

  \begin{center}
    \begin{tikzpicture}
      \node (f) at (0,0) {$ \pdv{f}{y} $};  
      \node (x) at (-1,-1.5) {$x$};  
      \node (y) at (1,-1.5) {$y$};  
      \node (ux) at (-2,-3) {$u$};  
      \node (vx) at (-0.5,-3) {$v$};  
      \node (uy) at (0.5,-3) {$u$};  
      \node (vy) at (2,-3) {$v$};  
      \draw (f) edge node[midway,above left] {$ \pdv[2]{f}{x}{y}$} (x) ;
      \draw (f) edge node[midway,above right] {$ \pdv[2]{f}{y}$} (y) ;
      \draw (x) edge node[midway,above left] {$ \pdv{x}{u}$} (ux) ;
      \draw (x) edge node[midway,above right] {$ \pdv{x}{v}$} (vx) ;
      \draw (y) edge node[midway,above left] {$ \pdv{y}{u}$} (uy) ;
      \draw (y) edge node[midway,above right] {$ \pdv{y}{v}$} (vy) ;
    \end{tikzpicture}  
  \end{center}
}{
  \begin{proof}
    \[
      \begin{aligned}
        \pdv[2]{f}{u} 
  &= \pdv{}{u}\left(\pdv{f}{u}\right) = \pdv{}{u} 
  \left( \pdv{f}{x} \pdv{x}{u} + \pdv{f}{y} \pdv{y}{u}\right) = 
  \pdv{}{u} \left(\pdv{f}{x} \pdv{x}{u}\right) + \pdv{}{u} 
  \left(\pdv{f}{y} \pdv{y}{u}\right) \\
  &= \pdv{}{u} \left( \pdv{f}{x}\right) \pdv{x}{u} + \pdv{f}{x} \pdv{}{u} \left(
  \pdv{x}{u} \right) + \pdv{}{u} \left(\pdv{f}{y} \right) \pdv{y}{u} + \pdv{f}{y} 
  \pdv{}{u} \left(\pdv{y}{u}\right) \\
  &=\left[ \pdv[2]{f}{x} \pdv{x}{u} + \pdv[2]{f}{x}{y} \pdv{y}{u} \right] \pdv{x}{u} +
  \pdv{f}{x} \pdv[2]{x}{u} + 
  \left[ \pdv[2]{f}{y}{x} \pdv{x}{u} + \pdv[2]{f}{y} \pdv{y}{u} \right] \pdv{y}{u} +
  \pdv{f}{y} \pdv[2]{y}{u} \\
  &= \pdv[2]{f}{x} \left(\pdv{x}{u}\right)^{2} + \pdv[2]{f}{x}{y} \pdv{y}{u}
  \pdv{x}{u} + \pdv{f}{x} \pdv[2]{x}{u} + \pdv[2]{f}{y}{x} \pdv{x}{u} \pdv{y}{u} + 
  \pdv[2]{f}{y} \left(\pdv{y}{u}\right)^{2} +
  \pdv{f}{y} \pdv[2]{y}{u} \\
  &= \pdv[2]{f}{x} \left(\pdv{x}{u}\right)^{2} + 2\pdv[2]{f}{x}{y} \pdv{x}{u}
  \pdv{y}{u} + \pdv[2]{f}{y} \left(\pdv{y}{u}\right)^{2} + \pdv{f}{x} \pdv[2]{x}{u} +
  \pdv{f}{y} \pdv[2]{y}{u} \\
      \end{aligned}
    \]
  \end{proof}
}


\subsection{Απόδειξη με το συμβολισμό των δεικτών}

Αποδεικνύουμε τον τύπο για την $ f_{uu} $ και ομοίως προκύπτουν και οι τύποι για τις  
$ f_{vv} $ και $ f_{uv} $.

\vspace{\baselineskip}

\twocolumnsideslc{
  \begin{center}
    \begin{tikzpicture}
      \node (f) at (0,0) {$f_{x}$};  
      \node (x) at (-1,-1.5) {$x$};  
      \node (y) at (1,-1.5) {$y$};  
      \node (ux) at (-2,-3) {$u$};  
      \node (vx) at (-0.5,-3) {$v$};  
      \node (uy) at (0.5,-3) {$u$};  
      \node (vy) at (2,-3) {$v$};  
      \draw (f) edge node[midway,above left] {$f_{xx}$} (x) ;
      \draw (f) edge node[midway,above right] {$f_{xy}$} (y) ;
      \draw (x) edge node[midway,above left] {$x_{u}$} (ux) ;
      \draw (x) edge node[midway,above right] {$x_{v}$} (vx) ;
      \draw (y) edge node[midway,above left] {$y_{u}$} (uy) ;
      \draw (y) edge node[midway,above right] {$y_{v}$} (vy) ;
    \end{tikzpicture}
  \end{center}

  \begin{center}
    \begin{tikzpicture}
      \node (f) at (0,0) {$f_{y}$};  
      \node (x) at (-1,-1.5) {$x$};  
      \node (y) at (1,-1.5) {$y$};  
      \node (ux) at (-2,-3) {$u$};  
      \node (vx) at (-0.5,-3) {$v$};  
      \node (uy) at (0.5,-3) {$u$};  
      \node (vy) at (2,-3) {$v$};  
      \draw (f) edge node[midway,above left] {$f_{yx}$} (x) ;
      \draw (f) edge node[midway,above right] {$f_{yy}$} (y) ;
      \draw (x) edge node[midway,above left] {$x_{u}$} (ux) ;
      \draw (x) edge node[midway,above right] {$x_{v}$} (vx) ;
      \draw (y) edge node[midway,above left] {$y_{u}$} (uy) ;
      \draw (y) edge node[midway,above right] {$y_{v}$} (vy) ;
    \end{tikzpicture}
  \end{center}
}{
  \begin{proof}
    \[
      \begin{aligned}
        f_{uu} &= (f_{u})_{u} = (f_{x}x_{u}+f_{y}y_{u})_{u} \\
               &=(f_{x}x_{u})_{u}+ (f_{y}y_{u})_{u} \\
               &=(f_{x})_{u}x_{u} + f_{x}(x_{u})_{u} + (f_{y})_{u}y_{u}+ 
               f_{y}(y_{u})_{u} \\
               &= (f_{xx}x_{u}+f_{xy}{y_{u}})x_{u} + f_{x} x_{uu} + 
               (f_{yx}x_{u}+f_{yy}y_{u})y_{u} + f_{y}y_{uu} \\
               &= f_{xx}(x_{u})^{2} + f_{xy}y_{u}x_{u}+ f_{x}x_{uu} + 
               f_{yx}x_{u}y_{u}+f_{yy}(y_{u})^{2}+ f_{y}y_{uu} \\
               &= f_{xx}(x_{u})^{2}+ 2f_{xy}x_{u}y_{u} + f_{yy}(y_{u})^{2} + 
               f_{x}x_{uu} + f_{y}y_{uu}
      \end{aligned}
    \] 
  \end{proof}
}

\begin{rem}
  Συγκεντρωτικά οι τύποι για τις παραγώγους 2ης τάξης της συνάρτησης 
  \[
    f_{uu}= f_{xx}(x_{u})^{2}+ 2f_{xy}x_{u}y_{u} + f_{yy}(y_{u})^{2} + f_{x}x_{uu} + 
    f_{y}y_{uu} 
  \] 
  \[
    f_{vv}= f_{xx}(x_{v})^{2}+ 2f_{xy}x_{v}y_{v} + f_{yy}(y_{v})^{2} + f_{x}x_{vv} + 
    f_{y}y_{vv} 
  \]
  \[
    f_{uv}= f_{xx}x_{u}x_{v}+ f_{xy}(x_{u}y_{v} + x_{v}y_{u}) + 
    f_{yy}y_{u}y_{v} + f_{x}x_{uv} + f_{y}y_{v} 
    f_{y}y_{vv} 
  \]
\end{rem}


\end{document}


