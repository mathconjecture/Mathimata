\documentclass[a4paper,12pt]{article}
\usepackage{etex}
%%%%%%%%%%%%%%%%%%%%%%%%%%%%%%%%%%%%%%
% Babel language package
\usepackage[english,greek]{babel}
% Inputenc font encoding
\usepackage[utf8]{inputenc}
%%%%%%%%%%%%%%%%%%%%%%%%%%%%%%%%%%%%%%

%%%%% math packages %%%%%%%%%%%%%%%%%%
\usepackage{amsmath}
\usepackage{amssymb}
\usepackage{amsfonts}
\usepackage{amsthm}
\usepackage{proof}

\usepackage{physics}

%%%%%%% symbols packages %%%%%%%%%%%%%%
\usepackage{bm} %for use \bm instead \boldsymbol in math mode 
\usepackage{dsfont}
\usepackage{stmaryrd}
%%%%%%%%%%%%%%%%%%%%%%%%%%%%%%%%%%%%%%%


%%%%%% graphicx %%%%%%%%%%%%%%%%%%%%%%%
\usepackage{graphicx}
\usepackage{color}
%\usepackage{xypic}
\usepackage[all]{xy}
\usepackage{calc}
\usepackage{booktabs}
\usepackage{minibox}
%%%%%%%%%%%%%%%%%%%%%%%%%%%%%%%%%%%%%%%

\usepackage{enumerate}

\usepackage{fancyhdr}
%%%%% header and footer rule %%%%%%%%%
\setlength{\headheight}{14pt}
\renewcommand{\headrulewidth}{0pt}
\renewcommand{\footrulewidth}{0pt}
\fancypagestyle{plain}{\fancyhf{}
\fancyhead{}
\lfoot{}
\rfoot{\small \thepage}}
\fancypagestyle{vangelis}{\fancyhf{}
\rhead{\small \leftmark}
\lhead{\small }
\lfoot{}
\rfoot{\small \thepage}}
%%%%%%%%%%%%%%%%%%%%%%%%%%%%%%%%%%%%%%%

\usepackage{hyperref}
\usepackage{url}
%%%%%%% hyperref settings %%%%%%%%%%%%
\hypersetup{pdfpagemode=UseOutlines,hidelinks,
bookmarksopen=true,
pdfdisplaydoctitle=true,
pdfstartview=Fit,
unicode=true,
pdfpagelayout=OneColumn,
}
%%%%%%%%%%%%%%%%%%%%%%%%%%%%%%%%%%%%%%

\usepackage[space]{grffile}

\usepackage{geometry}
\geometry{left=25.63mm,right=25.63mm,top=36.25mm,bottom=36.25mm,footskip=24.16mm,headsep=24.16mm}

%\usepackage[explicit]{titlesec}
%%%%%% titlesec settings %%%%%%%%%%%%%
%\titleformat{\chapter}[block]{\LARGE\sc\bfseries}{\thechapter.}{1ex}{#1}
%\titlespacing*{\chapter}{0cm}{0cm}{36pt}[0ex]
%\titleformat{\section}[block]{\Large\bfseries}{\thesection.}{1ex}{#1}
%\titlespacing*{\section}{0cm}{34.56pt}{17.28pt}[0ex]
%\titleformat{\subsection}[block]{\large\bfseries{\thesubsection.}{1ex}{#1}
%\titlespacing*{\subsection}{0pt}{28.80pt}{14.40pt}[0ex]
%%%%%%%%%%%%%%%%%%%%%%%%%%%%%%%%%%%%%%

%%%%%%%%% My Theorems %%%%%%%%%%%%%%%%%%
\newtheorem{thm}{Θεώρημα}[section]
\newtheorem{cor}[thm]{Πόρισμα}
\newtheorem{lem}[thm]{λήμμα}
\theoremstyle{definition}
\newtheorem{dfn}{Ορισμός}[section]
\newtheorem{dfns}[dfn]{Ορισμοί}
\theoremstyle{remark}
\newtheorem{remark}{Παρατήρηση}[section]
\newtheorem{remarks}[remark]{Παρατηρήσεις}
%%%%%%%%%%%%%%%%%%%%%%%%%%%%%%%%%%%%%%%




\input{definitions_ask.tex}

\pagestyle{askhseis}

\begin{document}

\begin{center}
  \minibox{\bfseries\large\color{Col2} Μερική Παράγωγος}
\end{center} 

\vspace{\baselineskip} 

\section*{Ορισμός}

\begin{enumerate}
  \item Έστω η συνάρτηση $ f(x,y) = xy $. Να υπολογιστούν με τη 
    βοήθεια του ορισμού οι $ f_{x}(1,1) $ και η $ f_{y}(1,1) $. 

    \hfill Απ: 
    \begin{tabular}{l}
      $f_{x}(1,1) = 1$ \\
      $f_{y}(1,1) = 1$
    \end{tabular} 

  \item Να υπολογιστούν οι μερικές παράγωγοι 1ης τάξης της συνάρτησης
    $
    f(x,y) = 
    \begin{cases}
      x^{2} \sin{\frac{y}{x}}, & x \neq 0 \\
      0, & x = 0 
    \end{cases}
    $ 

    \hfill Απ: 
    $  f_{x} = 
    \begin{cases}
      2x \sin{\frac{y}{x}} - y \cos{\frac{y}{x}}, & 
      x \neq 0  \\ 
      0, & x = 0 
    \end{cases} $ \quad 
    και \quad
    $ f_{y} = 
    \begin{cases}
      x \cos{\frac{y}{x}}, & x \neq 0 \\
      0, & x = 0 
    \end{cases} $    

\end{enumerate}


\section*{Κανόνες Παραγώγισης}

\begin{enumerate}
  \item Με τη βοήθεια των κανόνων παραγώγισης να υπολογιστούν οι μερικές 
    παράγωγοι 1ης τάξης των παρακάτω συναρτήσεων:

    \begin{enumerate}[i)]
      \item $f(x,y)=y\sin (xy)$ \hfill Απ: \begin{tabular}{l}
          $f_x=y^2\cos(xy)$ \\ 
          $f_y=\sin(xy)+yx\cos(xy)$
        \end{tabular}

      \item $f(x,y)=\ln(x+y)$\hfill Απ: \begin{tabular}{l}
          $f_x=\frac{y}{x+y}$ \\ 
          $f_y=\ln(x+y)+\frac{y}{x+y}$
        \end{tabular}

      \item $f(x,y)=\arcsin(\frac{x}{y})$\hfill Απ: \begin{tabular}{l}
          $f_x=\frac{1}{y^2-x^2}$ \\ 
          $f_y=-\frac{x}{y\sqrt{y^2-x^2}}$
        \end{tabular}
      \item $ f(x,y,z) = (x+y^{2}) \sin{(xz)} $ \hfill Απ: \begin{tabular}{l}
          $ f_{x} = \sin{(xz)} + z(x+y^{2}) \sin{(xz)} $ \\
          $ f_{y} = 2y \sin{(xz)} $ \\
          $ f_{z} = x(x+y^{2}) \sin{(xz)} $
        \end{tabular} 
    \end{enumerate}

  \item Έστω η συνάρτηση $f(x,y)=\ln\left(\cos y+x\cos x\right)$.  Να υπολογισθεί 
    η $ f_{xy} $ στο σημείο $\left(\pi,-\frac{\pi}{2}\right)$.
    \hfill Απ: $\frac{1}{\pi^2}$

  \item Έστω η συνάρτηση $ f(x,y) = x^{y} $. Να δείξετε ότι ισχύει ότι το
    \textbf{θεώρημα Schwartz}, δηλαδή ότι $ f_{xy} = f_{yx} $.
    %hlektr
  \item Να δείξετε ότι η συνάρτηση $ f(x,y) = (y+3x)^{1/2} - 
    (y-3x)^{2} $ ικανοποιεί τη σχέση $ f_{xx} - 9 f_{yy} = 0 $.
    %span
  \item Να δείξετε ότι η συνάρτηση $ f(x,y) = \cos{(x+y)} + \cos{(x-y)} $ 
    επαληθεύει την διαφορική εξίσωση $ z_{xx} - z_{yy} = 0 $.

  \item Να δείξετε ότι η συνάρτηση $ f(x,y) = x \arctan{\frac{y}{x}} $ 
    ικανοποιεί την διαφορική εξίσωση $ x^{2} f_{xx} + 2xyf_{xy} + y^{2} f_{yy} = 0 $ 
\end{enumerate}

\section*{Διαφορισιμότητα}

\begin{enumerate}
  \item Να εξεταστεί αν οι παρακάτω συναρτήσεις είναι διαφορίσιμες στο σημείο 
    $ (0,0) $.

    \begin{enumerate*}[i),itemjoin=\hspace{1.5cm}]
      \item  $ f(x,y) = 
        \begin{cases} 
          \frac{xy}{\sqrt{x^{2}+y^{2}}}, & (x,y) \neq (0,0) \\
          0, & (x,y) = (0,0)
        \end{cases} $ 
      \item $ g(x,y) = 
        \begin{cases} 
          \frac{(x+y)^{2}}{x^{2}+y^{2}}, & (x,y) \neq (0,0) \\
          0, & (x,y) = (0,0)
        \end{cases} $
    \end{enumerate*}

    \hfill Απ: 
    \begin{enumerate*}[i),itemjoin=\hspace{10pt}]
      \item όχι
      \item όχι
    \end{enumerate*}

  \item Να εξετάσετε αν η συνάρτηση 
    $
    f(x,y) = 
    \begin{cases} 
      xy \frac{x^{2}-y^{2}}{x^{2}+y^{2}}, & (x,y) 
      \neq (0,0) \\ 
      0, & (x,y) = (0,0)
    \end{cases}
    $ 
    είναι διαφορίσιμη στο σημείο $ (0,0) $. 
    \hfill Απ: ναι 
\end{enumerate}

\section*{Αρμονικές Συναρτήσεις}

\begin{enumerate}

  \item Να δείξετε ότι οι παρακάτω συναρτήσεις είναι αρμονικές:
    \begin{enumerate}[(i)]
      \item $ f(x,y) = x^{3}-3xy^{2} $
      \item $ f(x,y) = \ln(x^{2} + y^{2}) $
      \item $ f(x,y) = \ln{(x^{2}+y^{2})} + \arctan(\frac{y}{x}) $
      \item $ f(x,y) = \frac{ x }{ 2 } \ln(x^{2} + y^{2}) - y 
        \arctan(\frac{ y }{ x } ) $
    \end{enumerate}

  \item Να αποδείξετε ότι αν μία συνάρτηση $f(x,y)$, που έχει συνεχείς 
    μερικές παραγώγους 2ης τάξης είναι αρμονική, τότε και οι συναρτήσεις 
    \begin{enumerate*}[i),itemjoin=\hspace{7pt}]
      \item $ f_{x} $ 
      \item $ f_{y} $
      \item $ xf_{x}+yf_{y} $
      \item $ xf_{x}-yf_{y} $
    \end{enumerate*}
    είναι αρμονικές.
\end{enumerate}

\section*{Ομογενείς Συναρτήσεις}

\begin{enumerate}
  \item Αν $ u = u(x,y) $ και $ v=v(x,y) $ ομογενείς βαθμού $ \rho $, 
    τότε να δείξετε ότι $ \forall f(u,v) $ με συνεχείς μερικές παραγώγους 1ης τάξης
    ισχύει 
    \[
      xf_{x}+yf_{y}= \rho (u f_{u}+vf_{v}) 
    \] 
  \item Να αποδείξετε ότι αν $f(x,y)$ είναι ομογενής συνάρτηση, βαθμού $ \rho $, τότε 
    οι συναρτήσεις $ f_{x}, f_{y} $ είναι επίσης ομογενείς, βαθμού $ \rho -1 $.
\end{enumerate}

\section*{Διαφορικό}

\begin{enumerate}

  \item Να βρεθεί το ολικό διαφορικό 1ης τάξης, της συνάρτησης 
    $f(x,y)=\ln(xy)+\cos(y^2)$ 

    \hfill Απ: $df=\frac{dx}{x}+\left(\frac{1}{y}-2y\sin(y^2)\right)dy$

  \item Να βρεθεί το ολικό διαφορικό 1ης τάξης, της συνάρτησης 
    $ f(x,y) = \arctan(\frac{ x+y }{ x-y }) $, αν $ x>0 $ και $ y>0 $.

    \hfill Απ: $df = \frac{ -ydx + xdy }{ x^{2} + y^{2} } $ 

  \item Να βρεθεί το ολικό διαφορικό της συνάρτησης $ f(x,y) = x^{y} \cdot y^{x} $, 
    αν $ x>0$ και $ y>0 $.

    \hfill Απ: $df =  (x^{y-1}\cdot y^{x+1} + x^{y}\cdot y^{x} \ln{y} )dx 
    + (x^{y}\cdot y^{x} \ln{x} + x^{y+1} \cdot y^{x-1})dy $ 

  \item Για τις παρακάτω παραστάσεις, να αποδείξετε ότι είναι \textbf{τέλεια
    διαφορικά} και να υπολογίσετε τη συνάρτηση δυναμικού.
    \begin{enumerate}[i)]
      \item $ \left(x+e^{x/y}\right)dx + e^{x/y}\left(1- \frac{x}{y}\right)dy $
        \hfill Απ: $ f(x,y) = \frac{x^{2}}{2} +y e^{x/y} + c $ 

      \item $\left(2e^{x}+\frac{1}{x}-3\sin y\right)dx+3(y^2-x\cos y)dy$ 
        \hfill  Απ: $ f(x,y,z) = y^{3}-3x \sin{y} + 2e^{x} + \ln{x} +c $.
        %spand (114)

      \item $(2xy+z)dx+(x^{2}+z^{2})dy+(x+2yz)dz$ 
        \hfill  Απ: $ f(x,y,z) = x^{2}+z^{2}+xz +c $.

        % \item $ (3x^{2}+3y-1)dx + (z^{2}+3x)dy+(2yz+1)dz $
        % \hfill Απ: $ f(x,y,z) = x^{3}+3xy-x+yz^{2}+z+c $

      \item $ \cos(x+yz)dx + z\cos(x+yz)dy+y\cos(x+yz)dz $
        \hfill Απ: $ f(x,y,z) = \sin(x+yz) + c $
    \end{enumerate}

  \item Να υπολογιστεί το $a$ ώστε η παράσταση $ \frac{ x + ay }{ (x-y)^{3} }dx 
    + \frac{ ax+y }{ (x-y)^{3} }dy $ να είναι \textbf{τέλειο διαφορικό}.

    \hfill Απ: $ a=-1 $
\end{enumerate}

\section*{Εφαρμογές του Διαφορικού}

\begin{enumerate}
  \item Να υπολογιστεί κατά προσέγγιση η τιμή των παραστάσεων
    \begin{enumerate}[i)]
      \item $A = (1,02)^{3,01} $ \hfill Απ: $ A \approx 1,06 $ 
      \item $B =  \sqrt{ 9(1,95)^{2} + (8,1)^{2} } $ 
        \hfill Απ: $ B \approx 9,99 $ 
    \end{enumerate}

  \item Θέλουμε να κατασκευάσουμε ένα ορθογώνιο παραλληλεπίπεδο με ακμές $ x = a $, 
    $ y=b $ και $ z=c $. Αν τα σφάλματα που έγιναν κατά την κατασκευή των ακμών 
    είναι $ \Delta x = 0,01a $, $ \Delta y = -0,02b $ και $ \Delta z = 0,03c $, 
    τότε να υπολογίσετε το σφάλμα που έγινε στον όγκο του παραλληλεπιπέδου.

    \hfill Απ: $ 0,02abc $ 
\end{enumerate}


\section*{Πολυώνυμο Taylor}

\begin{enumerate}
  \item Να βρεθούν τα αναπτύγματα \textbf{Taylor}, μέχρι και όρους 
    \textbf{2ης τάξης}, των συναρτήσεων:

    \begin{enumerate}[i)]
      \item  $f(x,y)=y\cos{xy} $, γύρω από το σημείο 
        $ \left(1, \frac{ \pi }{ 2 }\right) $.

        \hfill Απ: $f(x,y)=-\frac{\pi^{2}}{4}(x-1) - \frac{ \pi }{ 2 } 
        \left(y - \frac{ \pi }{2 }\right) - \pi(x-1)
        \left(y-\frac{\pi}{2}\right)- \left(y- \frac{ \pi }{ 2} \right)^{2} $

      \item $ f(x,y)=e^{x}\tan{y} $ σε δυνάμεις των $ (x-1) $ και 
        $ \left(y - \frac{ \pi }{ 4 }\right) $

        \hfill Απ: $ f(x,y) = e + e(x-1) + 2e\left(y- \frac{ \pi }{ 4 }\right)
        + \frac{1}{ 2 } \left(e(x-1)^{2}+4e(x-1)\left(y- \frac{ \pi }{ 4 }
        \right) + 4e\left(y- \frac{ \pi }{ 4 } \right)^{2}\right) $
    \end{enumerate}

  \item Να βρεθεί το ανάπτυγμα \textbf{Maclaurin}, μέχρι όρους \textbf{2ης
    τάξης}, της συνάρτησης $ f(x,y) = e^{x}\ln(1+y)$.

    \hfill Απ: $ f(x,y)=y + xy - \frac{1}{ 2 } y^{2} + \frac{1}{ 2 } x^{2}y - 
    \frac{1}{ 2 } xy^{2} + \frac{1}{ 3 } y^{3} $
\end{enumerate}


\section*{Ακρότατα}

\begin{enumerate}
  \item Να βρεθούν και να χαρακτηριστούν τα κρίσιμα σημεία  των παρακάτω συναρτήσεων:
    \begin{enumerate}[i)]
      \item $ f(x,y) = x^{3} + y^{3} + 3xy $ 
        \hfill Απ: max: $(-1,-1)  $, σάγμα: $ (0,0) $
      \item $ f(x,y) = x^{2}+y^{4} $ 
        \hfill Απ: min: $ (0,0) $ 
      \item $ f(x,y) = x^{3} + y^{3} - 3x -12y + 50 $ 
        \hfill Απ: max: $ (-1,-2)$, min: $ (1,2) $, 
        σάγμα: $ (1,-2), (-1,2) $
      \item $ f(x,y) = x^{3} + y^{3} -3x -3y + 1 $ 
        \hfill Απ: max: $(-1,-1)  $, min: $ (1,1) $
      \item $ f(x,y) = x^{3} + 4xy -4y^{2} $ 
        \hfill Απ: max: $ (-2/3, -1/3)  $
      \item $ f(x,y) = x^{4} + y^{4} -2(x-y)^{2}$  
        \hfill Απ: min: $ (\sqrt{2} , -\sqrt{2}), (-\sqrt{2} , \sqrt{2}) $, 
        σάγμα: $ (0,0) $
      \item $ f(x,y) = (x^{2}-3y^{2})e^{1-x^{2}-y^{2}} $ 
        \hfill Απ: max: $ (1,0)$, min: $ (0,1), (0,-1) $, 
        σάγμα: $ (0,0) $
    \end{enumerate}

  \item Να βρεθούν και να χαρακτηριστούν τα κρίσιμα σημεία της συνάρτησης 
    $ f(x,y,z) = x^{3} + y^{3}+z^{3} + 3xy +3yz + 3xz $
    \hfill Απ: σάγμα: $(0,0,0)$, max: $(-2,-2,-2)$  

  \item Δίνεται τρίγωνο ΑΒΓ. Να βρεθεί σημείο P, στο επίπεδο του τριγώνου, ώστε 
    το άθροισμα των τετραγώνων των αποστάσεών του από τις κορυφές του τριγώνου 
    να είναι ελάχιστο.

  \item Να βρεθεί η ελάχιστη απόσταση της αρχής του συστήματος συντεταγμένων από το 
    επίπεδο με εξίσωση $ x+y+z=4 $. 
    \hfill Απ: $ d_{\min}(4/3,4/3) = 4\frac{\sqrt{3}}{3} $  

  \item Να βρεθεί η απόσταση του σημείου $ P(1,0,-2) $ από το επίπεδο με εξίσωση 
    $ x+2y+z=4 $. 
    
    \hfill Απ: $ d_{min}(11/6,5/3) = 5 \sqrt{6} /6 $ 
\end{enumerate}


\section*{Δεσμευμένα Ακρότατα}

\begin{enumerate}
  \item Να υπολογιστούν τα τοπικά ακρότατα της συνάρτησης $ f(x,y,z) = x^{2}+y^{2}+z^{2}
    $ που ικανοποιούν τον περιορισμό $ x+y+z+1=0 $.
    \hfill Απ: min: $ (-1/3,-1/3,-1/3) $ 

  \item Να υπολογιστούν τα τοπικά ακρότατα της συνάρτησης 
    $ f(x,y,z) = x^{2}+y^{2}+z^{2}-2x-2y-z+ \frac{5}{4} $ που ικανοποιούν τον 
    περιορισμό $ x^{2}+y^{2}-z=0  $.
    \hfill Απ: min: $ (1/ \sqrt[3]{4} , 1/ \sqrt[3]{4}) $ 

  \item Να υπολογιστούν τα τοπικά ακρότατα της συνάρτησης 
    $f(x,y,z)=xyz$ που ικανοποιούν την εξίσωση $x+y+z-1=0$.  
    \hfill Απ: max: $ (1/3,1/3,1/3) $ 

  \item Να υπολογιστούν τα τοπικά ακρότατα της συνάρτησης 
    $ f(x,y,z) = x^{2}+y^{2}+z^{2} $, που ικανοποιούν τους περιορισμούς 
    $ zx+zy=-2 $ και $ xy=1 $.
    \hfill Απ: min: $ (-1,-1,1) $, max: $ (1,1,-1) $ 
\end{enumerate}


\section*{Ολικά Ακρότατα}

\begin{enumerate}
  \item Να υπολογιστούν τα ολικά ακρότατα της συνάρτησης $ f(x,y) = \sqrt{x^{2}+y^{2}} +
    y^{2}-1 $, στον κυκλικό δίσκο $ x^{2}+y^{2} \leq 9 $. 
    \hfill Απ: min: $ (3,0), (-3,0) $, max: $ (0,3), (0,-3) $ 
\end{enumerate}





\end{document}
