\input{preamble.tex}
\newcommand{\vect}[2]{(#1_1,\ldots, #1_#2)}
%%%%%%% nesting newcommands $$$$$$$$$$$$$$$$$$$
\newcommand{\function}[1]{\newcommand{\nvec}[2]{#1(##1_1,\ldots, ##1_##2)}}

\newcommand{\linode}[2]{#1_n(x)#2^{(n)}+#1_{n-1}(x)#2^{(n-1)}+\cdots +#1_0(x)#2=g(x)}

\newcommand{\vecoffun}[3]{#1_0(#2),\ldots ,#1_#3(#2)}

\newcommand{\suma}{\sum_{n=0}^{\infty}a_n x^n}

\newcommand{\sumb}{\sum_{n=1}^{\infty}a_n n x^{n-1}}

\newcommand{\sumc}{\sum_{n=2}^{\infty}a_n n (n-1) x^{n-2}}

\newcommand{\varsum}[2]{\sum_{n=#1}^{#2}}


\pagestyle{askhseis}
\everymath{\displaystyle}
% \geometry{top=2cm}

\begin{document}

\begin{center}
  \minibox{\large\bfseries \textcolor{Col1}{Ασκήσεις Ολοκληρώματα}}
\end{center}

\vspace{\baselineskip} 


\section*{Αρμονικές Συναρτήσεις}

\begin{enumerate}

  \item Δίνεται η συνάρτηση $ u(x,y) = x^{2} +4x-y^{2}+2y $. 
    \begin{enumerate}[i)]
      \item Να δείξετε ότι η συνάρτηση $ u $ είναι αρμονική.
      \item Να βρεθεί η συνάρτηση $ v(x,y) $ τέτοια ώστε η μιγαδική συνάρτηση 
        $ f(z) = u(x,y) + iv(x,y) $ να είναι αναλυτική. 
      \item Να γραφεί η αντίστοιχη συνάρτηση $ f(z) $.
    \end{enumerate}

    \hfill Απ: $ f(z) = z^{2}+4z-2iz+a $ 

  \item Δίνεται η συνάρτηση $ u(x,y) = \ln{\sqrt{x^{2}+y^{2}}} $ που είναι ορισμένη 
    στο σύνολο $ \mathbb{R}^{2}- \{ 0,0 \} $. 
    \begin{enumerate}[i)]
      \item Να δείξετε ότι η συνάρτηση $ u $ είναι αρμονική.
      \item Να βρεθεί μια συζυγής αρμονική συνάρτηση της $u$.
      \item Να γραφεί η αντίστοιχη ολόμορφη συνάρτηση.
    \end{enumerate}

    \hfill Απ: $ f(z) = \ln{\sqrt{x^{2}+y^{2}}} + i(\arctan{\frac{y}{x}} + c)  $ 

  \item Δίνεται η συνάρτηση $ u(x,y) = 2x-x^{3}+axy^{2} $. 
    \begin{enumerate}[i)]
      \item Να βρεθεί για ποια τιμή του $a$, η συνάρτηση είναι αρμονική. 
      \item Για αυτήν την τιμή του $a$ να βρεθεί η συζυγής αρμονική συνάρτηση της $u$.
    \end{enumerate} 

    \hfill Απ: $ a=3 $, $ v(x,y) = 2y-3x^{2}y+y^{3}+c $ 

  \item Να εξεταστούν με χρήση πολικών συντεταγμένων, αν οι παρακάτω συναρτήσεις είναι
    αναλυτικές. 
    \begin{enumerate}[i)]
      \item $ f(z)=z^{3}+z $ \hfill Απ: ναι 
      \item $ f(z) = r + i \phi $ \hfill Απ: όχι 
    \end{enumerate} 
\end{enumerate}


\section*{Ολοκληρώματα}

\begin{enumerate}
\item Να υπολογιστούν τα παρακάτω ολοκληρώματα.

  \begin{enumerate}[i)]
    % spand ex 8 p.151
    \item $ \int \limits_c\frac{e^{z}}{z-2} \,{dz} $, \quad όπου 
      \begin{enumerate*}[i),itemjoin=\hspace{\baselineskip}]
        \item $ c: \abs{z} = 3 $
        \item $ c: \abs{z} = 1 $. 
      \end{enumerate*}
      \hfill Απ: 
      \begin{enumerate*}[i),itemjoin=\hspace{\baselineskip}]
        \item $ 2 \pi i e^{2} $ 
        \item $0$ 
      \end{enumerate*}
      % spand ex 9 p.153
    \item $ \int\limits_c \frac{1}{z^{2}+1} \,{dz} $, όπου $ c:\abs{z} = 2 $.
      \hfill Απ: 0 
      % spand ex 8 p.278
    \item $ \int \limits_{c}\frac{1}{z^{3}(z+4)} \,{dz} $, \quad όπου $c:\abs{z}=2 $ 
      \hfill Απ: $ \frac{2 \pi i}{4^{3}} $  
      % spand ex 1 p.145
    \item $\int\limits_c\frac{5z^2-2}{z(e^z-1)}\,dz$, \quad όπου $c:\abs{z}=2$ 
      \hfill Απ: $2\pi i$
      % spand ex 4 p.254 
    \item $\int\limits_c\tan z\,dz$, \quad όπου $c:\abs{z}=2$ \hfill Απ: $-4\pi i$
      % χρειάζεται να έχω πει ολοκληρωτικό τύπο Cauchy αλλιως παραγωγίσεις δυσκολες
      % \item $\int\limits_c\frac{2z-3}{z(z-1)^2(z^2+4)}\,dz$, \quad οπου $c:
      % \abs{z}=3$   \hfill Απ: 0
  \end{enumerate}

\pagebreak

\item Να υπολογιστούν τα παρακάτω ολοκληρώματα.

  \begin{enumerate}[i)]
    \item $\int\limits_0^{2\pi}\frac{1}{2+\cos x}\,dx$ \hfill Απ:$\frac{2
      \pi}{\sqrt{3}}$
    \item $\int\limits_0^{2\pi}\frac{1}{(5-3\sin x)^2}\,dx$ \hfill Απ: $\frac{5}{32}\pi$
    \item $\int\limits_0^{2\pi}\frac{\cos 3x}{5-4\cos x}\,dx$ 
      \hfill Απ: $\frac{\pi}{12}$
    \item $\int\limits_0^{2\pi}\frac{1}{3-2\cos x+\sin x}\,dx$ \hfill Απ: $\pi$
      % \item $\int\limits_0^{2\pi}\frac{1}{1-2a\cos x+a^2}\,dx$ 
      %   \hfill Απ: \begin{tabular}{c}$\abs{a}<1\Rightarrow I=\frac{2\pi}{1-a^2}$ \\ 
      %   $\abs{a}>1\Rightarrow I=\frac{2\pi}{a^2-1}$\end{tabular}
  \end{enumerate}

\item Να υπολογιστούν τα παρακάτω ολοκληρώματα.

  \begin{enumerate}[i)]
    \item $\int_{-\infty}^{+\infty}\frac{x\sin x}{1+x^2}\,dx$ 
      \hfill  Απ: $\frac{\pi}{e}$
    \item $\int_{-\infty}^{+\infty}\frac{\cos 2x}{(1+x^2)^2}\,dx$ 
      \hfill  Απ: $\frac{3\pi}{2e^2}$
  \end{enumerate}
\end{enumerate}

\end{document}

