\documentclass[a4paper,12pt]{article}
\usepackage{etex}
%%%%%%%%%%%%%%%%%%%%%%%%%%%%%%%%%%%%%%
% Babel language package
\usepackage[english,greek]{babel}
% Inputenc font encoding
\usepackage[utf8]{inputenc}
%%%%%%%%%%%%%%%%%%%%%%%%%%%%%%%%%%%%%%

%%%%% math packages %%%%%%%%%%%%%%%%%%
\usepackage{amsmath}
\usepackage{amssymb}
\usepackage{amsfonts}
\usepackage{amsthm}
\usepackage{proof}

\usepackage{physics}

%%%%%%% symbols packages %%%%%%%%%%%%%%
\usepackage{bm} %for use \bm instead \boldsymbol in math mode 
\usepackage{dsfont}
\usepackage{stmaryrd}
%%%%%%%%%%%%%%%%%%%%%%%%%%%%%%%%%%%%%%%


%%%%%% graphicx %%%%%%%%%%%%%%%%%%%%%%%
\usepackage{graphicx}
\usepackage{color}
%\usepackage{xypic}
\usepackage[all]{xy}
\usepackage{calc}
\usepackage{booktabs}
\usepackage{minibox}
%%%%%%%%%%%%%%%%%%%%%%%%%%%%%%%%%%%%%%%

\usepackage{enumerate}

\usepackage{fancyhdr}
%%%%% header and footer rule %%%%%%%%%
\setlength{\headheight}{14pt}
\renewcommand{\headrulewidth}{0pt}
\renewcommand{\footrulewidth}{0pt}
\fancypagestyle{plain}{\fancyhf{}
\fancyhead{}
\lfoot{}
\rfoot{\small \thepage}}
\fancypagestyle{vangelis}{\fancyhf{}
\rhead{\small \leftmark}
\lhead{\small }
\lfoot{}
\rfoot{\small \thepage}}
%%%%%%%%%%%%%%%%%%%%%%%%%%%%%%%%%%%%%%%

\usepackage{hyperref}
\usepackage{url}
%%%%%%% hyperref settings %%%%%%%%%%%%
\hypersetup{pdfpagemode=UseOutlines,hidelinks,
bookmarksopen=true,
pdfdisplaydoctitle=true,
pdfstartview=Fit,
unicode=true,
pdfpagelayout=OneColumn,
}
%%%%%%%%%%%%%%%%%%%%%%%%%%%%%%%%%%%%%%

\usepackage[space]{grffile}

\usepackage{geometry}
\geometry{left=25.63mm,right=25.63mm,top=36.25mm,bottom=36.25mm,footskip=24.16mm,headsep=24.16mm}

%\usepackage[explicit]{titlesec}
%%%%%% titlesec settings %%%%%%%%%%%%%
%\titleformat{\chapter}[block]{\LARGE\sc\bfseries}{\thechapter.}{1ex}{#1}
%\titlespacing*{\chapter}{0cm}{0cm}{36pt}[0ex]
%\titleformat{\section}[block]{\Large\bfseries}{\thesection.}{1ex}{#1}
%\titlespacing*{\section}{0cm}{34.56pt}{17.28pt}[0ex]
%\titleformat{\subsection}[block]{\large\bfseries{\thesubsection.}{1ex}{#1}
%\titlespacing*{\subsection}{0pt}{28.80pt}{14.40pt}[0ex]
%%%%%%%%%%%%%%%%%%%%%%%%%%%%%%%%%%%%%%

%%%%%%%%% My Theorems %%%%%%%%%%%%%%%%%%
\newtheorem{thm}{Θεώρημα}[section]
\newtheorem{cor}[thm]{Πόρισμα}
\newtheorem{lem}[thm]{λήμμα}
\theoremstyle{definition}
\newtheorem{dfn}{Ορισμός}[section]
\newtheorem{dfns}[dfn]{Ορισμοί}
\theoremstyle{remark}
\newtheorem{remark}{Παρατήρηση}[section]
\newtheorem{remarks}[remark]{Παρατηρήσεις}
%%%%%%%%%%%%%%%%%%%%%%%%%%%%%%%%%%%%%%%




\newcommand{\vect}[2]{(#1_1,\ldots, #1_#2)}
%%%%%%% nesting newcommands $$$$$$$$$$$$$$$$$$$
\newcommand{\function}[1]{\newcommand{\nvec}[2]{#1(##1_1,\ldots, ##1_##2)}}

\newcommand{\linode}[2]{#1_n(x)#2^{(n)}+#1_{n-1}(x)#2^{(n-1)}+\cdots +#1_0(x)#2=g(x)}

\newcommand{\vecoffun}[3]{#1_0(#2),\ldots ,#1_#3(#2)}

\newcommand{\mysum}[1]{\sum_{n=#1}^{\infty}


\usepackage{systeme}

\pagestyle{askhseis}


\setlength{\itemsep}{\baselineskip}

\begin{document}


\section*{Γραμμικά Συστήματα}

\begin{enumerate}

  \item Να λυθούν τα παρακάτω γραμμικά συστήματα $ 2 \times 2 $ με αντικατάσταση ή με 
    μέθοδο αντίθετων συντελεστών.

    \begin{enumerate}[i)]
      \item $ 
      \sysdelim.\}
      \systeme{
        x + 2y = 5, 
        4x + y = 6
      } $ 
      \hfill Απ: 
      \begin{tabular}{l}  
        $x=1 $ \\ 
        $y=2 $ 
      \end{tabular}

    \item $ 
    \sysdelim.\}
    \systeme{
      4x+3y=11,
      5x+7y=17
    } $ 
    \hfill Απ: 
    \begin{tabular}{l}  
      $x=2 $ \\ 
      $y=1 $ 
    \end{tabular}
\end{enumerate}

  \item Να λυθούν τα παρακάτω γραμμικά συστήματα με τη μέθοδο Crammer.

    \begin{enumerate}[i)]
      \item $ 
      \sysdelim.\}
      \systeme{
        x+y+z=0,
        x-2y-2z=-3,
        2x+y+z=-1
      } $ 
      \hfill Απ: \begin{tabular}{l}  
        $x=-1 $ \\ 
        $ y=1-z $ \\
        $z \in \mathbb{R}  $
      \end{tabular}

    \item $ 
    \sysdelim.\}
    \systeme{
      x+y-3z=-1,
      4x-2y+6z=8,
      15x-3y+9z=21
    } $ 
    \hfill Απ: \begin{tabular}{l}  
      $x=1 $ \\ 
      $ y=-2+3z $ \\
      $z \in \mathbb{R}  $
    \end{tabular}

  \item $ 
  \sysdelim.\}
  \systeme{
    x+2y+z=2,
    x+y+2z=-1,
    x+3y=6
  } $ 
  \hfill Απ: αδύνατο 

\item $ 
\sysdelim.\}
\systeme{
  -y+z=0,
  x-z=0,
  x-y=-1
} $ 
\hfill Απ: αδύνατο 
\end{enumerate}
\end{enumerate}


\section*{Εξισώσεις}

\subsection*{1ου βαθμού}

\begin{enumerate}
  %papadakis p180
  \item  Να λυθούν οι παρακάτω εξισώσεις.
    \begin{enumerate}[i)]
      \item $ 6x-2=9x-5 $ \hfill Απ: $ x=1 $
      \item $ 3(x+4)-(x-2) = 2+x $ \hfill Απ: $ x=-12 $ 
      \item $ 3(2x+1) - 14 = 2(3x+5) $ \hfill Απ: αδύνατη 
      \item $ 10x-2(4+5x) =-8 $ \hfill Απ: ταυτότητα 
    \end{enumerate}
\end{enumerate}

\subsection*{2ου βαθμού}

\begin{enumerate}
  \item  Να λυθούν οι παρακάτω εξισώσεις. (\textcolor{Col1}{Τριώνυμα})
    \begin{enumerate}[i)]
      \item $ x^{2} + 5x + 4 = 0 $ \hfill Απ:  $ -4, -1 $
      \item $ x^{2} - 10x + 25 = 0 $ \hfill Απ: $ 5 $
      \item $ x^{2} - 4x + 5 = 0 $ \hfill Απ:  αδύνατη
      \item $ -x^{2} - 8x + 9 = 0 $ \hfill Απ: $ -9, 1 $
      \item $ 4x^{2} + 20x + 25 = 0 $ \hfill Απ: $ - \frac{5}{2} $ 
    \end{enumerate}

  \item Να λυθούν οι παρακάτω εξισώσεις. (\textcolor{Col1}{Θυμάμαι:} Αν $ a\cdot b=0 $
    τότε $a=0 \; \text{ή} \; b=0 , \; \forall a,b \in \mathbb{R} $)
    \begin{enumerate}[i)]
      \item $ 3x^{2} - 2x + 1 = 2 $ \hfill Απ: $ 1, - \frac{1}{3} $  
      \item $ 3x^{2} + 2x - 4 = 3x - 7 $ \hfill Απ:  αδύνατη
      \item $ (x^{2} - 4)(x^{2} - 5x + 6) = 0 $ \hfill Απ: $ -2, 3, 2 $ 
      \item $ (x^{2} + 5x)(3x^{2} - 2x - 8) = 0 $ \hfill Απ: $ 0, -5, 2, -\frac{4}{3} $
    \end{enumerate}
\end{enumerate}
\end{document}

