\input{preamble_ask.tex}
\newcommand{\vect}[2]{(#1_1,\ldots, #1_#2)}
%%%%%%% nesting newcommands $$$$$$$$$$$$$$$$$$$
\newcommand{\function}[1]{\newcommand{\nvec}[2]{#1(##1_1,\ldots, ##1_##2)}}

\newcommand{\linode}[2]{#1_n(x)#2^{(n)}+#1_{n-1}(x)#2^{(n-1)}+\cdots +#1_0(x)#2=g(x)}

\newcommand{\vecoffun}[3]{#1_0(#2),\ldots ,#1_#3(#2)}

\newcommand{\suma}{\sum_{n=0}^{\infty}a_n x^n}

\newcommand{\sumb}{\sum_{n=1}^{\infty}a_n n x^{n-1}}

\newcommand{\sumc}{\sum_{n=2}^{\infty}a_n n (n-1) x^{n-2}}

\newcommand{\varsum}[2]{\sum_{n=#1}^{#2}}


% \DeclareMathOperator{\r}{r}

\begin{document}

\chapter*{Πίνακες}

\section*{Αντιστρέψιμοι Πίνακες}

\begin{dfn}
  Έστω $ A \in \textbf{M}_{n}(\mathbb{R}) $. Ο πίνακας $ A $ λέγεται
  \textcolor{Col1}{αντιστρέψιμος} (ή μη ιδιάζων) αν υπάρχει πίνακας 
  $ B \in \textbf{M}_{n}(\mathbb{R}) $ τέτοιος ώστε 
  \[
    A\cdot B = B \cdot A = I_{n} 
  \] 
  Στην περίπτωση αυτή, ο πίνακας $ B $ καλείται \textcolor{Col1}{αντίστροφος} του $A$. 
  Αν ο Α δεν έχει αντίστροφο, τότε λέγεται μη αντιστρέψιμος (ή ιδιάζων).
\end{dfn}

\begin{prop}
  Έστω $ A \in \textbf{M}_{n}(\mathbb{R}) $. Αν ο $A$ έχει αντίστροφο, τότε αυτός είναι 
  \textbf{μοναδικός}. 
\end{prop}

\begin{rem}
  Για αυτό, αν $ A \in \textbf{M}_{n}(\mathbb{R}) $ αντιστρέψιμος, τότε τον μοναδικό 
  αντίστροφο του $ A $ τον συμβολίζουμε με $ A^{-1} $.
\end{rem}

\begin{dfn}
  Έστω $ A \in \textbf{M}_{n}(\mathbb{R}) $ αντιστρέψιμος πίνακας και έστω $ k \in
  \mathbb{N} $. Ορίζουμε $ A^{-k} = (A^{-1})^{k} $ 
\end{dfn}

\begin{prop}
  Ισχύουν τα ακόλουθα:
  \begin{myitemize}
    \item Έστω $ A \in \textbf{M}_{n}(\mathbb{R}) $ αντιστρέψιμος πίνακας. Τότε ο 
      $ A^{-1} $ είναι αντιστρέψιμος και $ (A^{-1})^{-1} = A $
    \item Για κάθε $ n \in \mathbb{N} $, ο $ I_{n} $ είναι αντιστρέψιμος και 
      $ (I_{n})^{-1}=I_{n} $.
  \end{myitemize}
\end{prop}

\begin{prop}
  Έστω $ A, B \in \textbf{M}_{n}(\mathbb{R}) $ δύο αντιστρέψιμοι πίνακες. Τότε το 
  γινόμενο $ AB $ είναι αντιστρέψιμος πίνακας και 
  \[ 
    (AB)^{-1}=B^{-1}\cdot A^{-1} 
  \]
\end{prop}

\begin{cor}
  Αν $ A_{1}, A_{2}, \ldots, A_{k} \in \textbf{M}_{n}(\mathbb{R}) $ είναι 
  αντιστρέψιμοι πίνακες, τότε το γινόμενο $ A_{1}\cdot A_{2}\cdots A_{k} $ είναι 
  αντιστρέψιμος πίνακας και 
  \[
    (A_{1}\cdot A_{2}\cdots A_{k})^{-1} = (A_{k})^{-1}\cdots (A_{2})^{-1}\cdot (A_{1})^{-1}
  \] 
\end{cor}

\begin{prop}
  Έστω $ A \in \textbf{M}_{n}(\mathbb{R}) $ αντιστρέψιμος και έστω $ k \in \mathbb{R}- \{
  0 \}$. Τότε ο πίνακας $ kA $ είναι αντιστρέψιμος και 
  \[
    (kA)^{-1} = k A^{-1} 
  \] 
\end{prop}

\begin{rem}
  Προσοχή! Αν $ A, B \in \textbf{M}_{n}(\mathbb{R}) $ αντιστρέψιμοι πίνακες, τότε 
  δεν ισχύει απαραίτητα ότι το άθροισμα $ A+B $ είναι αντιστρέψιμος πίνακας. Πράγματι, 
  οι πίνακες $ A=I_{2} $ και $ B=-I_{2} $, είναι αντιστρέψιμοι, όμως $ A+B= O_{2} $ μη 
  αντιστρέψιμος, αφού για κάθε $ C \in \textbf{M}_{2}(\mathbb{R}) $ έχουμε ότι 
  $ O_{2}\cdot C=O_{2} $.
\end{rem}

\section*{Ικανές και Αναγκαίες Συνθήκες για να είναι ένας Πίνακας Αντιστρέψιμος}

\begin{thm}
  Έστω $ A \in \textbf{M}_{n}(\mathbb{R}) $. Τότε ο $A$ είναι αντιστρέψιμος αν και 
  μόνον αν $ R_{A}=I_{n} $.
\end{thm}

\begin{cor}
  Έστω $ A \in \textbf{M}_{n}(\mathbb{R}) $. Τότε ο $A$ είναι αντιστρέψιμος αν και 
  μόνον αν $ r(A)=n $, όπου $ r(A) $ ο βαθμός του $A$.
\end{cor}

\begin{thm}
  Έστω $ A \in \textbf{M}_{n}(\mathbb{R}) $. Τότε ο $A$ είναι αντιστρέψιμος αν και 
  μόνον αν ο $A$ είναι γινόμενο στοιχειωδών πινάκων.
\end{thm}

\begin{thm}
  Έστω $ A \in \textbf{M}_{n}(\mathbb{R}) $. Τότε ο $A$ είναι αντιστρέψιμος αν και 
  μόνον αν το ομογενές σύστημα $ AX=Ο_{n\times 1} $ έχει μοναδική λύση τη μηδενική 
  (τετριμμένη).
\end{thm}

\begin{thm}
  Έστω $ A \in \textbf{M}_{n}(\mathbb{R}) $. Τότε ο $A$ είναι αντιστρέψιμος αν και 
  μόνον αν για κάθε $ B \in \textbf{M}_{n\times 1}(\mathbb{R}) $ το σύστημα 
  $ AX=B $ είναι συμβιβαστό.
\end{thm}

\begin{thm}
  Έστω $ A \in \textbf{M}_{n}(\mathbb{R}) $. Τότε ο $A$ είναι αντιστρέψιμος αν και 
  μόνον αν για κάθε $ B \in \textbf{M}_{n\times 1}(\mathbb{R}) $ το σύστημα 
  $ AX=B $ έχει μοναδική λύση.
\end{thm}

\section*{Ανάστροφος - Συμμετρικός - Ορθογώνιος}

\begin{dfn}
  Έστω $ A = (a_{ij}) \in \textbf{M}_{m \times n}(\mathbb{R}) $. Ο ανάστροφος του $A$, 
  συμβολικά $ A^{T} $, είναι ο $ n \times m $ πίνακας που προκύπτει δια εναλλαγής των
  γραμμών και στηλών του $A$. Συνεπώς, $ A^{T}=(a_{ji}), \; \forall i=1,2,\ldots, n $ 
  και $ j=1,2,\ldots, m $. 
\end{dfn}

\begin{rem}
  Αν $ A \in \textbf{M}_{n}(\mathbb{R}) $ είναι διαγώνιος, τότε $ A^{T}=A $. 
  Συγκεκριμένα, $ I_{n}^{T}=I_{n} $, για κάθε $ n \in \mathbb{N} $. 
\end{rem}

\begin{thm}[Ιδιότητες Ανάστροφου Πίνακα]
\item {}
  \begin{enumerate}[i)]
    \item $ (A^{T})^{T} = A, \; \forall A \in \textbf{M}_{m \times n}(\mathbb{R}) $ 
    \item $ (kA)^{T} = kA^{T}, \; \forall A \in \textbf{M}_{m \times n}(\mathbb{R}), \;
      \forall k \in \mathbb{R} $ 
    \item $ (A+B)^{T} = A^{T}+B^{T}, \; \forall A,B \in \textbf{M}_{m \times
      n}(\mathbb{R}) $ 
    \item $ (AB)^{T} = B^{T}\cdot A^{T}, \; \forall A \in \textbf{M}_{m \times
      n}(\mathbb{R}) $ και $ B \in \textbf{M}_{n \times p}(\mathbb{R}) $
  \end{enumerate}
\end{thm}

\begin{thm}
  Έστω $ A \in \textbf{M}_{n}(\mathbb{R}) $. Τότε ο $ A^{T} $ είναι αντιστρέψιμος 
  αν και μόνον αν ο $A$ είναι αντιστρέψιμος. Επιπλέον, $ (A^{T})^{-1}=(A^{-1})^{T} $.
\end{thm}

\begin{dfn}
  Ένας $ n \times n $ πίνακας $A$, είναι:
  \begin{myitemize}
    \item συμμετρικός αν $ A^{T}=A  $
    \item αντισυμμετρικός αν $ A^{T}=-A  $
    \item ορθογώνιος αν $ AA^{T}=A^{T}A = I_{n} $ ισοδύναμα αν $ A^{T}=A^{-1} $.
  \end{myitemize}
\end{dfn}

\begin{rem}
\item {}
  \begin{myitemize}
    \item $ A=(a_{i}) \in \textbf{M}_{n}(\mathbb{R}) $ συμμετρικός, αν $ a_{ij} = a_{ji},
      \; \forall i,j=1,2,\ldots, n $ 
    \item $ A=(a_{i}) \in \textbf{M}_{n}(\mathbb{R}) $ αντισυμμετρικός, αν $ a_{ij} =
      -a_{ji}, \; \forall i,j=1,2,\ldots, n $ 
  \end{myitemize}
\end{rem}

\begin{prop}
\item {}
  \begin{myitemize}
    \item  Κάθε τετραγωνικός πίνακας γράφεται ως άθροισμα ενός συμμετρικού και ενός 
      αντισυμμετρικού πίνακα. Πράγματι, αν $ A \in \textbf{M}_{n}(\mathbb{R}) $, τότε
  \[
    A = \underbrace{\frac{1}{2} \left(A+A^{T}\right)}_{\text{συμμετρικός}} +
  \underbrace{\frac{1}{2} \left(A-A^{T}\right)}_{\text{αντισυμμετρικός}} 
\]
\end{myitemize}
\end{prop}

\section*{Ισοδύναμοι}
 
\begin{dfn}
  Έστω $ A,B \in \textbf{M}_{m \times n}(\mathbb{R}) $. Τότε οι $ A $ και $ B $ 
  λέγονται \textcolor{Col1}{ισοδύναμοι}, αν υπάρχουν \textbf{αντιστρέψιμοι} πίνακες 
  $ P \in \textbf{M}_{m}(\mathbb{R}) $ και $ Q \in \textbf{M}_{n}(\mathbb{R}) $ τέτοιοι 
  ώστε
  \[
     B = PAQ 
   \] 
\end{dfn}

\begin{rem}
\item {}
  \begin{myitemize}
    \item Για κάθε $ A \in \textbf{M}_{m \times n}(\mathbb{R}) $, ο $A$ είναι ισοδύναμος 
      με τον εαυτό του. Πράγματι, $ A = I_{m}AI_{n} $, όπου $ I_{n}, I_{m} $ 
      αντιστρέψιμοι.
    \item Για κάθε $ A,B \in \textbf{M}_{m \times n}(\mathbb{R}) $ αν $ A $ ισοδύναμος με
      τον $B$, τότε και $B$ ισοδύνμος με τον $A$. Πράγματι, $A$ ισοδύναμος με $B$ $
      \Rightarrow B=PAQ $ για κάποιους αντιστρέψιμους πίνακες $ P $ και $Q$, οπότε 
      $ A=P^{-1}BQ^{-1} $ (με $ P^{-1}, Q^{-1} $ επίσης αντιστρέψιμοι), οπότε και 
      $B$ ισοδύναμος με τον $A$.
    \item Για κάθε $ A,B, \Gamma \in \textbf{M}_{m \times n}(\mathbb{R}) $, αν $ A $, 
      ισοδύναμος με $ B $ και $ B $, ισοδύναμος με $\Gamma $, τότε και $ A$ ισοδύναμος
      με $ \Gamma $. Πράγματι, αν $ B=PAQ $ για κάποιους αντιστρέψιμους πίνακες 
      $ P $ και $ Q $, και αν $ \Gamma = P'BQ' $ για κάποιους ισοδύναμους πίνακες 
      $ P' $ και $ Q' $, τότε $ \Gamma = (P'P)A(QQ') $ και $ P'P, QQ'$ είναι
      αντιστρέψιμοι. Άρα $ A,\Gamma $ είναι ισοδύναμοι).
  \end{myitemize}
\end{rem}

\begin{rem}
  Ανάλογα με τους στοιχειώδεις μετασχηματισμούς γραμμών σε έναν πίνακα $ A \in
  \textbf{M}_{m \times n}(\mathbb{R}) $, ορίζονται οι στοιχειώδεις μετασχηματισμοί 
  στηλών.
\end{rem}

\begin{dfn}
  Αν $ A, B \in \textbf{M}_{m \times n}(\mathbb{R}) $, τότε ο $B$ λέγεται στηλοισοδυνάμος
  με τον $A$ αν προκύπτει από τον $A$ εφαρμόζοντας μια πεπερασμένη ακολουθία από 
  στοιχειώδεις μετασχηματισμούς στηλών.
\end{dfn}

\begin{thm}
  Αν ένας $ m \times n $ πίνακας $ B $ είναι στηλοισοδύναμος με έναν $ m \times n $ 
  πίνακα $A$, τότε υπάρχει ένα πεπερασμένο πλήθος στοιχειωδών πινάκων $ E_{1}, E_{2},
  \ldots, E_{k} $ έτσι ώστε 
  \[
    B = A \cdot E_{1} \cdot E_{2} \cdots E_{k} 
  \] 
  Επομένως υπάρχει ένας αντιστρέψιμος $ n \times n $ πίνακας $ Q $ τέτοιος ώστε 
  $ B=AQ $ με $ (Q=E_{1}E_{2}\cdots E_{k}) $.
\end{thm}
\end{document}
