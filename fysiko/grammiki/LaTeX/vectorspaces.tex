\input{preamble.tex}
\newcommand{\vect}[2]{(#1_1,\ldots, #1_#2)}
%%%%%%% nesting newcommands $$$$$$$$$$$$$$$$$$$
\newcommand{\function}[1]{\newcommand{\nvec}[2]{#1(##1_1,\ldots, ##1_##2)}}

\newcommand{\linode}[2]{#1_n(x)#2^{(n)}+#1_{n-1}(x)#2^{(n-1)}+\cdots +#1_0(x)#2=g(x)}

\newcommand{\vecoffun}[3]{#1_0(#2),\ldots ,#1_#3(#2)}

\newcommand{\suma}{\sum_{n=0}^{\infty}a_n x^n}

\newcommand{\sumb}{\sum_{n=1}^{\infty}a_n n x^{n-1}}

\newcommand{\sumc}{\sum_{n=2}^{\infty}a_n n (n-1) x^{n-2}}

\newcommand{\varsum}[2]{\sum_{n=#1}^{#2}}
\input{tikz.tex}

\usepackage{anyfontsize}
\pagestyle{vangelis}
\setcounter{chapter}{1}

\linespread{1.1}

\begin{document}

\begin{center}
  \minibox{\large \bfseries \textcolor{Col1}{Ασκήσεις στους Διανυσματικούς Χώρους}}
\end{center}

\vspace{\baselineskip}

\begin{enumerate}
  \item Να εξετάσετε αν τα παρακάτω υποσύνολα είναι υπόχωροι. 

    \twocolumnsidee{
      \begin{enumerate}[(i)]
        \item $ W = \{(x,y)\in \mathbb{R}^{2} \; \mid \; x,y \geq 0 \} $ 
          \hfill Απ: όχι 
        \item $ W = \{ (x,y) \in \mathbb{R}^{2} \; \mid \; x = 3y \} $ \hfill Απ: ναι
        \item $ W = \{ (x,y) \in \mathbb{R}^{2} \; \mid \; x+y=1 \} $ 
          \hfill Απ: όχι 
        \item $ W = \{ (x,y) \in \mathbb{R}^{2} \; \mid \; y = x^{2} \} $ 
          \hfill Απ: όχι 
        \item $ W = \{ (x,y,z) \in \mathbb{R}^{3} \; \mid \; x\geq 0 \} $ \hfill Απ:  όχι
        \item $ W = \{ (x,y,z) \in \mathbb{R}^{3} \; \mid \; x + y + z = 0 \} $ 
          \hfill Απ: ναι
      \end{enumerate}
      }{
      \begin{enumerate}[(i),start=7]
        \item $ W = \{ (x,y,z) \in \mathbb{R}^{3} \; \mid \; x + y + z = 1 \} $ 
          \hfill Απ: όχι
        \item $ W = \{ (x,y,z) \in \mathbb{R}^{3} \; \mid \; x^{2} + y^{2} + z^{2} 
          \leq 1  \} $ \hfill Απ: όχι
        \item $ W = \{ (x,y,z) \in \mathbb{R}^{3} \; \mid \; x = y = z \} $ 
          \hfill Απ: ναι
          % \item $ W = \{ (x,y,z) \in \mathbb{R}^{3} \mid x \leq y \leq q \} $ 
          % \hfill Απ: όχι
        \item $ W = \{ (x,y,z) \in \mathbb{R}^{3} \; \mid \; x = 2y = 3z \} $ 
          \hfill Απ: ναι 
        \item $ W = \{ A \in M_{2}(\mathbb{R}) \; \mid \; \det{A}=1 \} $ 
          \hfill Απ: όχι 
        \item $ W = \{ A \in M_{2}(\mathbb{R}) \; \mid \; \det{A}=0 \} $ 
          \hfill Απ: όχι 
      \end{enumerate}
    }

  \item Να δείξετε ότι τα παρακάτω υποσύνολα είναι υπόχωροι
    \begin{enumerate}[(i)]
      \item $ W = \{ (x,y) \in \mathbb{R}^{2} \mid y = 3x \} $ 
      \item $ W = \{(x,y,z)\in \mathbb{R}^{3} \; : \; 2x-3y+z=0 \} $
    \end{enumerate}

  \item\label{ask:eksart3} Έστω τα διανύσματα $ \mathbf{u}_{1} = (1,2,0) $, 
    $ \mathbf{u}_{2} = (-1,1,2) $ 
    και $ \mathbf{u}_{3} = (3,0,-4) $. Να βρείτε ποια συνθήκη πρέπει να ικανοποιούν τα 
    $a$, $b$, $c$, ώστε το διάνυσμα $ \mathbf{b} = (a,b,c) $ να ανήκει στον υπόχωρο 
    $ \langle \mathbf{u}_{1}, \mathbf{u}_{2}, \mathbf{u}_{3} \rangle $.

    \hfill Απ: $ 4a -2b + 3c = 0 $ 

  \item\label{ask:baseeks} Να εξετάσετε αν τα διανύσματα $ \mathbf{u} = (1,1,1)$, 
    $ \mathbf{v} = (1,2,3)$, $ \mathbf{w} = (1,5,3) $ είναι βάση του $\mathbb{R}^{3}$. 
    \hfill Απ: ναι 

  \item\label{ask:parag} Να εξετάσετε αν τα διανύσματα $ \mathbf{u_{1}} = (1,1,3), 
    \mathbf{u_{2}} = (1,0,2), \mathbf{u_{3}}= (1,-2,0)$ παράγουν τον $ \mathbb{R}^{3} $.
    \hfill Απ: όχι 

  \item\label{ask:isoi} Έστω τα διανύσματα $ \mathbf{u}_{1} = (1,2,-1,3)$, 
    $\mathbf{u}_{2} = (2,4,1,-2)$, $ \mathbf{u} _{3} = (3,6,3,-7) $ και 
    $ \mathbf{w}_{1} = (1,2,-4,11)$, $ \mathbf{w}_{2} = (2,4,-5,14) $. 
    Αν $ U = \Span \{ \mathbf{u}_{1}, \mathbf{u}_{2}, \mathbf{u}_{3} \} $ και 
    $ W = \Span \{ \mathbf{w}_{1}, \mathbf{w}_{2} \} $, τότε να δείξετε ότι $ U=W $.

  \item\label{ask:parag2} Για τα παρακάτω διανύσματα να βρεθεί μια βάση και η διάσταση 
    του χώρου που παράγουν.
    \begin{enumerate}[(i)]
      \item $ \mathbf{u}_{1} = (1,1,1,2,3) $, $ \mathbf{u}_{2} = (1,2,-1,-2,1) $, 
        $ \mathbf{u} _{3} = (3,5,-1,-2,5) $, $ \mathbf{u}_{4} = (1,2,1,-1,4) $

        \hfill Απ: $ B = \{ \mathbf{u}_{1}, \mathbf{u}_{2}, \mathbf{u}_{3}\} $ 

      \item $ \mathbf{u}_{1} = (1,-2,1,3,-1) $, $ \mathbf{u}_{2} = (-2,4,-2,-6,2) $, $
        \mathbf{u}_{3} = (1,-3,1,2,1) $, $ \mathbf{u}_{4} = (3,-7,3,8,-1) $

        \hfill Απ: $ B = \{ \mathbf{u}_{1}, \mathbf{u}_{2}, \mathbf{u}_{3} \} $ 

      \item $ \mathbf{u}_{1} = (1,0,1,0,1) $, $ \mathbf{u}_{2} = (1,1,2,1,0) $, 
        $ \mathbf{u} _{3} = (1,2,3,1,1,) $, $ \mathbf{u}_{4} = (1,2,1,1,1) $
        \hfill Απ: $ B = \{ \mathbf{u}_{1}, \mathbf{u}_{2}, \mathbf{u}_{3}, 
        \mathbf{u}_{4} \} $ 

      \item $ \mathbf{u}_{1} = (1,0,1,1,1) $, $ \mathbf{u}_{2} = (2,1,2,0,1) $, 
        $ \mathbf{u} _{3} = (1,1,2,3,4) $, $ \mathbf{u}_{4} = (4,2,5,4,6) $
        \hfill Απ: $ B = \{ \mathbf{u}_{1}, \mathbf{u}_{2}, \mathbf{u}_{3} \} $ 
    \end{enumerate}

  \item Θεωρήστε τους υποχώρους 
    $ U = \{ (x,y,z,w) \in \mathbb{R}^{3} \mid y - 2z + w = 0 \} $ και 
    $ W = \{ (x,y,z,w) \in \mathbb{R}^{3} \mid x = w, y = 2z \} $ του 
    $\mathbb{R}^{4}$. Να βρείτε μια βάση και τη διάσταση των υποχώρων:
    \begin{enumerate}[(i)]
      \item $ U $ \hfill Απ: $ B_{U} = \{ (1,0,0,0), (0,2,1,0), (0,-1,0,1) \} $ 
      \item $ W $ \hfill Απ: $ B_{W} = \{ (0,2,1,0), (1,0,0,1) \} $ 
      \item $ U \cap W $ \hfill Απ: $ B_{U\cap W} = \{ (0,2,1,0) \} $ 
    \end{enumerate}	

  \item Έστω $ W $ ο υπόχωρος του $\mathbb{R}^{4}$ που παράγεται από τα διανύσματα 
    $ \mathbf{u} _{1} = (1,-2,5,-3), \mathbf{u}_{2} = (2,3,1,-4) $, 
    $ \mathbf{u}_{3} = (3,8,-3,-5) $.
      Να βρείτε μια βάση και τη διάσταση του $ W $.

    \hfill Απ:  \begin{tabular}{l}
      $B = \{ (1,-2,5,-3), (0,7,-9,2) \}$ \\
    \end{tabular} 
\end{enumerate}


\end{document}

