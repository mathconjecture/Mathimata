\documentclass[a4paper,12pt]{article}
\usepackage{etex}
%%%%%%%%%%%%%%%%%%%%%%%%%%%%%%%%%%%%%%
% Babel language package
\usepackage[english,greek]{babel}
% Inputenc font encoding
\usepackage[utf8]{inputenc}
%%%%%%%%%%%%%%%%%%%%%%%%%%%%%%%%%%%%%%

%%%%% math packages %%%%%%%%%%%%%%%%%%
\usepackage{amsmath}
\usepackage{amssymb}
\usepackage{amsfonts}
\usepackage{amsthm}
\usepackage{proof}

\usepackage{physics}

%%%%%%% symbols packages %%%%%%%%%%%%%%
\usepackage{bm} %for use \bm instead \boldsymbol in math mode 
\usepackage{dsfont}
\usepackage{stmaryrd}
%%%%%%%%%%%%%%%%%%%%%%%%%%%%%%%%%%%%%%%


%%%%%% graphicx %%%%%%%%%%%%%%%%%%%%%%%
\usepackage{graphicx}
\usepackage{color}
%\usepackage{xypic}
\usepackage[all]{xy}
\usepackage{calc}
\usepackage{booktabs}
\usepackage{minibox}
%%%%%%%%%%%%%%%%%%%%%%%%%%%%%%%%%%%%%%%

\usepackage{enumerate}

\usepackage{fancyhdr}
%%%%% header and footer rule %%%%%%%%%
\setlength{\headheight}{14pt}
\renewcommand{\headrulewidth}{0pt}
\renewcommand{\footrulewidth}{0pt}
\fancypagestyle{plain}{\fancyhf{}
\fancyhead{}
\lfoot{}
\rfoot{\small \thepage}}
\fancypagestyle{vangelis}{\fancyhf{}
\rhead{\small \leftmark}
\lhead{\small }
\lfoot{}
\rfoot{\small \thepage}}
%%%%%%%%%%%%%%%%%%%%%%%%%%%%%%%%%%%%%%%

\usepackage{hyperref}
\usepackage{url}
%%%%%%% hyperref settings %%%%%%%%%%%%
\hypersetup{pdfpagemode=UseOutlines,hidelinks,
bookmarksopen=true,
pdfdisplaydoctitle=true,
pdfstartview=Fit,
unicode=true,
pdfpagelayout=OneColumn,
}
%%%%%%%%%%%%%%%%%%%%%%%%%%%%%%%%%%%%%%

\usepackage[space]{grffile}

\usepackage{geometry}
\geometry{left=25.63mm,right=25.63mm,top=36.25mm,bottom=36.25mm,footskip=24.16mm,headsep=24.16mm}

%\usepackage[explicit]{titlesec}
%%%%%% titlesec settings %%%%%%%%%%%%%
%\titleformat{\chapter}[block]{\LARGE\sc\bfseries}{\thechapter.}{1ex}{#1}
%\titlespacing*{\chapter}{0cm}{0cm}{36pt}[0ex]
%\titleformat{\section}[block]{\Large\bfseries}{\thesection.}{1ex}{#1}
%\titlespacing*{\section}{0cm}{34.56pt}{17.28pt}[0ex]
%\titleformat{\subsection}[block]{\large\bfseries{\thesubsection.}{1ex}{#1}
%\titlespacing*{\subsection}{0pt}{28.80pt}{14.40pt}[0ex]
%%%%%%%%%%%%%%%%%%%%%%%%%%%%%%%%%%%%%%

%%%%%%%%% My Theorems %%%%%%%%%%%%%%%%%%
\newtheorem{thm}{Θεώρημα}[section]
\newtheorem{cor}[thm]{Πόρισμα}
\newtheorem{lem}[thm]{λήμμα}
\theoremstyle{definition}
\newtheorem{dfn}{Ορισμός}[section]
\newtheorem{dfns}[dfn]{Ορισμοί}
\theoremstyle{remark}
\newtheorem{remark}{Παρατήρηση}[section]
\newtheorem{remarks}[remark]{Παρατηρήσεις}
%%%%%%%%%%%%%%%%%%%%%%%%%%%%%%%%%%%%%%%




\newcommand{\vect}[2]{(#1_1,\ldots, #1_#2)}
%%%%%%% nesting newcommands $$$$$$$$$$$$$$$$$$$
\newcommand{\function}[1]{\newcommand{\nvec}[2]{#1(##1_1,\ldots, ##1_##2)}}

\newcommand{\linode}[2]{#1_n(x)#2^{(n)}+#1_{n-1}(x)#2^{(n-1)}+\cdots +#1_0(x)#2=g(x)}

\newcommand{\vecoffun}[3]{#1_0(#2),\ldots ,#1_#3(#2)}

\newcommand{\mysum}[1]{\sum_{n=#1}^{\infty}




\everymath{\displaystyle}


\begin{document}

\begin{center}
  \fbox{\large\bf Εφαπτόμενο Επίπεδο μιας επιφάνειας }
\end{center}

\vspace{\baselineskip}

\begin{itemize}
  \item Αν $S: F(x,y,z)=0$ τότε η διανυσματική εξίσωση του εφαπτόμενου επιπέδου της $S$ στο σημείο
    της $ P(x_{0}, y_{0}, z_{0}) $ είναι
    \begin{equation}\label{eq:tan}
      (\mathbf{r} - \mathbf{r}_{0})\cdot \grad F(P_{0})=0 
    \end{equation} 
    όπου $ \mathbf{r}=x \mathbf{i}+y \mathbf{j}+z \mathbf{k} $ και $ r_{0}=x_{0} \mathbf{i}+ y
    _{0} \mathbf{j}+ z_{0} \mathbf{k} $.

    Σε καρτεσιανές συντεταγμένες η εξίσωση ~\eqref{eq:tan} γράφεται:

    \[
      \boxed{		
        \pdv{F}{x}\eval_{P_{0}} (x-x_{0}) + \pdv{F}{y}\eval_{P_{0}} (y-y_{0}) +
        \pdv{F}{z}\eval_{P_{0}} (z-z_{0}) =0 
      }
    \]
  \item Ένα κάθετο διάνυσμα στην επιφάνεια $ S $ στο σημείο $ P_{0} $ είναι το διάνυσμα της κλίσης της
    συνάρτησης 
    \[
      \grad F{(P_{0})} = \left(\pdv{f}{x}, \pdv{f}{y}, \pdv{f}{z}\right)_{P_{0}}
    \] 

  \item Αν $ S: z=f(x,y) $ τότε θέτουμε $ F(x,y,z) = z - f(x,y) = 0 $. Οπότε σε
    αυτήν την περίπτωση η εξίσωση ~\eqref{eq:tan} γράφεται
    \[
      \boxed{
        -\pdv{f}{x}\eval_{P_{0}} (x - x_{0}) - \pdv{f}{y}\eval_{P_{0}} (y - y_{0}) + (z - z_{0})
      }
    \] 
\end{itemize}

\end{document}
