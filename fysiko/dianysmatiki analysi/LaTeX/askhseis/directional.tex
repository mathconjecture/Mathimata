\documentclass[a4paper,12pt]{article}
\usepackage{etex}
%%%%%%%%%%%%%%%%%%%%%%%%%%%%%%%%%%%%%%
% Babel language package
\usepackage[english,greek]{babel}
% Inputenc font encoding
\usepackage[utf8]{inputenc}
%%%%%%%%%%%%%%%%%%%%%%%%%%%%%%%%%%%%%%

%%%%% math packages %%%%%%%%%%%%%%%%%%
\usepackage{amsmath}
\usepackage{amssymb}
\usepackage{amsfonts}
\usepackage{amsthm}
\usepackage{proof}

\usepackage{physics}

%%%%%%% symbols packages %%%%%%%%%%%%%%
\usepackage{bm} %for use \bm instead \boldsymbol in math mode 
\usepackage{dsfont}
\usepackage{stmaryrd}
%%%%%%%%%%%%%%%%%%%%%%%%%%%%%%%%%%%%%%%


%%%%%% graphicx %%%%%%%%%%%%%%%%%%%%%%%
\usepackage{graphicx}
\usepackage{color}
%\usepackage{xypic}
\usepackage[all]{xy}
\usepackage{calc}
\usepackage{booktabs}
\usepackage{minibox}
%%%%%%%%%%%%%%%%%%%%%%%%%%%%%%%%%%%%%%%

\usepackage{enumerate}

\usepackage{fancyhdr}
%%%%% header and footer rule %%%%%%%%%
\setlength{\headheight}{14pt}
\renewcommand{\headrulewidth}{0pt}
\renewcommand{\footrulewidth}{0pt}
\fancypagestyle{plain}{\fancyhf{}
\fancyhead{}
\lfoot{}
\rfoot{\small \thepage}}
\fancypagestyle{vangelis}{\fancyhf{}
\rhead{\small \leftmark}
\lhead{\small }
\lfoot{}
\rfoot{\small \thepage}}
%%%%%%%%%%%%%%%%%%%%%%%%%%%%%%%%%%%%%%%

\usepackage{hyperref}
\usepackage{url}
%%%%%%% hyperref settings %%%%%%%%%%%%
\hypersetup{pdfpagemode=UseOutlines,hidelinks,
bookmarksopen=true,
pdfdisplaydoctitle=true,
pdfstartview=Fit,
unicode=true,
pdfpagelayout=OneColumn,
}
%%%%%%%%%%%%%%%%%%%%%%%%%%%%%%%%%%%%%%

\usepackage[space]{grffile}

\usepackage{geometry}
\geometry{left=25.63mm,right=25.63mm,top=36.25mm,bottom=36.25mm,footskip=24.16mm,headsep=24.16mm}

%\usepackage[explicit]{titlesec}
%%%%%% titlesec settings %%%%%%%%%%%%%
%\titleformat{\chapter}[block]{\LARGE\sc\bfseries}{\thechapter.}{1ex}{#1}
%\titlespacing*{\chapter}{0cm}{0cm}{36pt}[0ex]
%\titleformat{\section}[block]{\Large\bfseries}{\thesection.}{1ex}{#1}
%\titlespacing*{\section}{0cm}{34.56pt}{17.28pt}[0ex]
%\titleformat{\subsection}[block]{\large\bfseries{\thesubsection.}{1ex}{#1}
%\titlespacing*{\subsection}{0pt}{28.80pt}{14.40pt}[0ex]
%%%%%%%%%%%%%%%%%%%%%%%%%%%%%%%%%%%%%%

%%%%%%%%% My Theorems %%%%%%%%%%%%%%%%%%
\newtheorem{thm}{Θεώρημα}[section]
\newtheorem{cor}[thm]{Πόρισμα}
\newtheorem{lem}[thm]{λήμμα}
\theoremstyle{definition}
\newtheorem{dfn}{Ορισμός}[section]
\newtheorem{dfns}[dfn]{Ορισμοί}
\theoremstyle{remark}
\newtheorem{remark}{Παρατήρηση}[section]
\newtheorem{remarks}[remark]{Παρατηρήσεις}
%%%%%%%%%%%%%%%%%%%%%%%%%%%%%%%%%%%%%%%




\input{definitions_ask.tex}


\pagestyle{askhseis}
\renewcommand{\vec}{\mathbf}

\linespread{1.3}

\begin{document}

\begin{center}
  \minibox{\large \bfseries \textcolor{Col1}{Ασκήσεις στην Παράγωγο κατά Κατεύθυνση}}
\end{center}

\vspace{\baselineskip}

\begin{enumerate}
  \item Να υπολογίσετε την παράγωγο κατά κατεύθυνση, των παρακάτω συναρτήσεων στο σημείο 
    και προς της κατεύθυνση που σας δίνεται κάθε φορά.
    \begin{enumerate}[i)]
      % \item $ f(x,y) = y \mathrm{e}^{-x} $, στο σημείο $ P_{0}(0,4) $ και προς την
      %   κατεύθυνση της \textbf{γωνίας} $ \theta = \frac{2\pi}{3} $. 
      %   \hfill Απ: $ 2 + \sqrt{3} / 2 $ 
      \item $ f(x,y) = \sin{(2x+3y)} $, στο σημείο $ P_{0}(-6,4) $ και προς την
        κατεύθυνση  $ \mathbf{u} = \sqrt{3} /2 \mathbf{i} - 1/2 \mathbf{j}$. 
        \hfill Απ: $ \sqrt{3} - 3/2 $ 
      \item $ f(x,y) = \sqrt{xy} $, στο σημείο $ P_{0}(2,8) $ και προς την
        κατεύθυνση του \textbf{σημείου} $P(5,4) $. \hfill Απ: $ 2 / 5 $ 
      \item $ f(x,y,z) = x \mathrm{e}^{2yz} $, στο σημείο $ P_{0}(3,0,2) $ και προς την
        κατεύθυνση  $ \mathbf{u} = (2/3, -2/3 , 1/3) $. 
        \hfill Απ: $-22/3$ 
      \item $ f(x,y,z) = xy+yz+zx $, στο σημείο $ P_{0}(1,-1,3) $ και προς την
        κατεύθυνση  του \textbf{σημείου} $P(2,4,5)$. 
        \hfill Απ: 22
    \end{enumerate}

  \item Να υπολογίσετε τον μέγιστο ρυθμό μεταβολής των παρακάτω συναρτήσεων, στο σημείο 
    που σας δίνεται κάθε φορά, καθώς και την κατεύθυνση προς την οποία αυτός 
    παρατηρείται.
    \begin{enumerate}[i)]
      \item $ f(x,y) = y^{2}/x $, στο σημείο $ P_{0}(2,4) $.
        \hfill Απ: $4 \sqrt{2}$ 
      \item $ f(x,y,z) = \sqrt{x^{2}+y^{2}+z^{2}} $, στο σημείο $ P_{0}(3,6,-2) $.
        \hfill Απ: $ 1 $
    \end{enumerate}

  \item 
    \begin{enumerate}[i)]
      \item Να δείξετε ότι μια συνάρτηση $ f(x,y) $, μειώνεται με το γρηγορότερο ρυθμό
        στο τυχαίο σημείο $ P_{0}(x,y) $, προς την κατεύθυνση που είναι αντίθετη από το 
        διάνυσμα της κλίσης της $ - \grad f(P_{0}) $.
        \hfill Απ: $ \theta = \pi $  
      \item Χρησιμοποιείστε το $ \rm{i)} $ ερώτημα, για να προσδιορίσετε την κατεύθυνση,
        προς την οποία η συνάρτηση με τύπο $ f(x,y) = x^{4}y-x^{2}y^{3} $ μειώνεται με 
        το γρηγορότερο ρυθμό στο σημείο $ P_{0}(2,-3) $. 
        \hfill Απ: $ - \grad f_{P_{0}} = (-12,92) $
    \end{enumerate}

  \item Η παράγωγος της συνάρτησης $ f(x,y) $, στο σημείο $ P(1,2) $ και προς την 
    κατεύθυνση $ \mathbf{u} = \mathbf{i} + \mathbf{j} $ είναι $ 2 \sqrt{2} $, ενώ 
    προς την κατεύθυνση $ \mathbf{v} = -2 \mathbf{j} $ είναι $ -3 $. Ποια είναι 
    η παράγωγος της $f$ προς την κατεύθυνση $ \mathbf{w} = - \mathbf{i}- 2\mathbf{j} $, 
    στο ίδιο σημείο; \hfill Απ: $ -7/ \sqrt{5} $

  \item Έστω ότι σε μια περιοχή του χώρου, το ηλεκτρικό δυναμικό $V$ σε κάθε σημείο 
    δίνεται από τη συνάρτηση \[ V(x,y,z) = 5x^{2}-3xy+xyz \] 
    \begin{enumerate}[i) ]
      \item Να υπολογίσετε το ρυθμό μεταβολής του ηλεκτρικού δυναμικού, στο σημείο 
        $ P(3,4,5) $, προς την κατεύθυνση του $ \mathbf{u} = \mathbf{i} + \mathbf{j} - 
        \mathbf{k} $. \hfill Απ: $ 32 / \sqrt{3} $  

      \item Προς ποια κατεύθυνση το ηλεκτρικό δυναμικό μεταβάλλεται πιο γρήγορα, στο 
        σημείο $ P $: \hfill Απ: $ (38,6,12) $ 
      \item Ποιος είναι ο μέγιστος ρυθμός μεταβολής του ηλεκτρικού δυναμικού, στο σημείο 
        $ P $; \hfill Απ: $ 2 \sqrt{406} $ 
    \end{enumerate}

  \item Η θερμοκρασία $T$ σε κάθε σημείο μιας μεταλλικής μπάλας, είναι αντιστρόφως 
    ανάλογη προς την απόσταση του σημείου αυτού από το κέντρο της, που θεωρούμε ότι 
    βρίσκεται στην αρχή των αξόνων. Αν η θερμοκρασία του σημείου $ (1,2,2) $ είναι 
    $ \SI{120}{\celsius} $, να υπολογίσετε:
    \begin{enumerate}[i)]
      \item Το ρυθμό μεταβολής της θερμοκρασίας $T$, στο σημείο $ (1,2,2) $ προς την 
        κατεύθυνση του σημείου $ (2,1,3) $. 

        \hfill Απ: $ - 40 /3 \sqrt{3} $ 
      \item Να δείξετε ότι σε κάθε σημείο της μεταλλικής μπάλας η κατεύθυνση προς την
        οποία η θερμοκρασία αυξάνεται με το μεγαλύτερο ρυθμό, δίνεται από ένα διάνυσμα 
        με κατεύθυνση προς την αρχή των αξόνων.
    \end{enumerate}
\end{enumerate}


  \end{document}
