\documentclass[a4paper,table]{report}
\input{preamble_ask.tex}
\input{definitions_ask.tex}


\pagestyle{askhseis}

\begin{document}

\begin{center}
\textcolor{Col1}{\minibox{\large\bfseries{Αλγεβρικές Δομές}}}
\end{center}

\vspace{\baselineskip}

\subsection*{Ομάδες}

\begin{enumerate}
   \item Αν $ "+" $ και $ "\cdot" $ είναι οι συνήθεις πράξεις πρόσθεσης και
     πολλαπλασιασμού να εξετάσετε αν τα παρακάτω σύνολα είναι ή όχι \textbf{ομάδες}.
     \begin{enumerate}[i)]
       \item $ (\{ x \in \mathbb{Z} \; : \; x<0 \},+) $ \hfill Απ: Όχι γιατί $ \nexists
         $ ουδέτερο 
       \item $ (\{ 3x \; : \; x \in \mathbb{Z}\},+) $ \hfill Απ: Ναι 
       \item $ (\{ x \in \mathbb{Z} \; : \; x \; \text{περιττός}\},\cdot) $ \hfill Απ:
         Όχι, γιατί $ \nexists $ συμμετρικό 
     \end{enumerate}

  \item Να δείξετε ότι το σύνολο 
    $G = \{ 2^{m}\cdot 3^{n} \; : \; m,n \in \mathbb{Z} \}$ αποτελεί \textbf{ομάδα} 
    ως προς τον συνήθη πολλαπλασιασμό πραγματικών αριθμών.

  \item Στο σύνολο $ G = \{ (a,b) \; : \; a \in \mathbb{R} \; \text{και} \; b \in
    \mathbb{R} - \{ 0 \} \} $ ορίζουμε την πράξη 
    \[
      (a_{1}, b_{1}) * (a_{2}, b_{2}) = (a_{1}+ a_{2}, b_{1} \cdot b_{2}), \quad \forall
      (a_{1}, b_{1}), \; (a_{2}, b_{2}) \in G
     \] 
     όπου $ "+" $ και $ "\cdot" $ είναι οι συνήθεις πράξεις στο $ \mathbb{R} $. 
     Να δείξετε ότι η δομή $ (G, *) $ είναι \textbf{αβελιανή ομάδα}.
\end{enumerate}

\subsection*{Δακτύλιοι}

\begin{enumerate}
  \item Στο σύνολο των ακεραίων $ \mathbb{Z} $ ορίζουμε τις ακόλουθες πράξεις: 
    \begin{gather*}
      a \oplus b = a+b+1, \quad \forall a,b \in \mathbb{Z} \\
      a \odot b = a+b+ab, \quad \forall a,b \in \mathbb{Z}
    \end{gather*} 
    όπου $ "+" $ και $ "\cdot" $ οι συνήθεις πράξεις στο $ \mathbb{Z} $. Να δείξετε ότι 
    η δομή $ (Z, \oplus , \odot) $ είναι δακτύλιος.
\end{enumerate}


\end{document}
