\documentclass[a4paper,11pt]{report}
\input{preamble_ask.tex}
\input{definitions_ask.tex}

\pagestyle{vangelis}
\everymath{\displaystyle}

\begin{document}

\begin{center}
  \textcolor{Col1}{\minibox[c]{\large \textbf{Εφαρμογές ΜΔΕ 2ης τάξης} \\ \textbf{Ομογενή
  προβλήματα}}}
\end{center}


\section*{Εξίσωση Κύματος}%

\begin{enumerate}
  \item Θεωρούμε μία χορδή μήκους $ L=1 $, η οποία είναι σταθερά στερεωμένη στα δύο 
    άκρα της. Στο χρόνο $ t=0 $ η χορδή είναι ακίνητη και περιγράφεται από τη συνάρτηση 
    $ f(x)= \sin{(\pi x)} $. Να υπολογιστεί η κίνηση της αν η τάση του νήματος, είναι 
    $ a = 1 $.
\end{enumerate}


\section*{Εξίσωση Διάχυσης}%

\begin{enumerate}
  \item Θεωρούμε χάλκινη ράβδο μήκους $ L=1 $ με πολύ μικρό πάχος, και με συντελεστή 
    διάχυσης $ D =  \SI{1.11d-4}{\metre^2/\second} $, η οποία είναι μονωμένη 
    περιφερειακά.  Η αρχική της θερμοκρασία είναι $ T_{0} = \SI{100}{\degreeCelsius} $, 
    ενώ στα άκρα εφαρμόζουμε σταθερή θερμοκρασία $ \SI{0}{\degreeCelsius} $, ώστε να 
    ψυχθεί η ράβδος. Να βρεθεί η θερμοκρασία της ράβδου την τυχαία χρονική στιγμή $ t $.
\end{enumerate}


\section*{Εξίσωση Laplace}%

\begin{enumerate}

  \item Θεωρούμε τετραγωνική πλάκα με πλευρές μήκους $ L=1 $. Οι τρεις πλευρές βρίσκονται
    σε θερμοκρασία μηδέν, ενώ η δεξιά κάθετη πλευρά έχει θερμοκρασία $ T=100 $. 
    Να γραφεί η εξίσωση που διέπει το πρόβλημα στη μόνιμη κατάσταση, να σημειωθούν οι 
    συνοριακές συνθήκες και να υπολογιστεί η κατανομή της θερμοκρασίας στην πλάκα.

  \item Θεωρούμε τετραγωνική πλάκα με αδιαβατικά τοιχώματα στις τρεις πλευρές και ροή
    θερμότητας που δίνεται από τη συνάρτηση $ T_{y}(x,0)= \sin{(2 \pi x)} $. 
    Να γραφεί η εξίσωση που διέπει το πρόβλημα στη μόνιμη κατάσταση, να σημειωθούν οι 
    συνοριακές συνθήκες και να υπολογιστεί η κατανομή της θερμοκρασίας στην πλάκα.


\end{enumerate}


\end{document}

