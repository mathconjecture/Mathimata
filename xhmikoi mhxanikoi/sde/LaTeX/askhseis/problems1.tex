\input{preamble_ask.tex}
\input{definitions_ask.tex}

\pagestyle{askhseis}
\everymath{\displaystyle}

\begin{document}



\begin{center}
  \minibox{\large \bfseries \color{Col1} Προβλήματα σδε 1ης τάξης} 
\end{center}

\vspace{\baselineskip}

\section*{Μεταφορά θερμότητας}

\begin{enumerate}
    %Petropoulou p573
  \item Μια ζεστή καλοκαιρινή μέρα, η θερμοκρασία περιβάλλοντος είναι 
    $ \SI{36}{\degree C} $, ενώ η θερμοκρασία εντός ενός δωματίου είναι 
    $ \SI{26}{\degree C} $. Το δωμάτιο διαθέτει κλιματιστικό το οποίο τίθεται 
    σε λειτουργία και μέσα σε 10 λεπτά, η θερμοκρασία του δωματίου μειώνεται ένα βαθμό. 
    Σε πόση ώρα η θερμοκρασία του δωματίου θα φτάσει τους $ \SI{22}{\degree C} $; 
    \hfill Απ: $ 35,3 $ λεπτά 

    %Petropoulou p574
  \item Η θερμοκρασία ενός δωματίου είναι $ \SI{21}{\degree C} $. Τοποθετούμε στο 
    μπαλκόνι έξω από το δωμάτιο ένα θερμόμετρο που προηγουμένως βρισκόταν μέσα στο 
    συγκεκριμένο δωμάτιο και ελέγχουμε τακτικά τις ενδείξεις του θερμομέτρου. Μετά 
    από 5 λεπτά, το θερμόμετρο δείχνει $ \SI{16}{\degree C} $, ενώ μετά από άλλα 
    5 λεπτά δείχνει $ \SI{13}{\degree C} $. Πόσο είναι η εξωτερική θερμοκρασία;
    \hfill Απ: $ \SI{8,5}{\degree C} $ 

    %Siafarikas p57
\item Η θερμοκρασία ενός φλυτζανιού με τσάι είναι αρχικά $ \SI{100}{\degree C} $ και 
  η θερμοκρασία του δωματίου μέσα στο οποίο βρίσκεται είναι σταθερή και ίση με 
  $ \SI{30}{\degree C} $. 
  \begin{enumerate}[i)]
    \item Αν μετά από 15 λεπτά η θερμοκρασία του τσαγιού είναι $ \SI{80}{\degree C} $ 
      να υπολογιστεί ο χρόνος που απαιτείται ώστε η θερμοκρασία του να φτάσει τους 
      $ \SI{50}{\degree C} $. 
    \item Υποθέτουμε, τώρα, ότι το ίδιο φλυτζάνι τσάι, πρώτα το αφήνουμε να κρυώσει για 
      20 λεπτά μέσα στο δωμάτιο. Στη συνέχεια το βάζουμε σ᾽ ένα ψυγείο με θερμοκρασία 
      $ \SI{15}{\degree C} $. Αν μετά από 10 λεπτά στο ψυγείο, η θερμοκρασία του 
      τσαγιού έχει πέσει στους $ \SI{60}{\degree C} $, να βρεθεί η θερμοκρασία του μετά 
    από 1 ώρα. 
    \hfill Απ: 
    \begin{enumerate*}[i)] 
      \item 56 λεπτά \item $ \SI{26}{\degree C} $  
    \end{enumerate*}
  \end{enumerate}
\end{enumerate}

\section*{Διάσπαση Ραδιενεργών Πυρήνων}

\begin{enumerate}
    %Siafarikas p204
\item Έστω ότι ο ρυθμός με τον οποίο διασπάται μια ραδιενεργός ουσία είναι ανάλογος 
  προς την υπάρχουσα ποσότητά της. Ένα συγκεκριμένο δείγμα με περιεκτικότητα $ 50 \% $ 
  της ουσίας διασπάται πλήρως σε μια περίοδο $ 1600 $ ετών (πχ. ράδιο-226). Ποιο 
  ποσοστό του αρχικού δείγματος θα διασπαστεί σε $ 800 $ έτη; Σε πόσα χρόνια θα απομείνει
  μόνο το 1/5 της αρχικής ποσότητας; 
  
  \hfill Απ: $ k\approx 0,000433 $, $ N(800) = 0,70723 $, $ t_{1}=3717 $ 


    %Siafarikas p204
\item Το Θόριο-234 είναι μια ραδιενεργή ουσία που διασπάται με ρυθμό ανάλογο της 
  αρχικής ποσότητας. Υποθέστε ότι $ \SI{1}{gr} $ αυτού του υλικού ελαττώνεται στα 
  $ \SI{0,8}{gr} $ σε μια εβδομάδα. Βρείτε το χρόνο υποδιπλασιασμού του Θορίου-234. 
  Πόση ποσότητα Θορίου-234 θα έχει απομείνει μετά από 10 βδομάδες;

  \hfill Απ: $ t_{1/2} = 3,1$, $ N(10)=0,107528 $  

    %Siafarikas p214
\item Με τη βοήθεια της χημικής ανάλυσης, υπολογίστηκε ότι η εναπομείνουσα ποσότητα 
  άνθρακα-14 που βρέθηκε στα δείγματα του ξυλάνθρακα που πάρθηκαν από το σπήλαιο 
  Lascaux ήταν $ 15 \% $ της αρχικής ποσότητας, τη στιγμή που το δέντρο πέθανε. Αν 
  είναι γνωστό ότι ο χρόνος ημίσειας ζωής του άνθρακα-14 είναι περίπου 5600 έτη και 
  ότι η ποσότητα $ N(t) $, του άνθρακα-14 σ᾽ ένα δείγμα ξυλάνθρακα ικανοποιεί την 
  εξίσωση $ N'(t)=-kN(t) $, να βρεθεί η ηλικία των τοιχογραφιών του σπηλαίου Lascaux.

  \hfill Απ: $ t_{1} = 15327 $ 

    %Siafarikas p215
\item Σε μια αρχαιολογική έρευνα οι επιστήμονες βρήκαν ένα αρχαίο εργαλείο κοντά σε ένα 
  απολιθωμένο ανθρώπινο οστό. Αν το εργαλείο και το απολίθωμα περιέχουν $ 65 \% $ και 
  $ 60 \% $ της αρχικής ποσότητας του άνθρακα-14, αντίστοιχα, να προσδιορισθεί αν 
  είναι δυνατόν το εργαλείο να είχε χρησιμοποιηθεί από αυτόν τον άνθρωπο.
  Δίνεται ότι ο χρόνος ημίσειας ζωής του άνθρακα-14 είναι περίπου 5600 έτη.

  \hfill Απ: $ t_{\text{εργ.}} = 3480 $, $ t_{\text{ανθρ.}} = 4127 $  

    %Siafarikas p216
\item Το 1960 οι New York Times με άρθρο τους ανακοίνωσαν ότι "οι αρχαιολόγοι 
  υποστηρίζουν ότι η κοινωνία των Σουμέριων κατοικούσε στην κοιλάδα του Τίγρη πριν από 
  5000 χρόνια". Υποθέτωντας ότι οι αρχαιολόγοι χρησιμοποίησαν την μέθοδο του άνθρακα-14 
  για να χρονολογήσουν την περιοχή, υπολογίστε το ποσοστό του άνθρακα-14 που βρέθηκε 
  στα δείγματά τους. Δίνεται ότι ο χρόνος ημίσειας ζωής του άνθρακα-14 είναι περίπου 
  5600 έτη.
  \hfill Απ: $ 53,8 \% $ 
\end{enumerate}

\section*{Μοντέλο Malthus}

\begin{enumerate}
    %Siafarikas p150
  \item Υποθέτουμε ότι ο πληθυσμός μιας πόλης είναι αρχικά 5000. Λόγω της κατασκευής 
    ενός αυτοκινητόδρομου που περνάει από την πόλη και συνδέει δυο μεγάλες επαρχίες ο
    πληθυσμός διπλασιάζεται τον επόμενο χρόνο. Αν ο ρυθμός αύξησης του πληθυσμού είναι 
    ανάλογος του υπάρχοντα πληθυσμού να βρεθεί σε πόσα χρόνια ο πληθυσμός θα γίνει 
    25000, καθώς και πόσος θα είναι ο πληθυσμός μετά από 5 χρόνια.
    \hfill Απ: $ t_{1}=2,3 $, $ y(5)=160000 $ 

    %Siafarikas p150
  \item Αν ο πληθυσμός σε μια καλλιέργεια μυκήτων τριπλασιάζεται κάθε επτά ημέρες, να 
    βρεθεί πόσος θα είναι ο πληθυσμός σε 35 ημέρες καθώς και πόσος χρόνος χρειάζεται 
    ώστε ο αρχικός πληθυσμός να δεκαπλασιαστεί.
    \hfill Απ: $ y(35)=243 y_{0} $, $ t_{1}=14,7 $ 

    %Siafarikas p150
  \item Σε μια καλλιέργεια κυττάρων μετά από μια ημέρα απομένουν τα 2/3 αυτών.
    Χρησιμοποιώντας αυτήν την πληροφορία να καθορισθεί ο αριθμός των ημερών που πρέπει να
    περάσουν ώστε ο αρχικός πληθυσμός να μειωθεί στο 1/3.
    \hfill Απ: $ t_{1}=2,7 $ 
\end{enumerate}



\end{document}
