\documentclass[a4paper,12pt]{article}
\usepackage{etex}
%%%%%%%%%%%%%%%%%%%%%%%%%%%%%%%%%%%%%%
% Babel language package
\usepackage[english,greek]{babel}
% Inputenc font encoding
\usepackage[utf8]{inputenc}
%%%%%%%%%%%%%%%%%%%%%%%%%%%%%%%%%%%%%%

%%%%% math packages %%%%%%%%%%%%%%%%%%
\usepackage{amsmath}
\usepackage{amssymb}
\usepackage{amsfonts}
\usepackage{amsthm}
\usepackage{proof}

\usepackage{physics}

%%%%%%% symbols packages %%%%%%%%%%%%%%
\usepackage{bm} %for use \bm instead \boldsymbol in math mode 
\usepackage{dsfont}
\usepackage{stmaryrd}
%%%%%%%%%%%%%%%%%%%%%%%%%%%%%%%%%%%%%%%


%%%%%% graphicx %%%%%%%%%%%%%%%%%%%%%%%
\usepackage{graphicx}
\usepackage{color}
%\usepackage{xypic}
\usepackage[all]{xy}
\usepackage{calc}
\usepackage{booktabs}
\usepackage{minibox}
%%%%%%%%%%%%%%%%%%%%%%%%%%%%%%%%%%%%%%%

\usepackage{enumerate}

\usepackage{fancyhdr}
%%%%% header and footer rule %%%%%%%%%
\setlength{\headheight}{14pt}
\renewcommand{\headrulewidth}{0pt}
\renewcommand{\footrulewidth}{0pt}
\fancypagestyle{plain}{\fancyhf{}
\fancyhead{}
\lfoot{}
\rfoot{\small \thepage}}
\fancypagestyle{vangelis}{\fancyhf{}
\rhead{\small \leftmark}
\lhead{\small }
\lfoot{}
\rfoot{\small \thepage}}
%%%%%%%%%%%%%%%%%%%%%%%%%%%%%%%%%%%%%%%

\usepackage{hyperref}
\usepackage{url}
%%%%%%% hyperref settings %%%%%%%%%%%%
\hypersetup{pdfpagemode=UseOutlines,hidelinks,
bookmarksopen=true,
pdfdisplaydoctitle=true,
pdfstartview=Fit,
unicode=true,
pdfpagelayout=OneColumn,
}
%%%%%%%%%%%%%%%%%%%%%%%%%%%%%%%%%%%%%%

\usepackage[space]{grffile}

\usepackage{geometry}
\geometry{left=25.63mm,right=25.63mm,top=36.25mm,bottom=36.25mm,footskip=24.16mm,headsep=24.16mm}

%\usepackage[explicit]{titlesec}
%%%%%% titlesec settings %%%%%%%%%%%%%
%\titleformat{\chapter}[block]{\LARGE\sc\bfseries}{\thechapter.}{1ex}{#1}
%\titlespacing*{\chapter}{0cm}{0cm}{36pt}[0ex]
%\titleformat{\section}[block]{\Large\bfseries}{\thesection.}{1ex}{#1}
%\titlespacing*{\section}{0cm}{34.56pt}{17.28pt}[0ex]
%\titleformat{\subsection}[block]{\large\bfseries{\thesubsection.}{1ex}{#1}
%\titlespacing*{\subsection}{0pt}{28.80pt}{14.40pt}[0ex]
%%%%%%%%%%%%%%%%%%%%%%%%%%%%%%%%%%%%%%

%%%%%%%%% My Theorems %%%%%%%%%%%%%%%%%%
\newtheorem{thm}{Θεώρημα}[section]
\newtheorem{cor}[thm]{Πόρισμα}
\newtheorem{lem}[thm]{λήμμα}
\theoremstyle{definition}
\newtheorem{dfn}{Ορισμός}[section]
\newtheorem{dfns}[dfn]{Ορισμοί}
\theoremstyle{remark}
\newtheorem{remark}{Παρατήρηση}[section]
\newtheorem{remarks}[remark]{Παρατηρήσεις}
%%%%%%%%%%%%%%%%%%%%%%%%%%%%%%%%%%%%%%%




\input{definitions_ask.tex}

\pagestyle{askhseis}
\everymath{\displaystyle}

\begin{document}

\begin{center}
  \minibox{\large\bf \textcolor{Col1}{Ασκήσεις Τριπλο Ολοκλήρωμα}}
\end{center}

\vspace{\baselineskip}

\section*{Παραλληλεπίπεδα Χωρία}
 

\begin{enumerate}
  \item Να υπολογιστουν τα παρακατω τριπλα ολοκληρωματα:
    \begin{enumerate}[i)]
      %spandagos p.251
      \item $ I=\int_{0}^{3}\!\!\int_{0}^{2}\!\!\int_{0}^{1}(x+y+z)\,dzdxdy $ 
        \hfill Απ: 18  
      \item $  I=\int_{1}^{2}\!\!\int_{3}^{4}\!\!\int_{5}^{6}xy\,dzdydx  $ 
        \hfill Απ: $ \frac{21}{4} $ 
        %spandagos p.251
      \item $ I=\int_{0}^{1}\!\!\int_{0}^{1}\!\!\int_{0}^{1}e^{x+y+z}\,dxdydz $ 
        \hfill Απ: $ (e-1)^{3} $
    \end{enumerate}
\end{enumerate}
    
    \section*{Γενικά Χωρία}

\begin{enumerate}
        %spandagos p.251
      \item $ I=\int_{-1}^{1}\!\!\int_{0}^{z}\!\int_{x-z}^{x+z}(x+y+z)\,dydxdz $ 
        \hfill Απ: 0  
        %spandagos p.278
      \item $ I=\iiint_{\Omega}(x+1)\,dxdydz $, \quad $ \Omega = \{(x,y,z)\in 
          \mathbb{R}^{3} \mid 0 \leq x \leq 1,\; 0 \leq y \leq x,\; -y^{2} \leq z 
        \leq x^{2} \}$ 
        \hfill Απ: $ \frac{3}{5} $ 
        %spandagos p.256
      \item $ I=\iiint_{\Omega} (x+y+z)\,dxdydz $, \quad $ \Omega= \{(x,y,z)\in 
        \mathbb{R}^{3} \mid x+y+z \leq 1,\; x \geq 0,\; y \geq 0,\; z \geq 0 \}  $ 
        \hfill Απ: $ \frac{1}{8} $ 
        %spandagos p.279
      \item $ I=\iiint_{\Omega} y\,dxdydz $, \quad $ \Omega = \{ (x,y,z) \in 
          \mathbb{R}^{3} \mid -5 + x^{2}+y^{2} \leq z \leq 3 - x^{2}-y^{2}, \; x 
        \geq 0, \; y \geq 0\}$ 
        \hfill Απ: $ \frac{128}{15} $  
        %spandagos p.283
      \item $ I = \iiint_{\Omega} (x^{2}+y^{2}+z) \,dxdydz $, \quad $\Omega = \{ 
          (x,y,z) \in \mathbb{R}^{3} \mid x^{2}+y^{2} \leq 1,\; z \leq x^{2}+y^{2}, 
        \; x \geq 0, y \geq 0, z \geq 0\} $ 

        \hfill Απ: $ \frac{\pi}{8} $ 
    \end{enumerate}


    \section*{Εφαρμογές του Τριπλού Ολοκληρώματος}

  \begin{enumerate}
      %spandagos p.258
  \item Να υπολογιστεί ο \textbf{όγκος} του παραλληλεπιπέδου με κορυφές τα σημεία 
    $ (0,0,0) $, $ (3,0,0) $, $ (0,2,0) $ και $ (3,2,1) $. 

    \hfill Απ: 9/2 

  \item Να υπολογιστεί ο \textbf{όγκος} του στερεού στο 1ο ογδοημόριο που περικλείεται 
    από τις επιφάνειες $y+z=2$ και $x=4-y^{2}$. 

    \hfill Απ: 20/3 

  \item Να υπολογιστεί ο \textbf{όγκος} του στερεού στο 1ο ογδοημόριο που περικλείεται 
    από τις επιφάνειες $ x+z=1 $ και $ y+2z=2 $. 

    \hfill Απ: 2/3 

      %spandagos p.263
  \item Να υπολογιστεί ο \textbf{όγκος} του στερεού που περικλείεται από τον κύλινδρο 
    $ x^{2}+y^{2}=1 $ και από τα επίπεδα $ z=0 $ και $ x+y+z=2 $, με 
    $ x \geq 0, \; y \geq 0, \; z \geq 0$ 

    \hfill Απ: $ {\pi}/{2} - {2}/{3} $ 

      %spandagos p.309
  \item Να υπολογιστεί ο \textbf{όγκος} του στερεού που περικλείεται από το 
    παραβολοειδές  $ z = (x-1)^{2}+y^{2} $ και το επίπεδο $ 2x+z=2 $.
   
    \hfill Απ: $ {\pi}/{2} $

    %spandagos p.293 
  \item Να υπολογιστεί ο \textbf{όγκος} του στερεού που περικλείεται από τα παραβολοειδή 
    $ z = 4 -y^{2} $, $ z = 2x^{2}+y^{2} $ στο 1ο τεταρτημόριο. 
   
    \hfill Απ: $ \pi $ 
\end{enumerate}


\end{document}
