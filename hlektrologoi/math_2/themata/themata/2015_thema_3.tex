\documentclass[a4paper,table]{report}
\input{preamble_themata}
\newcommand{\vect}[2]{(#1_1,\ldots, #1_#2)}
%%%%%%% nesting newcommands $$$$$$$$$$$$$$$$$$$
\newcommand{\function}[1]{\newcommand{\nvec}[2]{#1(##1_1,\ldots, ##1_##2)}}

\newcommand{\linode}[2]{#1_n(x)#2^{(n)}+#1_{n-1}(x)#2^{(n-1)}+\cdots +#1_0(x)#2=g(x)}

\newcommand{\vecoffun}[3]{#1_0(#2),\ldots ,#1_#3(#2)}

\newcommand{\suma}{\sum_{n=0}^{\infty}a_n x^n}

\newcommand{\sumb}{\sum_{n=1}^{\infty}a_n n x^{n-1}}

\newcommand{\sumc}{\sum_{n=2}^{\infty}a_n n (n-1) x^{n-2}}

\newcommand{\varsum}[2]{\sum_{n=#1}^{#2}}
\input{myboxes}
\input{tikz}

\geometry{left=12.5mm,right=12.5mm,top=30.25mm,bottom=30.25mm,footskip=24.16mm,headsep=24.16mm}

\newcommand{\twocolumnsidelcc}[2]{\begin{minipage}[c]{0.35\linewidth}\raggedright
        #1
        \end{minipage}\hfill\begin{minipage}[c]{0.60\linewidth}\raggedright
        #2
    \end{minipage}
}

\mdfdefinestyle{myboxstyle1}{linecolor=Col1!75,linewidth=0pt,
  backgroundcolor=Col1!25, %background color of the box
  shadow=false,shadowcolor=Col1,shadowsize=5pt,% shadows
}


\newcommand{\twocolumnsiden}[2]{\vspace{0.5\baselineskip}\begin{minipage}[t]{0.30\linewidth}
    #1
    \end{minipage}\begin{minipage}[t]{0.30\linewidth}
    #2
  \end{minipage}
  \vspace{0.5\baselineskip}
}

\everymath{\displaystyle}
\pagestyle{vangelis}


\begin{document}

\begin{mybox3}
  \begin{thema}
        Θεωρούμε το επίπεδο σύστημα συντεταγμένων $ Ox'y' $, το οποίο προκύπτει από την 
        αριστερόστροφη περιστροφή του συστήματος $ Oxy $ κατά γωνία $\theta$. 
        Έστω επίσης $ \mathbf{r} $ το διάνυσμα θέσης του σημείου $ P(x,y) $. 
    \begin{enumerate}[i)]
      \item Να βρείτε τη σχέση που συνδέει τις συντεταγμένες στα δύο αυτά συστήματα.
      \item Με κατάλληλη περιστροφή και μεταφορά των αξόνων, να βρεθεί σε κανονική μορφή 
        η έλλειψη:
        \[
          5x^{2} + 5y^{2} + 8xy - 18x - 18y +9 = 0
        \] 
    \end{enumerate}
  \end{thema}
\end{mybox3}
\begin{solution}
\item {}
  \twocolumnsidelcc{
    \includegraphics[]{coordinates.pdf}
  }{
    \begin{enumerate}[i)]
      \item Από τα ορθογώνια τρίγωνα $ OxP $ και $ Ox'P $ έχουμε αντίστοιχα:
    \end{enumerate}
    \[
      \begin{matrix}
        x = r \cos{\omega} \\
        y = r \sin{\omega}
      \end{matrix} 
      \quad \text{και} \quad 
      \begin{matrix}
        x' = r \cos{(\omega - \theta)} = \overbrace{r \cos{\omega}}^{x} \cos{\theta} +
        \overbrace{r \sin{\omega}}^{y}
        \sin{\theta} \\
        y' = r \sin{(\omega - \theta)} = \underbrace{r \sin{\omega}}_{y} \cos{\theta} -
        \underbrace{r \cos{\omega}}_{x}
        \sin{\theta}
      \end{matrix} 
    \]
    όπου χρησιμοποιήσαμε τις γνωστές τριγωνομετρικές ταυτότητες
    \begin{mybox1}
      \[
        \begin{aligned}
          \cos{(a-b)} = \cos{a} \cos{b} + \sin{a} \sin{b} \\
          \sin{(a-b)} = \sin{a} \cos{b} - \cos{a} \sin{b}
        \end{aligned}
      \]
    \end{mybox1}
  }

  Επομένως
  \[
    \left.
      \begin{matrix}
        x' = x \cos{\theta} + y \sin{\theta} \\
        y' = y \cos{\theta} - x \sin{\theta}
      \end{matrix} 
    \right\} 
    \Leftrightarrow 
    \begin{pmatrix*}[r] x' \\ y' \end{pmatrix*} = 
    \begin{pmatrix*}[r]
      \cos{\theta} & \sin{\theta} \\
      - \sin{\theta} & \cos{\theta}
    \end{pmatrix*}
    \cdot 
    \begin{pmatrix*}[r] x \\ y \end{pmatrix*} 
    \Leftrightarrow 
    X' = A\cdot X
  \] 
  Όμως ισχύει ότι $ X = A^{-1} \cdot X' $ όπου $ A^{-1} = 
  \begin{pmatrix*}[r]
    \cos{\theta} & - \sin{\theta} \\
    \sin{\theta} & \cos{\theta}
  \end{pmatrix*} $ 
  είναι ο αντίστροφος πίνακας του $A$.
  \begin{mybox1}
    Θυμάμαι, αν $ A=  
    \begin{pmatrix*}[r]
      a & b \\
      c & d
    \end{pmatrix*}
    $ με $ \abs{A} = 
    \begin{vmatrix*}[r]
      a & b \\
      c & d
    \end{vmatrix*} = ad - bc \neq 0
    $, τότε $ A^{-1} = \frac{1}{\abs{A}}
    \begin{pmatrix*}[r]
      d & - b \\
      -c & a
    \end{pmatrix*}
    $ είναι ο αντίστροφος του $A$.
  \end{mybox1}

  Άρα 
  \[
    \begin{pmatrix*}[r] x \\ y \end{pmatrix*} = 
    \begin{pmatrix*}[r]
      \cos{\theta} & - \sin{\theta} \\
      \sin{\theta} & \cos{\theta}
    \end{pmatrix*}
    \cdot 
    \begin{pmatrix*}[r] x' \\ y' \end{pmatrix*} 
    \Leftrightarrow 
        \left.
          \begin{matrix}
            x = x' \cos{\theta} - y' \sin{\theta} \\
            y = x' \sin{\theta} + y' \cos{\theta}
          \end{matrix}
      \right\}
  \] 
  Επομένως οι σχέσεις που συνδέουν τις συντεταγμένες του τυχαίου σημείου $ P(x,y) $ 
  στα δύο αυτά συστήματα, είναι:
  \begin{empheq}[box=\mathboxg]{equation*}
    \left.
      \begin{matrix}
        x' = x \cos{\theta} + y \sin{\theta} \\
        y' = y \cos{\theta} - x \sin{\theta}
      \end{matrix} 
    \right\} 
    \quad \text{ή} \quad 
    \left.
      \begin{matrix}
        x = x' \cos{\theta} - y' \sin{\theta} \\
        y = x' \sin{\theta} + y' \cos{\theta}
      \end{matrix} 
    \right\} 
  \end{empheq}
  ή ισοδύναμα με τη βοήθεια πινάκων
  \begin{empheq}[box=\mathboxg]{equation*}
    \begin{pmatrix*}[r] x' \\ y' \end{pmatrix*} = 
    \begin{pmatrix*}[r]
      \cos{\theta} & \sin{\theta} \\
      - \sin{\theta} & \cos{\theta}
    \end{pmatrix*}
    \cdot 
    \begin{pmatrix*}[r] x \\ y \end{pmatrix*} 
    \quad \text{ή} \quad 
    \begin{pmatrix*}[r] x \\ y \end{pmatrix*} = 
    \begin{pmatrix*}[r]
      \cos{\theta} & - \sin{\theta} \\
      \sin{\theta} & \cos{\theta}
    \end{pmatrix*}
    \cdot 
    \begin{pmatrix*}[r] x' \\ y' \end{pmatrix*} 
  \end{empheq}

  \begin{enumerate}[i),start=2]
    \item Η εξίσωση της έλλειψης $ 5x^2+5y^2+8xy-18x-18y+9=0 $, γίνεται:
      \begin{equation*}
        \begin{split}
          5(x' \cos{\theta} &- y' \sin{\theta} )^2+5(x' \sin{\theta +y' \cos{\theta} } )^2+
          8(x' \cos{\theta} -y' \sin{\theta} )(x' \sin{\theta} +y' \cos{\theta}) - 18
          (x' \cos{\theta -y' \sin{\theta}} ) \\
                            &-18(x' \sin{\theta} +y' \cos{\theta} ) +9=0 \Leftrightarrow 
        \end{split}
      \end{equation*}
      \begin{equation*}
        \begin{split}
          5({x'}^2 \cos^{2}{\theta} &+ {y'}^2 \sin^{2}{\theta} -2x'y' \sin{\theta \cos{\theta}
          } )+5({x'}^2 \sin^{2}{\theta} +{y'}^2 \cos^{2}{\theta} + 2x'y' \sin{\theta}
          \cos{\theta} ) + 8{x'}^2 \sin{\theta} \cos{\theta} \\ 
          &-8{y'}^2 \sin{\theta} \cos{\theta}+8x'y' \cos^{2}{\theta} -8x'y' \sin^{2}{\theta} - 18x' \cos{\theta}
          +18y' \sin{\theta} -18x' \sin{\theta} -18y' \cos{\theta} +9 = 0 \Leftrightarrow 
      \end{split}
    \end{equation*}
    \begin{equation*}
      {x'}^2(5+8 \sin{\theta} \cos{\theta} )+{y'}^2(5-8 \sin{\theta}
      \cos{\theta})+8x'y'\underbrace{(\cos^{2}{\theta} - \sin^{2}{\theta}
      )}_{\text{θέλω } 0}-18x'(\cos{\theta} +
      \sin{\theta})+18y'(- \cos{\theta} + \sin{\theta} )+9=0 
    \end{equation*}
    Θέτουμε $ \cos^{2}{\theta} = \sin^{2}{\theta} \overset{\theta \in
    [0, \pi /2]}{\Leftrightarrow}
    \cos{\theta} = \sin{\theta} \Leftrightarrow \theta = \frac{\pi}{4} $. Άρα 
    για $ \theta = \frac{\pi}{4} $, έχουμε ότι $ \cos{\theta} = \sin{\theta} =
    \frac{\sqrt{2}}{2} $ και η εξίσωση, μετά τη στροφή του συστήματος συντεταγμένων,
    κατά γωνία $ \theta $, γίνεται: 
    \begin{equation*}
      9{x'}^2+{y'}^2-18 \sqrt{2} x'+9=0 
    \end{equation*}
    Κάνουμε μεταφορά της αρχής του συστήματος συντεταγμένων, θέτοντας:
    \[
      \left.
        \begin{matrix}
          x''=x'- x_{0} \\
          y''=y'- y_{0}
        \end{matrix} 
      \right\} \Leftrightarrow 
      \left.
        \begin{matrix}
          x'= x_{0}+ x'' \\
          y'= y_{0}+ y''
        \end{matrix} 
      \right\}
    \]
    και προκύπτει:
    \begin{align*}
      9(x_{0}+x'')^2+(y_{0}+y'')^2-18 \sqrt{2} (x_{0}+x'')+9=0 \Leftrightarrow \\
      9 x_{0}^2+ 18 x_{0}x'' +9{x''}^2+ y_{0}^2+2 y_{0}y''+{y''}^2- 18 \sqrt{2} x_{0}- 18
      \sqrt{2} x''+9=0 \Leftrightarrow \\
      9{x''}^2+{y''}^2+x''\underbrace{(18 x_{0}-18 \sqrt{2})}_{\text{θέλω }0}+
      y''\underbrace{2y_{0}}_{\text{θέλω }0}+9 x_{0}^2-18 \sqrt{2} x_{0}+ y_{0}^2+9=0
      \Leftrightarrow \\
    \end{align*}
    Οπότε, θέτουμε:
    \[
      \left.
        \begin{matrix}
          18 x_{0}= 18 \sqrt{2} \\
          2 y_{0}= 0 
        \end{matrix} 
      \right\} \Leftrightarrow 
      \left.
        \begin{matrix}
          x_{0} = \sqrt{2} \\
          y_{0}=0
        \end{matrix} 
      \right\} 
    \]
    Οπότε η εξίσωση της έλλειψης, γίνεται:
    \[
      9{x''}^2+{y''}^2 + 9 (\sqrt{2} )^2 - 18 \sqrt{2} \sqrt{2} + 9 = 0 \Leftrightarrow 
      9{x''}^2+{y''}^2=9 \Leftrightarrow 
      {x''}^2+ \frac{{y''}^2}{3^2} = 1
    \]
    Άρα η κανονική μορφή της εξίσωσης της έλλειψης, είναι:
    \begin{empheq}[box=\mathboxg]{equation*}
      {x''}^2+ \frac{{y''}^2}{3^2} = 1
    \end{empheq}
\end{enumerate}
\end{solution}


\end{document}
