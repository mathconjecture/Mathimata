\documentclass[a4paper,12pt]{article}
\usepackage{etex}
%%%%%%%%%%%%%%%%%%%%%%%%%%%%%%%%%%%%%%
% Babel language package
\usepackage[english,greek]{babel}
% Inputenc font encoding
\usepackage[utf8]{inputenc}
%%%%%%%%%%%%%%%%%%%%%%%%%%%%%%%%%%%%%%

%%%%% math packages %%%%%%%%%%%%%%%%%%
\usepackage{amsmath}
\usepackage{amssymb}
\usepackage{amsfonts}
\usepackage{amsthm}
\usepackage{proof}

\usepackage{physics}

%%%%%%% symbols packages %%%%%%%%%%%%%%
\usepackage{bm} %for use \bm instead \boldsymbol in math mode 
\usepackage{dsfont}
\usepackage{stmaryrd}
%%%%%%%%%%%%%%%%%%%%%%%%%%%%%%%%%%%%%%%


%%%%%% graphicx %%%%%%%%%%%%%%%%%%%%%%%
\usepackage{graphicx}
\usepackage{color}
%\usepackage{xypic}
\usepackage[all]{xy}
\usepackage{calc}
\usepackage{booktabs}
\usepackage{minibox}
%%%%%%%%%%%%%%%%%%%%%%%%%%%%%%%%%%%%%%%

\usepackage{enumerate}

\usepackage{fancyhdr}
%%%%% header and footer rule %%%%%%%%%
\setlength{\headheight}{14pt}
\renewcommand{\headrulewidth}{0pt}
\renewcommand{\footrulewidth}{0pt}
\fancypagestyle{plain}{\fancyhf{}
\fancyhead{}
\lfoot{}
\rfoot{\small \thepage}}
\fancypagestyle{vangelis}{\fancyhf{}
\rhead{\small \leftmark}
\lhead{\small }
\lfoot{}
\rfoot{\small \thepage}}
%%%%%%%%%%%%%%%%%%%%%%%%%%%%%%%%%%%%%%%

\usepackage{hyperref}
\usepackage{url}
%%%%%%% hyperref settings %%%%%%%%%%%%
\hypersetup{pdfpagemode=UseOutlines,hidelinks,
bookmarksopen=true,
pdfdisplaydoctitle=true,
pdfstartview=Fit,
unicode=true,
pdfpagelayout=OneColumn,
}
%%%%%%%%%%%%%%%%%%%%%%%%%%%%%%%%%%%%%%

\usepackage[space]{grffile}

\usepackage{geometry}
\geometry{left=25.63mm,right=25.63mm,top=36.25mm,bottom=36.25mm,footskip=24.16mm,headsep=24.16mm}

%\usepackage[explicit]{titlesec}
%%%%%% titlesec settings %%%%%%%%%%%%%
%\titleformat{\chapter}[block]{\LARGE\sc\bfseries}{\thechapter.}{1ex}{#1}
%\titlespacing*{\chapter}{0cm}{0cm}{36pt}[0ex]
%\titleformat{\section}[block]{\Large\bfseries}{\thesection.}{1ex}{#1}
%\titlespacing*{\section}{0cm}{34.56pt}{17.28pt}[0ex]
%\titleformat{\subsection}[block]{\large\bfseries{\thesubsection.}{1ex}{#1}
%\titlespacing*{\subsection}{0pt}{28.80pt}{14.40pt}[0ex]
%%%%%%%%%%%%%%%%%%%%%%%%%%%%%%%%%%%%%%

%%%%%%%%% My Theorems %%%%%%%%%%%%%%%%%%
\newtheorem{thm}{Θεώρημα}[section]
\newtheorem{cor}[thm]{Πόρισμα}
\newtheorem{lem}[thm]{λήμμα}
\theoremstyle{definition}
\newtheorem{dfn}{Ορισμός}[section]
\newtheorem{dfns}[dfn]{Ορισμοί}
\theoremstyle{remark}
\newtheorem{remark}{Παρατήρηση}[section]
\newtheorem{remarks}[remark]{Παρατηρήσεις}
%%%%%%%%%%%%%%%%%%%%%%%%%%%%%%%%%%%%%%%




\newcommand{\vect}[2]{(#1_1,\ldots, #1_#2)}
%%%%%%% nesting newcommands $$$$$$$$$$$$$$$$$$$
\newcommand{\function}[1]{\newcommand{\nvec}[2]{#1(##1_1,\ldots, ##1_##2)}}

\newcommand{\linode}[2]{#1_n(x)#2^{(n)}+#1_{n-1}(x)#2^{(n-1)}+\cdots +#1_0(x)#2=g(x)}

\newcommand{\vecoffun}[3]{#1_0(#2),\ldots ,#1_#3(#2)}

\newcommand{\mysum}[1]{\sum_{n=#1}^{\infty}


\pagestyle{askhseis}

\begin{document}

\begin{center}
  \minibox{\bfseries\large \textcolor{Col1}{Ασκήσεις Επανάληψης}}
\end{center}

\vspace{\baselineskip}

\begin{enumerate}
  \item Βρείτε τις συντεταγμένες των σημείων όπου η ευθεία $ x-2y=2 $ τέμνει 
    τους άξονες.
  \item Βρείτε το σημείο τομής των ευθειών, $4x+3y=11$ και $2x+y=5$.
    \hfill Απ: $ x=2,\; y=1 $ 
  \item Προσδιορίστε ποιες δυο από τις παρακάτω ευθείες είναι παράλληλες.
    \begin{enumerate}[i)]
      \item $ 3x+5y=2 $
      \item $ 5x-3y=1 $
      \item $ 5x+3y=13 $
      \item $ 10x-6y=9 $
      \item $ y=0,6x+2 $ \hfill Απ: ii) και iv) 
    \end{enumerate}
  \item Ένα αρτοποιείο ανακάλυψε ότι αν μειώσει την τιμή στις τούρτες
    γενεθλίων κατά 1 ευρώ, πουλάει 12 παραπάνω τούρτες τον μήνα.
    \begin{enumerate}[i)]
      \item Υποθέτωντας ότι οι μηνιαίες πωλήσεις, $M$ σχετίζονται
        με τις τιμές, $P$ , με μία γραμμική σχέση της μορφής
        $ M=aP+b $, υπολογίστε την τιμή $ a $.

      \item Αν το αρτοποιείο πουλάει 240 τούρτες τον μήνα, υπολογίστε
        την τιμή του $b$ αν η τιμή της τούρτας είναι 14 ευρώ.

      \item Εκτιμήστε τις μηνιάιες πωλήσεις όταν η τιμή είναι 9 ευρώ.

      \item Αν το αρτοποιείο μπορεί να φτιάξει μόνο 168 τούρτες
        υπολογίστε την τιμή που χρειάζεται έτσι ώστε να τις πουλήσεις
        όλες.
        \hfill Απ: $ a=-12,\; b=408,\; M=300,\; P=20 $ 
    \end{enumerate}


  \item Μια τράπεζα προσφέρει $ 7\% $ ετήσιο ανατοκισμό. Υπολογίστε 
    την μελλοντική αξία ενός κεφαλαίου 4500 ευρώ μετά από 6 χρόνια. 
    Ποια η συνολική ποσοστιαία αύξηση κατά τη διάρκεια αυτής της 
    περιόδου; (\textbf{Υποδειξη:} ποσοστιαία μεταβολή = 
    $ (\frac{x_{\text{τελ.}}}{ x_{\text{αρχ.}}} - 1)\cdot 100\% $)
    \hfill Απ: 6753,29 ευρώ, $ 50\% $ 

  \item Υπολογίστε την μελλοντική αξία που θα έχουν τα 20000 ευρώ σε δύο 
    χρόνια αν τοκίζονται τριμηνιαία με τόκο $ 8\% $.

    \hfill Απ: 23433,19 ευρώ 


  \item Πόσο καιρό θα χρειαστεί ένα χρηματικό ποσό για να διπλασιασθεί αν 
    επενδυθεί με $ 5\% $ ετήσιο ανατοκισμό;

    \hfill Απ: 15 χρόνια 

  \item Υπάρχει υποτίμηση στην αξία ενός μηχανήματος κατά $ 5\% $ κάθε 
    χρόνο. Καθορίστε την αξία του σε 3 χρόνια αν η τρέχουσα αξία του
    είναι 50000 ευρώ.
    (\textbf{Υπόδειξη:} για υποτίμηση παρε $ K=K_{0}(1- \frac{r}{100})^{n} $)
    \hfill Απ: 42868,75 ευρώ 

  \item Ένα κεφάλαιο, 7000 ευρώ, επενδύεται με τόκο $ 9\% $ για 8 
    χρόνια. Καθορίστε την μελλοντική του αξία αν ο ανατοκισμός είναι
    ετήσιος, εξαμηνιαίος, μηνιαίος και συνεχής.
    \hfill Απ: $ 13947,94,\; 14156,59,\; 14342,45,\; 14381,03 $

  \item Ποιος από τους ακόλουθους ταμιευτικούς λογαριασμούς προσφέρει 
    καλύτερη απόδοση;
    Λογαριασμός Α με ετήσιο επιτόκιο $ 8,05\% $ και εξαμηνιαίο 
    ανατοκισμό ή Λογαριασμός Β με ετήσιο επιτόκιο $ 7,95\% $ και 
    μηνιαίο ανατοκισμό;

    \hfill Απ: $B$ 

  \item Υπολογίστε την μελλοντική αξία του ποσού των 100 ευρώ με 
    συνεχή ανατοκισμό $ 6\% $ για 12 χρόνια.

    \hfill Απ: 205,44 ευρώ 

  \item Πόσος καιρός χρειάζεται για να τριπλασιαστεί σε αξία ένα 
    κεφάλαιο αν επενδυθεί με ετήσιο επιτόκιο $ 3\% $ και συνεχή
    ανατοκισμό;
    \hfill Απ: 36,6 χρόνια 

  \item Αν υπάρχει συνεχής υποτίμηση της τάξης του $ 4\% $ σε ένα
    μηχάνημα, πόσα χρόνια χρειάζονται για να υποδιπλασιαστεί η αξία του;
    (υποδειξη: για συνεχή υποτίμηση παρε $ K=K_{0}e^{-rt} $)
    \hfill Απ: 17 χρόνια 


  \item Η συνάρτηση ζήτησης ενός αγαθού είναι $ P+Q = 30 $ και η συνάρτηση 
    συνολικού κόστους είναι $ TC = \frac{1}{2} Q^{2} + 6Q + 7 $.
    Υπολογίστε το επίπεδο παραγωγής $ Q $ που μεγιστοποιεί τα έσοδα.
    Υπολογίστε το επίπεδο παραγωγής $ Q $ που μεγιστοποιεί το κέρδος.
    (υποδειξη: (Εσοδα) $TR = PQ$, (Κερδος) $\text{Π} = TR-TC$)

    \hfill Απ: $ TR= 30Q-Q^{2},\; Q_{1}^{*}=15, \text{Π} = - \frac{3}{2} Q^{2} + 
    24Q -7,\; Q_{2}^{*}=8 $ 

\end{enumerate}

\end{document}
