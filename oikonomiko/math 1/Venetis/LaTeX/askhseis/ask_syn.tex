\input{preamble_ask.tex}
\input{definitions_ask.tex}
\input{tikz.tex}

\pagestyle{vangelis}

\begin{document}

\begin{center}
  \minibox{\large\bfseries \textcolor{Col1}{Ασκήσεις στα Σύνολα}}
\end{center}

\vspace{\baselineskip}

\begin{enumerate}

  \item Έστω τα σύνολα $A=\{1,2,3\}, B=\{1,2\}, C=\{1,3\}, D=\{2,3\}, E=\{1\}, 
    F=\{2\}, G=\{3\}$. Να βρεθούν τα παρακάτω σύνολα.
    \begin{enumerate}[label=(\roman*)]
      \item $A\cup B$ \hfill Απ: $ \{ 1,2,3 \} $ 
      \item $A\cap B$ \hfill Απ: $ \{ 1,2 \} $ 
      \item $A\cap (B\cap C)$ \hfill Απ: $ \{ 1 \} $ 
      \item $(C\cup A)\cap B$ \hfill Απ: $ \{ 1,2 \} $ 
      \item $A\setminus B$ \hfill Απ: $ \{ 3 \} $ 
      \item $C\setminus A$ \hfill Απ: $ \emptyset $  
      \item $D\triangle F = (D\setminus F)\cup(F\setminus D)$ \hfill Απ: $ \{ 3 \} $ 
    \end{enumerate}

  \item {\bfseries(Θέμα Σεπ '15)} Δίνονται τα παρακάτω σύνολα. Να υπολογίσετε 
    τα καρτεσιανά γινόμενα $A_i\times B_i$, για $i=1,2,3$ και σχεδιάστε τα γινόμενα 
    σε ένα \textbf{γράφημα} στο $\mathbb{R}^2$.

    \twocolumnsidesc{
      \begin{tabular}{l}
        $A_1=\{x:x\in\{1\}\}$\\
        $A_2=\{x:x\in[1,3]\}$ \\
        $A_3=\{x:x\in \{3\}\}$ \\
        $B_1=\{y:y\in[1,2]\}$\\
        $B_2=\{y:y=2.5-0.5x, \; \text{όπου} \; x \in A_{2}\}$ \\
        $B_3=\{y:y\in[1,3]$\}
      \end{tabular}
    }
    {
      \hfill Απ: 
      \begin{tikzpicture}[baseline=0pt,scale=0.7]
        \coordinate (0) at (0, 0) ;
        \coordinate (1) at (0, 4) ;
        \node at (1) [left] {$y$} ;
        \coordinate (2) at (4, 0) ;
        \node at (2) [below] {$x$} ;
        \coordinate (3) at (0, 2) ;
        \node at (3) [left] {$2$} ;
        \coordinate (4) at (0, 1) ;
        \node at (4) [left] {$1$} ;
        \coordinate (5) at (1, 0) ;
        \node at (5) [below] {$1$} ;
        \coordinate (6) at (3, 0) ;
        \node at (6) [below] {$3$} ;
        \coordinate (7) at (1, 2) ;
        \coordinate (8) at (1, 1) ;
        \coordinate (9) at (0, 3) ;
        \node at (9) [left] {$3$} ;
        \coordinate (10) at (3, 3) ;
        \coordinate (11) at (3, 1) ;

        \draw[-latex] (0.center) to (2.center);
        \draw[-latex] (0.center) to (1.center);
        \draw[ultra thick,Col1] (7.center) to node[midway,left,xshift=1pt] {\tiny{$A_{1} 
          \times B_{1}$}} (8.center) ;
        \draw[ultra thick,Col1] (10.center) to node[midway,right] {\tiny{$A_{3} 
          \times B_{3}$}}(11.center);
        \draw[ultra thick,Col2] (7.center) to node[midway,above,xshift=5pt] 
          {\tiny{$A_{2} \times B_{2}$}} (11.center);
        \draw[dashed] (8) -- (4) ;
        \draw[dashed] (7) -- (3) ;
        \draw[dashed] (8) -- (5) ;
        \draw[dashed] (11) -- (6) ;
        \draw[dashed] (10) -- (9) ;
        \draw[dashed] (11) -- (8) ;
        \draw[fill,Col1] (7) circle[r=2pt] ;
        \draw[fill,Col1] (8) circle[r=2pt] ;
        \draw[fill,Col1] (10) circle[r=2pt] ;
        \draw[fill,Col1] (11) circle[r=2pt] ;
      \end{tikzpicture}
    }
    

  \item Να βρεθεί η \textbf{ένωση} και η \textbf{τομή} των συνόλων $A$ και $B$ στις 
    παρακάτω περιπτώσεις.
    \begin{enumerate}[(\roman*)]
      \item Αν είναι $A=\mathbb{R}\setminus\{1,2\}$ και $B=\mathbb{R}\setminus\{1,3\}$
        \hfill Απ: $ A \cup B = \mathbb{R} \setminus \{ 1 \}, \; A \cap B = \mathbb{R}
        \setminus \{ 1,2,3 \} $ 
      \item Αν είναι $A=\mathbb{R}\setminus\{1,2\}$ και $B=[1,+\infty) \setminus 
        \{ 3 \} $
        \hfill Απ: $ A \cup B = \mathbb{R}, \; A \cap B = (1,+\infty) 
        \setminus \{ 2,3 \}$ 
      \item Αν είναι $A=(3,+\infty)$ και $B=(-\infty,5]$
        \hfill Απ: $ A \cup B = \mathbb{R}, \; A \cap B = (3,5]  $ 
    \end{enumerate}

  \item Δίνεται το γενικό σύνολο $S=\{x\in \mathbb{N} : 0<x<10\}$ και τα υποσύνολα 
    του $A=\{x\in \mathbb{N} : 1\leq x\leq 5\}$ και 
    $B=\{x\in \mathbb{N} : 3\leq x\leq 8\}$. 
    Να παραστήσετε αναλυτικά τα παρακάτω σύνολα.  Τι παρατηρείτε;
    \begin{enumerate}[(\roman*)]
      \item $A\cup B$ \hfill Απ: $ \{ 1,2,3,4,5,6,7,8 \} $ 
      \item $A\cap B$ \hfill Απ: $ \{ 3,4,5 \} $ 
      \item $A^c$ \hfill Απ: $ \{ 6,7,8,9 \} $ 
      \item $B^c$ \hfill Απ: $ \{ 1,2,9 \} $ 
      \item $(A\cup B)^c$ \hfill Απ: $ \{ 9 \} $ 
      \item $(A\cap B)^c$ \hfill Απ: $ \{ 1,2,6,7,8,9 \} $ 
      \item $A^c\cap B^c$ \hfill Απ: $ \{ 9 \} $  
      \item $A^c\cup B^c$ \hfill Απ: $ \{ 1,2,6,7,8,9 \} $ 
    \end{enumerate}
    \begin{rem*}
      Παρατηρούμε ότι ισχύουν οι παρακάτω σχέσεις, γνωστοί ως νόμοι De Morgan
      \[ (A \cup B)^{c} = A^{c} \cap B^{c} \quad \text{και} \quad 
      (A \cap B)^{c} = A^{c} \cup B^{c} \]
    \end{rem*}
\end{enumerate}



\end{document}
