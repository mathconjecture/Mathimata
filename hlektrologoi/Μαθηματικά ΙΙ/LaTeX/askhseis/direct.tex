\input{preamble_ask.tex}
\input{definitions_ask.tex}

\pagestyle{askhseis}
\everymath{\displaystyle}

\begin{document}



\begin{center}
  \minibox{\large\bfseries \textcolor{Col1}{Ασκήσεις στην Παράγωγο κατά Κατεύθυνση}}
\end{center}

\vspace{\baselineskip}

\begin{enumerate}
  \item Να βρεθεί η παράγωγος κατά κατεύθυνση της συνάρτησης $ f(x,y) = x^{2}y^{3}-4y $ 
    στο σημείο $ P(2,-1) $ κατά τη διεύθυνση του διανύσματος 
    $ \mathbf{u}=2 \mathbf{i}+5 \mathbf{j} $.	

    \hfill Απ: $ \frac{32}{\sqrt{ 29 }} $ 

  \item Να βρεθεί η παράγωγος κατά κατεύθυνση της συνάρτησης 
    $ f(x,y,z) = xe^{y^{2}-z^{2}} $ στο σημείο $ P(1,2,-2) $ κατά την κατεύθυνση 
    του διανύσματος $ \mathbf{u} = \mathbf{i}-2 \mathbf{k} $.

    \hfill Απ: $ - \frac{7}{\sqrt{ 5 }} $ 

  \item Δίνεται η πραγματική συνάρτηση $ f(x,y) = xe^{y} $.
    \begin{enumerate}[i)]
      \item Να υπολογίσετε την παράγωγο κατά κατεύθυνση της $ f(x,y) $ στο σημείο 
        $ P(2,0) $ και προς την κατεύθυνση του σημείου $ Q({1}/{2}, 2) $.
      \item Να προσδιορίσετε την κατεύθυνση προς την οποία η $ f(x,y) $ παρουσιάζει το
        μεγαλύτερο ρυθμό αύξησης στο σημείο $ P $.
      \item Να υπολογίσετε τη μέγιστη τιμή της παραγώγου κατά κατεύθυνση της 
        $ f(x,y) $ στο σημείο $ P(2,0) $.
    \end{enumerate}

    \hfill Απ: $ \rm{i}) 1, \; ii) \mathbf{u} = (1,2), \; iii) \sqrt{ 5 } $ 

\end{enumerate}

\end{document}
