\documentclass[a4paper,table]{report}
\documentclass[a4paper,12pt]{article}
\usepackage{etex}
%%%%%%%%%%%%%%%%%%%%%%%%%%%%%%%%%%%%%%
% Babel language package
\usepackage[english,greek]{babel}
% Inputenc font encoding
\usepackage[utf8]{inputenc}
%%%%%%%%%%%%%%%%%%%%%%%%%%%%%%%%%%%%%%

%%%%% math packages %%%%%%%%%%%%%%%%%%
\usepackage{amsmath}
\usepackage{amssymb}
\usepackage{amsfonts}
\usepackage{amsthm}
\usepackage{proof}

\usepackage{physics}

%%%%%%% symbols packages %%%%%%%%%%%%%%
\usepackage{bm} %for use \bm instead \boldsymbol in math mode 
\usepackage{dsfont}
\usepackage{stmaryrd}
%%%%%%%%%%%%%%%%%%%%%%%%%%%%%%%%%%%%%%%


%%%%%% graphicx %%%%%%%%%%%%%%%%%%%%%%%
\usepackage{graphicx}
\usepackage{color}
%\usepackage{xypic}
\usepackage[all]{xy}
\usepackage{calc}
\usepackage{booktabs}
\usepackage{minibox}
%%%%%%%%%%%%%%%%%%%%%%%%%%%%%%%%%%%%%%%

\usepackage{enumerate}

\usepackage{fancyhdr}
%%%%% header and footer rule %%%%%%%%%
\setlength{\headheight}{14pt}
\renewcommand{\headrulewidth}{0pt}
\renewcommand{\footrulewidth}{0pt}
\fancypagestyle{plain}{\fancyhf{}
\fancyhead{}
\lfoot{}
\rfoot{\small \thepage}}
\fancypagestyle{vangelis}{\fancyhf{}
\rhead{\small \leftmark}
\lhead{\small }
\lfoot{}
\rfoot{\small \thepage}}
%%%%%%%%%%%%%%%%%%%%%%%%%%%%%%%%%%%%%%%

\usepackage{hyperref}
\usepackage{url}
%%%%%%% hyperref settings %%%%%%%%%%%%
\hypersetup{pdfpagemode=UseOutlines,hidelinks,
bookmarksopen=true,
pdfdisplaydoctitle=true,
pdfstartview=Fit,
unicode=true,
pdfpagelayout=OneColumn,
}
%%%%%%%%%%%%%%%%%%%%%%%%%%%%%%%%%%%%%%

\usepackage[space]{grffile}

\usepackage{geometry}
\geometry{left=25.63mm,right=25.63mm,top=36.25mm,bottom=36.25mm,footskip=24.16mm,headsep=24.16mm}

%\usepackage[explicit]{titlesec}
%%%%%% titlesec settings %%%%%%%%%%%%%
%\titleformat{\chapter}[block]{\LARGE\sc\bfseries}{\thechapter.}{1ex}{#1}
%\titlespacing*{\chapter}{0cm}{0cm}{36pt}[0ex]
%\titleformat{\section}[block]{\Large\bfseries}{\thesection.}{1ex}{#1}
%\titlespacing*{\section}{0cm}{34.56pt}{17.28pt}[0ex]
%\titleformat{\subsection}[block]{\large\bfseries{\thesubsection.}{1ex}{#1}
%\titlespacing*{\subsection}{0pt}{28.80pt}{14.40pt}[0ex]
%%%%%%%%%%%%%%%%%%%%%%%%%%%%%%%%%%%%%%

%%%%%%%%% My Theorems %%%%%%%%%%%%%%%%%%
\newtheorem{thm}{Θεώρημα}[section]
\newtheorem{cor}[thm]{Πόρισμα}
\newtheorem{lem}[thm]{λήμμα}
\theoremstyle{definition}
\newtheorem{dfn}{Ορισμός}[section]
\newtheorem{dfns}[dfn]{Ορισμοί}
\theoremstyle{remark}
\newtheorem{remark}{Παρατήρηση}[section]
\newtheorem{remarks}[remark]{Παρατηρήσεις}
%%%%%%%%%%%%%%%%%%%%%%%%%%%%%%%%%%%%%%%




\newcommand{\vect}[2]{(#1_1,\ldots, #1_#2)}
%%%%%%% nesting newcommands $$$$$$$$$$$$$$$$$$$
\newcommand{\function}[1]{\newcommand{\nvec}[2]{#1(##1_1,\ldots, ##1_##2)}}

\newcommand{\linode}[2]{#1_n(x)#2^{(n)}+#1_{n-1}(x)#2^{(n-1)}+\cdots +#1_0(x)#2=g(x)}

\newcommand{\vecoffun}[3]{#1_0(#2),\ldots ,#1_#3(#2)}

\newcommand{\mysum}[1]{\sum_{n=#1}^{\infty}


\let\vec\mathbf

\pagestyle{vangelis}

\begin{document}


\chapter{Γραμμική Ανεξαρτησία}

\section{Υπόχωροι που παράγονται από διανύσματα}

\begin{dfn}
  Έστω $ V $ ένας διανυσματικός χώρος επί του σώματος $ \mathbb{F} $ και έστω 
  $ \mathbf{v} \in V $. Λέμε ότι το διάνυσμα $ \mathbf{v}$ είναι 
  \textcolor{Col2}{γραμμικός συνδυασμός} των διανυσμάτων 
  $ \mathbf{u_{1}}, \mathbf{u_{2}}, \ldots \mathbf{u}_{n} $, αν υπάρχουν 
  $ \lambda _{1}, \lambda _{2}, \ldots, \lambda _{n} \in \mathbb{F} $ τέτοιοι ώστε 
  \[
    \mathbf{v} = \lambda _{1} \mathbf{u_{1}}+ \lambda_{2} \mathbf{u_{2}}+ 
    \cdots \lambda _{k} \mathbf{u}_{k} \Leftrightarrow \mathbf{v} = 
    \sum_{i=1}^{k} \lambda _{i} \mathbf{u}_{i} 
  \]
  Τα στοιχεία $ \lambda _{1}, \lambda _{2}, \ldots, \lambda _{k} $ ονομάζονται 
  \textcolor{Col2}{συντελεστές} τους γραμμικού συνδυασμόυ.
\end{dfn}

\begin{thm}
  Έστω $ V $ ένας διανυσματικός χώρος επί του σώματος $ \mathbb{F} $ και έστω 
  $ S = \{ \mathbf{u_{1}}, \mathbf{u_{2}}, \ldots, \mathbf{u}_{k}\} \subseteq V $.
  Θέτουμε $ W $ να είναι το σύνολο όλων των γραμμικών συνδυασμών των στοιχείων του 
  $S$, με συντελεστές από το σώμα $ \mathbb{F} $.  Δηλαδή 
  \[
    W = L(S) = \{ w \in V \; : \; \exists \lambda _{1}, \ldots, \lambda _{k} \in 
      \mathbb{F} \; \text{ώστε} \; \mathbf{w} = \lambda _{1} 
    \mathbf{u_{1}}+ \cdots \lambda _{k} \mathbf{u}_{k}\} 
  \] 
  Τότε
  \begin{enumerate}[i)]
    \item $ S \subseteq W $
    \item $ W \leq V $ 
    \item $ W $ είναι ο μικρότερος υπόχωρος του $V$ που περιέχει το $S$.
  \end{enumerate}
\end{thm}

\begin{dfn}
  Έστω $ V $ ένας διανυσματικός χώρος επί του $ \mathbb{F} $ και έστω $ S = 
  \{ \mathbf{u}_{1}, \ldots, \mathbf{u}_{k} \} \subseteq V$. Τότε ο υπόχωρος $ W $ 
  του προηγούμενου θεωϱηματος ονομάζεται ο υπόχωρος του $V$ που 
  \textcolor{Col2}{παράγεται} από το $ S $, ή γραμμική θήκη του $S$ και 
  συμβολίζεται με $ W = < S > $ ή $ W = \Span \{ v_{1},\ldots,v_{n}  \}  $.  
  Λέμε επίσης ότι το σύνολο $S$ \textcolor{Col2}{παράγει} τον υπόχωρο $W$. 
  Επειδή $S$  πεπερασμένο λέμε ότι $W$ είναι \textcolor{Col2}{πεπερασμένα 
  γεννώμενος} υπόχωρος του  $V$ και τα στοιχεία 
  $ \mathbf{u_{1}}, \ldots, \mathbf{u_{n}} $ του $S$ λέγονται 
  \textcolor{Col2}{γεννήτορες} του $W$.
\end{dfn}

\begin{examples}
\item {}
  \begin{enumerate}
    \item Έστω ο $ \mathbb{R} $-χώρος $ \mathbb{R}^{n} $ και για $ i= 1,\ldots,n $
      έστω $ \mathbf{e_{i}} $ το διάνυσμα του $ \mathbb{R}^{n} $ το οποίο έχει 
      1 στην $ i $-θέση και 0 στις υπόλοιπες. Τότε $ \mathbb{R}^{n} = 
      < \mathbf{e_{1}}, \mathbf{e_{2}}, \ldots \mathbf{e_{n}} >  $. 
      Πράγματι, έστω $ \mathbf{u} = (u_{1},u_{2},\ldots,u_{n}) \in 
      \mathbb{R}^{n} $. Τότε 
      \begin{align*}
        \mathbf{u} = (u_{1},u_{2},\ldots,u_{n}) 
                &=  u_{1} (1,0,\ldots,0) + u_{2} (0,1,\ldots,0) + 
                \cdots + u_{n} (0,0,\ldots,1) \\
                &= u_{1} \mathbf{e_{1}} + u_{2} \mathbf{e_{2}} + \cdots + u_{n} 
                \mathbf{e_{n}} \in < \mathbf{e_{1}}, \mathbf{e_{2}}, 
                \ldots, \mathbf{e_{n}} >  
      \end{align*} 
      Άρα $ \mathbb{R}^{n} = < \mathbf{e_{1}}, \mathbf{e_{2}}, 
      \ldots, \mathbf{e_{n}} >  $.
    \item Έστω $ A \in \textbf{M}_{m \times n}(\mathbb{R}) $ και έστω 
      $ \mathbf{r}_{1}, \ldots \mathbf{r}_{m} $ οι $m$ - γραμμές του $A$.  
      Ο υπόχωρος $ < \mathbf{r}_{1}, \ldots, \mathbf{r}_{m} >  $ του 
      $ \mathbb{R}^{n} $ που παράγεται από τις γραμμές του $A$ ονομάζεται 
      χώρος γραμμών του $A$ και συμβολίζεται με $ R(A) $ ή $ \Gamma_{A} $.
    \item Έστω $ A \in \textbf{M}_{m \times n}(\mathbb{R}) $ και έστω 
      $ \mathbf{c}_{1}, \ldots \mathbf{c}_{n} $ οι $n$ - στήλες του $A$.  
      Ο υπόχωρος $ < \mathbf{c}_{1}, \ldots, \mathbf{c}_{n} >  $ του 
      $ \mathbb{R}^{m} $ που παράγεται από τις στήλες του $A$ ονομάζεται 
      χώρος στήλών του $A$ και συμβολίζεται με $ C(A) $ ή $ \Sigma_{A} $.
    \item Έστω $ \mathbf{P_{n}}(\mathbb{R}) $ ο $ \mathbb{R} $ - χώρος των 
      πολυωνύμων βαθμού $ \leq n $ και έστω $ p_{0} = 1, p_{1}=x, p_{2}=x^{2}, 
      \ldots, p_{n}=x^{n} $. Τότε $ \mathbf{P_{n}}(\mathbb{R}) = 
      < 1,x,x^{2},\ldots,x^{n}> $.
  \end{enumerate}
\end{examples}

\begin{thm}\label{thm:more}
  Έστω $V$ ένας διανυσματικός χώρος επί του $ \mathbb{F} $ και έστω $ S, S' $ δύο 
  μη-κενά, πεπερασμένα υποσύνολα του $V$. Τότε 
  \[ < S > = < S' > \Leftrightarrow S \subseteq < S' >  \; \text{και} \; 
  S' \subseteq < S >  \]
\end{thm}

% \begin{proof}
% \item {}
%   \begin{description}
%     \item[($\Rightarrow$)] Έστω ότι $ < S > = < S' >   $. Γνωρίζουμε από το προηγούμενο θεώρημα
%       ότι $ S \subseteq < S >  $ και $ S' \subseteq < S' >  $. Άρα από την 
%       υπόθεση έχουμε ότι και $ S \subseteq < S' >  $ και $ S' \subseteq < S >  $.
%     \item[($\Leftarrow$)] Έστω ότι $ S \subseteq < S' >  $ και $ S' \subseteq < S >  $. Επειδή $
%       < S >  $ είναι ο μικρότερος υπόχωρος του $V$ που περιέχει το $S$ και 
%       $ S \subseteq < S' >  $ έπεται ότι $ < S > \leq < S' >   $. Ομοίως 
%       $ < S' > \leq < S >   $, οπότε $ < S > = < S' >   $.
%   \end{description}
% \end{proof}

\begin{rem}
  Από το θεώρημα~\ref{thm:more} καταλαβαίνουμε ότι είναι δυνατόν περισσότερα από ένα 
  υποσύνολα του $V$ να παϱάγουν τον ίδιο υπόχωρο του $V$. Για παράδειγμα, για τον 
  $ \mathbb{R} $- χώρο $ \mathbb{R} $ έχουμε ότι $ \mathbb{R} = < x > , 
  \; \forall x \in  \mathbb{R} \setminus \{ 0 \} $.
\end{rem}

\section{Γραμμική Ανεξαρτησία}

\begin{dfn}
  Έστω $ (V,+,\cdot) $ ένας $ \mathbb{F} $- χώρος και έστω 
  $ S = \{ \mathbf{v_{1}}, \ldots, \mathbf{v}_{n} \} \subseteq V $. Λέμε ότι 
  τα διανύσματα $ \mathbf{v_{1}}, \ldots, \mathbf{v_{n}} $ είναι 
  \textcolor{Col2}{γραμμικώς ανεξάρτητα} ή ότι το σύνολο $ S $ είναι 
  \textcolor{Col2}{γραμμικώς ανεξάρτητο}, αν οποτεδήποτε έχουμε
  \[
    \comb{v}{k} = \mathbf{0} \; \text{τότε} \; \lambda _{1} = 
    \lambda _{2} = \cdots = \lambda _{n} = 0
  \]
  Αν τα διανύσματα $ \mathbf{v_{1}}, \ldots, \mathbf{v_{n}} $ δεν είναι γραμμικώς 
  ανεξάρτητα τότε λέγονται \textcolor{Col2}{γραμμικώς εξαρτημένα} και ισχύει 
  ότι υπάρχει μη τετριμμένος γραμμικός συνδυασμός στοιχείων του $S$ που είναι 
  ίσος με $ \mathbf{0} $, δηλαδή
  \[
    \exists  \lambda _{1}, \ldots, \lambda _{k} \in \mathbb{F} \; 
    \text{όχι όλα μηδέν, ώστε} \; \comb{u}{k} = \mathbf{0}
  \]
\end{dfn}

\begin{examples}
\item {}
  \begin{enumerate}
    \item Στον $ \mathbb{R}^{2} $ δύο διανύσματα $ \mathbf{u} $ και $ \mathbf{v} $ 
      είναι γραμμικώς ανεξάρτητα αν και μόνον αν δεν είναι παράλληλα. Πράγματι 
      έστω ότι $ \mathbf{u}, \mathbf{v} $ είναι γραμμικώς εξαρτημένα. Τότε 
      \[
        \exists \lambda _{1}, \lambda _{2} \in \mathbb{R} \; 
        \text{όχι και τα δύο μηδέν, ώστε} 
        \; \lambda_{1} \mathbf{u} + \lambda_{2} \mathbf{v} = \mathbf{0} 
      \]
      Έτσι, αν έστω ότι $ \lambda_{1} \neq 0 $, έχουμε 
      \[
        \mathbf{u} = - \frac{\lambda _{2}}{\lambda _{1}} \mathbf{v} \quad  
        \text{(παράλληλα)}
      \] 
  \end{enumerate}
\end{examples}

\begin{prop}
  Το $ \emptyset $ είναι γραμμικώς ανεξάρτητο. 
\end{prop}
\begin{proof}
  Έστω ότι το $ \emptyset $ είναι γραμμικώς εξαρτημένο. Άρα υπάρχει μη τετριμμένος 
  γραμμικός συνδυασμός στοιχείων του $ \emptyset $ που να είναι 
  $ \mathbf{0} $. Άτοπο, γιατί το $ \emptyset $ δεν έχει στοιχεία.
\end{proof}

\begin{prop}
  Κάθε μη-μηδενικό διάνυσμα ένος $ \mathbb{F} $- χώρου $V$ είναι γραμμικώς 
  ανεξάρτητο.
\end{prop}
\begin{proof}
  Πράγματι. Έστω $ \mathbf{u} \in V $ και $ \lambda \in \mathbb{F} $ με 
  $ \lambda \mathbf{u} = \mathbf{0} $. Τότε έχουμε $ \lambda = 0 $, αφού 
  $ \mathbf{u} \neq \mathbf{0} $.
\end{proof}

\begin{prop}
  Έστω $V$ ένας $ \mathbb{F} $- χώρος και 
  $ S = \{ \mathbf{v_{1}}, \mathbf{v_{2}}, \ldots, \mathbf{v_{n}}  \} \subseteq V $.
  Αν $ \mathbf{0} \in S $ τότε το $S$ είναι γραμμικώς εξαρτημένο. Συγκεκριμένα 
  $ \{ \mathbf{0} \} $ είναι γραμμικώς εξαρτημένο.
\end{prop}
\begin{proof}
  Πράγματι, έστω ότι $ \mathbf{u}_{i} = \mathbf{0} $, για κάποιο $ i $ με 
  $ 1 \leq i \leq k $. Τότε
  \[
    0 \mathbf{u_{1}}+ \cdots + 0 \mathbf{u}_{i-1} + 1 \mathbf{u}_{i} + 0 
    \mathbf{u}_{i+1} + \cdots + 0 \mathbf{u}_{k} = \mathbf{0}  
  \]
  είναι ένας γραμμικός συνδυασμός στοιχειών του $S$ που είναι $ \mathbf{0} $, 
  χωρίς να είναι μηδέν όλοι οι συντελεστές.
\end{proof}

\begin{examples}
\item {}
  \begin{enumerate}
    \item 
      Έστω $ \mathbb{R} $- χώρος $ \mathbb{R}^{n} $. Τότε οποιαδήποτε $ k $ 
      το πλήθος διανύσματα του $ \mathbb{R}^{n} $ με $ k >n $ είναι 
      γραμμικώς εξαρτημένα. 

    \item Έστω ο $ \mathbb{R} $-χώρος $ \mathbb{R}^{n} $ και για $ i= 1,\ldots,n $
      έστω $ \mathbf{e_{i}} $ το διάνυσμα του $ \mathbb{R}^{n} $ το οποίο έχει 
      1 στην $ i $-θέση και 0 στις υπόλοιπες. Τότε τα διανύσματα 
      $ \mathbf{e_{1}}, \mathbf{e_{2}}, \ldots, \mathbf{e_{n}} $ είναι 
      γραμμικώς ανεξάρτητα. 

    \item Έστω $ \mathbb{R} $- χώρος $ \mathbf{P_{n}}(\mathbb{R}), \; n 
      \in \mathbb{N} $. Τότε το σύνολο $ \{1,x,x^{2}, \ldots,x^{n} \} $ 
      είναι γραμμικώς ανεξάρτητο. 
  \end{enumerate}
\end{examples}

\begin{thm}\label{thm:s1s2depend}
  Έστω $V$ ένας $ \mathbb{F} $- χώρος και έστω 
  $ S = \{ \mathbf{u}_{1}, \ldots, \mathbf{u_{n}} \} $ μη μηδενικό σύνολο 
  διανυσμάτων του $V$. Τότε το $S$ είναι γραμμικώς εξαρτημένο αν και μόνον αν 
  κάποιο από τα διανύσματα του $S$ είναι γραμμικός συνδυασμός των υπολοίπων. 
\end{thm}
\begin{proof}
\item {}
  \begin{description}
    \item [($ \Rightarrow $)] Έστω ότι το $S$ είναι γραμμικώς εξαρτημένο. Τότε 
      υπάρχουν $ \lambda _{1}, \ldots, \lambda_{n} \in \mathbb{F} $ όχι 
      όλα μηδέν, ώστε 
      \[
        \lambda _{1} \mathbf{u_{1}} + \cdots + \lambda _{n} \mathbf{u}_{n} = 
        \mathbf{0}  
      \] 
      Έστω ότι $ \lambda _{i} \neq 0 $ με $ 1 \leq i \leq n $. Τότε έχουμε
      \begin{align*}
        \lambda _{1} \mathbf{u}_{1} + \cdots + \lambda _{i-1} 
        \mathbf{u}_{i-1} + \lambda _{i} \mathbf{u}_{i} + \lambda _{i+1} 
        \mathbf{u}_{i+1} + \cdots + \lambda _{n}
        \mathbf{u}_{n} = \mathbf{0} \Leftrightarrow  \\
        \mathbf{u}_{i} = - \frac{\lambda _{1}}{\lambda _{i}} \mathbf{u}_{1} +
        \cdots - \frac{\lambda _{i-1}}{\lambda _{i}} \mathbf{u}_{i-1} - 
        \frac{\lambda _{i+1}}{\lambda _{i}} \mathbf{u}_{i+1} - 
        \cdots - \frac{\lambda _{n}}{\lambda _{i}} \mathbf{u}_{n} 
      \end{align*} 
      Δηλαδή το διάνυσμα $ \mathbf{u}_{i} $ είναι γραμμικός συνδυασμός των 
      υπολόιπων.
    \item [($\Leftarrow$)] 
      Έστω ότι για κάποιο $ i $ με $ 1 \leq i \leq n $ το διάνυσμα 
      $ \mathbf{u}_{i} $ είναι γραμμικός συνδυασμός των υπολοίπων διανυσμάτων.
      Άρα υπάρχουν $ \lambda _{1}, \ldots, \lambda _{n} \in \mathbb{F} $ 
      ώστε 
      \begin{gather*}
        \mathbf{u}_{i} = \lambda _{1} \mathbf{u}_{1} + \cdots + 
        \lambda _{n} \mathbf{u}_{n} \Leftrightarrow \\
        \mathbf{u}_{i} = \lambda _{1} \mathbf{u_{1}} + \cdots + 
        \lambda _{i-1}
        \mathbf{u}_{i-1} + \lambda _{i+1} \mathbf{u}_{i+1} + \cdots + 
        \lambda _{n} \mathbf{u}_{n} \Leftrightarrow \\
        \lambda _{1} \mathbf{u_{1}} + \cdots + \lambda _{i-1}
        \mathbf{u}_{i-1} + (-1) \cdot \mathbf{u}_{i} + \lambda _{i+1} 
        \mathbf{u}_{i+1} + \cdots + \lambda _{n} \mathbf{u}_{n} = \mathbf{0}
      \end{gather*}
      Επομένως υπάρχει μη τετριμμένος γραμμικός συνδυασμός των στοιχειών του 
      $S$ που ειναι ο μηδενικός, άρα το $S$ είναι γραμμικώς εξαρτημένο.
  \end{description}
\end{proof}


\begin{thm}
  Έστω $V$ ένας $ \mathbb{F} $- χώρος και έστω $ S_{1}, S_{2} $ δύο μη κενά, 
  πεπερασμένα υποσύνολα του $V$ με $ S_{1} \subseteq S_{2} $. 
  \begin{myitemize}
    \item Τότε, αν $ S_{1} $ είναι γραμμικώς εξαρτημένο, τότε και το $ S_{2} $ 
  είναι γραμμικώς εξαρτημένο.
\item Τότε αν $ S_{2} $ είναι γραμμικώς ανεξάρτητο τότε και το $ S_{1} $ είναι γραμμικώς 
  ανεξάρτητο.
  \end{myitemize}
\end{thm}


\section{Βάση και Διάσταση Διανυσματικού Χώρου}

\begin{dfn}
  Έστω $ V $ ένας $ \mathbb{F} $- χώρος πεπερασμένης διάστασης.

  Ένα υποσύνολο $ B = \{ \mathbf{u}_{1} , \ldots, \mathbf{u}_{n} \} \subseteq V $ 
  ονομάζεται  {\color {Col2} $ \mathbb{F} $- βάση} του $V$, ή απλώς 
  \textcolor{Col2}{βάση} του $V$, αν 
  \begin{enumerate}[i)]
    \item Το σύνολο $B$ παράγει το χώρο $V$, δηλαδή $ V = < B >  $ 
    \item Το $B$ είναι γραμμικώς ανεξάρτητο υποσύνολο του $V$
  \end{enumerate}
\end{dfn}

\begin{rems}
\item {}
  \begin{enumerate}
    \item Αν $B$ είναι μια βάση ενός διανυσματικού χώρου $V$ τότε 
      $ \mathbf{0}_{V} \not \in B $, γιατί αλλιώς το $B$ θα ήταν γραμμικώς 
      εξαρτημένο.
    \item $ \emptyset $ είναι βάση του τετριμμένου χώρου $ \{ \mathbf{0}_{V} \} $.
  \end{enumerate}
\end{rems}

\begin{examples}
\item {}
  \begin{enumerate}
    \item Έστω ο $ \mathbb{R} $- χώρος $ \mathbb{R}^{n} $. Τότε το σύνολο 
      $ B = \{ \mathbf{e_{1}}, \ldots, \mathbf{e_{n}} \} $ είναι βάση του και 
      ονομάζεται η συνήθης ή η κανονική βάση του $ \mathbb{R}^{n} $.
    \item Έστω ο $ \mathbb{R} $- χώρος $ \mathbf{P_{n}}(\mathbb{R}) $. Τότε το 
      σύνολο $ B = \{ 1, x, x^{2}, \ldots, x^{n} \} $ είναι βάση του και 
      ονομάζεται η συνήθης ή η κανονική βάση του $ \mathbf{P_{n}}(\mathbb{R}) $.
    \item Έστω ο $ \mathbb{R} $- χώρος $ \textbf{M}_{m \times n}(\mathbb{R}) $. 
      Τότε το σύνολο 
      $ B = \{ E_{ij} \in \textbf{M}_{m \times n}(\mathbb{R}) \; : \; 1 \leq i
      \leq m \; \text{και} \; 1 \leq j \leq n\}  $, όπου $ E _{ij} $ είναι ο 
      $ m \times n $ πίνακας ο οποίος έχει 1 στην $ ij $- θέση και 0 στις άλλες 
      είναι βάση του και ονομάζεται η συνήθης ή η κανονική βάση του 
      $ \textbf{M}_{m \times n}(\mathbb{R}) $.
  \end{enumerate}
\end{examples}

\begin{thm}[Χαρακτηρισμός των Βάσεων]
\item {}
  Έστω $V$ ένας $ \mathbb{F} $- χώρος πεπερασμένης διάστασης και έστω 
  $ B = \{ \mathbf{u_{1}}, \ldots, \mathbf{u}_{n} \} $ ένα υποσύνολο του V. Τότε 
  το $B$ είναι βάση του $V$ αν και μόνον αν κάθε διάνυσμα του $V$ γράφεται 
  κατά μοναδικό τϱόπο ως γραμμικός συνδυασμός των στοιχείων του $B$.
\end{thm}

\begin{rem}
  Αν $B = \{ \mathbf{u}_{1}, \ldots, \mathbf{u_{n}} \} $ είναι μια βάση ενός 
  $ \mathbb{F} $ - χώρου $V$, τότε από το προηγούμενο θεώρημα έχουμε ότι 
  $ \forall \mathbf{u} \in V, \; \exists $ μοναδικό διάνυσμα 
  $ (u_{1}, \ldots, u_{n}) \in \mathbb{F}^{n} $ τέτοιο ώστε 
  $ \mathbf{u} = u_{1} \mathbf{u_{1}} + \cdots u_{n} \mathbf{u}_{n} $. 
  Το διάνυσμα $ (u_{1}, \ldots, u_{n}) \in \mathbb{F}^{n} $ λέγεται 
  \textcolor{Col2}{διάνυσμα συντεταγμένων} του $ \mathbf{u} $ και γράφουμε 
  \[ 
    [\mathbf{u}]_{B} = (u_{1}, \ldots, u_{n})
\] 
και  οι συντελεστές $ u_{1}, \ldots, u_{n} \in \mathbb{F} $ λέγονται 
\textcolor{Col2}{συντεταγμένες} του διανύσματος $ \mathbf{u} $ ως προς τη βάση $B$.
\end{rem}

\begin{prop}
  Αν το σύνολο $ S = \{ \mathbf{u}_{1}, \ldots, \mathbf{u_{n}} \} $ παράγει 
  το χώρο $V$, τότε κάποιο υποσύνολο του $S$ είναι βάση του $V$.
\end{prop}

\begin{thm}
  Έστω $ B = \{ \mathbf{e_{1}}, \ldots, \mathbf{e_{n}} \} $ μια βάση του $V$. 
  Τότε 
  \begin{enumerate}[i)]
    \item Οποιαδήποτε $ n+1 $ στοιχεία του $V$ είναι γραμμικώς εξαρτημένα.
    \item Όλες οι βάσεις του χώρου $V$ έχουν το ίδιο πλήθος στοιχείων.
  \end{enumerate}
\end{thm}

\begin{dfn}
  Όπως είδαμε, αν ένας $ \mathbb{F} $ - χώρος $V$ έχει μια βάση με 
  $ n $ στοιχεία, τότε και κάθε άλλη βάση του, θα έχει επίσης $n$ στοιχεία. 
  Ο αριθμός $n$ λέγεται \textcolor{Col2}{διάσταση} του χώρου $V$ επί του 
  $ \mathbb{F} $ και γράφουμε $ \dim V = n $.
\end{dfn}

\begin{prop}
\item {}
  Έστω $V$ ένας $ \mathbb{F} $ - χώρος διάστασης $n$. Τότε
  \begin{enumerate}[i)]
    \item $n$ γραμμικώς ανεξάρτητα στοιχεία του $V$, αποτελούν βάση αυτού.
    \item Οποιαδήποτε $ n-1 $ στοιχεία του $V$ δεν παράγουν τον χώρο $V$.
    \item Οποιαδήποτε $ m < n $ γραμμικώς ανεξάρτητα στοιχεία του $V$, 
      συμπληρώνονται με $ n-m $ άλλα στοιχεία, ώστε να προκύψει μια 
      βάση του χώρου.
  \end{enumerate}
\end{prop}

\begin{prop}
  Έστω $V$ ένας διανυσματικός χώρος πεπερασμένης διάστασης. Έστω $ W $ ένας 
  υπόχωρος του $V$ με βάση $ \{ \mathbf{e_{1}}, \ldots, \mathbf{e_{m}} \} $. 
  Τότε υπάρχει βάση του χώρου $ V $ της μορφής 
  $ \{ \mathbf{e_{1}}, \ldots, \mathbf{e_{m}}, \mathbf{e_{m+1}}, \ldots, 
  \mathbf{e_{n}} \} $, με $ n \geq m $.
\end{prop}

\begin{prop}
  Αν $W$ υπόχωρος του $V$ και $ \dim W = dim V $, τότε $ W=v $.
\end{prop}
\begin{proof}
  Γιατί, αν $ \{ \mathbf{e_{1}}, \ldots, \mathbf{e_{m}}\}  $ είναι βάση του 
  $ W $, θα είναι και βάση του $V$. Έτσι ο 
  $ V = < \mathbf{e_{1}}, \ldots, \mathbf{e}_{m} > = W$.
\end{proof}

\begin{prop}
  Εστω $ A = (a_{ij}) \in \textbf{M}_{n}(\mathbb{R}) $. Τότε 
  \begin{enumerate}[i)]
    \item Οι στήλες $ \mathbf{c}_{1}, \ldots, \mathbf{c}_{n} $ είναι γραμμικώς 
      εξαρτημένα διανύσματα του $ \mathbb{R}^{n} $ αν $ \abs{A} = 0 $.
    \item Οι γραμμές $ \mathbf{r}_{1}, \ldots, \mathbf{r}_{n} $ είναι γραμμικώς 
      εξαρτημένα διανύσματα του $ \mathbb{R}^{n} $ αν $ \abs{A} = 0 $.
  \end{enumerate}
\end{prop}

\begin{rem}
  Από το θεώρημα έπεται ότι για να ελέγξουμε αν $ n $ το πλήθος διανύσματα 
  του $ \mathbb{R}^{n} $ είναι γραμμικώς ανεξάρτητα, τότε αρκεί να ελέγξω την 
  ορίζουσα του πίνακα $A$ που έχει τα διανύσματα αυτά ως στήλες ή ως γραμμές. 
  Αν $ \abs{A} = 0 $, τότε θα είναι γραμμικώς εξαρτημένα.
\end{rem}

 \end{document}

