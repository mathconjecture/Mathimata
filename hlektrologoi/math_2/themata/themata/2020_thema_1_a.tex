\documentclass[a4paper,table]{report}
\input{preamble_themata}
\newcommand{\vect}[2]{(#1_1,\ldots, #1_#2)}
%%%%%%% nesting newcommands $$$$$$$$$$$$$$$$$$$
\newcommand{\function}[1]{\newcommand{\nvec}[2]{#1(##1_1,\ldots, ##1_##2)}}

\newcommand{\linode}[2]{#1_n(x)#2^{(n)}+#1_{n-1}(x)#2^{(n-1)}+\cdots +#1_0(x)#2=g(x)}

\newcommand{\vecoffun}[3]{#1_0(#2),\ldots ,#1_#3(#2)}

\newcommand{\suma}{\sum_{n=0}^{\infty}a_n x^n}

\newcommand{\sumb}{\sum_{n=1}^{\infty}a_n n x^{n-1}}

\newcommand{\sumc}{\sum_{n=2}^{\infty}a_n n (n-1) x^{n-2}}

\newcommand{\varsum}[2]{\sum_{n=#1}^{#2}}
\input{myboxes}

\setcounter{chapter}{1}

\newcommand{\twocolumnsiden}[2]{\vspace{0.5\baselineskip}\begin{minipage}[t]{0.30\linewidth}
    #1
    \end{minipage}\begin{minipage}[t]{0.30\linewidth}
    #2
  \end{minipage}
  \vspace{0.5\baselineskip}
}

\everymath{\displaystyle}
\pagestyle{vangelis}


\begin{document}

\begin{mybox3}
  \begin{thema}
    Έστω ότι $u$ και $v$ είναι συναρτήσεις των $x$ και $y$, οι οποίες ορίζονται μέσω των 
    σχέσεων 
    \[
      \begin{rcases}  
        u \cos{v} = x+1 \\
        u \sin{v} = x+y
      \end{rcases}
    \]
    Να δείξετε ότι η Ιακωβιανή ορίζουσα $ \pdv{(u,v)}{(x,y)} = \frac{1}{u} $.
  \end{thema}
\end{mybox3}
\begin{solution}
  \begin{description}
    \item [Α τρόπος:] Θεωρούμε το σύστημα των πεπλεγμένων εξισώσεων
      \[
        \begin{rcases}
          F(x,y,u,v) = u \cos{v} - x- 1 = 0 \\
          G(x,y,u,v) = u \sin{v} - x - y = 0 
        \end{rcases}
      \]
      Έχουμε δύο εξισώσεις με τέσσερις μεταβλητές. Ζητάμε να υπολογίσουμε την ορίζουσα
      \[
        \pdv{(u,v)}{(x,y)} = \begin{vmatrix*}[r]
          u_{x} & u_{y} \\
          v_{x} & v_{y}
        \end{vmatrix*}
      \]
      επομένως, αρκεί να υπολογίσουμε τις μερικές παραγώγους $ u_{x}, u_{y}, v_{x} $ και
      $ v_{y} $. Γι᾽ αυτό θεωρούμε ως εξαρτημένες μεταβλητές τις $ u,v $. Άρα οι μεταβλητές 
      $ x $ και $ y $ είναι οι ανεξάρτητες.
      Έχουμε ότι οι συναρτήσεις 

      \twocolumnsiden{
        \begin{myitemize}
          \item $ F_{x} = -1 $
          \item $ F_{y} = 0 $
          \item $ F_{u} = \cos{v} $
          \item $ F_{v} = -u\sin{v} $
        \end{myitemize}
      }{
        \begin{myitemize}
          \item $ G_{x} = -1 $
          \item $ G_{y} = -1 $
          \item $ G_{u} = \sin{v} $
          \item $ G_{v} = u\cos{v} $
        \end{myitemize}
      }

      είναι συνεχείς. Επίσης η ορίζουσα 
      \[
        J = \pdv{(F,G)}{(u,v)} = \begin{vmatrix*}[r]
          F_{u} & F_{v} \\
          G_{u} & G_{v} 
          \end{vmatrix*} = \begin{vmatrix*}[r]
          cosv & - u \sin{v} \\
          \sin{v} & u \cos{v}
        \end{vmatrix*} = 
        u \cos^{2}{v} + u \sin^{2}{v} = u( \cos^{2}{v} + \sin^{2}{v}) = u 
        \neq 0 \Leftrightarrow u \neq 0
      \]
      Οπότε σε αυτήν την περίπτωση ισχύουν οι τύποι του θεωρήματος πεπλεγμένης συνάρτησης 
      για τις συναρτήσεις $ u(x,y) $ και $ v(x,y) $, δηλαδή:
      \begin{align*}
        u_{x} &= - \frac{\pdv{(F,G)}{(x,v)}}{J} = - \frac{\begin{vmatrix*}[r]
            F_{x} & F_{v} \\
            G_{x} & G_{v}
          \end{vmatrix*}
          }{J} = - \frac{\begin{vmatrix*}[r]
            -1 & -u \sin{v} \\
            -1 & u \cos{v}
          \end{vmatrix*}
        }{u} = - \frac{-u \cos{v} - u \sin{v}}{u} = 
        \cos{v} + \sin{v} \\[0.5\baselineskip]
        u_{y} &= - \frac{\pdv{(F,G)}{(y,v)}}{J} = - \frac{\begin{vmatrix*}[r]
            F_{y} & F_{v} \\
            G_{y} & G_{v}
          \end{vmatrix*}
          }{J} = - \frac{\begin{vmatrix*}[r]
            0 & -u \sin{v} \\
            -1 & u \cos{v}
          \end{vmatrix*}
        }{u} = - \frac{-u \sin{v}}{u} = \sin{v} \\[0.5\baselineskip]
        v_{x} &= - \frac{\pdv{(F,G)}{(u,x)}}{J} = - \frac{\begin{vmatrix*}[r]
            F_{u} & F_{x} \\
            G_{u} & G_{x}
          \end{vmatrix*}
          }{J} = - \frac{\begin{vmatrix*}[r]
            \cos{v} & -1 \\
            \sin{v} & -1
          \end{vmatrix*}
        }{u} = - \frac{- \cos{v} + \sin{v}}{u} = \frac{\cos{v} - \sin{v}}{u}
        \\[0.5\baselineskip]
        v_{y} &= - \frac{\pdv{(F,G)}{(u,y)}}{J} = - \frac{\begin{vmatrix*}[r]
            F_{u} & F_{y} \\
            G_{u} & G_{y}
          \end{vmatrix*}
          }{J} = - \frac{\begin{vmatrix*}[r]
            \cos{v} & 0 \\
            \sin{v} & -1
          \end{vmatrix*}
        }{u} = - \frac{- \cos{v}}{u} = \frac{\cos{v}}{u}
      \end{align*} 
      Άρα, έχουμε
      \[
        \pdv{(u,v)}{(x,y)} = 
        \begin{vmatrix*}[r]
          u_{x} & u_{y} \\
          v_{x} & v_{y}
        \end{vmatrix*} = 
        \begin{vmatrix*}[r]
          \cos{v} + \sin{v} & \sin{v} \\[5pt]
          \frac{\cos{v} - \sin{v}}{u} & \frac{\cos{v}}{u}
        \end{vmatrix*} = \frac{\cos^{2}{v} + \sin{v} \cos{v}}{u} - \frac{\sin{v}
        \cos{v} - \sin^{2}{v}}{u} = \frac{1}{u}
      \]

    \item [Β τρόπος:]

      Λύνουμε τον δοσμένο μετασχηματισμό, ως προς $x$ και $y$
      \[
        \begin{rcases}  
          u \cos{v} = x+1 \\
          u \sin{v} = x+y
        \end{rcases} \Leftrightarrow 
        \begin{rcases}
          x = 1 - u \cos{v} \\
          y = u \sin{v} - x
        \end{rcases} \Leftrightarrow 
        \begin{rcases}
          x = 1 - u \cos{v} \\
          y = u \sin{v} - 1 + u \cos{v}
        \end{rcases}
      \]
      Η Ιακωβιανή ορίζουσα του μετασχηματισμού, αυτού είναι:
      \[
        \pdv{(x,y)}{(u,v)} = 
        \begin{vmatrix*}[r]
          x_{u} & x_{v} \\
          y_{u} & y_{v}
        \end{vmatrix*} = 
        \begin{vmatrix*}[c]
          \cos{v} & u \sin{v} \\
          \sin{v} + \cos{v} & u \cos{v} - u \sin{v}
        \end{vmatrix*} = u \cos^{2}{v} - u \sin{v} \cos{v} + u \sin^{2}{v}
        + u \sin{v} \cos{v} = u
      \]
      Αν $ u \neq 0 $ τότε ο μετασχηματισμός είναι ομαλός, άρα αντιστρέψιμος και
      τότε, για την ορίζουσα του αντίστροφου μετασχηματισμού, ισχύει:
      \[
        \pdv{(u,v)}{(x,y)} = \frac{1}{\pdv{(x,y)}{(u,v)}} =
        \frac{1}{u}
      \] 
  \end{description}
\end{solution}


\end{document}

