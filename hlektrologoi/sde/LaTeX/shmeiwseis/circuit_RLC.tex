\documentclass[a4paper,table]{report}
\input{preamble_ask.tex}
\input{definitions_ask.tex}
\input{myboxes.tex}


\usepackage[RPvoltages]{circuitikz}

\everymath{\displaystyle}
\thispagestyle{empty}


\begin{document}



% \chapter*{Ανώμαλα Σημεία}

\begin{center}
  \minibox{\large\bfseries \textcolor{Col1}{Κύκλωμα $ RLC $}}
\end{center}

\vspace{\baselineskip} 


Θεωρούμε ένα ηλεκτρικό κύκλωμα σε σειρά $RLC$, το οποίο περιλαμβάνει μια ωμική αντίσταση 
$R$, ένα πηνίο με αυτεπαγωγή $ L $ και έναν πυκνωτή με χωρητικότητα $ C $ καθώς και μια 
ηλεκτρεγερτική δύναμη τάσης $ E(t) $. Όταν το κύκλωμα είναι κλειστό, διαρρέεται από 
ρεύμα $ i(t) $ όπου $t$ ο χρόνος. Η εξίσωση που διέπει το ρεύμα προκύπτει λαμβάνοντας 
υπόψη τις πτώσεις τάσης οι οποίες οφείλονται στα επί μέρους στοιχεία του κυκλώματος και 
είναι αντίστοιχα:

\twocolumnsidesc{
\begin{myitemize}
  \item $ V_{L} = L \dv{i}{t} $
  \item $ V_{R} = iR $
  \item $ V_{C} = \frac{q}{C} = \frac{1}{C} \int i(t) \,{dt}  $
\end{myitemize}
}{
\begin{center}
  \begin{circuitikz}[american,cute inductors,scale=0.8]
    \draw (0,0) to[short,i=$i$,battery1,l=$E$] (0,3) to[L=$L$]  (4,3) to[C=$C$]
    (4,0) to[R=$R$]  (0,0) ;
  \end{circuitikz}
\end{center}
}
όπου $ q $ είναι το ηλεκτρικό φορτίο του πυκνωτή και $ i(t) = \dv{q}{t} $.  
Σύμφωνα με το νόμο του Kirchhoff, και αν υποθέσουμε ότι $ E(t) = E_{0}
\sin{\omega t} $, όπου $ E_{0} = $ σταθ.\ έχουμε ότι 
\begin{equation}\label{eq:eq1}
  V_{L} + V_{R} + V_{C} = E(t) \Leftrightarrow L \dv{i}{t} + Ri + \frac{1}{C}
  \int i(t) \,{dt} = E_{0} \sin{\omega t}
\end{equation}
όπου $ R,L $ και $ C $ είναι γνωστές σταθερές.
Παραγωγίζουμε την~\eqref{eq:eq1} ως προς $t$ και προκύπτει
\begin{equation}\label{eq:eq2}
  L \dv[2]{i}{t} + R \dv{i}{t} + \frac{1}{C} i = E_{0} \omega \cos{\omega t} 
  \Leftrightarrow \boxed{L i'' + R i' + \frac{1}{C} i = E_{0} \omega \cos{\omega t}}
\end{equation} 
\begin{rem}
\item {}
  \begin{myitemize}
    \item Παρατηρούμε ότι το $R$ παίζει το ρόλο του συντελεστή απόσβεσης.
    \item Λάμβάνοντας υπόψη τις σχέσεις $ \int i(t) \,{dt} = q $, $ i= q' $ και 
      $ i' = q'' $ η εξίσωση~\eqref{eq:eq1} μπορεί να εκφρασθεί και σε 
      όρους φορτίου $ q(t) $, έτσι ώστε
      \begin{equation}\label{eq:eq3}
        \boxed{L q'' + Rq' + \frac{1}{C} q = E_{0} \sin{\omega t}}
      \end{equation}
  \end{myitemize}
\end{rem}

Οι εξισώσεις~\eqref{eq:eq2}, και~\eqref{eq:eq3} είναι Γραμμικές, 2ης τάξης, με σταθερούς
συντελεστές. 
 
%todo να συνεχίσω τη θεωρία στα κυκλώματα

\end{document}
