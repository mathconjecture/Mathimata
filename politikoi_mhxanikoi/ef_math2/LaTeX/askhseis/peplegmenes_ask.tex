\documentclass[a4paper,table]{report}
\input{preamble_ask.tex}
\newcommand{\vect}[2]{(#1_1,\ldots, #1_#2)}
%%%%%%% nesting newcommands $$$$$$$$$$$$$$$$$$$
\newcommand{\function}[1]{\newcommand{\nvec}[2]{#1(##1_1,\ldots, ##1_##2)}}

\newcommand{\linode}[2]{#1_n(x)#2^{(n)}+#1_{n-1}(x)#2^{(n-1)}+\cdots +#1_0(x)#2=g(x)}

\newcommand{\vecoffun}[3]{#1_0(#2),\ldots ,#1_#3(#2)}

\newcommand{\mysum}[1]{\sum_{n=#1}^{\infty}





\begin{document}

\begin{center}
    \minibox{\bfseries\large\color{Col1} Σύνθετη Παράγωγος - Πεπλεγμένες}
\end{center} 


\section*{Παράγωγος Σύνθετων Συναρτήσεων}

\begin{enumerate}
    \item Έστω $ f(x,y) = x^{2}+y $ με $ x = t \sin{t} $ και $ y = \cos{t} $. 
        Να υπολογιστεί η $ \dv{f}{t} $. 

        \hfill Απ: $ \dv{f}{t} = 2t \sin^{2}{t} + 2t^{2} \sin{t} \cos{t} - \sin{t} $ 

    \item  Έστω $ f(x,y) = x^{2}+y^{2} $ με $ x = r \cos{\theta} $ και 
        $ y = r \sin{\theta} $. Να υπολογιστούν οι $ \pdv{f}{r} $ και 
        $ \pdv{f}{\theta} $
        \hfill Απ: $ \pdv{f}{r} = 2r $, $ \pdv{f}{\theta} = 0 $  
    \item Έστω $ f(x,y) = \sin{(x-y)} $ με $ x = uv $ και $ y = u-v $. Να 
        υπολογιστούν οι $ \pdv[2]{f}{u} $ και $ \pdv[2]{f}{v} $.

        \hfill Απ: \begin{tabular}{l}
            $ \pdv[2]{f}{u} = -(v-1)^{2} \sin{(x-y)} $ \\
            $ \pdv[2]{f}{v} = -(u+1)(v-1) \sin{(x-y)} + \cos{(x-y)} $
        \end{tabular}

    \item Έστω $ z = e^{uv} $ με $ u = \ln{(x+y)} $ και $ v = \arctan{\frac{x}{y}
        } $. Να υπολογιστούν οι $ \pdv{z}{x} $ και $ \pdv{z}{y} $. 
        \hfill Απ: \begin{tabular}{l}
            $ z_{x} = \frac{ve^{uv}}{x+y} + \frac{yue^{uv}}{x^{2}+y^{2}} $ \\
            $ z_{y} = \frac{ve^{uv}}{x+y} - \frac{xe^{uv}}{x^{2}+y^{2}} $
        \end{tabular}

\end{enumerate}

\section*{Παράγωγος Σύνθετων Συναρτήσεων} 

\begin{enumerate}
  \item Αν $ f(x,y) = \arctan{\frac{x}{y}} $ με $ x=u+v $ και $ y = u-v $ 
    να δείξετε ότι $ f_{u} + f_{v} = \frac{u-v}{u^{2}+v^{2}} $.

  \item  Αν $ z=z(u,v) $ με $ u=x \cos{a} - y \sin{a} $ και $ v= x \sin{a} + y \cos{a} $
    να δείξετε ότι $ (z_{x})^{2} + (z_{y})^{2} = (z_{u})^{2} + (z_{v})^{2} $.

  \item Αν $ z=f(x,y) $, με $ x=u+v $ και $ y = u-v $ να δείξετε ότι 
    $ (z_{x})^{2} - (z_{y})^{2} = z_{u}\cdot z_{v} $ 

  \item Έστω $ z=z(x,y) $ συνάρτηση δύο μεταβλητών, με $ x+y= \ln{u+v} $ και 
    $ x-y = \ln{(u-v)} $. Να δείξετε ότι 
    \[
      z_{xx}+z_{yy} = (u^{2}-v^{2})(z_{uu}-z_{vv}) 
    \]
  \item Δίνεται η συνάρτηση δύο μεταβλητών, με τύπο
    $ z = f(u) + g(v) $, όπου $ u = ax + by $ και $ v = ax - by $, όπου 
    $ a,b \in \mathbb{R} $. Να αποδείξετε ότι 
    \[
      a^{2} z_{yy} - b^{2}z_{xx} = 0 
    \] 
  \item Αν $ z = f(x,y) $, με $ x=r \cos{\theta} $ και $ y = r \sin{\theta} $, 
    τότε να δείξετε ότι
    \[
      \left(\pdv{f}{x}\right)^{2} + \left(\pdv{f}{y}\right)^{2} = 
      \left(\pdv{f}{r}\right)^{2} + \frac{1}{ r^{2}} 
      \left(\pdv{f}{\theta}\right)^{2} 
    \] 
  \item Έστω $ g(x,y) = f(x^{2} - y^{2}, y^{2} - x^{2}) $, όπου η συνάρτηση 
    $f$ είναι διαφορίσιμη. Να αποδείξετε ότι η συνάρτηση $g$ ικανοποιεί 
    την διαφορική εξίσωση με μερικές παραγώγους
    \[
      x\pdv{g}{y} + y\pdv{g}{x} = 0
    \] 
  \item Να δείξετε ότι η συνάρτηση $ z(x,y) = xyf\left(\frac{ x-y }{ xy }\right) $
    ικανοποιεί τη σχέση $ x^{4} z_{xx} = y^{4} z_{yy} $.


\end{enumerate}

\section*{Θεωρήματα Πεπλεγμένων Συναρτήσεων}

\subsection*{Εξίσωση της μορφής \ensuremath{F(x,y)=0}}

\begin{enumerate}
    \item Έστω ότι η εξίσωση $ x-1-2y+e^{xy} = 0 $ ορίζει πεπλεγμένη συνάρτηση της 
        μορφής $ y = y(x) $. Να υπολογιστούν οι μερικές παράγωγοι 
        1ης και 2ης τάξης, αυτής της συνάρτησης.

        \hfill Απ: $ \dv{y}{x} = \frac{1+ye^{xy}}{2-xe^{xy}} $ και $ \dv[2]{y}{x}
        = \frac{(y+xy')^{2}+2y'}{2-xe^{xy}} $ 

    \item Έστω ότι η εξίσωση $ y^{3} - yx $ ορίζει πεπλεγμένη συνάρτηση της 
        μορφής $ y = y(x) $. Να υπολογιστεί η μερική παράγωγος 2ης τάξης, αυτής 
        της συνάρτησης.

        \hfill Απ: $ \dv[2]{y}{x} = \frac{ -2xy }{ (3y^{2} - x)^{3} }  $ 

    \item Να δείξετε ότι η εξίσωση $ e^{x+y} + y- x = 0 $ ορίζει πεπλεγμένη συνάρτηση 
        $ y=y(x) $ σε μια περιοχή του σημείου $ \left(\frac{1}{2}, - 
        \frac{1}{2}\right) $, και να βρεθεί η παράγωγός $ \dv{y}{x} $.

        \hfill Απ: $ \dv{y}{x} = \frac{1-e^{x+y}}{1+e^{x+y}} $  
\end{enumerate}

\subsection*{Εξίσωση της μορφής \ensuremath{F(x,y,z)=0}}

\begin{enumerate}

    % \item Να δείξετε ότι η εξίσωση $ y^{2} + xz + z^{2} - e^{z} - 4 = 0 $ ορίζει 
    %     πεπλεγμένη συνάρτηση $ z = z(x,y) $ σε μια περιοχή του σημείου $ M(0,e,2) $ και 
    %     να υπολογίσετε τις μερικές παραγώγους $ z_{x} $ και $ z_{y} $ στο σημείο $M$.

    %     \hfill Απ: $ z_{x} = - \frac{2}{4-e^{2}} $ και $ z_{y} = - \frac{2e}{4-e^{2}} $ 

    \item Να δείξετε ότι η εξίσωση $ 2x-y-z- \ln{(x+y+z-2)} e^{x+y} = 0 $ ορίζει 
        πεπλεγμένη συνάρτηση $ z = z(x,y) $ σε μια περιοχή του σημείου $ M(1,1,1) $ και 
        να υπολογίσετε τις μερικές παραγώγους $ z_{x} $ και $ z_{y} $ στο σημείο $M$.

        \hfill Απ: $ z_{x} =  \frac{2-e^{2}}{1+e^{2}} $ και $ z_{y} = -1 $ 

    \item Να δείξετε ότι η εξίσωση $ x \cos{y} + y \cos{z} + z \cos{x} -1 = 0  $ ορίζει 
        πεπλεγμένη συνάρτηση $ z=z(x,y) $ σε μια περιοχή του σημείου 
        $ (0,0) $, τέτοια ώστε $ z(0,0) = 1 $ και να υπολογιστούν οι μερικές παράγωγοι 
        $ z_{x} $ και $ z_{y} $.

        \hfill Απ: $ z_{x} = - \frac{\cos{y} - z \sin{x}}{\cos{x} - y \sin{z}} $ και 
        $ z_{y} = - \frac{\cos{z} - x \sin{y}}{\cos{x} - y \sin{z}} $ 

    \item Έστω η εξίσωση $ z^{3} - xz - y = 0 $. 
        \begin{enumerate}[i)]
            \item Να βρεθούν τα σημεία $ (x,y,z) \in \mathbb{R}^{3} $ για τα οποία 
                ορίζεται πεπλεγμένη συνάρτηση $ z=z(x,y) $. 
            \item Να υπολογιστούν οι $ z_{x}(0,1) $ και $ z_{yy}(0,1) $. 
            \item Να υπολογιστεί το ανάπτυγμα Taylor της συνάρτησης $ z(x,y) $ 
                γύρω από το σημείο $ (0,1) $, μέχρι όρους 2ης τάξης.
        \end{enumerate}

        \hfill Απ: \begin{tabular}{l}
            $ \mathrm{i)} \; 3{z_{0}}^{2} \neq x_{0} $ \\
            $ \mathrm{ii)} \; z_{x}(0,1) = \frac{1}{3} $ και 
            $ z_{y}(0,1) = -\frac{2}{9} $ \\
            $ \mathrm{iii)} \; z(x,y) = 1 + \frac{1}{3} x + \frac{1}{3}(y-1) 
            - \frac{1}{9} x(y-1) - \frac{2}{9} (y-1)^{2} + \cdots $
        \end{tabular}

    \item Έστω η εξίσωση $ y - 2xz + z^{3} = 0 $. Να αποδείξετε ότι η εξίσωση 
        ορίζει μια πεπλεγμένη συνάρτηση $ z = z(x,y) $ στην περιοχή του σημείου 
        $ P(1,1,1) $ και στη συνέχεια να υπολογίσετε το πολυώνυμο Taylor αυτής 
        της συνάρτησης γύρω από το σημείο $ (1,1) $, μέχρι και όρους 2ης τάξης. 

        \hfill Απ: $ z(x,y) =  1 + 2(x-1) - (y-1) - 8(x-1)^{2} + 10(x-1)(y-1) - 
        3(y-1)^{2} + \cdots $ 

\end{enumerate}


\section*{Σύστημα Εξισώσεων}

\begin{enumerate}


  \item Να υπολογίσετε τις παραγώγους $ \dv{y}{x} $ και $ \dv{z}{x}$ των πεπλεγμένων 
    συναρτήσεων $ y=y(x) $ και $ z=z(x) $ που ορίζονται από το σύστημα των εξισώσεων
    \[
      \begin{rcases}
        x^{3} + y^{3} + z^{3} -3xyz = 0 \\
        x+y+z=4 
      \end{rcases}
    \]

    \hfill Απ: $ \dv{y}{x} = \frac{ z-x }{ y-z }, \dv{z}{x} = \frac{ x-y }{ y-z } $ 

  \item Να υπολογίσετε τις μερικές παραγώγους $ z_{x}$, $ z_{y} $ και $ w_{x} $, 
    $ w_{y} $ των συναρτήσεων $ z = z(x,y) $ και $ w = w(x,y) $ που ορίζονται 
    πεπλεγμένα από το σύστημα 
    \[
      \begin{rcases}
        -2 \sin{x} + 2y + z + w = 0  \\
        2x + 2 \cos{y} + z - w = 0 
      \end{rcases} 
    \]

    \hfill Απ: \begin{tabular}{l}
      $ z_{x} = -1 + \cos{x}, z_{y} = -1 + \sin{y} $ \\
      $ w_{x} = 1 + \cos{x}, w_{y} = -1 - \sin{y}  $ 
    \end{tabular}

  \item Να δείξετε, επαληθεύοντας τις προϋποθέσεις του θεωρήματος πεπλεγμένων
    συναρτήσεων, ότι το σύστημα 
    \[
      \begin{rcases}
        2x-y+z^{3}-w^{2}-1=0 \\
        x+y+z^{2}+w^{3}-4=0
      \end{rcases} 
    \]
    ορίζει τις πεπλεγμένες συναρτήσεις $ z = z(x,y) $ και $ w = w(x,y) $
    σε μια περιοχή του σημείου $ (1,1) $ οι οποίες περνούν από 
    το σημείο $ P(1,1,1,1) $.

  \item Να δείξετε ότι το σύστημα των εξισώσεων 
    \[
      \begin{rcases}
        x^{2}-y^{2}+uv-v^{2}+3 = 0 \\
        x+y^{2}+u^{2}+uv-2=0
      \end{rcases}
    \]
    μπορεί να λυθεί ως προς $ u = u(x,y) $ και $ v=v(x,y) $ σε κατάλληλη 
    περιοχή του σημείου $ (2,1,-1,2) $, και να υπολογιστούν οι μερικές παράγωγοι 
    $ u_{x} $, $ u_{y} $ και $ v_{x} $, $ v_{y} $ στο σημείο αυτό.

    \hfill Απ: \begin{tabular}{l}
      $ u_{x} = 8 $ και $ u_{y} = 4 $ \\
      $ v_{x} = 1 $ και $ v_{y} = 2 $
    \end{tabular} 

  \item Έστω ότι οι εξισώσεις για τις οποίες ισχύει ότι $ u \neq v $
    \[
      \begin{rcases}
        x=u+v \\
        y=u^{2}+v^{2} \\
        z=u^{3}+v^{3}
      \end{rcases}
    \] 
    ορίζουν πεπλεγμένη συνάρτηση της μορφής $ z = z(x,y) $. Να υπολογιστούν οι 
    μερικές παράγωγοι $ z_{x} $ και $ z_{y} $ αυτής της συνάρτησης.

    \hfill Απ: $ z_{x} = -3uv $ και $ z_{y} = \frac{3}{2} (u+v) $  
\end{enumerate}




\end{document}
