\documentclass[a4paper,12pt]{article}
\usepackage{etex}
%%%%%%%%%%%%%%%%%%%%%%%%%%%%%%%%%%%%%%
% Babel language package
\usepackage[english,greek]{babel}
% Inputenc font encoding
\usepackage[utf8]{inputenc}
%%%%%%%%%%%%%%%%%%%%%%%%%%%%%%%%%%%%%%

%%%%% math packages %%%%%%%%%%%%%%%%%%
\usepackage{amsmath}
\usepackage{amssymb}
\usepackage{amsfonts}
\usepackage{amsthm}
\usepackage{proof}

\usepackage{physics}

%%%%%%% symbols packages %%%%%%%%%%%%%%
\usepackage{bm} %for use \bm instead \boldsymbol in math mode 
\usepackage{dsfont}
\usepackage{stmaryrd}
%%%%%%%%%%%%%%%%%%%%%%%%%%%%%%%%%%%%%%%


%%%%%% graphicx %%%%%%%%%%%%%%%%%%%%%%%
\usepackage{graphicx}
\usepackage{color}
%\usepackage{xypic}
\usepackage[all]{xy}
\usepackage{calc}
\usepackage{booktabs}
\usepackage{minibox}
%%%%%%%%%%%%%%%%%%%%%%%%%%%%%%%%%%%%%%%

\usepackage{enumerate}

\usepackage{fancyhdr}
%%%%% header and footer rule %%%%%%%%%
\setlength{\headheight}{14pt}
\renewcommand{\headrulewidth}{0pt}
\renewcommand{\footrulewidth}{0pt}
\fancypagestyle{plain}{\fancyhf{}
\fancyhead{}
\lfoot{}
\rfoot{\small \thepage}}
\fancypagestyle{vangelis}{\fancyhf{}
\rhead{\small \leftmark}
\lhead{\small }
\lfoot{}
\rfoot{\small \thepage}}
%%%%%%%%%%%%%%%%%%%%%%%%%%%%%%%%%%%%%%%

\usepackage{hyperref}
\usepackage{url}
%%%%%%% hyperref settings %%%%%%%%%%%%
\hypersetup{pdfpagemode=UseOutlines,hidelinks,
bookmarksopen=true,
pdfdisplaydoctitle=true,
pdfstartview=Fit,
unicode=true,
pdfpagelayout=OneColumn,
}
%%%%%%%%%%%%%%%%%%%%%%%%%%%%%%%%%%%%%%

\usepackage[space]{grffile}

\usepackage{geometry}
\geometry{left=25.63mm,right=25.63mm,top=36.25mm,bottom=36.25mm,footskip=24.16mm,headsep=24.16mm}

%\usepackage[explicit]{titlesec}
%%%%%% titlesec settings %%%%%%%%%%%%%
%\titleformat{\chapter}[block]{\LARGE\sc\bfseries}{\thechapter.}{1ex}{#1}
%\titlespacing*{\chapter}{0cm}{0cm}{36pt}[0ex]
%\titleformat{\section}[block]{\Large\bfseries}{\thesection.}{1ex}{#1}
%\titlespacing*{\section}{0cm}{34.56pt}{17.28pt}[0ex]
%\titleformat{\subsection}[block]{\large\bfseries{\thesubsection.}{1ex}{#1}
%\titlespacing*{\subsection}{0pt}{28.80pt}{14.40pt}[0ex]
%%%%%%%%%%%%%%%%%%%%%%%%%%%%%%%%%%%%%%

%%%%%%%%% My Theorems %%%%%%%%%%%%%%%%%%
\newtheorem{thm}{Θεώρημα}[section]
\newtheorem{cor}[thm]{Πόρισμα}
\newtheorem{lem}[thm]{λήμμα}
\theoremstyle{definition}
\newtheorem{dfn}{Ορισμός}[section]
\newtheorem{dfns}[dfn]{Ορισμοί}
\theoremstyle{remark}
\newtheorem{remark}{Παρατήρηση}[section]
\newtheorem{remarks}[remark]{Παρατηρήσεις}
%%%%%%%%%%%%%%%%%%%%%%%%%%%%%%%%%%%%%%%




\newcommand{\vect}[2]{(#1_1,\ldots, #1_#2)}
%%%%%%% nesting newcommands $$$$$$$$$$$$$$$$$$$
\newcommand{\function}[1]{\newcommand{\nvec}[2]{#1(##1_1,\ldots, ##1_##2)}}

\newcommand{\linode}[2]{#1_n(x)#2^{(n)}+#1_{n-1}(x)#2^{(n-1)}+\cdots +#1_0(x)#2=g(x)}

\newcommand{\vecoffun}[3]{#1_0(#2),\ldots ,#1_#3(#2)}

\newcommand{\mysum}[1]{\sum_{n=#1}^{\infty}




\begin{document}


\begin{center}
  \minibox{\large \bfseries \textcolor{Col1}{Θεώρημα Πεπλεγμένων Συναρτήσεων (Σύστημα)}}
\end{center}

\vspace{\baselineskip}


\begin{example}
  Δίνεται το παρακάτω σύστημα εξισώσεων:
  \[
    \begin{rcases*}
      F^1(x,y,w;z)=0 \\
      F^2(x,y,w;z)=0 \\
      F^3(x,y,w;z)=0
    \end{rcases*}
    \Leftrightarrow
    \begin{rcases*}
      xy-w &=0 \\
      y-w^3-3z &=0 \\
      w^3 + z^3 -2zw &=0
    \end{rcases*}
  \]
  οι οποίες ικανοποιούνται στο σημείο $P(x,y,w;z)=(\frac{1}{4},4,1,1)$. 
  Να βρεθεί η $\dv{x}{z}$.
\end{example}
\begin{solution}
  Αρχικά, εξετάζουμε τις προυποθέσεις του θεωρήματος πεπλεγμένης συνάρτησης.
  \begin{enumerate}[i)]
    \item Το σημείο $P(x,y,w;z)=(\frac{1}{4},4,1,1)$ ικανοποιεί το σύστημα των 
      εξισώσεων.  (από υπόθεση) 
    \item Οι μερικές παράγωγοι των $F^1, F^2, F^3$ ως προς όλες τις μεταβλητές υπάρχουν 
      και είναι όλες συνεχείς. (οι $F^1, F^2, F^3$ είναι πολυωνυμικές)
    \item Βρίσκουμε την Ιακωβιανή Ορίζουσα.
      \[
        |J|=
        \begin{vmatrix}
          F^1_x & F^1_y & F^1_w \\
          F^2_x & F^2_y & F^2_w \\
          F^3_x & F^3_y & F^3_w
          \end{vmatrix} = \begin{vmatrix}
          y & x & -1 \\
          0 & 1 & -3w^2 \\
          0 & 0 & 3w^2-2z
        \end{vmatrix}= y(3w^2-2z).
      \]
      Στη συνέχεια εξετάζουμε την Ιακωβιανή Ορίζουσα στο σημείο $P$.
      \[
        |J|_P=4\cdot(3\cdot 1^2-2\cdot 1)=4\neq 0.
      \]
  \end{enumerate}

  Άρα ικανοποιούνται οι συνθήκες του θεωρήματος πεπλεγμένης συνάρτησης:
  Υπάρχουν συναρτήσεις της μορφής:

  \begin{enumerate}[i)]
    \item $x=f^1(z)$
    \item $y=f^2(z)$
    \item $w=f^3(z)$.
  \end{enumerate}

  Εμείς ενδιαφερόμαστε για την συνάρτηση $x=f^1(z)$ γιατί ζητάμε την παράγωγο 
  $\dv{x}{z}$.

  Έτσι εφαρμόζουμε τον κανόνα πεπλεγμένης συνάρτησης:

  \begin{align*}
    \begin{vmatrix}
      F^1_x & F^1_y & F^1_w \\[2pt]
      F^2_x & F^2_y & F^2_w \\[2pt]
      F^3_x & F^3_y & F^3_w
    \end{vmatrix}\cdot 
    \begin{pmatrix}
      \dv{x}{z} \\[2pt]
      \dv{y}{z} \\[2pt]
      \dv{w}{z}
      \end{pmatrix}&=\begin{pmatrix}
      -\pdv{F^1}{z} \\[2pt]
      -\pdv{F^2}{z} \\[2pt]
      -\pdv{F^3}{z}
    \end{pmatrix}\Leftrightarrow \\
    \begin{pmatrix}
      y & x & -1 \\
      0 & 1 & -3w^2 \\
      0 & 0 & 3w^2-2z
      \end{pmatrix}\cdot \begin{pmatrix}
      \dv{x}{z} \\[2pt]
      \dv{y}{z} \\[2pt]
      \dv{w}{z}
      \end{pmatrix} &= \begin{pmatrix}
      0 \\[2pt]
      3\\[2pt]
      2w-3z^2
    \end{pmatrix}
  \end{align*}

  Χρησιμοποιώντας τον κανόνα του Crammer λύνουμε ως προς $\dv{x}{z}$ και έχουμε:
  \[
    \dv{x}{z}=\frac{|J_x|}{|J|}=\frac{\begin{vmatrix}
        0 & x & -1 \\
        3 & 1 & -3w^2 \\
        2w-3z^2 & 0 & 3w^2-2z 
        \end{vmatrix}}{|J|}=\frac{\begin{vmatrix}
        0 & \frac{1}{4} & -1 \\
        3 & 1 & -3 \\
        -1 & 0 & 1
    \end{vmatrix}}{4}=-\frac{1}{4}<0.
  \]

  Επομένως η συνάρτηση $x$ είνα μια φθίνουσα συνάρτηση του $z$ και άρα μια αύξηση του 
  $z$ κατά 1 μονάδα θα επιφέρει μείωση του $x$ κατά $\frac{1}{4}$ μονάδες.
    \end{solution}


    \end{document}
