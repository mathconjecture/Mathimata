\documentclass[a4paper,12pt]{article}
\usepackage{etex}
%%%%%%%%%%%%%%%%%%%%%%%%%%%%%%%%%%%%%%
% Babel language package
\usepackage[english,greek]{babel}
% Inputenc font encoding
\usepackage[utf8]{inputenc}
%%%%%%%%%%%%%%%%%%%%%%%%%%%%%%%%%%%%%%

%%%%% math packages %%%%%%%%%%%%%%%%%%
\usepackage{amsmath}
\usepackage{amssymb}
\usepackage{amsfonts}
\usepackage{amsthm}
\usepackage{proof}

\usepackage{physics}

%%%%%%% symbols packages %%%%%%%%%%%%%%
\usepackage{bm} %for use \bm instead \boldsymbol in math mode 
\usepackage{dsfont}
\usepackage{stmaryrd}
%%%%%%%%%%%%%%%%%%%%%%%%%%%%%%%%%%%%%%%


%%%%%% graphicx %%%%%%%%%%%%%%%%%%%%%%%
\usepackage{graphicx}
\usepackage{color}
%\usepackage{xypic}
\usepackage[all]{xy}
\usepackage{calc}
\usepackage{booktabs}
\usepackage{minibox}
%%%%%%%%%%%%%%%%%%%%%%%%%%%%%%%%%%%%%%%

\usepackage{enumerate}

\usepackage{fancyhdr}
%%%%% header and footer rule %%%%%%%%%
\setlength{\headheight}{14pt}
\renewcommand{\headrulewidth}{0pt}
\renewcommand{\footrulewidth}{0pt}
\fancypagestyle{plain}{\fancyhf{}
\fancyhead{}
\lfoot{}
\rfoot{\small \thepage}}
\fancypagestyle{vangelis}{\fancyhf{}
\rhead{\small \leftmark}
\lhead{\small }
\lfoot{}
\rfoot{\small \thepage}}
%%%%%%%%%%%%%%%%%%%%%%%%%%%%%%%%%%%%%%%

\usepackage{hyperref}
\usepackage{url}
%%%%%%% hyperref settings %%%%%%%%%%%%
\hypersetup{pdfpagemode=UseOutlines,hidelinks,
bookmarksopen=true,
pdfdisplaydoctitle=true,
pdfstartview=Fit,
unicode=true,
pdfpagelayout=OneColumn,
}
%%%%%%%%%%%%%%%%%%%%%%%%%%%%%%%%%%%%%%

\usepackage[space]{grffile}

\usepackage{geometry}
\geometry{left=25.63mm,right=25.63mm,top=36.25mm,bottom=36.25mm,footskip=24.16mm,headsep=24.16mm}

%\usepackage[explicit]{titlesec}
%%%%%% titlesec settings %%%%%%%%%%%%%
%\titleformat{\chapter}[block]{\LARGE\sc\bfseries}{\thechapter.}{1ex}{#1}
%\titlespacing*{\chapter}{0cm}{0cm}{36pt}[0ex]
%\titleformat{\section}[block]{\Large\bfseries}{\thesection.}{1ex}{#1}
%\titlespacing*{\section}{0cm}{34.56pt}{17.28pt}[0ex]
%\titleformat{\subsection}[block]{\large\bfseries{\thesubsection.}{1ex}{#1}
%\titlespacing*{\subsection}{0pt}{28.80pt}{14.40pt}[0ex]
%%%%%%%%%%%%%%%%%%%%%%%%%%%%%%%%%%%%%%

%%%%%%%%% My Theorems %%%%%%%%%%%%%%%%%%
\newtheorem{thm}{Θεώρημα}[section]
\newtheorem{cor}[thm]{Πόρισμα}
\newtheorem{lem}[thm]{λήμμα}
\theoremstyle{definition}
\newtheorem{dfn}{Ορισμός}[section]
\newtheorem{dfns}[dfn]{Ορισμοί}
\theoremstyle{remark}
\newtheorem{remark}{Παρατήρηση}[section]
\newtheorem{remarks}[remark]{Παρατηρήσεις}
%%%%%%%%%%%%%%%%%%%%%%%%%%%%%%%%%%%%%%%




\input{definitions_ask.tex}

\pagestyle{askhseis}
\geometry{top=2.7cm}

\renewcommand{\vec}{\mathbf}

\begin{document}

\begin{center}
  \minibox{\large \bfseries \textcolor{Col1}{Ασκήσεις στα Διανύσματα}}
\end{center}

\vspace{\baselineskip}

\begin{enumerate}[itemsep=0.7\baselineskip]

  \item Να υπολογιστεί η \textbf{γωνία} $ \theta $, $ 0\leq \theta \leq \pi $ μεταξύ των 
    διανυσμάτων $ \vec{a} = (3,0,3) $ και $ \vec{b} = (0,6,6) $.
    \hfill Απ: $ \pi/{3} $

  \item Να δειχθεί με τη βοήθεια του εσωτερικού και εξωτερικού γινομένου ότι
    \begin{enumerate}[i)]
      \item τα διανύσματα $ \vec{a}_1 = (5,-2,1) $ και $ \vec{b}_1 = (-15,6,-3) $ είναι 
        \textbf{παράλληλα}.
      \item τα διανύσματα $ \vec{a}_2 = (4,-3,-2) $ και $ \vec{b}_2 = (-1,-2,1) $ είναι 
        \textbf{κάθετα}.
    \end{enumerate}

  \item Να υπολογιστεί το εξωτερικό γινόμενο των διανυσμάτων 
    $ \vec{a} = (2,-1,1) $ και $ \vec{b} = (-1,2,-3) $.
    \hfill Απ: $ (1,5,3) $

  \item Να υπολογιστεί το \textbf{εμβαδό} του παραλληλογράμμου που σχηματίζεται από τα 
    διανύσματα $ \vec{a} = (3,-2,-2) $ και $ \vec{b} = (-1,0,-1) $.
    \hfill Απ: $ \sqrt{33} $ 

  \item Να γραφεί το διάνυσμα $ \vec{a}=(-3,10,10) $ ως \textbf{γραμμικός συνδυασμός} 
    των διανυσμάτων $\vec{b}=(1,-1,2)$, $ \vec{c}=(2,0,3) $ και $ \vec{d}=(-1,4,3) $.  
    \hfill Υπόδειξή: Λύση $ 3\times 3 $ σύστημα. 

    \hfill Απ: $ \vec{a}=2\vec{b}-\vec{c}+3\vec{d} $

  \item Να βρεθεί διάνυσμα $ \vec{b} $ που έχει μέτρο $3$ και είναι αντίρροπο του 
    διανύσματος $ \vec{a}=(4,2,-4)$.
    \hfill Υπόδειξη: $ \vec{b} = \lambda \vec{a} $		

    \hfill Απ: $ \vec{b}=(-2,-1,2) $

  \item Να βρεθεί η \textbf{προβολή} του διανύσματος 
    $ \mathbf{v} = 6 \mathbf{i}+3 \mathbf{j} + 2 \mathbf{k} $ πάνω στο διάνυσμα 
    $ \mathbf{u} = \mathbf{i}-2 \mathbf{j} - \mathbf{k} $.

    \hfill Απ: $ \rm{pr}_{\mathbf{u}}{\mathbf{v}} = -\frac{1}{3} (\mathbf{i}-2
    \mathbf{j}- \mathbf{k}) $ 

  \item Να βρεθεί η \textbf{συνιστώσα} του διανύσματος 
    $ \mathbf{v} = 6 \mathbf{i}+3 \mathbf{j} + 2 \mathbf{k} $ πάνω στο διάνυσμα 
    $ \mathbf{u} = \mathbf{i}-2 \mathbf{j} - \mathbf{k} $.  
    \hfill Απ: $ -2/ \sqrt{6} $ 

  \item Να βρεθεί το \textbf{εμβαδό} του τριγώνου, με κορυφές τα σημεία $ A(1,-1,0) $, 
    $ B(2,1,-1) $ και $ C(-1,1,2) $. 
    \hfill Απ: $ 3 \sqrt{2} $ 

  % \item Έστω διανύσματα $ \vec{a}=(3,2,1) $ και $ \vec{b}=(1,-1,2) $.
  %   \begin{enumerate}[(i)]
  %     \item Να βρεθούν όλα τα διανύσματα του $ \mathbb{R}^{3} $ που είναι κάθετα στα $ 
  %       \vec{a} $ και $ \vec{b} $.
  %     \item Ποια από τα παραπάνω έχουν μέτρο ίσο με $3$; 
  %   \end{enumerate}
    % \hfill Υπ: Έστω $ \vec{c} = (x,y,z) $
    % \hfill Απ: \begin{tabular}{l}
    %   (i) $ \vec{c} = (x,-x,-x), x\in \mathbb{R}$, $a\times b = (5,-5,-5)$ \\
    %   (ii) $ (\sqrt{3}, - \sqrt{3}, - \sqrt{3}), (-\sqrt{3}, \sqrt{3}, \sqrt{3})$ 
    % \end{tabular}

  \item Να δείξετε ότι 
    \begin{enumerate}[(i)]
      \item $ \norm{\mathbf{u}+ \mathbf{v}}^{2} - \norm{\mathbf{u}- \mathbf{v}}^{2} 
        = 4 \mathbf{u}\cdot \mathbf{v} $
      \item $ \norm{\mathbf{u}+ \mathbf{v}}^{2} + \norm{\mathbf{u}- \mathbf{v}}^{2} 
        = 2 \norm{\mathbf{u}} ^{2} + 2 \norm{\mathbf{v}} ^{2} $
    \end{enumerate}

  \item Να δείξετε ότι οι παρακάτω ισότητες αληθεύουν αν και μόνον αν 
    τα διανύσματα $ \mathbf{a}, \mathbf{b} $ είναι γραμμικώς εξαρτημένα.
    \begin{enumerate}[(i)]
      \item $(\vec{a}\cdot \vec{b})^{2} = \vec{a}^{2}\cdot \vec{b}^{2}$
      \item $|\vec{a}\cdot \vec{b}| = ||\vec{a}|| \cdot ||\vec{b}||$
    \end{enumerate}

  \item Να δείξετε ότι $ (\vec{a}\times \vec{b})^{2} + (\vec{a}\cdot \vec{b})^{2} =
    \vec{a}^{2}\cdot \vec{b}^{2} $  

  \item Να δείξετε ότι
    \begin{enumerate}[i)]
      \item $ ( \vec{a} - \vec{b} ) \times ( \vec{a} + \vec{b} ) = 2 (\vec{a} \times 
        \vec{b}) $
      \item Πως ερμηνεύεται γεωμετρικά το παραπάνω αποτέλεσμα αν τα διανύσματα 
        $ \vec{a} $ και $ \vec{b} $ είναι γραμμικώς ανεξάρτητα;
    \end{enumerate}

  % \item Αν $ \vec{a} = (1,1,0) $ και $ \vec{b} = (1,2,-1) $ και η γωνία που 
    % σχηματίζουν 
  %   είναι $ \phi\neq \frac{\pi}{2} $ να υπολογιστεί η $ \tan{\phi} $ 
  %   \hfill Υπ: $ \tan{\phi} = \frac{\sin{\phi}}{\cos{\phi}} $
  %   \hfill Απ: $ \tan{\phi} = \frac{\sqrt{3}}{3}  $

  \item Να δείξετε ότι τα σημεία $ A(1,2,3), B(4,2,4), C(2,4,0) $ και $ D(2,1,5) $ 
    είναι συνεπίπεδα.

  \item Να βρείτε το $x$ και $y$ ώστε τα διανύσματα $ \vec{a} = (x^{2}+y^{2},2,13) $, 
    $ \vec{b} = (1,2x-3y,1) $ να είναι μεταξύ τους κάθετα. 

    \hfill Απ: $ x=-2 $, $ y=3 $

  \item Να δείξετε ότι τα σημεία $ A(1,2,3) $, $ B(4,2,4) $, $ C(2,4,0) $ και 
    $ D(-1,1,5) $ είναι κορυφές τετραέδρου και στη συνέχεια να υπολογίσετε τον όγκο του.
    \hfill Απ: $ V=1 $

  \item Να βρεθεί μοναδιαίο διάνυσμα, συνεπίπεδο με τα $ \vec{a} = (1,2,3) $ και 
    $ \vec{b} = (-1,0,2) $ και κάθετο στο $ \vec{b} $.

    \hfill Απ: $ (\frac{2}{3}, \frac{2}{3}, \frac{1}{3}) $,$ (-\frac{2}{3},- 
    \frac{2}{3}, -\frac{1}{3})$  
\end{enumerate}

\end{document}

