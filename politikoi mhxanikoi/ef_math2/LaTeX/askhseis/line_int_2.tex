\input{/home/vangelis/Desktop/MyNotesLaTeX/preamble.tex}
\newcommand{\vect}[2]{(#1_1,\ldots, #1_#2)}
%%%%%%% nesting newcommands $$$$$$$$$$$$$$$$$$$
\newcommand{\function}[1]{\newcommand{\nvec}[2]{#1(##1_1,\ldots, ##1_##2)}}

\newcommand{\linode}[2]{#1_n(x)#2^{(n)}+#1_{n-1}(x)#2^{(n-1)}+\cdots +#1_0(x)#2=g(x)}

\newcommand{\vecoffun}[3]{#1_0(#2),\ldots ,#1_#3(#2)}

\newcommand{\suma}{\sum_{n=0}^{\infty}a_n x^n}

\newcommand{\sumb}{\sum_{n=1}^{\infty}a_n n x^{n-1}}

\newcommand{\sumc}{\sum_{n=2}^{\infty}a_n n (n-1) x^{n-2}}

\newcommand{\varsum}[2]{\sum_{n=#1}^{#2}}

\thispagestyle{empty}
\everymath{\displaystyle}

\begin{document}

\begin{center}
\fbox{\large\bf Ασκήσεις Επικαμπυλιο Ολοκλήρωμα (IΙου Ειδους)}
\end{center}

\vspace{\baselineskip}

\begin{enumerate} 

\item Να υπολογιστουν τα παρακατω επικαμπυλια ολοκληρωματα:

\begin{enumerate}[i)]

\item $\int\limits_C \vec{F}\,d\vec{r}$, οπου $\vec{F}(x,y,z)=(3x^2-6yz)\vec{i}+(2y+3xz)\vec{j}+(1-4xyz^2)\vec{k}$ και $C$ η καμπυλη με παραμετρικες εξισωσεις $x(t)=t, y(t)=t^2, z(t)=t^3$, με $t\in [0,1]$

\hfill Απ: $2$

\end{enumerate}

\item Να υπολογιστει το επικαμπυλιο ολοκληρωμα $I=\int\limits_C xdy-ydx$ απο το σημειο $A(0,0)$ εως το σημειο $B(1,1)$ κατα μηκος των παρακατω καμπυλων. (Τι παρατηρειτε?)

\begin{enumerate}[i)]
\item της ευθειας $y=x$
\item των ευθειων $y=0$ και $x=1$
\item των ευθειων $x=0$ και $y=1$
\item της παραβολης $y=x^2$

\hfill Απ: \begin{tabular}{l}
$I=0$ \\
$I=1$ \\
$I=-1$ \\
$I=\sfrac{1}{3}$
\end{tabular}
\end{enumerate}

\item Να υπολογιστει το επικαμπυλιο ολοκληρωμα $\int\limits_C (y+z)dx$, οπου $C$ ειναι η καμπυλη με διανυσματικη παραμετρικη εξισωση $\vec{r}(t)=\cos t\vec{i}+\sin t\vec{k}$, με $t\in [0,2\pi]$.

\hfill Απ: $-\pi$

\item Να υπολογιστει το επικαμπυλιο ολοκληρωμα $I=\int\limits_C (xy+z)dx+(xy+z)dy+(xy+z)dz$, οπου $C$ η καμπυλη $y^2+z^2=a^2$, $x=0$ με αρχη το σημειο $A(0,0,0$ και περας το σημειο $B(0,a,0)$

\hfill Απ: $-\frac{a^2}{2}+\frac{\pi a^2}{4}$

\underline{\textbf{Υποδειξη:}}
Οταν η καμπυλη δινεται ως συναρτηση της μορφης $y=f(x)$, τοτε για την παραμετρικοποιηση της, θετω: $x=t$ και $y=f(t)$. 

πχ. Αν $C:y=3x^2$, με $x\in [2,3]$ τοτε θετω: $x(t)=t$ και $y(t)=3t^2$ με $t\in [2,3]$.

\end{enumerate}


\end{document}