\documentclass[a4paper,table]{report}
\input{preamble_themata}
\newcommand{\vect}[2]{(#1_1,\ldots, #1_#2)}
%%%%%%% nesting newcommands $$$$$$$$$$$$$$$$$$$
\newcommand{\function}[1]{\newcommand{\nvec}[2]{#1(##1_1,\ldots, ##1_##2)}}

\newcommand{\linode}[2]{#1_n(x)#2^{(n)}+#1_{n-1}(x)#2^{(n-1)}+\cdots +#1_0(x)#2=g(x)}

\newcommand{\vecoffun}[3]{#1_0(#2),\ldots ,#1_#3(#2)}

\newcommand{\mysum}[1]{\sum_{n=#1}^{\infty}

\input{myboxes}

\everymath{\displaystyle}
\pagestyle{vangelis}


\begin{document}

\begin{mybox3}
  \begin{thema}
    Να βρεθούν τα μοναδιαία διανύσματα $ T,N,B $ της καμπύλης $ \mathbf{r}(t)=t^{2}\,
    \mathbf{i} + 2t\, \mathbf{j} + t \, \mathbf{k} $ κατά τη χρονική στιγμή που 
    το μοναδιαίο κάθετο διάνυσμά της είναι παράλληλο προς το επίπεδο $ x+y+z=0 $.
  \end{thema}
\end{mybox3}
\begin{solution}
  \item {}
    Το μοναδιαίο κάθετο διάνυσμα της καμπύλης είναι το $ \mathbf{N} $. 
    \begin{myitemize}
      \item $ \mathbf{r}'(t) = 2t \mathbf{i} + 2 \mathbf{j} + 1 \mathbf{k} $ 
      \item $ \norm{\mathbf{r}'(t)} = \sqrt{(2t)^{2}+2^{2}+1^{2}} = \sqrt{4t^{2}+5} $
      \item $ \mathbf{T} = \frac{\mathbf{r}'(t)}{\norm{\mathbf{r}'(t)}} =  
        \frac{2t}{\sqrt{4t^{2}+5}} \mathbf{i} + \frac{2}{\sqrt{4t^{2}+5}}
        \mathbf{j} + \frac{1}{\sqrt{4t^{2}+5}} \mathbf{k}$
    \end{myitemize}
    Επειδή η παραγώγιση του $ \mathbf{T} $ θα μας οδηγήσει σε παράγωγο πηλίκου,
    προκειμένου να το αποφύγουμε, θα βρούμε το $ \mathbf{N} $ μέσω του τύπου $ \mathbf{N}
    = \mathbf{B} \times \mathbf{T} $, αφού πρώτα υπολογίσουμε το $ \mathbf{B} $ μέσω του 
    τύπου $ \mathbf{B} = \frac{\mathbf{r}' \times \mathbf{r}''}{\norm{\mathbf{r}'
    \times \mathbf{r}''}} $.
    \begin{myitemize}
      \item $ \mathbf{r}''(t) = 2 \mathbf{i} + 0 \mathbf{j} + 0 \mathbf{k} $
      \item $ \mathbf{r}' \times \mathbf{r}'' = 
        \begin{vmatrix}
          \mathbf{i} & \mathbf{j} & \mathbf{k} \\
          2t & 2 & 1 \\
          2 & 0 & 0
        \end{vmatrix} = 2 \mathbf{j} - 4 \mathbf{k}
        $
      \item $ \norm{\mathbf{r}' \times \mathbf{r}''} = \sqrt{0^{2}+2^{2}+(-4)^{2}} =
        \sqrt{20} = 2 \sqrt{5}$
      \item $ \mathbf{B} = \frac{\mathbf{r}' \times \mathbf{r}''}{\norm{\mathbf{r}' 
        \times \mathbf{r}''}} = \frac{2}{2 \sqrt{5}} \mathbf{j} - 
        \frac{4}{2 \sqrt{5}} \mathbf{k} = \frac{1}{\sqrt{5}} \mathbf{j} - 
        \frac{2}{\sqrt{5}} \mathbf{k} $
      \item $ \mathbf{N} = \mathbf{B} \times \mathbf{T} = 
        \begin{vmatrix*}
          \mathbf{i} & \mathbf{j} & \mathbf{k} \\
          0 & \frac{1}{\sqrt{5}} & - \frac{2}{\sqrt{5}} \\
          \frac{2t}{\sqrt{4t^{2}+5}} & \frac{2}{\sqrt{4t^{2}+5}} & 
          \frac{1}{\sqrt{4t^{2}+5}} 
        \end{vmatrix*} = \frac{5}{\sqrt{5} \sqrt{4t^{2}+5}} \mathbf{i} -
        \frac{4t}{\sqrt{5} \sqrt{4t^{2}+5}} \mathbf{j} - \frac{2t}{\sqrt{5} 
        \sqrt{4t^{2}+ 5}}
        $ 
    \end{myitemize}
    Γνωρίζουμε ότι ένα διάνυσμα \textbf{κάθετο} στο επίπεδο $ \Pi: x+y+z = 0 $ είναι το $
    (1,1,1) $ 

    \begin{mybox1}
      \vspace{0.5\baselineskip}
      \textcolor{Col1}{\textbf{Θυμάμαι:}} 
      Αν $ \textcolor{Col2}{a}x+ \textcolor{Col2}{b} y+ \textcolor{Col2}{c} z=0 $ είναι 
      εξίσωση επιπέδου, τότε ένα διάν. κάθετο στο επίπεδο είναι το $ \vec{n} 
      = ( \textcolor{Col2}{a} , \textcolor{Col2}{b} , \textcolor{Col2}{c}) $. 
    \end{mybox1}

    Επομένως $ \mathbf{N} \parallel \Pi \Leftrightarrow \mathbf{N} \perp (1,1,1)
    \Leftrightarrow \mathbf{N} \cdot (1,1,1) = 0 \Leftrightarrow $
    \begin{align*}
      \Bigl(\frac{5}{\sqrt{5} \sqrt{4t^{2}+5}} \mathbf{i} -
        \frac{4t}{\sqrt{5} \sqrt{4t^{2}+5}} \mathbf{j} - \frac{2t}{\sqrt{5} 
        \sqrt{4t^{2}+ 5}}\Bigr) \cdot (1,1,1) = 0 \Leftrightarrow \frac{5-4t-2t}{\sqrt{5}
      \sqrt{4t^{2}+5}} = 0 \Leftrightarrow 5-4t-2t = 0 \Leftrightarrow 
    \end{align*} 
    \begin{empheq}[box=\mathboxg]{equation*}
      t= \frac{5}{6}
    \end{empheq}
    Άρα 
    \begin{myitemize}
      \item $ \mathbf{T}\Bigl(\frac{5}{6}\Bigr) = \cdots $
      \item $ \mathbf{N}\Bigl(\frac{5}{6}\Bigr) = \cdots $
      \item $ \mathbf{B}\Bigl(\frac{5}{6}\Bigr) = \cdots $
    \end{myitemize}

  \end{solution}





  \end{document}
