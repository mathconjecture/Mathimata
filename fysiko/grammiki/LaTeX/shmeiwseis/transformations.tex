\documentclass[a4paper]{report}


\input{preamble_ask.tex}
\newcommand{\vect}[2]{(#1_1,\ldots, #1_#2)}
%%%%%%% nesting newcommands $$$$$$$$$$$$$$$$$$$
\newcommand{\function}[1]{\newcommand{\nvec}[2]{#1(##1_1,\ldots, ##1_##2)}}

\newcommand{\linode}[2]{#1_n(x)#2^{(n)}+#1_{n-1}(x)#2^{(n-1)}+\cdots +#1_0(x)#2=g(x)}

\newcommand{\vecoffun}[3]{#1_0(#2),\ldots ,#1_#3(#2)}

\newcommand{\mysum}[1]{\sum_{n=#1}^{\infty}




\begin{document}

\begin{example}
  Έστω τελεστής $ T \colon \mathbb{R}^{2} \to \mathbb{R}^{2} $ με 
  $ T(x,y) = (x+y,-x-2y) $. 
  \begin{enumerate}[i)]
    \item Να βρεθεί η παράσταση του τελεστή $ T $ ως προς τη συνήθη βάση, 
      $\beta = \{ (1,0), (0,1) \} $.
    \item Να βρεθεί η παράσταση του τελεστή ως προς τη βάση 
      $ \gamma = \{ (1,1), (1,2) \} $.
  \end{enumerate}
\end{example}
\begin{solution}
\item {}
  \begin{enumerate}[i)]
    \item 
      \begin{myitemize}
        \item $ T(1,0) = (1+0,-1-2\cdot0) = (1,-1) $
        \item $ T(0,1) = (0+1,-0-2\cdot1) = (1,-2) $
      \end{myitemize}
      Άρα 
      \[
        [T]_{\beta} 
        \begin{pmatrix*}[r]
          1 & 1 \\
          -1 & -2
        \end{pmatrix*}
      \] 
    \item 
      \begin{myitemize}
        \item $ T(1,1) = (1+1,-1-2\cdot1) = (2,-3) $
        \item $ T(1,2) = (1+2,-1-2\cdot2) = (3,-5) $
      \end{myitemize}
      Όμως τα διανύσματα $ (2,-3) $ και $ (3,-5) $ είναι εκφρασμένα ως προς τη συνήθη 
      βάση. Πρέπει να τα εκφράσουμε ως προς τη βάση $ \gamma $. Γι᾽ αυτό έχουμε:
      \[
        T(1,1)=(2,-3) = a(1,1)+b(1,2) = (a,a)+(b,2b)=(a+b,a+2b) \Leftrightarrow 
        \left.
          \begin{matrix}
            a+b=2 \\
            a+2b=-3
          \end{matrix} 
        \right\} 
      \] 
      άρα $ a=7 \quad \text{και} \quad b=-5 $.

  \end{enumerate}
\end{solution}

\end{document}
