\input{preamble_ask.tex}
\input{definitions_ask.tex}
\input{tikz.tex}
\input{myboxes.tex}

% \geometry{left=1.5cm,right=1.5cm,top=1.0cm}
\pagestyle{askhseis}

\begin{document}

\begin{center}
  \minibox{\bfseries\large \textcolor{Col1}{Ασκήσεις στις Καμπύλες}} 
\end{center} 

\vspace{\baselineskip}


\section*{Μήκος Τόξου}

\begin{enumerate}
  \item Για τις παρακάτω καμπύλες, να βρείτε το μοναδιαίο εφαπτόμενο διάνυσμα, 
    καθώς και το μήκος, για το ζητούμενο τμήμα της καμπύλης.
    \begin{enumerate}[i)]
      %thomas 13.3 ex. 1 p. 727
      \item $ \mathbf{r}(t)=2 \cos{t}\, \mathbf{i} + 2 \sin{t}\, \mathbf{j} +
        \sqrt{5} t \, \mathbf{k}$, \; με $ 0 \leq t \leq \pi $. \hfill Απ: $ 3 \pi $  
        %thomas 13.3 ex. 2 p. 727
      \item $ \mathbf{r}(t)=6 \sin{2t}\, \mathbf{i} + 6 \cos{2t}\, \mathbf{j} + 5t \, 
        \mathbf{k} $, \; με $ 0 \leq t \leq \pi $. \hfill Απ: $ 13 \pi $  
        %thomas 13.3 ex. 4 p. 727
      \item $ \mathbf{r}(t)=(2+t)\, \mathbf{i} - (t+1) \, \mathbf{j} + t \, \mathbf{k} $,
        \; με $ 0 \leq t \leq 3 $. \hfill Απ: $ 3 \sqrt{3} $ 
        %thomas 13.3 ex. 5 p. 727
      \item $ \mathbf{r}(t)= \cos^{3}{t}\, \mathbf{j} + \sin^{3}{t}\, \mathbf{k} $, 
        \; με $ 0 \leq t \leq \pi /2 $. \hfill Απ: $ 3/2 $ 
    \end{enumerate}

    %thomas 13.3 ex. 10 p. 727
  \item Να βρείτε το σημείο πάνω στην καμπύλη $ \mathbf{r}(t)=5 \sin{t}\, \mathbf{i} + 5
    \cos{t}\, \mathbf{j} + 12t \, \mathbf{k} $, που απέχει απόσταση $ 26 \pi $, κατά 
    μήκος της καμπύλης, από το σημείο $ (0,5,0) $. \hfill Απ: $ (0,5,24 \pi) $  

  \item Να υπολογίσετε το μήκος τόξου της καμπύλης $ \mathbf{r}(t)= \sqrt{2} t\, 
    \mathbf{i} + \sqrt{2} t \, \mathbf{j} + (1-t^{2}) \, \mathbf{k} $ από το σημείο 
    $ (0,0,1) $ έως το σημείο $ (\sqrt{2} , \sqrt{2} , 0) $. 
    \hfill Απ:  $ \sqrt{2} + \ln{(1+ \sqrt{2})} $

  \item  Θεωρούμε τις παρακάτω καμπύλες. Να γραφούν οι εξισώσεις με παράμετρο 
    το μήκος τόξου $s$.
    \begin{enumerate}[i)]
      \item  $\mathbf{r_{1}}(t) = \cos t \mathbf{i} + \sin t \mathbf{j} + t \mathbf{k}$, 
        $t\in [0,2\pi]$
      \item  $\mathbf{r_{2}}(t) = e^{t}\cos t \mathbf{i} + e^{t}\sin t \mathbf{j} + e^t
        \mathbf{k}$, $t\in [0,\pi]$
    \end{enumerate}
    Για την $ \mathbf{r_{2}}(t) $ να δείξετε ότι κάθε διάνυσμα του τριέδρου 
    Frenet της καμπύλης, στο τυχαίο σημείο, σχηματίζει σταθερή γωνία με τον 
    άξονα $ z $.

    \hfill  Απ: \begin{tabular}{l}
      $   \rm{i)}\quad \mathbf{r_{1}}(s) = \cos \frac{s}{\sqrt{2}} \mathbf{i} + \sin
      \frac{s}{\sqrt{2}} \mathbf{j} + \frac{s}{\sqrt{2}} \mathbf{k} $ \\
      $\rm{ii)}\quad \mathbf{r_{2}}(s) = {\color{red}q}\cos\ln({{\color{red}q}}) 
      \mathbf{i} + {\color{red}q}\sin\ln({{\color{red}q}}) \mathbf{j} + 
      {\color{red}q} \mathbf{k},\;\text{όπου}\; 
      {\color{red}{q}}=\left(\frac{s}{\sqrt{3}}+1\right)$
    \end{tabular}
\end{enumerate}

\section*{Τρίεδρο Frenet}

\begin{enumerate}
  \item Να βρείτε τα διανύσματα $ T $ και $N$, καθώς και την καμπυλότητα των παρακάτω 
    επίπεδων καμπυλών.
    %Thomas 13.4 ex. 1 p. 733
    \begin{enumerate}[i)]
      \item $ \mathbf{r}(t)=(2t+3)\, \mathbf{i} + (5-t^{2})\, \mathbf{j} $ 
        \hfill Απ: \begin{tabular}{l}
          $ \mathbf{T} =  \frac{1}{\sqrt{1+t^{2}}}\,\mathbf{i} -
          \frac{1}{\sqrt{1+t^{2}}}\,\mathbf{j} $
          \\
          $ \mathbf{N}=  \frac{-t}{\sqrt{1+t^{2}}}\,\mathbf{i} + 
          \frac{-1}{\sqrt{1+t^{2}}} \,\mathbf{j} $ \\
          $ k= \frac{1}{2(1+t^{2})^{3/2}} $ 
        \end{tabular} 
      \item $ \mathbf{r}(t)=(\cos{t} + t \sin{t})\, \mathbf{i} + (\sin{t} - t
        \cos{t})\, \mathbf{j} , \quad t>0 $
        \hfill Απ: \begin{tabular}{l}
          $ \mathbf{T}=  \cos{t}\,\mathbf{i} + \sin{t}\,\mathbf{j} $ \\
          $ \mathbf{N}=  - \sin{t}\,\mathbf{i} + \cos{t}\,\mathbf{j} $ \\
          $ k = \frac{1}{t} $
        \end{tabular} 
    \end{enumerate}

  \item Να βρείτε τα διανύσματα $ T $, $N$ και $B$ καθώς και την καμπυλότητα και στρέψη  
    των παρακάτω καμπυλών στο χώρο.
    \begin{enumerate}[i)]
      \item $ \mathbf{r}(t)=(3 \sin{t})\, \mathbf{i} + (3 \cos{t})\, \mathbf{j} + 4t 
        \, \mathbf{k} $ 
        \hfill Απ: \begin{tabular}{l}
          $ \mathbf{T}=  3/5 \cos{t}\,\mathbf{i} - 3/5 \sin{t} \,\mathbf{j} + 4/5 
          \mathbf{k} $ \\
          $ \mathbf{N}=  - \sin{t}\,\mathbf{i} - \cos{t}\,\mathbf{j} $ \\
          $ \mathbf{Β}=  4/5 \cos{t}\,\mathbf{i} - 4/5\sin{t}\,\mathbf{j}-3/5 
          \mathbf{k} $ \\
          $ k = 3/25 $, $ \sigma = -4/25 $
        \end{tabular} 
      \item $ \mathbf{r}(t)=(\cos^{3}{t})\, \mathbf{i} + (\sin^{3}{t})\, \mathbf{j} + \,
        \mathbf{k} , \quad 0<t< \pi /2$ 
        \hfill Απ: \begin{tabular}{l}
          $ \mathbf{T}=  - \cos{t}\,\mathbf{i} + \sin{t}\,\mathbf{j}$ \\
          $ \mathbf{N}=  \sin{t}\,\mathbf{i} + \cos{t}\,\mathbf{j} $ \\
          $ \mathbf{B}= -\mathbf{k} $ \\
          $ k = \frac{1}{3 \cos{t} \sin{t}} $, $ \sigma = 0 $ 
        \end{tabular} 
    \end{enumerate}
\end{enumerate}



\end{document}
