\documentclass[a4paper,12pt]{article}
\usepackage{etex}
%%%%%%%%%%%%%%%%%%%%%%%%%%%%%%%%%%%%%%
% Babel language package
\usepackage[english,greek]{babel}
% Inputenc font encoding
\usepackage[utf8]{inputenc}
%%%%%%%%%%%%%%%%%%%%%%%%%%%%%%%%%%%%%%

%%%%% math packages %%%%%%%%%%%%%%%%%%
\usepackage{amsmath}
\usepackage{amssymb}
\usepackage{amsfonts}
\usepackage{amsthm}
\usepackage{proof}

\usepackage{physics}

%%%%%%% symbols packages %%%%%%%%%%%%%%
\usepackage{bm} %for use \bm instead \boldsymbol in math mode 
\usepackage{dsfont}
\usepackage{stmaryrd}
%%%%%%%%%%%%%%%%%%%%%%%%%%%%%%%%%%%%%%%


%%%%%% graphicx %%%%%%%%%%%%%%%%%%%%%%%
\usepackage{graphicx}
\usepackage{color}
%\usepackage{xypic}
\usepackage[all]{xy}
\usepackage{calc}
\usepackage{booktabs}
\usepackage{minibox}
%%%%%%%%%%%%%%%%%%%%%%%%%%%%%%%%%%%%%%%

\usepackage{enumerate}

\usepackage{fancyhdr}
%%%%% header and footer rule %%%%%%%%%
\setlength{\headheight}{14pt}
\renewcommand{\headrulewidth}{0pt}
\renewcommand{\footrulewidth}{0pt}
\fancypagestyle{plain}{\fancyhf{}
\fancyhead{}
\lfoot{}
\rfoot{\small \thepage}}
\fancypagestyle{vangelis}{\fancyhf{}
\rhead{\small \leftmark}
\lhead{\small }
\lfoot{}
\rfoot{\small \thepage}}
%%%%%%%%%%%%%%%%%%%%%%%%%%%%%%%%%%%%%%%

\usepackage{hyperref}
\usepackage{url}
%%%%%%% hyperref settings %%%%%%%%%%%%
\hypersetup{pdfpagemode=UseOutlines,hidelinks,
bookmarksopen=true,
pdfdisplaydoctitle=true,
pdfstartview=Fit,
unicode=true,
pdfpagelayout=OneColumn,
}
%%%%%%%%%%%%%%%%%%%%%%%%%%%%%%%%%%%%%%

\usepackage[space]{grffile}

\usepackage{geometry}
\geometry{left=25.63mm,right=25.63mm,top=36.25mm,bottom=36.25mm,footskip=24.16mm,headsep=24.16mm}

%\usepackage[explicit]{titlesec}
%%%%%% titlesec settings %%%%%%%%%%%%%
%\titleformat{\chapter}[block]{\LARGE\sc\bfseries}{\thechapter.}{1ex}{#1}
%\titlespacing*{\chapter}{0cm}{0cm}{36pt}[0ex]
%\titleformat{\section}[block]{\Large\bfseries}{\thesection.}{1ex}{#1}
%\titlespacing*{\section}{0cm}{34.56pt}{17.28pt}[0ex]
%\titleformat{\subsection}[block]{\large\bfseries{\thesubsection.}{1ex}{#1}
%\titlespacing*{\subsection}{0pt}{28.80pt}{14.40pt}[0ex]
%%%%%%%%%%%%%%%%%%%%%%%%%%%%%%%%%%%%%%

%%%%%%%%% My Theorems %%%%%%%%%%%%%%%%%%
\newtheorem{thm}{Θεώρημα}[section]
\newtheorem{cor}[thm]{Πόρισμα}
\newtheorem{lem}[thm]{λήμμα}
\theoremstyle{definition}
\newtheorem{dfn}{Ορισμός}[section]
\newtheorem{dfns}[dfn]{Ορισμοί}
\theoremstyle{remark}
\newtheorem{remark}{Παρατήρηση}[section]
\newtheorem{remarks}[remark]{Παρατηρήσεις}
%%%%%%%%%%%%%%%%%%%%%%%%%%%%%%%%%%%%%%%




\newcommand{\vect}[2]{(#1_1,\ldots, #1_#2)}
%%%%%%% nesting newcommands $$$$$$$$$$$$$$$$$$$
\newcommand{\function}[1]{\newcommand{\nvec}[2]{#1(##1_1,\ldots, ##1_##2)}}

\newcommand{\linode}[2]{#1_n(x)#2^{(n)}+#1_{n-1}(x)#2^{(n-1)}+\cdots +#1_0(x)#2=g(x)}

\newcommand{\vecoffun}[3]{#1_0(#2),\ldots ,#1_#3(#2)}

\newcommand{\mysum}[1]{\sum_{n=#1}^{\infty}



\everymath{\displaystyle}

\pagestyle{askhseis}


\begin{document}

\begin{center}
  \minibox{\large\bfseries \textcolor{Col1}{Ασκήσεις στις Συναρτήσεις}}
\end{center}

\vspace{\baselineskip}

\section*{Πεδίο Ορισμού}

\begin{enumerate}
  \item  Να βρείτε το πεδίο ορισμού των παρακάτω συναρτήσεων.
    \begin{enumerate}[i)]
      \item $ f(x) = x^{3} - 5x^{2} + 2x -3 $ \hfill Απ: $ \mathbb{R} $
      \item $ f(x) = \frac{x-2}{x-3} $ \hfill Απ: $ \mathbb{R} \setminus 
        \{ 3 \} $ 
      \item $ f(x) = \frac{x}{2x-1} $ \hfill Απ: 
        $ \mathbb{R} \setminus \left\{ \frac{1}{2} \right\} $ 
      \item $ f(x) = 2x^{5} - \frac{1}{x} + \frac{x^{3}}{x+3}  $ \hfill 
        Απ: $ \mathbb{R} \setminus \{ 0, -3 \} $
      \item $ f(x) = \frac{x-1}{x^{2} - 5x + 6} $ \hfill Απ: 
        $ \mathbb{R} \setminus \{ 2,3 \}  $ 
      \item $ f(x) = \frac{4}{x^{2}+x+1} $ \hfill Απ: $ \mathbb{R} $ 
      \item $ f(x) = \frac{2x-1}{x^{3}-8} $ \hfill Απ: 
        $ \mathbb{R} \setminus \{ 1 \}  $ 
      % \item $ f(x) = \frac{x+4}{x^{3}-4x} -2 + \frac{1}{x^{2}+2x} $ 
      %   \hfill Απ: $ \mathbb{R} \setminus \{ 0,-2,2 \} $ 
    \end{enumerate}

  \item  Να βρείτε το πεδίο ορισμού των παρακάτω συναρτήσεων.
    \begin{enumerate}[i)]
      \item $ f(x) = \sqrt{x-2} $ \hfill Απ: $[2,+\infty)$
      \item $ f(x) = \sqrt{4-x} -3x \sqrt{x+2} $ \hfill Απ: $ [-2,4] $ 
      \item $ f(x) = \sqrt{x^{2}-2x-3} $ \hfill Απ: $ (-\infty,-1] 
        \cup [3,+\infty) $  
      \item $ f(x) = \frac{\sqrt{x+3}}{x+1} $ \hfill Απ: 
        $ [-3,-1) \cup (-1,+\infty)  $ 
      \item $ f(x) = \frac{\sqrt{x-2}}{\sqrt{x-3}} $ \hfill Απ:
        $ ( 3, +\infty ) $  
      \item $ f(x) = \frac{\sqrt{\abs{x}-2}}{x-3} $ \hfill 
        Απ: $ ( -\infty, -2 ] \cup [2,3) \cup (3,+\infty) $ 
      \item $ f(x) = \frac{\sqrt{\abs{x}-x}}{\abs{x}-2} $ \hfill Απ: 
        $ (-\infty,-2) \cup (-2,0] $
    \end{enumerate}

  \item Να υπολογίσετε τις τιμές του πραγματικού αριθμού $ \lambda $, για τις 
    οποίες η συνάρτηση $ f(x) = \frac{x^{3}-8}{x^{2}-4x- \lambda} $ έχει πεδίο 
    ορισμού το $ \mathbb{R} $.

    \hfill Απ: $ \lambda < -4 $ 

  \item  Να βρείτε το πεδίο ορισμού των παρακάτω συναρτήσεων.
    \begin{enumerate}[i)]
      \item $ f(x) = \ln{(x-3)} $ \hfill Απ: $(3,+\infty)$
      \item $ f(x) = \ln{(-x^{2}+5x-6)} $ \hfill Απ: $ (2,3) $
      \item $ f(x) = \ln{(4x-x^{2})} $ \hfill Απ: $ (0,4) $
      % \item $ f(x) = \ln{(-x^{2}+3x-2)} $ \hfill Απ: $ (1,2) $ 
      \item $ f(x) = \frac{1}{\ln{(x-1)}} $ \hfill Απ: $ (1,2) \cup (2,+\infty) $ 
      \item $ f(x) = \frac{1+e^{x}}{1-e^{x}} $ \hfill Απ: 
        $ \mathbb{R} \setminus \{ 0 \} $ 
    \end{enumerate}
\end{enumerate}


\section*{Σύνολο Τιμών}


\begin{enumerate}
  \item Να υπολογίσετε το σύνολο τιμών των παρακάτω συναρτήσεων.
    \begin{enumerate}[i)]
      \item $ f(x) = 3x-5 $ \hfill Απ: $ \mathbb{R} $ 
      \item $ f \colon [-2,1) \to \mathbb{R} $ με $ f(x) = 3x-5 $ 
        \hfill Απ: $ [-11,-2) $ 
      \item $ f(x) = \frac{3x+1}{x-2} $ \hfill Απ: 
        $ \mathbb{R} \setminus \{ 3 \} $ 
      \item $ f \colon (-2,2) \to \mathbb{R} $ με $ f(x) = \frac{3x+2}{2x-5} $
        \hfill Απ: $ \left(-8, \frac{4}{9}\right) $ 
      \item $ f(x) = x^{2}-2x+5 $ \hfill Απ: $ [-6,+\infty) $ 
      \item $ f \colon [2,4) \to \mathbb{R} $ με $ f(x) = x^{2}-4x+3 $
        \hfill Απ: $ [-1,3) $
      \item $ f(x) = \frac{x^{2}-2x+2}{x-1} $ \hfill Απ:
        $ (-\infty,-2] \cup [2,+\infty) $
      % \item $ f(x) = \frac{x^{2}-x-2}{x^{2}-5x+6} $ \hfill Απ:
      %   $ \mathbb{R} \setminus \{ -3,1 \} $
      % \item $ f(x) = 5 + \sqrt{x^{2}-2x+3} $ \hfill Απ: 
      %   $ [5+ \sqrt{2}, +\infty) $
      % \item $ f(x) = \frac{1-a^{x}}{1+a^{x}} $, $ a>0 $ \hfill Απ: 
      %   $ (-1,1) $
      % \item $ f(x) = \ln{\left(\frac{a-x}{a+x}\right)} $, $ a>0 $ \hfill Απ: 
      %   $ \mathbb{R} $
    \end{enumerate}
\end{enumerate}





\end{document}

