\documentclass[a4paper]{report}
\input{preamble_ask.tex}
\newcommand{\vect}[2]{(#1_1,\ldots, #1_#2)}
%%%%%%% nesting newcommands $$$$$$$$$$$$$$$$$$$
\newcommand{\function}[1]{\newcommand{\nvec}[2]{#1(##1_1,\ldots, ##1_##2)}}

\newcommand{\linode}[2]{#1_n(x)#2^{(n)}+#1_{n-1}(x)#2^{(n-1)}+\cdots +#1_0(x)#2=g(x)}

\newcommand{\vecoffun}[3]{#1_0(#2),\ldots ,#1_#3(#2)}

\newcommand{\suma}{\sum_{n=0}^{\infty}a_n x^n}

\newcommand{\sumb}{\sum_{n=1}^{\infty}a_n n x^{n-1}}

\newcommand{\sumc}{\sum_{n=2}^{\infty}a_n n (n-1) x^{n-2}}

\newcommand{\varsum}[2]{\sum_{n=#1}^{#2}}

\geometry{left=1cm,right=1cm}
\pagestyle{askhseis}

\newcommand{\twocolumnsidesss}[2]{\begin{minipage}[t]{0.52\linewidth}\raggedright
    #1
    \end{minipage}\hspace{2pt}\hfill{\color{Col1}{\vrule width 1pt}}\hfill\begin{minipage}[t]{0.47\linewidth}\raggedright
    #2
  \end{minipage}
}



\begin{document}

\begin{center}
\minibox{\large\bfseries \textcolor{Col1}{Ασκήσεις Taylor και Maclaurin}}
\end{center}

\vspace{\baselineskip}

\begin{enumerate}


  \item Να υπολογιστεί το προσεγγιστικό πολυώνυμο Maclaurin 3ης τάξης, των 
    παρακάτω συναρτήσεων.

    \twocolumnsidesss{
      \begin{enumerate}[i)]
        \item $ y= \mathrm{e}^{-x} $ 
          \hfill Απ: $ \mathrm{e}^{-x} \approx 1-x+ \frac{1}{2} x^{2} - \frac{1}{6} x^{3} $ 
        \item $ y= \ln{(x+2)} $ \;
          \hfill Απ: $ \ln{(x+2)} \approx \ln{2} + \frac{1}{2} x - \frac{1}{8} x^{2} +
          \frac{1}{24} x^{3} $ 
      \end{enumerate}
    }{
      \begin{enumerate}[i),start=3]
        \item $ y= \frac{1}{x-1} $ \hfill Απ: $ \frac{1}{x-1} \approx -1 -x -x^{2} - x^{3} $ 
        \item $ y= \sqrt{x+1} $ \hfill Απ: $ \sqrt{x+1} \approx 1 + \frac{1}{2} x -
          \frac{1}{8} x^{2} + \frac{1}{16} x^{3} $ 
      \end{enumerate}
    }

  \item  Να υπολογιστεί το προσεγγιστικό πολυώνυμο Taylor 3ης τάξης, γύρω από 
    το σημείο $ x_{0}=1 $, των παρακάτω συναρτήσεων.
    \begin{enumerate}[i)]
      \item $ y= \ln{(x+1)} $ 
        \hfill Απ: $ \ln{(x+1)} \approx \ln{2} + \frac{1}{2} (x-1) - \frac{1}{8}
        (x-1)^{2} + \frac{1}{24} (x-1)^{3} $ 
      \item $ y= \frac{1}{x+1} $ \hfill Απ: $ \frac{1}{x+1} \approx \frac{1}{2} -
        \frac{1}{4} (x-1) + \frac{1}{8} (x-1)^{2} - \frac{1}{16} (x-1)^{3} $ 
    \end{enumerate}

  \item Να βρείτε μια πολυωνυμική προσέγγιση μέχρι και 
    όρους 2ης τάξης της συνάρτησης που ορίζεται πεπλεγμένα από την εξίσωση 
    $ x^{2} - xy + y^{2} = 3$ στο σημείο $ (1,-1) $.
    \hfill Απ: $f(x) \cong -1 + (x-1) + \frac{(x-1)^{2}}{3} $

  \item Έστω $ f(x) = \ln{(1+x)} $, $ x>-1 $. Να υπολογιστεί το ανάπτυγμα
    \textbf{Maclaurin} 4ης τάξης και στη συνέχεια να υπολογιστεί το αντίστοιχο 
    σφάλμα για $ x = 0,1 $.

    \hfill Απ: \begin{tabular}{l}
      $ \ln(1+x) \cong x - \frac{1}{2} x^{2} + \frac{1}{3}x^{3} - 
      \frac{1}{4} x^{4} $ \\ 
      $ \abs{R_{4}(0,1)} < 0,000002$	
    \end{tabular}

\end{enumerate}

\end{document}
