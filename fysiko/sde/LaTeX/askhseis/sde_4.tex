\input{preamble_ask.tex}
\input{definitions_ask.tex}

\geometry{left=15.63mm,right=15.63mm,top=32.25mm,bottom=36.25mm,footskip=24.16mm,headsep=24.16mm}

\everymath{\displaystyle}
\pagestyle{askhseis}

\begin{document}

\begin{center}
  \minibox{\large\bf \textcolor{Col1}{Ασκήσεις στο Μετασχηματισμό Laplace}}
\end{center}

\vspace{\baselineskip}

\begin{enumerate}

  \item Με χρήση μόνο των ιδιοτήτων και του πίνακα των βασικών μετασχηματισμών να βρεθεί 
    ο μετασχηματισμός Laplace των παρακάτω συναρτήσεων:
    \begin{enumerate}[i)]
      \item $f(t)=(\sin t-\cos t)^2$\hfill Απ: $F(s)=\frac{1}{s}-\frac{2}{s^2+4}$
      \item $f(t)=\cosh^2 4t$ \hfill Απ: $F(s)=\frac{1}{2s}+\frac{s}{2(s^2-64)}$
      \item $f(t)=e^{2t}\sin 3t$ \hfill Απ: $F(s)=\frac{3}{(s-2)^2+9}$
      \item $f(t)=t^3\sin 3t$ \hfill Απ: $\frac{72s(s^2-9)}{(s^2+9)^4}$
    \end{enumerate}

  \item Να βρεθεί, εφόσον υπάρχει, ο \textbf{αντίστροφος} μετασχηματισμός Laplace 
    των παρακάτω συναρτήσεων:
    \begin{enumerate}[i)]
      % \item $\frac{s^2}{s^2+1}$\hfill Απ: Δεν υπάρχει
      \item $\frac{6}{s^4}$\hfill Απ: $f(t)=t^3$
      \item $\frac{1}{s}+\frac{1}{s^2}$\hfill Απ: $f(t)=1+t$
      \item $\frac{1}{s-3}$\hfill Απ: $f(t)=e^{3t}$
      \item $\frac{1}{s^2-2s+10}$\hfill Απ: $f(t)=\frac{1}{3}e^t\sin 3t$
      \item $\frac{1}{s(s+1)}$\hfill Απ: $f(t)=1-e^{-t}$
      \item $\frac{1}{s(s+2)^2}$\hfill Απ: $f(t)=\frac{1}{4}-\frac{1}{4}e^{-2t}-
        \frac{1}{2}te^{-2t}$
      \item $\frac{1}{s(s^2+1)}$\hfill Απ: $f(t)=1-\cos t$
    \end{enumerate}

  \item Να υπολογίσετε το μετασχηματισμό Laplace των παρακάτω συναρτήσεων:
    \begin{enumerate}[i)]
      \item $f(t)=\begin{dcases} 0, & t<1 \\ t, & t>1 \end{dcases}$ \hfill 
        $\mathcal{L}\{f(t)\}=e^{-s}\Bigl(\frac{1}{s^2}+\frac{1}{s}\Bigr)$
      \item $g(t)=\begin{dcases} 0, & t<\frac{\pi}{2} \\ \sin t, & t>\frac{\pi}{2} 
        \end{dcases}$ \hfill $\mathcal{L}\{g(t)\}=e^{-\frac{s\pi}{2}}\frac{s}{s^2+1}$
      \end{enumerate}

    \item Να επιλυθούν με τη βοήθεια του μετασχηματισμού Laplace τα ακόλουθα προβλήματα 
      αρχικών τιμών:
      \begin{enumerate}[i)]
        \item $y'-y=e^{-t},\quad y(0)=0$\hfill Απ: $y(t)=\sinh t$
        \item $y'+y=\sin t,\quad y(0)=1$\hfill Απ: $y(t)=\frac{3}{2}e^{-t}+
          \frac{1}{2}(\sin t-\cos t)$
        \item $y''-5y'+6y=0,\quad y(0)=0, \; y'(0)=1$\hfill Απ: $y(t)=e^{3t}-e^{2t}$
        \item $y''+y=2e^{-t},\quad t>0,\; y(0)=-1,\; y'(0)=2$
          \hfill Απ: $y(t)=2e^{-t}-2\cos t+3 \sin t$
      \end{enumerate}

    \item\label{it:ekth} Να επιλυθούν τα ακόλουθα προβλήματα αρχικών τιμών, υποθέτωντας 
      ότι η συνάρτηση $ y(t) $ είναι εκθετικού τύπου.
      \begin{enumerate}[i)]
        \item $y''-ty'+y=5, \quad t>0,\; y(0)=5,\; y'(0)=3$\hfill Απ: $y(t)=3t+5$
        \item $y'' +5ty'-10y=2, \quad y(0)=1,\; y'(0)=0$\hfill Απ: $y(t)=6t^2+1$
        \item $y'' +ty'-2y=4, \quad y(0)=0,\; y'(0)=0$\hfill Απ: $y(t)=2t^2$
      \end{enumerate}

    \item Να επιλυθούν με τη βοήθεια του μετασχηματισμού Laplace τα ακόλουθα προβλήματα 
      αρχικών τιμών:

      \begin{enumerate}[i)]
        \item $y'+2y=g(t),\quad t>0,\quad y(0)=0$, όπου $g(t)=\begin{cases}3, & 
          0\leq t<1 \\ 1, & t\geq 1\end{cases}$

          \hfill Απ: $y(t)=\begin{cases} \frac{3}{2}-\frac{3}{2}e^{-2t},& 0\leq t<1 \\ 
          \frac{1}{2}+\left(e^2-\frac{3}{2}\right)e^{-2t}, & t\geq 1\end{cases}$

        \item $y''-3y'+2y=g(t),\quad t>0,\; y(0)=1,\; y'(0)=0$, όπου 
          $g(t)=\begin{cases} 0, & 0\leq t<2 \\ 1, & t\geq 2\end{cases}$

          \hfill Απ: $y(t)=2e^t-e^{2t}=\frac{1}{2}H(t-2)+\frac{1}{2}e^{2(t-2)}
          H(t-2)-e^{t-2}H(t-2)$
      \end{enumerate}

    \item Να επιλυθούν τα ακόλουθα συστήματα με τη μέθοδο του μετασχηματισμού Laplace.
      \begin{enumerate}[i)]
        \item \begin{tabular}{l}
            $ y_{1}'=2 y_{1}+ y_{2}, \quad y_{1}(0)=1 $ \\
            $ y_{2}'= 3 y_{1}+ 4 y_{2}, \quad y_{2}(0)= 0 $ 
          \end{tabular} \hfill Απ: 
          \begin{tabular}{l}
            $ y_{1}(t) = \frac{1}{4} \left(3 \mathrm{e}^{t} + \mathrm{e}^{5t}\right) $ \\
            $ y_{2}(t) = -\frac{3}{4} \left( \mathrm{e}^{t} -  \mathrm{e}^{5t}\right) $ 
          \end{tabular}
        \item \begin{tabular}{l}
            $ y_{1}'+ y_{2} = \mathrm{e}^{-t}, \quad y_{1}(0)=0 $ \\
            $ y_{2}' - y_{1} =3\mathrm{e}^{-t}, \quad y_{2}(0)=1 $
          \end{tabular} \hfill Απ: 
          \begin{tabular}{l}
            $ y_{1}(t) = (2 \cos{t} - 2 \sin{t} -2 \mathrm{e}^{-t}) $ \\
            $ y_{2}(t) = ( - \mathrm{e}^{-t} + 2 \cos{t} + 2 \sin{t}) $
          \end{tabular}
        \item \begin{tabular}{l}
            $ y_{1}' = 3 y_{1}- 3 y_{2}+2, \quad y_{1}(0)=1 $ \\
            $ y_{2}' = -6 y_{1} - t, \quad y_{2}(0)=-1 $
          \end{tabular}\hfill \hfill Απ: 
          \begin{tabular}{l}
            $y_{1}(t)=\frac{1}{108}\left(133 \mathrm{e}^{6t} - 28 \mathrm{e}^{-3t} + 3 -18t
            \right)$ \\
            $y_{2}(t)=-\frac{1}{108}\left(133 \mathrm{e}^{6t} + 56 \mathrm{e}^{-3t} + 18t -81
            \right)$ 
          \end{tabular} 
      \end{enumerate}

    \item\label{it:conv} 
      Να υπολογιστούν με τη βοήθεια της συνέλιξης, οι αντίστροφοι μετασχηματισμοί 
      Laplace των παρακάτω συναρτήσεων.
      \begin{enumerate}[i)]
        \item $ F(s) = \frac{1}{s^{2}(s+1)} $ \hfill Απ: $ y(t) = t-1+ \mathrm{e}^{-t} $ 
        \item $ F(s) = \frac{1}{(s-1)(s-2)} $ \hfill Απ: $ y(t) = - \mathrm{e}^{t} +
          \mathrm{e}^{2t} $ 
        \item $ F(s) = \frac{1}{s(s^{2}+4)} $ \hfill Απ: $ y(t) = \frac{1}{4} (1-
          \cos{2t}) $ 
        \item $ F(s) = \frac{s}{(s-1)(s^{2}+1)} $ \hfill Απ: $ y(t) = \frac{1}{2}
          (\sin{t} - \cos{t} + \mathrm{e}^{t}) $  
        \item\label{it:conv1} 
          $ F(s) = \frac{1}{(s^{2}+a^{2})^{2}} $ \hfill Απ: $ \frac{1}{2a^{3}} 
          (\sin{at} - at \cos{at}) $ 
      \end{enumerate}

      \end{enumerate}

      \vspace{\baselineskip}

      \section*{Υποδείξεις}

      \begin{enumerate}
        \item Για την άσκηση~\ref{it:ekth}, να χρησιμοποιήσετε τη σχέση
          $ \lim\nolimits_{s\to+\infty}F(s)=0 $, για να προσδιορίσετε τη σταθερά $c$, που 
          προκύπτει από την επίλυση της σδε 1ης τάξης.
        \item Για την άσκηση~\ref{it:conv}, στο ερώτημα~\ref{it:conv1} να χρησιμοποιήσετε
          τον τύπο $ \sin{(a-b)} = \sin{a} \cos{b} - \cos{a} \sin{b} $ 
      \end{enumerate}


            \end{document}
