\documentclass[a4paper,table]{report}
\input{preamble_ask.tex}
\input{definitions_ask.tex}

\pagestyle{askhseis}

\renewcommand{\vec}{\mathbf}

\begin{document}

\begin{center}
  \minibox{\large \bfseries \textcolor{Col1}{Ασκήσεις στα Ακρότατα}}
\end{center}

\vspace{\baselineskip}

\section*{Τοπικά Ακρότατα}

\begin{enumerate}
  \item Να βρεθούν και να χαρακτηριστούν τα κρίσιμα σημεία  των παρακάτω συναρτήσεων:
    \begin{enumerate}[i)]
      \item $ f(x,y) = x^{3} + y^{3} + 3xy $ 
        \hfill Απ: max: $(-1,-1)  $, σάγμα: $ (0,0) $
      \item $ f(x,y) = x^{2}+y^{4} $ 
        \hfill Απ: min: $ (0,0) $ 
      \item $ f(x,y) = x^{3} + y^{3} - 3x -12y + 50 $ 
        \hfill Απ: max: $ (-1,-2)$, min: $ (1,2) $, 
        σάγμα: $ (1,-2), (-1,2) $
      \item $ f(x,y) = x^{3} + y^{3} -3x -3y + 1 $ 
        \hfill Απ: max: $(-1,-1)  $, min: $ (1,1) $,
        σάγμα: $ (1,-1), (-1,1) $
      \item $ f(x,y) = x^{3} + 4xy -4y^{2} $ 
        \hfill Απ: max: $ (-2/3, -1/3)  $, σάγμα: $ (0,0) $
      \item $ f(x,y) = x^{4} + y^{4} -2(x-y)^{2}$  
        \hfill Απ: min: $ (\sqrt{2} , -\sqrt{2}), (-\sqrt{2} , \sqrt{2}) $, 
        σάγμα: $ (0,0) $
      \item $ f(x,y) = (x^{2}-3y^{2})e^{1-x^{2}-y^{2}} $ 
        \hfill Απ: max: $ (1,0), (-1,0) $, min: $ (0,1), (0,-1) $, 
        σάγμα: $ (0,0) $
    \end{enumerate}

  \item Να βρεθεί η ελάχιστη απόσταση του επιπέδου με εξίσωση $ x+y+z=4 $, από την 
    αρχή των αξόνων.

    \hfill Απ: $ d_{\min}(4/3,4/3) = 4\frac{\sqrt{3}}{3} $  

  \item Να βρεθεί η ελάχιστη απόσταση του επιπέδου με εξίσωση $ 3x+2y+z=6 $, από την 
    αρχή των αξόνων.

    \hfill Απ: $ d_{\min}(9/7,6/7) = 3\frac{\sqrt{14}}{7} $  

  \item Να βρεθεί η ελάχιστη απόσταση του σημείου $ P(2,-1,1) $ από το επίπεδο με 
    εξίσωση $ x+y-z=2 $. 
    
    \hfill Απ: $ d_{min}(8/3,-1/3) = 2 /\sqrt{3} $ 

  \item Να βρεθεί η ελάχιστη απόσταση του σημείου $ P(-6,4,0) $ από τον κώνο με 
    εξίσωση $ z = \sqrt{x^{2}+y^{2}} $. 
    
    \hfill Απ: $ d_{min}(-3,2) = \sqrt{26} $ 

  % \item Δίνεται τρίγωνο ΑΒΓ. Να βρεθεί σημείο P, στο επίπεδο του τριγώνου, ώστε 
  %   το άθροισμα των τετραγώνων των αποστάσεών του από τις κορυφές του τριγώνου 
  %   να είναι ελάχιστο.
\end{enumerate}


\section*{Ακρότατα Υπό Συνθήκη}

\subsection*{Ακρότατα με ένα περιορισμό}


\begin{enumerate}
    %Thomas 12th 14.8 ex.1 
  \item Να βρείτε τα ακρότατα της συνάρτησης $ f(x,y) = xy $ πάνω στην έλλειψη 
    $ x^{2}+2y^{2}=1 $.

    \hfill Απ: 
    \begin{tabular}{l}
      max: $ f(\sqrt{2} /2, 1/2) = f(- \sqrt{2} /2, -1/2) = \frac{\sqrt{2}}{2} $ \\
      min $ f(\sqrt{2} /2, -1/2) = f(- \sqrt{2} /2, 1/2) = -\frac{\sqrt{2}}{2} $ \\
    \end{tabular}

    %Thomas 12th 14.8 ex.14 
  \item Να βρείτε τα ακρότατα της συνάρτησης $ f(x,y) = 3x-y+6 $ υπό τον περιορισμό 
    $ x^{2}+y^{2}=4 $.

    \hfill Απ:  
    \begin{tabular}{l}
      max: $ f(\frac{6}{\sqrt{10}} , - \frac{2}{\sqrt{10}}) = 2 \sqrt{10} +6 $ \\
      min $ f(-\frac{6}{\sqrt{10}} , + \frac{2}{\sqrt{10}}) = -2 \sqrt{10} +6 $ \\
    \end{tabular}

    %Thomas 12th 14.8 ex.15 
  \item Η θερμοκρασία σε κάθε σημείο μιας μεταλλικής πλάκας δίνεται από τη σχέση
    \[
      T(x,y) = 4x^{2}-4xy+y^2 
    \]
    Ένα μυρμήγκι, που βρίσκεται πάνω στην πλάκα, περπατά στην περιφέρεια κύκλου, 
    με κέντρο την αρχή των αξόνων και ακτίνας 5. Να υπολογίσετε την μέγιστη και την 
    ελάχιστη θερμοκρασία που θα συναντήσει το μυρμήγκι κατά τη διαδρομή του.

    \hfill Απ:  
    \begin{tabular}{l}
      max $ f(2 \sqrt{5} , - \sqrt{5}) = f(-2 \sqrt{5} , \sqrt{5}) = 125 $ \\
      min $ f(\sqrt{5} , 2 \sqrt{5}) = f(- \sqrt{5} , - 2 \sqrt{5}) = 0 $ 
    \end{tabular}
\end{enumerate}


\subsection*{Ακρότατα με δύο περιορισμούς}

\begin{enumerate}
  \item Να μεγιστοποιήσετε τη συνάρτηση $ f(x,y,z) = x^{2}+2y-z^{2} $ υπο τους 
    περιορισμούς $ 2x-y=0 $ και $ y+z=0 $. 
    \hfill Απ: $ f_{max}(2/3,4/3,-4/3) = 4/3 $

  \item Να βρεθεί το σημείο της ευθείας τομής των επιπέδων $ y+2z=12 $ και $ x+y=6 $,
    που βρίσκεται πιο κοντά στην αρχή των αξόνων. 
    \hfill Απ: $(2,4,4)$ 
\end{enumerate}


\section*{Απόλυτα Ακρότατα (Σε κλειστή και φραγμένη περιοχή)}

\begin{enumerate}
  %Thomas 12th 14.7 ex.41 
  \item Μια λεπτή, επίπεδη, μεταλλική πλάκα, σε σχήμα κυκλικού δίσκου 
    $ x^{2}+y^2 \leq 1 $, θερμαίνεται, έτσι ώστε η θερμοκρασία σε κάθε σημείο της να 
    δίνεται από τη συνάρτηση 
    $ T(x,y) = 2y^{2}+x^{2}-x $.
    Να βρείτε τη θερμοκρασία στα πιο θερμά και πιο ψυχρά σημεία αυτής της πλάκας.

    \hfill Απ:  
    \begin{tabular}{l}
      max $ T(-1/2, \sqrt{3} /2) = T(-1/2, - \sqrt{3} /2) = 9/4 $ \\
      min $ f(1/2,0) = -1/4 $ 
    \end{tabular}

    %Thomas 12th 14.7 ex.32 
  \item Να υπολογιστούν τα ακρότατα της συνάρτησης $ f(x,y) = x^{2}-xy+y^{2}+1 $ 
    στην κλειστή, τριγωνική περιοχή του επιπέδου, που περικλείεται από τις καμπύλες 
    $ x=0 $, $ y=4 $ και $ y=x $.

    \hfill Απ:  
    \begin{tabular}{l}
      max $ f(0,4) = f(4,4) = 17 $ \\
      min $ f(0,0) = 1 $ 
    \end{tabular}

    %Thomas 12th 14.7 ex.31 
  \item Να υπολογιστούν τα ακρότατα της συνάρτησης $ f(x,y) = 2x^{2}-4x+y^{2}-4y+1 $ 
    στην κλειστή, τριγωνική περιοχή του επιπέδου, που περικλείεται από τις καμπύλες 
    $ x=0 $, $ y=2 $ και $ y=2x $.

    \hfill Απ:  
    \begin{tabular}{l}
      max $ f(0,0) = 1 $ \\
      min $ f(1,2) = -5 $ 
    \end{tabular}

    %Thomas 12th 14.7 ex.31 
  \item Να υπολογιστούν τα ακρότατα της συνάρτησης $ f(x,y) = x^{2}-2xy+2y $ 
    στην κλειστή, ορθογώνια περιοχή $ D = \{ (x,y) \in \mathbb{R}^{2} \; : \; 0 \leq x
    \leq 3, \; 0 \leq y \leq 2 \} $.

    \hfill Απ:  
    \begin{tabular}{l}
      max $ f(3,0) = 9 $ \\
      min $ f(0,0) = f(2.2) = 0 $ 
    \end{tabular}
\end{enumerate}




\end{document}
