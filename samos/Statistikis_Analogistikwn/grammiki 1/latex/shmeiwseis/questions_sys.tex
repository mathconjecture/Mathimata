\input{preamble_ask.tex}
\input{definitions_ask.tex}
\input{tikz}

\pagestyle{askhseis}


\begin{document}

\begin{center}
  \minibox{\large\bfseries \textcolor{Col2}{Συστήματα Γραμμικών Εξισώσεων}}
\end{center}

% \section*{Ακολουθίες}

\vspace{\baselineskip}

\hfill \textcolor{Col1}{\textbf{Σωστό}} \quad \textcolor{Col1}{\textbf{Λάθος}}
\begin{enumerate}[itemsep=.5\baselineskip]
  \item \textcolor{Col1}{Κάθε γραμμικό σύστημα $n$ εξισώσεων με $n$ αγνώστους έχει 
    \textbf{μοναδική} λύση}. 
    \hfill $\textcolor{Col1}{\square} \qquad \qquad \textcolor{Col1}{\blacksquare}$

    Απ: Λάθος. Αν $ |A| \neq 0 $, τότε έχει άπειρες λύσεις ή αδύνατο (Crammer).

  \item \textcolor{Col1}{Κάθε γραμμικό σύστημα $n$ εξισώσεων με $n$ αγνώστους, 
    είναι \textbf{συμβιβαστό}}.
    \hfill $\textcolor{Col1}{\square} \qquad \qquad \textcolor{Col1}{\blacksquare}$

    Λάθος. Για παράδειγμα το σύστημα $ \left.
      \begin{matrix}
        x+y=1 \\
        x+y=2
      \end{matrix} 
    \right\}$ είναι αδύνατο.

  \item \textcolor{Col1}{Ένα γραμμικό σύστημα $m$ εξισώσεων με $n$ αγνώστους, όπου 
    $ m>n $, μπορεί να έχει \\ \textbf{άπειρο} πλήθος λύσεων}.
    \hfill $\textcolor{Col1}{\blacksquare} \qquad \qquad \textcolor{Col1}{\square}$

    Απ: Σωστό. Για παράδειγμα το σύστημα $ \left.
      \begin{matrix}
        x+y=1 \\
        2x+2y=2 \\
        3x+3y=3
      \end{matrix} 
    \right\} $ έχει άπειρο πλήθος λύσεων.

  \item \textcolor{Col1}{Ένα γραμμικό σύστημα $m$ εξισώσεων με $n$ αγνώστους, όπου 
    $ m<n $, μπορεί να είναι \textbf{αδύνατο}}.
    \hfill $\textcolor{Col1}{\blacksquare} \qquad \qquad \textcolor{Col1}{\square}$

    Απ: Σωστό. Για παράδειγμα το σύστημα $ \left.
      \begin{matrix}
        x+y+z=0 \\
        x+y+z=1
      \end{matrix} 
    \right\} $ είναι αδύνατο.

  \item \textcolor{Col1}{Ένα γραμμικό σύστημα με πίνακα συντελεστών $ A $ έχει 
      άπειρο πλήθος λύσεων αν και μόνον αν \\ ο $A$ είναι γραμμοϊσοδύναμος με κλιμακωτό 
    πίνακα που περιέχει στήλη χωρίς ηγετική μονάδα}.
    \hfill $\textcolor{Col1}{\square} \qquad \qquad \textcolor{Col1}{\blacksquare}$

    Απ: Λάθος. Το σύστημα $ \left.
      \begin{matrix}
        x+y=0 \\
        x+y=1
      \end{matrix} 
    \right\}$ είναι αδύνατο αλλά η 2η στήλη του $R_{A}$ δεν περιέχει ηγετικό.

  \item \textcolor{Col1}{Αν ο $A$ είναι αντιστρέψιμος, τότε ο $ A+A^{T} $ είναι
    αντιστρέψιμος}.
    \hfill $\textcolor{Col1}{\square} \qquad \qquad \textcolor{Col1}{\blacksquare}$

    Απ: Λάθος. Είναι δυνατόν $ A + A^{T} = O \Leftrightarrow A^{T}=-A $. π.χ. $ A = 
    \begin{pmatrix*}[r]
      0 & 1 \\
      -1 & 0
    \end{pmatrix*} $

  \item \textcolor{Col1}{Αν ο $A$ και $ B $ αντιστρέψιμοι πίνακες, τότε ο $ A+Β $ 
      είναι αντιστρέψιμος}.
    \hfill $\textcolor{Col1}{\square} \qquad \qquad \textcolor{Col1}{\blacksquare}$

    Απ: Λάθος. π.χ. $ A=I_{2} $ και $ B=-I_{2} $ αντιστρέψιμοι, όμως $ A+B=O_{2} $ 
    μη αντιστρέψιμος.

  \item \textcolor{Col1}{Αν $ A+B $ αντιστρέψιμος, τότε $ A $ και $ B $ είναι 
    αντιστρέψιμοι}.
    \hfill $\textcolor{Col1}{\square} \qquad \qquad \textcolor{Col1}{\blacksquare}$

    Απ: Λάθος. π.χ. $ A= 
    \begin{pmatrix}
      1 & 0 \\
      0 & 0
    \end{pmatrix} $ και $ B = 
    \begin{pmatrix}
      0 & 0 \\
      0 & 1
    \end{pmatrix} $ μη αντιστρέψιμοι, όμως $ A+B=I_{2} $ αντιστρέψιμος.
\end{enumerate}



\end{document}

