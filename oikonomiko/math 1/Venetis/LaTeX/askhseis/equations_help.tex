\input{preamble_ask.tex}
\input{definitions_ask.tex}

% \everymath{\displaystyle}
\pagestyle{vangelis}

\begin{document}

\begin{center}
  \minibox{\large\bfseries \textcolor{Col1}{Ευθείες}}
\end{center}

\vspace{\baselineskip}

\begin{enumerate}
  \item  Δίνεται η ευθεία ε:  $ 3x - 4y = 12 $.
    \begin{enumerate}[i)]
      \item Να εξετάσετε αν η ευθεία διέρχεται από τα σημεία $\left(2,- \frac{3}{2}
        \right)$, $ (2,-2) $ \hfill Απ:  ναι, όχι 
      \item  Να υπολογίσετε τα σημεία τομής της ευθείας $ \varepsilon $  
        με τους άξονες  $ xx' $  και  $ yy' $ \hfill Απ: $ xx': 4, yy':-3 $ 
      \item Να γραφεί η $ \varepsilon $ στη μορφή $ y=ax+b $ και να 
        προσδιορίσετε την κλίση της. 
        \hfill Απ: $ \lambda_{\varepsilon} = \frac{3}{4} $ 
    \end{enumerate}

  % \item  Δίνονται οι ευθείες $ \varepsilon_1: y = 3x-6 $, 
  %   $ \varepsilon_2: y = -x+2 $  και $ \varepsilon_3:   y = -4x+8 $
  %   \begin{enumerate}[i)]
  %     \item  Να βρείτε το σημείο τομής των ευθειών.  \hfill Απ: $ (2,0) $
  %     \item  Να δείξετε ότι οι τρεις ευθείες διέρχονται από το ίδιο σημείο.
  %   \end{enumerate}

  \item  Να βρείτε την εξίσωση της ευθείας  $ \varepsilon $, η οποία: 
    \begin{enumerate}[i)]
      \item διέρχεται από το σημείο  $ A(2,-1) $ και έχει συντελεστή διεύθυνσης 
        $ -3 $.  \hfill Απ: $ y = -3x+5 $ 
      \item διέρχεται από το σημείο  $ A(1,2) $ και έχει συντελεστή διεύθυνσης 
        $ \frac{1}{3} $.  \hfill Απ: $ y = \frac{1}{3} x + \frac{5}{3} $ 
      \item διέρχεται από τα σημεία  $ A(1,1) $  και  $ B(3,5) $. 
    \hfill Απ: $ y=2x-1 $ 
      \item διέρχεται από τα σημεία  $ A(-1,2) $  και  $ B(-3,2) $. 
    \hfill Απ: $ y=2 $ 
    \end{enumerate}

  \item  Να βρείτε τα  σημεία τομής των ευθειών:
    \begin{enumerate}[i)]
      \item $ \varepsilon_{1}: x + 2y = 5 $   και  $\varepsilon_{2}: 4x + y = 6 $.
        \hfill Απ: $ (1,2) $ 
      \item $ \varepsilon_{1}: 4x+3y=11 $ και  $ \varepsilon_{2}: 5x+7y=17 $ 
        \hfill Απ: $ (2,1) $ 
    \end{enumerate}

  \item Να βρείτε τα σημεία ισορροπίας των παρακάτω υποδειγμάτων αγοράς για ένα προϊόν.
    \begin{enumerate}[i),itemsep=\baselineskip]
      \item $ 
        \left.
          \begin{matrix}
            Q_{d} = 30-2P \\
            Q_{s} = -6+5P
          \end{matrix} 
        \right\}$ 
        \hfill Απ: $ P^{*} = \frac{36}{7} \approx 5.14, \; Q^{*}= \frac{138}{7} 
        \approx 19.71 $

      \item $ 
        \left.
          \begin{matrix}
            Q_{d} = 51 - 3P \\
            Q_{s} = 6P - 10
          \end{matrix} 
        \right\}$ 
        \hfill Απ: $ P^{*} = \frac{25}{11} \approx 2.27, \; Q^{*}= \frac{156}{11} \approx
        14.18 $

    \end{enumerate}
\end{enumerate}


\begin{center}
  \minibox{\large\bfseries  \textcolor{Col1}{Εξισώσεις}}
\end{center}

\subsection*{1ου βαθμού}

\begin{enumerate}
  %papadakis p180
  \item  Να λυθούν οι παρακάτω εξισώσεις.
    \begin{enumerate}[i)]
      \item $ 6x-2=9x-5 $ \hfill Απ: $ x=1 $
      \item $ 3-2(x-1)=x-4 $ \hfill Απ: $ x=3 $ 
      \item $ x-3(2-x)=-6 $ \hfill Απ: $ x=0 $ 
      \item $ 3(x+4)-(x-2) = 2+x $ \hfill Απ: $ x=-12 $ 
    \end{enumerate}

  \item  Να λυθούν οι παρακάτω εξισώσεις.
    \begin{enumerate}[i)]
      \item $ \frac{x-2}{3} = \frac{9-x}{4} $ \hfill Απ: $ x=5 $ 
      \item $ 3 - \frac{x+2}{4} = x+10 $ \hfill Απ: $ x=-6 $ 
      \item $ \frac{x-2}{4} - \frac{x-3}{2} = \frac{2-x}{3} $ \hfill Απ: $ x=-4 $ 
      \item $ \frac{1}{3} - \frac{2(x-1)}{9} = \frac{x+5}{18} $ \hfill Απ: $ x=1 $ 
    \end{enumerate}

  \item  Να λυθούν οι παρακάτω εξισώσεις.
    \begin{enumerate}
      \item $ \frac{4x-3}{3} - \frac{3x-1}{4} = 3 - \frac{7x-9}{6} $ 
        \hfill Απ: $ x=3 $ 
      \item $ \frac{x-2}{3} = \frac{x+1}{2} - \frac{3x-1}{6} $ \hfill Απ: $ x=4 $ 
      \item $ 2x- \frac{x+1}{2} - \frac{x-1}{4} = 2 - \frac{x+7}{8} $ 
        \hfill Απ: $ x=1 $
      \item $ 1 - \frac{2x-5}{9} = -3(x+1) - \frac{x+8}{6} $ \hfill Απ: $ x=-2 $  
    \end{enumerate}

  \item  Να λυθούν οι παρακάτω εξισώσεις (αδύνατες, αόριστες).
    \begin{enumerate}[i)]
      \item $ 3(2x+1) - 14 = 2(3x+5) $ \hfill Απ: αδύνατη 
      \item $ 10x-2(4+5x) =-8 $ \hfill Απ: ταυτότητα 
      \item $ 9x-3(3x-5)=15 $ \hfill Απ: ταυτότητα 
      \item $ 10x-2(4+5x) = 8 $ \hfill Απ: αδύνατη 
    \end{enumerate}

  \item  Να λυθούν οι παρακάτω εξισώσεις (αδύνατες, αόριστες).
    \begin{enumerate}[i)]
      \item $ \frac{x+4}{2} - \frac{5-x}{3} = x - \frac{x-2}{6} $ \hfill Απ: ταυτότητα
      \item $ \frac{3x-2}{4} - \frac{x+1}{2} = 3 - \frac{3-x}{4} $ \hfill Απ: αδύνατη 
      \item $ \frac{6-x}{3} - \frac{x-1}{6} = \frac{2}{3} - \frac{x-3}{2} $ 
        \hfill Απ: ταυτότητα
      \item $ \frac{x+1}{2} - \frac{2x-5}{3} = 1 - \frac{x-1}{6} $ \hfill Απ: αδύνατη 
    \end{enumerate}
\end{enumerate}

\subsection*{2ου βαθμού}

\begin{enumerate}
  \item  Να λυθούν οι παρακάτω εξισώσεις.
    \begin{enumerate}[i)]
      \item $ 2x^{2} + 5x - 3 = 0 $ \hfill  Απ: $ \frac{1}{2}, -3 $ 
      \item $ x^{2} + 5x + 4 = 0 $ \hfill Απ:  $ -4, -1 $
      \item $ x^{2} - 10x + 25 = 0 $ \hfill Απ: $ 5 $
      \item $ x^{2} - 7x + 6 = 0 $ \hfill Απ: $ 6, 1 $
      \item $ x^{2} - 4x + 5 = 0 $ \hfill Απ:  αδύνατη
      \item $ -x^{2} - 8x + 9 = 0 $ \hfill Απ: $ -9, 1 $
      \item $ -x^{2} + x + 42 = 0 $ \hfill Απ: $ \frac{1}{3}, -1 $  
      \item $ 4x^{2} + 20x + 25 = 0 $ \hfill Απ: $ - \frac{5}{2} $ 
    \end{enumerate}

  \item Να λυθούν οι παρακάτω εξισώσεις.
    \begin{enumerate}[i)]
      \item $ (x^{2} - 4)(x^{2} - 5x + 6) = 0 $ \hfill Απ: $ -2, 3, 2 $ 
      \item $ (x^{2} + 5x)(3x^{2} - 2x - 8) = 0 $ \hfill Απ: $ 0, -5, 2, -\frac{4}{3} $
      \item $ (x^{2} - 7x + 6)(4x^{2}- 8x + 3) = 0 $ \hfill Απ: $ 6, 1, \frac{1}{2},
        \frac{3}{2} $
      \item $ (9x^{2} - 6x + 1)(x^{2} - x - 2) = 0 $ \hfill Απ: $ \frac{1}{3}, 2, -1  $ 

    \end{enumerate}

  \item Να λυθούν οι παρακάτω εξισώσεις.
    \begin{enumerate}[i)]
      \item $ 3x^{2} - 2x + 1 = 2 $ \hfill Απ: $ 1, - \frac{1}{3} $  
      \item $ 3x^{2} + 2x - 4 = 3x - 7 $ \hfill Απ:  αδύνατη
      \item $ 4x^{2} - 6x + 5 = 4 - 5x^{2} $ \hfill Απ: $ \frac{1}{3} $
      \item $ 5x^{2} - 3x = 2x + x^{2} - 1 $ \hfill Απ: $ \frac{1}{4}, 1 $ 
    \end{enumerate}

  \item Να λυθούν οι παρακάτω εξισώσεις.
    \begin{enumerate}[i)]
      \item $ (x-1)(x-2) = x(x-4)+(x+2)^{2} $ \hfill Απ: $ -1, -2 $
      \item $ (2x-1)^{2} - (x-3)^{2} = (x+1)(2x-1) - 1 $ \hfill Απ: $ 2, -3 $
      \item $ (2-x)^{2} - (x-1)(x+1) = (x+3)^{2} - 6x $ \hfill Απ: $ -2 $
      \item $ 2(x-2)(x+3) + (-x-1)^{2} = (x-1)^{2} - (x-2)^{2} $ \hfill Απ: $ -2,
        \frac{4}{3} $ 
    \end{enumerate}

  \item Να λυθούν οι παρακάτω εξισώσεις.
    \begin{enumerate}[i)]
      \item $ \frac{x^{2}}{6} - \frac{2x}{3} = \frac{3x-10}{4} $ \hfill Απ: $
        \frac{5}{2}, 6 $
      \item $ \frac{x-2}{2} = \frac{(x-1)(x+1)}{3} - \frac{2x+1}{6} $ \hfill Απ: $
        \frac{3}{2}, 1 $ 
      \item $ \left(\frac{x-1}{2}\right)^{2} - \frac{2x}{5} = 
        \frac{2x+3}{2} - \frac{(5x-1)(2x+5)}{20} $ \hfill Απ: $ 2, -1 $ 
    \end{enumerate}

  \item Να λυθούν οι παρακάτω εξισώσεις.
    \begin{enumerate}[i)]
      \item $ 1 - \frac{3}{x} = \frac{10}{x^{2}} $ \hfill Απ: $ -2, 5 $
      \item $ 2 - \frac{x+4}{x} = \frac{21}{x^{2}} $ \hfill Απ: $ 7, -3 $
      \item $ \frac{7}{6x} = \frac{1}{2x^{2}} - 1 $ \hfill Απ: $ \frac{1}{3},
        -\frac{3}{2} $ 
      \item $ 1 - \frac{2}{3x} = \frac{2}{x} - \frac{2-x}{x^{2}} $ \hfill Απ: $ 3,
        \frac{2}{3} $  
      \item $ \frac{240}{x+2} = \frac{240}{x} - 10 $ \hfill Απ: $ 6, -8 $
    \end{enumerate}
\end{enumerate}


\end{document}

