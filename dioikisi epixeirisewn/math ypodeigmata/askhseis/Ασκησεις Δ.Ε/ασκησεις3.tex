\documentclass[a4paper,12pt]{article}

\usepackage[english,greek]{babel}
\usepackage[utf8]{inputenc}


\usepackage{amsmath}
\usepackage{amssymb}
\usepackage{amsfonts}
\usepackage{amsthm}
\usepackage{physics}
\usepackage{graphicx}

\usepackage[top=2cm,bottom=2cm,left=2cm,right=2cm]{geometry}







\begin{document}


\thispagestyle{empty}


\begin{center}
\begin{tabular}{@{}c@{}}
\includegraphics{logo.jpg}
\end{tabular}
\end{center}

\vspace{5\baselineskip}

\begin{center}
{\LARGE\textbf{Δυναμικά Μαθηματικά Υποδείγματα}}\\[5\baselineskip]



{\large\textbf{ Εργασίες - Ασκήσεις προς Επίλυση \\[5pt]Κεφάλαια 15-16: Διαφορικές Εξισώσεις}}
\end{center}

\vspace{15\baselineskip}

\begin{center}
Καθηγητής: Γεώργιος Ανδρουλάκης




\vspace{\baselineskip}
\noindent

Φοιτήτρια: Ματίνα Συλλιγάρδου\\
Α.Μ.: 2793\\
Έτος: $4$ο\\
Εξάμηνο: $8$ο
\end{center}



\vfill





\section{Ασκηση}

{\bfseries Να λυθούν οι ακόλουθες γραμμικές διαφορικές εξισώσεις πρώτου βαθμού με σταθερούς συντελεστές και δεδομένο σταθερό όρο.}

\vspace{2\baselineskip}

\begin{description}

\item [$\alpha$)] $\boxed{\frac{dy}{dx} + y = 4$, όπου $y(0) = 0}$

\textbf{Λύση}:
\vspace{\baselineskip}

Έχω $a=1\neq 0, b=4$, επομένως:

\begin{minipage}{0.4\textwidth}
\begin{align*}
y(t) &= y_c + y_p \\
&=Ae^{-at} + \frac{b}{a} \\
&= Ae^{-t} + 4 \quad\text{γενική λύση}
\end{align*}
\end{minipage}\hfill\begin{minipage}{0.4\textwidth}
\begin{align*}
y(0) &= 0 \\
Α+4 &= 0 \\
A=&-4
\end{align*}
\end{minipage}

Άρα
\[
\underline{y(t) = -4e^{-t} + 4 \quad \text{η ορισμένη λύση.}}
\]

\vspace{\baselineskip}

\item [$\beta$)] $\boxed{\frac{dy}{dx} = 23$, όπου $y(0) = 1}$

\textbf{Λύση}:

\vspace{\baselineskip}

Έχω $a=0, b=23$, επομένως:

\begin{minipage}{0.4\textwidth}
\begin{align*}
y(t) &= y_c + y_p \\
&=A + bt \\
&= A + 23t \quad\text{γενική λύση}
\end{align*}
\end{minipage}\hfill\begin{minipage}{0.4\textwidth}
\begin{align*}
y(0) &= 1 \\
Α &= 1 \\
\end{align*}
\end{minipage}

Άρα
\[
\underline{y(t) = 1 + 23t \quad \text{η ορισμένη λύση.}}
\]

\vspace{\baselineskip}

\item [$\gamma$)] $\boxed{\frac{dy}{dx} -5 y = 0$, όπου $y(0) = 6}$

\textbf{Λύση}:

\vspace{\baselineskip}

Έχω $a=-5\neq 0, b=0 \quad(\text{ομογενής}),$ επομένως:

\begin{minipage}{0.4\textwidth}
\begin{align*}
y(t) &= Ae^{-at} \\
&=Ae^{5t} \quad\text{γενική λύση}
\end{align*}
\end{minipage}\hfill\begin{minipage}{0.4\textwidth}
\begin{align*}
y(0) &= 6 \\
Α &= 6 \\
\end{align*}
\end{minipage}

Άρα
\[
\underline{y(t) = 6e^{5t} \quad \text{η ορισμένη λύση.}}
\]

\vspace{\baselineskip}

\item [$\delta$)] $\boxed{\frac{dy}{dx} +3 y = 2$, όπου $y(0) = 4}$

\textbf{Λύση}:

\vspace{\baselineskip}

Έχω $a=3\neq 0, b=2,$ επομένως:

\begin{minipage}{0.4\textwidth}
\begin{align*}
y(t) &= y_c + y_p \\
&=Ae^{-at} + \frac{b}{a} \\
&= Ae^{-3t} + \frac{2}{3} \quad\text{γενική λύση}
\end{align*}
\end{minipage}\hfill\begin{minipage}{0.4\textwidth}
\begin{align*}
y(0) &= 4 \\
Α+\frac{2}{3} &= 4 \\
A&=\frac{10}{3}
\end{align*}
\end{minipage}

Άρα
\begin{align*}
y(t) &= \frac{10}{3}e^{-3t} + \frac{2}{3} \\
&= \underline{\frac{2}{3} (5e^{-3t} + 1)\quad \text{η ορισμένη λύση.}}
\end{align*}
\end{description}

\vspace{2\baselineskip}

\section{Άσκηση}{\bfseries Δείξτε ότι η εξίσωση $\frac{dP}{dt} + j(\beta+\delta)P = j(\alpha+\gamma)$ μπορεί να γραφεί και ως $\frac{dP}{dt} + k(P-P^*) = 0$.
Εαν θεωρήσουμε ότι $P-P^* \equiv\Delta$ (αξιοσημείωτη απόκλιση) ώστε
$\frac{d\Delta}{dt} = \frac{dP}{dt}$ τότε η διαφορική εξίσωση μπορεί να γραφεί ως
\[
\frac{d\Delta}{dt} + k\Delta =0.
\]}

\vspace{2\baselineskip}

\textbf{Λύση}:

\vspace{\baselineskip}

Ξέρουμε ότι η τιμή ισορροπίας στο υπόδειγμα αγοράς δίνεται από $P^*=\frac{\alpha+\gamma}{\beta+\delta}$ με $\alpha, \beta, \gamma, \delta >0, \;\text{άρα} \; \neq 0$ και επίσης για τον συντελεστή προσαρμογής $j$ ισχύει επίσης ότι $j>0$.
Άρα έχουμε ότι $j(\beta+\delta)\neq 0$. Επομένως, διαιρώντας:
\begin{align*}
\frac{dP}{dt} + j(\beta+\delta) &= j(\alpha + \gamma) \\
\frac{\frac{dP}{dt}}{j(\beta+\delta)} + P &= \frac{\alpha+\gamma}{\beta+\delta} \\
\frac{\frac{dP}{dt}}{j(\beta+\delta)} + P &= P^* \\
\frac{\frac{dP}{dt}}{j(\beta+\delta)} + P - P^* &= 0 \\
\frac{dP}{dt} + j(\beta+\delta)(P-P^*) &=0 \\
\end{align*}
και αν θέσουμε $k\equiv j(\beta+\delta)$ προκύπτει η ζητούμενη σχέση:
\[
\frac{dP}{dt} + k(P-P^*) = 0.
\]
Εαν θεωρήσουμε ότι $P-P^*\equiv \Delta$ ώστε $\frac{d\Delta}{dt} = \frac{dP}{dt}$ τότε η παραπάνω εξίσωση που μόλις αποδείξαμε γίνεται:
\[
\frac{d\Delta}{dt} + k\Delta = 0.
\]

\vspace{2\baselineskip}

\section{Άσκηση}{\bfseries Να επιλυθούν οι ακόλουθες διαφορικές εξισώσεις πρώτης τάξης.}

\vspace{2\baselineskip}

\begin{description}

\item [($1$)] $\boxed{\frac{dy}{dt} + 5y = 15}$

\textbf{Λύση}:

\vspace{\baselineskip}

Έχω $a=5\neq 0, b=15$, επομένως:

\begin{minipage}{0.4\textwidth}
\begin{align*}
y(t) &= y_c + y_p \\
&=Ae^{-at} + \frac{b}{a} \\
&= \underline{Ae^{-5t} + 3 \quad\text{γενική λύση}}
\end{align*}
\end{minipage}

\vspace{\baselineskip}


\item [($2$)] $\boxed{\frac{dy}{dt} + 2ty = 0}$

\textbf{Λύση}:

\vspace{\baselineskip}

Έχω $u(t)=2t, w(t)=0 \quad(\text{ομογενής}),$ επομένως:

\[\int u(t)dt = \int 2tdt = 2\int tdt = 2\frac{t^2}{2} = t^2.\]

\begin{minipage}{0.4\textwidth}
Άρα
\begin{align*}
y(t) &= Ae^{-\int u(t)dt} \\
&= \underline{Ae^{-t^2} \quad\text{γενική λύση}}
\end{align*}
\end{minipage}

\vspace{\baselineskip}


\item [($3$)] $\boxed{\frac{dy}{dt} + 2ty = t}$

\textbf{Λύση}:

\vspace{\baselineskip}

Έχω $u(t)=2t, w(t)=t$, επομένως:

\[
\int u(t)dt = \int 2tdt = 2\int tdt = 2\frac{t^2}{2} = t^2.
\]

\[
\int w(t)e^{\int u(t)dt}dt = \int te^{t^2}dt \underset{\substack{dx=2tdt\\tdt=\frac{dx}{2}}}{\overset{x=t^2}{=}} \frac{1}{2}\int e^xdx = \frac{1}{2} e^{t^2}.
\]
\begin{minipage}{0.4\textwidth}
Άρα
\begin{align*}
y(t) &= e^{-\int u(t)dt}\left(A + \int w(t)e^{\int u(t)dt}dt\right) \\
&= e^{-t^2}(A+\frac{1}{2}e^{t^2}) \\
&= Ae^{-t^2}+\frac{1}{2} \quad\text{γενική λύση}.
\end{align*}
\end{minipage}\hfill\begin{minipage}{0.4\textwidth}
\begin{align*}
y(0) &= \frac{3}{2} \\
Α+ \frac{1}{2}&= \frac{3}{2} \\
A&=1
\end{align*}
\end{minipage}

Τελικά
\[
\underline{y(t) = e^{-t^2} + \frac{1}{2} \quad \text{η ορισμένη λύση.}}
\]

\vspace{\baselineskip}


\item [($4$)] $\boxed{\frac{dy}{dt} + t^2y = 5t^2}$

\textbf{Λύση}:

\vspace{\baselineskip}

Έχω $u(t)=t^2, w(t)=5t^2$, επομένως:

\[
\int u(t)dt = \int t^2dt = \frac{t^3}{3}.
\]

\[
\int w(t)e^{\int u(t)dt}dt = \int 5t^2e^{\frac{t^3}{3}}dt = 5\int t^2e^{\frac{t^3}{3}}dt = \underset{dx=t^2dt}{\overset{x=\frac{t^3}{3}}{=}} 5\int e^xdx = 5 e^{\frac{t^3}{3}}.
\]
\begin{minipage}{0.4\textwidth}
Άρα
\begin{align*}
y(t) &= e^{-\int u(t)dt}\left(A + \int w(t)e^{\int u(t)dt}dt\right) \\
&= e^{-\frac{t^3}{3}}(A+5e^{\frac{t^3}{3}}) \\
&= Ae^{-\frac{t^3}{3}}+5 \quad\text{γενική λύση}.
\end{align*}
\end{minipage}\hfill\begin{minipage}{0.4\textwidth}
\begin{align*}
y(0) &= 6 \\
Α+5 &= 0 \\
A&=1
\end{align*}
\end{minipage}

Τελικά
\[
\underline{y(t) = e^{-\frac{t^3}{3}}+5 \quad \text{η ορισμένη λύση.}}
\]


\end{description}

\vspace{2\baselineskip}

\section{Άσκηση} {\bfseries Είναι οι ακόλουθες διαφορικές εξισώσεις ακριβείς? Εάν δεν είναι είναι ακριβείς πειραματιστείτε με τα $t,y$ και $y^2$ ως πιθανούς παράγοντες ολοκλήρωσης.}

\vspace{2\baselineskip}

\begin{description}

\item [$(\alpha)$] $\boxed{2(t^3+1)dy + 3yt^2dt = 0}$

\textbf{Λύση}:

\vspace{\baselineskip}

Έχουμε:
$M=2(t^3+1)$ και $N=3yt^2$.
\[
\pdv{M}{t} = 6t^2 \neq
\pdv{N}{y}= 3t^2
\]
Άρα η δ.ε. δεν είναι ακριβής. Γι' αυτό πολλαπλασιάζουμε με $y$ και έχουμε:
\[
2y(t^3+1)dy + 3y^2t^2dt = 0
\]
η οποία είναι ακριβής, γιατί:
$M'=2y(t^3+1)$ και $N'=3y^2t^2$.
\[
\pdv{M'}{t}= 6yt^2 = \pdv{N'}{y}=6yt^2
\]

\vspace{\baselineskip}

\item [$(\beta)$] $\boxed{4y^3tdy + 2(y^4+3t)dt = 0}$

\textbf{Λύση}:

\vspace{\baselineskip}

Έχουμε:
$M=4y^3t$ και $N=2y^4+6t$.
\[
\pdv{M}{t} = 4y^3 \neq
\pdv{N}{y}=8y^3
\]
Άρα η δ.ε. δεν είναι ακριβής. Γι' αυτό πολλαπλασιάζουμε με $t$ και έχουμε:
\[
4y^3t^2dy + (2y^4t+6t^2)dt = 0
\]
η οποία είναι ακριβής, γιατί:
$M'=4y^3t^2$ και $N'=2y^4t+6t^2$.
\[
\pdv{M'}{t}= 8y^3t = \pdv{N'}{y}=8y^3t
\]
\end{description}

\vspace{2\baselineskip}

\section{Άσκηση}{\bfseries  Προσδιορίστε για κάθε διαφορική εξίσωση τα ακόλουθα:
\begin{description}
\item [($1$)]Εάν οι μεταβλητές είναι διαχωρίσιμες
\item [($2$)]Εάν η εξίσωση είναι γραμμική ή έαν είναι δυνατό να μετασχηματιστεί σε γραμμική.
\end{description}}

\vspace{2\baselineskip}



\begin{description}
\item [$(\alpha)$] $\boxed{2tdy + 2ydt=0}$

\textbf{Απάντηση:}

\vspace{\baselineskip}

Οι μεταβλητές είναι διαχωρίσιμες, γιατί η δ.ε. αν διαιρέσουμε με $2y$ και στη συνέχεια με $t$ γράφεται ως:

\[
\frac{dy}{y}+\frac{dt}{t}=0
\]
Είναι επίσης και γραμμική, γιατί:

\[
\frac{dy}{y}+\frac{dt}{t}=0 \Rightarrow
\frac{dy}{dt}+\frac{y}{t}=0 \Rightarrow
\frac{dy}{dt}+\frac{1}{t}y=0
\]
η οποία είναι ομογενής με $u(t)=\frac{1}{t}$.

\vspace{\baselineskip}

\item [$(\beta)$] $\boxed{\frac{y}{y+t}dy + \frac{2t}{y+t}dt=0}$


\textbf{Απάντηση:}

\vspace{\baselineskip}

Την πολλαπλασιάζω με $y+t$ και έχω:
\[
ydy+2tdt=0
\]
η οποία είναι χωριζομένων μεταβλητών. Επίσης η δ.ε. γράφεται ως:
\[
\frac{dy}{dt}=-\frac{2t}{y} \Rightarrow \frac{dy}{dt}=-2ty^{-1}
\]
η οποία είναι μια δ.ε. \textlatin{Bernoulli} με $m=-1$ και η οποία μπορεί να αναχθεί σε γραμμική.

\vspace{\baselineskip}

\item [$(\gamma)$] $\boxed{\frac{dy}{dt}=\frac{-t}{y}}$

\textbf{Απάντηση:}

\vspace{\baselineskip}

\begin{align*}
\frac{dy}{dt}&=\frac{-t}{y} \\
ydy&=-tdt\\
ydy&+tdt=0
\end{align*}
η οποία είναι μια δ.ε. χωριζομένων μεταβλητών. Μπορεί επίσης να αναχθεί σε γραμμική, αφού γράφεται και ως:

\[
\frac{dy}{dt}=-ty^{-1}
\]
η οποία είναι μια δ.ε. \textlatin{Bernoulli}.

\vspace{\baselineskip}

\item [$(\delta)$] $\boxed{\frac{dy}{dt}=3y^2t}$

\textbf{Απάντηση:}

\vspace{\baselineskip}

Έχουμε:
\begin{align*}
\frac{dy}{dt} &=3y^2t \\
y'&=3y^2t \\
\frac{y'}{y^2} &= 3t
\end{align*}
Θέτουμε $v=-\frac{1}{y}\Rightarrow v'=\frac{y'}{y^2}$ και με αντικατάσταση βρίσκουμε ότι:
\begin{align*}
-v'&=3t \\
v'&=-3t
\end{align*}
η οποία είναι γραμμική $1$ης τάξης ως προς $v$ με $u(t)=0$.




\end{description}

\vspace{2\baselineskip}

\section{Άσκηση} {\bfseries Βρείτε τη γενική λύση για τις ακόλουθες διαφορικές εξισώσεις και στη συνέχεια προσδιορίστε την λύση τους για τις αρχικές συνθήκες $y(0)=4$ και $y'(0)=2$. }

\vspace{2\baselineskip}

\begin{description}

\item [$(\alpha)$] $\boxed{y''(t)+3y'(t) -4y=12}$

\textbf{Λύση}:

\vspace{\baselineskip}

$a_1=3, a_2=-4\neq 0, b=12$.

$y_p=\frac{b}{a_2}=\frac{12}{4}=-3$

Για τη συμπληρωματική συνάρτηση, λύνω την αντίστοιχη ομογενή

$y''(t) + 3y'(t) -4y = 0$

Θεωρώ το χαρακτηριστικό πολυώνυμο:
\begin{align*}
r^2+3r-4&=0 \Leftrightarrow \\
(r-1)(r+4)&=0
\end{align*}

Άρα $y_c=Ae^{r_1t} + Be^{r_2t} = Ae^t + Be^{-4t}$

Δηλαδή,

\begin{align*}
y(t) &= Ae^t + Be^{-4t} -3 \quad \text{γενική λύση} \\
y'(t) &= Ae^t-5Be^{-4t}
\end{align*}

\[
  \left.\begin{aligned}
y(0)&=4 \\
y'(0)&=2\\
  \end{aligned}\:\right\}\Leftrightarrow \quad
 \left.\begin{aligned}
A+B-3=4 \\
A-4B=2\\
  \end{aligned}\:\right\}\Leftrightarrow \quad
 \left.\begin{aligned}
A+B&=7 \\
A-4B&=2
  \end{aligned}\:\right\}\Leftrightarrow \quad
 \left.\begin{aligned}
-A-B&=-7 \\
A-4B&=2
  \end{aligned}\:\right\}\Rightarrow
\]
$-5Β=-5$.

Άρα $B=1$ και $A=6$

Τελικά: $\underline{y(t) = 6e^t+e^{-4t} -3 \quad \text{ορισμένη λύση}}$.

\vspace{\baselineskip}

\item [$(\beta)$] $\boxed{y''(t)+6y'(t) +5y=10}$

\textbf{Λύση}:

\vspace{\baselineskip}

$a_1=6, a_2=5\neq 0, b=10$.

$y_p=\frac{b}{a_2}=\frac{10}{5}=2$

Για τη συμπληρωματική συνάρτηση, λύνω την αντίστοιχη ομογενή

$y''(t)+6y'(t) +5y = 0$

Θεωρώ το χαρακτηριστικό πολυώνυμο:
\begin{align*}
r^2+6r+5&=0 \Leftrightarrow \\
(r+1)(r+5)&=0
\end{align*}

Άρα $y_c=Ae^{-t} + Be^{-5t} $

Δηλαδή,

\begin{align*}
y(t) &= Ae^{-t} + Be^{-5t} +2 \quad \text{γενική λύση} \\
y'(t) &= -Ae^{-t}-5Be^{-5t}
\end{align*}

\[
  \left.\begin{aligned}
y(0)&=4 \\
y'(0)&=2\\
  \end{aligned}\:\right\}\Leftrightarrow \quad
 \left.\begin{aligned}
A=B+2&=4 \\
-A-5B&=2\\
  \end{aligned}\:\right\}\Leftrightarrow \quad
 \left.\begin{aligned}
A+B&=2\\
-A-5B&=2
  \end{aligned}\:\right\}\Leftrightarrow \quad
 \left.\begin{aligned}
A&=3 \\
B&=-1
  \end{aligned}\:\right\}
\]


Τελικά: $\underline{y(t) = 3e^{-t}-e^{-4t} +2 \quad \text{ορισμένη λύση}}$.


\vspace{\baselineskip}

\item [$(\gamma)$] $\boxed{y''(t)-2y'(t) +y=3}$

\textbf{Λύση}:

\vspace{\baselineskip}

$a_1=-2, a_2=1\neq 0, b=3$.

$y_p=\frac{b}{a_2}=\frac{3}{1}=3$

Για τη συμπληρωματική συνάρτηση, λύνω την αντίστοιχη ομογενή

$y''(t)+-2y'(t) +y = 0$

Θεωρώ το χαρακτηριστικό πολυώνυμο:
\begin{align*}
r^2-2r+1&=0 \Leftrightarrow \\
(r-1)^2&=0 \quad \text{δηλαδή $r=1$ είναι διπλή ρίζα}.
\end{align*}

Άρα $y_c=Ae^{t} + Bτe^{t} $

Δηλαδή,

\begin{align*}
y(t) &= Ae^{t} + Bte^{t} +3 \quad \text{γενική λύση} \\
y'(t) &= Ae^{t} + Be^{t} + Bte^t
\end{align*}

\[
  \left.\begin{aligned}
y(0)&=4 \\
y'(0)&=2\\
  \end{aligned}\:\right\}\Leftrightarrow \quad
 \left.\begin{aligned}
A+3&=4\\
A+B&=2\\
  \end{aligned}\:\right\}\Leftrightarrow \quad
 \left.\begin{aligned}
A&=1\\
B&=1
  \end{aligned}\:\right\}
\]


Τελικά: $\underline{y(t) = e^{t}+te^{t} +3 = e^t(1+t) +3 \quad \text{ορισμένη λύση}}$.


\vspace{\baselineskip}

\item [$(\delta)$] $\boxed{y''(t)+8y'(t) +16y=0}$

\textbf{Λύση}:

\vspace{\baselineskip}

$a_1=8, a_2=16\neq 0, b=0$.

$y_p=\frac{b}{a_2}=0$

Για τη συμπληρωματική συνάρτηση, λύνω την αντίστοιχη ομογενή

$y''(t)+8y'(t) + 16 = 0$

Θεωρώ το χαρακτηριστικό πολυώνυμο:
\begin{align*}
r^2+8r+16&=0 \Leftrightarrow \\
(r+4)^2&=0 \quad \text{δηλαδή $r=4$ είναι διπλή ρίζα}.
\end{align*}

Άρα $y_c=Ae^{4t} + Bte^{4t} $

Δηλαδή,

\begin{align*}
y(t) &= Ae^{4t} + Bte^{4t} \quad \text{γενική λύση} \\
y'(t) &= 4Ae^{4t} + Be^{4t} + 4B4te^{4t}
\end{align*}

\[
  \left.\begin{aligned}
y(0)&=4 \\
y'(0)&=2\\
  \end{aligned}\:\right\}\Leftrightarrow \quad
 \left.\begin{aligned}
A&=4\\
4A+B&=2\\
  \end{aligned}\:\right\}\Leftrightarrow \quad
 \left.\begin{aligned}
A&=4\\
B&=-14
  \end{aligned}\:\right\}
\]


Τελικά: $\underline{y(t) = 4e^{4t}-14te^{4t} \quad \text{ορισμένη λύση}}$.
\end{description}


\vspace{2\baselineskip}

\section{Άσκηση} {\bfseries Βρείτε τα $y_p$ και $y_c$ καθώς και τη γενική και μερική λύση για τις ακόλουθες εξισώσεις.}

\vspace{2\baselineskip}

\begin{description}

\item [$(1)$] $\boxed{y''(t)-4y'(t) +8y=0}$

\textbf{Λύση}:

\vspace{\baselineskip}

$a_1=-4, a_2=8\neq 0, b=0$.

$y_p=\frac{b}{a_2}=0$

Για τη συμπληρωματική συνάρτηση, λύνω την αντίστοιχη ομογενή

$y''(t)-4y'(t) + 8 = 0$

Θεωρώ το χαρακτηριστικό πολυώνυμο:
\begin{align*}
r^2-4r+8&=0 \Leftrightarrow \\
r_1=2+2i&, r_2 = 2-2i
\end{align*}

Άρα $y_c=y_c=e^{2t}(A\cos 2t +B\sin 2t)$

Δηλαδή,

\begin{align*}
y(t) &= e^{2t}(A\cos 2t +B\sin 2t) \quad \text{γενική λύση} \\
y'(t) &= 2e^{2t}(A\cos 2t +B\sin 2t)+e^{2t}(-4\sin 2t + 4B\cos 2t)
\end{align*}

\[
  \left.\begin{aligned}
y(0)&=3 \\
y'(0)&=7\\
  \end{aligned}\:\right\}\Leftrightarrow \quad
 \left.\begin{aligned}
A&=3\\
2A+4B&=7\\
  \end{aligned}\:\right\}\Leftrightarrow \quad
 \left.\begin{aligned}
A&=3\\
B&=\frac{1}{4}
  \end{aligned}\:\right\}
\]


Τελικά: $\underline{y(t) = e^{2t}(3\cos 2t +\frac{1}{4}\sin 2t) \quad \text{ορισμένη λύση}}$.

\vspace{\baselineskip}



\item [$(2)$] $\boxed{y''(t)+4y'(t) +8y=2}$

\textbf{Λύση}:

\vspace{\baselineskip}

$a_1=4, a_2=8\neq 0, b=2$.

$y_p=\frac{b}{a_2}=\frac{1}{4}$

Για τη συμπληρωματική συνάρτηση, λύνω την αντίστοιχη ομογενή

$y''(t) +4y'(t) + 8 = 0$

Θεωρώ το χαρακτηριστικό πολυώνυμο:
\begin{align*}
r^2+4r+8&=0 \Leftrightarrow \\
r_1=-2+2i&, r_2 = -2-2i
\end{align*}

Άρα $y_c=e^{-2t}(A\cos 2t +B\sin 2t)$

Δηλαδή,

\begin{align*}
y(t) &= e^{-2t}(A\cos 2t +B\sin 2t) + \frac{1}{4} \quad \text{γενική λύση} \\
y'(t) &= -2e^{-2t}(A\cos 2t +B\sin 2t)+e^{-2t}(-4\sin 2t + 4B\cos 2t)
\end{align*}

\[
  \left.\begin{aligned}
y(0)&=2\frac{1}{4} \\
y'(0)&=4\\
  \end{aligned}\:\right\}\Leftrightarrow \quad
 \left.\begin{aligned}
A+\frac{1}{4}&=2\frac{1}{4}\\
-2A+4B&=4\\
  \end{aligned}\:\right\}\Leftrightarrow \quad
 \left.\begin{aligned}
A&=2\\
B&=2
  \end{aligned}\:\right\}
\]


Τελικά: \begin{align*}y(t) &= e^{-2t}(2\cos 2t +2\sin 2t) + \frac{1}{4}\\ &= \underline{2e^{-2t}(\cos 2t +\sin 2t) + \frac{1}{4}\quad \text{ορισμένη λύση}}.\end{align*}



\vspace{\baselineskip}


\item [$(3)$] $\boxed{y''(t)+3y'(t) -4y=12}$

\textbf{Λύση}:

\vspace{\baselineskip}

$a_1=3, a_2=-4\neq 0, b=12$.

$y_p=\frac{b}{a_2}=\frac{12}{-4}=-3$

Για τη συμπληρωματική συνάρτηση, λύνω την αντίστοιχη ομογενή

$y''(t) +3y'(t) -4 = 0$

Θεωρώ το χαρακτηριστικό πολυώνυμο:
\begin{align*}
r^2+3r-4&=0 \Leftrightarrow \\
(r+4)(r-1) &= \Leftrightarrow \\
r_1=-4&, r_2 = 1
\end{align*}

Άρα $y_c=Ae^{t} +Be^{-4t}$

Δηλαδή,

\begin{align*}
y(t) &=Ae^{t} +Be^{-4t} -3 \quad \text{γενική λύση} \\
y'(t) &= Ae^{t} -4Be^{-4t}
\end{align*}

\[
  \left.\begin{aligned}
y(0)&=2 \\
y'(0)&=2\\
  \end{aligned}\:\right\}\Leftrightarrow \quad
 \left.\begin{aligned}
A+B-3&=2\\
A-4B&=2\\
  \end{aligned}\:\right\}\Leftrightarrow \quad
 \left.\begin{aligned}
A+B&=5\\
A-4B&=2
  \end{aligned}\:\right\}\Leftrightarrow \quad
 \left.\begin{aligned}
A&=\frac{22}{5} \\
B&=\frac{3}{5}
  \end{aligned}\:\right\}
\]


Τελικά: \begin{align*}y(t) &= \underline{\frac{1}{5}(22e^t -12e^{-4t})-3 \quad \text{ορισμένη λύση}}.\end{align*}

\vspace{\baselineskip}


\item [$(4)$] $\boxed{y''(t)+9y=3}$

\textbf{Λύση}:

\vspace{\baselineskip}

$a_1=0, a_2=9\neq 0, b=3$.

$y_p=\frac{b}{a_2}=\frac{1}{3}$

Για τη συμπληρωματική συνάρτηση, λύνω την αντίστοιχη ομογενή

$y''(t)+9y = 0$

Θεωρώ το χαρακτηριστικό πολυώνυμο:
\begin{align*}
r^2+9&=0 \Leftrightarrow \\
r^2&=-9 \Leftrightarrow \\
r&=\pm i\sqrt{9} \Leftrightarrow \\
r&=\pm 3i
\end{align*}

Άρα $y_c=e^{0}(A\cos 3t +B\sin 3t) = A\cos 3t +B\sin 3t$

Δηλαδή,

\begin{align*}
y(t) &= A\cos 3t +B\sin 3t + \frac{1}{3} \quad \text{γενική λύση} \\
y'(t) &= -3A\sin 3t +3B\cos 3t
\end{align*}

\[
  \left.\begin{aligned}
y(0)&=1 \\
y'(0)&=8\\
  \end{aligned}\:\right\}\Leftrightarrow \quad
 \left.\begin{aligned}
A+\frac{1}{3}&=1\frac{1}{4}\\
3B&=8\\
  \end{aligned}\:\right\}\Leftrightarrow \quad
 \left.\begin{aligned}
A&=\frac{2}{3}\\
B&=\frac{8}{3}
  \end{aligned}\:\right\}
\]


Τελικά: \begin{align*}y(t) &= \frac{2}{3}\cos 3t +\frac{8}{3}\sin 3t) + \frac{1}{3}\\ &= \underline{\frac{2}{3}(\cos 3t +4\sin 3t) + \frac{1}{3}\quad \text{ορισμένη λύση}}.\end{align*}
\end{description}

\vspace{2\baselineskip}

\section{Άσκηση} {\bfseries Έστω $m,n,u$ και $w$ διάφοροι του μηδενός.
\begin{align*}
Q_d&=\alpha -\beta P +mP' +nP'' \quad\text{όπου $\alpha,\beta >0$}\\
Q_s&=\gamma +\delta P +uP'+wP'' \quad\text{όπου $\gamma,\delta >0$}
\end{align*}
Θεωρώντας ότι η προσφορά είναι ίση με τη ζήτηση σε κάθε χρονική στιγμή να γραφεί η νεά διαφορική εξίσωση του μοντέλου.

Βρείτε την τιμή ισορροπίας που είναι ανεξάρτητη από το χρόνο.}

\textbf{Λύση}:

\vspace{\baselineskip}

Σε κατάσταση ισορροπίας έχω:
\[
Q_d=Q_s \Leftrightarrow
\]
\begin{align*}
\alpha -\beta P +mP' + nP'' =-\gamma + \delta P +uP'+wP'' &\Leftrightarrow\\
\alpha+\gamma-(\beta+\delta)P +(m-n)P''+(n-w)P'' = 0 &\Leftrightarrow \\
(n-w)P'' + (m-n)P'-(\beta+\delta)P =-(\alpha+\gamma)&
\end{align*}
Αν $n-w\neq0$ έχω:
\[
P''+\frac{m-n}{n-w}P'-\frac{\beta+\delta}{n-w}P = -\frac{\alpha+\gamma}{n-w}
\]
Άρα η νέα δ.ε. του μοντέλου είναι $2$ης τάξης με σταθερούς συντελεστές. Επειδή $a_2=-\frac{\beta+\delta}{n-w}\neq0$ με $\beta,\delta>0$ και $b=-\frac{\alpha+\gamma}{n-w}$ ο σταθερός όρος του β' μέλους, έχω:
\[
y_p=P_p=\frac{b}{a_2}=-\frac{\alpha+\gamma}{\beta+\delta}
\]

\vspace{2\baselineskip}

\section{Άσκηση}  {\bfseries Βρείτε την ειδική λύση των ΔΕ.}

\vspace{2\baselineskip}

\begin{description}
\item [($\alpha$)] $\boxed{y''(t)+2y'(t) +y=t}$

\textbf{Λύση:}

\vspace{\baselineskip}

Ζητάμε την ειδική λύση $y_p$ της ΔΕ.

Δοκιμάζουμε: $y(t)=a_0+a_1t$

Οπότε:
\begin{align*}
y'(t)&=a_1\\
y''(t)&=0
\end{align*}
Άρα η ΔΕ. με αντικατάτασταση των $y,y',y''$ γίνεται:
\[
0+2a_1+a_0+a_1t=t
\]
και από την ισότητα των πολυωνύμων, έχω:
\[
\left.\begin{aligned}
a_1&=1\\
2a_1+a_0&=0
  \end{aligned}\:\right\}\Leftrightarrow \quad
 \left.\begin{aligned}
a_1&=1\\
a_0&=-2
  \end{aligned}\:\right\}
\]
Άρα $\underline{y_p=-2+t = t-2}$.

\vspace{\baselineskip}

\item [($\beta$)] $\boxed{y''(t)+4y'(t) +y=2t^2}$

\textbf{Λύση:}

\vspace{\baselineskip}

Ζητάμε την ειδική λύση $y_p$ της ΔΕ.

Δοκιμάζουμε: $y(t)=a_0+a_1t+a_2t^2$

Οπότε:
\begin{align*}
y'(t)&=a_1+2a_2t\\
y''(t)&=2a_2
\end{align*}
Άρα η ΔΕ. με αντικατάτασταση των $y,y',y''$ γίνεται:
\begin{align*}
2a_2+4(a_1+2a_2t)+a_0+a_1t+a_2t^2&=2t^2 \Leftrightarrow\\
2a_2+4a_1 + 8a_2t+a_0+a_1t+a_2t^2&=2t^2 \Leftrightarrow\\
2a_2+4a_1+a_0+(8a_2+a_1)t+a_2t^2 &= 2t^2 \Leftrightarrow\\
\end{align*}
και από την ισότητα των πολυωνύμων, έχω:
\[
\left.\begin{aligned}
2a_2+4a_1+a_0&=0 \\
8a_2+a_1 &=0 \\
a_2&=2
  \end{aligned}\:\right\}\Leftrightarrow \quad
 \left.\begin{aligned}
a_0&=60\\
a_1&=-16 \\
a_2&=2
  \end{aligned}\:\right\}
\]
Άρα $\underline{y_p=60-16t+2t^2}$.

\vspace{\baselineskip}

\item [($\gamma$)] $\boxed{y''(t)+y'(t) +2y=e^t}$

\textbf{Λύση:}

\vspace{\baselineskip}

Ζητάμε την ειδική λύση $y_p$ της ΔΕ.

Δοκιμάζουμε: $y(t)=Ae^t$

Οπότε:
\begin{align*}
y'(t)&=Ae^t\\
y''(t)&=Ae^t
\end{align*}
Άρα η ΔΕ. με αντικατάτασταση των $y,y',y''$ γίνεται:
\begin{align*}
Ae^t+Ae^t+2Ae^t&=e^t \Leftrightarrow\\
4Ae^t&=e^t \Leftrightarrow\\
4A&=1 \Leftrightarrow\\
A&=\frac{1}{4}
\end{align*}

Άρα $y_p\underline{=\frac{1}{4}e^t}$.

\vspace{\baselineskip}

\item [($\delta$)] $\boxed{y''(t)+y'(t) +3y=\sin t}$

\textbf{Λύση:}

\vspace{\baselineskip}

Ζητάμε την ειδική λύση $y_p$ της ΔΕ.

Δοκιμάζουμε: $y(t)=A\cos t+B\sin t$

Οπότε:
\begin{align*}
y'(t)&=-A\sin t+B\cos t\\
y''(t)&=-A\cos t-B\sin t
\end{align*}
Άρα η ΔΕ. με αντικατάτασταση των $y,y',y''$ γίνεται:
\begin{align*}
-A\cos t-B\sin t -A\sin t+B\cos t+3A\cos t+3B\sin t &=\sin t \Leftrightarrow\\
(2A+B)\cos t +(2A-B)\sin t&=\sin t
\end{align*}
και εξισώνοντας, έχω:
\[
\left.\begin{aligned}
2Α+Β &=0\\
2Β-Α &=1 \\
  \end{aligned}\:\right\}\Leftrightarrow \quad
 \left.\begin{aligned}
Β&=-2Α\\
-4Α-Α&=1 \\
  \end{aligned}\:\right\}\Leftrightarrow \quad
  \left.\begin{aligned}
B&=\frac{2}{5}\\
A&=-\frac{1}{5} \\
  \end{aligned}\:\right\}\Leftrightarrow \quad
\]
Άρα $\underline{y_p=-\frac{1}{5}\cos t +\frac{2}{5}\sin t = \frac{1}{5}(2\sin t-\cos t)}$.
\end{description}

\vspace{2\baselineskip}

\section{Άσκηση}{\bfseries Βρείτε τη λύση των ΔΕ.}

\vspace{2\baselineskip}

\begin{description}

\item [$(a)$] $\boxed{y'''(t)+2y''(t)+y'(t)+2y = 8}$

\textbf{Λύση:}

\vspace{\baselineskip}

$a_1=2, a_2=1, a_3=2\neq 0, b=8$

Άρα $y_p=\frac{8}{2}=4$

Για το $y_c$ έχω:

\begin{align*}
r^3+2r^2+r+2&=0 \Leftrightarrow \\
r^2(r+2)+r+2 &=0 \Leftrightarrow\\
(r+2)(r^2+1)&=0 \Leftrightarrow\\
r=-2 &, r=\pm i
\end{align*}
Αρα $y_c = A_1e^{-2t}+A_2\cos t+A_3\sin t$

Τελικά $\underline{y(t) = A_1e^{-2t}+A_2\cos t+A_3\sin t +4}$.


\vspace{\baselineskip}

\item [$(b)$] $\boxed{y'''(t)+y''(t)+3y'(t) = 1}$

\textbf{Λύση:}

\vspace{\baselineskip}

$a_1=1, a_2=3\neq 0, a_3=0, b=1$

Άρα $y_p=\frac{b}{a_2}t = \frac{1}{3}t$

Για το $y_c$ έχω:

\begin{align*}
r^3+r^2+3r&=0 \Leftrightarrow \\
r(r^2+r+3r) &=0 \Leftrightarrow\\
(r+2)(r^2+1)&=0 \Leftrightarrow\\
r=-2 &, r=-\frac{1}{2}\pm \frac{\sqrt{11}}{2}i
\end{align*}
Αρα $y_c = A_1e^0+e^{-\frac{1}{2}}(A_2\cos\frac{\sqrt{11}}{2}t+A_3\sin\frac{\sqrt{11}}{2}t)$

Τελικά $\underline{y(t) = A_1+e^{-\frac{1}{2}}(A_2\cos\frac{\sqrt{11}}{2}t+A_3\sin\frac{\sqrt{11}}{2}t)+\frac{1}{3}t}$.

\vspace{\baselineskip}

\item [$(c)$] $\boxed{
3y'''(t)+9y''(t) = 1  \Leftrightarrow
y'''(t)+3y''(t) = \frac{1}{3}
}$

\textbf{Λύση:}

\vspace{\baselineskip}

$a_1=3\neq 0, a_2=0, a_3=0, b=\frac{1}{3}$

Δοκιμάζουμε $y_p=kt^2$, και έχω: $y_p'(t)=2kt$, $y_p''(t)=2k$ και $y_p'''(t)=0$.
Άρα η δ.ε. μας δίνει $3\cdot 2k = \frac{1}{3}$ και άρα $k=\frac{1}{18}.$

Επομένως $y_p(t)=\frac{1}{18}t^2$.

Για το $y_c$ έχω:

\begin{align*}
r^3+3r^2&=0 \Leftrightarrow \\
r^2(r+3) &=0 \Leftrightarrow\\
r^2=0 &, r=-3
\end{align*}
Αρα $y_c = A_1e^0+A_2te^0+A_3e^{-3t}=A_1+A_2t+A_3e^{-3t}$.


Τελικά $\underline{y(t) = A_1+A_2t+A_3e^{-3t} +\frac{1}{18}t^2}$.

\vspace{\baselineskip}

\item [$(d)$] $\boxed{3y^{(4)}(t)+y''(t) = 4}$

\textbf{Λύση:}

\vspace{\baselineskip}

$a_1=0, a_2=1\neq 0, a_3=0, a_4=0, b=4$

Δοκιμάζουμε $y_p=kt^2$ και έχουμε $y_p'(t)=2kt$, $y_p''(t)=2k$ και $y_p'''(t)=y_p^{(4)}(t)=0$.

Η δ.ε. μας δίνει $\frac{1}{3}2k=\frac{4}{3}$ και άρα $k=2$.

Άρα $y_p(t)=2t^2$.

Για το $y_c$ έχω:

\begin{align*}
r^4+r^2&=0 \Leftrightarrow \\
r^2(r^2+1) &=0 \Leftrightarrow\\
r=0 \:\text{(διπλή ρίζα)} &, r=\pm i
\end{align*}
Αρα $y_c = A_1e^0+A_2te^0+e^{0}(A_3\cos t+A_4\sin t)=A_1+A_2t+A_3\cos t+A_4\sin t$.


Τελικά $\underline{y(t) =A_1+A_2t+A_3\cos t+A_4\sin t +2t^2}$.

\end{description}






\end{document} 