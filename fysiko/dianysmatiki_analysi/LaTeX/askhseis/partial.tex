\documentclass[a4paper,12pt]{article}
\usepackage{etex}
%%%%%%%%%%%%%%%%%%%%%%%%%%%%%%%%%%%%%%
% Babel language package
\usepackage[english,greek]{babel}
% Inputenc font encoding
\usepackage[utf8]{inputenc}
%%%%%%%%%%%%%%%%%%%%%%%%%%%%%%%%%%%%%%

%%%%% math packages %%%%%%%%%%%%%%%%%%
\usepackage{amsmath}
\usepackage{amssymb}
\usepackage{amsfonts}
\usepackage{amsthm}
\usepackage{proof}

\usepackage{physics}

%%%%%%% symbols packages %%%%%%%%%%%%%%
\usepackage{bm} %for use \bm instead \boldsymbol in math mode 
\usepackage{dsfont}
\usepackage{stmaryrd}
%%%%%%%%%%%%%%%%%%%%%%%%%%%%%%%%%%%%%%%


%%%%%% graphicx %%%%%%%%%%%%%%%%%%%%%%%
\usepackage{graphicx}
\usepackage{color}
%\usepackage{xypic}
\usepackage[all]{xy}
\usepackage{calc}
\usepackage{booktabs}
\usepackage{minibox}
%%%%%%%%%%%%%%%%%%%%%%%%%%%%%%%%%%%%%%%

\usepackage{enumerate}

\usepackage{fancyhdr}
%%%%% header and footer rule %%%%%%%%%
\setlength{\headheight}{14pt}
\renewcommand{\headrulewidth}{0pt}
\renewcommand{\footrulewidth}{0pt}
\fancypagestyle{plain}{\fancyhf{}
\fancyhead{}
\lfoot{}
\rfoot{\small \thepage}}
\fancypagestyle{vangelis}{\fancyhf{}
\rhead{\small \leftmark}
\lhead{\small }
\lfoot{}
\rfoot{\small \thepage}}
%%%%%%%%%%%%%%%%%%%%%%%%%%%%%%%%%%%%%%%

\usepackage{hyperref}
\usepackage{url}
%%%%%%% hyperref settings %%%%%%%%%%%%
\hypersetup{pdfpagemode=UseOutlines,hidelinks,
bookmarksopen=true,
pdfdisplaydoctitle=true,
pdfstartview=Fit,
unicode=true,
pdfpagelayout=OneColumn,
}
%%%%%%%%%%%%%%%%%%%%%%%%%%%%%%%%%%%%%%

\usepackage[space]{grffile}

\usepackage{geometry}
\geometry{left=25.63mm,right=25.63mm,top=36.25mm,bottom=36.25mm,footskip=24.16mm,headsep=24.16mm}

%\usepackage[explicit]{titlesec}
%%%%%% titlesec settings %%%%%%%%%%%%%
%\titleformat{\chapter}[block]{\LARGE\sc\bfseries}{\thechapter.}{1ex}{#1}
%\titlespacing*{\chapter}{0cm}{0cm}{36pt}[0ex]
%\titleformat{\section}[block]{\Large\bfseries}{\thesection.}{1ex}{#1}
%\titlespacing*{\section}{0cm}{34.56pt}{17.28pt}[0ex]
%\titleformat{\subsection}[block]{\large\bfseries{\thesubsection.}{1ex}{#1}
%\titlespacing*{\subsection}{0pt}{28.80pt}{14.40pt}[0ex]
%%%%%%%%%%%%%%%%%%%%%%%%%%%%%%%%%%%%%%

%%%%%%%%% My Theorems %%%%%%%%%%%%%%%%%%
\newtheorem{thm}{Θεώρημα}[section]
\newtheorem{cor}[thm]{Πόρισμα}
\newtheorem{lem}[thm]{λήμμα}
\theoremstyle{definition}
\newtheorem{dfn}{Ορισμός}[section]
\newtheorem{dfns}[dfn]{Ορισμοί}
\theoremstyle{remark}
\newtheorem{remark}{Παρατήρηση}[section]
\newtheorem{remarks}[remark]{Παρατηρήσεις}
%%%%%%%%%%%%%%%%%%%%%%%%%%%%%%%%%%%%%%%




\input{definitions_ask.tex}

\pagestyle{askhseis}

\renewcommand{\vec}{\mathbf}

\begin{document}

\begin{center}
  \minibox{\large \bfseries \textcolor{Col1}{Μερική Παράγωγος - Διαφορικό}}
\end{center}

\vspace{\baselineskip}

\section*{Κανόνες Παραγώγισης}

\begin{enumerate}
  % \item Έστω η συνάρτηση $ f(x,y) = x^{y} $. Να δείξετε ότι ισχύει ότι το
  %   \textbf{θεώρημα Schwartz}, δηλαδή ότι $ f_{xy} = f_{yx} $.
    %span
  \item Να δείξετε ότι η συνάρτηση $ f(x,y) = \cos{(x+y)} + \cos{(x-y)} $ 
    επαληθεύει την διαφορική εξίσωση $ f_{xx} - f_{yy} = 0 $.

  \item Να δείξετε ότι η συνάρτηση $ f(x,y) = x \arctan{\frac{y}{x}} $ 
    ικανοποιεί την διαφορική εξίσωση $ x^{2} f_{xx} + 2xyf_{xy} + y^{2} f_{yy} = 0 $. 

  \item Να δείξετε ότι η συνάρτηση $ \psi(x,y,z,t) = \sin{(x+y+z- c \sqrt{3}t)} $ 
    ικανοποιεί την εξίσωση κύματος $ c^{2}(\psi _{xx}+ \psi _{yy} + \psi _{zz}) = \psi
    _{tt} $.

  % \item Να δείξετε ότι η συνάρτηση $ z = \mathrm{e}^{y} \phi (y
  %   \mathrm{e}^{\frac{x^{2}}{2y^{2}}}) $ ικανοποιεί τη διαφορική εξίσωση 
  %   $ (x^{2}-y^{2}) \pdv{z}{x} + xy \pdv{z}{y} = xyz $.
\end{enumerate}


\section*{Διαφορικό}

\begin{enumerate}

  \item Να βρεθεί το ολικό διαφορικό 1ης τάξης, της συνάρτησης 
    $f(x,y)=\ln(xy)+\cos(y^2)$ 

    \hfill Απ: $df=\frac{dx}{x}+\left(\frac{1}{y}-2y\sin(y^2)\right)dy$

  \item Να βρεθεί το ολικό διαφορικό 1ης τάξης, της συνάρτησης 
    $ f(x,y) = \arctan(\frac{ x+y }{ x-y }) $, αν $ x>0 $ και $ y>0 $.

    \hfill Απ: $df = \frac{ -ydx + xdy }{ x^{2} + y^{2} } $ 

    % \item Να βρεθεί το ολικό διαφορικό της συνάρτησης $ f(x,y) = x^{y} \cdot y^{x} $, 
    %   αν $ x>0$ και $ y>0 $.

    %   \hfill Απ: $df =  (x^{y-1}\cdot y^{x+1} + x^{y}\cdot y^{x} \ln{y} )dx 
    %   + (x^{y}\cdot y^{x} \ln{x} + x^{y+1} \cdot y^{x-1})dy $ 

  \item Για τις παρακάτω παραστάσεις, να αποδείξετε ότι είναι \textbf{τέλεια
    διαφορικά} και να υπολογίσετε τη συνάρτηση δυναμικού.
    \begin{enumerate}[i)]
      \item $ \left(x+e^{x/y}\right)dx + e^{x/y}\left(1- \frac{x}{y}\right)dy $
        \hfill Απ: $ f(x,y) = \frac{x^{2}}{2} +y e^{x/y} + c $ 

      \item $\left(2e^{x}+\frac{1}{x}-3\sin y\right)dx+3(y^2-x\cos y)dy$ 
        \hfill  Απ: $ f(x,y,z) = y^{3}-3x \sin{y} + 2e^{x} + \ln{x} +c $.
        %spand (114)

      \item $(2xy+z)dx+(x^{2}+z^{2})dy+(x+2yz)dz$ 
        \hfill  Απ: $ f(x,y,z) = x^{2}+z^{2}+xz +c $.

        % \item $ (3x^{2}+3y-1)dx + (z^{2}+3x)dy+(2yz+1)dz $
        % \hfill Απ: $ f(x,y,z) = x^{3}+3xy-x+yz^{2}+z+c $

      \item $ \cos(x+yz)dx + z\cos(x+yz)dy+y\cos(x+yz)dz $
        \hfill Απ: $ f(x,y,z) = \sin(x+yz) + c $
    \end{enumerate}

  \item Να υπολογιστεί το $a$ ώστε η παράσταση $ \frac{ x + ay }{ (x-y)^{3} }dx 
    + \frac{ ax+y }{ (x-y)^{3} }dy $ να είναι \textbf{τέλειο διαφορικό}.
    \hfill Απ: $ a=-1 $

  \item \textbf{(παλιό θέμα)} Μια συνάρτηση $ f(x,y) $ στο σημείο $ (2,5) $ παίρνει την 
    τιμή $ f(2,5) = 7 $ και οι μερικές παράγωγοι τις τιμές $ f_{x}(2,5) = 3 $ και 
    $ f_{y}(2,5) = 1 $. Να υπολογιστεί κατά προσέγγιση η τιμή της στα σημεία 
    $ (2.2,5.1) $ και $ (1.9,5.2) $.

  \item Να υπολογιστεί κατά προσέγγιση, με τη βοήθεια του διαφορικού, η τιμή των 
   παρακάτω  παραστάσεων.
    \begin{enumerate}[i)]
      \item $A = (1,02)^{3,01} $ \hfill Απ: $ A \approx 1,06 $ 
      \item $B =  \sqrt{ 9(1,95)^{2} + (8,1)^{2} } $ 
        \hfill Απ: $ B \approx 9,99 $ 
    \end{enumerate}
\end{enumerate}


\section*{Αρμονικές Συναρτήσεις}

\begin{enumerate}

  \item Να δείξετε ότι οι παρακάτω συναρτήσεις είναι αρμονικές:
    \begin{enumerate}[(i)]
      \item $ f(x,y) = x^{3}-3xy^{2} $
      \item $ f(x,y) = \ln(x^{2} + y^{2}) $
      % \item $ f(x,y) = \ln{(x^{2}+y^{2})} + \arctan(\frac{y}{x}) $
      % \item $ f(x,y) = \frac{ x }{ 2 } \ln(x^{2} + y^{2}) - y 
      %   \arctan(\frac{ y }{ x } ) $
    \end{enumerate}
\end{enumerate}

\section*{\textcolor{Col1}{Υποδείξεις}}

\begin{myitemize}
  \item Μια συνάρτηση $ f(x,y) $ είναι \textbf{αρμονική} αν ικανοποιεί την εξίσωση 
    Laplace, δηλαδή αν $ \Delta f = 0 \Leftrightarrow f_{xx}+f_{yy} = 0 $.
  \item Συνάρτηση Δυναμικού: 
    $ f(x,y) = \int_{x_{0}}^{x} P(t,y) \,{dt} + \int_{y_{0}}^{y} Q(x_{0},t) \,{dt} $ 
  \item Συνάρτηση Δυναμικού: $ f(x,y,z) = \int _{x_{0}}^{x} P(t,y,z) \,{dt} + 
    \int _{y_{0}}^{y} Q(x_{0},t,z) \,{dt} + \int _{z_{0}}^{z} R(x_{0}, y_{0},t) \,{dt} $
\end{myitemize}

\end{document}
