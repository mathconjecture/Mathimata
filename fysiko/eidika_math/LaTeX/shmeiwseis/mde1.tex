\input{preamble_ask.tex}
\newcommand{\vect}[2]{(#1_1,\ldots, #1_#2)}
%%%%%%% nesting newcommands $$$$$$$$$$$$$$$$$$$
\newcommand{\function}[1]{\newcommand{\nvec}[2]{#1(##1_1,\ldots, ##1_##2)}}

\newcommand{\linode}[2]{#1_n(x)#2^{(n)}+#1_{n-1}(x)#2^{(n-1)}+\cdots +#1_0(x)#2=g(x)}

\newcommand{\vecoffun}[3]{#1_0(#2),\ldots ,#1_#3(#2)}

\newcommand{\mysum}[1]{\sum_{n=#1}^{\infty}

\input{tikz.tex}
\input{myboxes.tex}

%todo να συνεχισω... να τα γραψω καλυτερα (δες Χατζηκωνσταντινου, οχι Λουκοπουλο)

\pagestyle{vangelis}
% \everymath{\displaystyle}
\setcounter{chapter}{1}

\input{insbox}



\begin{document}

\chapter*{Σημειώσεις Θεωρίας}

\section{Μερικές Διαφορικές Εξισώσεις}

\begin{mybox1}
  \begin{dfn}
    Ονομάζουμε \textcolor{Col1}{μερική διαφορική εξίσωση}, για συντομία μδε, κάθε 
    εξίσωση που περιέχει μια άγνωστη συνάρτηση $ u=u(\mathbf{x}) $, της διανυσματικής 
    μεταβλητής 
    $ \mathbf{x} = (x_{1}, x_{2}, \ldots, x_{n}) \in \Omega \subseteq \mathbb{R}^{n} $ 
    με $ n \geq 2 $, και πεπερασμένο πλήθος από μερικές παραγώγους αυτής, δηλαδή 
    \begin{equation}
      \label{eq:pde}
      F\left(\mathbf{x}, u, \pdv{u}{x_{1}} , \pdv{u}{x_{2}}, \ldots, 
        \pdv{u}{x_{n}}, \ldots, \frac{\partial^{(m)}{u}}{\partial x_{1}^{\lambda_ 1} 
      \partial x_{2}^{\lambda_ 2}\ldots \partial x_{n}^{\lambda _{n}}}\right) = 0
    \end{equation} 
    όπου $ \lambda _1+ \lambda _2 + \cdots \lambda _n = m $, με $ \lambda _{1}, \lambda
    _{2} \ldots, \lambda _{n}, m =1,2,\ldots$
  \end{dfn}
\end{mybox1}

\begin{rem}
  Οι μερικές διαφορικές εξισώσεις, με δύο ανεξάρτητες μεταβλητές $ x $ και $ y $, έχουν 
  τη γενική μορφή
  \[
    F\left(x,y,u, \pdv{u}{x} , \pdv{u}{y} , \ldots, 
      \frac{\partial ^{(m)}u}{\partial x^{\lambda _1} \partial
    y^{\lambda _2}} \right) = 0, \; \text{όπου} \; \lambda _{1}+ \lambda _{2}=m 
  \]
\end{rem}

\begin{mybox1}
  \begin{dfn}
    Η \textcolor{Col1}{τάξη} της μερικής διαφορικής εξίσωσης καθορίζεται 
    από την \textbf{μεγαλύτερη} τάξη των μερικών παραγώγων που εμφανίζονται στην εξίσωση.
  \end{dfn}
\end{mybox1}

\begin{rem}
  Συνδυάζοντας τους ορισμούς των μδε και της τάξης τους, συμπεραίνουμε, ότι:
  \begin{myitemize}
    \item Μια μερική διαφορική εξίσωση 1ης τάξης, δύο μεταβλητών, έχει τη γενική μορφή
      \[
        F(x,y,u,u_{x},u_{y}) = 0 \; \text{όπου} \; u=u(x,y)
      \] 
    \item Μια μερική διαφορική εξίσωση 2ης τάξης, δύο μεταβλητών, έχει τη γενική μορφή
      \[
        F(x,y,u,u_{x},u_{y},u_{xx},u_{xy},u_{yy}) = 0 \; \text{όπου} \; u=u(x,y)
      \] 
  \end{myitemize}
\end{rem}

\begin{mybox1}
  \begin{dfn}
    Ονομάζεται \textcolor{Col1}{βαθμός} μιας μερικής διαφορικής εξίσωσης, ο
    \textbf{εκθέτης} στον οποίο είναι υψωμένη η μεγαλύτερης τάξης μερική παράγωγος που 
    περιέχεται στην εξίσωση.
  \end{dfn}
\end{mybox1}

\begin{examples}
\item {}
  \begin{enumerate}
    \item Η μδε $ uu_{x}+2xyu_{y} = 0 $ είναι 1ης τάξης, 1ου βαθμού.
    \item Η μδε $ u_{xx}+xu_{y} = 0 $ είναι 2ης τάξης, 1ου βαθμού.
    \item Η μδε $ (\pdv{u}{x})^{3} + (\pdv[2]{u}{x}{y})^{2} + xy-u=0 $ είναι 2ης τάξης,
      2ου βαθμού.
    \item Η μδε$ \frac{\partial^{3}u}{\partial x^{2}\partial y} + y\pdv[2]{u}{y}=0 $ 
      είναι 3ης τάξης, 1ου βαθμού.
  \end{enumerate}
\end{examples}


\subsection*{Γραμμικές Μερικές Διαφορικές Εξισώσεις}

\begin{mybox1}
  \begin{dfn}
    Μια μερική διαφορική εξίσωση της μορφής~\eqref{eq:pde} ονομάζεται
    \textcolor{Col1}{γραμμική}, όταν η συνάρτηση $F$ είναι γραμμική ως προς τη μεταβλητή 
    $u$ και ως προς όλες τις μερικές παραγώγους της $u$ που εμφανίζονται στην εξίσωση.
  \end{dfn}
\end{mybox1}

\begin{rem}
  Γενικά, σε μια γραμμική διαφορική εξίσωση, δεν εμφανίζονται καθόλου δυνάμεις, οι άλλες 
  συναρτήσεις της άγνωστης συνάρτησης $ u $ και των μερικών παραγώγων της, αλλά ούτε και
  γινόμενα των μερικών παραγώγων μεταξύ τους. Αν μια μδε δεν είναι γραμμική, τότε 
  λέγεται \textcolor{Col1}{μη γραμμική}.
\end{rem}

\begin{rem} \item {}
  Συνδυάζοντας τους ορισμούς των γραμμικών μδε και της τάξης τους, συμπεραίνουμε, ότι:
  \begin{myitemize}
    \item Μια γραμμική μδε 1ης τάξης, δύο μεταβλητών, έχει τη μορφή
      \begin{equation} 
        a(x,y) u_{x} + b(x,y)u_{y} + c(x,y)u = d(x,y), \; \text{όπου} \; u=u(x,y) 
      \end{equation}
      Αν $ d(x,y)=0 $, τότε η εξίσωση λέγεται \textcolor{Col1}{ομογενής}.
    \item Μια γραμμική μδε 1ης τάξης, τριών μεταβλητών, έχει τη μορφή
      \begin{equation} 
        a(x,y,z) u_{x} + b(x,y,z)u_{y} + c(x,y,z)u_{z} + d(x,y,z)u = e(x,y,z), 
        \; \text{όπου} \; u=u(x,y,z)  
      \end{equation} 
      Αν $ e(x,y,z)=0 $, τότε η εξίσωση λέγεται \textcolor{Col1}{ομογενής}.
    \item Μια γραμμική μδε 2ης τάξης, δύο μεταβλητών, έχει τη μορφή
      \begin{equation} 
        a(x,y)u_{xx} + b(x,y)u_{xy} + c(x,y)u_{yy} + d(x,y)u_{x} + e(x,y)u_{y} +
        f(x,y)u =g(x,y), \; \text{όπου} \; u=u(x,y) 
      \end{equation} 
      Αν $ g(x,y)=0 $, τότε η εξίσωση λέγεται \textcolor{Col1}{ομογενής}.
  \end{myitemize}
\end{rem}

\begin{examples} \item {}
  \begin{enumerate}
    \item Η μδε $ yu_{x}-xu_{y}=xy $, όπου $ u=u(x,y) $  είναι γραμμική.
    \item Η μδε $ xu_{x}+yu_{y}+zu_{z}=2u $, όπου $ u=u(x,y,z) $  είναι γραμμική,
      ομογενής.
  \end{enumerate}
\end{examples}


\subsection*{Ημιγραμμικές Μερικές Διαφορικές Εξισώσεις 1ης τάξης}

\begin{mybox1}
  \begin{dfn}
    Μια μερική διαφορική εξίσωση 1ης τάξης, ονομάζεται \textcolor{Col1}{ημιγραμμική}, 
    όταν είναι γραμμική, μόνο ως προς τις μερικές παραγώγους μέγιστης τάξης που 
    εμφανίζονται στην εξίσωση.
  \end{dfn}
\end{mybox1}

\begin{rem}
  Μια ημιγραμμική μδε 1ης τάξης, δύο μεταβλητών, έχει τη μορφή
  \[ a(x,y,u)u_{x} + b(x,y,u)u_{y} = c(x,y,u), \; \text{όπου} \; u=u(x,y) \]
\end{rem}

\begin{examples} \item {}
  \begin{enumerate}
    \item Η μδε $uu_{x}+u_{y}=1$, όπου $ u=u(x,y) $  είναι ημιγραμμική.
    \item Η μδε $ (y-u)u_{x} + (u-x)u_{y} = x-y $, όπου $ u=u(x,y) $  είναι ημιγραμμική.
  \end{enumerate}
\end{examples}


\subsection*{Σχεδόν Γραμμικές Μερικές Διαφορικές Εξισώσεις 1ης τάξης}

\begin{mybox1}
  \begin{dfn}
    Μια μερική διαφορική εξίσωση 1ης τάξης, ονομάζεται 
    \textcolor{Col1}{σχεδόν γραμμική}, όταν οι συντελεστές των μερικών παραγώγων 
    μέγιστης τάξης, εξαρτώνται μόνο από τις ανεξάρτητες μεταβλητές.
  \end{dfn}
\end{mybox1}

\begin{rem}
  Μια σχεδόν γραμμική μδε 1ης τάξης, δύο μεταβλητών, έχει τη μορφή
  \[  a(x,y) u_{x} + b(x,y) u_{y} = c(x,y,u) \; \text{όπου} \; u=u(x,y) \]
\end{rem}

\begin{examples} \item {}
  \begin{enumerate}
    \item Η μδε $yu_{x}+xu_{y}=u^{2}$, όπου $ u=u(x,y) $  είναι σχεδόν γραμμική.
  \end{enumerate}
\end{examples}


\section*{Λύσεις μιας Μερικής Διαφορικής Εξίσωσης}

\begin{mybox1}
  \begin{dfn}
    Ονομάζουμε \textcolor{Col1}{λύση} μιας μερικής διαφορικής εξίσωσης (μδε), σε μιας 
    περιοχή $ \Omega $, οποιαδήποτε συνάρτηση που μαζί με τις μερικές παραγώγους της, 
    ικανοποιεί την εξίσωση, σε κάθε σημείο του $ \Omega $.
    Η λύση μιας μδε $n$ μεταβλητών, παριστάνει μια \textbf{υπερεπιφάνεια} στο χώρο 
    $ \mathbb{R}^{n+1} $.
  \end{dfn}
\end{mybox1}

\begin{rem}
  Μια μερική διαφορική εξίσωσης έχει άπειρες λύσεις.
\end{rem}

\begin{mybox1}
  \begin{dfn}
    Ονομάζουμε \textcolor{Col1}{γενική λύση (ή γενικό ολοκλήρωμα)} μιας μδε τάξης $m$, 
    μια λύση της, η οποία περιέχει $m$ το πλήθος αυθαίρετες συναρτήσεις, κάθε μία από 
    τις οποίες εξαρτάται από $ n-1 $ μεταβλητές.
  \end{dfn}
\end{mybox1}

\begin{mybox1}
  \begin{dfn}
    Κάθε λύση που προκύπτει από τη γενική λύση της μδε με συγκεκριμένη επιλογή των 
    αυθαίρετων συναρτήσεων ονομάζεται \textcolor{Col1}{μερική ή ειδική λύση (ή μερικό
    ολοκλήρωμα)}.
  \end{dfn}
\end{mybox1}

\begin{mybox1}
  \begin{dfn}
    Μια λύση ονομάζεται \textcolor{Col1}{ιδιάζουσα λύση (ή ιδιάζων ολοκλήρωμα)} μιας 
    μδε, αν δεν προκύπτει από τη γενική λύση της.
  \end{dfn}
\end{mybox1}

\begin{mybox1}
  \begin{dfn}
    Πλήρης λύση, μιας μδε, ονομάζεται το σύνολο των λύσεών της.
  \end{dfn}
\end{mybox1}

\begin{rem}
  Η πλήρης λύση μιας μδε, περιέχει τόσες αυθαίρετες 
  σταθερές όσες είναι οι ανεξάρτητες μεταβλητές της άγνωστης συνάρτησης.
\end{rem}


\section*{Μέθοδοι Επίλυσης μιας ΜΔΕ}

\subsection*{Γενικές Μέθοδοι}

Με τις μεθόδους αυτές επιδιώκεται η άμεση λύση μιας μδε, 
η οποία θα ικανοποιεί τις συνοριακές συνθήκες.

\subsection*{Χωρισμός Μεταβλητών}

Η άγνωστη συνάρτηση εκφράζεται ως γινόμενο συναρτήσεων, 
κάθε μία από τις οποίες εξαρτάται μόνο από μία μεταβλητή και η μδε ανάγεται σε 
σύστημα συνήθων διαφορικών εξισώσεων. Η μέθοδος αυτή εφαρμόζεται σε ομογενείς μδε.


\subsection*{Αλλαγή Μεταβλητών} 

Η μέθοδος αυτή μετασχηματίζει μια μδε, σε μια άλλη μδε, η 
οποία είναι ευκολότερη ή σε μια συνήθη διαφορική εξίσωση. Εφαρμόζεται για γραμμικές
αλλά και μη γραμμικές μδε.

\subsection*{Ολοκληρωτικοί Μετασχηματισμοί} 

Η μέθοδος μετασχηματίζει μια μδε με $n$
ανεξάρτητες μεταβλητές σε μια άλλη με $ (n-1) $ ανεξάρτητες μεταβλητές. Για 
παράδειγμα ο Μετασχηματισμός Fourier.

\subsection*{Ολοκληρωτικές Εξισώσεις} 

Μια μδε μετασχηματίζεται σε μια ολοκληρωτική εξίσωση, 
όπου η άγνωστη συνάρτηση εμφανίζεται μέσα σε ένα ολοκλήρωμα, το οποίο στη συνέχεια 
επιλύεται με διάφορες μεθόδους. Για παράδειγμα Η συνάρτηση Green.

\subsection*{Μέθοδος Ιδιοτιμών} 

Με τη μέθοδο αυτή μια μδε μετασχηματίζεται σε ένα πρόβλημα 
ιδιοτιμών και οι ζητούμενες λύσεις εκφράζονται ως άθροισμα των αντίστοιχων
ιδιοσυναρτήσεων.


\section*{Προβλήματα Συνοριακών Τιμών}

Σε κάθε φυσικό πρόβλημα, που μοντελοποιείται μέσω μιας μδε, συνήθως ζητείται η επίλυση 
της εξίσωσης σε ένα συγκεκριμένο χωρίο $ \Omega $, φραγμένο ή μη, στο σύνορο $ \partial
\Omega $ του οποίου καθορίζονται συγκεκριμένες συνθήκες που ισχύουν για την 
συνάρτηση $ u $. Ένα τέτοιο πρόβλημα, καλείται \textcolor{Col1}{Πρόβλημα Συνοριακών 
Τιμών} (ΠΣΤ).

\subsection*{Εξωτερικό Πρόβλημα}

Τα προβλήματα στα οποία ένα μέρος του συνόρου, βρίσκεται στο \textbf{άπειρο}, 
ονομάζονται \textcolor{Col1}{εξωτερικά} προβλήματα συνοριακών τιμών και οι λύσεις τους 
πρέπει να ικανοποιούν συνθήκες, οι οποίες δίνονται σε οριακή ή ασυμπτωτική μορφή.

\section*{Προβλήματα Αρχικών Τιμών}

Προβλήματα στα οποία επιβάλλονται συνοριακές συνθήκες σε ένα μόνο σημείο του 
διαστήματος ολοκλήρωσης μιας ανεξάρτητης μεταβλητής ονομάζονται προβλήματα αρχικών 
τιμών.

\section*{Συνοριακές Συνθήκες}

Οι συνοριακές συνθήκες, που συνοδεύουν ένα Πρόβλημα Συνοριακών Τιμών, ανήκουν σε έναν 
από τους παρακάτω τύπους, οι οποίοι για λόγους ευκολίας, παρουσιάζονται στο δισδιάστατο 
χώρο.

\subsection*{Τύπου Dirichlet}

Η λύση $u$ της μδε σε μια δισδιάστατη περιοχή $D$, απαιτείται να ικανοποιεί μια συνθήκη η
οποία καθορίζει την τιμή της συνάρτησης επί του συνόρου της περιοχής $D$, που συνήθως
είναι κάποια καμπύλη $C$
\[
  u= f(x,y), \quad \forall (x,y) \in C \; \text{με} \; C = \partial D
\]
όπου $ f(x,y) $ είναι γνωστή συνάρτηση.

\subsection*{Τύπου Newmann}

Η λύση $u$ της μδε σε μια δισδιάστατη περιοχή $D$, απαιτείται να ικανοποιεί μια συνθήκη η
οποία καθορίζει την τιμή παραγώγων της συνάρτησης επί του συνόρου της περιοχής $D$, που 
συνήθως είναι κάποια καμπύλη $C$
\[
  \pdv{u}{\mathbf{n}} = f(x,y), \quad \forall (x,y) \in C \; \text{με} \; C = \partial D
\] 
όπου $ f(x,y) $ είναι γνωστή συνάρτηση και $ \pdv{u}{\mathbf{n}} $ είναι η παράγωγος
κατά κατεύθυνση της συνάρτησης $ u $ προς τη διεύθυνση του διανύσματος $ \mathbf{n} $, 
που είναι κάθετο στη συνοριακή καμπύλη σε κάθε σημείο της και με προσανατολισμό, προς το
εξωτερικό της περιοχής $D$.

\subsection*{Τύπου Robin}

Η λύση $u$ της μδε σε μια δισδιάστατη περιοχή $D$, απαιτείται να ικανοποιεί δύο συνθήκες
της μορφής
\[
  u= f_{1}(x,y), \quad \forall (x,y) \in C_{1} \quad \text{και} \quad \pdv{u}{\mathbf{n}}
  = f_{2}(x,y), \; \quad \forall (x,y) \in C_{2} 
\]
όπου $ f_{1}(x,y) $ και $ f_{2}(x,y) $ είναι γνωστές συναρτήσεις και $ C_{1} $, $ C_{2}
$ είναι δυο διαφορετικά τμήματα τμήματα της συνοριακής καμπύλης $C$, τέτοια ώστε 
$ C=C_{1}+C_{2} $.

\subsection*{Μεικτού Τύπου}

Η λύση $u$ της μδε σε μια δισδιάστατη περιοχή $D$, απαιτείται να ικανοποιεί μια συνθήκη
της μορφής
\[
  \pdv{u}{\mathbf{n}} + h(x,y)u = f(x,y)
\]
όπου $ h(x,y) $ και $ f(x,y) $ είναι γνωστές \textbf{συνεχείς} συναρτήσεις.
όπου  $ f_{1}(x,y) $ και $ f_{2}(x,y) $ είναι γνωστές συναρτήσεις και $ C_{1} $, $ C_{2}
$ είναι δυο διαφορετικά τμήματα τμήματα της συνοριακής καμπύλης $C$, τέτοια ώστε 
$ C=C_{1}+C_{2} $.


\section*{Καλώς Διατυπωμένο Πρόβλημα (Hadamard)}

\begin{mybox1}
  \begin{dfn}
    Ένα πρόβλημα είναι \textcolor{Col1}{καλώς διατυπωμένο} αν πληρούνται οι ακόλουθες 
    απαιτήσεις:
    \begin{enumerate}[i)]
      \item \textbf{Υπάρχει} λύση του προβλήματος
      \item Η λύση είναι \textbf{μοναδική}
      \item Η λύση εξαρτάται \textbf{συνεχώς} από τις αρχικές και συνοριακές συνθήκες 
        του προβλήματος
    \end{enumerate}
  \end{dfn}
\end{mybox1}

\begin{rem}
  Η έννοια της συνέχειας συνδέεται με το πρόβλημα της ευστάθειας της λύσης και είναι
  ουσιώδους σημασίας γιατί πρέπει μια μικρή αλλαγή στις αρχικές ή συνοριακές συνθήκες του
  προβλήματος να παράγει μια μικρή μεταβολή της λύσης σε όλη την περιοχή ολοκλήρωσης.
\end{rem}


\subsection*{1ης τάξης}

\begin{example}
  Έστω η μδε $ \pdv{u}{x} = 0, \; u=u(x,y) $ με αρχική συνθήκη $ u(x_{0},y) = f(y) $, 
  όπου η $ f(y) $ θεωρείται γνωστή συνάρτηση. Με ολοκλήρωση βρίσκουμε εύκολα ότι 
  $ u(x,y) = c(y) $, όπου $ c(y) $ αυθαίρετη συνάρτηση. Αν $ y= \text{σταθερά} $ τότε 
  $ f(y)= \text{σταθερά} $ και η λύση $ u(y)$ είναι επίσης σταθερά. Επομένως υπάρχει 
  μία μοναδική λύση 
  \[
    u(x,y)=f(y) 
  \]
  η οποία επαληθεύει την αρχική συνθήκη. Επομένως η διαφορική εξίσωση $ u_{x}=0 $ 
  με την αρχική συνθήκη $ u(x_{0},y) = f(y) $ είναι ένα καλώς τοποθετημένο πρόβλημα.
\end{example}

\begin{example}
  Έστω ότι θέλουμε να επιλύσουμε την ίδια διαφορική εξίσωση του προηγούμενου
  παραδείγματος αλλά με την αρχική συνθήκη $ u(x,0) = f(x) $, όπου $ f(x) $ είναι γνωστή 
  συνάρτηση. 
  Είναι προφανές ότι η λύση της διαφορικής εξίσωσης $ u=c(y) $ για $ y=0 $ είναι 
  σταθερά. Επομένως αυτή η λύση δεν μπορεί να επαληθεύει την αρχική συνθήκη $ u(x,0) = f(x)
  $, παρά μόνο αν η $ f(x) $ είναι ταυτοτικά ίση με μια σταθερά. Από τα ανωτέρω 
  προκύπτει ότι η εξίσωση $ u_{x}=0 $ με την αρχική συνθήκη $ u(x,0)=f(x) $ δεν είναι 
  ένα καλώς τοποθετημένο πρόβλημα.
\end{example}



\section*{Προβλήματα Ιδιοτιμών}

Προβλήματα της μορφής $ Ly= \lambda y $, όπου ένας γραμμικός διαφορικός τελεστής $L$ 
δρα σε μια συνάρτηση και μου δίνει ως αποτέλεσμα την ίδια τη συνάρτηση, πολλαπλασιασμένη 
επί μία σταθερά, ονομάζονται \textcolor{Col1}{προβλήματα ιδιοτιμών}.
Οι τιμές του $ \lambda $ για τις οποίες κάποια συνάρτηση $ y $ ικανοποιεί την εξίσωση 
$ Ly= \lambda y $ ονομάζονται \textcolor{Col1}{ιδιοτιμές} και οι αντίστοιχες 
συναρτήσεις, ονομάζονται \textcolor{Col1}{ιδιοσυναρτήσεις}.

\subsection*{Φάσμα}

Το σύνολο των ιδιοτιμών ενός τέτοιου προβλήματος ονομάζεται φάσμα. Το φάσμα ενός 
προβλήματος ιδιοτιμών, μπορεί να είναι \textbf{διακριτό}, ή \textbf{συνεχές}. Συνεχές 
φάσμα ιδιοτιμών, προκύπτει συνήθως σε προβλήματα όπου το πεδίο ορισμού 
δεν είναι πεπερασμένο αλλά μπορεί το ένα ή και τα δύο όρια να τείνουν στο άπειρο.

\begin{rem}
  Το αντικείμενο μιας μαθηματικής μελέτης του προβλήματος ιδιοτιμών, είναι να 
  προβλέψει γενικές ιδιότητες του φάσματος, για παράδειγμα αν οι ιδιοτιμές είναι 
  πραγματικές ή όχι καθώς και γενικές ιδιότητες του συστήματος των ιδιοσυναρτήσεων.
\end{rem}

\subsection*{Ιδιότητες των Ιδιοσυναρτήσεων}

Οι ιδιοσυναρτήσεις, ενός προβλήματος ιδιοτιμών, υπό τις κατάλληλες προυποθέσεις, συνήθως 
ικανοποιούν:
\begin{myitemize}
  \item συνθήκες \textbf{ορθογωνιότητας}, με την έννοια ότι το εσωτερικό γινόμενο 
    δύο διαφορετικών τέτοιων συναρτήσεων σε κάποιο διάστημα $[a,b]$ είναι μηδέν
    \[
      \int _{a}^{b}y_{n}(x)\cdot y_{m}(x) \,{dx} = 0, \quad n \neq m
    \] 
  \item συνθήκες \textbf{πληρότητας}, με την έννοια ότι κάθε συνάρτηση $ f(x) $ μπορεί να 
    αναπτυχθεί σε σειρά ιδιοσυναρτήσεων της μορφής
    \[
      f(x) = \sum_{n=1}^{\infty} c_{n} y_{n}(x) 
    \]
\end{myitemize}

Οι προυποθέσεις που πρέπει να ισχύουν ώστε οι ιδιοσυναρτήσεις ενός προβλήματος ιδιοτιμών
να είναι πραγματικές και οι αντίστοιχες ιδιοσυναρτήσεις να έχουν τις παραπάνω πολύ
σημαντικές ιδιότητες, είναι:
\begin{myitemize}
  \item Κανένα σημείο του κλειστού διαστήματος $ [a,b] $ δεν είναι ιδιόμορφο (ή
    ανώμαλο) σημείο του διαφορικού τελεστή 
    \[ L = a_{1}(x) \dv[2]{}{x} + a_{2}(x) \dv{}{x} + a_{3}(x) \] 
  \item Οι Συνοριακές Συνθήκες ανήκουν σε κάποιον από τους ακόλουθους τύπους:
    \begin{myitemize}
      \item \textbf{Τύπου Ι}
        \begin{enumerate}[i)]
          \item $ y(a)=y(b)=0 $
          \item $ y'(a)=y'(b)=0 $
          \item $ y'(a)=y(b)=0 $
          \item $ y(a)=y'(b)=0 $
        \end{enumerate}
      \item \textbf{Τύπου ΙΙ}
        \begin{enumerate}[i)]
          \item $ y(a)=y(b) $ και $ y'(a)=y'(b) $ (περιοδικές)
        \end{enumerate}
    \end{myitemize}
\end{myitemize}


\section*{Πρότυπες Μορφές Προβλημάτων Ιδιοτιμών}

Το πρόβλημα ιδιοτιμών, μπορεί να πάρει πρότυπες μορφές, όπως είναι η μορφή Liouville 
και η μορφή Schrodinger.

\subsection*{Μορφή Liouville}

Έστω το πρόβλημα ιδιοτιμών $ Ly= \lambda y $, όπου 
\[
  L = a_{1}(x) \dv[2]{}{x} + a_{2}(x) \dv{}{x} + a_{3}(x) 
\]
Άρα, έχουμε την εξίσωση ιδιοτιμών $ a_{1}(x)y''+ a_{2}(x)y'+ (a_{3}(x)- \lambda)y = 0 $. 
Πολλαπλασιάζουμε την εξίσωση με τον ολοκληρωτικό παράγοντα 
\[
  \mu (x) = \frac{1}{a_{1}} \mathrm{e}^{\int \frac{a_{2}}{a_{1}}} \,{dx}
\] 
και η εξίσωση γίνεται
\[
  \mu(x) a_{1}(x) y'' + \mu(x) a_{2}(x) y' + (\mu(x) a_{3}(x) - \mu(x) \lambda)y = 0
\]
Θέτουμε $ \mu(x) a_{1}(x) = p(x) $, και παρατηρούμε ότι $ \mu(x) a_{2}(x) = p'(x) $, 
οπότε αν θέσουμε επίσης $ \mu (x) a_{3}(x) = - u(x) $ και $ - \mu (x) = w(x) $, η 
εξίσωση γίνεται
\[
  p(x) y'' + p'(x)y' + (\lambda w(x) - u(x))y = 0 \quad \text{(\textbf{Μορφή Liouville})}
\] 
ή ισοδύναμα
\[
  [p(x)y']' +  (\lambda w(x) - u(x))y = 0
\] 

\enlargethispage{2\baselineskip}

\subsection*{Μορφή Schrodinger}

\[
  y'' + (\lambda - u(x))y = 0 \quad \text{(\textbf{Μορφή Schrodinger})}
\]
όπου $ p(x)=1 $ και άρα $ p'(x)=0 $.


\end{document}

