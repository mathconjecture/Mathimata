
\documentclass[a4paper]{article}
\usepackage{etex}
%%%%%%%%%%%%%%%%%%%%%%%%%%%%%%%%%%%%%%
% Babel language package
\usepackage[english,greek]{babel}
% Inputenc font encoding
\usepackage[utf8]{inputenc}
%%%%%%%%%%%%%%%%%%%%%%%%%%%%%%%%%%%%%%

%%%%% math packages %%%%%%%%%%%%%%%%%%
\usepackage{amsmath}
\usepackage{amssymb}
\usepackage{amsfonts}
\usepackage{amsthm}
\usepackage{proof}

\usepackage{physics}

%%%%%%% symbols packages %%%%%%%%%%%%%%
\usepackage{bm} %for use \bm instead \boldsymbol in math mode 
\usepackage{dsfont}
\usepackage{stmaryrd}
%%%%%%%%%%%%%%%%%%%%%%%%%%%%%%%%%%%%%%%


%%%%%% graphicx %%%%%%%%%%%%%%%%%%%%%%%
\usepackage{graphicx}
\usepackage{color}
%\usepackage{xypic}
\usepackage[all]{xy}
\usepackage{calc}
\usepackage{booktabs}
%\usepackage{minibox}
%%%%%%%%%%%%%%%%%%%%%%%%%%%%%%%%%%%%%%%

\usepackage{enumerate}

\usepackage{fancyhdr}
%%%%% header and footer rule %%%%%%%%%
\setlength{\headheight}{14pt}
\renewcommand{\headrulewidth}{0pt}
\renewcommand{\footrulewidth}{0pt}
\fancypagestyle{plain}{\fancyhf{}
\fancyhead{}
\lfoot{}
\rfoot{\small \thepage}}
\fancypagestyle{vangelis}{\fancyhf{}
\rhead{\small \leftmark}
\lhead{\small }
\lfoot{}
\rfoot{\small \thepage}}
%%%%%%%%%%%%%%%%%%%%%%%%%%%%%%%%%%%%%%%

\usepackage{hyperref}
\usepackage{url}
%%%%%%% hyperref settings %%%%%%%%%%%%
\hypersetup{pdfpagemode=UseOutlines,hidelinks,
bookmarksopen=true,
pdfdisplaydoctitle=true,
pdfstartview=Fit,
unicode=true,
pdfpagelayout=OneColumn,
}
%%%%%%%%%%%%%%%%%%%%%%%%%%%%%%%%%%%%%%

\usepackage[space]{grffile}

\usepackage{geometry}
\geometry{left=25.63mm,right=25.63mm,top=36.25mm,bottom=36.25mm,footskip=24.16mm,headsep=24.16mm}

%\usepackage[explicit]{titlesec}
%%%%%% titlesec settings %%%%%%%%%%%%%
%\titleformat{\chapter}[block]{\LARGE\sc\bfseries}{\thechapter.}{1ex}{#1}
%\titlespacing*{\chapter}{0cm}{0cm}{36pt}[0ex]
%\titleformat{\section}[block]{\Large\bfseries}{\thesection.}{1ex}{#1}
%\titlespacing*{\section}{0cm}{34.56pt}{17.28pt}[0ex]
%\titleformat{\subsection}[block]{\large\bfseries{\thesubsection.}{1ex}{#1}
%\titlespacing*{\subsection}{0pt}{28.80pt}{14.40pt}[0ex]
%%%%%%%%%%%%%%%%%%%%%%%%%%%%%%%%%%%%%%

%%%%%%%%% My Theorems %%%%%%%%%%%%%%%%%%
\newtheorem{thm}{Θεώρημα}[section]
\newtheorem{cor}[thm]{Πόρισμα}
\newtheorem{lem}[thm]{λήμμα}
\theoremstyle{definition}
\newtheorem{dfn}{Ορισμός}[section]
\newtheorem{dfns}[dfn]{Ορισμοί}
\theoremstyle{remark}
\newtheorem{remark}{Παρατήρηση}[section]
\newtheorem{remarks}[remark]{Παρατηρήσεις}
%%%%%%%%%%%%%%%%%%%%%%%%%%%%%%%%%%%%%%%

\begin{document}

\begin{center}
\fbox{\bfseries\large Ολοκληρώματα Ρητών και Άρρητων Συναρτήσεων}
\end{center}

\vspace{2\baselineskip}
\everymath{\displaystyle}
\pagestyle{empty}

\begin{enumerate}

\item Να αναλυθούν σε άθροισμα απλών κλασμάτων οι παρακάτω ρητές συναρτήσεις.

\begin{enumerate}[i)]
\item $\frac{f(x)}{g(x)}=\frac{x^2+4x+7}{(x+2)(x+3)^2}$ \hfill Απ: $A_1=3, A_2=-2, A_3=-4$
\item $\frac{f(x)}{g(x)}=\frac{x^5+1}{(x^2-x-1)^3}$ \hfill Απ: \begin{tabular}{l} 
$A_1=\phantom{-}1, B_1=\phantom{-}2$ \\ $A_2=\phantom{-}1, B_2=-3$ \\ $A_3=-1, B_3=\phantom{-}2$
\end{tabular}
\item $\frac{f(x)}{g(x)}=\frac{x^3-3x}{(x+1)^2(x^2+x+2)}$ \hfill Απ: $A_1=\frac{1}{2}, A_2=1, A_3=\frac{1}{2}, B_3=-3$
\end{enumerate}

\item Να υπολογιστούν τα παρακάτω ολοκληρώματα Ρητών συναρτήσεων.

\begin{enumerate}[i)]
\item $\int\frac{dx}{x^2-4}$\hfill Απ: $\frac{1}{4}\ln\abs{\frac{x-2}{x+2}}+c$
\item $\int\frac{x^2+x-1}{(x-2)(x^2+1)}\,dx$ \hfill Απ: $\ln\abs{x-2}+\arctan x+c$
\item $\int\frac{x+1}{x^3+x^2-6x}\,dx$ \hfill Απ: $\ln\frac{\abs{x-2}^{\frac{3}{10}}}{x^{\frac{1}{6}}\abs{x+3}^{\frac{2}{15}}}$
\item $\int\frac{x^2}{1-x^4}\,dx$ \hfill Απ: $\frac{1}{4}\ln\abs{\frac{1+x}{1-x}}-\frac{1}{2}\arctan x+c$
\item $\int\frac{x^4-x^3-x-1}{x^3-x^2}\,dx$ \hfill Απ: $\frac{x^2}{2}+2\ln\abs{x}-\frac{1}{x}-2\ln\abs{x-1}+c$
\item $\int \frac{x^2-x+1}{(x+1)^2}\,dx$\hfill Απ: $-3\ln\abs{x+1}-\frac{3}{x+1}+x+c$
\item $\int\frac{1}{e^{2x}-3e^{x}}\,dx$ \hfill Απ: $-\frac{2}{9}x+\frac{1}{9}\ln\abs{e^x-3}+c$
\end{enumerate}



\item Να υπολογιστούν τα παρακάτω ολοκληρώματα με τη βοήθεια κατάλληλων τριγωνομετρικών αντικαταστάσεων.

\begin{enumerate}[i)]

\item $\int\frac{dx}{x\sqrt{4+x^2}}$ \hfill Απ: $-\frac{\sqrt{4+x^2}}{4x}+c$
\item $\int\frac{dx}{(x^2+1)^{\frac{3}{2}}}$ \hfill Απ: $\frac{x}{\sqrt{x^2+1}}+c$
\item $\int\frac{dx}{x^2\sqrt{9-x^2}}\,,\quad \abs{x}<3$ \hfill Απ: $-\frac{\sqrt{9-x^2}}{9x}+c$
\item $\int\frac{x^2}{\sqrt{x^2-4}}\,dx,\quad x>2$ \hfill Απ: $\frac{x}{2}\sqrt{x^2+4}+2\ln(x+\sqrt{x^2+4})+c$.
\item $\int \frac{dx}{x^{2}\sqrt{1-x^{2}}}$ \hfill Απ: $-\frac{\sqrt{1-x^{2}}}{x}$

\end{enumerate}

\end{enumerate}



\end{document}