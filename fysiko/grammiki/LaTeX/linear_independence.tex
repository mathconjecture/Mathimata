\documentclass[a4paper,12pt]{article}
\usepackage{etex}
%%%%%%%%%%%%%%%%%%%%%%%%%%%%%%%%%%%%%%
% Babel language package
\usepackage[english,greek]{babel}
% Inputenc font encoding
\usepackage[utf8]{inputenc}
%%%%%%%%%%%%%%%%%%%%%%%%%%%%%%%%%%%%%%

%%%%% math packages %%%%%%%%%%%%%%%%%%
\usepackage{amsmath}
\usepackage{amssymb}
\usepackage{amsfonts}
\usepackage{amsthm}
\usepackage{proof}

\usepackage{physics}

%%%%%%% symbols packages %%%%%%%%%%%%%%
\usepackage{bm} %for use \bm instead \boldsymbol in math mode 
\usepackage{dsfont}
\usepackage{stmaryrd}
%%%%%%%%%%%%%%%%%%%%%%%%%%%%%%%%%%%%%%%


%%%%%% graphicx %%%%%%%%%%%%%%%%%%%%%%%
\usepackage{graphicx}
\usepackage{color}
%\usepackage{xypic}
\usepackage[all]{xy}
\usepackage{calc}
\usepackage{booktabs}
\usepackage{minibox}
%%%%%%%%%%%%%%%%%%%%%%%%%%%%%%%%%%%%%%%

\usepackage{enumerate}

\usepackage{fancyhdr}
%%%%% header and footer rule %%%%%%%%%
\setlength{\headheight}{14pt}
\renewcommand{\headrulewidth}{0pt}
\renewcommand{\footrulewidth}{0pt}
\fancypagestyle{plain}{\fancyhf{}
\fancyhead{}
\lfoot{}
\rfoot{\small \thepage}}
\fancypagestyle{vangelis}{\fancyhf{}
\rhead{\small \leftmark}
\lhead{\small }
\lfoot{}
\rfoot{\small \thepage}}
%%%%%%%%%%%%%%%%%%%%%%%%%%%%%%%%%%%%%%%

\usepackage{hyperref}
\usepackage{url}
%%%%%%% hyperref settings %%%%%%%%%%%%
\hypersetup{pdfpagemode=UseOutlines,hidelinks,
bookmarksopen=true,
pdfdisplaydoctitle=true,
pdfstartview=Fit,
unicode=true,
pdfpagelayout=OneColumn,
}
%%%%%%%%%%%%%%%%%%%%%%%%%%%%%%%%%%%%%%

\usepackage[space]{grffile}

\usepackage{geometry}
\geometry{left=25.63mm,right=25.63mm,top=36.25mm,bottom=36.25mm,footskip=24.16mm,headsep=24.16mm}

%\usepackage[explicit]{titlesec}
%%%%%% titlesec settings %%%%%%%%%%%%%
%\titleformat{\chapter}[block]{\LARGE\sc\bfseries}{\thechapter.}{1ex}{#1}
%\titlespacing*{\chapter}{0cm}{0cm}{36pt}[0ex]
%\titleformat{\section}[block]{\Large\bfseries}{\thesection.}{1ex}{#1}
%\titlespacing*{\section}{0cm}{34.56pt}{17.28pt}[0ex]
%\titleformat{\subsection}[block]{\large\bfseries{\thesubsection.}{1ex}{#1}
%\titlespacing*{\subsection}{0pt}{28.80pt}{14.40pt}[0ex]
%%%%%%%%%%%%%%%%%%%%%%%%%%%%%%%%%%%%%%

%%%%%%%%% My Theorems %%%%%%%%%%%%%%%%%%
\newtheorem{thm}{Θεώρημα}[section]
\newtheorem{cor}[thm]{Πόρισμα}
\newtheorem{lem}[thm]{λήμμα}
\theoremstyle{definition}
\newtheorem{dfn}{Ορισμός}[section]
\newtheorem{dfns}[dfn]{Ορισμοί}
\theoremstyle{remark}
\newtheorem{remark}{Παρατήρηση}[section]
\newtheorem{remarks}[remark]{Παρατηρήσεις}
%%%%%%%%%%%%%%%%%%%%%%%%%%%%%%%%%%%%%%%




\input{definitions_ask.tex}
\input{tikz.tex}

\usepackage{anyfontsize}
\pagestyle{vangelis}
\setcounter{chapter}{1}

\linespread{1.1}

\begin{document}

\begin{center}
  \minibox{\large \bfseries \textcolor{Col1}{Ασκήσεις στους Διανυσματικούς Χώρους}}
\end{center}

\vspace{\baselineskip}



\section*{Γραμμική Ανεξαρτησία}


\begin{enumerate}

  \item\label{ask:lineks} Να εξετάσετε αν τα παρακάτω διανύσματα είναι γραμμικώς 
    ανεξάρτητα.
    \begin{enumerate}[(i)]
      \item $ \mathbf{u} = (1,2) $, $ \mathbf{v} = (3,-5) $ \hfill Απ: ναι
      \item $ \mathbf{u} = (1,2,-3) $ $ \mathbf{v} = (4,5,-6) $ \hfill Απ: ναι
      \item $ \mathbf{u} = (1,1,0)$, $ \mathbf{v} = (1,3,2)$, $ \mathbf{w} = (4,9,5) $ 
        \hfill Απ: όχι 
      \item $ \mathbf{u} = (1,2,3)$, $ \mathbf{v} = (2,5,7)$, $ \mathbf{w} = (1,3,5) $ 
        \hfill Απ: ναι 
      \item $ \mathbf{u} = (1,2,-3,1) $, $ \mathbf{v} = (3,7,1,-2) $, $ \mathbf{w} =
        (1,3,7,-4) $ \hfill Απ: όχι
      \item $ \mathbf{u} = (1,3,1,-2) $, $ \mathbf{v} = (2,5,-1,3) $, $ \mathbf{w} =
        (1,3,7,-2) $ \hfill Απ: ναι
    \end{enumerate}

  \item Αν $ \mathbf{u} $, $ \mathbf{v} $, $ \mathbf{w} $ είναι γραμμικώς 
    ανεξάρτητα διανύσματα τότε να εξετάσετε αν τα παρακάτω σύνολα διανυσμάτων 
    είναι επίσης γραμμικώς ανεξάρτητα.
    \begin{enumerate}[(i)]
      \item $ S = \{ \mathbf{u} + \mathbf{v} - 2 \mathbf{w}, \mathbf{u} - \mathbf{v} -
        \mathbf{w}, \mathbf{u} + \mathbf{w} \} $ \hfill Απ: ναι 
      \item $ S = \{ \mathbf{u} + \mathbf{v} - 3 \mathbf{w}, \mathbf{u} + 3 \mathbf{v} -
        \mathbf{w}, \mathbf{v} + \mathbf{w}\}  $ \hfill Απ: όχι 
    \end{enumerate}

  \item\label{ask:eksart} Να εκφράσετε το διάνυσμα $ \mathbf{b} = (1,-2,5) $ ως γραμμικό 
    συνδυασμό των διανυσμάτων $ \mathbf{u} = (1,1,1)$, $ \mathbf{v} = (1,2,3)$, 
    $ \mathbf{w} = (2,-1,1) $.
    \hfill Απ: $ \mathbf{b} = -6 \mathbf{u} + 3 \mathbf{v} +2 \mathbf{w} $ 	

  \item\label{ask:eksart2} Να εξετάσετε αν το διάνυσμα $ \mathbf{b} = (2,5,-3) $ 
    γράφεται ως γραμμικός συνδυασμός των διανυσμάτων $ \mathbf{u} = (1,-3,2)$, 
    $ \mathbf{v} = (2,4,1)$, $ \mathbf{w} = (1,-5,7) $.
    \hfill Απ: ναι 

  \item Να εξετάσετε αν το διάνυσμα $ \mathbf{b} = (-3,0,12,15) $ μπορεί να γραφεί 
    ως γραμμικός συνδυασμός των διανυσμάτων $ \mathbf{u}_{1} = (1,2,-2,1) $, 
    $ \mathbf{u}_{2} = (1,3,-1,4) $ και $ \mathbf{u}_{3} = (2,1,-7,-7) $. 
    Αν ναι, τότε να βρεθεί ένας τέτοιος γραμμικός συνδυασμός.

    \hfill Απ: $ \mathbf{b} = -9 \mathbf{u}_{1} + 6 \mathbf{u}_{2} $ 

  \item\label{ask:eksart3} Έστω τα διανύσματα $ \mathbf{u}_{1} = (1,2,0) $, 
    $ \mathbf{u}_{2} = (-1,1,2) $ 
    και $ \mathbf{u}_{3} = (3,0,-4) $. Να βρείτε ποια συνθήκη πρέπει να ικανοποιούν τα 
    $a$, $b$, $c$, ώστε το διάνυσμα $ \mathbf{b} = (a,b,c) $ να ανήκει στον υπόχωρο 
    $ \Span \{ \mathbf{u}_{1}, \mathbf{u}_{2}, \mathbf{u}_{3} \} $.

    \hfill Απ: $ 4a -2b + 3c = 0 $ 

  \item\label{ask:synd} Να δείξετε ότι τα διανύσματα $ \mathbf{u}_{1} = (1,3,0,5) $, 
    $ \mathbf{u}_{2} = (1,2,1,4) $, $ \mathbf{u}_{3} = (1,1,2,3)$, 
    $ \mathbf{u}_{4} = (1,1,2,3) $, $ \mathbf{u}_{5} = (1,-3,6,-1) $ 
    είναι γραμμικώς εξαρτημένα και να βρεθεί μια σχέση που τα συνδέει. 
    \hfill Απ: $ \mathbf{u}_{1} - 2 \mathbf{u}_{2} + \mathbf{u}_{3} = 0 $ 

\end{enumerate}


\end{document}

