\input{preamble_ask.tex}
\input{definitions_ask.tex}


\begin{document}



\pagestyle{askhseis}

\begin{center}
  \minibox{\large \bfseries \textcolor{Col1}{Προβλήματα στο ρυθμό μεταβολής}}
\end{center}

\vspace{\baselineskip}

\begin{enumerate}

	\item  Φουσκώνουμε με αέρα ένα σφαιρικό μπαλόνι, έτσι
		ώστε ο όγκος του να αυξάνει με ρυθμό $\SI{100}{cm^{3}\per s}$. Πόσο γρήγορα
		αυξάνει η ακτίνα του μπαλονιού όταν η διάμετρος του είναι $\SI{50}{cm}$?

		\hfill Απ: $\frac{1}{25\pi}\si{cm\per s}$


	\item  Μια σκάλα μήκους $\SI{10}{m}$ είναι ακουμπησμένη
		σε ένα κατακόρυφο τοίχο. Εάν το κάτω μέρος (βάση) της σκάλας ολισθαίνει
		οριζόντια, απομακρυνόμενη από τον τοίχο με ρυθμό $\SI{1}{m\per s}$, πόσο
		γρήγορα η κορυφή της σκάλας πέφτει (ολισθαίνοντας κατακόρυφα στον τοίχο), όταν η
		βάση της σκάλας απέχει $\SI{6}{m}$ από τον τοίχο?

		\hfill Απ: $\frac{3}{4}\si{m\per s}$

	\item  Μια δεξαμενή νερού έχει το σχήμα ενός ανάποδου
		κυκλικού κώνου με ακτίνα βάσης $2$ $\si{m}$ και ύψος $4$ $\si{m}$. Αν
		στη δεξαμενή εισέρχεται ποσότητα νερού με ρυθμό $2$ $\si{m^{3}/min}$, να
		υπολογίσετε το ρυθμό με τον οποίο ανέρχεται η στάθμη του νερού στο εσωτερικό της
		δεξαμενής, όταν το νερό έχει βάθος $3$ $\si{m}$.

		\hfill Απ: $\frac{8}{9\pi}\si{m\per s}$

	\item   Αυτοκίνητο $A$ ταξιδεύει δυτικά με ταχύτητα $50$
		$\si{km\per h}$ και αυτοκίνητο $B$ ταξιδεύει βόρεια με ταχύτητα $60$
		$\si{km/h}$. Και τα δύο αυτοκίνητα κινούνται προς την διαστάυρωση των δύο δρόμων. Με τι ρυθμό τα δύο αυτοκίνητα πλησιάζουν το ένα το άλλο, όταν το $A$ βρίσκεται $0.3$ $\si{km}$ και το $B$ $0.4$ $\si{km}$ από τη διασταύρωση?

		\hfill Απ: $\SI{78}{km/h}$

	\item  Ένας άνδρας περπατά σε ευθύγραμμο μονοπάτι με
		ταχύτητα $\SI{4}{m/s}$. Μια δέσμη leiser η οποία είναι τοποθετημένη σε απόσταση $\SI{20}{m}$ από το μονοπάτι, τον σημαδεύει συνεχώς. Με ποιο ρυθμό περιστρέφεται η δέσμη leiser οταν ο άνδρας βρίσκεται σε απόσταση $\SI{15}{m}$ από εκείνο το σημείο στο μονοπάτι το οποίο απέχει ελάχιστη απόσταση από την πηγή leiser?

		\hfill Απ: $\SI{0.128}{rad/s}$

	\item  Πλοίο Α ξεκινάει από ένα λιμάνι στις 12 μ.μ. και κατευθύνεται δυτικά
		με ταχύτητα  $\SI{9}{km/h}$. Πλοίο Β ξεκινάει από το ίδιο λιμάνι στη 1
		μ.μ. και κατευθύνεται νότια με ταχύτητα $\SI{12}{km/h}$. Με τι ρυθμό
		απομακρύνονται μεταξύ τους τα δύο πλοία στις 3 μ.μ.?

		\hfill Απ: $\SI{14,70}{km/h}$

\end{enumerate}


\end{document}
