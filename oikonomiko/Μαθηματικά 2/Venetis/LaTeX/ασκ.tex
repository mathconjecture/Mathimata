\documentclass[a4paper,12pt]{article}


\usepackage{amsmath}
\usepackage{amssymb}
\usepackage{amsfonts}
\usepackage{amsthm}
\usepackage{physics}
\usepackage{graphicx}

\usepackage[top=2cm,bottom=2cm,left=2cm,right=2cm]{geometry}



\begin{document}

\thispagestyle{empty}

\begin{center}
\fbox{\large \bfseries Ασκήσεις στους Πίνακες κ Ορίζουσες}
\end{center}

\vspace{\baselineskip}


\begin{enumerate}

\item {\bfseries Να υπολογιστούν οι ορίζουσες των παρακάτω πινάκων και να βρεθούν οι αντίστροφοι, αν υπάρχουν.}

\renewcommand{\theenumii}{\roman{enumii}}
\begin{enumerate}
\item 
\(
A=\begin{pmatrix}
1 & -2 \\ 
2 & \phantom{-}5
\end{pmatrix}
\) \hfill Απ: $|A|=9,\quad A^{-1}= \begin{pmatrix}
5 & -2 \\ 
2 & -1
\end{pmatrix}$
\item  
\(
B=
\begin{pmatrix}
-1 & 4 \\ 
-2 & 5
\end{pmatrix}
\)\hfill Απ: $|B|=3,\quad B^{-1}= \begin{pmatrix}
\frac{5}{3} & -\frac{4}{3} \\[2pt] 
\frac{2}{3} & -\frac{1}{3}
\end{pmatrix}$
\item 
\(
C=
\begin{pmatrix}
\phantom{-}1 & -2 \\ 
-3 & \phantom{-}6
\end{pmatrix}
\)\hfill Απ: $|C|=0$
\end{enumerate}



\item {\bfseries Ομοίως και για τους παρακάτω $3\times 3$ πίνακες.}

\begin{enumerate}

\item 
\(
A=\begin{pmatrix}
1 & -1 & \phantom{-}1\\ 
1 & \phantom{-}1 & \phantom{-}0\\ 
1 & \phantom{-}0 & \phantom{-}1
\end{pmatrix}
\)\hfill Απ: $|A|=1,\quad A^{-1}=\begin{pmatrix}
\phantom{-}1 & \phantom{-}1 & -1\\ 
-1 & \phantom{-}0 & \phantom{-}1\\ 
-1 & -1 & \phantom{-}2
\end{pmatrix}$
\item 
\(
B=\begin{pmatrix}
1 & -1 & \phantom{-}2\\ 
2 & \phantom{-}0 & \phantom{-}1\\ 
1 & \phantom{-}2 & -1
\end{pmatrix}
\)\hfill Απ: $|B|=3,\quad B^{-1}=\begin{pmatrix}
-\frac{2}{3} & \phantom{-}1 & -\frac{1}{3}\\ 
\phantom{-}1 & -1 & \phantom{-}1\\ 
\phantom{-}\frac{4}{3} & -1 & \phantom{-}\frac{2}{3}
\end{pmatrix} $
\item 
\(
C=\begin{pmatrix}
4 & 2 & 3\\ 
2 & 1 & 2\\ 
1 & 1 & 1
\end{pmatrix}
\)\hfill Απ: $|C|=-1,\quad C^{-1}= \begin{pmatrix}
\phantom{-}1 & -1 & -1\\ 
\phantom{-}0 & -1 & \phantom{-}2\\ 
-1 & \phantom{-}2 & \phantom{-}0
\end{pmatrix}$


\end{enumerate}

\end{enumerate}



\end{document}

