\documentclass[a4paper,table]{report}
\input{preamble_ask.tex}
\input{definitions_ask.tex}


\pagestyle{askhseis}
\everymath{\displaystyle}

% \geometry{top=30.25mm,bottom=30.25mm}


\begin{document}

\begin{center}
  \minibox{{\large\bfseries \textcolor{Col1}{Ασκήσεις στις Μερικές Παραγώγους}}}
\end{center}


\section*{Μερική Παράγωγος}

\begin{enumerate}

  \item Με τη βοήθεια των κανόνων παραγώγισης να υπολογιστούν οι μερικές 
    παράγωγοι 1ης και 2ης  τάξης των παρακάτω συναρτήσεων:
    \begin{enumerate}[i)]
      \item $f(x,y)=y^2\cos (x)$ \hfill Απ: \begin{tabular}{ll}
          $f_x=-y^2\sin(x)$ & $f_{xx}=_-y^2\cos(x)$ \\
          $f_y=2y\cos(x)$ & $f_{yy}=2\cos(x)$ \\
                          & $f_{xy}=-2y\sin(x)$ 
        \end{tabular}

      \item $f(x,y)= \mathrm{e}^{xy}$ \hfill Απ: \begin{tabular}{ll}
          $f_x= y \mathrm{e}^{xy} $ & $f_{xx}=_y^2 \mathrm{e}^{xy} $ \\
          $f_y= x \mathrm{e}^{xy} $ & $f_{yy}=_x^2 \mathrm{e}^{xy} $ \\
                                    & $f_{xy}= \mathrm{e}^{xy} (1+xy) $
        \end{tabular}
    \end{enumerate}

  \item Να υπολογιστούν οι μερικές παράγωγοι 1ης τάξης των παρακάτω συναρτήσεων.

    \begin{enumerate}[i)]
      \item $ f(x,y) = y^{3} \mathrm{e}^{x^{2}y} $ \hfill Απ: $ 
        \begin{matrix*}[l]
          f_{x} = 2xy^{4} \mathrm{e}^{x^{2}y} \\ 
          f_{y} = y^{2} \mathrm{e}^{x^{2}y} (3+x^{2}y) 
        \end{matrix*} $

      % \item $ f(x,y) = y \sin{(x^{3}y^{2})} $ \hfill Απ: $ 
      %   \begin{matrix*}[l]
      %     f_{x} = 3y^{3}x^{2} \cos{(x^{3}y^{2})} \\[5pt]
      %     f_{y} = \sin{(x^{3}y^{2})} + 2x^{3}y^{2} \cos{(x^{3}y^{2})}
      %   \end{matrix*} $

      \item $ f(x,y) = x^{2} \ln{(x^{2}y)} $ \hfill Απ: $ 
        \begin{matrix*}[l]
          f_{x} = 2x \ln{(x^{2}y)} + 2x \\
          f_{y} = \frac{x^{2}}{y} 
        \end{matrix*} $

      \item $ f(x,y) = \frac{1}{x} \sin{(xy^{2})} $ \hfill Απ: $ 
        \begin{matrix*}[l]
          f_{x} = - \frac{\sin{(xy^{2})}}{x^{2}} + \frac{y^{2} \cos{(xy^{2})}}{x} 
          \\[5pt]
          f_{y} = 2y \cos{(xy^{2})}
        \end{matrix*} $

      \item $ f(x,y) = x \sqrt{1 - xy}$ \hfill Απ: $ 
        \begin{matrix*}[l]
          f_{x} = \frac{-3xy+2}{2 \sqrt{1-xy}} \\ 
          f_{y} = -\frac{x^{2}}{2 \sqrt{1-xy}} 
        \end{matrix*} $

      \item $ f(x,y) = \ln{\Bigl(\frac{y}{1+xy}\Bigr)} $ \hfill Απ: $ 
        \begin{matrix*}[l]
          f_{x} = \frac{-y}{1 + xy} \\ 
          f_{y} = \frac{1}{y(1 + xy)} 
        \end{matrix*} $
    \end{enumerate}

    %span
  \item Να δείξετε ότι η συνάρτηση $ f(x,y) = \cos{(x+y)} + \cos{(x-y)} $ 
    επαληθεύει την διαφορική εξίσωση $ f_{xx} - f_{yy} = 0 $.
\end{enumerate}


\section*{Διαφορικό}

\begin{enumerate}

  \item Να βρεθεί το ολικό διαφορικό 1ης τάξης, της συνάρτησης 
    $f(x,y)=\ln(xy)+\cos(y^2)$ 

    \hfill Απ: $df=\frac{dx}{x}+\left(\frac{1}{y}-2y\sin(y^2)\right)dy$

  \item Για τις παρακάτω παραστάσεις, να αποδείξετε ότι είναι \textbf{τέλεια
    διαφορικά} και να υπολογίσετε τη συνάρτηση δυναμικού.
    \begin{enumerate}[i)]
      \item $ \left(x+e^{x/y}\right)dx + e^{x/y}\left(1- \frac{x}{y}\right)dy $
        \hfill Απ: $ f(x,y) = \frac{x^{2}}{2} +y e^{x/y} + c $ 

      \item $ (6x^{3}- \sin{x})dx + (9x^{2}y^{2}+ \cos{y})dy $ 
        \hfill Απ: $ f(x,y) = 3x^{2}y^{3}+ \cos{x} + \sin{y} + c $  

      \item $\left(2e^{x}+\frac{1}{x}-3\sin y\right)dx+3(y^2-x\cos y)dy$ 
        \hfill  Απ: $ f(x,y,z) = y^{3}-3x \sin{y} + 2e^{x} + \ln{x} +c $.
        %spand (114)
    \end{enumerate}

  \item Να υπολογιστεί το $a$ ώστε η παράσταση, να είναι \textbf{τέλειο διαφορικό}.
    \[ \frac{ x + ay }{ (x-y)^{3} }dx + \frac{ ax+y }{ (x-y)^{3} }dy \]
    \hfill Απ: $ a=-1 $
\end{enumerate}

\end{document}

