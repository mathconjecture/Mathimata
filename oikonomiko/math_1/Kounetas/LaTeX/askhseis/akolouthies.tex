\input{preamble_ask.tex}
\input{definitions_ask.tex}


\everymath{\displaystyle}

\pagestyle{askhseis}


\begin{document}

\begin{center}
  \minibox{\large\bfseries \textcolor{Col1}{Ασκήσεις στις Ακολουθίες}}
\end{center}

\vspace{\baselineskip}



\section*{Ακολουθίες}

\begin{enumerate}
  \item Να γράψετε τους 5 πρώτους όρους της ακολουθίας
    $
      a_{n} = \frac{(-1)^{n-1}}{2n-1} 
    $ 
    \hfill Απ: $1,-1//3,1/5,-1/7,1/9$
  \item Δίνονται οι ακολουθίες 
    $
      a_{n} = \frac{1}{(-1)^{n}+2}$ και $b_{n} = \frac{1 \cdot 3
      \cdot 5 \cdots (2n-1)}{2\cdot 4 \cdot 6 \cdots (2n)}  
    $ 
    Να υπολογιστούν $ a_{20} $, $ a_{33} $ και $ b_{3} $ και $ b_{5} $.

    \hfill Απ:  $ a_{20} = 1/3 $, $ a_{33} = 1 $, $ b_{3} = 5/16 $, $ b_{5} = 63/256 $

  \item Δίνεται η ακολουθία:
    $
      a_{n}= \frac{2n (n-1)}{n^{2}+3}  
    $ 
    Να εξετάσετε αν ο $ 10/7 $ είναι όρος της ακολουθίας.
    \hfill Απ: ναι $ a_{5}=10/7 $ 
\end{enumerate}



\section*{Αναδρομικές Ακολουθίες}

\begin{enumerate}
  \item Να βρεθεί ο 4ος και ο 6ος όρος της ακολουθίας $ a_{1}=1 $, $ a_{2}=8 $ και 
    $ a_{n+2}=4(a_{n+1}-a_{n}) $
    \hfill Απ: $ a_{4}=80 $, $ a_{6}=512 $ 

  \item Να βρεθεί ο $n$-οστός όρος της ακολουθίας 
    $
      a_{1}=3 $ και $ 3a_{n+1} = a_{n}, \; \forall n \in \mathbb{N}
    $ 
    \hfill Απ: $ a_{n} = 3 \cdot \left(\frac{1}{3} \right)^{n-1} = 3^{2-n} $ 

  \item Να ορίσετε αναδρομικά της ακολουθία $ a_{n} = 3n-2 $.
    \hfill Απ: $ a_{1}=1 $ και $ a_{n+1}=a_{n}+3 $ 
\end{enumerate}



\section*{Αριθμητική Πρόοδος}

\begin{enumerate}
  \item Δίνεται η αριθμητική πρόοδος $ 1,5/3,7/3, \ldots $
    \begin{enumerate}[i)]
      \item Να βρείτε το $n$-οστό της προόδου.
      \item Να βρείτε το 16ο όρο της προόδου.
    \end{enumerate}
    \hfill Απ: $ a_{n}= \frac{2}{3} n + \frac{1}{3} $, $ a_{16}=11 $

  \item Να υπολογιστεί το άθροισμα των 30 πρώτων όρων της προόδου $ -6,-2,2,6,\ldots $.
    \hfill Απ: $ S_{30}=1560 $ 

  \item Να υπολογιστεί το άθροισμα $ 1+7+13+\cdots +121 $.
    \hfill Απ: $ S_{21} = 1281 $ 

  \item Σε μια αριθμητική πρόοδο, δίνονται $ \omega = -12 $, $ S_{8}=456 $. Να βρεθούν 
    οι $ a_{1} $ και $ a_{8} $.
    \hfill Απ: $ a_{1}= 99 $, $ a_{8}=95 $ 

  \item Να βρεθεί ο $ x $ ώστε οι αριθμοί $ x+5 $, $ 2-x $ και $ 7x+9 $ να είναι 
    διαδοχικοί όροι αριθμητικής προόδου.
    \hfill Απ: $ x=-1 $ 
\end{enumerate}



\section*{Γεωμετρική Πρόοδος}

\begin{enumerate}
  \item Να βρείτε το $n$-οστό όρο της γεωμετρικής προόδου
    $
      1/2,-1/3,2/9,-4/27,\ldots 
    $
    \hfill Απ: $\lambda = -2/3 $, $ a_{n}= \frac{2^{n-2}}{(-3^{n-1})} $ 

  \item Να υπολογίσετε το άθροισμα των 7 πρώτων όρων των γεωμετρικών προόδων
    $ 2,6,18, \ldots $ και $-64,32,-16,\ldots $ 

    \hfill Απ: $ S_{7}=2186 $, $ S_{7} = -43 $ 

  \item Σε μια γεωμετρική πρόοδο, δίνονται $ S_{n}=315 $, $ a_{1}=5 $ και 
    $ \lambda = 2 $. Να βρείτε το πλήθος των όρων.
    \hfill Απ: $ n=6 $ 

  \item Να υπολογίσετε το άθροισμα $ \frac{1}{5} + (-1) + 5 + \cdots + 3125 $.
    \hfill Απ: $ S_{7} = 13021/5 $ 

  \item Να βρεθεί ο $x$ ώστε οι αριθμοί $ x-4 $, $ x+1 $ και $ x-19 $ να είναι 
    διαδοχικοί όροι γεωμετρικής προόδου. 
    \hfill Απ: $ x=3 $ 
\end{enumerate}


\end{document}

