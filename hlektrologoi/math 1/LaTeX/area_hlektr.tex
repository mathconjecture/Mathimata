\documentclass[a4paper,12pt]{article}
\usepackage{etex}
%%%%%%%%%%%%%%%%%%%%%%%%%%%%%%%%%%%%%%
% Babel language package
\usepackage[english,greek]{babel}
% Inputenc font encoding
\usepackage[utf8]{inputenc}
%%%%%%%%%%%%%%%%%%%%%%%%%%%%%%%%%%%%%%

%%%%% math packages %%%%%%%%%%%%%%%%%%
\usepackage{amsmath}
\usepackage{amssymb}
\usepackage{amsfonts}
\usepackage{amsthm}
\usepackage{proof}

\usepackage{physics}

%%%%%%% symbols packages %%%%%%%%%%%%%%
\usepackage{bm} %for use \bm instead \boldsymbol in math mode 
\usepackage{dsfont}
\usepackage{stmaryrd}
%%%%%%%%%%%%%%%%%%%%%%%%%%%%%%%%%%%%%%%


%%%%%% graphicx %%%%%%%%%%%%%%%%%%%%%%%
\usepackage{graphicx}
\usepackage{color}
%\usepackage{xypic}
\usepackage[all]{xy}
\usepackage{calc}
\usepackage{booktabs}
\usepackage{minibox}
%%%%%%%%%%%%%%%%%%%%%%%%%%%%%%%%%%%%%%%

\usepackage{enumerate}

\usepackage{fancyhdr}
%%%%% header and footer rule %%%%%%%%%
\setlength{\headheight}{14pt}
\renewcommand{\headrulewidth}{0pt}
\renewcommand{\footrulewidth}{0pt}
\fancypagestyle{plain}{\fancyhf{}
\fancyhead{}
\lfoot{}
\rfoot{\small \thepage}}
\fancypagestyle{vangelis}{\fancyhf{}
\rhead{\small \leftmark}
\lhead{\small }
\lfoot{}
\rfoot{\small \thepage}}
%%%%%%%%%%%%%%%%%%%%%%%%%%%%%%%%%%%%%%%

\usepackage{hyperref}
\usepackage{url}
%%%%%%% hyperref settings %%%%%%%%%%%%
\hypersetup{pdfpagemode=UseOutlines,hidelinks,
bookmarksopen=true,
pdfdisplaydoctitle=true,
pdfstartview=Fit,
unicode=true,
pdfpagelayout=OneColumn,
}
%%%%%%%%%%%%%%%%%%%%%%%%%%%%%%%%%%%%%%

\usepackage[space]{grffile}

\usepackage{geometry}
\geometry{left=25.63mm,right=25.63mm,top=36.25mm,bottom=36.25mm,footskip=24.16mm,headsep=24.16mm}

%\usepackage[explicit]{titlesec}
%%%%%% titlesec settings %%%%%%%%%%%%%
%\titleformat{\chapter}[block]{\LARGE\sc\bfseries}{\thechapter.}{1ex}{#1}
%\titlespacing*{\chapter}{0cm}{0cm}{36pt}[0ex]
%\titleformat{\section}[block]{\Large\bfseries}{\thesection.}{1ex}{#1}
%\titlespacing*{\section}{0cm}{34.56pt}{17.28pt}[0ex]
%\titleformat{\subsection}[block]{\large\bfseries{\thesubsection.}{1ex}{#1}
%\titlespacing*{\subsection}{0pt}{28.80pt}{14.40pt}[0ex]
%%%%%%%%%%%%%%%%%%%%%%%%%%%%%%%%%%%%%%

%%%%%%%%% My Theorems %%%%%%%%%%%%%%%%%%
\newtheorem{thm}{Θεώρημα}[section]
\newtheorem{cor}[thm]{Πόρισμα}
\newtheorem{lem}[thm]{λήμμα}
\theoremstyle{definition}
\newtheorem{dfn}{Ορισμός}[section]
\newtheorem{dfns}[dfn]{Ορισμοί}
\theoremstyle{remark}
\newtheorem{remark}{Παρατήρηση}[section]
\newtheorem{remarks}[remark]{Παρατηρήσεις}
%%%%%%%%%%%%%%%%%%%%%%%%%%%%%%%%%%%%%%%




\newcommand{\vect}[2]{(#1_1,\ldots, #1_#2)}
%%%%%%% nesting newcommands $$$$$$$$$$$$$$$$$$$
\newcommand{\function}[1]{\newcommand{\nvec}[2]{#1(##1_1,\ldots, ##1_##2)}}

\newcommand{\linode}[2]{#1_n(x)#2^{(n)}+#1_{n-1}(x)#2^{(n-1)}+\cdots +#1_0(x)#2=g(x)}

\newcommand{\vecoffun}[3]{#1_0(#2),\ldots ,#1_#3(#2)}

\newcommand{\mysum}[1]{\sum_{n=#1}^{\infty}


\everymath{\displaystyle}
\pagestyle{askhseis}

\begin{document}

\begin{center}
  \minibox{\large\bfseries \textcolor{Col1}{Εμβαδό Επίπεδων Χωρίων}}
\end{center}

\vspace{\baselineskip}



\begin{enumerate}

	\item Να υπολογιστεί το εμβαδό του επίπεδου χωρίου που περικλείεται από τις 
        καμπύλες $ y^{2} = 4x $ και $ x^{2} = 4y $.

		\hfill Απ: $ \frac{16}{3} $

	\item Να υπολογιστεί το εμβαδό του επίπεδου χωρίου που περικλείεται από την 
        καμπύλη  $ y = x^{3} $ και τις ευθείες $ y = -\frac{x}{2} $ και $ y=1 $.

		\hfill Απ: $ \frac{7}{4} $

	\item Να υπολογιστεί το εμβαδό του επίπεδου χωρίου που περικλείεται από την καμπύλη 
		$ y^{2} = x $ και την ευθεία $ y = x-6 $.

		\hfill Απ: $ \frac{125}{6} $

	\item Να υπολογιστεί το εμβαδό του επίπεδου χωρίου που περικλείεται από τις 
        καμπύλες $ x=-2y^{2} $ και $ x=1-3y^{2} $. 

        \hfill Απ: $ \frac{4}{3} $  

	\item Να υπολογιστεί το εμβαδό του χωρίου που περικλείεται από τις καμπύλες $
		x^{2} + y^{2} = 8 $ και $ y^{2} = 2x $, στο δεξιό ημιεπίπεδο.

		\hfill Απ: $ 2 \pi + \frac{4}{3} $

	\item Να υπολογιστεί το εμβαδό ενός τόξου της κυκλοειδούς καμπύλης 
        $ x = a(t - \sin{t}) $, $ y = a(1 - \cos{t}), \; a>0 $.

        \hfill Απ: $3 \pi a^{2} $

	\item Να υπολογιστεί το εμβαδό του επίπεδου χωρίου που περικλείεται από την 
        αστροειδή καμπύλη με παραμετρικές εξισώσεις $ x = a \cos^{3}{t}  $ και 
        $ y = a \sin^{3}{t}, \; t \in [0, 2 \pi] $.

        \hfill Απ: $ \frac{3 a^{2} \pi}{8} $ 

    \item Να υπολογιστεί το εμβαδό της έλλειψης, $c$ αν:
        \begin{enumerate}[i)]
            \item $ c: \frac{x^{2}}{a^{2}} + \frac{y^{2}}{b^{2}} = 1 $ 
            \item $ c: x = a \cos{t}$ και $ y = b \sin{t} $ 
        \end{enumerate}

                \hfill Απ: $ ab \pi $ 

    \item Να υπολογιστεί το εμβαδό του επίπεδου χωρίου που περικλείεται από την 
        καμπύλη με παραμετρικές εξισώσεις $ x = t^{2} - 1 $ και $ y = t^{3} - t $.

        \hfill Απ: $ \frac{8}{15} $ 

\end{enumerate}


\end{document}


