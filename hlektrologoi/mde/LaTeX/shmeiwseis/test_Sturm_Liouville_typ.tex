\documentclass[a4paper,table]{report}
\input{preamble_ask.tex}
\input{definitions_ask.tex}
\input{myboxes.tex}

\geometry{top=1cm,margin=1cm}

\pagestyle{vangelis}
\everymath{\displaystyle}
\setcounter{chapter}{1}

\begin{document}

\chapter*{Προβλήματα Συνοριακών Τιμών}

\section*{Βασικό Παράδειγμα ομαλού Sturm Liouville}


\begin{mybox3}
\begin{example}
  $ y''(x) + \lambda y(x)=0 $ με $ y(0)=y(L)=0 $.
\end{example}
\end{mybox3}
\begin{solution}
  Διακρίνουμε περιπτώσεις για τις διάφορες τιμές του $ \lambda $.
  \begin{myitemize}
    \item Αν $ \bm{\lambda = 0} $ τότε η εξίσωση γίνεται 
      $ y''(x)=0 \Rightarrow y'(x) = c_{1} \Rightarrow y(x) = c_{1}x + c_{2} $. 
      Εφαρμόζουμε τις συνοριακές συνθήκες και έχουμε:
      \[
        y(0)=0 \Leftrightarrow c_{2}=0 \Rightarrow y(x) = c_{1}x 
      \] 
      Εφαρμόζουμε και τη 2η συνθήκη και έχουμε:
      \[
        y(L)=0 \Leftrightarrow c_{1}L=0 \overset{L>0}{\Leftrightarrow} c_{1}=0 
      \]
      Άρα $ y(x)=0 $ απορρίπτεται ως τετριμμένη λύση.
    \item Αν $ \bm{\lambda < 0} \Rightarrow \lambda = -k^{2} $ με $ k>0 $ και τότε η 
      εξίσωση γίνεται $ y''(x) -k^{2}y(x)=0$ με λύση 
      \[ 
        y(x) = c_{1} \cosh{(kx)} + c_{2} \sinh{(kx)} 
      \]
      Εφαρμόζουμε τις συνοριακές συνθήκες και έχουμε:
      \[
        y(0)=0 \Leftrightarrow c_{1}=0 \Rightarrow y(x) = c_{2} \sinh{(kx)}  
      \] 
      Εφαρμόζουμε και τη 2η συνθήκη και έχουμε:
      \[
        y(L)=0 \Leftrightarrow c_{2} \sinh{(kL)} = 0 \overset{kL>0}{\Leftrightarrow} 
        c_{2}=0 
      \]
      Άρα $ y(x)=0 $ απορρίπτεται ως τετριμμένη λύση.
    \item Αν $ \bm{\lambda > 0} \Rightarrow \lambda = k^{2} $ με $ k>0 $ και τότε η 
      εξίσωση γίνεται $ y''(x) +k^{2}y(x)=0$ με λύση 
      \[ 
        y(x) = c_{1} \cos{(kx)} + c_{2} \sin{(kx)} 
      \]
      Εφαρμόζουμε τις συνοριακές συνθήκες και έχουμε:
      \[
        y(0)=0 \Leftrightarrow c_{1}=0 \Rightarrow y(x) = c_{2} \sin{(kx)}  
      \] 
      Εφαρμόζουμε και τη 2η συνθήκη και έχουμε:
      \[
        y(L)=0 \Leftrightarrow c_{2} \sin{(kL)} = 0 \overset{c_{2} 
        \neq 0}{\Leftrightarrow} \sin{(kL)} =0 \Rightarrow \sin{(kL)} = \sin{(n \pi)}, \;
        n \in \mathbb{N} 
      \]
      οπότε 
      \[
        k_{n} = \frac{n \pi}{L} , \; n \in \mathbb{N} \quad \text{άρα} \quad 
        \lambda _{n} = \frac{n^{2} \pi ^{2}}{L^{2}} 
      \] 
      και $ y_{n}(x) = c_{n} \sin{\left(\frac{n \pi x}{L} \right)}  $ 
  \end{myitemize}
\end{solution}

\begin{mybox3}
\begin{example}
  $ y''(x) + \lambda y(x)=0 $ με $ y'(0)=y'(L)=0 $.
\end{example}
\end{mybox3}
\begin{solution}
  Διακρίνουμε περιπτώσεις για τις διάφορες τιμές του $ \lambda $.
\begin{myitemize}
  \item Αν $ \bm{\lambda = 0} $ τότε η εξίσωση γίνεται 
    $ y''(x)=0 \Rightarrow y'(x) = c_{1} \Rightarrow y(x) = c_{1}x + c_{2} $. 
      Εφαρμόζουμε τις συνοριακές συνθήκες και έχουμε:
      \[
        y'(0)=0 \Leftrightarrow c_{1}=0 \Rightarrow y(x) = c_{2} 
      \] 
      Εφαρμόζουμε και τη 2η συνθήκη και έχουμε:
      \[
        y'(L)=0 \Leftrightarrow c_{1}=0 
      \]
      Άρα $ \lambda_{0}=0 $ είναι ιδιοτιμή και έστω $ y_{0}(x)=c_{0} $ η αποδεκτή, 
      μη τετριμμένη λύση. 
    \item Αν $ \bm{\lambda < 0} \Rightarrow \lambda = -k^{2} $ με $ k>0 $ και τότε η 
      εξίσωση γίνεται $ y''(x) -k^{2}y(x)=0$ με λύση 
      \begin{align*} 
        y(x) &= c_{1} \cosh{(kx)} + c_{2} \sinh{(kx)} \\
        y'(x) &= c_{1}k \sinh{(kx)} + c_{2}k \cosh{(kx)}
      \end{align*}
      Εφαρμόζουμε τις συνοριακές συνθήκες και έχουμε:
      \[
        y'(0)=0 \Leftrightarrow c_{2}k=0 \overset{k>0}{\Leftrightarrow} c_{2} = 0
      \] 
      Άρα $ y(x)= c_{1} \cosh{(kx)} $ και $ y'(x) = c_{1}k \sinh{(kx)} $. 
      Εφαρμόζουμε και τη 2η συνθήκη και έχουμε:
      \[
        y'(L)=0 \Leftrightarrow c_{1}k \sinh{(kL)} = 0 \overset{kL>0}{\Leftrightarrow} 
        c_{1}k=0 \overset{k>0}{\Leftrightarrow} c_{1}=0  
      \]
      Άρα $ y(x)=0 $ απορρίπτεται ως τετριμμένη λύση.
    \item Αν $ \bm{\lambda > 0} \Rightarrow \lambda = k^{2} $ με $ k>0 $ και τότε η 
      εξίσωση γίνεται $ y''(x) +k^{2}y(x)=0$ με λύση 
      \begin{align*} 
        y(x) &= c_{1} \cos{(kx)} + c_{2} \sin{(kx)} \\
        y'(x) &= -c_{1}k \sin{(kx)} + c_{2}k \cos{(kx)} 
      \end{align*}
      Εφαρμόζουμε τις συνοριακές συνθήκες και έχουμε:
      \[
        y'(0)=0 \Leftrightarrow c_{2}k=0 \overset{k>0}{\Leftrightarrow} c_{2}=0 
      \] 
      Άρα $ y(x)= c_{1} \cos{(kx)} $ και $ y'(x) = -c_{1}k \sin{(kx)} $. 
      Εφαρμόζουμε και τη 2η συνθήκη και έχουμε:
      \[
        y'(L)=0 \Leftrightarrow -c_{1}k \sin{(kL)} = 0 \overset{c_{1} 
        \neq 0}{\Leftrightarrow} \sin{(kL)} =0 \Rightarrow \sin{(kL)} = \sin{(n \pi)}, \;
        n \in \mathbb{N} 
      \]
      οπότε 
      \[
        k_{n} = \frac{n \pi}{L} , \; n \in \mathbb{N} \quad \text{άρα} \quad 
        \lambda _{n} = \frac{n^{2} \pi ^{2}}{L^{2}} 
      \] 
      και $ y_{n}(x) = c_{n} \cos{\left(\frac{n \pi x}{L} \right)}  $ 
  \end{myitemize}
\end{solution}

\begin{mybox3}
\begin{example}
  $ y''(x) + \lambda y(x)=0 $ με $ y(0)=y'(L)=0 $.
\end{example}
\end{mybox3}
\begin{solution}
  Διακρίνουμε περιπτώσεις για τις διάφορες τιμές του $ \lambda $.
  \begin{myitemize}
    \item Αν $ \bm{\lambda = 0} $ τότε η εξίσωση γίνεται 
      $ y''(x)=0 \Rightarrow y'(x) = c_{1} \Rightarrow y(x) = c_{1}x + c_{2} $. 
      Εφαρμόζουμε τις συνοριακές συνθήκες και έχουμε:
      \[
        y(0)=0 \Leftrightarrow c_{2}=0 \Rightarrow y(x) = c_{1}x 
      \] 
      Εφαρμόζουμε και τη 2η συνθήκη και έχουμε:
      \[
        y'(L)=0 \Leftrightarrow c_{1}=0 
      \]
      Άρα $ y(x)=0 $ απορρίπτεται ως τετριμμένη λύση.
    \item Αν $ \bm{\lambda < 0} \Rightarrow \lambda = -k^{2} $ με $ k>0 $ και τότε η 
      εξίσωση γίνεται $ y''(x) -k^{2}y(x)=0$ με λύση 
      \begin{align*} 
        y(x) = c_{1} \cosh{(kx)} + c_{2} \sinh{(kx)} \\
        y'(x) = c_{1}k \sinh{(kx)} + c_{2}k \cosh{(kx)} 
      \end{align*}
      Εφαρμόζουμε τις συνοριακές συνθήκες και έχουμε:
      \[
        y(0)=0 \Leftrightarrow c_{1}=0 
      \] 
      Άρα $ y(x) = c_{2} \sinh{(kx)} $ και $ y'(x) = c_{2}k \cosh{(kx)} $ 
      Εφαρμόζουμε και τη 2η συνθήκη και έχουμε:
      \[
        y'(L)=0 \Leftrightarrow c_{2}k \sinh{(kL)} = 0 \overset{kL>0}{\Leftrightarrow} 
        c_{2}k=0 \overset{k>0}{\Leftrightarrow} c_{2}=0
      \]
      Άρα $ y(x)=0 $ απορρίπτεται ως τετριμμένη λύση.
    \item Αν $ \bm{\lambda > 0} \Rightarrow \lambda = k^{2} $ με $ k>0 $ και τότε η 
      εξίσωση γίνεται $ y''(x) +k^{2}y(x)=0$ με λύση 
      \begin{align*} 
        y(x) &= c_{1} \cos{(kx)} + c_{2} \sin{(kx)} \\
        y'(x) &= -c_{1} \sin{(kx)} + c_{2} \cos{(kx)} 
      \end{align*}
      Εφαρμόζουμε τις συνοριακές συνθήκες και έχουμε:
      \[
        y(0)=0 \Leftrightarrow c_{1}=0 
      \] 
      Άρα $ y(x)= c_{2} \sin{(kx)} $ και $ y'(x) = c_{2} \cos{(kx)} $
      Εφαρμόζουμε και τη 2η συνθήκη και έχουμε:
      \[
        y'(L)=0 \Leftrightarrow c_{2} \cos{(kL)} = 0 \overset{c_{2} 
          \neq 0}{\Leftrightarrow} \cos{(kL)} =0 \Rightarrow \cos{(kL)} = 
          \cos{\left(\frac{(2n-1) \pi}{2}\right)}, \; n \in \mathbb{N} 
      \]
      οπότε 
      \[
        k_{n} = \frac{(2n-1) \pi}{2L} , \; n \in \mathbb{N} \quad \text{άρα} \quad 
        \lambda _{n} = \frac{(2n-1)^{2} \pi ^{2}}{4L^{2}} 
      \] 
      και $ y_{n}(x) = c_{n} \sin{\left(\frac{(2n-1) \pi x}{2L} \right)}  $ 
  \end{myitemize}
\end{solution}

\begin{mybox3}
\begin{example}
  $ y''(x) + \lambda y(x)=0 $ με $ y'(0)=y(L)=0 $.
\end{example}
\end{mybox3}
\begin{solution}
  Διακρίνουμε περιπτώσεις για τις διάφορες τιμές του $ \lambda $.
  \begin{myitemize}
    \item Αν $ \bm{\lambda = 0} $ τότε η εξίσωση γίνεται 
      $ y''(x)=0 \Rightarrow y'(x) = c_{1} \Rightarrow y(x) = c_{1}x + c_{2} $. 
      Εφαρμόζουμε τις συνοριακές συνθήκες και έχουμε:
      \[
        y'(0)=0 \Leftrightarrow c_{1}=0 \Rightarrow y(x) = c_{2} 
      \] 
      Εφαρμόζουμε και τη 2η συνθήκη και έχουμε:
      \[
        y(L)=0 \Leftrightarrow c_{2}=0 
      \]
      Άρα $ y(x)=0 $ απορρίπτεται ως τετριμμένη λύση.
    \item Αν $ \bm{\lambda < 0} \Rightarrow \lambda = -k^{2} $ με $ k>0 $ και τότε η 
      εξίσωση γίνεται $ y''(x) -k^{2}y(x)=0$ με λύση 
      \begin{align*} 
        y(x) = c_{1} \cosh{(kx)} + c_{2} \sinh{(kx)} \\
        y'(x) = c_{1}k \sinh{(kx)} + c_{2}k \cosh{(kx)} 
      \end{align*}
      Εφαρμόζουμε τις συνοριακές συνθήκες και έχουμε:
      \[
        y'(0)=0 \Leftrightarrow c_{2}k=0 \overset{k>0}{\Leftrightarrow} c_{2}=0
      \] 
      Άρα $ y(x) = c_{1} \cosh{(kx)} $ και $ y'(x) = c_{1}k \sinh{(kx)} $ 
      Εφαρμόζουμε και τη 2η συνθήκη και έχουμε:
      \[
        y(L)=0 \Leftrightarrow c_{1}k \cosh{(kL)} = 0 \Leftrightarrow c_{1}k=0 
        \overset{k>0}{\Leftrightarrow} c_{1}=0
      \]
      Άρα $ y(x)=0 $ απορρίπτεται ως τετριμμένη λύση.
    \item Αν $ \bm{\lambda > 0} \Rightarrow \lambda = k^{2} $ με $ k>0 $ και τότε η 
      εξίσωση γίνεται $ y''(x) +k^{2}y(x)=0$ με λύση 
      \begin{align*} 
        y(x) &= c_{1} \cos{(kx)} + c_{2} \sin{(kx)} \\
        y'(x) &= -c_{1} \sin{(kx)} + c_{2} \cos{(kx)} 
      \end{align*}
      Εφαρμόζουμε τις συνοριακές συνθήκες και έχουμε:
      \[
        y'(0)=0 \Leftrightarrow c_{2}k =0 \overset{k>0}{\Leftrightarrow} c_{2}=0
      \] 
      Άρα $ y(x)= c_{1} \cos{(kx)} $ και $ y'(x) = -c_{1}k \sin{(kx)} $
      Εφαρμόζουμε και τη 2η συνθήκη και έχουμε:
      \[
        y(L)=0 \Leftrightarrow c_{1} \cos{(kL)} = 0 \overset{c_{1} 
          \neq 0}{\Leftrightarrow} \cos{(kL)} =0 \Rightarrow \cos{(kL)} = 
          \cos{\left(\frac{(2n-1) \pi}{2}\right)}, \; n \in \mathbb{N} 
      \]
      οπότε 
      \[
        k_{n} = \frac{(2n-1) \pi}{2L} , \; n \in \mathbb{N} \quad \text{άρα} \quad 
        \lambda _{n} = \frac{(2n-1)^{2} \pi ^{2}}{4L^{2}} 
      \] 
      και $ y_{n}(x) = c_{n} \cos{\left(\frac{(2n-1) \pi x}{2L} \right)}  $ 
  \end{myitemize}
\end{solution}

\section*{Περιοδικές Συνοριακές Συνθήκες}


\begin{mybox3}
\begin{example}
  $ y''(x) + \lambda y(x)=0 $ με $ y(0)=y(L) $ και $ y'(0)=y'(L) $.
\end{example}
\end{mybox3}
\begin{solution}
  Διακρίνουμε περιπτώσεις για τις διάφορες τιμές του $ \lambda $.
  \begin{myitemize}
    \item Αν $ \bm{\lambda = 0} $ τότε η εξίσωση γίνεται 
      $ y''(x)=0 \Rightarrow y'(x) = c_{1} \Rightarrow y(x) = c_{1}x + c_{2} $ και 
      $ y'(x) = c_{1} $. 
      Εφαρμόζουμε τις συνοριακές συνθήκες και έχουμε:
      \[
        y(0)=y(L) \Leftrightarrow c_{2}= c_{1}L + c_{2} \Rightarrow c_{1}L=0
        \overset{L>0}{\Leftrightarrow} c_{1} = 0
      \] 
      Άρα $ y(x) = c_{2} $ και $ y'(x) = 0 $
      Εφαρμόζουμε και τη 2η συνθήκη και έχουμε:
      \[
        y'(0)=y'(L) \Leftrightarrow 0 = 0
      \]
      Άρα $ \lambda_{0}=0 $ είναι ιδιοτιμή και έστω $ y_{0}(x)=c_{0} $ η αποδεκτή, 
      μη τετριμμένη λύση, με αντίστοιχη ιδιοσύναρτηση την $y_{0}=1 $. 
    \item Αν $ \bm{\lambda < 0} \Rightarrow \lambda = -k^{2} $ με $ k>0 $ και τότε η 
      εξίσωση γίνεται $ y''(x) -k^{2}y(x)=0$ με λύση 
      \begin{align*} 
        y(x) = c_{1} \cosh{(kx)} + c_{2} \sinh{(kx)} \\
        y'(x) = c_{1}k \sinh{(kx)} + c_{2}k \cosh{(kx)} 
      \end{align*}
      Εφαρμόζουμε τις συνοριακές συνθήκες και έχουμε:
      \[
        y(0)=y(L) \Leftrightarrow  c_{1} = c_{1} \cosh{(kL)} + c_{2} \sinh{(kL)} 
        \Leftrightarrow c_{1}(\cosh{(kL)} -1) + c_{2} \sinh{(kL)} = 0
      \] 
      Εφαρμόζουμε και τη 2η συνθήκη και έχουμε:
      \begin{gather*}
        y'(0)=y'(L) \Leftrightarrow c_{2}k = c_{1}k \sinh{(kL)} + c_{2}k \cosh{(kL)} 
        \overset{k > 0}{\Leftrightarrow} 
         c_{1} \sinh{(kL)} + c_{2} (\cosh{(kL)} -1) = 0
      \end{gather*}
      Οπότε, προκύπτει το ομογενές σύστημα 
      \[
        \left.
          \begin{matrix}
            c_{1}(\cosh{(kL)} -1) + c_{2} \sinh{(kL)} = 0 \\
             c_{1} \sinh{(kL)} + c_{2} (\cosh{(kL)} -1) = 0
          \end{matrix} 
        \right\} \Leftrightarrow 
        \begin{pmatrix}
          \cosh{(kL)} -1 & \sinh{(kL)} \\
          \sinh{(kL)} & (\cosh{(kL)} -1)
        \end{pmatrix} \cdot 
        \begin{pmatrix*}[r]
          c_{1} \\ c_{2} 
        \end{pmatrix*} = 
        \begin{pmatrix*}[r]
          0 \\ 0
        \end{pmatrix*}
      \]
      το οποίο θα έχει μη μηδενική λύση, αν η ορίζουσά του είναι μηδέν, δηλαδή
      \[
        \begin{vmatrix}
          \cosh{(kL)} -1 & \sinh{(kL)} \\
          \sinh{(kL)} & (\cosh{(kL)} -1)
        \end{vmatrix} = 0 \Leftrightarrow  (\cosh{(kL)} -1)^{2}- \sinh^{2}{(kL)} = 0
        \Leftrightarrow 2(1- \cosh{(kL)}) = 0 \Leftrightarrow \cosh{(kL)} = 1
      \] 
      Όμως $ \cosh{(kL)} = 1 \Leftrightarrow kL=0 $ το οποίο είναι άτοπο, γιατί $ k,L>0 $
      Άρα μοναδική λύση του συστήματος η $ y(x)=0 $, απορρίπτεται ως τετριμμένη λύση.
    \item Αν $ \bm{\lambda > 0} \Rightarrow \lambda = k^{2} $ με $ k>0 $ και τότε η 
      εξίσωση γίνεται $ y''(x) +k^{2}y(x)=0$ με λύση 
      \begin{align*} 
        y(x) &= c_{1} \cos{(kx)} + c_{2} \sin{(kx)} \\
        y'(x) &= -c_{1}k \sin{(kx)} + c_{2}k \cos{(kx)} 
      \end{align*}
      Εφαρμόζουμε τις συνοριακές συνθήκες και έχουμε:
      \[
        \left.
          \begin{matrix}
            y(0)=y(L) \\
            y'(0)=y'(L)
          \end{matrix} 
        \right\} \Leftrightarrow 
        \left.
          \begin{matrix}
            c_{1} = c_{1} \cos{(kL)} + c_{2} \sin{(kL)} \\
            c_{2} k = - c_{1}k \sin{(kL)} + c_{2} k\cos{(kL)}
          \end{matrix} 
        \right\} \Leftrightarrow 
        \left.
          \begin{matrix}
            c_{1} (\cos{(kL)} -1) + c_{2} \sin{(kL)} = 0 \\
            - c_{1} k \sin{(kL)} + c_{2} k (\cos{(-kL)} - 1) = 0 
          \end{matrix} 
        \right\} \overset{k>0}{\Leftrightarrow} 
      \]
      \[
        \left.
          \begin{matrix}
            c_{1} (\cos{(kL)} -1) + c_{2} \sin{(kL)} = 0 \\
            - c_{1} \sin{(kL)} + c_{2} (\cos{(-kL)} - 1) = 0 
          \end{matrix} 
        \right\} \Leftrightarrow 
        \begin{pmatrix}
          \cos{(kL)} -1 & \sin{(kL)} \\
          - \sin{(kL)} & \cos{(-kL)} - 1 
        \end{pmatrix} \cdot
        \begin{pmatrix}
          c_{1} \\ c_{2} 
        \end{pmatrix} = 
        \begin{pmatrix}
          0 \\ 0 
        \end{pmatrix}
      \]
      το οποίο θα έχει μη μηδενική λύση, αν η ορίζουσά του είναι μηδέν, δηλαδή
      \[
        \begin{vmatrix}
          \cos{(kL)} -1 & \sin{(kL)} \\
          - \sin{(kL)} & \cos{(-kL)} - 1 
        \end{vmatrix} = 0 \Leftrightarrow  (\cos{(kL)} -1)^{2}+ \sin^{2}{(kL)} = 0
        \Leftrightarrow 2(1- \cos{(kL)}) = 0 \Leftrightarrow \cos{(kL)} = 1
      \] 
      Όμως $ \cos{(kL)} = 1 \Leftrightarrow kL= 2n \pi, \forall n \in \mathbb{N} $, άρα
      \[
        k_{n} = \frac{2n \pi}{L} , \; n \in \mathbb{N} \quad \text{άρα} \quad 
        \lambda _{n} = \frac{4 n^{2} \pi ^{2}}{L^{2}}, \; n \in \mathbb{N}
      \]
      και ιδιοσυναρτήσεις είναι οι
      $ y_{n}(x) = c_{n} \cos{\left(\frac{2n \pi}{L} x\right)} + d_{n} 
      \sin{\left(\frac{2n \pi}{L} x\right)} $ για κάθε $ c_{1}, c_{2} \in \mathbb{R} $ 
      και για κάθε $ n = 1,2,3,\ldots$.
  \end{myitemize}
\end{solution}

\begin{mybox3}
\begin{example}
  $ y''(x) + \lambda y(x)=0 $ με $ y(-L)=y(L) $ και $ y'(-L)=y'(L) $.
\end{example}
\end{mybox3}
\begin{solution}
  Διακρίνουμε περιπτώσεις για τις διάφορες τιμές του $ \lambda $.
  \begin{myitemize}
    \item Αν $ \bm{\lambda = 0} $ τότε η εξίσωση γίνεται 
      $ y''(x)=0 \Rightarrow y'(x) = c_{1} \Rightarrow y(x) = c_{1}x + c_{2} $. 
      Εφαρμόζουμε τις συνοριακές συνθήκες και έχουμε:
      \[
        y(-L)=y(L) \Leftrightarrow -c_{1}L+ c_{2} = c_{1}L + c_{2} \Rightarrow 2c_{1}L=0
        \overset{L>0}{\Leftrightarrow} c_{1} = 0
      \] 
      Εφαρμόζουμε και τη 2η συνθήκη και έχουμε:
      \[
        y'(-L)=y'(L) \Leftrightarrow c_{2} = c_{2}
      \]
      Άρα $ \lambda_{0}=0 $ είναι ιδιοτιμή και έστω $ y_{0}(x)=c_{0} $ η αποδεκτή, 
      μη τετριμμένη λύση, με αντίστοιχη ιδιοσύναρτηση την $y_{0}=1 $. 
    \item Αν $ \bm{\lambda < 0} \Rightarrow \lambda = -k^{2} $ με $ k>0 $ και τότε η 
      εξίσωση γίνεται $ y''(x) -k^{2}y(x)=0$ με λύση 
      \begin{align*} 
        y(x) = c_{1} \cosh{(kx)} + c_{2} \sinh{(kx)} \\
        y'(x) = c_{1}k \sinh{(kx)} + c_{2}k \cosh{(kx)} 
      \end{align*}
      Εφαρμόζουμε τις συνοριακές συνθήκες και έχουμε:
      \[
        y(-L)=y(L) \Leftrightarrow  c_{1} \cosh{(k(-L))} + c_{2} \sinh{(k(-L))} = 
        c_{1} \cosh{(kL)} + c_{2} \sinh{(kL)} \Leftrightarrow 2 c_{2} \sinh{(kL)} = 0 
        \overset{kL>0}{\Leftrightarrow} c_{2} = 0
      \] 
      Άρα $ y(x) = c_{1} \cosh{(kL)} $ και $ y'(x) = c_{1}k \sinh{(kL)} $.  
      Εφαρμόζουμε και τη 2η συνθήκη και έχουμε:
      \[
        y'(-L)=y'(L) \Leftrightarrow c_{1}k \sinh{(kL)} = c_{1}k \sinh{(k(-L))}
        \Leftrightarrow 2 c_{1}k \sinh{(kL)} = 0 \overset{k>0}{\Leftrightarrow} c_{1}=0
      \]
      Άρα $ y(x)=0 $ απορρίπτεται ως τετριμμένη λύση.
    \item Αν $ \bm{\lambda > 0} \Rightarrow \lambda = k^{2} $ με $ k>0 $ και τότε η 
      εξίσωση γίνεται $ y''(x) +k^{2}y(x)=0$ με λύση 
      \begin{align*} 
        y(x) &= c_{1} \cos{(kx)} + c_{2} \sin{(kx)} \\
        y'(x) &= -c_{1}k \sin{(kx)} + c_{2}k \cos{(kx)} 
      \end{align*}
      Εφαρμόζουμε τις συνοριακές συνθήκες και έχουμε:
      \[
        \left.
          \begin{matrix}
            y(-L)=y(L) \\
            y'(-L)=y'(L)
          \end{matrix} 
        \right\} \Leftrightarrow 
        \left.
          \begin{matrix}
            c_{1} \cos{(k(-L))} + c_{2} \sin{(k(-L))} = c_{1} \cos{(kL)} + 
            c_{2} \sin{(kL)} \\
            - c_{1} k \sin{(k(-L))} + c_{2} k \cos{(k(-L))} = - c_{1}k \sin{(kL)} + 
            c_{2} k\cos{(kL)}
          \end{matrix} 
        \right\} \Leftrightarrow 
      \]
      \[
        \left.
          \begin{matrix}
            c_{1} \cos{(-kL)} + c_{2} \sin{(-kL)} = c_{1} \cos{(kL)} + c_{2} 
            \sin{(kL)} \\
            - c_{1} k \sin{(-kL)} + c_{2} k \cos{(-kL)} = - c_{1}k \sin{(kL)} + 
            c_{2} k\cos{(kL)}
          \end{matrix} 
        \right\} \Leftrightarrow 
      \]
      \[
        \left.
          \begin{matrix}
            c_{1} \cos{(kL)} - c_{2} \sin{(kL)} = c_{1} \cos{(kL)} + c_{2} \sin{(kL)} \\
            c_{1} k \sin{kL} + c_{2} k \cos{kL} = - c_{1}k \sin{(kL)} + c_{2} k\cos{(kL)}
          \end{matrix}
        \right\} \Leftrightarrow 
      \]
      \[
        \left.
          \begin{matrix}
            2 c_{2} \sin{(kL)} = 0 \\
            2 c_{1} k \sin{(kL)} = 0
          \end{matrix} 
        \right\} \Leftrightarrow 
        \left.
          \begin{matrix}
            c_{2} = 0 \quad \text{ή} \quad \sin{(kL)} = 0 \\
            c_{1} = 0 \quad \text{ή} \quad \sin{(kL)} = 0
          \end{matrix} 
        \right\} \Leftrightarrow 
      \]
      \[
        \left.
          \begin{matrix}
            c_{2} = 0 \\
            c_{1} = 0
          \end{matrix} 
        \right\} \; \text{ή} \; 
        \left.
          \begin{matrix}
            c_{2} = 0 \\
            \sin{(kL)} = 0 
          \end{matrix} 
        \right\} \; \text{ή} \; 
        \left.
          \begin{matrix}
            \sin{(kL)} = 0 \\
            c_{1} = 0 
          \end{matrix} 
        \right\} \; \text{ή} \; 
        \left.
          \begin{matrix}
            \sin{(kL)} = 0 \\
            \sin{(kL)} = 0 
          \end{matrix} 
        \right\} \Leftrightarrow 
      \]
      \[
        \text{απορ.} \; \text{ή} \; 
        \left.
          \begin{matrix}
            k_{n} = \frac{n \pi}{L} \\
            y_{n}(x) = c_{n} \cos{\left(\frac{n \pi}{L}\right)} 
          \end{matrix} 
        \right\} \; \text{ή} \; 
        \left.
          \begin{matrix}
            k_{n} = \frac{n \pi}{L} \\
            y_{n}(x) = c_{n} \sin{\left(\frac{n \pi}{L}\right)} 
          \end{matrix} 
        \right\} \; \text{ή} \;
        \left.
          \begin{matrix}
            k_{n} = \frac{n \pi}{L} \\
            y_{n}(x) = c_{1} \cos{\left(\frac{n \pi}{L} x\right)} + c_{2} 
            \sin{\left(\frac{n \pi}{L} x\right)} 
          \end{matrix}
        \right\} 
      \]
      οπότε, το πρόβλημα έχει ιδιοτιμές $ \lambda _{n} = \frac{n^{2} \pi ^{2}}{L^{2}} $
      και ιδιοσυναρτήσεις τις 
      $ y_{n}(x) = c_{1} \cos{\left(\frac{n \pi}{L} x\right)} + c_{2} 
      \sin{\left(\frac{n \pi}{L} x\right)} $ για κάθε $ c_{1}, c_{2} \in \mathbb{R} $ 
      και για κάθε $ n = 1,2,3,\ldots$.
  \end{myitemize}
\end{solution}







\end{document}


