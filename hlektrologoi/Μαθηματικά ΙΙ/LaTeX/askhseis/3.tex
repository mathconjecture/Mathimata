\input{preamble.tex}
\input{definitions_ask.tex}

\pagestyle{askhseis}
\everymath{\displaystyle}
\renewcommand{\vec}{\mathbf}

\begin{document}

\begin{center}
  \minibox{\large\bf \textcolor{Col1}{Ασκήσεις Κλίση, Απόκλιση, Στροβιλισμός}}
\end{center}

\vspace{\baselineskip}

\begin{enumerate}

\item Να αποδείξετε τις παρακάτω σχέσεις για τους διαφορικούς τελεστές

\begin{enumerate}[i)]
\item $\curl(\grad f)=0$
\item $\div(\curl F)=0$
\end{enumerate}

\item Δίνεται το βαθμωτό πεδίο 
  $f(x,y,z)=3y^4z^2\vec{i}+4x^3z^2\vec{j}-3x^2y^2\vec{k}$. Να δείξετε ότι το 
  πεδίο είναι σωληνοειδές.

\item Δίνεται το διανυσματικό πεδίο 
  $\boldsymbol{F}(x,y,z)=(x^4+yz^3)\vec{i}+x^3\vec{j}+(-1+x^2+y^2)\vec{k}$. 
  Να υπολογιστούν:
\begin{enumerate}[i)]
\item $\div F$
\item $\grad(\div F)$
\item $\curl(\curl F)$

\hfill Απ: \begin{tabular}{l}
  $\mathrm{i)}\; \div F=4x^3$ \\
  $\mathrm{ii)}\; \grad(\div F)=12x^2\vec{i}+0\vec{j}+0\vec{k}$ \\
  $\mathrm{iii)}\; \curl(\curl F)=(-6yz)\vec{i}+(-6x)\vec{j}+(-2y^2-2x^2)\vec{k}$
\end{tabular}

\end{enumerate}

\item Δίνεται το διανυσματικό πεδίο $\boldsymbol{F}=x\vec{i}+y\vec{j}+z\vec{k}$. 
\begin{enumerate}[i)]
\item Να δείξετε ότι το πεδίο είναι αστρόβιλο.
\item Να βρειτε το $\div F$.
\item Να βρειτε το $\grad(\div F)$.
\end{enumerate}

\hfill Απ: \begin{tabular}{l}
  $\mathrm{ii)}\; \div F=3$ \\
  $\mathrm{iii)}\; \grad(\div F)=0$

\end{tabular}

\item Θεωρούμε το διανυσματικό πεδίο
\[
\boldsymbol{F}(x,y,z)=e^{2x}\vec{i}+f(x,z)\vec{j}+e^z\vec{k}
\]
\begin{enumerate}[i)]
\item Να βρεθεί η συνάρτηση $f(x,z)$ ώστε το πεδίο να είναι συντηρητικό ($\curl F=0$).
\item Να βρείτε τη συνάρτηση δυναμικού.

\hfill Απ: \begin{tabular}{l}
  $\mathrm{i)}\; f(x,z)=c$ (σταθ.) \\
  $\mathrm{ii)}\;f(x,y,z)=\frac{e^{2x}}{2}+cy+e^z-\frac{3}{2}$
\end{tabular}

\end{enumerate}
\end{enumerate}

\end{document}
