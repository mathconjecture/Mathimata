\documentclass[a4paper,table]{report}
\input{preamble.tex}
\input{definitions_ask.tex}

\pagestyle{askhseis}

\renewcommand{\vec}{\mathbf}

\begin{document}

\begin{center}
  \minibox{\large \bfseries \textcolor{Col1}{Ασκήσεις στα Ακρότατα}}
\end{center}

\vspace{\baselineskip}

\section*{Τοπικά Ακρότατα}

\begin{enumerate}

\item Να βρεθεί η ελάχιστη απόσταση του επιπέδου με εξίσωση $ x+y+z=4 $, από την 
  αρχή των αξόνων.

  \hfill Απ: $ d_{\min}(4/3,4/3) = 4\frac{\sqrt{3}}{3} $  

\item Να βρεθεί η ελάχιστη απόσταση του επιπέδου με εξίσωση $ 3x+2y+z=6 $, από την 
  αρχή των αξόνων.

  \hfill Απ: $ d_{\min}(9/7,6/7) = 3\frac{\sqrt{14}}{7} $  

\item Να βρεθεί η ελάχιστη απόσταση του σημείου $ P(2,-1,1) $ από το επίπεδο με 
  εξίσωση $ x+y-z=2 $. 

  \hfill Απ: $ d_{min}(8/3,-1/3) = 2 /\sqrt{3} $ 

% \item Να βρεθεί η ελάχιστη απόσταση του σημείου $ P(-6,4,0) $ από τον κώνο με 
%   εξίσωση $ z = \sqrt{x^{2}+y^{2}} $. 

%   \hfill Απ: $ d_{min}(-3,2) = \sqrt{26} $ 

\end{enumerate}


\section*{Ακρότατα Υπό Συνθήκη}

\begin{enumerate}

    %Thomas 12th 14.8 ex.1 
  \item Να βρείτε τα ακρότατα της συνάρτησης $ f(x,y) = xy $ πάνω στην έλλειψη 
    $ x^{2}+2y^{2}=1 $.

    \hfill Απ: 
    \begin{tabular}{l}
      max: $ f(\sqrt{2} /2, 1/2) = f(- \sqrt{2} /2, -1/2) = \frac{\sqrt{2}}{2} $ \\
      min $ f(\sqrt{2} /2, -1/2) = f(- \sqrt{2} /2, 1/2) = -\frac{\sqrt{2}}{2} $ \\
    \end{tabular}

    %Thomas 12th 14.8 ex.14 
  \item Να βρείτε τα ακρότατα της συνάρτησης $ f(x,y) = 3x-y+6 $ υπό τον περιορισμό 
    $ x^{2}+y^{2}=4 $.

    \hfill Απ:  
    \begin{tabular}{l}
      max: $ f(\frac{6}{\sqrt{10}} , - \frac{2}{\sqrt{10}}) = 2 \sqrt{10} +6 $ \\
      min $ f(-\frac{6}{\sqrt{10}} , + \frac{2}{\sqrt{10}}) = -2 \sqrt{10} +6 $ \\
    \end{tabular}

    %%Thomas 12th 14.8 ex.15 
  %\item Η θερμοκρασία σε κάθε σημείο μιας μεταλλικής πλάκας δίνεται από τη σχέση
    %\[
    %  T(x,y) = 4x^{2}-4xy+y^2 
    %\]
    %Ένα μυρμήγκι, που βρίσκεται πάνω στην πλάκα, περπατά στην περιφέρεια κύκλου, 
    %με κέντρο την αρχή των αξόνων και ακτίνας 5. Να υπολογίσετε την μέγιστη και την 
    %ελάχιστη θερμοκρασία που θα συναντήσει το μυρμήγκι κατά τη διαδρομή του.

    %\hfill Απ:  
    %\begin{tabular}{l}
    %  max $ f(2 \sqrt{5} , - \sqrt{5}) = f(-2 \sqrt{5} , \sqrt{5}) = 125 $ \\
    %  min $ f(\sqrt{5} , 2 \sqrt{5}) = f(- \sqrt{5} , - 2 \sqrt{5}) = 0 $ 
    %\end{tabular}

  % \item Να υπολογιστούν τα τοπικά ακρότατα της συνάρτησης $ f(x,y,z) = x^{2}+y^{2}+z^{2}
  %   $ που ικανοποιούν τον περιορισμό $ x+y+z+1=0 $.
  %   \hfill Απ: min: $ (-1/3,-1/3,-1/3) $ 

  % \item Να υπολογιστούν τα τοπικά ακρότατα της συνάρτησης 
  %   $ f(x,y,z) = x^{2}+y^{2}+z^{2}-2x-2y-z+ \frac{5}{4} $ που ικανοποιούν τον 
  %   περιορισμό $ x^{2}+y^{2}-z=0  $.
  %   \hfill Απ: min: $ (1/ \sqrt[3]{4} , 1/ \sqrt[3]{4}) $ 

  % \item Να υπολογιστούν τα τοπικά ακρότατα της συνάρτησης 
  %   $f(x,y,z)=xyz$ που ικανοποιούν την εξίσωση $x+y+z-1=0$.  
  %   \hfill Απ: max: $ (1/3,1/3,1/3) $ 

  % \item Να υπολογιστούν τα τοπικά ακρότατα της συνάρτησης 
  %   $ f(x,y,z) = x^{2}+y^{2}+z^{2} $, που ικανοποιούν τους περιορισμούς 
  %   $ zx+zy=-2 $ και $ xy=1 $.
  %   \hfill Απ: min: $ (-1,-1,1) $, max: $ (1,1,-1) $ 
\end{enumerate}



\section*{Απόλυτα Ακρότατα (Σε κλειστή και φραγμένη περιοχή)}

\begin{enumerate}
    %Thomas 12th 14.7 ex.41 
  % \item Μια λεπτή, επίπεδη, μεταλλική πλάκα, σε σχήμα κυκλικού δίσκου 
  %   $ x^{2}+y^2 \leq 1 $, θερμαίνεται, έτσι ώστε η θερμοκρασία σε κάθε σημείο της να 
  %   δίνεται από τη συνάρτηση 
  %   $ T(x,y) = 2y^{2}+x^{2}-x $.
  %   Να βρείτε τη θερμοκρασία στα πιο θερμά και πιο ψυχρά σημεία αυτής της πλάκας.

  %   \hfill Απ:  
  %   \begin{tabular}{l}
  %     max $ T(-1/2, \sqrt{3} /2) = T(-1/2, - \sqrt{3} /2) = 9/4 $ \\
  %     min $ f(1/2,0) = -1/4 $ 
  %   \end{tabular}

  \item Να υπολογιστούν τα ολικά ακρότατα της συνάρτησης $ f(x,y) = \sqrt{x^{2}+y^{2}} +
    y^{2}-1 $, στον κυκλικό δίσκο $ x^{2}+y^{2} \leq 9 $. 

    \hfill Απ: 
    \begin{tabular}{l}
      max: $ (0,3), (0,-3) $ \\
      min: $ (3,0), (-3,0) $, 
    \end{tabular}

    %Thomas 12th 14.7 ex.32 
  \item Να υπολογιστούν τα ακρότατα της συνάρτησης $ f(x,y) = x^{2}-xy+y^{2}+1 $ 
    στην κλειστή, τριγωνική περιοχή του επιπέδου, που περικλείεται από τις καμπύλες 
    $ x=0 $, $ y=4 $ και $ y=x $.

    \hfill Απ:  
    \begin{tabular}{l}
      max $ f(0,4) f(4,4) = 17 $ \\
      min $ f(0,0) = 1 $ 
    \end{tabular}

    %Thomas 12th 14.7 ex.31 
  \item Να υπολογιστούν τα ακρότατα της συνάρτησης $ f(x,y) = 2x^{2}-4x+y^{2}-4y+1 $ 
    στην κλειστή, τριγωνική περιοχή του επιπέδου, που περικλείεται από τις καμπύλες 
    $ x=0 $, $ y=2 $ και $ y=2x $.

    \hfill Απ:  
    \begin{tabular}{l}
      max $ f(0,0) = 1 $ \\
      min $ f(1,2) = -5 $ 
    \end{tabular}
\end{enumerate}



\end{document}


