\documentclass[a4paper,table]{report}
\input{preamble_themata}
\newcommand{\vect}[2]{(#1_1,\ldots, #1_#2)}
%%%%%%% nesting newcommands $$$$$$$$$$$$$$$$$$$
\newcommand{\function}[1]{\newcommand{\nvec}[2]{#1(##1_1,\ldots, ##1_##2)}}

\newcommand{\linode}[2]{#1_n(x)#2^{(n)}+#1_{n-1}(x)#2^{(n-1)}+\cdots +#1_0(x)#2=g(x)}

\newcommand{\vecoffun}[3]{#1_0(#2),\ldots ,#1_#3(#2)}

\newcommand{\mysum}[1]{\sum_{n=#1}^{\infty}

\input{myboxes}

\everymath{\displaystyle}
\pagestyle{vangelis}


\begin{document}

\begin{mybox3}
  \begin{thema}
    Έστω η επιφάνεια $ S \colon F(x,y,z)=0 $ και έστω ότι η γενική εξίσωση του 
    εφαπτόμενου επιπέδου στο τυχαίο σημείο $ (x_{0}, y_{0}, z_{0}) $ της επιφάνειας έχει 
    τη μορφή:
    \[
      (x_{0}+ z_{0})(x- x_{0}) - (y_{0}+ z_{0})(y- y_{0}) + (x_{0}- y_{0})(z- z_{0})=0 
    \] 
    Αν η επιφάνεια διέρχεται από τη σημείο $ \textcolor{Col1}{(1,2,3)} $, τότε η εξίσωσή 
    της είναι:
  \end{thema}
\end{mybox3}
\begin{solution}
  \item []
    \begin{mybox1}
      \vspace{0.5\baselineskip}
      \textcolor{Col1}{\textbf{Θυμάμαι:}} 
      Αν $ F(x,y,z)=c $ είναι επιφάνεια, τότε η εξίσωση του εφαπτόμενου επιπέδου 
      στο σημείο $ (\textcolor{Col1}{x_{0}},
      \textcolor{Col1}{y_{0}}, \textcolor{Col1}{z_{0}}) $ της επιφάνειας είναι: 
      $ {F_{x}(x_{0}, y_{0}, z_{0})}(x- \textcolor{Col1}{x_{0}}) +
      {F_{y}(x_{0}, y_{0}, z_{0})}(y- \textcolor{Col1}{y_{0}}) + 
      {F_{z}(x_{0}, y_{0}, z_{0})}(z- \textcolor{Col1}{z_{0}}) = 0 $.
    \end{mybox1}
    Από την εξίσωση του εφαπτόμενου επιπέδου που μας δίνεται και από την προηγούμενη
    παρατήρηση, συμπεραίνουμε ότι:
    \begin{align*}
      F_{x}(x,y,z) &= x+z = P(x,y,z) \\
      F_{y}(x,y,z) &= -y-z = Q(x,y,z) \\
      F_{z}(x,y,z) &= x-y = R(x,y,z)
    \end{align*} 
    Οι συναρτήσεις, $ P,Q $ και $ R $ είναι συνεχείς με συνεχείς μερικές παραγώγους, ως
    πολυωνυμικές συναρτήσεις και επίσης ισχύει:
    \[
      P_{y} = 0 = Q_{x} \quad \text{και} \quad Q_{z}=-1=R_{y} \quad \text{και} \quad
      P_{z} = 1 = R_{x}
     \] 
    Οπότε μπορούμε να υπολογίσουμε τη συνάρτηση $ F(x,y,z) $ από τον γνωστό τύπο:
    \begin{align*}
      F(x,y,z) &= \int _{x_{0}}^{x} P(t,y,z) \,{dt} + \int _{y_{0}}^{y} Q(x_{0},t,z) 
      \,{dt} + \int _{z_{0}}^{z} R(x_{0}, y_{0}, t) \,{dt} \\
               &= \int _{\textcolor{Col1}{1}}^{x} (t+z) \,{dt} + \int
               _{\textcolor{Col1}{2}}^{y} (-t-z) \,{dt} + 
               \int _{\textcolor{Col1}{3}}^{z} (1-2) \,{dt} \\
               &= \left[\frac{t^2}{2} + zt\right]_{1}^{x} + \left[- \frac{t^2}{2} -zt\right]_{2}^{y} +
               \left[-t\right]_{3}^{z} \\
               &= \frac{x^{2}}{2} + zx - \frac{1}{2} -\cancel{z} - \frac{y^{2}}{2} - zy +
               2 + \cancel{2z}
               -\cancel{z} +3 \\
               &= \frac{x^{2}}{2} +zx - \frac{y^{2}}{2} - zy + \frac{9}{2}
    \end{align*} 
  \end{solution}



\end{document}
