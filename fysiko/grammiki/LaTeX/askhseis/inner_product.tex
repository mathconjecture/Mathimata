\documentclass[a4paper,table]{report}
\documentclass[a4paper,12pt]{article}
\usepackage{etex}
%%%%%%%%%%%%%%%%%%%%%%%%%%%%%%%%%%%%%%
% Babel language package
\usepackage[english,greek]{babel}
% Inputenc font encoding
\usepackage[utf8]{inputenc}
%%%%%%%%%%%%%%%%%%%%%%%%%%%%%%%%%%%%%%

%%%%% math packages %%%%%%%%%%%%%%%%%%
\usepackage{amsmath}
\usepackage{amssymb}
\usepackage{amsfonts}
\usepackage{amsthm}
\usepackage{proof}

\usepackage{physics}

%%%%%%% symbols packages %%%%%%%%%%%%%%
\usepackage{bm} %for use \bm instead \boldsymbol in math mode 
\usepackage{dsfont}
\usepackage{stmaryrd}
%%%%%%%%%%%%%%%%%%%%%%%%%%%%%%%%%%%%%%%


%%%%%% graphicx %%%%%%%%%%%%%%%%%%%%%%%
\usepackage{graphicx}
\usepackage{color}
%\usepackage{xypic}
\usepackage[all]{xy}
\usepackage{calc}
\usepackage{booktabs}
\usepackage{minibox}
%%%%%%%%%%%%%%%%%%%%%%%%%%%%%%%%%%%%%%%

\usepackage{enumerate}

\usepackage{fancyhdr}
%%%%% header and footer rule %%%%%%%%%
\setlength{\headheight}{14pt}
\renewcommand{\headrulewidth}{0pt}
\renewcommand{\footrulewidth}{0pt}
\fancypagestyle{plain}{\fancyhf{}
\fancyhead{}
\lfoot{}
\rfoot{\small \thepage}}
\fancypagestyle{vangelis}{\fancyhf{}
\rhead{\small \leftmark}
\lhead{\small }
\lfoot{}
\rfoot{\small \thepage}}
%%%%%%%%%%%%%%%%%%%%%%%%%%%%%%%%%%%%%%%

\usepackage{hyperref}
\usepackage{url}
%%%%%%% hyperref settings %%%%%%%%%%%%
\hypersetup{pdfpagemode=UseOutlines,hidelinks,
bookmarksopen=true,
pdfdisplaydoctitle=true,
pdfstartview=Fit,
unicode=true,
pdfpagelayout=OneColumn,
}
%%%%%%%%%%%%%%%%%%%%%%%%%%%%%%%%%%%%%%

\usepackage[space]{grffile}

\usepackage{geometry}
\geometry{left=25.63mm,right=25.63mm,top=36.25mm,bottom=36.25mm,footskip=24.16mm,headsep=24.16mm}

%\usepackage[explicit]{titlesec}
%%%%%% titlesec settings %%%%%%%%%%%%%
%\titleformat{\chapter}[block]{\LARGE\sc\bfseries}{\thechapter.}{1ex}{#1}
%\titlespacing*{\chapter}{0cm}{0cm}{36pt}[0ex]
%\titleformat{\section}[block]{\Large\bfseries}{\thesection.}{1ex}{#1}
%\titlespacing*{\section}{0cm}{34.56pt}{17.28pt}[0ex]
%\titleformat{\subsection}[block]{\large\bfseries{\thesubsection.}{1ex}{#1}
%\titlespacing*{\subsection}{0pt}{28.80pt}{14.40pt}[0ex]
%%%%%%%%%%%%%%%%%%%%%%%%%%%%%%%%%%%%%%

%%%%%%%%% My Theorems %%%%%%%%%%%%%%%%%%
\newtheorem{thm}{Θεώρημα}[section]
\newtheorem{cor}[thm]{Πόρισμα}
\newtheorem{lem}[thm]{λήμμα}
\theoremstyle{definition}
\newtheorem{dfn}{Ορισμός}[section]
\newtheorem{dfns}[dfn]{Ορισμοί}
\theoremstyle{remark}
\newtheorem{remark}{Παρατήρηση}[section]
\newtheorem{remarks}[remark]{Παρατηρήσεις}
%%%%%%%%%%%%%%%%%%%%%%%%%%%%%%%%%%%%%%%




\newcommand{\vect}[2]{(#1_1,\ldots, #1_#2)}
%%%%%%% nesting newcommands $$$$$$$$$$$$$$$$$$$
\newcommand{\function}[1]{\newcommand{\nvec}[2]{#1(##1_1,\ldots, ##1_##2)}}

\newcommand{\linode}[2]{#1_n(x)#2^{(n)}+#1_{n-1}(x)#2^{(n-1)}+\cdots +#1_0(x)#2=g(x)}

\newcommand{\vecoffun}[3]{#1_0(#2),\ldots ,#1_#3(#2)}

\newcommand{\mysum}[1]{\sum_{n=#1}^{\infty}




\begin{document}

\begin{center}
  \minibox{\large \textcolor{Col1}{\textbf{Ασκήσεις στο Εσωτερικό Γινόμενο}}}
\end{center}

\vspace{\baselineskip}

\begin{enumerate}
  \item Θεωρείστε τα πολυώνυμα $ f(t) = 3t-5 $ και $ g(t) = t^{2} $ στο χώρο των 
    πολυωνύμων $ P_{2}[t] $ με εσωτερικό γινόμενο 
    \[
      \langle f, g \rangle = \int _{0}^{1}f(t)g(t) \,{dt}
    \] 
    \begin{enumerate}[i)]
      \item Να υπολογίσετε το $ \langle f(t), g(t)\rangle $.
      \item Να υπολογίσετε τα $ \norm{f} $ και $ \norm{g} $.
    \end{enumerate}
    \hfill Απ: $ \langle f, g\rangle = - \frac{11}{12} $, $ \norm{f} = 13 $, $
    \norm{g} = \frac{1}{5} $ 

  \item Να βρείτε ένα μη μηδενικό, μοναδιαίο διάνυσμα $ \mathbf{w} $, που να είναι 
    ορθογώνιο προς τα διανύσματα $ \mathbf{u} = (1,2,1) $ και $ \mathbf{v}=(2,5,4) $ 
    στον $ \mathbb{R}^{3} $.

    \hfill Απ: $ \mathbf{w} = \bigl(\frac{3}{\sqrt{14}} , - \frac{2}{\sqrt{14}} ,
    \frac{1}{\sqrt{14}}\bigr) $, για $ z=1 $

  \item Να δείξετε ότι η παρακάτω σχέση, ορίζει εσωτερικό γινόμενο στον 
    $ \mathbb{R}^{2} $.
    \[
      \langle \mathbf{u}, \mathbf{v}\rangle = x_{1} y_{1} - x_{1} y_{2} - x_{2} y_{1} + 
      3 x_{2} y_{2}
    \] 
    όπου $ \mathbf{u}=(x_{1}, x_{2}) $ και $ \mathbf{v}=(y_{1}, y_{2}) $.

  \item Να δείξετε ότι η παρακάτω σχέση, ορίζει εσωτερικό γινόμενο στον 
    $ \mathbb{R}^{2} $
    \[
      \langle \mathbf{u}, \mathbf{v}\rangle = 4 x_{1} y_{1} + 3 x_{2} y_{2} 
    \] 
    και να βρεθεί ο πίνακας του εσωτερικού γινομένου ως προς την κανονική βάση του $
    \mathbb{R}^{2} $, αν $ \mathbf{u}=(x_{1}, x_{2}) $ και $ \mathbf{v}=(y_{1}, y_{2}) $.


  \item Θεωρείστε τα παρακάτω πολυώνυμα στον $ P_{2}[t] $ με εσωτερικό γινόμενο 
    $ \langle f, g\rangle = \int _{0}^{1}f(t)g(t) \,{dt} $.
    \[
      f(t)=t+2, \quad g(t)=3t-2, \quad h(t)=t^{2}-2t-3 
    \]
    \begin{enumerate}[i)]
      \item Να υπολογίσετε τα $ \langle f, g\rangle $ και $ \langle f, h\rangle $. 
        \hfill Απ: $ \langle f, g\rangle = -1 $, $ \langle f, h\rangle = -
        \frac{37}{4} $ 
      \item Να υπολογίσετε τα $ \norm{f} $ και $ \norm{g} $. 
        \hfill Απ: $ \norm{f} = \sqrt{\frac{19}{3}} $, $ \norm{g} = 1 $  
      \item Να κανονικοποιήσετε τα $ f $ και $ g $. 
        \hfill Απ: $ \hat{f} = \sqrt{\frac{3}{19}} (t+2) $, $ \hat{g}= 3t-2 $  
    \end{enumerate}

  \item Θεωρείστε τον υπόχωρο $ W = < (1,1,1,1), (1,1,2,4), (1,2,-4,-3)> $ του 
    $ \mathbb{R}^{4} $. Να βρείτε μια ορθοκανονική βάση του W.

    \hfill Απ: $ \frac{1}{2} (1,1,1,1), \frac{1}{\sqrt{6}} (-1,-1,0,2) , \frac{1}{5
    \sqrt{2}} (1,3-6,2) $.

  \item Θεωρείστε τον διανυσματικό χώρο $ P_{2}[t] $ με εσωτερικό γινόμενο 
    \[
      \langle p(t), q(t)\rangle = \int _{-1}^{1}p(t)q(t) \,{dt} 
    \] 
    και βάση $ B = \{ 1,t,t^{}2 \} $. Να βρεθεί μια ορθοκανονική βάση με ακέραιους
    συντελεστές.

    \hfill Απ: $ 1, 2t-1, 6t^{2}-6t+1 $. 

  \item Στο διανυσματικό χώρο $ C[a,b] = \{ f \colon [a,b] \to \mathbb{R} \; : \; f
    \; \text{συνεχής} \} $, με τις γνωστές πράξεις του αθροίσματος συναρτήσεων και του
    πολλαπλασιασμού αριθμού επί συνάρτηση, ορίζουμε τη σχέση 
    \[
      \langle f, g\rangle = \int _{a}^{b} f(x)g(x) \,{dx} 
    \] 
    Να δείξετε ότι η σχέση αυτή, ορίζει εσωτερικό γινόμενο.

\end{enumerate}
\end{document}

