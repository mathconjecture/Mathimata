\documentclass[a4paper,table]{report}
\input{preamble_themata}
\newcommand{\vect}[2]{(#1_1,\ldots, #1_#2)}
%%%%%%% nesting newcommands $$$$$$$$$$$$$$$$$$$
\newcommand{\function}[1]{\newcommand{\nvec}[2]{#1(##1_1,\ldots, ##1_##2)}}

\newcommand{\linode}[2]{#1_n(x)#2^{(n)}+#1_{n-1}(x)#2^{(n-1)}+\cdots +#1_0(x)#2=g(x)}

\newcommand{\vecoffun}[3]{#1_0(#2),\ldots ,#1_#3(#2)}

\newcommand{\mysum}[1]{\sum_{n=#1}^{\infty}

\input{myboxes}

\everymath{\displaystyle}
\pagestyle{vangelis}



\begin{document}

\begin{mybox3}
  \setcounter{thm}{1}
  \begin{thema}
    Θεωρούμε την καμπύλη του $ \mathbb{R}^{3} $
    \[
      (c) \colon \mathbf{r}(t)= \cos{t}
      \,
      \mathbf{i} + t \, \mathbf{j} +  \sin{t} \, \mathbf{k}, \; t \in [0, 2 \pi]
    \]
  \begin{enumerate}[i)]
    \item Να βρείτε για κάθε $ t \in [0,2 \pi] $, την οξεία γωνία που σχηματίζει το
      εφαπτόμενο διάνυσμα $ \mathbf{r'}(t) $ της $ (c) $ με τον άξονα $ y'y $. 
    \item Να βρείτε τα σημεία της $ (c) $, στα οποία το πρώτο κάθετο διάνυσμα 
      $ \mathbf{N} $ της $ (c) $ είναι κάθετο με τον άξονα $ x'x $.
    \item Να βρείτε την ακτίνα του εφαπτόμενου κύκλου της $ (c) $ σε κάθε σημείο της.
    \item Θεωρούμε επίσης την καμπύλη του $ \mathbb{R}^{3} $
      \[
        (\gamma) \colon
        \mathbf{\phi}(t)= \sin{t} \, \mathbf{i} + t \, \mathbf{j} + \cos{t} \, \mathbf{k},
        \; t \in [0,2 \pi]
      \]
      Έστω ακόμη $ (s) $ μια επιφάνεια του $ \mathbb{R}^{3} $, η 
      οποία περιέχει τις καμπύλες $ (c), (\gamma) $ και ορίζεται από την εξίσωση $ z=f(x,y)
      $ με $f$ διαφορίσιμη συνάρτηση. Αν $ A\Bigl(\frac{\sqrt{2}}{2} , \frac{\pi}{4} ,
      \frac{\sqrt{2}}{2}\Bigr) $, τότε:
      \begin{enumerate}[i)]
        \item Να δείξετε ότι το $A$ ανήκει στην $ (s) $.
        \item Να βρείτε την εξίσωση της κάθετης ευθείας της $ (s) $ στο σημείο $A$.
      \end{enumerate}
  \end{enumerate}
  \end{thema}
\end{mybox3}
\begin{solution}
  \begin{enumerate}[i)]
    \item 
      Βρίσκουμε:
      \begin{align*}
        \mathbf{r'}(t)&=- \sin{t}\, \mathbf{i} +\, \mathbf{j} + \cos{t} \, \mathbf{k} \\
        \norm{\mathbf{r'}(t)} &= \sqrt{\sin^{2}{t} + 1^{2} + \cos^{2}{t}} = \sqrt{2}
      \end{align*} 
      Η γωνία $\theta$ που σχηματίζει το εφαπτόμενο διάνυσμα τον άξονα $ y'y $, 
      δίνεται από: 
      \[
        \cos{\theta} = \frac{\mathbf{r'}(t) \cdot \mathbf{j}}{\norm{\mathbf{r'}(t)}} =
        \frac{1}{\sqrt{2}} = \frac{\sqrt{2}}{2} \Rightarrow \theta = \frac{\pi}{4}
      \]
    \item Θα βρούμε το πρώτο κάθετο διάνυσμα $ \mathbf{N} $.
      \begin{align*}
        \mathbf{T} &= \frac{\mathbf{r'}(t)}{\norm{\mathbf{r'}(t)}} =  -
        \frac{\sin{t}}{\sqrt{2}}\,\mathbf{i} +
        \frac{1}{\sqrt{2}}\,\mathbf{j} + \frac{\cos{t}}{\sqrt{2}}\,\mathbf{k} \\
        \dv{\mathbf{T}}{t} &=  - \frac{\cos{t}}{\sqrt{2}}\,\mathbf{i} + 0
        \,\mathbf{j} - \frac{\sin{t}}{\sqrt{2}}\,\mathbf{k} \\
        \norm{\dv{\mathbf{T}}{t}} &= \sqrt{\frac{\cos^{2}{t}}{2} + 0^{2} +
        \frac{\sin^{2}{t}}{2}} = \sqrt{\frac{1}{2}} = \frac{\sqrt{2}}{2} \\
          \mathbf{N} &= \frac{{d\mathbf{T}}/{dt}}{\norm{{d\mathbf{T}}/{dt}}} =
          \frac{- \frac{\cos{t}}{\sqrt{2}}}{\frac{\sqrt{2}}{2}} \, \mathbf{i} + 0 \,
          \mathbf{j} + \frac{- \frac{\sin{t}}{\sqrt{2}}}{\frac{\sqrt{2}}{2}} \, \mathbf{k}
          = - \cos{t} \, \mathbf{i} + 0 \, \mathbf{j} - \sin{t} \, \mathbf{k} 
        \end{align*}
        Το διάνυσμα $ \mathbf{N} $ θα είναι κάθετο προς τον άξονα $ x'x $ ανν 
        \[
          \mathbf{N}\cdot \mathbf{i} = 0 \Leftrightarrow - \cos{t} = 0 \Leftrightarrow
          \cos{t} = 0 \Leftrightarrow t= \frac{\pi}{2} \; \text{ή} \; t= 
          \frac{3\pi}{2} 
        \]
      \item Γνωρίζουμε, ότι ο κύκλος που εφάπτεται στην καμπύλη, σε κάθε σημείο της,
        είναι ο κύκλος καμπυλότητας. Άρα, ζητάμε την ακτίνα καμπυλότητας.
        \[
          k = \frac{\norm{d \mathbf{T}/dt}}{\norm{\mathbf{r'}(t)}} =
          \frac{\frac{\sqrt{2}}{2}}{\sqrt{2}} = \frac{1}{2}
        \] 
        Επομένως, η ακτίνα καμπυλότητας, είναι
        \[
          \rho = \frac{1}{k} = 2 
        \] 
      \item Το σημείο $ A\Bigl(\frac{\sqrt{2}}{2} , \frac{\pi}{4} ,
        \frac{\sqrt{2}}{2}\Bigr) $ ανήκει στην $ (s) $ αφού ανήκει στις δύο καμπύλες 
        που περιέχονται σε αυτήν. Πράγματι:
        \begin{myitemize}
          \item Για $ t= \frac{\pi}{4} $: \quad
            $\mathbf{r}\Bigl(\frac{\pi}{4}\Bigr)= \cos{\frac{\pi}{4}}\, \mathbf{i} +
            \frac{\pi}{4}\, \mathbf{j} + \sin{\frac{\pi}{4}}  \, \mathbf{k} =
            \mathbf{r}(t)= \frac{\sqrt{2}}{2}\, \mathbf{i} + \frac{\pi}{4}\, \mathbf{j} +
            \frac{\sqrt{2}}{2} \, \mathbf{k} = \Bigl(\frac{\sqrt{2}}{2} , \frac{\pi}{4} ,
            \frac{\sqrt{2}}{2}\Bigr) = A$
          \item Για $ t= \frac{\pi}{4} $: \quad
            $\mathbf{\phi}\Bigl(\frac{\pi}{4}\Bigr)= \sin{\frac{\pi}{4}}\, \mathbf{i} +
            \frac{\pi}{4}\, \mathbf{j} + \cos{\frac{\pi}{4}}  \, \mathbf{k} =
            \mathbf{r}(t)= \frac{\sqrt{2}}{2}\, \mathbf{i} + \frac{\pi}{4}\, \mathbf{j} +
            \frac{\sqrt{2}}{2} \, \mathbf{k} = \Bigl(\frac{\sqrt{2}}{2} , \frac{\pi}{4} ,
            \frac{\sqrt{2}}{2}\Bigr) = A$
        \end{myitemize}  
        Για την κάθετη ευθεία στην επιφάνεια $ (s) $ στο σημείο $ A $, χρειαζόμαστε 
        ένα κάθετο διάνυσμα της $ (s) $, στο $A$. Έχουμε:
        \[
          \mathbf{\phi'}(t) =  \cos{t}\,\mathbf{i} + 0\,\mathbf{j} - \sin{t} \,\mathbf{k}
        \] 
        \begin{myitemize}
          \item Το διάνυσμα $ \mathbf{r'}\Bigl(\frac{\pi}{4}\Bigr) =  - \frac{\sqrt{2}}{2}\,\mathbf{i} +
            \,\mathbf{j} + \frac{\sqrt{2}}{2} \,\mathbf{k} $ είναι
            εφαπτόμενο στη $ (c) $ και άρα στην $ (s) $ στο $A$.
          \item Το διάνυσμα $ \mathbf{\phi'}\Bigl(\frac{\pi}{4}\Bigr) =  \frac{\sqrt{2}}{2}\,\mathbf{i} +
            \,\mathbf{j} - \frac{\sqrt{2}}{2} \,\mathbf{k} $ είναι
            εφαπτόμενο στη $ (\gamma) $ και άρα στην $ (s) $ στο $A$.
        \end{myitemize}
        Επομένως το διάνυσμα 
        \[
          \mathbf{r'}\Bigl(\frac{\pi}{4}\Bigr) \times \mathbf{\phi'}\Bigl(\frac{\pi}{4}\Bigr) = 
          \begin{vmatrix*}[c]
            \mathbf{i} & \mathbf{j} & \mathbf{k} \\[8pt]
            - \frac{\sqrt{2}}{2} & 1 & \frac{\sqrt{2}}{2} \\[8pt]
            \frac{\sqrt{2}}{2} & 1 & - \frac{\sqrt{2}}{2} 
          \end{vmatrix*} = - \sqrt{2} \, \mathbf{i} - 0 \, \mathbf{j} - \sqrt{2} \,
          \mathbf{k}
         \] 
         είναι κάθετο στην $ (s) $, και άρα η εξίσωση της κάθετης ευθείας, είναι:
         \[
           k: \begin{rcases} 
             x(t) = \frac{\sqrt{2}}{2} - \sqrt{2} t \\
             y(t) = \frac{\pi}{4} \\
             z(t) = \frac{\sqrt{2}}{2} - \sqrt{2} t
           \end{rcases}
          \] 
    \end{enumerate} 
\end{solution}

\end{document}
