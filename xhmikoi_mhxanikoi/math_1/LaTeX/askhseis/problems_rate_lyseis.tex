\documentclass[a4paper,table]{report}
\input{preamble_ask.tex}
\input{definitions_ask.tex}
\input{tikz}
\input{myboxes}

\newcommand{\twocolumnsidescc}[2]{\begin{minipage}[c]{0.20\linewidth}\raggedright
    #1
    \end{minipage}\hfill\begin{minipage}[c]{0.75\linewidth}\raggedright
    #2
  \end{minipage}
}

\newcommand{\twocolumnsidescd}[2]{\begin{minipage}[c]{0.25\linewidth}\raggedright
    #1
    \end{minipage}\hfill\begin{minipage}[c]{0.75\linewidth}\raggedright
    #2
  \end{minipage}
}

\newcommand{\twocolumnsidescm}[2]{\begin{minipage}[c]{0.265\linewidth}\raggedright
    #1
    \end{minipage}\hfill\begin{minipage}[c]{0.71\linewidth}\raggedright
    #2
  \end{minipage}
}

\newcommand{\twocolumnsidescq}[2]{\begin{minipage}[c]{0.28\linewidth}\raggedright
    #1
    \end{minipage}\hfill\begin{minipage}[c]{0.70\linewidth}\raggedright
    #2
  \end{minipage}
}







\pagestyle{askhseis}


\begin{document}

% \chapter*{Ρυθμός Μεταβολής}

\begin{center}
  \large{\bfseries \textcolor{Col1}{Προβλήματα στο Ρυθμό Μεταβολής}}
\end{center}

\vspace{\baselineskip}

\section*{Λύσεις των Προβλημάτων}

% \vspace{\baselineskip}

\begin{mybox3}
  \begin{problem}
    Φουσκώνουμε με αέρα ένα σφαιρικό μπαλόνι, έτσι
    ώστε ο όγκος του να αυξάνει με ρυθμό $\SI{100}{cm^{3}\per s}$. Πόσο γρήγορα
    αυξάνει η ακτίνα του μπαλονιού όταν η διάμετρος του είναι $\SI{50}{cm}$?
  \end{problem}
\end{mybox3}
\begin{solution}
  \item {}

    \twocolumnsidescc{
      \begin{tikzpicture}[scale=0.80]
        \shade[ball color = Col1,opacity=0.55] (0,0) circle (2cm);
        \draw (0,0) circle (2cm);
        \draw (-2,0) arc (180:360:2 and 0.6);
        \draw[dashed,opacity=0.6] (2,0) arc (0:180:2 and 0.6);
        \fill[fill=black] (0,0) circle (1.2pt);
        \draw[dashed,opacity=0.6] (0,0 ) -- node[above,opacity=1]{$r$} (2,0);
      \end{tikzpicture}
    }{
      Έστω $V$ ο όγκος του μπαλονιού και $r$ η ακτίνα του.
      Γνωρίζουμε ότι
      \[
        V=\frac{4}{3}\pi r^{3}
      \]
      Παραγωγίζοντας την παραπάνω σχέση ως προς $t$ έχουμε:
      \[
        \dv{V}{t} = 4\pi r^{2}\dv{r}{t} \Leftrightarrow
        \dv{r}{t} = \frac{1}{4\pi r^{2}}\dv{V}{t}
      \]
    }

    Με αντικατάσταση των τιμών $r=25$ και $\dv{V}{t}=100$ έχουμε
    \[
      \dv{r}{t} = \frac{1}{4\pi (25)^{2}}100 = \frac{1}{25\pi}.
    \]
    Επομένως, η ακτίνα του μπαλονιού αυξάνει με ρυθμό $\frac{1}{25\pi}\si{cm\per s}$.
  \end{solution}

  \begin{mybox3}
    \begin{problem}
      Μια σκάλα μήκους $\SI{10}{m}$ είναι ακουμπισμένη
      σε ένα κατακόρυφο τοίχο. Εάν το κάτω μέρος (βάση) της σκάλας ολισθαίνει
      οριζόντια, απομακρυνόμενη από τον τοίχο με ρυθμό $\SI{1}{m\per s}$, πόσο
      γρήγορα η κορυφή της σκάλας πέφτει (ολισθαίνοντας κατακόρυφα στον τοίχο), όταν η
      βάση της σκάλας απέχει $\SI{6}{m}$ από τον τοίχο?
    \end{problem}
  \end{mybox3}
  \begin{solution}
  \item {}

    \twocolumnsidescd{
      \begin{tikzpicture}[scale=0.85]
        \node  (0) at (0.3, 0) {};
        \node  (1) at (4, 0) {};
        \node  (2) at (4, 4.2) {};
        \node  (3) at (4, 3.5) {};
        \node  (4) at (1.5, 0) {};
        \node  (5) at (0.3, -0.25) {};
        \node  (6) at (4.25, -0.25) {};
        \node  (7) at (4.25, 4.2) {};
        \coordinate (1a) at ($(1)-(0,0.6)$) ;
        \coordinate (4a) at ($(4)-(0,0.6)$) ;
        \coordinate (4b) at ($(4)+(-0.15,0.2)$) ;
        \coordinate (4l) at ($(4b)-(0.8,0)$) ;
        \draw (0.center) to (4.center);
        \draw (4.center) to node[above] {$x$} (1.center);
        \draw (1.center) to node[left] {$y$} (3.center);
        \draw (3.center) to (2.center);
        \draw[line width=2.5pt,Col1] (3.center) to node[above,sloped]{$\SI{10}{m}$} 
          (4.center);
        \draw (0.center) to (5.center);
        \draw (5.center) to (6.center);
        \draw (6.center) to (7.center);
        \draw (2.center) to (7.center);
        \draw[pattern=north west lines] (0.center) -- (1.center) -- 
          (2.center) -- (7.center) -- (6.center) -- (5.center);
        \path[fill=Col2!75,opacity=0.4] (0.center) -- (1.center) -- 
          (2.center) -- (7.center) -- (6.center) -- (5.center);
        \draw[{Stealth}-{Stealth}] (1a.center) node {$|$} --
          node[below]{$\SI{6}{m}$} (4a.center) node {$|$} ;
        \draw[yshift=10pt] (4b.center) edge[-stealth,Col1,ultra thick] 
          node[above] {$\SI{1}{m \per s}$} (4l.center) ;
      \end{tikzpicture}
    }{
      Έστω $x=x(t)$ η απόσταση της βάσης της σκάλας από τον τοίχο και $y=y(t)$ η 
      απόσταση της κορυφής της σκάλας από το έδαφος.

      Από το Πυθαγόρειο θεώρημα έχουμε ότι
      \begin{align}\label{eq:pyth}
        x^{2}+y^{2}&=10^{2}  \notag \Leftrightarrow \\
        x^{2}+y^{2}&=100
      \end{align}
      Παραγωγίζοντας την σχέση~\eqref{eq:pyth} ως προς το χρόνο $t$ έχουμε
      \[
        2x\dv{x}{t} + 2y\dv{y}{t} = 0
      \]
    }
    Λύνουμε ως προς το ζητούμενο ρυθμό μεταβολής και έχουμε
    \begin{equation}\label{eq:rateofy}
      \dv{y}{t} = -\frac{x}{y}\dv{x}{t}
    \end{equation}

    Όταν $x=6$, τότε από τη σχέση~\eqref{eq:pyth} έχουμε ότι $y=8$. Με αντικατάσταση 
    αυτών των τιμών στη σχέση~\eqref{eq:rateofy} έχουμε
    \[
      \dv{y}{t} = -\frac{6}{8}(1) = -\SI[quotient-mode=fraction]{3/4}{m/s}
    \]

    Το γεγονός ότι ο ρυθμός μεταβολής $\dv*{y}{t}$ είναι αρνητικός σημαίνει
    ότι η απόσταση $y$ της κορυφής της σκάλας από το έδαφος,
    \emph{μειώνεται} με ρυθμό ${3}/{4}$ $\si{m\per s}$. Με άλλα λόγια η
    κορυφή της σκάλας πέφτει με ρυθμό ${3}/{4}$ $\si{m\per s}$.
  \end{solution}

  \begin{mybox3}
    \begin{problem}
      Μια δεξαμενή νερού έχει το σχήμα ενός ανάποδου
      κυκλικού κώνου με ακτίνα βάσης $2$ $\si{m}$ και ύψος $4$ $\si{m}.$ Αν
      στη δεξαμενή εισέρχεται ποσότητα νερού με ρυθμό $2$ $\si{m^{3}/min}$, να
      υπολογίσετε το ρυθμό με τον οποίο ανέρχεται η στάθμη του νερού στο εσωτερικό της
      δεξαμενής, όταν το νερό έχει βάθος $3$ $\si{m}$.
    \end{problem}
  \end{mybox3}
  \begin{solution}
  \item {}

    Ο όγκος του κυκλικού κώνου δίνεται από τη σχέση

    \twocolumnsidescc{
      \begin{tikzpicture}
        \node (0) at (0, 0) {};
        \node at (0) [below] {$A$};
        \node (1) at (-1, 3) {};
        \node (2) at (1, 3) {};
        \node at (2) [right] {\smaller$C$};
        \node (5) at (-0.5, 1.5) {};
        \node (8) at (0.5, 1.5) {};
        \node at (8) [right] {\smaller$E$};
        \node (9) at (0, 3) {};
        \node at (9) [left] {\smaller$B$};
        \node (10) at (0, 1.5) {};
        \node at (10) [left] {\smaller$D$};
        \draw [bend left=90, looseness=0.50] (1.center) to (2.center);
        \draw [bend right=90, looseness=0.50] (1.center) to (2.center);
        \draw [bend left=90, looseness=0.75] (5.center) to (8.center);
        \draw [bend right=90, looseness=0.75] (5.center) to (8.center);
        \draw (0.center) to (5.center);
        \draw (5.center) to (1.center);
        \draw (0.center) to (8.center);
        \draw (8.center) to (2.center);
        \draw (9.center) to node[pos=0.5,fill=white] {\smaller$2$} (2.center);
        \draw (10.center) to node[pos=0.5,fill=blue!25,inner sep=1.5pt] 
          {\smaller$r$} (8.center);
        \draw (9.center) to (10.center);
        \draw (10.center) to (0.center);
        \begin{scope}
          \draw[{Stealth}-{Stealth}] ([xshift=-5pt]5.center) node {$-$} --
            node[fill=white]{\smaller$h$} ([xshift=-5pt]5|-0) node {$-$} ;
          \draw[{Stealth}-{Stealth}] ([xshift=-5pt]1.center) node {$-$} -- 
            node[fill=white]{\smaller$4$} ([xshift=-5pt]1|-0) node {$-$} ;
          \begin{pgfonlayer}{background}
            \fill[blue!25] (0.center) -- (5.center) to [bend left=90, looseness=0.75] 
              (8.center) -- (0.center) ;
          \end{pgfonlayer}
        \end{scope}
      \end{tikzpicture}
    }{
      \begin{equation} \label{eq:cylvol}
        V=\frac{1}{3}\pi r^{2} h
      \end{equation}
      όπου $r$ είναι η ακτίνα της βάσης και $h$ το ύψος του κώνου. 
      Από την ομοιότητα των τριγώνων $ADE$ και $ABC$ έχουμε ότι
      \[
        \frac{r}{h} = \frac{2}{4} \Rightarrow r=\frac{h}{2}
      \]
      και έτσι η σχέση~\eqref{eq:cylvol} γίνεται
      \[
        V=\frac{1}{3}\pi\Bigl(\frac{h}{2}\Bigr)^{2}h = \frac{\pi}{12}h^{3}
      \]
    }
    Παραγωγίζοντας, τώρα αυτή τη σχέση ως προς $t$ έχουμε
    \begin{align}\label{eq:volrate}
      \dv{V}{t}&=\frac{\pi}{4}h^{2}\dv{h}{t}  \notag \Leftrightarrow \\
      \dv{h}{t}&=\frac{4}{\pi h^{2}}\dv{V}{t}
    \end{align}
    και αντικαθιστώντας $h=3$ και $\dv{V}{t}=2$ έχουμε τελικά
    \[
      \dv{h}{t}=\frac{4}{\pi (3)^{2}}\cdot 2 = \frac{8}{9\pi}
    \]
    Επομένως η στάθμη του νερού στο εσωτερικό της δεξαμενής ανεβαίνει με ρυθμό
    $\frac{8}{9\pi}$ $\si{m/min}$.
  \end{solution}

  \begin{mybox3}
    \begin{problem}
      Αυτοκίνητο $A$ ταξιδεύει δυτικά με ταχύτητα $50$
      $\si{km\per h}$ και αυτοκίνητο $B$ ταξιδεύει βόρεια με ταχύτητα $60$
      $\si{km/h}$. Και τα δύο αυτοκίνητα κινούνται προς την διαστάυρωση των δύο δρόμων. 
      Με τι ρυθμό τα δύο αυτοκίνητα πλησιάζουν το ένα το άλλο, όταν το $A$ βρίσκεται 
      $0.3$ $\si{km}$ και το $B$ $0.4$ $\si{km}$ από τη διασταύρωση?
    \end{problem}
  \end{mybox3}
  \begin{solution}
  \item {}

    Έστω $C$ η διασταύρωση των δύο δρόμων. Έστω επίσης $x=x(t)$ και $y=y(t)$ η 
    απόσταση των $A$ και $B$ αντίστοιχα, από τη διασταύρωση και $s=s(t)$ η απόσταση 
    των δύο αυτοκινήτων κατά την χρονική στιγμή $t$.

    \twocolumnsidescm{
      \begin{tikzpicture}
        \node (0) at (0, 0) {};
        \fill[Col1] (0) circle (3pt) ; 
        \node (1) at (0, 4) {};
        \node[Col1] at (1) {$\times$} ; 
        \node at (1.west) [left] {$C$} ; 
        \node (2) at (3, 4) {};
        \coordinate (0a) at ($(0)+(0,1.0)$) ;
        \coordinate (2a) at ($(2)-(1.0,0)$) ;
        \coordinate (o) at ($(0)+(0,0.2)$) ;
        \coordinate (b) at ($(2)-(0.2,0)$) ;
        \fill[Col1] (2) circle (3pt) ; 
        \draw[dashed,Col1,thick] (o) -- node[sloped,above,pos=0.6] {$\SI{0.4}{km}$}
          node[black,right,pos=0.65] {$x$} (1) ;
        \draw[dashed,Col1,thick] (b) -- node[sloped,above,pos=0.6] {$\SI{0.3}{km}$}
          node[black,below,pos=0.6] {$y$} (1) ;
        \draw[Col1] (o) edge[-latex,ultra thick] node[right] {$\SI{60}{km\per h}$} (0a) ;
        \draw[Col1] (b) edge[-latex,ultra thick] node[below] {$\SI{50}{km\per h}$} (2a) ;
        \node[left] at (0.west) {$B$} ;
        \node[right] at (2.east) {$A$} ;
      \end{tikzpicture}
    }{
      Από τα δεδομένα του προβλήματος έχουμε ότι $\dv{x}{t}=-50$ \si{km/h} και
      $\dv{y}{t}=-60$ $\si{km/h}$. Οι παράγωγοι είναι αρνητικές γιατί οι αποστάσεις $x$ 
      και $y$ συνεχώς μειώνονται. Ζητούμενο είναι ο ρυθμός $\dv{s}{t}$

      Από το Πυθαγόρειο θεώρημα έχουμε
      \[
        s^{2}=x^{2}+y^{2}
      \]
      και παραγωγίζοντας ως προς $t$
      \begin{align} \label{eq:distrate}
        2s\dv{s}{t} &= 2x\dv{x}{t} + 2y\dv{y}{t} \notag \Rightarrow \\
        \dv{s}{t} &= \frac{1}{s}\left(x\dv{x}{t}+y\dv{y}{t}\right)
      \end{align}
    }

    Όταν $x=0.3$ και $y=0.4$ το Πυθαγόρειο θεώρημα δίνει $s=0.5$, οπότε με 
    αντικατάσταση στη σχέση~\eqref{eq:distrate} έχουμε
    \[
      \dv{s}{t}=\frac{1}{0.5}[0.3(-50)+0.4(-60)] = \SI{-78}{km/h}
    \]

    Επομένως τα δύο αυτοκίνητα πλησιάζουν το ένα το άλλο με ρυθμό $\SI{78}{km/h}$.
  \end{solution}

  \begin{mybox3}
    \begin{problem}
      Πλοίο Α ξεκινάει από ένα λιμάνι στις 12 μ.μ. και 
      κατευθύνεται δυτικά
      με ταχύτητα  $\SI{9}{km/h}$. Πλοίο Β ξεκινάει από το ίδιο λιμάνι στη 1
      μ.μ. και κατευθύνεται νότια με ταχύτητα $\SI{12}{km/h}$. Με τι ρυθμό
      απομακρύνονται μεταξύ τους τα δύο πλοία στις 3 μ.μ.?
    \end{problem}
  \end{mybox3}
  \begin{solution}
  \item {}

    Έστω $s$ η απόσταση των δύο πλοίων κατά τη χρονική στιγμή $ t $ και έστω 
    $x$ και $y$ οι αποστάσεις των πλοίων $ A $ και $B$ αντίστοιχα, από το λιμάνι. 
    Τότε

    \twocolumnsidescq{
      \begin{tikzpicture}
        \node (0) at (0, 0) {};
        \fill[Col1] (0) circle (3pt) ; 
        \node (2) at (0, 0) {};
        \node (w) at (-3,0) {} ;
        \node (s) at (0,-3) {} ;
        \coordinate (0a) at ($(0)-(0,1.0)$) ;
        \coordinate (2a) at ($(2)-(1.0,0)$) ;
        \coordinate (o) at ($(0)-(0,0.2)$) ;
        \coordinate (b) at ($(2)-(0.2,0)$) ;
        \fill[Col1] (2) circle (3pt) ; 
        \draw[Col1] (o) edge[-latex,ultra thick] 
          node[right] {$\SI{12}{km\per h}$} (0a) ;
        \draw[Col1] (b) edge[-latex,ultra thick] 
          node[above,xshift=-10pt] {$\SI{9}{km\per h}$} (2a) ;
        \node[above] at (0.north) {$A$} ;
        \node[right] at (2.east) {$B$} ;
        \draw[dashed,Col1,thick] (0a) -- node[left] {$y$} (s) ;
        \draw[dashed,Col1,thick] (2a) -- node[below] {$x$} (w) ;
      \end{tikzpicture}
    }{
      \[
        s^{2} = x^{2} + y^{2}
      \]
      Παραγωγίζοντας ως προς $t$ έχουμε
      \begin{align*}
        2s \dv{s}{t} & = 2x\dv{x}{t} + 2y\dv{y}{t} \Rightarrow \\
        s \dv{s}{t} & = x\dv{x}{t} + y\dv{y}{t} \Rightarrow 
      \end{align*}
      Οι αποστάσεις που έχουν διανύσει τα δύο πλοία ως τις $3$ μ.μ. είναι
      \begin{align*}
        x &= v\cdot t = 9 \cdot 3 = 27 \\
        y &= v\cdot t = 12 \cdot 2 = 24
      \end{align*}
    }

    Έχουμε ότι $ s = \sqrt{x^{2} + y^{2}} = \sqrt{27^{2} + 24^{2}} =
    \sqrt{729 + 576} = \sqrt{1305} = 36,12$.
    Άρα 
    \begin{align}
      36,12 \cdot \dv{s}{t} &= 27\cdot 9 + 24 \cdot 12 \\
      \dv{s}{t} &= \frac{243 + 288}{36,12} = \frac{531}{34,12} = 14,70
    \end{align}
    Επομένως τα δύο πλοία απομακρύνονται με ρυθμό $\SI{14,70}{km/h}$.
  \end{solution}

  \begin{mybox3}
    \begin{problem}
      Ένας άνδρας περπατά σε ευθύγραμμο μονοπάτι με
      ταχύτητα $\SI{4}{m/s}$. Μια δέσμη leiser η οποία είναι τοποθετημένη σε απόσταση 
      $\SI{20}{m}$ από το μονοπάτι, τον σημαδεύει συνεχώς. Με ποιο ρυθμό περιστρέφεται η 
      δέσμη leiser οταν ο άνδρας βρίσκεται σε απόσταση $\SI{15}{m}$ από εκείνο το σημείο 
      στο μονοπάτι το οποίο απέχει ελάχιστη απόσταση από την πηγή leiser?
    \end{problem}
  \end{mybox3}
  \begin{solution}
  \item {}

    \twocolumnsidescm{
      \begin{tikzpicture}
        \node  (0) at (1.5, 0) {};
        \node  (0b) at (1.5, -0.2) {};
        \node  (1) at (2.5, 0) {};
        \node  (2) at (4.5, 0) {};
        \node at (2) [above right] {$O$} ;
        \node  (3) at (5, 0) {};
        \node  (3b) at (5, -0.2) {};
        \node  (4) at (4.5, 2.5) {};
        \draw[Col2,very thick] (0.center) to node[above,pos=0.6,black] {$x$} 
          (3.center);
        \draw[dashed] (4.center) to (1.center);
        \draw[dashed] (4.center) to node[right] {$20$} (2.center);
        \draw[pattern=north west lines] (0.center) -- (3.center) -- (3b.center) --
          (0b.center) -- cycle ;
        \fill[Col2!75,opacity=0.4] (0.center) -- (3.center) -- (3b.center) -- (0b.center) -- cycle;
        \fill[Col1] (1) circle (3.0pt) ;
        \fill[magenta!75] (4) circle (2.5pt) ;
        \draw (4) pic[-latex,draw,fill=blue!25,angle radius=20pt,angle
          eccentricity=1.35,"$\phi$"] {angle=1--4--2} ;
        \draw (1) -- ++ (0,0.8) node (man) {} ;
        \fill (man) circle (3.0pt) ;
        \draw (man.south) -- +(-0.2,-0.30);
        \draw (man.south) -- +(0.2,-0.30);
        \draw (2) pic[angle radius=8pt,draw] {right angle=4--2--1} ;
      \end{tikzpicture}
    }{
      Έστω $O$ το σημείο στο μονοπάτι που απέχει λιγότερο από την πηγή leiser.
      Έστω $x$ η απόσταση του άνδρα από το σημείο $O$ και $\phi$ η γωνία που σχηματίζει 
      η δέσμη που σημαδεύει τον άνδρα με την κάθετη ευθεία στο μονοπάτι.

      Έχουμε
      \[
        \frac{x}{20}=\tan\phi \Rightarrow x=20\tan\phi
      \]
    }

    Παραγωγίζοντας ως προς $t$ έχουμε
    \begin{align*}
      \dv{x}{t} &= 20\cdot \frac{1}{\cos^{2}\phi}\dv{\phi}{t} \notag \Rightarrow \\
      \dv{\phi}{t}&=\frac{1}{20}\cos^{2}\phi \dv{x}{t} =\frac{1}{20}\cos^{2}\phi\cdot 4 =\frac{1}{5}\cos^{2}\phi
    \end{align*}
    Άρα
    \begin{equation} \label{eq:anglerate}
      \dv{\phi}{t}=\frac{1}{5}\cos^{2}\phi
    \end{equation}
    Όταν $x=15$ τότε το μήκος της δέσμης είναι $25$, όποτε $\cos\phi = \frac{4}{5}$ 
    και επομένως
    \[
      \dv{\phi}{t} = \frac{1}{5}\Bigl(\frac{4}{5}\Bigr)^{2} = \frac{16}{125} = 0.128
    \]
    Επομένως η δέσμη περιστρέφεται με ρυθμό $\SI{0.128}{rad/s}$.

  \end{solution}




  \end{document}

