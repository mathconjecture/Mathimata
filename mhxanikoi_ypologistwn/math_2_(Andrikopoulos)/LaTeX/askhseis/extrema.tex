\documentclass[a4paper,12pt]{article}
\usepackage{etex}
%%%%%%%%%%%%%%%%%%%%%%%%%%%%%%%%%%%%%%
% Babel language package
\usepackage[english,greek]{babel}
% Inputenc font encoding
\usepackage[utf8]{inputenc}
%%%%%%%%%%%%%%%%%%%%%%%%%%%%%%%%%%%%%%

%%%%% math packages %%%%%%%%%%%%%%%%%%
\usepackage{amsmath}
\usepackage{amssymb}
\usepackage{amsfonts}
\usepackage{amsthm}
\usepackage{proof}

\usepackage{physics}

%%%%%%% symbols packages %%%%%%%%%%%%%%
\usepackage{bm} %for use \bm instead \boldsymbol in math mode 
\usepackage{dsfont}
\usepackage{stmaryrd}
%%%%%%%%%%%%%%%%%%%%%%%%%%%%%%%%%%%%%%%


%%%%%% graphicx %%%%%%%%%%%%%%%%%%%%%%%
\usepackage{graphicx}
\usepackage{color}
%\usepackage{xypic}
\usepackage[all]{xy}
\usepackage{calc}
\usepackage{booktabs}
\usepackage{minibox}
%%%%%%%%%%%%%%%%%%%%%%%%%%%%%%%%%%%%%%%

\usepackage{enumerate}

\usepackage{fancyhdr}
%%%%% header and footer rule %%%%%%%%%
\setlength{\headheight}{14pt}
\renewcommand{\headrulewidth}{0pt}
\renewcommand{\footrulewidth}{0pt}
\fancypagestyle{plain}{\fancyhf{}
\fancyhead{}
\lfoot{}
\rfoot{\small \thepage}}
\fancypagestyle{vangelis}{\fancyhf{}
\rhead{\small \leftmark}
\lhead{\small }
\lfoot{}
\rfoot{\small \thepage}}
%%%%%%%%%%%%%%%%%%%%%%%%%%%%%%%%%%%%%%%

\usepackage{hyperref}
\usepackage{url}
%%%%%%% hyperref settings %%%%%%%%%%%%
\hypersetup{pdfpagemode=UseOutlines,hidelinks,
bookmarksopen=true,
pdfdisplaydoctitle=true,
pdfstartview=Fit,
unicode=true,
pdfpagelayout=OneColumn,
}
%%%%%%%%%%%%%%%%%%%%%%%%%%%%%%%%%%%%%%

\usepackage[space]{grffile}

\usepackage{geometry}
\geometry{left=25.63mm,right=25.63mm,top=36.25mm,bottom=36.25mm,footskip=24.16mm,headsep=24.16mm}

%\usepackage[explicit]{titlesec}
%%%%%% titlesec settings %%%%%%%%%%%%%
%\titleformat{\chapter}[block]{\LARGE\sc\bfseries}{\thechapter.}{1ex}{#1}
%\titlespacing*{\chapter}{0cm}{0cm}{36pt}[0ex]
%\titleformat{\section}[block]{\Large\bfseries}{\thesection.}{1ex}{#1}
%\titlespacing*{\section}{0cm}{34.56pt}{17.28pt}[0ex]
%\titleformat{\subsection}[block]{\large\bfseries{\thesubsection.}{1ex}{#1}
%\titlespacing*{\subsection}{0pt}{28.80pt}{14.40pt}[0ex]
%%%%%%%%%%%%%%%%%%%%%%%%%%%%%%%%%%%%%%

%%%%%%%%% My Theorems %%%%%%%%%%%%%%%%%%
\newtheorem{thm}{Θεώρημα}[section]
\newtheorem{cor}[thm]{Πόρισμα}
\newtheorem{lem}[thm]{λήμμα}
\theoremstyle{definition}
\newtheorem{dfn}{Ορισμός}[section]
\newtheorem{dfns}[dfn]{Ορισμοί}
\theoremstyle{remark}
\newtheorem{remark}{Παρατήρηση}[section]
\newtheorem{remarks}[remark]{Παρατηρήσεις}
%%%%%%%%%%%%%%%%%%%%%%%%%%%%%%%%%%%%%%%




\input{definitions_ask.tex}

\pagestyle{askhseis}

\renewcommand{\vec}{\mathbf}

\begin{document}

\begin{center}
  \minibox{\large \bfseries \textcolor{Col1}{Ασκήσεις στα Ακρότατα}}
\end{center}

\vspace{\baselineskip}

\section*{Τοπικά Ακρότατα}

\begin{enumerate}
  \item Να βρεθούν και να χαρακτηριστούν τα κρίσιμα σημεία  των παρακάτω συναρτήσεων:
    \begin{enumerate}[i)]
      \item $ f(x,y) = x^{3} + y^{3} + 3xy $ 
        \hfill Απ: max: $(-1,-1)  $, σάγμα: $ (0,0) $
      \item $ f(x,y) = x^{2}+y^{4} $ 
        \hfill Απ: min: $ (0,0) $ 
      \item $ f(x,y) = x^{3} + y^{3} - 3x -12y + 50 $ 
        \hfill Απ: max: $ (-1,-2)$, min: $ (1,2) $, 
        σάγμα: $ (1,-2), (-1,2) $
      \item $ f(x,y) = x^{3} + y^{3} -3x -3y + 1 $ 
        \hfill Απ: max: $(-1,-1)  $, min: $ (1,1) $,
        σάγμα: $ (1,-1), (-1,1) $
      \item $ f(x,y) = x^{3} + 4xy -4y^{2} $ 
        \hfill Απ: max: $ (-2/3, -1/3)  $, σάγμα: $ (0,0) $
      % \item $ f(x,y) = x^{4} + y^{4} -2(x-y)^{2}$  
      %   \hfill Απ: min: $ (\sqrt{2} , -\sqrt{2}), (-\sqrt{2} , \sqrt{2}) $, 
      %   σάγμα: $ (0,0) $
      \item $ f(x,y) = \mathrm{e}^{2x} - 2x + 2y^{2} +3 $ \hfill Απ: max: $ (0,0) $  
      \item $ f(x,y) = (x^{2}-3y^{2})e^{1-x^{2}-y^{2}} $ 
        \hfill Απ: max: $ (1,0), (-1,0) $, min: $ (0,1), (0,-1) $, 
        σάγμα: $ (0,0) $
      \item $ f(x,y,z) = 2-x^{2}+2xy-3y^{2}-2z^{2} $ \hfill Απ:  max $ (0,0,0) $ 
      \item $ f(x,y,z) = x^{2}+y^{2}+z^{2}-2x-1 $ \hfill Απ:  min: $ (1,0,0) $ 
      \item $ f(x,y,z) = 2x^{2}+xy+4y^{2}+xz+z^{2}+2 $ \hfill Απ: min: $ (0,0,0) $ 
    \end{enumerate}

  \item Να βρεθούν και να χαρακτηριστούν τα κρίσιμα σημεία της συνάρτησης 
    $ f(x,y,z) = x^{3} + y^{3}+z^{3} + 3xy +3yz + 3xz $

    \hfill Απ: σάγμα: $(0,0,0)$, max: $(-2,-2,-2)$  
\end{enumerate}


\section*{Ακρότατα Υπό Συνθήκη}

\begin{enumerate}

  \item 
    \begin{enumerate}[i)]
      \item $ z=xy $ με περιορισμό $ x+2y=2 $ \hfill Απ: max: $ (1,1/2) $ 
      \item $ z=7-y+x^{2} $ με περιορισμό $ x+y=0 $ \hfill Απ: min: $ (-1/2,1/2) $ 
      \item $ z=x-3y-xy $ με περιορισμό $ x+y=6 $ \hfill Απ: min: $ (1,5) $ 
    \end{enumerate}

  \item \textbf{(θέμα 2021)} 
    Να υπολογιστούν τα τοπικά ακρότατα της συνάρτησης $ f(x,y,z) = x^{2}+y^{2}+z^{2}
    $ που ικανοποιούν τον περιορισμό $ x+y+z+1=0 $.
    \hfill Απ: min: $ (-1/3,-1/3,-1/3) $ 

  \item Να υπολογιστούν τα τοπικά ακρότατα της συνάρτησης 
    $ f(x,y,z) = x^{2}+y^{2}+z^{2}-2x-2y-z+ \frac{5}{4} $ που ικανοποιούν τον 
    περιορισμό $ x^{2}+y^{2}-z=0  $.
    \hfill Απ: min: $ (1/ \sqrt[3]{4} , 1/ \sqrt[3]{4}) $ 

  \item Να υπολογιστούν τα τοπικά ακρότατα της συνάρτησης 
    $f(x,y,z)=xyz$ που ικανοποιούν την εξίσωση $x+y+z-1=0$.  
    \hfill Απ: max: $ (1/3,1/3,1/3) $ 

    %Thomas 12th 14.8 ex.1 
  \item Να βρείτε τα ακρότατα της συνάρτησης $ f(x,y) = xy $ πάνω στην έλλειψη 
    $ x^{2}+2y^{2}=1 $.

    \hfill Απ: 
    \begin{tabular}{l}
      max: $ f(\sqrt{2} /2, 1/2) = f(- \sqrt{2} /2, -1/2) = \frac{\sqrt{2}}{2} $ \\
      min $ f(\sqrt{2} /2, -1/2) = f(- \sqrt{2} /2, 1/2) = -\frac{\sqrt{2}}{2} $ \\
    \end{tabular}

    %Thomas 12th 14.8 ex.14 
  \item Να βρείτε τα ακρότατα της συνάρτησης $ f(x,y) = 3x-y+6 $ υπό τον περιορισμό 
    $ x^{2}+y^{2}=4 $.

    \hfill Απ:  
    \begin{tabular}{l}
      max: $ f(\frac{6}{\sqrt{10}} , - \frac{2}{\sqrt{10}}) = 2 \sqrt{10} +6 $ \\
      min $ f(-\frac{6}{\sqrt{10}} , + \frac{2}{\sqrt{10}}) = -2 \sqrt{10} +6 $ \\
    \end{tabular}
\end{enumerate}





\end{document}

