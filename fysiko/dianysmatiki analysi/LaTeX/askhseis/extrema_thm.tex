\input{preamble_ask.tex}
\input{definitions_ask.tex}

\pagestyle{askhseis}

% \linespread{1.3}
\renewcommand{\vec}{\mathbf}

\begin{document}

\begin{center}
  \minibox{\large \bfseries \textcolor{Col1}{θέματα στα Ακρότατα Συναρτήσεων πολλών
  Μεταβλητών}}
\end{center}

\vspace{\baselineskip}

\section*{Ακρότατα}

\begin{enumerate}
  \item \textbf{(Ιουν. 2019)} Να βρεθούν τα τοπικά ακρότατα της συνάρτησης 
    $ f(x,y) = x^{4}+y^{4}-4xy+1 $.

    \hfill Απ: $ f_{\rm{suddle}}(0,0) = 1, \; f_{\min}(1,1)=-1, \; f_{\min}(-1,-1)=-1 $

  \item \textbf{(Σεπ 2014)} Χαρακτηρίστε τα κρίσιμα σημεία της συνάρτησης 
    $ f(x,y) = x^{3}+y^{2} -6xy+6x+3y $.

    \hfill Απ: $ f_{\rm{suddle}}(1,3/2) = 19/4, \; f_{\min}(5,27/2) = -109/4 $ 
\end{enumerate}


\section*{Ακρότατα Υπό Συνθήκη}

\begin{enumerate}
  \item \textbf{(Ιουν. 2018)} Να βρεθούν τα ακρότατα της συνάρτησης 
    $ f(x,y) = x^{2}+y^{2} $ με τη συνθήκη 
    $ \frac{x^{2}}{a^{2}} + \frac{y^{2}}{b^{2}} = 1 $ και να χαρακτηριστούν. 
    Να εξεταστεί η περίπτωση $ a=b $.

  \item \textbf{(Ιαν. 2017)} Βρείτε τα ακρότατα της συνάρτησης $ f(x,y) = 3x+4y-1 $ 
    υπό την συνθήκη $ x^{2}+y^{2}=4 $. Να χαρακτηριστούν τα ακρότατα.

  \item \textbf{(Ιουν. 2014)} Να βρεθούν τα ακρότατα της συνάρτησης 
    $ f(x,y) = 2+2x+2y-x^{2}-y^{2} $ στις παρακάτω περιοχές:
    \begin{enumerate}[i)]
      \item στην περιοχή $ R $ του δεξιού ημικυκλίου $ x^{2}+y^{2}=4 $.
      \item στην περιοχή $ R $ του κάτω ημικυκλίου $ x^{2}+y^{2}=4 $.
    \end{enumerate}
\end{enumerate}


\section*{Ακρότατα σε κλειστή και φραγμένη περιοχή}

\begin{enumerate}
  \item \textbf{(Ιουν. 2015)} Θεωρούμε έναν μεταλλικό κυκλικό δίσκο με εξίσωση 
    $ x^{2}+y^{2}=4 $. Ο δίσκος θερμαίνεται έτσι ώστε η θερμοκρασία $ T $ σε κάθε 
    σημείο του δίσκο $ (x,y) $ να δίνεται από τη σχέση
    \[
      T(x,y) = 2x^{2}+y^{2}-y.
    \] 
    Βρείτε τα πιο θερμά και τα πιο ψυχρά σημεία του μεταλλικού δίσκου καθώς και τη
    θερμοκρασία τους.

  \item \textbf{(Ιουν. 2013)} Θεωρούμε μια ηλεκτρικά φορτισμένη υλική πλάκα, της οποίας 
    η περίμετρος ορίζεται από την εξίσωση $ \frac{x^{2}}{3} + y^{2} = 1 $. Η 
    πυκνότητα του ηλεκτρικού φορτίου δίνεται από την σχέση
    \[
      p(x,y) = 10 - (x-1)^{2}-y^{2}. 
    \]  
    Να βρεθούν τα σημεία της υλικής πλάκας όπου η πυκνότητα $ p(x,y) $ του φορτίου 
    είναι μέγιστη ή ελάχιστη.
\end{enumerate}

\end{document}
