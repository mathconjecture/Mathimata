\input{preamble_ask.tex}
\input{definitions_ask.tex}
\input{myboxes.tex}

\geometry{top=1cm}

\pagestyle{vangelis}
\everymath{\displaystyle}
\setcounter{chapter}{1}

\begin{document}

\chapter*{Προβλήματα Συνοριακών Τιμών}

\section*{Sturm Liouville}

\subsection*{Ιδιότητες Ιδιοτιμών και Ιδιοσυναρτήσεων}

\begin{enumerate}
  \item Οι ιδιοτιμές του ΠΣΤ είναι πραγματικές, απλές, αριθμήσιμες, διατεταγμένες και 
    υπάρχει ελάχιστη ιδιοτιμή, δηλαδή ισχύει:
    \[
      \lambda _{1} < \lambda _{2} < \lambda _{3} \cdots 
     \]
     και επιπλέον $ \lim_{n \to \infty} \lambda _{n} = \infty $

   \item Έστω $ \lambda _{n} $ ιδιοτιμή με αντίστοιχη ιδιοσυνάρτηση $ \phi _{n}(x) $. 
     Η $ \phi _{n}(x) $ έχει ακριβώς $ n-1 $ σημεία μηδενισμού στο $ (a,b) $.

   \item Το σύνολο των ιδιοσυναρτήσεων είναι πλήρες, δηλαδή κάθε $ f \in L^{2}(a,b) $ 
     μπορεί να αναπαρασταθεί ως γενικευμένη σειρά Fourier 
     \[
       f(x) = \sum_{n=1}^{\infty} c_{n} \phi _{n}(x) 
      \] 

%todo να συνεχίσω τη θεωρια Sturm Liouville


     
\end{enumerate}

\end{document}
