\documentclass[a4paper,table]{report}
\input{preamble_ask.tex}
\input{definitions_ask.tex}
\input{tikz.tex}
\input{myboxes.tex}

\pagestyle{vangelis}
% \everymath{\displaystyle}



\begin{document}

\chapter*{Σειρές \textlatin{Fourier} - Παραδείγματα}


\begin{mybox3}
  \begin{example}
  \item{}
  \item{}
    \begin{enumerate}[i)]
      \item Να βρείτε τη σειρά \textlatin{Fourier} της συνάρτησης $ f(t)=t $ με $ 0 \leq t < 2 \pi $.
      \item Να δείξετε ότι $ \sum_{n=1}^{\infty} \frac{(-1)^{n-1}}{2n-1} = 
        \frac{\pi}{4} $, δηλαδή ότι $ 1 - \frac{1}{3} + \frac{1}{5} - \frac{1}{7}
        + \cdots = \frac{\pi}{4} $.
    \end{enumerate}
  \end{example}
\end{mybox3}
\begin{solution}
\item {}
  \begin{enumerate}[i)]
    \item 
      \begin{minipage}[t]{0.53\textwidth}
        \begin{myitemize}
          \item $f$ συνεχής στο $ (0,2 \pi) $, με $ f(0^{+}) = 0 $ και 
            $ f(2 \pi ^{-}) = 2 \pi $ \hfill\tikzmark{a}
          \item $ f'(x)=1 $ συνεχής στο $ (0,2 \pi ) $ και 
            $ f'(0^{+}) = f'(2 \pi ^{-}) = 1 $. \hfill\tikzmark{b}
        \end{myitemize}
        \mybrace{a}{b}[$f$ τμηματικά λεία στο $ [0.2 \pi] $]
      \end{minipage}

      Επεκτείνουμε κατάλληλα την $f$ ώστε να είναι περιοδική στο $\mathbb{R}$ 
      με περίοδο $ T=2 \pi $.
      \[
        a_{0} = \frac{1}{\pi} \int _{0}^{2 \pi} t\,{dt} = \frac{1}{\pi}
        \left[\frac{t^{2}}{2}\right]_{0}^{2 \pi} = \frac{1}{\pi} \cdot 
        \frac{4 \pi ^{2}}{2} = 2 \pi
      \] 
      \begin{align*}
        a_{n} &= \frac{1}{\pi} \int _{0}^{2 \pi} t \cos{(nt)} \,{dt} = \frac{1}{\pi} 
        \int _{0}^{2 \pi} t \left(\frac{\sin{(nt)}}{n}\right)' \,{dt} = \frac{1}{\pi}
        \left(\left[\frac{t \sin{(nt)}}{n} \right]_{0}^{2 \pi} - \int _{0}^{2 \pi}
        \frac{\sin{(nt)}}{n} \,{dt}\right) \\ 
              &= \frac{1}{\pi} \left(0 - \left[- \frac{\cos{(nt)}}{n^{2}}\right]_{0}^{2 
                \pi}\right) = \frac{1}{\pi} \left(\frac{\cos{(2 \pi n)}}{n^{2}} - 
                  \frac{1}{n^{2}}\right) = \frac{1}{\pi} \left(\frac{1}{n^{2}} - 
                \frac{1}{n^{2}}\right) = 0
      \end{align*} 
      \begin{align*}
        b_{n} &= \frac{1}{\pi} \int _{0}^{2 \pi} t \sin{(nt)} \,{dt} = \frac{1}{\pi} 
        \int _{0}^{2 \pi } t \left(- \frac{\cos{(nt)}}{n}\right)' \,{dt} = \frac{1}{\pi}
        \left(\left[- \frac{t \cos{(nt)}}{n} \right]_{0}^{2 \pi} + \int _{0}^{2 \pi}
        \frac{\cos{(nt)}}{n} \,{dt}\right) \\
              &= \frac{1}{\pi} \left(- \frac{2 \pi \cos{(2 \pi n)}}{n} + \left[
              \frac{\sin{(nt)}}{n^{2}}\right]_{0}^{2 \pi}\right) = 
              \frac{1}{\pi} \left(- \frac{2 \pi}{n} + 0\right) = - \frac{2}{n} 
      \end{align*}
      Άρα η σειρά \textlatin{Fourier} της $f$, είναι:
      \begin{align*}
        f(t) = t &= \pi - 2\sum_{n=1}^{\infty} \frac{1}{n} \sin{(nt)} , 
        \quad \text{για κάθε $t$ σημείο συνέχειας της $f$} \\
        \frac{f(0^{-})+f(0^{+})}{2} = \frac{2 \pi - 0 }{2} = \pi 
                 &= \pi - 2\sum_{n=1}^{\infty} \frac{1}{n} \sin{(nt)} 
                 \quad \text{για $t=0$} 
      \end{align*}

    \item Για $ t= \pi /2 $, όπου είναι σημείο συνέχειας της $f$, έχουμε:
      \[
        \frac{\pi}{2} = \pi - 2 \sum_{n=1}^{\infty} \frac{1}{n} \sin{\frac{n \pi}{2}} =
        \pi - 2 \sum_{n=1}^{\infty} \frac{1}{n} \cdot 
        \begin{cases}
          1, & n=1,5,9,13,\ldots \\
          0, &n=0,2,4,6,8,\ldots \\
          -1, & n=3,7,11,15,\ldots 
        \end{cases} \Leftrightarrow  
        \frac{\pi}{2} = \pi - 2 \sum_{\substack{n=1 \\ (n \; \text{περιτ.})}}^{\infty}
        \frac{(-1)^{\frac{n-1}{2}}}{n} 
      \]
      Οπότε, θέτοντας $ n=2k-1, \; k=1,2,3,\ldots $, έχουμε ότι
      \[
        \frac{\pi}{2} = \pi - 2 \sum_{k=1}^{\infty} 
        \frac{(-1)^{\frac{(2k-1)-1}{2}}}{2k-1} = \pi -2 
          \sum_{k=1}^{\infty} \frac{(-1)^{k-1}}{2k-1} \Leftrightarrow
          \sum_{k=1}^{\infty} \frac{(-1)^{k-1}}{2k-1} = \frac{\pi}{4}
        \]
        Δηλαδή, ισοδύναμα 
        \[
          \sum_{n=1}^{\infty} \frac{(-1)^{n-1}}{2n-1} = \frac{\pi}{4}  
        \]
    \end{enumerate}
\end{solution}

\begin{mybox3}
  \begin{example}
  \item{}
  \item{}
    \begin{enumerate}[i)]
      \item Να βρείτε τη \textbf{συνημιτονική} σειρά \textlatin{Fourier} της συνάρτησης 
        $ f(t)=t^{2} $ με $ 0 \leq t \leq 2 $.
      \item Να δείξετε ότι $ \sum_{n=1}^{\infty} \frac{(-1)^{n-1}}{n^{2}} = 
        \frac{\pi ^{2}}{12} $, δηλαδή ότι $ 1 - \frac{1}{4} + \frac{1}{9} - \frac{1}{16}
        + \cdots = \frac{\pi ^{2}}{12} $.
      \item Να υπολογίσετε το άθροισμα $ \sum_{n=1}^{\infty} \frac{1}{n^{2}} $. 
    \end{enumerate}
  \end{example} 
\end{mybox3}
\begin{solution}
\item {}
  \begin{enumerate}[i)]
    \item 
      Επεκτείνουμε την $f$ στο διάστημα $ [-2,0] $, θέτοντας $ f(t)=t^{2}, \; 
      \forall t \in [-2,0] $ ώστε να είναι \textbf{άρτια} στο $ [-2.2] $.

      Στη συνέχεια επεκτείνουμε κατάλληλα την $f$, ώστε ναι είναι \textbf{περιοδική} στο 
      $ \mathbb{R} $ με $ T=4 \Leftrightarrow 2L=4 \Leftrightarrow \boxed{L=2} $.

      \vspace{\baselineskip}
      \begin{minipage}[t]{0.61\textwidth}
        \begin{myitemize}
          \item $f$ συνεχής στο $ [-2,2] $ \hfill \tikzmark{a}
          \item $ f'(t)=2t $ συνεχής στο $ (-2,2) $ με  και $ f'(-2^{+}) = -4 $ και 
            $ f'(2^{-}) = 4 $ \hfill \tikzmark{b}
        \end{myitemize}
        \mybrace{a}{b}[$f$ τμηματικά λεία στο $[-2,2]$]
      \end{minipage}

      Έχουμε ότι $b_{n}=0, \; \forall n \in \mathbb{N} $, αφού $f$ άρτια.  
      \[
        a_{0} = \frac{1}{2} \int _{-2}^{2} t^{2} \,{dt} =  
        \frac{1}{2} \left[\frac{t^{3}}{3} \right]^{2}_{-2} = \frac{1}{2} 
        \left(\frac{8}{3} - \frac{-8}{3}\right) = \frac{8}{3}
      \]
      \begin{align*}
        a_{n} &= \frac{2}{2} \int _{0}^{2} t^{2} \cos{\left( \frac{n \pi t}{2}\right)} 
        \,{dt} = \int _{0}^{2} t^{2} 
        \left(\frac{\sin{\left(\frac{n \pi t}{2}\right)}}{\frac{n \pi}{2}} \right)' 
        \,{dt} = \left[t^{2} \left(\frac{\sin{\left(\frac{n \pi t}{2}\right)}}{\frac{n \pi}{2}}
          \right)\right]_{0}^{2} - \int _{0}^{2} 2t\left(\frac{\sin{\left(\frac{n \pi t}{2}\right)}}{\frac{n \pi}{2}}
        \right) \,{dt} \\ 
              &= 0 - \frac{4}{n \pi} \int _{0}^{2} 
              t \left( - \frac{\cos{\left(\frac{n \pi t}{2}\right)}}{\frac{n \pi}{2}}
                \right)' \,{dt} = -\frac{4}{n \pi} \left(\left[-t \frac{\cos{\left(\frac{n \pi
                  t}{2}\right)}}{\frac{n \pi}{2}}
                  \right]_{0}^{2}+ \int _{0}^{2}  \frac{\cos{\left(\frac{n \pi
              t}{2}\right)}}{\frac{n \pi}{2}} \,{dt}\right) \\ 
              &= -\frac{4}{n \pi} \left(-2
                \frac{\cos{(n \pi)}}{\frac{n \pi}{2}} + 0 + \frac{2}{n \pi} \left[
              \frac{\sin{(\frac{n \pi t}{2})}}{\frac{n \pi}{2}}\right]_{0}^{2}\right) =
              \frac{16}{n^{2} \pi ^{2}} \cos{(n \pi)} = (-1)^{n} \frac{16}{n^{2} \pi ^{2}} 
      \end{align*}
      Επομένως, η σειρά \textlatin{Fourier}
      \[
        t^{2} =\frac{4}{3} + \frac{16}{\pi ^{2}}\sum_{n=1}^{\infty}\frac{(-1)^{n}}{n^{2}}  
        \cos{\left(\frac{n \pi t}{2}\right)}, \; \forall t \in [-2,2]
      \] 
    \item 
      Για $ t=0 $, έχουμε 
      \[
        0 =  \frac{4}{3} + \frac{16}{\pi ^{2}}\sum_{n=1}^{\infty}\frac{(-1)^{n}}{n^{2}}  
        \cos{0} \Leftrightarrow - \frac{16}{\pi ^{2}}
        \sum_{n=1}^{\infty} \frac{(-1)^{n}}{n^{2}} = \frac{4}{3} \Leftrightarrow
        \sum_{n==1}^{\infty} \frac{(-1)^{n-1}}{n^{2}} = \frac{\pi ^{2}}{12}
      \] 
    \item 
      Για $ t=2 $, έχουμε
      \[
        2^{2} = \frac{4}{3} + \frac{16}{\pi ^{2}}\sum_{n=1}^{\infty}\frac{(-1)^{n}}{n^{2}}  
        \cos{n \pi} \Leftrightarrow \frac{8}{3} = \frac{16}{\pi^{2}} \sum_{n=1}^{\infty}
        \frac{(-1)^{n}}{n^{2}} (-1)^{n} \Leftrightarrow 
        \sum_{n=1}^{\infty} \frac{(-1)^{2n}}{n^{2}} = \frac{\pi ^{2}}{6} \Leftrightarrow 
        \sum_{n=1}^{\infty} \frac{1}{n^{2}} = \frac{\pi ^{2}}{6} 
      \] 
  \end{enumerate}
\end{solution}

\begin{mybox3}
  \begin{example}
  \item{}
  \item{}
    \begin{enumerate}[i)]
      \item Να βρείτε την \textbf{ημιτονική} σειρά \textlatin{Fourier} της συνάρτησης 
        $ f(t)=t $ με $ 0 \leq t \leq 2 $.
      \item Να δείξετε ότι $ \sum_{n=1}^{\infty} \frac{(-1)^{n-1}}{2n-1} = 
        \frac{\pi}{4} $, δηλαδή ότι $ 1 - \frac{1}{3} + \frac{1}{5} - \frac{1}{7}
        + \cdots = \frac{\pi}{4} $.
    \end{enumerate}
  \end{example}
\end{mybox3}
\begin{solution}
\item {}
  \begin{enumerate}[i)]
    \item 
      Επεκτείνουμε την $f$ στο διάστημα $ [-2,0] $, θέτοντας $ f(t)=t, \; 
      \forall t \in [-2,0] $ ώστε να είναι \textbf{περιττή} στο $ [-2.2] $.

      Στη συνέχεια επεκτείνουμε κατάλληλα την $f$, ώστε ναι είναι 
      \textbf{περιοδική} στο 
      $ \mathbb{R} $ με $ T=4 \Leftrightarrow 2L=4 \Leftrightarrow \boxed{L=2} $.

      \vspace{\baselineskip}
      \begin{minipage}[t]{0.53\textwidth}
        \begin{myitemize}
          \item $f$ συνεχής στο $ [-2,2] $ \hfill \tikzmark{a}
          \item $ f'(t)=1 $ συνεχής στο $ (-2,2) $ και $ f'(-2^{+}) = 
            f'(2^{-}) = 1 $ \hfill \tikzmark{b}
        \end{myitemize}
        \mybrace{a}{b}[$f$ τμηματικά λεία στο $[-2,2]$]
      \end{minipage}

      Έχουμε ότι $ a_{0}=0 $ και $α_{n}=0, \; \forall n \in \mathbb{N} $, 
      αφού $f$ περιττή.  
      \begin{align*}
        b_{n} &= \frac{2}{2} \int _{0}^{2} t \sin{\left(\frac{n \pi t}{2}\right)}
        \,{dt} = \int _{0}^{2} t \left(-
        \frac{\cos{(\frac{n \pi t}{2})}}{\frac{n \pi}{2}}\right)' \,{dt} =
        \left(\left[- \frac{t \cos{(\frac{n \pi t}{2})}}{\frac{n \pi}{2}}
          \right]_{0}^{2} + \frac{2}{n \pi}\int _{0}^{2}
        \cos{\left(\frac{n \pi t}{2}\right)} \,{dt}\right) \\
              &= \frac{2}{n \pi} \left( -2 \cos{(n \pi)} + 0 + 
                \left[\frac{\sin{(\frac{n \pi t}{2})}}{\frac{n \pi}{2}}
              \right]_{0}^{2}\right) = \frac{2}{n \pi} \left(-2 (-1)^{n} + 0\right) = 
              \frac{4 (-1)^{n+1}}{n \pi} 
      \end{align*}
      Επομένως, η σειρά \textlatin{Fourier}
      \[
        t = \frac{4}{\pi} \sum_{n=1}^{\infty} \frac{(-1)^{n+1}}{n} \sin{\left(\frac{n \pi t}{2}\right)}, 
        \quad \forall t \in (-2,2)
      \] 
      \[
        \frac{f(-2^{+})+ f(2^{-})}{2} = \frac{-2 + 2}{2} = 0 = \frac{4}{\pi}
        \sum_{n=1}^{\infty} \frac{(-1)^{n+1}}{n} \sin{0}, \quad \text{για } t = -2,2
      \] 
    \item 
      Για $ t=1 $ έχουμε
      \[
        1 = \frac{4}{\pi} \sum_{n=1}^{\infty} \frac{(-1)^{n+1}}{n}\sin{\left( \frac{n
        \pi}{2} \right)} \Leftrightarrow 
        1 = \frac{4}{\pi} \sum_{n=1}^{\infty} \frac{(-1)^{n+1}}{n} \cdot 
        \begin{cases}
          1, & n=1,5,9,13,\ldots \\
          0, &n=0,2,4,6,8,\ldots \\
          -1, & n=3,7,11,15,\ldots 
        \end{cases}
      \] 
      \[
        1 = \frac{4}{\pi} \sum_{\substack{n=1 \\ n \; \text{περιτ.}}}^{\infty} 
        \frac{(-1)^{n+1}}{n} (-1)^{\frac{n-1}{2}} \Leftrightarrow 1 = \frac{4}{\pi}
        \sum_{n=1}^{\infty} \frac{(-1)^{2n}}{2n-1} (-1)^{n-1}
        \Leftrightarrow 
        \sum_{n=1}^{\infty} \frac{(-1)^{n-1}}{2n-1} = \frac{\pi}{4}
      \] 
  \end{enumerate}
\end{solution}

\begin{mybox3}
  \begin{example}
    Να βρεθεί η σειρά \textlatin{Fourier} της συνάρτησης $ f(t) = e^{at} $, με $ a \neq 0 $, αν 
    $ - \pi < t \leq \pi $.
  \end{example}
\end{mybox3}
\begin{solution}
  \[
    a_{0} = \frac{1}{\pi} \int _{- \pi }^{\pi} \mathrm{e}^{at} \,{dx} = 
    \frac{1}{\pi} \left[\frac{\mathrm{e}^{at}}{a} \right] _{- \pi }^{\pi} = 
    \frac{1}{a\pi} (\mathrm{e}^{a \pi} - \mathrm{e}^{-a \pi}) = \frac{2}{a \pi} 
    \sinh{(a \pi)} 
  \]
  \[
    a_{n} = \frac{1}{\pi} \int_{-\pi}^{\pi} \mathrm{e}^{at} \cos{(nt)} \,{dt}
  \]
  Οπότε, για τον υπολογισμό του ολοκληρώματος, έχουμε:
  \begin{align*}
    \underbrace{\int_{-\pi}^{\pi} \mathrm{e}^{at} \cos{(nt)}}_{I} \,{dt} 
    &= \int_{-\pi}^{\pi} \left(\frac{\mathrm{e}^{at}}{a} \right)' \cos{(nt)}\,{dt} 
    = \left[\frac{\mathrm{e}^{at}}{a} \cos{(nt)} 
    \right]_{- \pi }^{\pi} + \int_{-\pi}^{\pi} \frac{\mathrm{e}^{at}}{a} n \sin{(nt)}
    \,{dt} \\
    &= \left[\frac{\mathrm{e}^{at}}{a} \cos{(nt)} 
    \right]_{- \pi }^{\pi} + \frac{n}{a}  \int_{-\pi}^{\pi}
    \left(\frac{\mathrm{e}^{at}}{a}\right)' \sin{(nt)} \,{dt} \\
    &= \left[\frac{\mathrm{e}^{at}}{a} \cos{(nt)} 
    \right]_{- \pi }^{\pi} + \frac{n}{a} 
    \left(\left[\frac{\mathrm{e}^{at}}{a} \sin{(nt)} 
      \right]_{- \pi }^{\pi }-\int_{-\pi}^{\pi} \frac{\mathrm{e}^{at}}{a} n 
    \cos{(nt)} \,{dt}\right) \\ 
    &= \left[\frac{\mathrm{e}^{at}}{a} \cos{(nt)} 
    \right]_{- \pi }^{\pi} + \frac{n}{a} 
    \left[\frac{\mathrm{e}^{at}}{a} \sin{(nt)} 
    \right]_{- \pi }^{\pi }- \frac{n^{2}}{a^{2}}
    \underbrace{\int_{-\pi}^{\pi}
    \mathrm{e}^{at} \cos{(nt)} \,{dt}}_{I} \Rightarrow \\ 
  \end{align*}
  \[
    I+ \frac{n^{2}}{a^{2}} I =  \left[\frac{\mathrm{e}^{at}}{a} \cos{(nt)} 
    \right]_{- \pi }^{\pi} \Leftrightarrow 
    I = \frac{a^{2}}{a^{2}+n^{2}} \frac{1}{a}
    \left(\mathrm{e}^{a \pi} \cos{(n \pi)} - \mathrm{e}^{-a \pi} \cos{(-n \pi)}\right)
  \]
  \[
    I = \frac{a^{2}}{a^{2}+n^{2}} \frac{1}{a}
    \left(\mathrm{e}^{a \pi} \cos{(n \pi)} - \mathrm{e}^{-a \pi} \cos{(n \pi)}\right)
  \] 
  \[
    I = \frac{a}{a^{2}+n^{2}} (-1)^{n} \left(\mathrm{e}^{a \pi} - 
    \mathrm{e}^{-a \pi}\right) = \frac{2a(-1)^{n}}{a^{2}+n^{2}} \sinh{(a \pi)} 
  \] 
  Άρα 
  \[
    a_{n} = (-1)^{n}\frac{2a}{a^{2}+n^{2}} \frac{\sinh{(a \pi)}}{\pi}
  \] 
  Ομοίως, 
  \[
    b_{n} = \frac{1}{\pi} \int_{-\pi}^{\pi} \mathrm{e}^{at} \sin{(nt)} \,{dt}
  \]
  Οπότε, για τον υπολογισμό του ολοκληρώματος, έχουμε:
  \begin{align*}
    \underbrace{\int_{-\pi}^{\pi} \mathrm{e}^{at} \sin{(nt)}}_{I} \,{dt} 
    &= \int_{-\pi}^{\pi} \left(\frac{\mathrm{e}^{at}}{a} \right)' \sin{(nt)}\,{dt} 
    = \left[\frac{\mathrm{e}^{at}}{a} \sin{(nt)} 
    \right]_{- \pi }^{\pi} - \int_{-\pi}^{\pi} \frac{\mathrm{e}^{at}}{a} n \cos{(nt)}
    \,{dt} \\
    &= \left[\frac{\mathrm{e}^{at}}{a} \sin{(nt)} 
    \right]_{- \pi }^{\pi} - \frac{n}{a}  \int_{-\pi}^{\pi}
    \left(\frac{\mathrm{e}^{at}}{a}\right)' \cos{(nt)} \,{dt} \\
    &= \left[\frac{\mathrm{e}^{at}}{a} \sin{(nt)} 
    \right]_{- \pi }^{\pi} - \frac{n}{a} 
    \left(\left[\frac{\mathrm{e}^{at}}{a} \cos{(nt)} 
      \right]_{- \pi }^{\pi }+\int_{-\pi}^{\pi} \frac{\mathrm{e}^{at}}{a} n 
    \sin{(nt)} \,{dt}\right) \\ 
    &= \left[\frac{\mathrm{e}^{at}}{a} \sin{(nt)} 
    \right]_{- \pi }^{\pi} - \frac{n}{a} 
    \left[\frac{\mathrm{e}^{at}}{a} \cos{(nt)} 
    \right]_{- \pi }^{\pi }- \frac{n^{2}}{a^{2}}
    \underbrace{\int_{-\pi}^{\pi}
    \mathrm{e}^{at} \sin{(nt)} \,{dt}}_{I} \Rightarrow \\ 
  \end{align*}
  \[
    I+ \frac{n^{2}}{a^{2}} I = - \frac{n}{a} \left[\frac{\mathrm{e}^{at}}{a} \cos{(nt)} 
    \right]_{- \pi }^{\pi} \Leftrightarrow 
    I = \frac{a^{2}}{a^{2}+n^{2}} \left(-\frac{n}{a^{2}}\right)
    \left(\mathrm{e}^{a \pi} \cos{(n \pi)} - \mathrm{e}^{-a \pi} \cos{(-n \pi)}\right)
  \]
  \[
    I = \frac{-n}{a^{2}+n^{2}} \left(\mathrm{e}^{a \pi} \cos{(n \pi)} - \mathrm{e}^{-a \pi} \cos{(n \pi)}\right)
  \] 
  \[
    I = \frac{-n}{a^{2}+n^{2}} (-1)^{n} \left(\mathrm{e}^{a \pi} - 
    \mathrm{e}^{-a \pi}\right) = \frac{2n(-1)^{n+1}}{a^{2}+n^{2}} \sinh{(a \pi)} 
  \] 
  Άρα 
  \[
    b_{n} = (-1)^{n+1}\frac{2n}{a^{2}+n^{2}} \frac{\sinh{(a \pi)}}{\pi}
  \] 
\end{solution}


\begin{mybox3}
  \begin{example}
    Να βρεθεί η μιγαδική σειρά \textlatin{Fourier} της συνάρτησης 
    $ f(t) = e^{at} $, με $ a \neq 0 $, αν $ - \pi < t \leq \pi $.
  \end{example}
\end{mybox3}
\begin{solution}
  Θα υπολογίσουμε τη \textbf{μιγαδική} σειρά \textlatin{Fourier,}  
  $ \sum\limits_{n=- \infty}^{\infty} c_{n} \mathrm{e}^{int} $. Οπότε, 
  \begin{align*}
    c_{n} &= \frac{1}{2 \pi} \int _{- \pi }^{\pi} \mathrm{e}^{at} \mathrm{e}^{-int } 
    \,{dt} = \frac{1}{2 \pi} \int _{- \pi }^{\pi} \mathrm{e}^{(a-in)t} \,{dt} = 
    \frac{1}{2 \pi} \left[\frac{\mathrm{e}^{(a-in)t}}{a- in} \right]_{- \pi }^{\pi} = 
    \frac{1}{2 \pi} \frac{1}{a-in} \left[\mathrm{e}^{at} (\cos{nt} - i 
    \sin{nt})\right]_{- \pi }^{\pi} \\
          &= \frac{1}{2 \pi} \frac{a+in}{a^{2}+n^{2}} 
          \left[\mathrm{e}^{a \pi} (\cos{(n \pi)- i \cancel{\sin{(n \pi)}}) - 
            \mathrm{e}^{-a \pi } (\cos{(n \pi )}}
          +i \cancel{\sin{(n \pi)}} )\right] = (-1)^{n} \frac{a+in}{a^{2}+n^{2}} 
          \frac{\sinh{(a \pi)}}{\pi}
  \end{align*}
  Από τις σχέσεις $ a_{0}= c_{0}, \; a_{n}= c_{n}+c_{-n} $ και 
  $ b_{n}=i(c_{n}-c_{-n}) $, προκύπτουν οι πραγματικοί συντελεστές \textlatin{Fourier}
  \begin{gather*}
    a_{0}= \frac{\sinh{(a \pi)}}{a\pi}  \\
    a_{n}= (-1)^{n} \frac{2a}{a^{2}+n^{2}} \frac{\sinh{(a \pi)}}{\pi} \\
    b_{n} = (-1)^{n+1} \frac{2n}{a^{2}+n^{2}} \frac{\sinh{(a \pi)}}{\pi} 
  \end{gather*} 
\end{solution}
Στα σημεία ασυνέχειας $ t_{m} = \pm (2m-1) \pi, \; m=1,2,3,\ldots $ η σειρά
\textlatin{Fourier} 
συγκλίνει στην τιμή 
\[
  \sum_{n=- \infty}^{\infty} c_{n} \mathrm{e}^{\pm in (2m-1) \pi} = 
  \frac{\mathrm{e}^{a \pi} + \mathrm{e}^{- a \pi}}{2} = \cosh{(a \pi)}  
\] 

\begin{mybox3}
\begin{example}
  Αν η σειρά \textlatin{fourier} της συνάρτησης $f$ συγκλίνει ομοιόμορφα προς την $f(x)$ στο 
  διάστημα $ (-L,L) $, να αποδείξετε την ισότητα \textlatin{Parseval}.
\end{example}
\end{mybox3}
\begin{solution}
  Αφού η σειρά \textlatin{fourier} συγκλίνει ομοιόμορφα προς στην $ f(x) $, έχουμε:
  \[
    f(x) = \frac{a_{0}}{2} + \sum_{n=1}^{\infty} \left[a_{n} 
    \cos{\left(\frac{n \pi x}{L}\right)} + b_{n} \sin{\left(\frac{n \pi x}{L}\right)}\right], 
    \quad \forall x \in \mathbb{R} 
  \] 
  Επομένως, πολλαπλασιάζουμε και τα δύο μέλη με $ f(x) $ και ολοκληρώνουμε όρο 
  προς όρο, δεξί μέλος, γεγονός που επιτρέπεται, λόγω της ομοιόμορφης σύγκλισης της 
  σειράς και έχουμε:
  \begin{gather*}
    \int _{-L}^{L} f^{2}(x) \,{dx} 
    = \frac{a_{0}}{2} \underbrace{\int _{-L}^{L} f(x) \,{dx}}_{a_{0}L} + 
    \sum_{n=1}^{\infty} \Biggl[a_{n} \underbrace{\int _{-L}^{L} f(x) 
      \cos{\left(\frac{n \pi x}{L}\right)} \,{dx}}_{a_{n}L} + b_{n} 
      \underbrace{\int _{-L}^{L} f(x) \sin{\left(\frac{n \pi x}{L}\right)} 
    \,{dx}}_{b_{n}L} \ \Biggr] \Leftrightarrow \\ 
    \int _{-L}^{L} f^{2}(x) \,{dx}  = \frac{a_{0}^{2}}{2} L + 
    \sum_{n=1}^{\infty} (a_{n}^{2}L+b_{n}^{2}L) \Leftrightarrow \\
    \frac{1}{L} \int _{-L}^{L} f^{2}(x) \,{dx}  = \frac{a_{0}^{2}}{2}
    + \sum_{n=1}^{\infty} (a_{n}^{2}+b_{n}^{2}) \\
  \end{gather*}
\end{solution}

\enlargethispage*{\baselineskip}

\begin{mybox3}
  \begin{example}
  \item{}
  \item{}
    \begin{enumerate}[i)]
      \item Να βρείτε την \textbf{συνημιτονική} σειρά \textlatin{Fourier} της συνάρτησης 
        $ f(t)=t $ με $ 0 < t < 2 $.
      \item Να δείξετε ότι $ \sum_{n=1}^{\infty} \frac{1}{(2n-1)^{4}} = 
        \frac{\pi^{4}}{96} $, δηλαδή ότι $ 1 + \frac{1}{3^{4}} + \frac{1}{5^{4}} +
        \frac{1}{7^{4}} + \cdots = \frac{\pi ^{4}}{96} $.
      \item Να δείξετε ότι $ \sum_{n=1}^{\infty} \frac{1}{n^{4}} = 
        \frac{\pi^{4}}{90} $, δηλαδή ότι $ 1 + \frac{1}{2^{4}} + \frac{1}{3^{4}} +
        \frac{1}{4^{4}} + \cdots = \frac{\pi ^{4}}{90} $.
    \end{enumerate}
  \end{example}
\end{mybox3}
\begin{solution}
\item {}
  \begin{enumerate}[i)]
    \item 
      Επεκτείνουμε την $f$ στο διάστημα $ [-2,0] $, θέτοντας $ f(t)=-t, \; 
      \forall t \in [-2,0] $ ώστε να είναι \textbf{άρτια} στο $ [-2.2] $.

      Στη συνέχεια επεκτείνουμε κατάλληλα την $f$, ώστε ναι είναι 
      \textbf{περιοδική} στο 
      $ \mathbb{R} $ με $ T=4 \Leftrightarrow 2L=4 \Leftrightarrow \boxed{L=2} $.

      \vspace{\baselineskip}
      \begin{minipage}[t]{0.53\textwidth}
        \begin{myitemize}
          \item $f$ συνεχής στο $ [-2,2] $ \hfill \tikzmark{a}
          \item $ f'(t)=1 $ συνεχής στο $ (-2,2) $ και $ f'(-2^{+}) = 
            f'(2^{-}) = 1 $ \hfill \tikzmark{b}
        \end{myitemize}
        \mybrace{a}{b}[$f$ τμηματικά λεία στο $[-2,2]$]
      \end{minipage}

      Έχουμε ότι $ b_{n}=0 , \; \forall n \in \mathbb{N} $, 
      αφού $f$ άρτια.  
      \begin{align*}
        a_{0} &= \frac{2}{2} \int _{0}^{2} t \,{dt} = 
        \left[\frac{t^{2}}{2}\right]_{0}^{2} = 2
      \end{align*}
      \begin{align*}
        a_{n} &= \frac{2}{2} \int _{0}^{2} t \cos{\left(\frac{n \pi t}{2}\right)}
        \,{dt} = \int _{0}^{2} t \left(
        \frac{\sin{(\frac{n \pi t}{2})}}{\frac{n \pi}{2}}\right)' \,{dt} =
        \left(\left[ \frac{t \sin{(\frac{n \pi t}{2})}}{\frac{n \pi}{2}}
          \right]_{0}^{2} - \frac{2}{n \pi}\int _{0}^{2}
        \sin{\left(\frac{n \pi t}{2}\right)} \,{dt}\right) \\
              &= -\frac{2}{n \pi}  
              \left[\frac{-\cos{(\frac{n \pi t}{2})}}{\frac{n \pi}{2}}
              \right]_{0}^{2} = \frac{4}{n^{2} \pi^{2}} (\cos{(n \pi)} - 1) = 
              \frac{4}{n^{2} \pi^{2}} \bigl((-1)^{n}- 1\bigr) = 
              \begin{cases}
                \frac{4}{n^{2} \pi ^{2}} (-2) , & \text{$n$ περιττός} \\
                0, & \text{$n$ άρτιος}
              \end{cases}
      \end{align*}
      Επομένως, η σειρά \textlatin{Fourier}
      \[
        t = 1 - \frac{8}{\pi^{2}} \sum_{n=1}^{\infty} \frac{1}{(2n-1)^{2}}
        \cos{\left(\frac{(2n-1) \pi t}{2}\right)}, 
        \quad \forall t \in \mathbb{R} 
      \] 
    \item Από την ταυτότητα \textlatin{Parseval} έχουμε
      \[
        \frac{1}{2} \int _{-2}^{2} x^{2} \,{dx} = \frac{2^{2}}{2} +
        \sum_{n=1}^{\infty} \frac{16}{n^{4} \pi ^{4}} (\cos{{ (n \pi )-1 }})^{2} 
        \Leftrightarrow \frac{8}{3} = 2 + \frac{16}{\pi ^{4}}
        \sum_{\substack{n=1 \\ n \; \text{περ.}}}^{\infty} 
        \frac{4}{n^{4}}  
        \Leftrightarrow \frac{2}{3} = \frac{64}{\pi ^{4}} \sum_{n=1}^{\infty}
        \frac{1}{(2n-1)^{4}} 
      \] 
      άρα 
      \[
        \sum_{n=1}^{\infty} \frac{1}{(2n-1)^{4}} = \frac{\pi^{4}}{96}
      \]
    \item 
      Για να βρούμε το άθροισμα 
      $ S = \sum_{n=1}^{\infty} \frac{1}{n^{4}} = \frac{1}{1^{4}} + \frac{1}{2^{4}} +
      \frac{1}{3{4}} + \cdots $, έχουμε ότι 
      \[
        S =   \frac{1}{1^{4}} + \frac{1}{2^{4}} + \frac{1}{3^{4}} + \cdots =
        \Bigl(\frac{1}{1^{4}} + \frac{1}{3^{4}} + \frac{1}{5^{4}} + \cdots\Bigr) +
        \Bigl(\frac{1}{2^{4}} + \frac{1}{4^{4}} + \frac{1}{6^{4}} + \cdots\Bigr) 
        = \frac{\pi ^{4}}{96} + 
        \frac{1}{2^{4}} \Bigl(\frac{1}{1^{4}} + \frac{1}{2^{4}} + \frac{1}{3^{4}} +
        \cdots\Bigr) 
      \]
      Άρα
      \[
      S = \frac{\pi ^{4}}{96} + \frac{1}{2^{4}} S \Rightarrow \frac{15}{16}S
      = \frac{\pi ^{4}}{96} \Rightarrow S = \frac{\pi ^{4}}{90}
      \]
  \end{enumerate}
\end{solution} 


%todo Να γράψω παραδείγματα με παραγωγιση και ολοκληρωση σειρων fourier


\end{document}

