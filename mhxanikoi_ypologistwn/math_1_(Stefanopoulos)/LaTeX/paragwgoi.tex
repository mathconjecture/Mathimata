\input{preamble_ask.tex}
\input{definitions_ask.tex}


\pagestyle{askhseis}

\begin{document}


\begin{center}
  \minibox{\large \bfseries \textcolor{Col1}{Ασκήσεις στις Παραγώγους}}
\end{center}

\vspace{\baselineskip}


\begin{enumerate}

  \section*{Παράγωγος}

  % \item Να βρείτε τα $ a, b \in \mathbb{R} $ έτσι ώστε η ευθεία $ y = 2x + 5
  %   $ να είναι εφαπτομένη της συνάρτησης $ f(x) = x^{2} + ax + b $ στο
  %   σημείο $ x_{0} = -1 $. 
  %   \hfill Απ: $ a = 4, b = 6 $

  % \item Να εξεταστεί πλήρως ως προς τη συνέχεια και την παραγωγισιμότητα η
  %   συνάρτηση $ f(x) = e^{\abs{x}} $.
  %   \hfill Απ: συνεχής, όχι παραγωγίσιμη 

  \item Να βρεθούν τα $ a, b \in \mathbb{R} $ έτσι ώστε η συνάρτηση 
    $
    f(x) = \begin{cases}
      x^{2}, & x\geq 2 \\
      ax+b , & x<2
    \end{cases}
    $
    να είναι παραγωγίσιμη στο $ x_{0} = 2 $.

    \hfill Απ: $ a=4, b=-4 $

  \item Να υπολογιστούν οι παράγωγοι των παρακάτω συναρτήσεων με 2 τρόπους.
    \begin{enumerate}[(i)]
      \item $ f(x) = \ln{\left(\sqrt{1+3x^{2}\strut}\right)} $ \hfill Απ: $
        \frac{3x}{1+3x^{2}} $
      \item $ f(x) = \ln({\sin({\cos{x}})}) $ \hfill Απ: $
        \frac{1}{\sin{(\cos{x})}} [\cos{(\cos{x})}] (- \sin{x}) $ 
      \item $ f(x)= \arcsin(\frac{1}{x}) $ \hfill Απ: $ - \frac{1}{x \sqrt{x^{2}-1}} $ 
      % \item $ f(x) = \arctan \left(\frac{x}{\sqrt{1 + x^{2}}}\right) $ \hfill Απ: $
      %   \frac{1}{(1+2x^{2})\sqrt{1 + x^{2}}} $
    \end{enumerate}

    \subsection*{Λογαριθμική Παραγώγιση}

  \item  Να υπολογιστούν οι παράγωγοι των παρακάτω συναρτήσεων

    \begin{enumerate}[(i)]
      \item $ f(x) = (\cos{x})^{\sin{2x}} $ \hfill Απ: $
        (\cos{x})^{\sin{2x}} 2(\cos{2x} \ln{(\cos{x})} - \sin^{2}{x}) $
      \item $ f(x) = \left(1 + \frac{1}{x} \right)^{x} $ \hfill Απ: $
        \left(1 + \frac{1}{x}\right)^{x}\left[\ln{(1 + \frac{1}{x})} -
        \frac{1}{x+1}\right] $
      \item $ f(x)=(\sin{x})^{x} $ \hfill Απ: $ (\sin{x})^{x}[\ln{(\sin{x}
        )} + x \cot{x}] $ 
      \item $ f(x)=\cos{x}^{x} $ \hfill Απ: $ (- \sin{x^{x}})x^{x} (1 +
        \ln{x}) $
    \end{enumerate}

    % \subsection*{Παράγωγος της Αντίστροφης}
    % \item Να βρεθούν οι παράγωγοι των αντίστροφων, των παρακάτω συναρτήσεων.
    % \begin{enumerate}[(i)]
    %   \twocolumnside{
    %     \item $ y = \cos{x} $ \hfill Απ: $ \frac{-1}{\sqrt{1 - y^{2}}} $
    %     \item $ y = \tan{x} $ \hfill Απ: $ \frac{1}{1 + y^{2}} $
    %       }{
    %     \item $ y = \cosh{x} $  \hfill Απ: $ \frac{1}{\sqrt{y^{2} - 1}} $
    %     \item $ y = \tanh{x} $ \hfill Απ: $ \frac{1}{x^{2} - 1} $
    %     }
    % \end{enumerate}

    \subsection*{Πεπλεγμένη Παραγώγιση}

    \item Δίνεται η σχέση $ x^{2} - xy + y^{2} = 3 $, $ y=y(x) $. Να βρεθεί η 1η
      και η 2η παράγωγος της $y$ ως προς $x$ στο σημείο $ (1,-1) $.

      \hfill Απ: $ y' = 1$, $ y'' = \frac{2}{3} $

    \item Δίνεται η σχέση $ 4x^{3} - 3xy^{2} + 6x^{2} - 5xy - 8 y^{2} + 9x + 14
      = 0$. Να βρείτε τις εξισώσεις της εφαπτομένης και της κάθετης ευθείας
      της καμπύλης στο σημείο $ (-2,3) $.

      \hfill Απ: $\varepsilon\colon y = \frac{9}{2} x - 6 $, 
      $\kappa\colon y = \frac{2}{9} x + \frac{31}{9} $.


      \subsection*{Διαφορικό}

    \item Να υπολογιστούν κατά προσέγγιση οι τιμές με τη βοήθεια του διαφορικού:
      \begin{enumerate}[i)]
        \item $\sqrt{104}$ \hfill Απ: $10,2$
        \item $\sqrt[4]{17}$ \hfill Απ: $\frac{1}{4}17^{-\frac{3}{4}}+2$
      \end{enumerate}

    \item Να υπολογιστεί κατά προσέγγιση, με τη βοήθεια του διαορικού, η μεταβολή 
      της συνάρτησης $ f(x) = x^{2/5} $, όταν:
      \begin{enumerate}[i)]
        \item Το $x$ μεταβάλλεται από το 32 στο 34 
        \item Το $x$ μεταβάλλεται από το 1 στο 0,9.
      \end{enumerate}

      \hfill Απ: 
      \begin{enumerate*}[i),itemjoin=\hspace{1em}]
        \item $ \Delta f \approx 0,1 $ 
        \item $ \Delta f \approx -0,04 $ 
      \end{enumerate*}


      \subsection*{L' Hospital}

    \item Να υπολογιστούν τα παρακάτω όρια.
      \begin{enumerate}[i)]
        \twocolumnside{
          %A Desmi p.217
          \item $ \lim_{x \to 0} \frac{\mathrm{e}^{x} - x -1}{x^{2}} $ \hfill Απ: $1/2$  
          \item $ \lim_{x \to \infty} x \ln{(1+ \frac{1}{x})} $ \hfill Απ: $1$ 
          \item $ \lim_{x \to \infty} (x - \ln{x}) $ \hfill Απ: $ \infty $  
          \item $ \lim_{x \to \infty} (1+2x)^{1/x} $ \hfill Απ: $ 1 $ 
          \item $ \lim_{x \to 0^{+}} (1+x)^{\cot{x}} $ \hfill Απ: $ \mathrm{e} $ 
            }{
          \item $ \lim_{x\to 1} \left(\frac{1}{\ln{x}} - \frac{1}{x-1}\right) $ \hfill
            Απ: $ \frac{1}{2} $
          \item $ \lim_{x\to +\infty} \left(1 + \frac{1}{x} +
            \frac{2}{x^{2}}\right)^{x} $ \hfill Απ: $ e $ 
          \item $ \lim_{x\to 0^{+}} \left(\frac{1}{x}\right)^{\sin{x}} $ \hfill $ 1 $
          \item $ \lim_{x\to 0} \left(\cos{2x}\right)^{\frac{3}{x^{2}}}  $ \hfill Απ:
            $ e^{-6} $
          }
      \end{enumerate}


      \section*{Θεωρήματα Διαφορικού Λογισμού}

      \subsection*{Ύπαρξη Ρίζας}

      %A Desmi p.179
    \item Να δείξετε ότι η εξίσωση $ 3x^{5}-x^{3}=5 $ έχει ακριβώς μία ρίζα στο διάστημα 
      $ (1,2) $.
      %A Desmi p.176
    \item Να δείξετε ότι η εξίσωση $ \cos{x} + 1 = x $ έχει ακριβώς μία ρίζα στο διάστημα 
      $ (0, \pi/2) $. 

    \item Να αποδείξετε ότι η εξίσωση $ x^{2} = x \sin{x} + \cos{x} $ 
      έχει δύο ακριβώς πραγματικές ρίζες $ x_{1} $, $ x_{2} $, με $ x_{1} \in (-\pi, 0) $, 
      και $x_{2} \in (0, \pi) $.

      \subsection*{Ανισότητες}

    \item Να αποδείξετε τις παρακάτω ανισότητες
      \begin{enumerate}[i)]
        \item $ 2 \sqrt{x} \geq 3 - \frac{1}{x} $, για κάθε $ x>0 $.
        \item $ \cos{x} > 1 - \frac{x^{2}}{2} $, για κάθε $ x>0 $.
        \item $ 1- \frac{1}{x} \leq \ln{x} \leq x-1 $, για κάθε $ x>0 $.
      \end{enumerate}

      \subsection*{Θεώρημα Μέσης Τιμής}

    \item Να αποδείξετε ότι για κάθε $x,y \in \mathbb{R}$ με $ x \neq y $, ισχύουν 
      \begin{enumerate}[i)]
        \item $ \abs{ \cos^{2}{x} - \cos^{2}{y}} \leq \abs{x-y} $ 
        \item $ \mathrm{e}^{x} < \frac{\mathrm{e}^{x} - \mathrm{e}^{y}}{x-y} <
          \mathrm{e}^{y} $, αν $ x<y $.
      \end{enumerate}

    \item Να αποδείξετε την παρακάτω ανισότητα   
      \[
        \frac{a - b}{\cos^{2}{b}} \leq \tan{a} - \tan{b}\leq \frac{a -
        b}{\cos^{2}{a}}, \qq{με}  0 < b \leq a < \frac{\pi}{2}
      \]

    \item Να αποδείξετε την ανισότητα 
      \[
        \frac{a-b}{a} \leq \ln{\frac{a}{b}} \leq \frac{a-b}{b}, \qq{με}  0<b\leq a 
      \]

      %A Desmi p.178
    \item Να αποδείξετε την παρακάτω ανισότητα 
      $ x+1 \leq \mathrm{e}^{x} \leq x \mathrm{e}^{x} + 1 $ για κάθε $ x \in 
      \mathbb{R} $ 

      \enlargethispage{3\baselineskip}

      \subsection*{Θεώρημα Taylor}

    \item Έστω $ f(x) = \ln{(1+x)} $, $ x>-1 $. Να υπολογιστεί το ανάπτυγμα
      Maclaurin μέχρι και όρους 4ης τάξης και στη συνέχεια να
      υπολογιστεί το αντίστοιχο σφάλμα για $ x = 0,1 $.

      \hfill Απ: \begin{tabular}{l}
        $ \ln(1+x) \cong x - \frac{1}{2} x^{2} + \frac{1}{3}x^{3} - 
        \frac{1}{4} x^{4} $ \\
        $ \abs{R_{4}(0,1)} < 0,000002$	
      \end{tabular}

    \item Να δείξετε ότι 
      \[
        \sin{x} \cong \sin{a} + \cos{a} (x-a) - \frac{\sin{a}}{2!} (x-a)^{2} -
        \frac{\cos{\xi} (x-a)^{3}}{3!}
      \]
      όπου $\xi$ μεταξύ $a$ και $x$. Στη συνέχεια να υπολογίσετε το $
      \sin{\ang{51}}$ καθώς και το διαπραττόμενο σφάλμα.

      \hfill Απ: $ \abs{R_{3}(\ang{51})} < 0,00019 $
  \end{enumerate}


  \end{document}
