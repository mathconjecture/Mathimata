\documentclass[a4paper,table]{report}
\input{preamble_themata}
\newcommand{\vect}[2]{(#1_1,\ldots, #1_#2)}
%%%%%%% nesting newcommands $$$$$$$$$$$$$$$$$$$
\newcommand{\function}[1]{\newcommand{\nvec}[2]{#1(##1_1,\ldots, ##1_##2)}}

\newcommand{\linode}[2]{#1_n(x)#2^{(n)}+#1_{n-1}(x)#2^{(n-1)}+\cdots +#1_0(x)#2=g(x)}

\newcommand{\vecoffun}[3]{#1_0(#2),\ldots ,#1_#3(#2)}

\newcommand{\mysum}[1]{\sum_{n=#1}^{\infty}

\input{myboxes}

\everymath{\displaystyle}
\pagestyle{vangelis}


\begin{document}

\begin{mybox3}
  \begin{thema}
    Να βρεθεί η παράγωγος κατά κατεύθυνση της συνάρτησης $ f(x,y,z) = x^{2}
    \mathrm{e}^{xyz} $ στο σημείο $ P(1,1,1) $ προς την κατεύθυνση του 
    \textbf{εφαπτόμενου} διανύσματος της καμπύλης που είναι η τομή των επιφανειών 
    \[
      \left.
        \begin{matrix}
      x^{2}+y^{2}-z^{2}=1 \\
      xyz=1
        \end{matrix} 
      \right\} 
    \]
  \end{thema}
\end{mybox3}
\begin{solution}
  Θέτουμε 
  \[
    \left.
      \begin{matrix}
        F(x,y,z) = x^{2}+y^{2}+z^{2} - 1 = 0 \\
        G(x,y,z) = xyz -1 =0
      \end{matrix} 
    \right\} 
  \]
  \begin{mybox1}
    \vspace{0.5\baselineskip}
    \textcolor{Col1}{\textbf{Θυμάμαι:}} ένα διάνυσμα κάθετο σε μία επιφάνεια 
    $ F(x,y,z)=0 $ σε ένα σημείο $ P $ είναι το διάνυσμα $ \grad{F(P)} $, της κλίσης 
    της σε αυτό το σημείο. 
  \end{mybox1}
  Οπότε, θα έχουμε:
  \begin{gather*}
    \grad F(P) \; \text{\textbf{κάθετο} στην επιφάνεια} \; F(x,y,z)=0 \; \text{στο 
    σημείο} \; P \\
    \grad G(P) \; \text{\textbf{κάθετο} στην επιφάνεια} \; G(x,y,z)=0 \; \text{στο 
    σημείο} \; P  
  \end{gather*} 
  Επομένως, το διάνυσμα $ \grad F(P) \times \grad G(P) $ θα είναι \textbf{παράλληλο}, 
  δηλαδή \textcolor{Col1}{εφαπτόμενο,} στην καμπύλη που είναι η τομή των δύο 
  επιφανειών, στο σημείο $ P $.
  \begin{gather*}
    \grad F = \Bigl(\pdv{F}{x}, \pdv{F}{y} , \pdv{F}{z} \Bigr) = (2x,2y,-2z) 
    \Rightarrow \grad F(P) = (2,2,-2) \\
    \grad G = \Bigl(\pdv{G}{x}, \pdv{G}{y} , \pdv{G}{z} \Bigr) = (yz,xz,xy) 
    \Rightarrow \grad G(P) = (1,1,1)
  \end{gather*} 
  \[
    \grad F(P) \times \grad G(P) = 
    \begin{vmatrix*}[r]
      \mathbf{i} & \mathbf{j} & \mathbf{k} \\
      2 & 2 & -2 \\
      1 & 1 & 1
    \end{vmatrix*} = 
    4 \mathbf{i} -4 \mathbf{j} + 0 \mathbf{k} = (4,-4,0)
  \] 
  Βρίσκουμε το μοναδιαίο διάνυσμα $ \hat{\mathbf{u}} $ και την κλίση της συνάρτησης 
  $ f(x,y,z) $ στο σημείο $P$
  \begin{gather*}
    \hat{\mathbf{u}} = \frac{1}{\norm{\mathbf{u}}} \mathbf{u} =
    \frac{1}{\sqrt{4^{2}+(-4)^{2}+0^{2}}} (4,-4,0) = \frac{1}{\sqrt{32}} (4,-4,0) =
  \frac{1}{4 \sqrt{2}} (4,-4,0) = \Bigl(\frac{1}{\sqrt{2}} , - \frac{1}{\sqrt{2}},0\Bigr)
  \\
  \grad f = \Bigl(\pdv{f}{x} , \pdv{f}{y} , \pdv{f}{z}\Bigr) = 
  (2x \mathrm{e}^{xyz} + x^{2}yz \mathrm{e}^{xyz}, x^{3}z \mathrm{e}^{xyz}, x^{3}y 
  \mathrm{e}^{xyz}) \Rightarrow \grad f (P) = (3 \mathrm{e}, \mathrm{e}, \mathrm{e}) 
   \end{gather*} 
   Άρα η παράγωγος κατά κατεύθυνση θα είναι
   \[
     \dv{f}{\mathbf{u}}\eval_{P} = \grad f(P) \cdot \hat{\mathbf{u}} = (3 \mathrm{e},
   \mathrm{e}, \mathrm{e}) \cdot \Bigl(\frac{1}{\sqrt{2}} , - \frac{1}{\sqrt{2}} , 0\Bigr) = 
     \frac{3 \mathrm{e}}{\sqrt{2}} - \frac{\mathrm{e}}{\sqrt{2}} = \frac{2
     \mathrm{e}}{\sqrt{2}}.
    \] 
\end{solution}

\end{document}
