\documentclass[a4paper,12pt]{article}
\usepackage{etex}
%%%%%%%%%%%%%%%%%%%%%%%%%%%%%%%%%%%%%%
% Babel language package
\usepackage[english,greek]{babel}
% Inputenc font encoding
\usepackage[utf8]{inputenc}
%%%%%%%%%%%%%%%%%%%%%%%%%%%%%%%%%%%%%%

%%%%% math packages %%%%%%%%%%%%%%%%%%
\usepackage{amsmath}
\usepackage{amssymb}
\usepackage{amsfonts}
\usepackage{amsthm}
\usepackage{proof}

\usepackage{physics}

%%%%%%% symbols packages %%%%%%%%%%%%%%
\usepackage{bm} %for use \bm instead \boldsymbol in math mode 
\usepackage{dsfont}
\usepackage{stmaryrd}
%%%%%%%%%%%%%%%%%%%%%%%%%%%%%%%%%%%%%%%


%%%%%% graphicx %%%%%%%%%%%%%%%%%%%%%%%
\usepackage{graphicx}
\usepackage{color}
%\usepackage{xypic}
\usepackage[all]{xy}
\usepackage{calc}
\usepackage{booktabs}
\usepackage{minibox}
%%%%%%%%%%%%%%%%%%%%%%%%%%%%%%%%%%%%%%%

\usepackage{enumerate}

\usepackage{fancyhdr}
%%%%% header and footer rule %%%%%%%%%
\setlength{\headheight}{14pt}
\renewcommand{\headrulewidth}{0pt}
\renewcommand{\footrulewidth}{0pt}
\fancypagestyle{plain}{\fancyhf{}
\fancyhead{}
\lfoot{}
\rfoot{\small \thepage}}
\fancypagestyle{vangelis}{\fancyhf{}
\rhead{\small \leftmark}
\lhead{\small }
\lfoot{}
\rfoot{\small \thepage}}
%%%%%%%%%%%%%%%%%%%%%%%%%%%%%%%%%%%%%%%

\usepackage{hyperref}
\usepackage{url}
%%%%%%% hyperref settings %%%%%%%%%%%%
\hypersetup{pdfpagemode=UseOutlines,hidelinks,
bookmarksopen=true,
pdfdisplaydoctitle=true,
pdfstartview=Fit,
unicode=true,
pdfpagelayout=OneColumn,
}
%%%%%%%%%%%%%%%%%%%%%%%%%%%%%%%%%%%%%%

\usepackage[space]{grffile}

\usepackage{geometry}
\geometry{left=25.63mm,right=25.63mm,top=36.25mm,bottom=36.25mm,footskip=24.16mm,headsep=24.16mm}

%\usepackage[explicit]{titlesec}
%%%%%% titlesec settings %%%%%%%%%%%%%
%\titleformat{\chapter}[block]{\LARGE\sc\bfseries}{\thechapter.}{1ex}{#1}
%\titlespacing*{\chapter}{0cm}{0cm}{36pt}[0ex]
%\titleformat{\section}[block]{\Large\bfseries}{\thesection.}{1ex}{#1}
%\titlespacing*{\section}{0cm}{34.56pt}{17.28pt}[0ex]
%\titleformat{\subsection}[block]{\large\bfseries{\thesubsection.}{1ex}{#1}
%\titlespacing*{\subsection}{0pt}{28.80pt}{14.40pt}[0ex]
%%%%%%%%%%%%%%%%%%%%%%%%%%%%%%%%%%%%%%

%%%%%%%%% My Theorems %%%%%%%%%%%%%%%%%%
\newtheorem{thm}{Θεώρημα}[section]
\newtheorem{cor}[thm]{Πόρισμα}
\newtheorem{lem}[thm]{λήμμα}
\theoremstyle{definition}
\newtheorem{dfn}{Ορισμός}[section]
\newtheorem{dfns}[dfn]{Ορισμοί}
\theoremstyle{remark}
\newtheorem{remark}{Παρατήρηση}[section]
\newtheorem{remarks}[remark]{Παρατηρήσεις}
%%%%%%%%%%%%%%%%%%%%%%%%%%%%%%%%%%%%%%%




\newcommand{\vect}[2]{(#1_1,\ldots, #1_#2)}
%%%%%%% nesting newcommands $$$$$$$$$$$$$$$$$$$
\newcommand{\function}[1]{\newcommand{\nvec}[2]{#1(##1_1,\ldots, ##1_##2)}}

\newcommand{\linode}[2]{#1_n(x)#2^{(n)}+#1_{n-1}(x)#2^{(n-1)}+\cdots +#1_0(x)#2=g(x)}

\newcommand{\vecoffun}[3]{#1_0(#2),\ldots ,#1_#3(#2)}

\newcommand{\mysum}[1]{\sum_{n=#1}^{\infty}






\begin{document}

\renewcommand{\arraystretch}{1.2}

\begin{center}
  \minibox{\bfseries\Large \textcolor{Col1}{Γραμμική Άλγεβρα}}
\end{center}

\vspace{\baselineskip}

\setcounter{chapter}{1}

\section{Διανυσματικοί Υπόχωροι}

Έστω $V$ ένας $\mathbb{K}$-χώρος.

\begin{dfn}
Ένα μη-κενό υποσύνολο $W$ του $V$ λέγεται διανυσματικός \textbf{\textit{υπόχωρος}} του $V$ αν το $W$ είναι $\mathbb{K}$-χώρος με τις ίδιες πράξεις πρόσθεσης και βαθμωτού πολλαπλασιασμού, όπως εκείνες του $V$. Συμβολίζουμε $W\leq V$.
\end{dfn}

Επομένως, για να αποδείξουμε ότι ένα μη-κενό υποσύνολο $W\subseteq V$ είναι υπόχωρος του $V$, αρκεί να δείξουμε ότι το $W$ είναι κλειστό ως προς την πρόσθεση και τον βαθμωτό πολλαπλασιασμό.

\begin{prop}
$W\neq \emptyset$, $W\subseteq V$, διανυσματικός υπόχωρος του $V$ $\Leftrightarrow$
\begin{enumerate}[i.]
\item $\lambda w\in W$, για κάθε $\lambda\in \mathbb{K}, w\in W$.
\item $w_1+w_2\in W$, για κάθε $w_1, w_2\in W$.
\end{enumerate}
\end{prop}

\begin{prop}
$\{0\}\leq V$ και $V\leq V$ για κάθε διανυσματικό χώρο $V$.
\end{prop}

\section{Βάση και Διάσταση Υπόχωρων}

Έστω $V$ ένας $\mathbb{K}$-χώρος.


\begin{dfn} Έστω $S=\{v_1,v_2,\ldots,v_n\}\subseteq V$. Ονομάζουμε \textbf{\textit{γραμμικό συνδυασμό}} των στοιχείων του $S$ κάθε παράσταση της μορφής:
\[
\sum_{i=1}^{n}a_iv_i=a_1v_1+a_2v_2+\cdots +a_nv_n \quad\text{όπου}\quad a_i\in \mathbb{K}
\]
\end{dfn}

Το σύνολο $U$ \textbf{όλων} των γραμμικών συνδυασμών των στοιχείων $v_1,v_2,\ldots, v_n$ του $S$ αποδεικνύεται ότι είναι ένας υπόχωρος του 
$V$ και συμβολίζεται με $U=<S>$. Τότε λέμε ότι το σύνολο $S$ \textit{παράγει} τον χώρο $U$ και τα διανύσματα του $S$ λέγονται \textit{γεννήτορες} του χώρου.

\begin{dfn} Έστω $S=\{v_1,v_2,\ldots,v_n\}\subseteq V$. Τα στοιχεία του υποσυνόλου $S$ λέγονται \textbf{\textit{γραμμικώς ανεξάρτητα}} ή το $S$ λέμε ότι είναι \textit{γραμμικώς ανεξάρτητο} πάνω στο $\mathbb{K}$, αν 
\[
a_1v_1+a_2v_2+\cdots +a_nv_n=0\Rightarrow a_1=a_2=\cdots=a_n=0
\]
για κάθε $a_i\in \mathbb{K}$ με  $i=1,\ldots,n$. 
\end{dfn}

\begin{dfn} Αν τα διανύσματα $v_1,v_2,\ldots,v_n$ δεν είναι γραμμικώς ανεξάρτητα, τότε λέμε ότι είναι \textbf{\textit{γραμμικώς εξαρτημένα}}. Επομένως, $v_1,v_2,\ldots,v_n$ είναι γραμμικώς εξαρτημένα, αν υπάρχουν  συντελεστές $a_i\in \mathbb{K}$ όχι όλοι μηδέν, έτσι ώστε $a_1v_1+a_2v_2+\cdots +a_nv_n=0$.
\end{dfn}

\begin{prop}
  \item {}
\begin{enumerate}[i.]
\item $S\subseteq V$ και $0\in S\Rightarrow S$ είναι γραμμικώς εξαρτημένο.
\item $S$ γραμμικώς ανεξάρτητο $\Rightarrow$ Κάθε υποσύνολο του $S$ είναι γραμμικώς ανεξάρτητο.
\item $v_1,v_2,\ldots v_n$ γραμμικώς εξαρτημένα $\Rightarrow$ κάποιο από τα $v_i$ είναι γραμμικός συνδυασμός των υπολοίπων.
\end{enumerate} 
\end{prop}

\begin{dfn} Ένα υποσύνολο $B\subseteq V$ λέγεται \textbf{\textit{βάση}} του $V$, αν:
\begin{enumerate}[i.]
\item Το $B$ παράγει τον χώρο $V$, δηλαδή $V=<B>$ 
\item Το $B$ είναι γραμμικώς ανεξάρτητο υποσύνολο του $V$.

Αποδεικνύεται ότι ένα υποσύνολο $B$ του διανυσματικού χώρου $V$ είναι μια βάση του $V$ αν και μόνον αν κάθε στοιχείο του χώρου $V$ γράφεται κατά \textbf{μοναδικό τρόπο} ως γραμμικός συνδυασμός των στοιχείων του $Β$.
\end{enumerate}
\end{dfn}


\begin{center}
\minibox{\bfseries\Large \textcolor{Col1}{Λυμένα Θέματα}}
\end{center}

\vspace{\baselineskip}

\begin{enumerate}[1.]
\item Δίνονται τα διανύσματα $v_1=(2,1,3,2)^T, v_2=(1,2,1,2)^T, v_3=(0,1,1,0)^T$.
\begin{enumerate}[i)]
\item Εξετάστε αν τα διανύσματα $v_1,v_2,v_3$ είναι γραμμικώς ανεξάρτητα ή γραμμικώς εξαρτημένα.
\item Δείξτε ότι το διάνυσμα $v_4=(4,0,8,2)^T$ ανήκει στο χώρο που παράγεται από τα $v_1,v_2,v_3$.
\item Γράψτε το $v_4=(4,0,8.2)^T$ σαν γραμμικό συνδυασμό των διανυσμάτων $v_1,v_2,v_3$.
\end{enumerate}

\vspace{\baselineskip}

\fbox{Λύση}

\vspace{\baselineskip}

\begin{enumerate}[i)]
\item \[
\begin{pmatrix}
2 & 1 & 3 & 2 \\
1 & 2 & 1 & 2 \\
0 & 1 & 1 & 0
\end{pmatrix}
\sim
\begin{pmatrix}
1 & 2 & 1 & 2 \\
2 & 1 & 3 & 2 \\
0 & 1 & 1 & 0 
\end{pmatrix}
\sim
\begin{pmatrix}
1 & 2 & 1 & 2 \\
0 & -3 & 1 & -2 \\
0 & 1 & 1 & 0
\end{pmatrix}
\sim
\begin{pmatrix}
1 & 2 & 1 & 2 \\
0 & 1 & 1 & 0 \\
0 & -3 & 1 & -2
\end{pmatrix}
\sim
\]
\[
\sim
\begin{pmatrix}
1 & 0 & -1 & 2 \\
0 & 1 & 1 & 0 \\
0 & 0 & 4 & -2
\end{pmatrix}
\sim
\begin{pmatrix}
1 & 0 & -1 & 2 \\
0 & 1 & 1 & 0 \\
0 & 0 & 1 & -\frac{1}{2}
\end{pmatrix}
\sim
\begin{pmatrix}
1 & 0 & 0 & \frac{3}{2} \\
0 & 1 & 0 & \frac{1}{2} \\
0 & 0 & 1 & -\frac{1}{2}
\end{pmatrix}
\]
Επομένως τα διανύσματα είναι γραμμικώς ανεξάρτητα.

\item 

Το διάνυσμα $v_4$ ανήκει στο χώρο που παράγεται από τα $v_1,v_2,v_3$ αν και μόνον αν, υπάρχουν $x,y,z\in \mathbb{R}$ τέτοια ώστε:
\[
xv_1+yv_2+zv_3=v_4 \Leftrightarrow 
\]
\[
x\begin{pmatrix}
2 \\ 1 \\ 3 \\2
\end{pmatrix}
+y\begin{pmatrix}
1 \\ 2 \\ 1 \\ 2
\end{pmatrix}
+z\begin{pmatrix}
0 \\ 1 \\ 1 \\ 0
\end{pmatrix}
=\begin{pmatrix}
4 \\ 0 \\ 8 \\2
\end{pmatrix} \Leftrightarrow
\]
\[
\left.
\begin{array}{c}
2x+y+0z=4\\
x+2y+z=0 \\
3x+y+z=8 \\
2x+2y+0z=2
\end{array}
\right\}\Leftrightarrow
\]
\[
\begin{pmatrix}
2 & 1 & 0 & 4 \\
1 & 2 & 1 & 0 \\
3 & 1 & 1 & 8 \\
2 & 2 & 0 & 2 
\end{pmatrix}
\sim
\begin{pmatrix}
1 & 2 & 1 & 0 \\
2 & 1 & 0 & 4 \\
3 & 1 & 1 & 8 \\
2 & 2 & 0 & 2
\end{pmatrix}
\sim 
\begin{pmatrix}
1 & 2 & 1 & 0 \\
0 & -3 & -2 & 4 \\
0 & -5 & -2 & 8 \\
0 & -2 & -2 & -2
\end{pmatrix}
\sim 
\begin{pmatrix}
1 & 2 & 1 & 0 \\
0 & 1 & \frac{2}{3} & -\frac{4}{3} \\
0 & -5 & -2 & 8 \\
0 & -2 & -2 & -2 
\end{pmatrix}
\sim 
\]
\[
\sim
\begin{pmatrix}
1 & 2 & 1 & 0 \\
0 & 1 & \frac{2}{3} & -\frac{4}{3} \\
0 & 0 & \frac{4}{3} & \frac{4}{3} \\
0 & 0 & -\frac{2}{3} & -\frac{2}{3}
\end{pmatrix}
\sim 
\begin{pmatrix}
1 & 2 & 1 & 0 \\
0 & 1 & \frac{2}{3} & -\frac{4}{3} \\
0 & 0 & 1 & 1 \\
0 & 0 & -\frac{2}{3} & -\frac{2}{3} \\ 
\end{pmatrix}
\sim 
\begin{pmatrix}
1 & 2 & 1 & 0 \\
0 & 1 & \frac{2}{3} & -\frac{4}{3} \\
0 & 0 & 1 & 1 & \\
0 & 0 & 0 & 0 
\end{pmatrix}
\sim 
\]
\[
\sim
\begin{pmatrix}
1 & 0 & -\frac{1}{3} & \frac{8}{3} \\
0 & 1 & \frac{2}{3} & -\frac{4}{3}\\
0 & 0 & 1 & 1 \\
0 & 0 & 0 & 0 
\end{pmatrix}
\sim 
\begin{pmatrix}
1 & 0 & 0 & 3 \\
0 & 1 & 0 & -2 \\
0 & 0 & 1 & 1 \\
0 & 0 & 0 & 0 
\end{pmatrix}\Leftrightarrow
\]
Παρατηρούμε ότι:
$\rank(A)=\rank(E)=3$, όπου $3$ είναι το πλήθος των αγνώστων του συστήματος. Επομένως το σύστημα έχει μοναδική λύση και άρα το διάνυσμα $v_4$ γράφεται ως γραμμικός συνδυασμός των $v_1,v_2,v_3$ και επομένως ανήκει στο χώρο που παράγουν τα διανύσματα αυτά.


\item 
Η μοναδική λύση του συστήματος, και άρα οι συντελεστές του γραμμικού συνδυασμού, είναι:
$ x=3, y=-2, z=1 $

\end{enumerate}


\item Έστω το σύνολο $V=\left\{v=(x,y,z,w)^T\in\mathbb{R}^4 \;:\; \minibox{$2x+3y+z=0$ \\ $x-y+w=0$}\right\}$.

Να δείξετε ότι το $V$ είναι υπόχωρος του $\mathbb{R}^4$ και να βρείτε μια βάση και τη διάστασή του.

\vspace{\baselineskip}

\fbox{Λύση}

\vspace{\baselineskip}

Καταρχάς παρατηρούμε ότι $0\in V$. 

Έστω $v=(x,y,z,w)\in V \Rightarrow 2x+3y+z=0$ και $x-y+w=0$.

Θα δείξουμε ότι $\lambda v\in V$. 

Πράγματι:
\[
\lambda v=\lambda(x,y,z,w)=(\lambda x,\lambda y, \lambda z, \lambda w)
\]
και έχουμε:
\[
2\lambda x+3\lambda y +\lambda z= \lambda(2x+3y+z)=\lambda 0=0
\]
\[
\lambda x- \lambda y +\lambda w=\lambda(x-y+w)=\lambda 0 =0
\]

Άρα $\lambda v\in V$.

Έστω $v_1=(x_1,y_1,z_1,w_1)\in V \Rightarrow 2x_1+3y_1+z_1=0$ και $x_1-y_1+w_1=0$.

Έστω $v_2=(x_2,y_2,z_2,w_2)\in V \Rightarrow 2x_2+3y_2+z_2=0$ και $x_2-y_2+w_2=0$.

Θα δείξουμε ότι \[v_1+v_2=(x_1+x_2,y_1+y_2,z_1+z_2,w_1+w_2)\in V\] 

Πράγματι:

\[2(x_1+x_2)+3(y_1+y_2)+z_1+z_2=2x_1+3y_1+z_1+2x_2+3y_2+z_2=0+0=0\]
\[(x_1+x_2)-(y_1+y_2)+w_1+w_2=x_1-y_1+w_1+x_2-y_2+w_2=0+0=0\]

Άρα $v_1+v_2\in V$.

Επομένως $V\leq \mathbb{R}^4$.

Έστω τώρα $v=(x,y,z,w)\in V \Rightarrow 2x+3y+z=0$ και $x-y+w=0$.

Επομένως $v$ ανήκει στο χώρο λύσεων του παραπάνω ομογενούς συστήματος. Άρα το $v$ ανήκει στο μηδενόχωρο του συστήματος και έτσι, αρκεί να βρώ μια βάση και τη διάσταση για τον χώρο αυτόν. 


\[
\begin{pmatrix}
2 & 3 & 1 & 0 \\
1 & -1 & 0 & 1 
\end{pmatrix}
\sim
\begin{pmatrix}
1 & -1 & 0 & 1 \\
2 & 3 & 1 & 0 
\end{pmatrix}
\sim 
\begin{pmatrix}
1 & -1 & 0 & 1 \\
0 & 5 & 1 & -2
\end{pmatrix}
\sim 
\]
\[
\sim
\begin{pmatrix}
1 & -1 & 0 & 1 \\
0 & 1 & \frac{1}{5} & -\frac{2}{5}
\end{pmatrix}
\sim 
\begin{pmatrix}
1 & 0 & \frac{1}{5} & \frac{3}{5} \\
0 & 1 & \frac{1}{5} & -\frac{2}{5}
\end{pmatrix}
\]

\end{enumerate}

Άρα μια βάση για το μηδενόχωρο του συστήματος είναι ή:

\[
Β=\left\{(-\frac{1}{5},-\frac{1}{5},1,0), (\frac{3}{5},-\frac{2}{5},0,1)\right\}
\]

και επομένως έχουμε ότι:

\[
\dim(V)=2
\]
\end{document}
