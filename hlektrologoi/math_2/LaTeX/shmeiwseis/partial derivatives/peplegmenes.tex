\chapter{Πεπλεγμένες Συναρτήσεις}


\section{Θεωρήματα Πεπλεγμένων συναρτήσεων}

\vspace{\baselineskip}

\subsection{Η εξίσωση \ensuremath{F(x,y) = 0}}

\subsubsection{Πεπλεγμένη συνάρτηση της μορφής \ensuremath{y=f(x)}}

Έστω $ F(x,y) = 0 $, όπου $ F\colon D \to \mathbb{R} $ μια συνάρτηση με πεδίο
ορισμού ένα ανοιχτό υποσύνολο $D$ του $\mathbb{R}^{2}$ και $ (x_0,y_0) $ ένα 
εσωτερικό σημείο του $D$.  Αν 
\begin{enumerate}[(i)]
  \item $F(x_0,y_0) = 0$ 
  \item $ F_x, F_y$ συνεχείς σε περιοχή του σημείου $ (x_0,y_0) $ 
  \item $ F_{\textcolor{Col1}{y}}(x_0,y_0) \neq 0 $
\end{enumerate}
τότε υπάρχει μοναδική συνάρτηση $\textcolor{Col1}{y=y(x)} $ ορισμένη στο 
$ I_0 \subseteq \mathbb{R} $ τέτοια ώστε:
\begin{myitemize}
  \item $y_0 = y(x_0)$
  \item $F(x,y(x)) = 0, \quad \forall x \in I_0$
  \item $ \dv{y}{x} = - \frac{F_x}{F_y}, \quad \forall x \in I_0  $
\end{myitemize}

\begin{rem}
  Ο παραπάνω τύπος για την παράγωγο $ \dv{y}{x} $ προκύπτει ως λύση της εξίσωσης
  \[
    \pdv{F}{x} + \pdv{F}{y}\dv{y}{x} = 0 
  \] 
  η οποία προκύπτει με παραγώγιση της $ F(x,y) = 0$, αν θεωρήσουμε ότι $ y=y(x) $.
\end{rem}

\subsubsection{Πεπλεγμένη συνάρτηση της μορφής \ensuremath{x=f(y)}}

Έστω $ F(x,y) = 0 $, όπου $ F\colon D \to \mathbb{R} $ μια συνάρτηση με πεδίο
ορισμού ένα ανοιχτό υποσύνολο $D$ του $\mathbb{R}^{2}$ και $ (x_0,y_0) $ ένα 
εσωτερικό σημείο του $D$.  Αν 
\begin{enumerate}[(i)]
  \item $F(x_0,y_0) = 0$ 
  \item $ F_x, F_y$ συνεχείς σε περιοχή του σημείου $ (x_0,y_0) $ 
  \item $ F_{\textcolor{Col1}{x}}(x_0,y_0) \neq 0 $
\end{enumerate}
τότε υπάρχει μοναδική συνάρτηση $ \textcolor{Col1}{x=x(y)} $ ορισμένη στο 
$ I_0 \subseteq \mathbb{R} $ τέτοια ώστε:
\begin{myitemize}
  \item $x_0 = x(y_0)$
  \item $F(x(y),y) = 0, \quad \forall y \in I_0$
  \item $ \dv{x}{y} = - \frac{F_y}{F_x}, \quad \forall y \in I_0  $
\end{myitemize}

\begin{rem}
  Ο παραπάνω τύπος για την παράγωγο $ \dv{x}{y} $ προκύπτει ως λύση της εξίσωσης
  \[
    \pdv{F}{y} + \pdv{F}{x}\dv{x}{y} = 0 
  \] 
  η οποία προκύπτει με παραγώγιση της $ F(x,y) = 0$, αν θεωρήσουμε ότι $ x=x(y) $.
\end{rem}
\begin{example}
  Να υπολογίσετε τα σημεία της καμπύλης $ F(x,y) = x^{3}+y^{3}-6xy=0 $ για τα οποία 
  ισχύει το θεώρημα πεπλεγμένης συνάρτησης.
\end{example}
\begin{solution}
  Αρχικά εξετάζουμε αν το $ y $ μπορεί να είναι πεπλεγμένη συνάρτηση του $x$.   
  Βρίσκουμε τα σημεία που επαληθεύουν την εξίσωση και για τα οποία $ F_{y}=0 $.
  \[
    F_{y} = 0 \Leftrightarrow 3y^{2}-6x=0 \Leftrightarrow \inlineequation[xsom]
    {x = \frac{1}{2} y^{2}}
  \] 
  Βρίσκουμε ποια από αυτά τα σημεία επαληθεύουν την εξίσωση και έχουμε:
  \[
    \left(\frac{1}{2} y^{2}\right)^{3}+y^{3}-6 \frac{1}{2} y^{2}y=0 
    \Leftrightarrow \frac{1}{8} y^{6}+y^{3} -3y^{3}=0 
    \Leftrightarrow y^{3}\left(\frac{1}{8} y^{3}-2\right)=0 \Leftrightarrow y=0 \quad
    \text{ή} \quad y= \sqrt[3]{16} = \sqrt[3]{8\cdot2} = 2 \sqrt[3]{2}
  \] 
  Άρα, από την εξίσωση~\eqref{xsom} έχουμε ότι 
  \[ 
    x=0  \quad \text{ή} \quad x = \frac{1}{2} \left(2 \sqrt[3]{2}\right)^{2} = 
    \frac{1}{2} \left(2 \cdot 2^{\frac{1}{3}}\right)^{2} = 2 \cdot 2^{\frac{2}{3}} 
    = 2 \sqrt[3]{2^{2}} = 2 \sqrt[3]{4}  
  \]
  Επομέvως το θεώρημα δεν εφαρμόζεται για τα σημεία $ (0,0) $ και $ (2 \sqrt[3]{4}
  , 2 \sqrt[3]{2}) $.

  Στη συνέχεια εξετάζουμε αν το $ x $ μπορεί να είναι πεπλεγμένη συνάρτηση του $y$.
  Βρίσκουμε τα σημεία που επαληθεύουν την εξίσωση και για τα οποία $ F_{x}=0 $.
  \[
    F_{x}=0 \Leftrightarrow 3x^{2}-6y=0 \Leftrightarrow \inlineequation[ysom]
    {y= \frac{1}{2} x^{2}}
  \] 
  Βρίσκουμε ποια από αυτά τα σημεία επαληθεύουν την εξίσωση και έχουμε:
  \[
    x^{3}+\left(\frac{1}{2} x^{2}\right)^{3}-6x \frac{1}{2} x^{2}=0 \Leftrightarrow 
    x^{3}+ \frac{1}{8} x^{6}-3x^{3}=0 \Leftrightarrow \frac{1}{8} x^{6}-3x^{3}=0
    \Leftrightarrow x^{3}\left(\frac{1}{8} x^{3}-2\right)=0 \Leftrightarrow x = 0 \quad
    \text{ή} \quad x=2 \sqrt[3]{2} 
  \] 
  Άρα, από την εξίσωση~\eqref{ysom} έχουμε ότι 
  \[ 
    y=0  \quad \text{ή} \quad y = 2 \sqrt[3]{4} 
  \]
  Επομέvως το θεώρημα δεν εφαρμόζεται για τα σημεία $ (0,0) $ και $ (2 \sqrt[3]{2}
  , 2 \sqrt[3]{4}) $.

  Οπότε συνολικά, το θεώρημα δεν εφαρμόζεται για τα σημεία 
  \[
    \boxed{(0,0), \quad (2 \sqrt[3]{4} , 2 \sqrt[3]{2}), \quad (2 \sqrt[3]{2} , 2
    \sqrt[3]{4})}
  \] 
\end{solution}

\subsection{Η εξίσωση \ensuremath{F(x,y,z) = 0}}

\subsubsection{Πεπλεγμένη συνάρτηση της μορφής \ensuremath{z=z(x,y)}}

Έστω $ F(x,y,z) = 0 $, όπου $F\colon D \to \mathbb{R}$ μια συνάρτηση με πεδίο ορισμού 
ένα ανοικτό υποσύνολο $ D $ του $ \mathbb{R}^{3}  $ και $ (x_0,y_0,z_0) $ ένα 
εσωτερικό σημείο του $ D $. Αν
\begin{enumerate}[(i)]
  \item $ F(x_0,y_0,z_0) $
  \item $ F_x, F_y, F_z $ συνεχείς σε περιοχή του σημείου $ (x_0,y_0,z_0) $
  \item $ F_{\textcolor{Col1}{z}}(x_0,y_0,z_0) \neq 0 $
\end{enumerate}
τότε υπάρχει μοναδική συνάρτηση, $ \textcolor{Col1}{z=z(x,y)} $ ορισμένη στο 
$ D_0 \subseteq \mathbb{R}^{2} $ τέτοια ώστε:
\begin{myitemize}
  \item $ z_0 = z(x_0,y_0) $
  \item $ F(x,y,z(x,y)) = 0,  \quad \forall (x,y)\in  D_0 $
  \item $ \pdv{z}{x} = - \frac{F_x}{F_z} $ και $ \pdv{z}{y} = - \frac{F_y}{F_z}, 
    \quad \forall (x,y) \in D_0$
\end{myitemize}

\begin{rem}
  Οι παραπάνω τύποι για τις παραγώγους $ \pdv{z}{x}, \pdv{z}{y} $ προκύπτουν 
  ως λύσεις των εξισώσεων  
  \begin{align*}	
    \pdv{F}{x} + \pdv{F}{z}\pdv{z}{x} &= 0 \\
    \pdv{F}{y} + \pdv{F}{z}\pdv{z}{y} &= 0 
  \end{align*}
  οι οποίες προκύπτουν με παραγώγιση της $ F(x,y,z) = 0 $, ως προς $x$ και $y$ 
  αντίστοιχα και  αν θεωρήσουμε ότι $ z=z(x,y) $.
\end{rem}

\subsubsection{Πεπλεγμένη συνάρτηση της μορφής \ensuremath{y=y(x,z)}}

Έστω $ F(x,y,z) = 0 $, όπου $F\colon D \to \mathbb{R}$ μια συνάρτηση με πεδίο ορισμού 
ένα ανοικτό υποσύνολο $ D $ του $ \mathbb{R}^{3}  $ και $ (x_0,y_0,z_0) $ ένα 
εσωτερικό σημείο του $ D $. Αν
\begin{enumerate}[(i)]
  \item $ F(x_0,y_0,z_0) $
  \item $ F_x, F_y, F_z $ συνεχείς σε περιοχή του σημείου $ (x_0,y_0,z_0) $
  \item $ F_{\textcolor{Col1}{y}}(x_0,y_0,z_0) \neq 0 $
\end{enumerate}
τότε υπάρχει μοναδική συνάρτηση, $ \textcolor{Col1}{y=y(x,z)} $ ορισμένη στο 
$ D_0 \subseteq \mathbb{R}^{2} $ τέτοια ώστε:
\begin{myitemize}
  \item $ y_0 = y(x_0,z_0) $
  \item $ F(x,y(x,z),z)) = 0,  \quad \forall (x,z) \in  D_0 $
  \item $ \pdv{y}{x} = - \frac{F_x}{F_y} $ και $ \pdv{y}{z} = - \frac{F_z}{F_y}, 
    \quad \forall (x,z) \in D_0$
\end{myitemize}

\begin{rem}
  Οι παραπάνω τύποι για τις παραγώγους $ \pdv{y}{x}, \pdv{y}{z} $ προκύπτουν 
  ως λύσεις των εξισώσεων  
  \begin{align*}	
    \pdv{F}{x} + \pdv{F}{y}\pdv{y}{x} &= 0 \\
    \pdv{F}{z} + \pdv{F}{y}\pdv{y}{z} &= 0 
  \end{align*}
  οι οποίες προκύπτουν με παραγώγιση της $ F(x,y,z) = 0 $, ως προς $x$ και $z$ 
  αντίστοιχα και  αν θεωρήσουμε ότι $ y=y(x,z) $.
\end{rem}

\subsubsection{Πεπλεγμένη συνάρτηση της μορφής \ensuremath{x=x(y,z)}}

Έστω $ F(x,y,z) = 0 $, όπου $F\colon D \to \mathbb{R}$ μια συνάρτηση με πεδίο ορισμού 
ένα ανοικτό υποσύνολο $ D $ του $ \mathbb{R}^{3}  $ και $ (x_0,y_0,z_0) $ ένα 
εσωτερικό σημείο του $ D $. Αν
\begin{enumerate}[(i)]
  \item $ F(x_0,y_0,z_0) $
  \item $ F_x, F_y, F_z $ συνεχείς σε περιοχή του σημείου $ (x_0,y_0,z_0) $
  \item $ F_{\textcolor{Col1}{x}}(x_0,y_0,z_0) \neq 0 $
\end{enumerate}
τότε υπάρχει μοναδική συνάρτηση, $ \textcolor{Col1}{x=x(y,z)} $ ορισμένη στο 
$ D_0 \subseteq \mathbb{R}^{2} $ τέτοια ώστε:
\begin{myitemize}
  \item $ x_0 = x(y_0,z_0) $
  \item $ F(x(y,z),y,z)) = 0,  \quad \forall (y,z) \in  D_0 $
  \item $ \pdv{x}{y} = - \frac{F_y}{F_x} $ και $ \pdv{x}{z} = - \frac{F_z}{F_x}, 
    \quad \forall (y,z) \in D_0$
\end{myitemize}

\begin{rem}
  Οι παραπάνω τύποι για τις παραγώγους $ \pdv{x}{y}, \pdv{x}{z} $ προκύπτουν 
  ως λύσεις των εξισώσεων  
  \begin{align*}	
    \pdv{F}{y} + \pdv{F}{x}\pdv{x}{y} &= 0 \\
    \pdv{F}{z} + \pdv{F}{x}\pdv{x}{z} &= 0 
  \end{align*}
  οι οποίες προκύπτουν με παραγώγιση της $ F(x,y,z) = 0 $, ως προς $y$ και $z$ 
  αντίστοιχα και  αν θεωρήσουμε ότι $ x=x(x,z) $.
\end{rem}
\begin{example}
  Έστω η εξίσωση $ F(x,y,z) = y^{2}+xz+z^{2}- \mathrm{e}^{z}-4 = 0$. 
  Να εξετάσετε αν ισχύει το θεώρημα πεπλεγμένης συνάρτησης, για την 
  $ z = z(x,y) $, στο σημείο $P(0,\mathrm{e},2) $ και να βρεθούν οι μερικές παράγωγοι 
  1ης και 2ης τάξης της συνάρτησης $ z = z(x,y) $ στο σημείο $ (0, \mathrm{e}) $.
\end{example}
\begin{solution}
\item {}
  \begin{enumerate}[i)]
    \item $ F(0, \mathrm{e}, 2) = \mathrm{e}^{2} + 0\cdot 2 + 2^{2} - 
      \mathrm{e}^{2} - 4 = 0  $ 
    \item 
      \begin{tabular}{l}
        $ F_{x} = z \phantom{\ +2z- \mathrm{e}^{z}} $ \tikzmark{a} \\
        $ F_{y} = 2y $ \\
        $ F_{z} = x+2z- \mathrm{e}^{z} \tikzmark{b} $
      \end{tabular}
      \mybrace{a}{b}[είναι συνεχείς ως πολυωνυμικές]
    \item $ F_{z}(0, \mathrm{e},2) = 0+2\cdot 2- \mathrm{e}^{2} 
      = 4- \mathrm{e}^{2} \neq 0 $
  \end{enumerate}
  Επομένως ικανοποιούνται οι προϋποθέσεις του θεωρήματος Πεπλεγμένης συνάρτησης 
  και άρα υπάρχει μοναδική συνάρτηση $ z=z(x,y) $, για την οποία ισχύουν:
  \begin{myitemize}
    \item $ \inlineequation[eq:perplex1]{2=z(0, \mathrm{e})} $
    \item $ z_{x} = - \frac{F_{x}}{F_{z}} = - \frac{z}{x+2z- \mathrm{e}^{z}} $ και 
      $ z_{y} = - \frac{F_{y}}{F_{z}} - \frac{2y}{x+2z- \mathrm{e}^{z}} $ σε μια 
      περιοχή του σημείου $ (0,\mathrm{e}) $.
  \end{myitemize}
  Επομένως στο σημείο $ (0, \mathrm{e}) $ έχουμε ότι $ z= z(0, \mathrm{e}) = 2 $ και 
  άρα:
  \[
    z_{x}(0, \mathrm{e}) \overset{\eqref{eq:perplex1}}{=}  
    - \frac{2}{0+2\cdot 2- e^{2}} = - \frac{2}{4-\mathrm{e}^{2}} \quad \text{και} 
    \quad z_{y}(0, \mathrm{e}) \overset{\eqref{eq:perplex1}}{=}  
    - \frac{2 \cdot \mathrm{e}}{0+2\cdot 2 - 
    \mathrm{e}^{2}} = - \frac{2 \mathrm{e}}{4- \mathrm{e}^{2}} 
  \]
  Για να βρούμε τις μερικές παραγώγους 2ης τάξης της συνάρτησης $ z(x,y) $, 
  παραγωγίζουμε ξανά, τις μερικές παραγώγους 1ης τάξης, θεωρούμε όμως ότι 
  $z=z(x,y)$ είναι συνάρτηση.
  \begin{align*}
    z_{xx} &= \left(\frac{-z}{x+2z- \mathrm{e}^{z}}\right) _{x} =
    \frac{(-z)_{x}(x+2z- \mathrm{e}^{z})-(-z)(x+2z- \mathrm{e}^{z} )_{x}}{(x+2z-
      \mathrm{e}^{z})^{2}} = \frac{-z_{x}(x+2z- \mathrm{e}^{z})+z(1+2z_{x}- 
    \mathrm{e}^{z} z_{x})}{(x+2z- \mathrm{e}^{z})^{2}}  \\ 
    z_{yy} &= \left(\frac{-2y}{x+2z- \mathrm{e}^{z}}\right)_{y} = 
    \frac{(-2y)_{y}(x+2z- \mathrm{e}^{z})- (-2y)(x+2z- \mathrm{e}^{z} )_{y}}{(x+2z-
      \mathrm{e}^{z} )^{2}} = \frac{(-2)(x+2z- \mathrm{e}^{z} )+2y(2z_{y}-
    \mathrm{e}^{z} z_{y})}{(x+2z- \mathrm{e}^{z})^{2}} \\  
    z_{xy}&= \left(\frac{-z}{x+2z+ \mathrm{e}^{z}}\right)_{y} = 
    \frac{(-z)_{y}(x+2z+ \mathrm{e}^{z})-(-z)(x+2z+ \mathrm{e}^{z} )_{y}}{(x+2z+
      \mathrm{e}^{z})^{2}} = \frac{-z_{y}(x+2z+ \mathrm{e}^{z})+z(+2z_{y}+ 
    \mathrm{e}^{z} z_{y})}{(x+2z+ \mathrm{e}^{z})^{2}}  \\ 
  \end{align*} 
\end{solution}

Με αντικατάσταση, όπου $ (x,y,z)=(0, \mathrm{e}, 2) $ αλλά και όπου $ z_{x}$ και 
$ z_{y} $ τις τιμές που μόλις βρήκαμε, υπολογίζουμε και τις τιμές των παραγώγων 2ης
τάξης στο σημείο $ (0, \mathrm{e}) $.

% \enlargethispage{\baselineskip}

% Επομένως στο σημείο $ (0, \mathrm{e}) $ έχουμε ότι $ z= z(0, \mathrm{e}) = 2 $, 
% $ z_{x}(0, \mathrm{e}) = - \frac{2}{4- \mathrm{e}^{2}} $ και $ z_{y}(0, \mathrm{e}) 
% = - \frac{2 \mathrm{e}}{4 - \mathrm{e}^{2}}$ και άρα:
% \begin{align*}
%   z_{xx}(0, \mathrm{e}) &= \frac{ \frac{2}{4-\mathrm{e}^{2}}(0+2\cdot 2-
%     \mathrm{e}^{2})+2[1+2 (-\frac{2}{4-\mathrm{e}^{2}}) - \mathrm{e}^{2}
%   (-\frac{2}{4-\mathrm{e}^{2}})]}{(0+2 \cdot 2 - \mathrm{e}^{2})^{2}} = 
%   \frac{2 + \frac{2 \mathrm{e}^{2}}{4- \mathrm{e}^{2}}}{(4- \mathrm{e}^{2})^{2}} 
%   = \frac{8}{(4- \mathrm{e}^{2})^{3}} \\
%     z_{yy}(0, \mathrm{e}) &= \frac{(-2)(0+2 \cdot 2 - \mathrm{e}^{2})+2 
%       \mathrm{e}[2\cdot (-\frac{2 \mathrm{e}}{4- \mathrm{e}^{2}}) - 
%       \mathrm{e}^{2} (-\frac{2 \mathrm{e}}{4- \mathrm{e}^{2}})]}{(4- 
%       \mathrm{e}^{2})^{2}} = \frac{-8+2 \mathrm{e}^{2}-
%       \frac{8e^{2}}{4- \mathrm{e}^{2}}+ \frac{4e^{3}}{4- \mathrm{e}^{2}}}{(4-
%       \mathrm{e}^{2})^{2}} = \frac{2(\mathrm{e}^{4} + 4 \mathrm{e}^{2} -16)}{(4-
%     \mathrm{e}^{2})^{2}} 
%     \end{align*}


\subsection{Συστήματα Εξισώσεων}

\subsubsection{1η περίπτωση}

Έστω το σύστημα εξισώσεων $(\Sigma):
\begin{cases}
  F(x,y,z) = 0  \\
  G(x,y,z) = 0
\end{cases}$
και έστω το σημείο $ P_0(x_0,y_0,z_0) $, όπου ισχύει:
\begin{enumerate}[(i)]
  \item  
    \begin{tabular}{l}
      $F(x_0,y_0,z_0) = 0$ \\
      $G(x_0,y_0,z_0) = 0$
    \end{tabular}
  \item Οι συναρτήσεις $ F, G $ είναι $ C^{1} $ τάξης 
  \item $ \eval{\pdv{(F,G)}{(y,z)}} _{P_{0}} \neq 0 $
\end{enumerate} 
τότε υπάρχουν μοναδικές συναρτήσεις $ y = y(x) $ και $ z = z(x) $ τάξης $ C^{1} $ 
τέτοιες ώστε:
\begin{myitemize}
  \item 
    \begin{tabular}{l}
      $ y_0 = y(x_0) $ \\
      $ z_0 = z(x_0) $
    \end{tabular}
  \item 
    \begin{tabular}{l}
      $ F(x,y(x),z(x)) = 0 $ \\
      $ G(x,y(x),z(x)) = 0 $
    \end{tabular}
  \item $ \dv{y}{x} = - \frac{\pdv{(F,G)}{(x,z)}}{\pdv{(F,G)}{(y,z)}} $ και 
    $ \dv{z}{x} = - \frac{\pdv{(F,G)}{(y,x)}}{\pdv{(F,G)}{(y,z)}} $
\end{myitemize}

\begin{rem}
  Οι παραπάνω τύποι για τις τιμές των μερικών παραγώγων $ \dv{y}{x}, \dv{z}{x}$ 
  προκύπτουν ως λύσεις του ακόλουθου $ 2 \times 2 $ συστήματος εξισώσεων οι οποίες 
  προκύπτουν με παραγώγιση των εξισώσεων του συστήματος $ (\Sigma) $, ως προς $x$.
  \renewcommand{\arraystretch}{2}
  \[
    \begin{aligned}
      (\Sigma_x): \left\{\begin{tabular}{l}
          $\pdv{F}{x} + \pdv{F}{y}\dv{y}{x} + \pdv{F}{z}\dv{z}{x} = 0$ \\
          $\pdv{G}{x} + \pdv{G}{y}\dv{y}{x} + \pdv{G}{z}\dv{z}{x} = 0$
        \end{tabular}
      \right.
    \end{aligned}
  \]
\end{rem}

\subsubsection{2η περίπτωση}

Έστω το σύστημα εξισώσεων $(\Sigma):
\begin{cases}
  F(x,y,z,w) = 0  \\
  G(x,y,z,w) = 0
\end{cases}$
και έστω το σημείο $ P_0(x_0,y_0,z_0,w_0) $, όπου ισχύει:
\begin{enumerate}[(i)]
  \item  \begin{tabular}{l}
      $F(x_0,y_0,z_0,w_0) = 0$ \\
      $G(x_0,y_0,z_0,w_0) = 0$
    \end{tabular}
  \item Οι συναρτήσεις $ F, G $ είναι $ C^{1} $ τάξης 
  \item $ \eval{\pdv{(F,G)}{(z,w)}}_{P_0} \neq 0 $ 
\end{enumerate}
τότε υπάρχουν μοναδικές συναρτήσεις $ z = z(x,y) $ και $ w = w(x,y) $ τάξης 
$ C^{1} $ τέτοιες ώστε:
\begin{myitemize}
  \item \begin{tabular}{l}
      $ z_0 = z(x_0,y_0) $ \\
      $ w_0 = w(x_0,y_0) $
    \end{tabular}
  \item \begin{tabular}{l}
      $ F(x,y,z(x,y), w(x,y) = 0 $ \\
      $ G(x,y,z(x,y), w(x,y)) = 0 $
    \end{tabular}
  \item $ \pdv{z}{x} = - \frac{\pdv{(F,G)}{(x,w)}}{\pdv{(F,G)}{(x,w)}} $, 
    \; $ \pdv{z}{y} = - \frac{\pdv{(F,G)}{(y,w)}}{\pdv{(F,G)}{(z,w)}} $, 
    \; $ \pdv{w}{x} = - \frac{\pdv{(F,G)}{(z,x)}}{\pdv{(F,G)}{(z,w)}} $ και 
    $ \pdv{w}{y} = - \frac{\pdv{(F,G)}{(z,y)}}{\pdv{(F,G)}{(z,w)}} $
\end{myitemize}

\begin{rem}
  Οι παραπάνω τύποι για τις τιμές των μερικών παραγώγων 
  $ \pdv{z}{x}, \pdv{z}{y}, \pdv{w}{x}, \pdv{w}{y} $ προκύπτουν ως λύσεις των 
  παρακάτω $ 2 \times 2 $ γραμμικών συστημάτων τα οποία προκύπτουν με παραγώγιση 
  των εξισώσεων του συστήματος $ (\Sigma) $, ως προς $x$ και $y$ αντίστοιχα.
  \renewcommand{\arraystretch}{2}
  \[
    \begin{aligned}
      (\Sigma_x): 
      \left\{\begin{tabular}{l}
          $\pdv{F}{x} + \pdv{F}{z}\pdv{z}{x} + \pdv{F}{w}\pdv{w}{x} = 0$ \\
          $\pdv{G}{x} + \pdv{G}{z}\pdv{z}{x} + \pdv{G}{w}\pdv{w}{x} = 0$
        \end{tabular}
      \right.  &\quad \text{και} \quad& (\Sigma_y): 
      \left\{\begin{tabular}{l}
          $\pdv{F}{y} + \pdv{F}{z}\pdv{z}{y} + \pdv{F}{w}\pdv{w}{y} = 0$ \\
          $\pdv{G}{y} + \pdv{G}{z}\pdv{z}{y} + \pdv{G}{w}\pdv{w}{y} = 0$
        \end{tabular}
      \right.
    \end{aligned}
  \]
\end{rem}




\section{Ιακωβιανές Ορίζουσες}

\subsection{Ορισμός}

Έστω $ \begin{cases} 
  f^{1}=f^{1}(x_{1},\ldots,x_{n}) \\
  f^{2}=f^{2}(x_{1},\ldots,x_{n}) \\
  \vdots \\
  f^{n} = f^{n}(x_{1}\ldots,x_{n}) 
\end{cases} 
$, τότε η Ιακωβιανή ορίζουσα, είναι 
$ J = \pdv{(f^{1},\ldots,f^{n})}{(x_{1}\ldots,x_{n})} = \begin{vmatrix}
  f^{1}_{x_{1}} & f^{1}_{x_{2}} & \cdots & f^{1}_{x_{n}} \\
  f^{2}_{x_{1}} & f^{2}_{x_{2}} & \cdots & f^{2}_{x_{n}} \\
  \vdots & \vdots & \cdots & \vdots \\
  f^{n}_{x_{1}} & f^{n}_{x_{2}} & \cdots & f^{n}_{x_{n}} \\
\end{vmatrix}$

\begin{rem}
  Η κύρια χρησιμότητά τους είναι στην εύρεση των μερικών 
  παραγώγων πεπλεγμένων συναρτήσεων, όπως εξηγείται στον 
  παρακάτω γενικό κανόνα.
\end{rem}


\subsection{Γενικός Μνημονικός Κανόνας}

Όταν ζητάμε την μερική Παράγωγο μιας εξαρτημένης μεταβλητής 
$ (\text{Ε.Μ.$^{*}$}) $, ως προς κάποια ανεξάρτητη 
μεταβλητή $ (\text{Α.Μ.$^{*}$}) $, τότε αυτή είναι 
ίση με μείον το πηλίκο της Ιακωβιανής ορίζουσας 
των Πεπλεγμένων συναρτήσεων 
ως προς τις εξαρτημένες μεταβλητές όπου όμως έχουμε αντικαταστήσει την 
εξαρτημένη μεταβλητή $ (\text{Ε.Μ.$^{*}$}) $ με την ανεξάρτητη μεταβλητή 
$ (\text{Α.Μ.$^{*}$}) $ προς την Ιακωβιανή ορίζουσα των 
Πεπλεγμένων συναρτήσεων ως προς τις εξαρτημένες μεταβλητές.

\[
  \pdv{(\text{Ε.Μ.}^{*})}{(\text{Α.Μ.}^{*})} = - 
  \frac{J \; (\text{Πεπλ.\ ως προς E.M.}^{*} 
  \to A.M.^{*})}{J \; (\text{Πεπλεγμένων ως προς E.M.)}} 
\] 

\begin{example}
\item {}
  \begin{enumerate}
    \item Έστω το σύστημα
      $ \begin{cases}
        F(u,v,w,x,y)  = 0 \\
        G(u,v,w,x,y)  = 0 \\
        H(u,v,w,x,y)  = 0
      \end{cases} $. Τότε έχουμε Ε.Μ.:3 (όσες και οι 
      εξισώσεις) και Α.Μ.:2 (οι υπόλοιπες). 
      \begin{myitemize}
        \item Οπότε, αν θεωρήσουμε ως ανεξάρτητες μεταβλητές τις $x$ και $y$, τότε, 
          έχουμε:
          \[
            \left.\pdv{u}{\textcolor{Col1}{x}}\right)_{y} = - 
            \frac{\pdv{(F,G,H)}{(\textcolor{Col1}{x},v,w)}}{\pdv{(F,G,H)}{(u,v,w)}}  
            \quad \text{και} \quad \left. \pdv{v}{\textcolor{Col1}{y}} \right)_{x} = - 
            \frac{\pdv{(F,G,H)}{(u,\textcolor{Col1}{y},w)}}{\pdv{(F,G,H)}{(u,v,w)}} 
            \quad \text{και} \quad
            \left.\pdv{w}{\textcolor{Col1}{y}}\right)_{x} = 
            - \frac{\pdv{(F,G,H)}{(u,v,\textcolor{Col1}{y})}}{\pdv{(F,G,H)}{(u,v,w)}} 
          \] 
        \item ενώ αν θεωρήσουμε ως ανεξάρτητες μεταβλητές είτε τις $v$, $w$, είτε τις 
          $u$, $y$, τότε έχουμε:
          \[
            \quad \text{και} \quad \left.  \pdv{x}{\textcolor{Col1}{v}} \right)_{w} = - 
            \frac{\pdv{(F,G,H)}{(u,\textcolor{Col1}{v},y)}}{\pdv{(F,G,H)}{(u,x,y)}} 
            \quad \text{και} \quad \left.  \pdv{w}{\textcolor{Col1}{u}} \right)_{y} = - 
            \frac{\pdv{(F,G,H)}{(v,\textcolor{Col1}{u},x)}}{\pdv{(F,G,H)}{(v,w,x)}} 
          \] 
      \end{myitemize}
  \end{enumerate}
\end{example}

\begin{rem}
  Ο συμβολισμός $ \textstyle{\left. \pdv{u}{\textcolor{Col1}{x}} \right)_{y}} $ 
  σημαίνει ότι υπολογίζουμε την μερική παράγωγο της εξαρτημένης μεταβλητής $u$ ως 
  προς την ανεξάρτητη μεταβλητή $x$, θεωρώντας και τη μεταβλητή $y$, ως ανεξάρτητη.
\end{rem}

\begin{example}[Θέμα Εξετάσεων]
  Θεωρούμε τις $ u= \sigma (x,y) $ και $ v=h(x,y) $ οι οποίες ορίζονται από τις σχέσεις
  $ v = \textstyle{\pdv{f(x,u)}{u}} $ και $ y= -\textstyle{\pdv{f(x,u)}{x}} $, 
  όπου $ f $ συνάρτηση των μεταβλητών $x$ και $u$. Να δείξετε ότι 
  \[
    \pdv{(u,v)}{(x,y)} = 1 
  \] 
\end{example}
\begin{solution}
  Σχηματίζουμε το σύστημα των πεπλεγμένων εξισώσεων, 4 μεταβλητών 
  \begin{align*}
    F(x,y,u,v) &= v - \pdv{f(x,u)}{u} = 0 \Leftrightarrow v - f_{u}(x,u) = 0\\
    G(x,y,u,v) &= y + \pdv{f(x,u)}{x} = 0 \Leftrightarrow y + f_{x}(x,u) = 0   
  \end{align*} 
  Για να υπολογίσουμε την ορίζουσα $ \pdv{(u,v)}{(x,y)} = 
  \begin{vmatrix*}
    u_{x} & u_{y} \\
    v_{x} & v_{y}
  \end{vmatrix*} $ 
  αρκεί να υπολογίσουμε τις μερ. παραγώγους $ u_{x},u_{y},v_{x} $ και $ v_{y} $. 
  \begin{align*}
    \pdv{(F,G)}{(u,v)} = 
    \begin{vmatrix*}
      F_u & F_v \\
      G_u & G_v
    \end{vmatrix*} = 
    \begin{vmatrix*}
      -f_{uu} & 1 \\
      f_{xu} & 0
    \end{vmatrix*} = 
    -f_{xu} \\
    \intertext{Επίσης}
    u_x = 
    -\frac{\pdv{(F,G)}{(x,v)}}{\pdv{(F,G)}{(u,v)}} = 
    -\frac{
      \begin{vmatrix*}
        F_x & F_v \\
        G_x & G_v
    \end{vmatrix*}}{-f_{xu}} = 
    \frac{
      \begin{vmatrix*}
        -f_{ux}  && 1 \\
        f_{xx} && 0
    \end{vmatrix*}}{f_{xu}} = 
    -\frac{f_{xx}}{f_{xu}} 
  \end{align*}
  \begin{align*}
    u_y = 
    -\frac{\pdv{(F,G)}{(y,v)}}{\pdv{(F,G)}{(u,v)}} = 
    -\frac{
      \begin{vmatrix*}
        F_y & F_v \\
        G_y & G_v
    \end{vmatrix*}}{-f_{xu}} = 
    \frac{
      \begin{vmatrix*}
        0 && 1 \\
        1 && 0
    \end{vmatrix*}}{f_{xu}} = 
    -\frac{1}{f_{xu}} 
  \end{align*}
  \begin{align*}
    v_x = 
    -\frac{\pdv{(F,G)}{(u,x)}}{\pdv{(F,G)}{(u,v)}} = 
    -\frac{
      \begin{vmatrix*}
        F_u & F_x \\
        G_u & G_x
    \end{vmatrix*}}{-f_{xu}} = 
    \frac{
      \begin{vmatrix*}
        -f_{uu} && -f_{ux} \\
        f_{xu} && f_{xx}            
    \end{vmatrix*}}{f_{xu}} = 
    \frac{-f_{uu}f_{xx}+ f^{2}_{xu}}{f_{xu}} 
  \end{align*}
  \begin{align*}
    v_y = 
    -\frac{\pdv{(F,G)}{(u,y)}}{\pdv{(F,G)}{(u,v)}} = 
    -\frac{
      \begin{vmatrix*}
        F_u & F_y \\
        G_u & G_y
    \end{vmatrix*}}{-f_{xu}} = 
    \frac{
      \begin{vmatrix*}
        -f_{uu} && 0 \\
        f_{xu} && 1
    \end{vmatrix*}}{f_{xu}} = -\frac{f_{uu}}{f_{xu}} 
  \end{align*}
  Άρα 
  \begin{align*} 
    \pdv{(u,v)}{(x,y)} &= 
    \begin{vmatrix*}
      u_{x} & u_{y} \\
      v_{x} & v_{y}
    \end{vmatrix*} = u_{x}v_{y}-u_{y}v_{x} = 
    \left( -\frac{f_{xx}}{f_{xu}} \right) \cdot 
    \left( -\frac{f_{uu}}{f_{xu}} \right)- 
    \left( - \frac{1}{f_{xu}} \right) \cdot 
    \left( \frac{-f_{uu}f_{xx}+ f^{2}_{xu}}{f_{xu}} \right) \\ 
                       &= \frac{f_{xx}f_{uu}}{f^{2}_{xu}} -
                       \frac{f_{uu}f_{xx} -
                       f^2_{xu}}{f^{2}_{xu}}=1
  \end{align*}
\end{solution}


\subsection{Θεωρήματα για Ιακωβιανές Ορίζουσες}

\begin{enumerate}
  \item Μια ικανή και αναγκαία συνθήκη ώστε το σύστημα, 
    \[
      \begin{cases}
        F(u,v,x,y,z) = 0 \\
        G(u,v,x,y,z) = 0
      \end{cases}
    \]
    να μπορεί να λυθεί, για παράδειγμα, ως προς 
    $u$ και $v$, είναι η Ιακωβιανή Ορίζουσα
    \[
      J = \pdv{(F,G)}{(u,v)} \neq 0 
    \] 

  \item Ο κανόνας που χρησιμοποιούμε για την εύρεση 
    Ιακωβιανών οριζουσών σύνθετων συναρτήσεων είναι 
    ίδιος με τον κανόνα αλυσίδας μερικών παραγώγων.

    Έστω 
    \[
      \begin{cases} x=x(u,v) \\
      y=y(u,v)\end{cases} \quad \text{και} \quad 
      \begin{cases} 
        u = u(r,s) \\
        v=v(r,s) 
      \end{cases} 
    \] 
    Τότε
    \[
      J = \pdv{(x,y)}{(r,s)} = \pdv{(x,y)}{(u,v)} \cdot \pdv{(u,v)}{(r,s)}  
    \] 
  \item Αν $ \begin{cases} u=u(x,y) \\ v=v(x,y) \end{cases} $ τότε μια ικανή 
    και αναγκαία συνθήκη ώστε να υπάρχει μια συναρτησιακή σχέση μεταξύ των 
    $ u $ και $v $ είναι η Ικανωβιανή ορίζουσα $ J = \pdv{(u,v)}{(x,y)} = 0 $.
    Δηλαδή:
    \[
      F(u,v)=0 \Leftrightarrow J = \pdv{(u,v)}{(x,y)} = 0 
    \] 
\end{enumerate}






