\input{preamble_ask.tex}
\input{definitions_ask.tex}
\input{myboxes.tex}

\geometry{top=1cm}

\pagestyle{vangelis}
\everymath{\displaystyle}
\setcounter{chapter}{1}

\begin{document}

\chapter*{ΜΔΕ 1ης τάξης}

\section*{Ταξινόμηση ΜΔΕ 1ης τάξης}

Η γενική μορφή μιας μδε 1ης τάξης είναι 
\begin{description}
  \item [στον $\mathbb{R}^{2}$:] $ F(x,y,u,u_{x},u_{y}) = 0, \quad (x,y) \in D 
    \subseteq \mathbb{R}^{2} $  
  \item [στον $\mathbb{R}^{3}:$] $ F(x,y,z,u,u_{x},u_{y},u_{z}) = 0, \quad (x,y,z) \in V 
    \subseteq \mathbb{R}^{3} $ 
\end{description} 

Οι γραμμικές εξισώσεις, συνήθως ταξινομούνται σε μία από τις παρακάτω περιπτώσεις:
\begin{description}
  [widest=Σχεδον Γραμμικες:,labelindent=1em,labelsep*=1em, itemindent=0pt,leftmargin=*]
\item [Γραμμικές:] $ a(x,y)u_{x}+b(x,y)u_{y}+c(x,y)u=d(x,y) $
\item [Σχεδόν Γραμμικές:] $ a(x,y)u_{x}+b(x,y)u_{y}=c(x,y,u) $
\item [Ημιγραμμικές:] $ a(x,y,u)u_{x}+b(x,y,u)u_{y}=c(x,y,u) $
\end{description}

\section*{Μέθοδος των Χαρακτηριστικών}

Αναζητούμε μία μέθοδο, ώστε να μπορούμε να λύνουμε την πιο γενική περίπτωση των 
γραμμικών μδε 1ης τάξης, που είναι οι Ημιγραμμικές
\begin{equation}\label{eq:hmigram}
  a(x,y,u)u_{x}+b(x,y,u)u_{y}=c(x,y,u) 
\end{equation}
Ας υποθέσουμε ότι βρισκόμαστε στον 3-διάστατο χώρο με συντεταγμένες $ (x,y,u) $. 
Τότε μία συνάρτηση $ u=u(x,y) $ η οποία είναι λύση της διαφορικής μας εξίσωσης, θα 
παριστάνει μία επιφάνεια σε αυτό το χώρο. Στη συνέχεια, παρατηρούμε ότι η
εξίσωση~\eqref{eq:hmigram} μπορεί να γραφεί με τη βοήθεια του εσωτερικού γινομένου, ως
εξής:
\begin{equation}\label{eq:perp}
  \begin{pmatrix*}[r] a(x,y,u) \\ b(x,y,u) \\ c(x,y,u) \end{pmatrix*}^{T} \cdot 
  \begin{pmatrix*}[r] u_{x} \\ u_{y} \\ -1 \end{pmatrix*} = 0 \Leftrightarrow 
  \begin{pmatrix*}[r] a(x,y,u) \\ b(x,y,u) \\ c(x,y,u) \end{pmatrix*} \perp 
  \begin{pmatrix*}[r] u_{x} \\ u_{y} \\ -1 \end{pmatrix*} = 0 
\end{equation}
Επομένως, αρκεί να κατανοήσουμε τη σχέση του διανύσματος $ (u_{x},u_{y},-1)^{T} $ 
με τη λύση $ u=u(x,y) $ της εξίσωσης.

Θεωρούμε τη συνάρτηση $ \Phi \colon \mathbb{R}^{3} \to \mathbb{R} $ με τύπο 
\[
  \Phi(x,y,u)=u-u(x,y) 
\] 
και παρατηρούμε ότι 
\[
  \begin{pmatrix*}[r] u_{x} \\ u_{y} \\ -1 \end{pmatrix*} = -
  \begin{pmatrix*}[r] \Phi _{x} \\ \Phi _{y} \\ \Phi _{u} \end{pmatrix*} = 
  - \grad \Phi 
\]
όπου η κλίση $ \grad \Phi $ (αρα και η $ - \grad \Phi $), είναι ένα κάθετο διάνυσμα 
της επιφάνειας $ \Phi = 0 \Leftrightarrow u=u(x,y) $. Επομένως, το διάνυσμα 
$ (u_{x},u_{y},-1)^{T} $, είναι κάθετο στην επιφάνεια $ u=u(x,y) $, η οποία είναι 
λύση της διαφορικής εξίσωσης.
Επειδή όμως, από τη σχέση~\eqref{eq:perp}, το διάνυσμα $ (a,b,c)^{T} $ 
είναι κάθετο στο $ (u_{x},u_{y},-1)^{T} $, και αυτό με τη σειρά του είναι κάθετο 
στην επιφάνεια $ u=u(x,y) $, αυτό σημαίνει ότι το διάνυσμα 
$ (a,b,c)^{T} $ πρέπει να είναι παράλληλο στην επιφάνεια $ u=u(x,y) $. Επομένως, 
αυτό που έχουμε δείξει είναι ότι η διαφορική εξίσωση~\eqref{eq:hmigram} είναι 
ισοδύναμη με την γεωμετρική απαίτηση, το διάνυσμα $ (a,b,c)^{T} $ να είναι παράλληλο
στην επιφάνεια $ u=u(x,y) $ η οποία είναι λύση της εξίσωσης. 
Κατά συνέπεια, κάθε ολοκληρωτική καμπύλη του διανυσματικού πεδίου $ (a,b,c)^{T} $,
δηλαδή κάθε καμπύλη της μορφής $ (x(t),y(t),u(t)) $ που ικανοποιεί τις εξισώσεις:
%todo des ζυγκιρίδη (το εξηγεί καλά)
\[
  \left.
    \begin{aligned}
  \dv{x}{t} = a(x,y,u) \\
  \dv{y}{t} = b(x,y,u) \\
  \dv{u}{t} = c(x,y,u) 
    \end{aligned} 
  \right\} \Leftrightarrow 
  \frac{dx}{a} = \frac{dy}{b} = \frac{du}{c}
 \]
% \begin{gather*}
%   \dv{x}{t} = a(x,y,u) \\
%   \dv{y}{t} = b(x,y,u) \\
%   \dv{u}{t} = c(x,y,u) 
% \end{gather*} 
πρέπει να περιέχεται σε κάποια από τις επιφάνειες που αποτέλουν τη λύση της μδε. 
Αντίστροφα, κάθε επιφάνεια, που σχηματίζεται από αυτές τις ολοκληρωτικές καμπύλες, 
θα πρέπει να είναι λύση της μδε.

Η ανάλυση που κάναμε παραπάνω, οδήγησε τον Lagrange, στην ακόλουθη μέθοδο επίλυσης, 
που ονομάζεται "μέθοδος των χαρακτηριστικών". 

\begin{mybox2}
\begin{thm}
  Η γενική λύση της ημιγραμμικής μδε 1ης τάξεως 
  \[
    a(x,y,u)u_{x} + b(x,y,u)u_{y} = c(x,y,u) 
  \] 
  είναι η $ F(\phi, \psi) = 0 $, όπου $F$ είναι μια αυθαίρετη συνάρτηση δύο μεταβλητών,
  και οι $ \phi (x,y,u) = c_{1} $ και $ \psi (x,y,u) = c_{2} $, αποτελούν λύσεις των 
  χαρακτηριστικών εξισώσεων
  \[
    \frac{dx}{a} = \frac{dy}{b} = \frac{du}{c} 
  \] 
\end{thm}
\end{mybox2}

%todo να συνεχισω, διαβασμα και γραψιμο απο το αρχειο best(ημιγραμ).pdf






\end{document}
