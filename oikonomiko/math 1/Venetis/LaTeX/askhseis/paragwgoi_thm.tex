\input{preamble_ask.tex}
\input{definitions_ask.tex}

\pagestyle{vangelis}

\begin{document}

\begin{center}
  \minibox{\large\bfseries \textcolor{Col1}{Θέματα Εξετάσεων}}
\end{center}

\vspace{\baselineskip}

\section{Παράγωγος}

\begin{enumerate}
  \item Αν $ p(t)= \frac{\mathrm{e}^{\beta t}}{t} $ δίνει την κίνηση των τιμών στο χρόνο 
    και $ \beta > 0 $, τότε 
    \begin{enumerate}[i)]
      \item Δώστε τον πληθωρισμό $ \pi (t) = \frac{\dv{p(t)}{t}}{p(t)} $ ως συνάρτηση του
        χρόνου 
      \item Υπολογίστε τη μεταβολή του πληθωρισμού στο χρόνο.
    \end{enumerate}
\end{enumerate}

\section{Κανόνας Αλυσίδας}

\begin{enumerate}
  \item {(2017)} Έστω ότι οι τιμές στο χρόνο κινούνται σύμφωνα με την $ P= A \mathrm{e}^{at} $. 
    Επίσης, έστω ότι η ζήτηση δίνεται από την $ Q = Bp^{- \epsilon} $, όπου $ A,B,
    \epsilon > 0 $ και $ 0<a<1 $. 
    \begin{enumerate}[i)]
      \item Χρησιμοποιήστε τον \textbf{αλυσωτό κανόνα} για να υπολογίσετε
        τη μεταβολή της ζήτησης ως προς το χρόνο.

        \hfill Απ: $ \dv{Q}{t} = - \epsilon aBA^{- \epsilon} \mathrm{e}^{-a\epsilon t} $ 

        \end{enumerate}

      \item Υπολογίστε με τον αλυσωτό κανόνα τη μεταβολή της ζήτησης του αγαθού $A$ στον
        χρόνο, $ \dv{Q(t)}{t} $, αν γνωρίζετε ότι 
        \begin{enumerate}[i)]
          \item η ζήτηση του αγαθού $A$ εξαρτάται από την τιμή του, $ Q= \alpha -
            \beta P_{A} $
          \item η τιμή του αγαθού είναι διασυνδεδεμένη με την τιμή ενός ανταγωνιστικού
            αγαθού $ P_{B} $ με βάση την $ P_{A}= \gamma + \delta P_{B} $
          \item η τιμή του ανταγωνιστικού αγαθού μεταβάλλεται στο χρόνο σύμφωνα με την 
            $ P_{B}=k + \mathrm{e}^{-mt} $. 
    \end{enumerate}

        Για τις σταθερές ισχύει ότι $ \alpha, \beta , \gamma , \delta , k , m >0 $
\end{enumerate}


\begin{center}
  \minibox{\large\bfseries \textcolor{Col1}{Ελαστικότητα}}
\end{center}

\vspace{\baselineskip}

\begin{enumerate}
  \item Αν $ A=1, \; B=1 $, ο ρυθμός μεγέθυνσης του πληθωρισμού είναι σταθερός στον
    χρόνο και ίσος με $ 2.5\% $, δηλαδή $ a=0.025 $, και η ελαστικότητα ζήτησης είναι
    μοναδιαία, υπολογίστε τον ρυθμό μεγέθυνσης της ζήτησης στο χρόνο και σχολιάστε
    σύντομα. 
    \hfill Απ: $ \frac{dQ/dt}{Q} = -0.025 $, ίσος με πληθ. 
  \item Μια μονοπωλιακή επιχείρηση αντιμετωπίζει γραμμική συνάρτηση ζήτησης $ Q = a-bP $ 
    με παραμέτρους $ a,b>0 $. 
    \begin{enumerate}[i)]
      \item Υπολογίστε την ελαστικότητα των εσόδων της επιχείρησης ως
        προς την τιμή.
        \hfill Απ: $ \varepsilon _{TR} = \frac{a-2bP}{a-bP} $ 
      \item Σχολιάστε την πρόταση ενός συναδέλφου ότι μια ποσοστιαία αύξηση στην τιμή
        οδηγεί πάντα σε αύξηση των εσόδων.
      \item Σχολιάστε την πρόταση ενός συναδέλφου ότι η ελαστικότητα των εσόδων στη
        συγκεκριμένη αγορά μπορεί να είναι ελαστική ανάλογα με τις παραμέτρους της
        γραμμικής ζήτησης.
    \end{enumerate}
\end{enumerate}

\begin{center}
  \minibox{\large\bfseries \textcolor{Col1}{Taylor}}
\end{center}

\vspace{\baselineskip}

\begin{enumerate}
  \item Υπολογίστε την προσέγγιση κατά Taylor 4ης τάξης της συνάρτησης $ f(x) =
    \frac{1}{1-x} $ γύρω από το σημείο $ x_{0}=0 $. \hfill Απ: $ f(x) \approx
    1+x+x^{2}+x^{3}+x^{4} $ 
  \item Για ποιές τιμές του $ x $, γύρω από το $ x_{0}=1 $, είναι το σφάλμα προσέγγισης
    Taylor 2ης τάξης της συνάρτησης $ f(x)=-2x^{3}+3x^{2}+11x-1 $ μικρότερο από 0.1;
    \hfill Απ: $ x \in (0.631597,1.368403) $ 
\end{enumerate}

\begin{center}
  \minibox{\large\bfseries \textcolor{Col1}{Ολοκληρώματα}}
\end{center}

\begin{enumerate}
  \item Υπολογίστε το ολοκλήρωμα $ \int _{0.5}^{1} \frac{1}{1-2x} \,{dx} $ 
    \hfill Απ: $ - \infty $ 
\end{enumerate}

\begin{center}
  \minibox{\large\bfseries \textcolor{Col1}{Πλεόνασμα Παραγωγού - Καταναλωτή}}
\end{center}

\begin{enumerate}
  \item Δίνονται οι συναρτήσεις $ Q = \sqrt{P} $ και $ Q = \frac{1}{\sqrt{P}} $ 
    με πεδία ορισμού, $ P \in [0, + \infty] $ και $ P \in (0,2] $ αντίστοιχα. 
    Υπολογίστε το κοινωνικό πλεόνασμα στην τιμή ισορροπίας.

    \hfill Απ: Π.Κ. $= 2(\sqrt{2} -1)$, Π.Π. $= \frac{2}{3} $, Κ.Π.=Π.Κ.+Π.Π $ \approx
    1.495 $ 
\end{enumerate}

\begin{center}
  \minibox{\large\bfseries \textcolor{Col1}{Ακρότατα}}
\end{center}

\begin{enumerate}
  \item Μια επιχείρηση, σε καθεστώς τέλειου ανταγωνισμού, έχει συνάρτηση κόστους 
    $ TC(Q)= \mathrm{e}^{\gamma Q} $ με $ \gamma > 0 $ και $ P > \gamma $. Προβείτε σε 
    βελτιστοποίηση της συνάρτησης κέρδους ως προς την ποσότητα παραγωγής. Χαρακτηρίστε το
    βέλτιστο σημείο. Δώστε τη συνάρτηση προσφοράς της επιχείρησης.

    \hfill Απ: $ Q^{*}= \frac{\ln{P/ \gamma}}{\gamma} $, ολικό και μοναδικό μέγιστο. 
    $ S(P)= 
     \begin{cases}
       0, & P< \gamma \\
       \frac{\ln{P/ \gamma}}{\gamma}, & P \geq \gamma
        \end{cases}$
\end{enumerate}

\end{document}
