\input{preamble_ask.tex}
\input{definitions_ask.tex}

\geometry{top=1.5cm}
\pagestyle{askhseis}

\begin{document}

\begin{center}
  \minibox{\bfseries\large\color{Col2} Μερική Παράγωγος}
\end{center} 

\vspace{\baselineskip} 

\section*{Ορισμός}

\begin{enumerate}
  \item Έστω η συνάρτηση $ f(x,y) = xy $. Να υπολογιστούν με τη 
    βοήθεια του ορισμού οι $ f_{x}(1,1) $ και η $ f_{y}(1,1) $. 

    \hfill Απ: 
    \begin{tabular}{l}
      $f_{x}(1,1) = 1$ \\
      $f_{y}(1,1) = 1$
    \end{tabular} 

  \item Να υπολογιστούν οι μερικές παράγωγοι 1ης τάξης της συνάρτησης
    $
    f(x,y) = 
    \begin{cases}
      x^{2} \sin{\frac{y}{x}}, & x \neq 0 \\
      0, & x = 0 
    \end{cases}
    $ 

    \hfill Απ: 
    $  f_{x} = 
    \begin{cases}
      2x \sin{\frac{y}{x}} - y \cos{\frac{y}{x}}, & 
      x \neq 0  \\ 
      0, & x = 0 
    \end{cases} $ \quad 
    και \quad
    $ f_{y} = 
    \begin{cases}
      x \cos{\frac{y}{x}}, & x \neq 0 \\
      0, & x = 0 
    \end{cases} $    

\end{enumerate}


\section*{Κανόνες Παραγώγισης}

\begin{enumerate}
  \item Με τη βοήθεια των κανόνων παραγώγισης να υπολογιστούν οι μερικές 
    παράγωγοι 1ης τάξης των παρακάτω συναρτήσεων:

    \begin{enumerate}[i)]
      \item $f(x,y)=y\sin (xy)$ \hfill Απ: \begin{tabular}{l}
          $f_x=y^2\cos(xy)$ \\ 
          $f_y=\sin(xy)+yx\cos(xy)$
        \end{tabular}

      \item $f(x,y)=\arcsin(\frac{x}{y})$\hfill Απ: \begin{tabular}{l}
          $f_x=\frac{1}{\sqrt{y^2-x^2}}$ \\ 
          $f_y=-\frac{x}{y\sqrt{y^2-x^2}}$
        \end{tabular}
      \item $ f(x,y,z) = (x+y^{2}) \sin{(xz)} $ \hfill Απ: \begin{tabular}{l}
          $ f_{x} = \sin{(xz)} + z(x+y^{2}) \sin{(xz)} $ \\
          $ f_{y} = 2y \sin{(xz)} $ \\
          $ f_{z} = x(x+y^{2}) \sin{(xz)} $
        \end{tabular} 
    \end{enumerate}

  \item Έστω η συνάρτηση $f(x,y)=\ln\left(\cos y+x\cos x\right)$.  Να υπολογισθεί 
    η $ f_{xy} $ στο σημείο $\left(\pi,-\frac{\pi}{2}\right)$.
    \hfill Απ: $\frac{1}{\pi^2}$

  \item Έστω η συνάρτηση $ f(x,y) = x^{y} $. Να δείξετε ότι ισχύει ότι το
    \textbf{θεώρημα Schwartz}, δηλαδή ότι $ f_{xy} = f_{yx} $.
    %hlektr
  \item Να δείξετε ότι η συνάρτηση $ f(x,y) = (y+3x)^{1/2} - 
    (y-3x)^{2} $ ικανοποιεί τη σχέση $ f_{xx} - 9 f_{yy} = 0 $.
    %span
  \item Να δείξετε ότι η συνάρτηση $ f(x,y) = \cos{(x+y)} + \cos{(x-y)} $ 
    επαληθεύει την διαφορική εξίσωση $ f_{xx} - f_{yy} = 0 $.

  \item Να δείξετε ότι η συνάρτηση $ f(x,y) = x \arctan{\frac{y}{x}} $ 
    ικανοποιεί την διαφορική εξίσωση $ x^{2} f_{xx} + 2xyf_{xy} + y^{2} f_{yy} = 0 $ 
\end{enumerate}

\section*{Διαφορισιμότητα}

\begin{enumerate}
  \item Να εξεταστεί αν οι παρακάτω συναρτήσεις είναι διαφορίσιμες στο σημείο 
    $ (0,0) $.

    \begin{enumerate*}[i),itemjoin=\hspace{1.5cm}]
      \item  $ f(x,y) = 
        \begin{cases} 
          \frac{xy}{\sqrt{x^{2}+y^{2}}}, & (x,y) \neq (0,0) \\
          0, & (x,y) = (0,0)
        \end{cases} $ 
      \item $ g(x,y) = 
        \begin{cases} 
          \frac{(x+y)^{2}}{x^{2}+y^{2}}, & (x,y) \neq (0,0) \\
          0, & (x,y) = (0,0)
        \end{cases} $
    \end{enumerate*}
    \hfill Απ: 
    \begin{enumerate*}[i),itemjoin=\hspace{10pt}]
      \item όχι
      \item όχι
    \end{enumerate*}

  \item Να εξετάσετε αν η συνάρτηση 
    $
    f(x,y) = 
    \begin{cases} 
      xy \frac{x^{2}-y^{2}}{x^{2}+y^{2}}, & (x,y) 
      \neq (0,0) \\ 
      0, & (x,y) = (0,0)
    \end{cases}
    $ 
    είναι διαφορίσιμη στο σημείο $ (0,0) $. 
    \hfill Απ: ναι 
\end{enumerate}

\section*{Αρμονικές Συναρτήσεις}

\begin{enumerate}

  \item Να δείξετε ότι οι παρακάτω συναρτήσεις είναι αρμονικές:
    \begin{enumerate}[(i)]
      \item $ f(x,y) = x^{3}-3xy^{2} $
      \item $ f(x,y) = \ln(x^{2} + y^{2}) $
      \item $ f(x,y) = \ln{(x^{2}+y^{2})} + \arctan(\frac{y}{x}) $
      % \item $ f(x,y) = \frac{ x }{ 2 } \ln(x^{2} + y^{2}) - y 
      %   \arctan(\frac{ y }{ x } ) $
    \end{enumerate}

  \item Να αποδείξετε ότι αν μία συνάρτηση $f(x,y)$, που έχει συνεχείς 
    μερικές παραγώγους 2ης τάξης είναι αρμονική, τότε και οι συναρτήσεις 
    \begin{enumerate*}[i),itemjoin=\hspace{7pt}]
      \item $ f_{x} $ 
      \item $ f_{y} $
      \item $ xf_{x}+yf_{y} $
      \item $ xf_{x}-yf_{y} $
    \end{enumerate*}
    είναι αρμονικές.
\end{enumerate}

\section*{Ομογενείς Συναρτήσεις}

\begin{enumerate}
  \item Αν $ u = u(x,y) $ και $ v=v(x,y) $ ομογενείς βαθμού $ \rho $, 
    τότε να δείξετε ότι $ \forall f(u,v) $ με συνεχείς μερικές παραγώγους 1ης τάξης
    ισχύει 
    \[
      xf_{x}+yf_{y}= \rho (u f_{u}+vf_{v}) 
    \] 
  \item Να αποδείξετε ότι αν $f(x,y)$ είναι ομογενής συνάρτηση, βαθμού $ \rho $, τότε 
    οι συναρτήσεις $ f_{x}, f_{y} $ είναι επίσης ομογενείς, βαθμού $ \rho -1 $.
\end{enumerate}

\section*{Διαφορικό}

\begin{enumerate}

  \item Να βρεθεί το ολικό διαφορικό 1ης τάξης, της συνάρτησης 
    $f(x,y)=\ln(xy)+\cos(y^2)$ 

    \hfill Απ: $df=\frac{dx}{x}+\left(\frac{1}{y}-2y\sin(y^2)\right)dy$

  \item Να βρεθεί το ολικό διαφορικό 1ης τάξης, της συνάρτησης 
    $ f(x,y) = \arctan(\frac{ x+y }{ x-y }) $, αν $ x>0 $ και $ y>0 $.

    \hfill Απ: $df = \frac{ -ydx + xdy }{ x^{2} + y^{2} } $ 

  % \item Να βρεθεί το ολικό διαφορικό της συνάρτησης $ f(x,y) = x^{y} \cdot y^{x} $, 
  %   αν $ x>0$ και $ y>0 $.

  %   \hfill Απ: $df =  (x^{y-1}\cdot y^{x+1} + x^{y}\cdot y^{x} \ln{y} )dx 
  %   + (x^{y}\cdot y^{x} \ln{x} + x^{y+1} \cdot y^{x-1})dy $ 

  \item Για τις παρακάτω παραστάσεις, να αποδείξετε ότι είναι \textbf{τέλεια
    διαφορικά} και να υπολογίσετε τη συνάρτηση δυναμικού.
    \begin{enumerate}[i)]
      \item $ \left(x+e^{x/y}\right)dx + e^{x/y}\left(1- \frac{x}{y}\right)dy $
        \hfill Απ: $ f(x,y) = \frac{x^{2}}{2} +y e^{x/y} + c $ 

      \item $\left(2e^{x}+\frac{1}{x}-3\sin y\right)dx+3(y^2-x\cos y)dy$ 
        \hfill  Απ: $ f(x,y,z) = y^{3}-3x \sin{y} + 2e^{x} + \ln{x} +c $.
        %spand (114)

      \item $(2xy+z)dx+(x^{2}+z^{2})dy+(x+2yz)dz$ 
        \hfill  Απ: $ f(x,y,z) = x^{2}y+z^{2}y+xz +c $.

        % \item $ (3x^{2}+3y-1)dx + (z^{2}+3x)dy+(2yz+1)dz $
        % \hfill Απ: $ f(x,y,z) = x^{3}+3xy-x+yz^{2}+z+c $

      \item $ \cos(x+yz)dx + z\cos(x+yz)dy+y\cos(x+yz)dz $
        \hfill Απ: $ f(x,y,z) = \sin(x+yz) + c $
    \end{enumerate}

  \item Να υπολογιστεί το $a$ ώστε η παράσταση $ \frac{ x + ay }{ (x-y)^{3} }dx 
    + \frac{ ax+y }{ (x-y)^{3} }dy $ να είναι \textbf{τέλειο διαφορικό}.

    \hfill Απ: $ a=-1 $
\end{enumerate}

\section*{Εφαρμογές του Διαφορικού}

\begin{enumerate}
  \item Να υπολογιστεί κατά προσέγγιση η τιμή των παραστάσεων
    \begin{enumerate}[i)]
      \item $A = (1,02)^{3,01} $ \hfill Απ: $ A \approx 1,06 $ 
      \item $B =  \sqrt{ 9(1,95)^{2} + (8,1)^{2} } $ 
        \hfill Απ: $ B \approx 9,99 $ 
    \end{enumerate}

  \item Θέλουμε να κατασκευάσουμε ένα ορθογώνιο παραλληλεπίπεδο με ακμές $ x = a $, 
    $ y=b $ και $ z=c $. Αν τα σφάλματα που έγιναν κατά την κατασκευή των ακμών 
    είναι $ \Delta x = 0,01a $, $ \Delta y = -0,02b $ και $ \Delta z = 0,03c $, 
    τότε να υπολογίσετε το σφάλμα που έγινε στον όγκο του παραλληλεπιπέδου.

    \hfill Απ: $ 0,02abc $ 
\end{enumerate}


\section*{Πολυώνυμο Taylor}

\begin{enumerate}
  \item Να βρεθούν τα αναπτύγματα \textbf{Taylor}, μέχρι και όρους 
    \textbf{2ης τάξης}, των συναρτήσεων:

    \begin{enumerate}[i)]
      \item  $f(x,y)=y\cos{xy} $, γύρω από το σημείο 
        $ \left(1, \frac{ \pi }{ 2 }\right) $.

        \hfill Απ: $f(x,y)=-\frac{\pi^{2}}{4}(x-1) - \frac{ \pi }{ 2 } 
        \left(y - \frac{ \pi }{2 }\right) - \pi(x-1)
        \left(y-\frac{\pi}{2}\right)- \left(y- \frac{ \pi }{ 2} \right)^{2} $

      \item $ f(x,y)=e^{x}\tan{y} $ σε δυνάμεις των $ (x-1) $ και 
        $ \left(y - \frac{ \pi }{ 4 }\right) $

        \hfill Απ: $ f(x,y) = e + e(x-1) + 2e\left(y- \frac{ \pi }{ 4 }\right)
        + \frac{1}{ 2 } \left(e(x-1)^{2}+4e(x-1)\left(y- \frac{ \pi }{ 4 }
        \right) + 4e\left(y- \frac{ \pi }{ 4 } \right)^{2}\right) $
    \end{enumerate}

  \item Να βρεθεί το ανάπτυγμα \textbf{Maclaurin}, μέχρι όρους \textbf{2ης
    τάξης}, της συνάρτησης $ f(x,y) = e^{x}\ln(1+y)$.

    \hfill Απ: $ f(x,y)=y + xy - \frac{1}{ 2 } y^{2} + \frac{1}{ 2 } x^{2}y - 
    \frac{1}{ 2 } xy^{2} + \frac{1}{ 3 } y^{3} $
\end{enumerate}




\end{document}
