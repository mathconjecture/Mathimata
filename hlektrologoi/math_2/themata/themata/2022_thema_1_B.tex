\documentclass[a4paper,table]{report}
\input{preamble_themata}
\newcommand{\vect}[2]{(#1_1,\ldots, #1_#2)}
%%%%%%% nesting newcommands $$$$$$$$$$$$$$$$$$$
\newcommand{\function}[1]{\newcommand{\nvec}[2]{#1(##1_1,\ldots, ##1_##2)}}

\newcommand{\linode}[2]{#1_n(x)#2^{(n)}+#1_{n-1}(x)#2^{(n-1)}+\cdots +#1_0(x)#2=g(x)}

\newcommand{\vecoffun}[3]{#1_0(#2),\ldots ,#1_#3(#2)}

\newcommand{\mysum}[1]{\sum_{n=#1}^{\infty}

\input{myboxes}

\everymath{\displaystyle}
\pagestyle{vangelis}
\geometry{left=1.5cm,right=1.5cm}

\begin{document}

\begin{mybox3}
  \begin{thema}
    Δίνεται η συνάρτηση $ f(x,y,z)=x^{3}+y^{3}+z^{3}+3xy+3yz+3xz $. Να βρεθούν τα 
    τοπικά ακρότατα της συνάρτησης.
  \end{thema}
\end{mybox3}
\begin{solution}
  \item []
  \begin{enumerate}
    \item Βρίσκουμε τις μερικές παραγώγους 1ης και 2ης τάξης.

      \twocolumnsidesl{
        \begin{myitemize}
          \item $ f_{x}=3x^{2}+3y+3z $
          \item $ f_{x}= 3y^{2} +3x+3z $
          \item $ f_{z}= 3z^{2} +3y+3x $
        \end{myitemize}
      }{
        \twocolumnsidesl{
          \begin{myitemize}
            \item $ f_{xx}=6x $
            \item $ f_{yy}=6y $
            \item $ f_{zz}=6z $
          \end{myitemize}
        }{
          \begin{myitemize}
            \item $ f_{xy}=f_{yx}=3 $
            \item $ f_{xz}=f_{zx}=3 $
            \item $ f_{yz}=f_{zy}=3 $
          \end{myitemize}
        }
      }

    \item Βρίσκουμε τα κρίσιμα σημεία της συνάρτησης.

      Επειδή η συνάρτηση $ f(x,y) $ είναι συνεχής και διαφορίσιμη σε κάθε σημείο 
      του $ \mathbb{R}^{2} $, αναζητούμε τα κρίσιμα σημεία της στις λύσεις του
      συστήματος:
      \[
        \left.
          \begin{matrix}
            f_{x}=0 \\
            f_{y}=0 \\
            f_{z}=0
          \end{matrix} 
        \right\} \Leftrightarrow 
        \left.
          \begin{matrix}
            3x^2+3y+3z=0 \\
            3y^{2} +3x+3z =0 \\
            3z^{2} +3y+3x =0 
          \end{matrix} 
        \right\} \Leftrightarrow 
        \left.
          \begin{matrix}
            x^2+y+z=0 \\
            y^{2} +x+z =0 \\
            z^{2} +y+x =0 
          \end{matrix} 
        \right\}
      \]
      Αφαιρούμε από την 2η εξίσωση την 3η και έχουμε:
      \[
        \left.
          \begin{matrix}
            x^2+y+z=0 \\
            y^2-z^2+z-y=0 \\
            z^{2} +y+x =0 
          \end{matrix} 
        \right\} \Leftrightarrow 
        \left.
          \begin{matrix}
            x^2+y+z=0 \\
            (y-z)(y+z)-(y-z)=0 \\
            z^{2} +y+x =0 
          \end{matrix} 
        \right\} \Leftrightarrow 
        \left.
          \begin{matrix}
            x^2+y+z=0 \\
            (y-z)(y+z-1)=0 \\
            z^{2} +y+x =0 
          \end{matrix} 
        \right\} \Leftrightarrow 
        \left.
          \begin{matrix}
            x^2+y+z=0 \\
            \textcolor{Col2}{y=z} \quad \text{ή} \quad 
            \textcolor{Col1}{y+z=1} \\
            z^{2} +y+x =0 
          \end{matrix} 
        \right\}
      \]
      Όπου τώρα προκύπτουν δύο συστήματα:
      \[
        \Sigma_1 : \left.
          \begin{matrix}
            x^2+y+z=0 \\
            \textcolor{Col2}{y=z} \\
            z^{2} +y+x =0 
          \end{matrix} 
        \right\} \quad \text{και} \quad 
        \Sigma_2 : \left.
          \begin{matrix}
            x^2+y+z=0 \\
            \textcolor{Col1}{y+z=1} \\
            z^{2} +y+x =0 
          \end{matrix} 
        \right\} \Leftrightarrow 
        \left.
          \begin{matrix}
            x^2+1=0 \\
            y+z=1 \\
            z^{2} +y+x =0 
          \end{matrix} 
          \right\} \quad \text{\begin{tabular}{c} \textcolor{Col1}{αδύνατο,} λόγω \\ 
              της 1ης εξίσωσης
          \end{tabular}}
        \] 
        Λύνουμε το $ \Sigma _1 $. Με αντικατάσταση της 2ης εξίσωσης στην 1η και 3η  
        έχουμε:
        \[
          \left.
            \begin{matrix}
              x^2+2y=0 \\
              z=y \\
              y^{2}+y+x=0
            \end{matrix} 
          \right\} \Leftrightarrow 
          \left.
            \begin{matrix}
              x^{2}+2y=0 \\
              z=y \\
              x=-y^2-y
            \end{matrix} 
          \right\} \Leftrightarrow 
          \left.
            \begin{matrix}
              (-y^2-y)^2+2y=0 \\
              z=y \\
              x=-y^2-y
            \end{matrix} 
          \right\} \Leftrightarrow 
          \left.
            \begin{matrix}
              y^4+2y^3+y^2+2y=0 \\
              z=y \\
              x=-y^2-y
            \end{matrix} 
          \right\} \Leftrightarrow 
        \] 
        \[
          \left.
            \begin{matrix}
              y(y^3+2y^2+y+2)=0 \\
              z=y \\
              x=-y^2-y
            \end{matrix} 
          \right\} \Leftrightarrow 
          \left.
            \begin{matrix}
              y=0 \quad \text{η} \quad y^3+2y^2+y+2=0 \\
              z=y \\
              x=-y^2-y
            \end{matrix} 
          \right\}
        \]
        Οπότε προκύπτουν δύο συστήματα:
        \[
          \Sigma_3 :\left.
            \begin{matrix}
              y^3+2y^2+y+2=0 \\
              z=y \\
              x=-y^2-y
            \end{matrix} 
          \right\}\quad \text{και} \quad 
          \Sigma_4 :\left.
            \begin{matrix}
              y=0 \\
              z=y \\
              x=-y^2-y
            \end{matrix} 
          \right\} \Leftrightarrow 
          \begin{tabular}{c} x=y=z=0 \\ \text{άρα } (0,0,0) \end{tabular}
        \]
        Λύνουμε το $ \Sigma _3 $
        \[
          \left.
            \begin{matrix}
              y^2(y+2) + y+2=0 \\
              z=y \\
              x=-y^2-y
            \end{matrix} 
          \right\} \Leftrightarrow 
          \left.
            \begin{matrix}
              (y+2)(y^2+1)=0 \\
              z=y \\
              x=-y^2-y
            \end{matrix} 
          \right\} \Leftrightarrow 
          \left.
            \begin{matrix}
              y=-2 \quad \text{ή} \quad y^2+1=0 \\
              z=y \\
              x=-y^2-y
            \end{matrix} 
          \right\} \Leftrightarrow 
          \left.
            \begin{matrix}
              y=-2 \\
              z=y \\
              x=-y^2-y
            \end{matrix} 
          \right\} \Leftrightarrow \!\! \begin{tabular}{c} x=y=z=-2 \\ 
          \text{άρα } (-2,-2,-2) \end{tabular}
        \]
        Επομένως η συνάρτηση έχει δύο στάσιμα σημεία. Το $ (0,0,0) $ και το $
        (-2,-2,-2) $.

      \item Βρίσκουμε τις Εσσιανές ορίζουσες.
        \begin{myitemize}
          \item Για το $ (-2,-2,-2) $, έχουμε:
            \begin{gather*}
              \abs{H_3} = \begin{vmatrix*}[r]
                f_{xx} & f_{xy} & f_{xz} \\
                f_{yx} & f_{yy} & f_{yz} \\
                f_{zx} & f_{zy} & f_{zz}
              \end{vmatrix*} = 
              \begin{vmatrix*}[r]
                -12 & 3 & 3 \\
                3 & -12 & 3 \\
                3 & 3 & -12
              \end{vmatrix*} = -1350 < 0 \\[\baselineskip]
              \abs{H_2} = \begin{vmatrix*}[r]
                f_{xx} & f_{xy} \\
                f_{yx} & f_{yy}
              \end{vmatrix*} = 
              \begin{vmatrix*}[r]
                -12 & 3 \\
                3 & -12
              \end{vmatrix*} = 135 > 0 
            \end{gather*}
            \[
              \abs{H_1} = f_{xx} = -12 < 0
            \] 
            Άρα στο σημείο $ (-2,-2,-2) $ η $f$ παρουσιάζει τοπικό μέγιστο.
          \item Για το $ (0,0,0) $, έχουμε:
            \[
              f_{xx} = 0 
            \]
            Οπότε δεν γνωρίζουμε αν η συνάρτηση έχει ακρότατο. 
            Γι᾽ αυτό εφαρμόζουμε τον ορισμό των τοπικών ακροτάτων. 
            Εξετάζουμε το πρόσημο της παραστασης $ f(x,y,z)-f(0,0,0) $. 
            Πράγματι:
            \[
              f(x,y,z)-f(0,0,0) = x^3+y^3+z^3+3xy+3yz+3xz 
            \] 
            Παρατηρούμε ότι
            \begin{myitemize}
              \item Για όλα τα σημεία $ (a,0,0) $ με $ a>0 $ και $ a $ μικρό,
                (δηλαδή για όλα τα σημεία $ (a,0,0) $ του άξονα $x$,  κοντά στο 
                $(0,0,0)$), έχουμε ότι 
                \[
                  f(x,y,z)-f(0,0,0)=a^{3}>0 
                \] 
              \item Για όλα τα σημεία $ (0,b,0) $ με $ b<0 $ και $ b $ μικρό,
                (δηλαδή για όλα τα σημεία $ (0,b,0) $ του άξονα $y$,  κοντά στο 
                $(0,0,0)$), έχουμε ότι 
                \[
                  f(x,y,z)-f(0,0,0)=b^{3}<0 
                \] 
                Επομένως η παράσταση $ f(x,y,z)-f(0,0,0) $ \textbf{δεν} διατηρεί 
                σταθερό πρόσημο σε κάθε σημείο κοντά στο $ (0,0,0) $ και γι᾽ αυτό η 
                συνάρτηση δεν παρουσιάζει σε αυτό τοπικό ακρότατο, αλλά
                \textbf{σαγματικό} σημείο.
            \end{myitemize}
        \end{myitemize}
    \end{enumerate}
  \end{solution}



  \end{document}
