\documentclass[a4paper,12pt]{article}
\usepackage{etex}
%%%%%%%%%%%%%%%%%%%%%%%%%%%%%%%%%%%%%%
% Babel language package
\usepackage[english,greek]{babel}
% Inputenc font encoding
\usepackage[utf8]{inputenc}
%%%%%%%%%%%%%%%%%%%%%%%%%%%%%%%%%%%%%%

%%%%% math packages %%%%%%%%%%%%%%%%%%
\usepackage{amsmath}
\usepackage{amssymb}
\usepackage{amsfonts}
\usepackage{amsthm}
\usepackage{proof}

\usepackage{physics}

%%%%%%% symbols packages %%%%%%%%%%%%%%
\usepackage{bm} %for use \bm instead \boldsymbol in math mode 
\usepackage{dsfont}
\usepackage{stmaryrd}
%%%%%%%%%%%%%%%%%%%%%%%%%%%%%%%%%%%%%%%


%%%%%% graphicx %%%%%%%%%%%%%%%%%%%%%%%
\usepackage{graphicx}
\usepackage{color}
%\usepackage{xypic}
\usepackage[all]{xy}
\usepackage{calc}
\usepackage{booktabs}
\usepackage{minibox}
%%%%%%%%%%%%%%%%%%%%%%%%%%%%%%%%%%%%%%%

\usepackage{enumerate}

\usepackage{fancyhdr}
%%%%% header and footer rule %%%%%%%%%
\setlength{\headheight}{14pt}
\renewcommand{\headrulewidth}{0pt}
\renewcommand{\footrulewidth}{0pt}
\fancypagestyle{plain}{\fancyhf{}
\fancyhead{}
\lfoot{}
\rfoot{\small \thepage}}
\fancypagestyle{vangelis}{\fancyhf{}
\rhead{\small \leftmark}
\lhead{\small }
\lfoot{}
\rfoot{\small \thepage}}
%%%%%%%%%%%%%%%%%%%%%%%%%%%%%%%%%%%%%%%

\usepackage{hyperref}
\usepackage{url}
%%%%%%% hyperref settings %%%%%%%%%%%%
\hypersetup{pdfpagemode=UseOutlines,hidelinks,
bookmarksopen=true,
pdfdisplaydoctitle=true,
pdfstartview=Fit,
unicode=true,
pdfpagelayout=OneColumn,
}
%%%%%%%%%%%%%%%%%%%%%%%%%%%%%%%%%%%%%%

\usepackage[space]{grffile}

\usepackage{geometry}
\geometry{left=25.63mm,right=25.63mm,top=36.25mm,bottom=36.25mm,footskip=24.16mm,headsep=24.16mm}

%\usepackage[explicit]{titlesec}
%%%%%% titlesec settings %%%%%%%%%%%%%
%\titleformat{\chapter}[block]{\LARGE\sc\bfseries}{\thechapter.}{1ex}{#1}
%\titlespacing*{\chapter}{0cm}{0cm}{36pt}[0ex]
%\titleformat{\section}[block]{\Large\bfseries}{\thesection.}{1ex}{#1}
%\titlespacing*{\section}{0cm}{34.56pt}{17.28pt}[0ex]
%\titleformat{\subsection}[block]{\large\bfseries{\thesubsection.}{1ex}{#1}
%\titlespacing*{\subsection}{0pt}{28.80pt}{14.40pt}[0ex]
%%%%%%%%%%%%%%%%%%%%%%%%%%%%%%%%%%%%%%

%%%%%%%%% My Theorems %%%%%%%%%%%%%%%%%%
\newtheorem{thm}{Θεώρημα}[section]
\newtheorem{cor}[thm]{Πόρισμα}
\newtheorem{lem}[thm]{λήμμα}
\theoremstyle{definition}
\newtheorem{dfn}{Ορισμός}[section]
\newtheorem{dfns}[dfn]{Ορισμοί}
\theoremstyle{remark}
\newtheorem{remark}{Παρατήρηση}[section]
\newtheorem{remarks}[remark]{Παρατηρήσεις}
%%%%%%%%%%%%%%%%%%%%%%%%%%%%%%%%%%%%%%%




\input{definitions_ask.tex}

\pagestyle{askhseis}

\renewcommand{\vec}{\mathbf}

\begin{document}

\begin{center}
  \minibox{\large \bfseries \textcolor{Col1}{Ασκήσεις στα Ακρότατα Με Περιορισμό}}
\end{center}

\vspace{\baselineskip}

\section*{Ασκήσεις}

\begin{enumerate}
  \item Να μελετηθούν τα ακρότατα της συνάρτησης $ f(x,y) = xy $ με περιορισμό 
    $ x+y=1 $. \hfill Απ: $ f_{\max}(1/2,1/2) $ 
    %pauln notes
  \item Να μελετηθούν τα ακρότατα της συνάρτησης $ f(x,y) = 5x-3y $ με περιορισμό 
    $x^{2}+y^{2}=136$. 

    \hfill Απ: $ f_{\max}(10,-6), \; f_{\min}(-10,6) $ 

  \item Να μελετηθούν τα ακρότατα της συνάρτησης $ f(x,y,z) = xyz $ με περιορισμό 
    $x^{2}+y^{2}+z^{2}=1 $, όπου $ x,y,z>0 $. 

    \hfill Απ: $ f_{\max}(\sqrt{3} /3,\sqrt{3} /3,\sqrt{3} /3) $ 

    %pauln notes
  \item Να μελετηθούν τα ακρότατα της συνάρτησης $ f(x,y,z) = xyz $ με περιορισμό 
    $x+y+z=1 $, όπου $ x,y,z \geq 0 $. 

    \hfill Απ: $ f_{\max}(1/3,1/3,1/3), f_{\min}(0,0,1), f_{min}(0,1,0), 
    f_{min}(1,0,0) $ 

  \item Να μελετηθούν τα ακρότατα της συνάρτησης $ f(x,y,z) = 4y-2z $ με περιορισμούς 
    $2x-y-z=2$, και $x^{2}+y^{2}=1$. 

    \hfill Απ: $ f_{\max}(-2/ \sqrt{13} , 3 / \sqrt{13}), 
    f_{\min}(2 / \sqrt{13}, - 3 / \sqrt{13})$ 
\end{enumerate}


\section*{Προβλήματα}

\begin{enumerate}
  \item Έστω ότι μια επιχείρηση έχει συνάρτηση ολικών κερδών
    \[
      \pi (x,y) = -4x^{2}-5y^{2}+20xy
     \] 
     όπου $ x $ και $ y $ είναι οι παραγόμενες και πωλούμενες ποσότητες των προιόντων 
     $ X $ και $ Y $. Για την παραγωγή κάθε μονάδας προιόντος $X$ απαιτούνται 2 μονάδες 
     ενός συντελεστή παραγωγής και για την παραγωγή κάθε μονάδας προιόντος $Y$ 
     απαιτούνται 5 μονάδες του συντελεστή αυτού που διατίθεται σε 40 μονάδες. Να 
     υπολογίσετε το μέγιστο κέρδος, υπό την παραπάνω συνθήκη.

\item Το συνολικό κόστος παραγωγής $x$ μονάδων ενός προιόντος $A$ και $y$ μονάδων ενός 
  προιόντος $B$ δίνεται από τη σχέση $ TC = 22x^{2} + 8 y^{2} - 5xy $. Αν η επιχείρηση 
  δεσμεύεται να παράγει 20 μονάδες συνολικά, γράψτε τον περιορισμό που συνδέει το 
  $x$ και το $y$ και στη συνέχεια υπολογίστε τον αριθμό κάθε είδους προιόντος που 
  πρέπει να παραχθεί ώστε να ελαχιστοποιηθεί το κόστος.

  \hfill Απ: $ x+y=20, \; x=6, \; y=14 $

\item Η συνάρτηση παραγωγής μιας επιχείρησης δίνεται από τη σχέση $ Q=50KL $. Το 
  κόστος για μια μονάδα κεφαλαίου και εργασίας είναι $ 2 $ ευρώ και $ 3 $ ευρώ 
  αντίστοιχα. Υπολογίστε τις τιμές των $ K $ και $ L $ οι οποίες ελαχιστοποιούν 
  το συνολικό κόστος εισροών αν ο στόχος παραγωγής είναι $1200$.

\hfill Απ: $K=6, \; L=4$  

\item Υπολογίστε τη μέγιστη τιμή της συνάρτησης χρησιμότητας $ u(x_{1}, x_{2}) = x_{1}
  x_{2} $ που υπόκειται στον εισοδηματικό περιορισμό $ x_{1}+4 x_{2}=360 $.

  \hfill Απ: $ 8100 $

\item Υπολογίστε τις τιμές των $ x_{1} $ και $ x_{2} $, που μεγιστοποιούν τη 
  συνάρτηση χρησιμότητας $ u(x_{1}, x_{2}) = \ln{x_{1}} + 2 \ln{x_{2}} $, η οποία  
  υπόκειται στον εισοδηματικό περιορισμό $ 2x_{1}+3 x_{2}=18 $.

  \hfill Απ: $x1 = 3, \; x_{2}=4$

\item Μια επιχείρηση παράγει δύο αγαθά, το $A$ και το $B$. Το εβδομαδιαίο κόστος
  παραγωγής $x$ μονάδων $A$ και $y$ μονάδων $B$ είναι $ TC=0.2 x^{2}+0.05y^{2} +0.1xy
  +2x+5y+1000 $.
  \begin{enumerate}[i)]
    \item Υπολογίστε την ελάχιστη τιμή του $ TC $ στην περίπτωση που δεν υπάρχουν 
      περιορισμοί.
    \item Υπολογίστε στην ελάχιστη τιμή του $ TC $ όταν η επιχείρηση δεσμεύεται να 
      παράγει συνολικά 500 αγαθά και από τα 2 είδη.

      \hfill Απ: $1000, \; 15985$
  \end{enumerate}

\item Η συνάρτηση παραγωγής μιας επιχείρησης δίνεται από τη σχέση $ Q=10K^{1/2}L^{1/4}
  $. Το κόστος ανά μονάδα κεφαλαίου και εργασίας είναι 4 ευρώ και 5 ευρώ αντίστοιχα 
  και η επιχείρηση ξοδεύει συνολικά 60 ευρώ σε αυτές τις εισροές. Βρείτε τις τιμές 
  των $K$ και $L$ που μεγιστοποιούν την παραγωγή.

  \hfill Απ: $ K=10, \; L=4 $ 

\item Η συνάρτηση παραγωγής μιας επιχείρησης δίνεται από τη σχέση 
  $ Q=2L^{1/2}+3K^{1/2} $, όπου τα $ Q, L $ και $K$ συμβολίζουν τον αριθμό των 
  μονάδων των παραγόμενων προιόντων, της εργασίας και του κεφαλαίου αντίστοιχα. 
Το κόστος εργασίας είναι 2 ευρώ ανά μονάδα και του κεφαλαίου 1 ευϱω ανά μονάδα, ενώ 
τα παραγόμενα προιόντα πωλούνται 8 ευρώ ανά μονάδα. Αν η επιχείρηση είναι πρόθυμη να 
ξοδύψε 99 ευρώ στο κόστος εισροών, να υπολογίσετε το μέγιστο κέρδος και τις τιμές 
των $K$ και $L$, για τις οποίες αυτό επιτυγχάνεται.

  \hfill Απ: $ K=81, \; L=9, \; \Pi = 165 $ 
\end{enumerate} 


\begin{enumerate}
  \item Να προσδιοριστούν τα ακρότατα της συνάρτησης $ f(x,y,z) = x+y+z $, υπό 
    τους περιορισμούς $ x^{2}+y^{2}=2 $ και $ x+z=1 $.
\end{enumerate}

\hfill Απ: $ f_{max}(0, \sqrt{2} , 1), \; f_{\min}(0, - \sqrt{2} , 1) $ 



\end{document}
