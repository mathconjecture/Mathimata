\input{preamble_ask.tex}
\input{definitions_ask.tex}


\geometry{top=2cm}
\pagestyle{askhseis}
\everymath{\displaystyle}


\begin{document}

\begin{center}
  \minibox{\large\bf \textcolor{Col1}{Ασκήσεις στις ευθείες και τα επίπεδα}}
\end{center}

% \vspace{\baselineskip}

\section*{Ευθείες και Επίπεδα στο χώρο}


\begin{enumerate}
  \item Να βρείτε τις παραμετρικές εξισώσεις της ευθείας, όταν:
    \begin{enumerate}[i)]
      \item διέρχεται από το σημείο $ P(3,-4,-1) $ και είναι παράλληλη στο διάνυσμα 
        $ \mathbf{u} = \mathbf{i}+ \mathbf{j}+ \mathbf{k} $. 

        \hfill Απ: $ x=3+t, \; y=-4+t, \; z=-1+t $

      \item διέρχεται από τα σημεία $ P(1,2,-1) $ και $ Q(-1,0,1) $.
        \hfill Απ: $ x=1-2t, \; y=2-2t, \; z=-1+2t $ 

      % \item διέρχεται από τα σημεία $ P(1,2,0) $ και $ Q(1,1,-1) $.
      %   \hfill Απ: $ x=1, \; y=2-t, \; z=-t $ 

      \item διέρχεται από την αρχή των αξόνων και είναι παράλληλη στο διάνυσμα 
        $ \mathbf{u} = 2 \mathbf{j}+ \mathbf{k} $.

        \hfill Απ: $ x=0, \; y=2t, \; z=t $ 

      \item διέρχεται από το σημείο $ P(3,-2,1) $ και είναι παράλληλη προς την ευθεία 
        $ x=1+2t, \; y=2-t, \; z=3t $.

        \hfill Απ: $ x=3+2t, \; y=-2-t, \; z=1+3t $ 

      \item διέρχεται από το σημείο $ P(2,4,5) $ και είναι κάθετη στο επίπεδο 
        $ 3x+7y-5z=21 $.

        \hfill Απ: $ x=2+3t, \; y=4+7t, \; z=5-5t $ 

      \item διέρχεται από το σημείο $ P(2,3,0) $ και είναι κάθετη στα διανύσματα 
        $ \mathbf{u} = \mathbf{i}+2 \mathbf{j}+3 \mathbf{k} $ και $ \mathbf{v} = 3
        \mathbf{i}+ 4 \mathbf{j}+5 \mathbf{k} $.

        \hfill Απ: $ x=2-2t, \; y=3+4t, \; z=-2t $  
    \end{enumerate}

  \item Να βρείτε τις παραμετρικές εξισώσεις του ευθύγραμμου τμήματος, που ενώνει 
    τα σημεία $A$ και $B$:
    \begin{enumerate}[i)]
      \item $ A(0,0,0), \; B(1,0,2) $ \hfill Απ: $ x=t, \; y=0, \; z=2t, \quad t \in
        [0,1] $  
      \item $ A(0,1,1), \; B(0,-1,1) $ \hfill Απ: $ x=0, \; y=1-2t, \; z=1, \quad t \in
        [0,1] $ 
      \item $ A(2,0,2), \; B(0,2,0) $ \hfill Απ: $ x=2-2t, \; y=2t, \; z=2-2t, 
        \quad t \in [0,1] $ 
      % \item $ A(1,0,-1), \; B(0,3,0) $ \hfill Απ: $ x=1-t, \; y=3t, \; z=-1+t, 
      %   \quad t \in [0,1] $ 
    \end{enumerate}

  \item Να βρείτε την εξίσωση του επιπέδου, όταν:
    \begin{enumerate}[i)]
      \item διέρχεται από το σημείο $ p(0,2,-1) $ και είναι κάθετο στο διάνυσμα 
        $ \vec{n} = 3 \mathbf{i}-2 \mathbf{j}- \mathbf{k} $. 
        \hfill Απ: $ 3x-2y-z=-3 $ 

      \item διέρχεται από το σημείο $ P(1,-1,3) $ και είναι παράλληλο προς το επίπεδο 
        $ 3x+y+z=7 $.

        \hfill Απ: $ 3x+y+z=5 $  

      \item διέρχεται από τα σημεία $ P(1,1,-1), \; Q(2,0,2) $ και $ R(0,-2,1) $.
        \hfill Απ: $ 7x-5y-4z=6 $ 

      \item διέρχεται από το σημείο $ P(2,4,5) $ και είναι κάθετο στην ευθεία με
        παραμετρικές εξισώσεις $ x=5+t, \; y=1+3t, \; z=4t $.
        \hfill Απ: $ x+3y+4z=34 $ 

      \item διέρχεται από το σημείο $ P(1,-2,1) $ και είναι κάθετο στο διάνυσμα θέσης 
        του σημείου $ P $.
        \hfill Απ: $ x-2y+z=6 $ 
    \end{enumerate}

  \item Να βρείτε το σημείο τομής των ευθειών, $ \varepsilon _{1}: x=2t+1, \; y=3t+2, \;
    z=4t+3$ και $ \varepsilon _{2}: x=s+2, \; y=2s+4, \; z=-4s-1 $, και στη συνέχεια να 
    βρείτε την εξίσωση του επιπέδου, που ορίζουν αυτές οι δύο ευθείες.

    \hfill Απ: $ -20x+12y+z=7 $. 
    
  % \item Να βρείτε τη γωνία που σχηματίζουν τα επίπεδα:
  %   \begin{enumerate}[i)]
  %     \item $ x+y=1, \; 2x+y-2z=2 $ \hfill Απ: $ \pi /4 $ 
  %     \item $5x+y-z=10, \; x-2y+3z=-1$ \hfill Απ: $ \pi /2 $ 
  %   \end{enumerate}
\end{enumerate}


\section*{Εφαπτόμενες ευθείες και επίπεδα}

\begin{enumerate}
  %Thomas 14.5 ex. 25-28 p.791
\item Να βρείτε την εξίσωση της εφαπτομένης των παρακάτω ισοσταθμικών καμπυλών, στο
  δοσμένο σημείο.
  \begin{enumerate}[i)]
    \item $ x^{2}+y^{2} = 4, \quad (\sqrt{2} , \sqrt{2}) $ 
      \hfill Απ: $ \sqrt{2} x + \sqrt{2} y = 4 $
    \item $ x^{2}-y=1, \quad (\sqrt{2} ,1) $ \hfill Απ: $ y=2 \sqrt{2} x - 3 $
    \item $ xy=-4, \quad (2.-2) $ \hfill Απ: $ y=x-4 $
    % \item $ x^{2}-xy+y^{2}=7, \quad (-1,2) $ \hfill Απ: $ -4x+5y=14 $
  \end{enumerate}

  %Thomas 14.6 ex. 1-6 p.791
\item Να βρείτε την εξίσωση του εφαπτόμενου επιπέδου και της κάθετης ευθείας, των 
  παρακάτω ισοσταθμικών επιφανειών, στο δοσμένο σημείο.
  \begin{enumerate}[i)]
    \item $ x^{2}+y^{2}+z^{2}=3, \quad P_{0}(1,1,1) $ 
      \hfill Απ: $ \Pi :  x+y+z=3, \; \kappa : x=1+2t,y=1+2t,z=1+2t $
    \item $ x^{2}+y^{2}-z^{2}=18, \quad P_{0}(3,5,-4) $
      \hfill Απ: $ \Pi : 3x+5y+4z=18, \; \kappa : x=3+6t,y=5+10t,z=-4+8t $
    \item $ 2z-x^{2}=0, \quad P_{0}(2,0,2) $ 
      \hfill Απ: $ \Pi : -2x+z+2=0, \; \kappa : x=2-4t,y=0,z=2+2t $
    % \item $ x^{2}-xy-y^{2}-z=0, \quad P_{0}(1,1,-1) $
    %   \hfill Απ: $ \Pi : x-3y-z=-1, \; \kappa x=1+t,y=1-3t,z=-1-t$
  \end{enumerate}

  \enlargethispage{3\baselineskip}

\item Να βρείτε τις παραμετρικές εξισώσεις της εφαπτόμενης ευθείας της καμπύλης, στο 
  δοσμένο σημείο, όταν ευθεία δίνεται ως τομή δύο επιφανειών.
  \begin{enumerate}[i)]
    \item Επιφάνειες: $ x+y^{2}+2z = 4 $, $ x=1 $ στο σημείο $ P(1,1,1) $.
      \hfill Απ: $ \varepsilon : x=1,\;y=1+2t,\;z=1-2t $ 
    % \item Επιφάνειες: $xyz=1$, $x^{2}+2y^{2}+3z^{2}=6$ στο σημείο $ P(1,1,1) $.
    %   \hfill Απ: $ \varepsilon : x=1+2t,\;y=1-4t,\;z=1+2t $ 
    \item Επιφάνειες: $x^{2}+y^{2}=4$, $x^{2}+y^{2}-z=0$ στο σημείο $ P(\sqrt{2} ,
      \sqrt{2} , 4) $.

      \hfill Απ: $ \varepsilon : x= \sqrt{2} -2 \sqrt{2}t,\;y= \sqrt{2} +2 \sqrt{2}t,
      \;z=4$ 
  \end{enumerate}
\end{enumerate}



\end{document}

