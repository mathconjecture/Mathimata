\documentclass[a4paper,12pt]{article}
\usepackage{etex}
%%%%%%%%%%%%%%%%%%%%%%%%%%%%%%%%%%%%%%
% Babel language package
\usepackage[english,greek]{babel}
% Inputenc font encoding
\usepackage[utf8]{inputenc}
%%%%%%%%%%%%%%%%%%%%%%%%%%%%%%%%%%%%%%

%%%%% math packages %%%%%%%%%%%%%%%%%%
\usepackage{amsmath}
\usepackage{amssymb}
\usepackage{amsfonts}
\usepackage{amsthm}
\usepackage{proof}

\usepackage{physics}

%%%%%%% symbols packages %%%%%%%%%%%%%%
\usepackage{bm} %for use \bm instead \boldsymbol in math mode 
\usepackage{dsfont}
\usepackage{stmaryrd}
%%%%%%%%%%%%%%%%%%%%%%%%%%%%%%%%%%%%%%%


%%%%%% graphicx %%%%%%%%%%%%%%%%%%%%%%%
\usepackage{graphicx}
\usepackage{color}
%\usepackage{xypic}
\usepackage[all]{xy}
\usepackage{calc}
\usepackage{booktabs}
\usepackage{minibox}
%%%%%%%%%%%%%%%%%%%%%%%%%%%%%%%%%%%%%%%

\usepackage{enumerate}

\usepackage{fancyhdr}
%%%%% header and footer rule %%%%%%%%%
\setlength{\headheight}{14pt}
\renewcommand{\headrulewidth}{0pt}
\renewcommand{\footrulewidth}{0pt}
\fancypagestyle{plain}{\fancyhf{}
\fancyhead{}
\lfoot{}
\rfoot{\small \thepage}}
\fancypagestyle{vangelis}{\fancyhf{}
\rhead{\small \leftmark}
\lhead{\small }
\lfoot{}
\rfoot{\small \thepage}}
%%%%%%%%%%%%%%%%%%%%%%%%%%%%%%%%%%%%%%%

\usepackage{hyperref}
\usepackage{url}
%%%%%%% hyperref settings %%%%%%%%%%%%
\hypersetup{pdfpagemode=UseOutlines,hidelinks,
bookmarksopen=true,
pdfdisplaydoctitle=true,
pdfstartview=Fit,
unicode=true,
pdfpagelayout=OneColumn,
}
%%%%%%%%%%%%%%%%%%%%%%%%%%%%%%%%%%%%%%

\usepackage[space]{grffile}

\usepackage{geometry}
\geometry{left=25.63mm,right=25.63mm,top=36.25mm,bottom=36.25mm,footskip=24.16mm,headsep=24.16mm}

%\usepackage[explicit]{titlesec}
%%%%%% titlesec settings %%%%%%%%%%%%%
%\titleformat{\chapter}[block]{\LARGE\sc\bfseries}{\thechapter.}{1ex}{#1}
%\titlespacing*{\chapter}{0cm}{0cm}{36pt}[0ex]
%\titleformat{\section}[block]{\Large\bfseries}{\thesection.}{1ex}{#1}
%\titlespacing*{\section}{0cm}{34.56pt}{17.28pt}[0ex]
%\titleformat{\subsection}[block]{\large\bfseries{\thesubsection.}{1ex}{#1}
%\titlespacing*{\subsection}{0pt}{28.80pt}{14.40pt}[0ex]
%%%%%%%%%%%%%%%%%%%%%%%%%%%%%%%%%%%%%%

%%%%%%%%% My Theorems %%%%%%%%%%%%%%%%%%
\newtheorem{thm}{Θεώρημα}[section]
\newtheorem{cor}[thm]{Πόρισμα}
\newtheorem{lem}[thm]{λήμμα}
\theoremstyle{definition}
\newtheorem{dfn}{Ορισμός}[section]
\newtheorem{dfns}[dfn]{Ορισμοί}
\theoremstyle{remark}
\newtheorem{remark}{Παρατήρηση}[section]
\newtheorem{remarks}[remark]{Παρατηρήσεις}
%%%%%%%%%%%%%%%%%%%%%%%%%%%%%%%%%%%%%%%




\newcommand{\vect}[2]{(#1_1,\ldots, #1_#2)}
%%%%%%% nesting newcommands $$$$$$$$$$$$$$$$$$$
\newcommand{\function}[1]{\newcommand{\nvec}[2]{#1(##1_1,\ldots, ##1_##2)}}

\newcommand{\linode}[2]{#1_n(x)#2^{(n)}+#1_{n-1}(x)#2^{(n-1)}+\cdots +#1_0(x)#2=g(x)}

\newcommand{\vecoffun}[3]{#1_0(#2),\ldots ,#1_#3(#2)}

\newcommand{\mysum}[1]{\sum_{n=#1}^{\infty}



\pagestyle{vangelis}



\begin{document}

\chapter{Γραμμική Ανεξαρτησία}

\begin{dfn}
  Έστω $ V $ ένας διανυσματικός χώρος επί του $ \mathbb{R} $ και έστω 
  $ \mathbf{v} \in V $. Λέμε ότι το διάνυσμα $ \mathbf{v}$ είναι 
  \textcolor{Col2}{γραμμικός συνδυασμός} των διανυσμάτων 
  $ \mathbf{u_{1}}, \mathbf{u_{2}}, \ldots \mathbf{u}_{n} $, αν υπάρχουν 
  $ \lambda _{1}, \lambda _{2}, \ldots, \lambda _{n} \in \mathbb{R} $ τέτοιοι ώστε 
  \[
    \mathbf{v} = \lambda _{1} \mathbf{u_{1}}+ \lambda_{2} \mathbf{u_{2}}+ 
    \cdots \lambda _{k} \mathbf{u}_{k} \Leftrightarrow \mathbf{v} = 
    \sum_{i=1}^{k} \lambda _{i} \mathbf{u}_{i} 
  \]
  Τα στοιχεία $ \lambda _{1}, \lambda _{2}, \ldots, \lambda _{k} $ ονομάζονται 
  \textcolor{Col2}{συντελεστές} τους γραμμικού συνδυασμόυ.
\end{dfn}

\begin{dfn}
  Έστω $ V $ ένας διανυσματικός χώρος και έστω 
  $ S = \{ \mathbf{v_{1}}, \ldots, \mathbf{v}_{n} \} \subseteq V $. Λέμε ότι 
  τα διανύσματα $ \mathbf{v_{1}}, \ldots, \mathbf{v_{n}} $ είναι 
  \textcolor{Col2}{γραμμικώς ανεξάρτητα} ή ότι το σύνολο $ S $ είναι 
  \textcolor{Col2}{γραμμικώς ανεξάρτητο}, αν οποτεδήποτε έχουμε
  \[
    \comb{v}{k} = \mathbf{0} \; \text{τότε} \; \lambda _{1} = 
    \lambda _{2} = \cdots = \lambda _{n} = 0
  \]
  Αν τα διανύσματα $ \mathbf{v_{1}}, \ldots, \mathbf{v_{n}} $ δεν είναι γραμμικώς 
  ανεξάρτητα τότε λέγονται \textcolor{Col2}{γραμμικώς εξαρτημένα} και ισχύει 
  ότι υπάρχει μη τετριμμένος γραμμικός συνδυασμός στοιχείων του $S$ που είναι 
  ίσος με $ \mathbf{0} $, δηλαδή
  \[
    \exists  \lambda _{1}, \ldots, \lambda _{k} \in \mathbb{K} \; 
    \text{όχι όλα μηδέν, ώστε} \; \comb{u}{k} = \mathbf{0}
  \]
\end{dfn}

\begin{examples}
\item {}
  \begin{enumerate}
    \item Στον $ \mathbb{R}^{2} $ δύο διανύσματα $ \mathbf{u} $ και $ \mathbf{v} $ 
      είναι γραμμικώς ανεξάρτητα αν και μόνον αν δεν είναι παράλληλα. Πράγματι 
      έστω ότι $ \mathbf{u}, \mathbf{v} $ είναι γραμμικώς εξαρτημένα. Τότε 
      \[
        \exists \lambda _{1}, \lambda _{2} \in \mathbb{R} \; 
        \text{όχι και τα δύο μηδέν, ώστε} 
        \; \lambda_{1} \mathbf{u} + \lambda_{2} \mathbf{v} = \mathbf{0} 
      \]
      Έτσι, αν έστω ότι $ \lambda_{1} \neq 0 $, έχουμε 
      \[
        \mathbf{u} = - \frac{\lambda _{2}}{\lambda _{1}} \mathbf{v} \quad  
        \text{(παράλληλα)}
      \] 

    \item Στον $ \mathbb{R}^{3} $ δύο διανύσματα είναι γραμμικώς ανεξάρτητα αν και 
      μόνον αν δεν είναι παράλληλα και τρία διανύσματα είναι γραμμικώς ανεξάρτητα αν 
      και μόνον αν δεν είναι συνεπίπεδα.
  \end{enumerate}
\end{examples}


\begin{prop}
  Κάθε μη-μηδενικό διάνυσμα ένος $ \mathbb{K} $- χώρου $V$ είναι γραμμικώς 
  ανεξάρτητο.
\end{prop}
\begin{proof}
  Πράγματι. Έστω $ \mathbf{u} \in V $ και $ \lambda \in \mathbb{K} $ με 
  $ \lambda \mathbf{u} = \mathbf{0} $. Τότε έχουμε $ \lambda = 0 $, αφού 
  $ \mathbf{u} \neq \mathbf{0} $.
\end{proof}

\begin{prop}
  Έστω $V$ ένας $ \mathbb{K} $- χώρος και 
  $ S = \{ \mathbf{v_{1}}, \mathbf{v_{2}}, \ldots, \mathbf{v_{n}}  \} \subseteq V $.
  Αν $ \mathbf{0} \in S $ τότε το $S$ είναι γραμμικώς εξαρτημένο. Συγκεκριμένα 
  $ \{ \mathbf{0} \} $ είναι γραμμικώς εξαρτημένο.
\end{prop}
\begin{proof}
  Πράγματι, έστω ότι $ \mathbf{u}_{i} = \mathbf{0} $, για κάποιο $ i $ με 
  $ 1 \leq i \leq k $. Τότε
  \[
    0 \mathbf{u_{1}}+ \cdots + 0 \mathbf{u}_{i-1} + 1 \mathbf{u}_{i} + 0 
    \mathbf{u}_{i+1} + \cdots + 0 \mathbf{u}_{k} = \mathbf{0}  
  \]
  είναι ένας γραμμικός συνδυασμός στοιχειών του $S$ που είναι $ \mathbf{0} $, 
  χωρίς να είναι μηδέν όλοι οι συντελεστές.
\end{proof}

\begin{exercise}
  Να εξετάσετε αν τα παρακάτω διανύσματα είναι γραμμικώς ανεξάρτητα.
\end{exercise}


\end{document}
