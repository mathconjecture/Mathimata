\documentclass[a4paper,table]{report}
\input{preamble_ask.tex}
\input{definitions_ask.tex}


\geometry{top=2.5cm}
\pagestyle{askhseis}
\everymath{\displaystyle}



\begin{document}

\begin{center}
  \minibox{\large\bf \textcolor{Col1}{Ασκήσεις Διπλό Ολοκλήρωμα}}
\end{center}


\section*{Πολικές Συντεταγμένες}

\begin{enumerate}
  \item Να υπολογίσετε τα παρακάτω διπλά ολοκληρώματα κάνοντας χρήση των πολικών 
    και ελλειπτικών συντεταγμένων.
    \begin{enumerate}[i)]
      \item $I=\iint_{D}e^{x^2+y^2}\,dxdy$, 
        \quad $D=\{(x,y)\in\mathbb{R}^2 \mid x^2+y^2\leq a^2,\; a>0\}$ 
        \hfill Απ: $\pi(e^{a^2}-1)$ %spandagos p.73 ex.1
      \item $I=\iint_{D}\sin(x^2+y^2)\,dxdy$, \quad $D= \{(x,y)\in \mathbb{R}^{2} 
        \mid x^{2}+y^{2} \leq a^{2} \}$ 
        \hfill Απ: $\pi (1 - \cos{a^{2}}) $
      \item $I=\iint_{D}(x+y)\,dxdy$, \quad $ D= \{(x,y)\in \mathbb{R}^{2} 
        \mid x^{2}+y^{2} \leq 1, x \geq 0, y \geq 0 \} $ 
        \hfill Απ: $2/3$
      \item $I=\iint_{D}\sqrt{x^2+y^2}\,dxdy$, \quad $ D= \{(x,y)\in \mathbb{R}^{2} 
        \mid 4 \leq x^{2}+y^{2} \leq 9 \} $ 
        \hfill Απ: $38 \pi /3$
      \item $I=\iint_{D}x\,dxdy$, \quad $ D= \{(x,y)\in \mathbb{R}^{2} \mid x^{2}+y^{2} 
        \leq 4,\; x^{2}+y^{2} \geq 1,\; x \geq 0,\; y \geq 0 \} $ 
        \hfill Απ: $ \frac{7}{3} $ %spandagos p.73 ex.2
      \item $I=\iint_{D}(x^2+y^2)\,dxdy$, \quad $ D= \{(x,y)\in \mathbb{R}^{2} 
        \mid (x-a)^{2}+y^{2} \leq a^{2},\; a>0 \}  $ 
        \hfill Απ: $\frac{3}{2}a^4\pi$
      \item $I=\int_{0}^{1} \int _{0}^{\sqrt{1-x^{2}}} (x^2+y^2)\,dydx$ 
        \hfill Απ: $ \frac{\pi}{8} $ 
    \end{enumerate}


    \section*{Εφαρμογές του διπλού ολοκληρώματος} 

    %thomas 15.2 ex.58
  \item Να υπολογιστεί ο \textbf{όγκος} του στερεού κάτω από την επιφάνεια
    $ z= 9- x^{2} -y^{2} $ και πάνω από την περιοχή $T$ του επιπέδου που 
    περικλείεται από τον κύκλο $ x^{2}+y^{2}=1 $.
    \hfill Απ: $ 17 \pi /2 $ 

  \item Να υπολογιστεί ο \textbf{όγκος} του στερεού που περικλείεται από το 
    παραβολοειδές με εξίσωση $ x^{2}+y^{2}+z=4 $ και από το επίπεδο $ z=0 $.
    \hfill Απ: $ 8\pi $ 

  \item Να υπολογιστεί ο \textbf{όγκος} του στερεού που περικλείεται από τις 
    επιφάνειες $ x^{2}+y^{2}=1 $ και $ y+z=2 $.
    \hfill Απ: $ 2 \pi $ 

  \item Να υπολογιστεί ο \textbf{όγκος} του στερεού που περικλείεται από τις 
    επιφάνειες $ z=x^{2}+y^{2} $ και $ x^{2}+y^{2}=2x $.
    \hfill Απ: $ \frac{3 \pi}{2} $ 

  \item \label{ask:area} 
    \textbf{(Κιουτσιούκης)} Να υπολογιστεί το εμβαδό της επίπεδης περιοχής που 
    περικλείεται από τον κύκλο $ x^{2}+y^{2} = 16 $ και την έλλειψη 
    $ \frac{x^{2}}{5^{2}} + \frac{y^{2}}{4^{2}} =1$. 
    \hfill Απ: $ 4 \pi $ 

  \item \textbf{(Κιουτσιούκης)} Να υπολογιστεί ο 
    \textbf{όγκος} του ημισφαιρίου $ x^{2}+y^{2}+z^{2}=\rho^{2}, \; z \geq 0 $.
    \hfill Απ: $ \frac{2}{3} \pi \rho^{3} $ 

  \item \textbf{(Κιουτσιούκης)} Να υπολογιστεί το εμβαδό της περιοχής που περικλείεται 
    από τον κύκλο $ x^{2}+y^{2}=4 $ και βρίσκεται πάνω από την ευθεία $ y=1 $ και κάτω 
    από την ευθεία $ y= \sqrt{3} x $.
    \hfill Απ: $ \frac{\pi - \sqrt{3}}{3} $ 
\end{enumerate}



\section*{Ασκήσεις στα Υλικά Χωρία}


\begin{enumerate}
  \item Να βρεθεί το \textbf{κέντρο μάζας} του υλικού χωρίου $D$ που ορίζεται από τις 
    καμπύλες $y^{2}=x$ και $y=x^{3}$, όταν η συνάρτηση πυκνότητας μάζας, είναι 
    $\delta(x,y)=y$
  \hfill Απ: $\overline{x}={7}/{12}$, $\overline{y}={14}/{25}$

\item Να βρεθούν οι συντεταγμένες του \textbf{κέντρου βάρους} του ομογενούς 
  $(\delta(x,y)=1)$ χωρίου που περικλείεται από τις καμπύλες $y^{2}=2x$, $x=2$ και $y=0$.
  \hfill Απ: $\overline{x}={6}/{5}$, $\overline{y}={3}/{4}$

\item  Να υπολογιστούν το \textbf{κέντρο μάζας} και η \textbf{ροπή αδράνειας} ως προς 
  την αρχή των αξόνων της επιφάνειας που περικλείεται από τις καμπύλες $y=x^{3}$ και  
  $y=4x$ με $x,y\geq 0$, όταν $\delta(x,y)=1$.

  \hfill Απ: $\overline{x}={16}/{15}$, $\overline{y}={64}/{21}$, 
  $I_{o}={848}/{15}$
\end{enumerate}


\end{document}
