\documentclass[a4paper,table]{report}
\input{preamble.tex}
\input{definitions_ask.tex}


\pagestyle{askhseis}


\begin{document}

\begin{center}
  \minibox{\large\bfseries \textcolor{Col1}{Ασκήσεις στις Ιδιοτιμές - Ιδιοδιανύσματα
  Πίνακα}}
\end{center}

\vspace{\baselineskip}

\begin{enumerate}

\item Να βρεθούν οι \textbf{ιδιοτιμές} και τα \textbf{ιδιοδιανύσματα} των παρακάτω 
  πινάκων:

\begin{enumerate}[i)]

\item $\begin{pmatrix}
3 & 2 \\
2 & 3
\end{pmatrix}$\hfill Απ: \begin{tabular}{l}
$\lambda_1=1$, $\lambda_2=5$ \\
$X_1=(-1,1)^T$ \\
$X_2=(1,1)^T$
\end{tabular}

\item $\begin{pmatrix}
1 & 2 \\
4 & 3
\end{pmatrix}$\hfill Απ: \begin{tabular}{l}
$\lambda_1=-1$, $\lambda_2=5$ \\
$X_1=(-1,1)^T$ \\
$X_2=(1,2)^T$
\end{tabular}

\item $\begin{pmatrix}
2 & 1 \\
1 & 2
\end{pmatrix}$\hfill Απ: \begin{tabular}{l}
$\lambda_1=-3$, $\lambda_2=1$ \\
$X_1=(1,1)^T$ \\
$X_2=(-1,1)^T$
\end{tabular}

\item $\begin{pmatrix}
2 & 1 \\
0 & 2
\end{pmatrix}$\hfill Απ: \begin{tabular}{l}
$\lambda_{1,2}=2 \; (\text{διπλή})$ \\
$X_1=(1,0)^T$ \\
\end{tabular}

\item $\begin{pmatrix}
2 & -3 \\
3 & \phantom{-}2
\end{pmatrix}$\hfill Απ: \begin{tabular}{l}
$\lambda_{1,2}=2\pm 3i$ \\
$X_1=(i,1)^T$ \\
$X_2=(-i,1)^T$
\end{tabular}

\end{enumerate}


\end{enumerate}



\end{document}

