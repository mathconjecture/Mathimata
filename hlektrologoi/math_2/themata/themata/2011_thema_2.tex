\documentclass[a4paper,table]{report}
\input{preamble_themata}
\newcommand{\vect}[2]{(#1_1,\ldots, #1_#2)}
%%%%%%% nesting newcommands $$$$$$$$$$$$$$$$$$$
\newcommand{\function}[1]{\newcommand{\nvec}[2]{#1(##1_1,\ldots, ##1_##2)}}

\newcommand{\linode}[2]{#1_n(x)#2^{(n)}+#1_{n-1}(x)#2^{(n-1)}+\cdots +#1_0(x)#2=g(x)}

\newcommand{\vecoffun}[3]{#1_0(#2),\ldots ,#1_#3(#2)}

\newcommand{\mysum}[1]{\sum_{n=#1}^{\infty}

\input{myboxes}

\everymath{\displaystyle}
\pagestyle{vangelis}


\begin{document}

\begin{mybox3}
  \begin{thema}
    Έστω η εξίσωση $ F(x,y,z)=z+x \mathrm{e}^{z} - y = 0 $. Να εξετάσετε αν μέσω της
    εξίσωσης αυτής ορίζεται 
    πεπλεγμένα η συνάρτηση $ z=z(x,y) $ σε μια περιοχή του σημείου $ P(0,0,0) $ και αν 
    ορίζεται να την προσεγγίσετε με το ανάπτυγμα \textbf{Maclaurin} μέχρι και όρους 2ης 
    τάξης.
  \end{thema}
\end{mybox3}
\begin{solution}
  \item {}
    Ελέγχουμε αν ικανοποιούνται οι προϋποθέσεις του θεωρήματος πεπλεγμένης 
    συνάρτησης για το σημείο $ P(0,0,0) $, 
    \begin{enumerate}[i)]
      \item $ F(0,0,0) = 0+0 \mathrm{e}^{0}-0 = 0 $
      \item $ F_{x} = \mathrm{e}^{z}, \; F_{y} = -1, \; F_{z} = 1+ x \mathrm{e}^{z}
        $ είναι συνεχείς στο σημείο $ P(0,0,0) $ 
      \item $ F_{z}(0,0,0) = 1 + 0 \mathrm{e}^{0} = 1 \neq 0 $
    \end{enumerate}
    Επομένως ικανοποιούνται οι προϋποθέσεις του θεωρήματος πεπλεγμένης συνάρτησης και 
    άρα υπάρχει μοναδική συνάρτηση $ z=z(x,y) $ με συνεχείς μερικές παραγώγους, για την 
    οποία ισχύουν:
    \begin{myitemize}
      \item \inlineequation[eq:zzero]{0 = z(0,0)} 
      \item $ z_{x} = - \frac{F_{x}}{F_{z}} = - \frac{\mathrm{e}^{z}}{1+x 
        \mathrm{e}^{z} } \Rightarrow \inlineequation[eq:zxzero]
        {z_{x}(0,0) \overset{\eqref{eq:zzero}}{=} - 
        \frac{\mathrm{e}^{0}}{1+ 0 \mathrm{e}^{0}} = \textcolor{Col1}{-1}} $
      \item $ z_{y} = - \frac{F_{y}}{F_{z}} = - \frac{-1}{1+x \mathrm{e}^{z}} =
        \frac{1}{1+x \mathrm{e}^{z}} \Rightarrow 
        \inlineequation[eq:zyzero]{z_{y}(0,0) \overset{\eqref{eq:zzero}}{=} 
        \frac{1}{1+0 \mathrm{e}^{0}} = \textcolor{Col1}{1}} $ 
    \end{myitemize}
    Για να βρούμε τις μερικές παραγώγους 2ης τάξης, ξαναπαραγωγίζουμε τις μερικές
    παραγώγους 1ης τάξης. Όμως, \textcolor{Col1}{προσοχή}, γιατί $ z=z(x,y) $, 
    είναι συνάρτηση και πρέπει να το λάβουμε υπόψη μας κατά την παραγώγιση.
    \begin{myitemize}
      \item $ z_{xx} = \Bigl(- \frac{\mathrm{e}^{z}}{1+ x \mathrm{e}^{z}}\Bigr)_{x} = - 
        \frac{(\mathrm{e}^{z})_{x} (1+x \mathrm{e}^{z}) - \mathrm{e}^{z}(1+x 
        \mathrm{e}^{z})_{x}}{(1+x \mathrm{e}^{z} )^{2}} = - 
        \frac{\mathrm{e}^{z}z_{x}(1+x \mathrm{e}^{z}) - 
        \mathrm{e}^{z}(\mathrm{e}^{z} + x \mathrm{e}^{z} z_{x})}{(1+x
      \mathrm{e}^{z})^{2}} \Rightarrow $

      $ z_{xx}(0,0) \overset{\eqref{eq:zzero}, \eqref{eq:zxzero}}{=} -
      \frac{\mathrm{e}^{0} (-1) (1+0 \mathrm{e}^{0}) - \mathrm{e}^{0} (\mathrm{e}^{0} + 0
      \mathrm{e}^{0})}{(1+ 0 \mathrm{e}^{0})^{2}} = - \frac{-1-1}{1} =
      \textcolor{Col1}{2}  $

    \item $ z_{yy} = \Bigl(\frac{1}{1+ x \mathrm{e}^{z}}\Bigr)_{y} = - \frac{1}{(1+x
        \mathrm{e}^{z} )^{2}} (1+x \mathrm{e}^{z} )_{y} = - \frac{x \mathrm{e}^{z}
      z_{y}}{(1+x \mathrm{e}^{z} )^{2}} \Rightarrow $

      $ z_{yy}(0,0) \overset{\eqref{eq:zzero}, \eqref{eq:zyzero}}{=} - \frac{0
      \mathrm{e}^{0} 1}{(1+0 \mathrm{e}^{0})^{2}} = \textcolor{Col1}{0} $

    \item $ z_{xy} = z_{yx} = \Bigl(\frac{1}{1+ x \mathrm{e}^{z}}\Bigr)_{x} = 
      - \frac{1}{(1+x \mathrm{e}^{z} )^{2}} (1+x \mathrm{e}^{z} )_{x} = -
      \frac{\mathrm{e}^{z} + x \mathrm{e}^{z} z_{x}}{(1+x \mathrm{e}^{z} )^{2}}
      \Rightarrow $ 

      $ z_{xy}(0,0) = z_{yx}(0,0)  \overset{\eqref{eq:zzero}, \eqref{eq:zxzero}}{=} 
      - \frac{\mathrm{e}^{0} + 0 \mathrm{e}^{0}(-1)}{(1+ 0 
      \mathrm{e}^{0})^{2}} = \textcolor{Col1}{-1} $
  \end{myitemize}

  Άρα το ανάπτυγμα Maclaurin της $ z(x,y) $ θα είναι:
  \begin{align*}
    z(x,y) &= z(0,0) + \frac{1}{1!} [ z_x(0,0)x+z_{y}(0,0)y ] + 
    \frac{1}{2!} [ z_{xx}(0,0)x^{2}+2z_{xy}(0,0)xy+z_{yy}(0,0)y^{2} ] \\
           &= 0 + \frac{1}{1!} [(-1)x+1y] + \frac{1}{2} [2x^{2}+2(-1)xy+0y^{2}] \\ 
           &= -x+y + x^{2}-xy
  \end{align*}

  Άρα $ z(x,y) = -x+y+z^{2}-xy $

\end{solution}


\end{document}
