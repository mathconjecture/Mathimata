\documentclass[a4paper,12pt]{article}
\usepackage{etex}
%%%%%%%%%%%%%%%%%%%%%%%%%%%%%%%%%%%%%%
% Babel language package
\usepackage[english,greek]{babel}
% Inputenc font encoding
\usepackage[utf8]{inputenc}
%%%%%%%%%%%%%%%%%%%%%%%%%%%%%%%%%%%%%%

%%%%% math packages %%%%%%%%%%%%%%%%%%
\usepackage{amsmath}
\usepackage{amssymb}
\usepackage{amsfonts}
\usepackage{amsthm}
\usepackage{proof}

\usepackage{physics}

%%%%%%% symbols packages %%%%%%%%%%%%%%
\usepackage{bm} %for use \bm instead \boldsymbol in math mode 
\usepackage{dsfont}
\usepackage{stmaryrd}
%%%%%%%%%%%%%%%%%%%%%%%%%%%%%%%%%%%%%%%


%%%%%% graphicx %%%%%%%%%%%%%%%%%%%%%%%
\usepackage{graphicx}
\usepackage{color}
%\usepackage{xypic}
\usepackage[all]{xy}
\usepackage{calc}
\usepackage{booktabs}
\usepackage{minibox}
%%%%%%%%%%%%%%%%%%%%%%%%%%%%%%%%%%%%%%%

\usepackage{enumerate}

\usepackage{fancyhdr}
%%%%% header and footer rule %%%%%%%%%
\setlength{\headheight}{14pt}
\renewcommand{\headrulewidth}{0pt}
\renewcommand{\footrulewidth}{0pt}
\fancypagestyle{plain}{\fancyhf{}
\fancyhead{}
\lfoot{}
\rfoot{\small \thepage}}
\fancypagestyle{vangelis}{\fancyhf{}
\rhead{\small \leftmark}
\lhead{\small }
\lfoot{}
\rfoot{\small \thepage}}
%%%%%%%%%%%%%%%%%%%%%%%%%%%%%%%%%%%%%%%

\usepackage{hyperref}
\usepackage{url}
%%%%%%% hyperref settings %%%%%%%%%%%%
\hypersetup{pdfpagemode=UseOutlines,hidelinks,
bookmarksopen=true,
pdfdisplaydoctitle=true,
pdfstartview=Fit,
unicode=true,
pdfpagelayout=OneColumn,
}
%%%%%%%%%%%%%%%%%%%%%%%%%%%%%%%%%%%%%%

\usepackage[space]{grffile}

\usepackage{geometry}
\geometry{left=25.63mm,right=25.63mm,top=36.25mm,bottom=36.25mm,footskip=24.16mm,headsep=24.16mm}

%\usepackage[explicit]{titlesec}
%%%%%% titlesec settings %%%%%%%%%%%%%
%\titleformat{\chapter}[block]{\LARGE\sc\bfseries}{\thechapter.}{1ex}{#1}
%\titlespacing*{\chapter}{0cm}{0cm}{36pt}[0ex]
%\titleformat{\section}[block]{\Large\bfseries}{\thesection.}{1ex}{#1}
%\titlespacing*{\section}{0cm}{34.56pt}{17.28pt}[0ex]
%\titleformat{\subsection}[block]{\large\bfseries{\thesubsection.}{1ex}{#1}
%\titlespacing*{\subsection}{0pt}{28.80pt}{14.40pt}[0ex]
%%%%%%%%%%%%%%%%%%%%%%%%%%%%%%%%%%%%%%

%%%%%%%%% My Theorems %%%%%%%%%%%%%%%%%%
\newtheorem{thm}{Θεώρημα}[section]
\newtheorem{cor}[thm]{Πόρισμα}
\newtheorem{lem}[thm]{λήμμα}
\theoremstyle{definition}
\newtheorem{dfn}{Ορισμός}[section]
\newtheorem{dfns}[dfn]{Ορισμοί}
\theoremstyle{remark}
\newtheorem{remark}{Παρατήρηση}[section]
\newtheorem{remarks}[remark]{Παρατηρήσεις}
%%%%%%%%%%%%%%%%%%%%%%%%%%%%%%%%%%%%%%%




\newcommand{\vect}[2]{(#1_1,\ldots, #1_#2)}
%%%%%%% nesting newcommands $$$$$$$$$$$$$$$$$$$
\newcommand{\function}[1]{\newcommand{\nvec}[2]{#1(##1_1,\ldots, ##1_##2)}}

\newcommand{\linode}[2]{#1_n(x)#2^{(n)}+#1_{n-1}(x)#2^{(n-1)}+\cdots +#1_0(x)#2=g(x)}

\newcommand{\vecoffun}[3]{#1_0(#2),\ldots ,#1_#3(#2)}

\newcommand{\mysum}[1]{\sum_{n=#1}^{\infty}

\input{tikz.tex}

% \geometry{top=3cm}

\usepackage[cmtip,all]{xy}
\usepackage{silence}
\WarningsOff[catoptions]
\everymath{\displaystyle}
\pagestyle{vangelis}

\geometry{left=9mm,right=9mm,top=30.00mm,bottom=34.00mm,footskip=24.16mm,headsep=24.16mm}

\begin{document}


\chapter{Μερική Παραγώγιση}

\section{Συναρτήσεις Μερικών Παραγώγων}

Έστω $ f(x,y) $ συνάρτηση δύο μεταβλητών. 
\begin{myitemize}
  \item Η \textcolor{Col1}{μερική παράγωγος της $f$ ως προς $x$} 
    υπολογίζεται παραγωγίζοντας την συνάρτηση $ f(x,y) $ ως προς $x$, 
    θεωρώντας το $y$ σταθερό. 
  \item Η \textcolor{Col1}{μερική παράγωγος της $f$ ως προς $y$} 
    υπολογίζεται παραγωγίζοντας την συνάρτηση $ f(x,y) $ ως προς $y$, 
    θεωρώντας το $x$ σταθερό. 
\end{myitemize}

\begin{rem}
  Γενικότερα η \textcolor{Col1}{μερική παράγωγος της $f$ ως προς $ x_{i} $} 
  υπολογίζεται παραγωγίζοντας τη συνάρτηση $ f(x_{1}, \ldots, x_{n}) $ ως προς 
  $ x_{i} $, θεωρώντας \textbf{όλες} τις υπόλοιπες μεταβλητές σταθερές.
\end{rem}

\section{Συμβολισμός}

\begin{align*}
  \eval{\pdv{f}{x} }_{(x_{0}, y_{0})} = \pdv{f(x_{0}, y_{0})}{x} = 
  f_{x}(x_{0}, y_{0}) = f'_{x}(x_{0}, y_{0} ) \quad \text{και} \quad
  \eval{\pdv{f}{y} }_{(x_{0}, y_{0})} = \pdv{f(x_{0}, y_{0})}{y} = 
  f_{y}(x_{0}, y_{0}) = f'_{y}(x_{0}, y_{0} ) 
\end{align*} 

\section{Κανόνες Παραγώγισης}

\twocolumnsides{
  \begin{myitemize}
    \item $ \pdv{x}(f+g) = \pdv{f}{x} + \pdv{g}{x} $
    \item $ \pdv{x}(af) = a \pdv{f}{x} $ 
  \end{myitemize}}{
  \begin{myitemize}
    \item $ \pdv{x}(f\cdot g) = \pdv{f}{x} \cdot g + f \cdot \pdv{g}{x} $
    \item $ \pdv{x}(\frac{f}{g}) = \frac{\pdv{f}{x} \cdot g - f \cdot 
      \pdv{g}{x}}{g^{2}} $
\end{myitemize}}

\begin{examples}
\item {}
  \begin{enumerate}
    \item Έστω $ f(x,y)=x^{2}y^{3}+4xy^{2}+4y+5 $. Να 
      υπολογιστούν οι μερικές παράγωγοι $ f_{x} $ και 
      $ f_{y} $.
      \begin{solution}
        \begin{align*}
          f_{x} &= (x^{2}y^{3}+4xy^{2}+4y+5)_{x} =
          (x^{2}y^{3})_{x}+(4xy^{2})_{x}+(4y)_{x}+(5)_{x} = 2xy^{3} + 4y^{2}
          \intertext{και}
          f_{y}&=(x^{2}y^{3}+4xy^{2}+4y+5)_{y} = 
          (x^{2}y^{3})_{y}+(4xy^{2})_{y}+(4y)_{y}+(5)_{y} = 3x^{2}y^{2} + 
          8xy + 4
        \end{align*} 
      \end{solution}
    \item Έστω $ f(x,y)=2x^{2}y+3 \cos{3y} +1 $. Να υπολογιστούν οι 
      μερικές παράγωγοι $ f_{x}$ και $ f_{y} $.
      \begin{solution}
        \[
          f_{x}=4xy \quad \text{και} \quad f_{y}=2x^{2}-3 \sin{3y} (3y)_{y} 
          = 2x^{2}-9 \sin{3y}
        \] 
      \end{solution}
    \item Έστω $ f(x,y,z)=x^{2}yz - y \cos{(xy)} $. Να υπολογιστούν οι 
      μερικές παράγωγοι $ f_{x}, f_{y}, f_{z} $. 
      \begin{solution}
      \item {}
        \begin{align*}
          f_{x}&=2xyz- \cos{(xy)}(xy)_{x} = 2xyz-y \cos{xy} \\
          f_{y}&=x^{2}z- \cos{xy}(xy)_{y}=x^{2}z - x \cos{xy} \\
          f_{z}&=x^{2}z
        \end{align*}
      \end{solution}
  \end{enumerate}
\end{examples}

\section{Μερικές Παράγωγοι Ανώτερης Τάξης}

\begin{example}
\item {}
  Έστω $ f(x,y,z) = 3x^{2}y^{2} + xy^{3} + 3x +1 $. 
  Να υπολογιστούν οι μερικές παράγωγοι 1ης και 2ης τάξης.
  \begin{solution}
  \item {} 
    \begin{align*}
      f_{x} &= 6xy^{2}+y^{3}+3 \quad \text{και} \quad 
      f_{y} = 6x^{2}y+3xy^{2} \\
      f_{xx} &= (f_{x})_{x} = (6xy^{2}+y^{3}+3)_{x} =
      6y^{2} \\
      f_{yy} &= (f_{y})_{y} = (6x^{2}y+3xy^{2})_{y} = 
      6x^{2}+6xy \\
      f_{xy} &= (f_{x})_{y} = (6xy^{2}+y^{3}+3)_{y} = 
      12xy = 3y^{2} \\
      f_{yx} &= (f_{y})_{x} = (6x^{2}y+3xy^{2})_{x} = 
      12xy+3y^{2} \; \; \,
    \end{align*}
  \end{solution}
\end{example}

\begin{rem}
\item {}
  Για τις μικτές παραγώγους $ f_{xy} $ και $ f_{yx} $ 
  ισχύει:
  \begin{align*}
    \pdv[2]{f}{x}{y} = \pdv{}{x} \left(\pdv{f}{y}\right) = \pdv{x} (f_{y}) 
    = (f_{y})_{x} = f_{yx}
    \intertext{και}
    \pdv[2]{f}{y}{x} = \pdv{}{y} \left(\pdv{f}{x}\right) = \pdv{y} (f_{x}) 
    = (f_{x})_{y} = f_{xy}
  \end{align*} 
\end{rem}

\begin{rem}
\item {}
  Οι πολυωνυμικές συναρτήσεις δύο (ή περισσότερων) μεταβλητών, 
  έχουν συνεχείς μερικές παραγώγους σε κάθε σημείο του $ \mathbb{R}^{2} $ 
  (ή $\mathbb{R}^{n}$).
  Οι λοιπές στοιχειώδεις συναρτήσεις $ \sin{f(x,y)}, \cos{f(x,y)}, a^{f(x,y)}, 
  \ln{f(x,y)} $ κ.λ.π. όπου $ f(x,y) $ πολυωνυμική συνάρτηση, έχουν 
  συνεχείς μερικές παραγώγους σε κάθε σημείο του πεδίου ορισμού τους.
  Για αυτές τις συναρτήσεις ισχύει $ f_{xy}=f_{yx} $.
\end{rem}

\mythmm{Schwartz}{Αν για τη συνάρτηση $ f(x,y) $ \textcolor{Col1}{υπάρχουν} 
  οι μερικές παράγωγοι $ f_{xy} $ και $ f_{yx} $ και είναι 
  \textcolor{Col1}{συνεχείς} σε μια περιοχή του σημείου 
$ (x_{0}, y_{0}) $, τότε $ f_{xy}=f_{yx} $ στην περιοχή αυτή.}

\section{Μερική Ολοκλήρωση}

\begin{rem}
\item {}
  Αν $ f_{x}(x,y) = g(x,y)$ και $ f_{y}(x,y)=h(x,y) $ τότε ισχύει:
  \begin{align*}
    f(x,y) = \int g(x,y) \,{dx} + c(y) \quad \text{και} \quad f(x,y) = 
    \int h(x,y) \,{dy} + c(x) 
  \end{align*} 
\end{rem}

\begin{example}
\item {}
  Έστω $ f(x,y)$ με $ f_{x}=e^{x+y} $ και $ f(0,y)=e^{y} $. 
  Να βρεθεί ο τύπος της $f$.
  \begin{solution}
    \begin{align*}
      f(x,y) = \int e^{x+y} \,{dx} = e^{x+y} + c(y) \; \tikzmark{a} \\ 
      f(0,y) = e^{y} \Leftrightarrow e^{y}+ c(y) = e^{y} \Rightarrow c(y) = 0 
      \; \tikzmark{b}
    \end{align*}
    \mybrace{a}{b}[ $f(x,y) = e^{x+y}$ ]
  \end{solution}
\end{example}

\begin{rem}
  Όταν ολοκληρώνουμε μια συνάρτηση πολλών μεταβλητών, ως προς κάποια από τις 
  μεταβλητές, τότε η σταθερά ολοκλήρωσης είναι \textcolor{Col1}{συνάρτηση} 
  των υπολοίπων μεταβλητών, οι οποίες θεωρούνται σταθερές κατά την ολοκλήρωση.
\end{rem}

\section{Ολικό Διαφορικό}

\mydfn{Έστω η συνάρτηση $ f(x,y) $. Το \textcolor{Col1}{ολικό διαφορικό 1ης και 
  2ης τάξης} της συνάρτηση $f$ είναι:
  \[
    \boxed{df = f_{x}dx + f_{y}dy} \quad \text{και} \quad 
    \boxed{d^{2}f = f_{xx}dx^{2}+2f_{xy}dxdy+f_{yy}dy^{2}}
  \] 
  Στην περίπτωση όπου $ f= f(x_{1}, x_{2}, \ldots, x_{n}) $, το ολικό 
  διαφορικό 1ης τάξης γίνεται: 
  \[
    df = f_{x_{1}}d{x_{1}} + f_{x_{2}}d{x_{2}} + \cdots f_{x_{n}} dx_{n}
  \]
}

\begin{example}
  Να βρείτε το ολικό διαφορικό της συνάρτησης $ f(x,y) = xye^{x+2y} $ 
  \begin{solution}
  \item {}
    Το ολικό διαφορικό δίνεται από τη σχέση $ df = f_{x} dx + f_{y} dy $.  
    Για τις μερικές παραγώγους έχουμε ότι: 
    \[
      f_{x} = ye^{x+2y}+xye^{x+2y} \quad \text{και} \quad f_{y} = xe^{x+2y} +
      2xye^{x+2y}
    \] 
    Επομένως
    \[
      df = y(1+x)e^{x+2y} dx + x(1+2y)e^{x+2y}dy
    \]
  \end{solution}
\end{example}

\section{Εφαρμογές του Διαφορικού}

\begin{example}
  Να υπολογίσετε το $ \Delta f $ και το $ df $ στο σημείο $ (x,y) = (1,1) $, 
  για τη συνάρτηση $ f(x,y) = x^{3}+y^{2} $ και για 
  \begin{enumerate}[i)]
    \item $ \Delta x = 0,1, \; \Delta y = 0,1 $
    \item $ \Delta x = 0,01, \; \Delta y = 0,01 $
  \end{enumerate}
  Τι παρατηρείτε?
\end{example}
\begin{solution}
\item {}
  \begin{enumerate}[i)]
    \item 
      \begin{align*} 
        \Delta f &= f(x+ \Delta x, y + \Delta y) - f(x,y) = f(1+0.1,1+0.1) - 
        f(1,1) = f(1.1,1.1) - f(1,1) \\ 
                 &= [(1.1)^{3}+(1.1)^{2}] - (1^{3}+1^{2}) = 0.541 \\
        df &= f_{x}dx + f_{y}dy  = 3x^{2} dx + 2y dy \Rightarrow df (1,1) = 
        3\cdot 1^{2} \cdot 0.1 + 2 \cdot 1 \cdot 0.1 = 0.5 
      \end{align*}
    \item 
      \begin{align*} 
        \Delta f &= f(x+ \Delta x, y + \Delta y) - f(x,y) = f(1+0.01,1+0.01) - 
        f(1,1) = f(1.01,1.01) - f(1,1) \\ 
                 &= [(1.01)^{3}+(1.01)^{2}] - (1^{3}+1^{2}) = 0.0504 \\
        df &= f_{x}dx + f_{y}dy  = 3x^{2} dx + 2y dy \Rightarrow df (1,1) = 
        3\cdot 1^{2} \cdot 0.01 + 2 \cdot 1 \cdot 0.01 = 0.05 
      \end{align*}
  \end{enumerate}
  Παρατηρούμε ότι στην 1η περίπτωση, έχουμε
  $ \Delta f - df \approx 0.04 $, ενώ στη 2η περίπτωση έχουμε $ \Delta f - df \approx 
  0,0004 $, δηλαδή στη 2η περίπτωση, όπου τα $ \Delta x $ και $ \Delta y $ επιλέχθηκαν 
  να είναι μικρότερα, το διαφορικό προσεγγίζει με μεγαλύτερη ακρίβεια τη διαφορά 
  $ \Delta f $.
\end{solution}



\section{Τύπος Taylor και Maclaurin}

\begin{example}
  Να υπολογιστεί το ανάπτυγμα της συνάρτησης $f(x,y)=x^3+y^3+xy^2$ γύρω από το 
  σημείο $ (1,2) $ (ή ισοδύναμα σε δυνάμεις του $(x-1)$ και $(y-2)$).
\end{example}
\begin{solution}
  Για να υπολογίσουμε το ανάπτυγμα της συνάρτησης σε δυνάμεις του $(x-1)$ και $(y-2)$ 
  αρκεί ισοδύναμα να υπολογίσουμε το ανάπτυγμα της συνάρτησης γύρω από το 
  σημείο $(x_0,y_0)=(1,2)$

  Παρατηρώ ότι δεν αναφέρεται στην εκφώνηση μέχρι τους όρους ποιας τάξης 
  xρειάζεται να βρω το ανάπτυγμα.  Γι' αυτό, μιας και η συνάρτηση είναι πολυωνυμική,
  βρίσκω μέχρι την τάξη όπου μηδενίζονται οι μερικές παράγωγοι: 

  (Δηλαδή στο συγκεκριμένο παράδειγμα μέχρι $3$ης τάξης, αφού όλες οι παράγωγοι 
  $4$ης και ανώτερης τάξης, θα είναι όλες μηδέν)

  \vspace{\baselineskip}

  \twocolumnsides{\begin{itemize}
      \item $f_x=3x^2+y^2\Rightarrow f_x(1,2)=7$
      \item $f_y=3y^2+2xy\Rightarrow f_y(1,2)=16$
      \item $f_{xx}=6x\Rightarrow f_{xx}(1,2)=6$
      \item $f_{xy}=2y\Rightarrow f_{xy}(1,2)=4$
      \item $f_{yy}=6y+2x\Rightarrow f_{yy}(1,2)=14$
      \end{itemize}}{\begin{itemize}
      \item $f_{xxx}=6$
      \item $f_{xxy}=f_{xyx}=0$
      \item $f_{xyy}=f_{yxy}=2$
      \item $f_{yyy}=6$ 
  \end{itemize}}

  \vspace{\baselineskip}

  Με αντικατάσταση των μερικών παραγώγων στον τύπο Taylor, έχουμε:
  \begin{align*}
    f(x,y)&=13+\Bigl(7(x-1)+16(y-2)\Bigr)+ \\ 
          &\quad +\frac{1}{2!}\Bigl(6(x-1)^2 +2\cdot 4(x-1)(y-2)+14(y-2)^2\Bigr)+ \\
          &\quad +\frac{1}{3!}\Bigl(6(x-1)^3+3\cdot 0(x-1)^2(y-2)+3
          \cdot 2(x-1)(y-2)^2+6(y-2)^3\Bigr).
  \end{align*}
  Και μετά τις πράξεις, έχουμε: 
  \begin{align*}
    f(x,y)&=13+7(x-1)(y-2)+16(y-2)+3(x-1)^2+4(x-1)(y-2) \\
          &\quad +7(y-2)^2+(x-1)^3+(x-1)(y-2)^2+(y-2)^3.
  \end{align*}
  Δεν κάνουμε άλλες πράξεις.

  Έχουμε το ανάπτυγμα της $f(x,y)$ σε δυνάμεις του $(x-1)$ και $(y-2)$ όπως ζητήθηκε.
\end{solution}

\section{Κλίση}

\begin{dfn}[Τελεστής Hamilton ή Ανάδελτα]
  \[ \grad = \pdv{}{x} \mathbf{i} + \pdv{}{y} \mathbf{j} + \pdv{}{z} \mathbf{k} \quad
  \text{ή} \quad \grad = \left(\pdv{}{x} , \pdv{}{y} , \pdv{}{z}\right)  \]
\end{dfn}

\begin{dfn}
  Έστω $ f(x,y,z) $ συνάρτηση. Τότε η διανυσματική συνάρτηση 
  \[ \grad f = \pdv{f}{x} \mathbf{i}+ \pdv{f}{y} \mathbf{j} + \pdv{f}{z} \mathbf{k} 
  \quad \text{ή} \quad \grad f = \left(\pdv{f}{x} , \pdv{f}{y} , \pdv{f}{z}\right)\]
  ονομάζεται \textcolor{Col1}{κλίση} της συνάρτησης $ f(x,y,z) $.
\end{dfn}

\section{Παράγωγος κατά Κατεύθυνση}

\begin{dfn}
  Η \textcolor{Col1}{παράγωγος κατά κατεύθυνση} της συνάρτησης $ f(x,y) $ στο σημείο 
  $ P(x_{0}, y_{0}) $ και προς την κατεύθυνση του διανύσματος $ \mathbf{u} $ 
  δίνεται από τον τύπο
  \[
    \dv{f}{\mathbf{u}} = D_{\mathbf{u}}(f)(P) = \grad f(P) \cdot \hat{\mathbf{u}} 
  \] 
  όπου $ \grad f (P) $ είναι η κλίση της συνάρτησης $f$ στο σημείο $P$ και 
  $ \mathbf{u} $ είναι το \textbf{μοναδιαίο} διάνυσμα $ \mathbf{u} $.
\end{dfn}
\begin{example}
  Έστω η συνάρτηση $ f(x,y) = 100-x^{2}-y^{2} $. 
  \begin{enumerate}[i)]
    \item Να βρείτε την παράγωγο της $f$ στο σημείο $ P_{0}(3,4) $ και προς την 
      κατεύθυνση του \textbf{διανύσματος} $ \mathbf{u} = 3 \mathbf{i}- 4 \mathbf{j} $.
    \item Να βρείτε την παράγωγο της $f$ στο σημείο $ P_{0}(3,4) $ και προς την 
      κατεύθυνση του \textbf{σημείου} $P(1,2)$.
    \item Προς ποιά κατεύθυνση η $f$ αυξάνει με το \textbf{μεγαλύτερο ρυθμό}, 
      στο σημείο $ P_{0} $; Ποιος είναι ο μέγιστος ρυθμός μεταβολής της $f$;
    \item Προς ποιά κατεύθυνση η παράγωγος κατά κατεύθυνση γίνεται \textbf{μηδέν}; στο 
      σημείο $ P_{0} $;
  \end{enumerate}
\end{example}
\begin{solution}
  \begin{enumerate}[i)]
    \item Έχουμε ότι $ D_{\mathbf{u}}f(P_{0}) = \grad f(P_{0}) \cdot 
      \mathbf{\widehat{u}} $
      \begin{align*}
        \grad f = (f_{x}, f_{y}) = (-2x, -2y) \Rightarrow \grad f(P_{0}) = (-6,-8) \\
        \mathbf{\widehat{u}} = \frac{\mathbf{u}}{\norm{\mathbf{u}}} =
        \frac{1}{\sqrt{3^{2}+(-4)^{2}}} (3,-4) = \left(\frac{3}{5} , - 
        \frac{4}{5}\right)
      \end{align*} 
      Επομένως
      \[
        D_{\mathbf{u}}f(P_{0}) = (-6,-8) \cdot \left(\frac{3}{5} , - \frac{4}{5}\right) 
        = - \frac{18}{5} + \frac{32}{5} = \frac{14}{5} > 0
      \] 
    \item Βρίσκουμε το διάνυσμα $ \vec{P_{0}P} = (1-3,2-4) = (-2,-2) $. Τότε το 
      μοναδιαίο $ \vec{\widehat{P_{0}P}} =
      \frac{\vec{P_{0}P}}{\norm{\vec{P_{0}P}}} = \frac{1}{\sqrt{8}} (-2,-2) = \left(-
      \frac{1}{\sqrt{2}} , - \frac{1}{\sqrt{2}}\right) $
      Άρα 
      \[
        D_{\mathbf{P_{0}P}}f(P_{0}) = \grad f(P_{0}) \cdot \vec{P_{0}P} = (-6,-8) 
        \cdot \left(- \frac{1}{\sqrt{2}} , - \frac{1}{\sqrt{2}}\right) = 
        \frac{6}{\sqrt{2}} + \frac{8}{\sqrt{2}} = \frac{14}{\sqrt{2}} > 0
      \] 
    \item Ζητάμε την κατεύθυνση $ \mathbf{u} $ προς την οποία η $f$ αυξάνει με το 
      μεγαλύτερο ρυθμό, δηλαδή η παράγωγος κατά κατεύθυνση γίνεται μέγιστη. Έχουμε:
      \[
        D_{\mathbf{u}}f(P_{0}) \max \Leftrightarrow \norm{\grad f (P_{0})}
        \norm{\mathbf{\widehat{u}}} \cos{\theta} \max \Leftrightarrow 
        \norm{\grad f(P_{0})} \cos{\theta} \max \Leftrightarrow \cos{\theta} \max
        \Leftrightarrow \theta = 0
      \]
      Δηλαδή $ \mathbf{u} $ παράλληλο με $ \grad f(P_{0}) $. Οπότε, η $f$ αυξάνει με 
      το μεγαλύτερο ρυθμό, προς την κατεύθυνση της κλίσης της. Μάλιστα, ισχύει:
      \[
        - \norm{\grad f(P_{0})} \leq D_{\mathbf{u}} f(P_{0}) \leq \norm{\grad f(P_{0})}
      \]
      για κάθε κατεύθυνση $ \mathbf{u} $ στο σημείο $ P_{0} $. 
      Δηλαδή, ο \textbf{μέγιστος ρυθμός} μεταβολής της $f$ στο σημείο $ P_{0} $ είναι $
      D_{\mathbf{u}}f(P_{0})_{\max} = \norm{\grad f(P_{0})} $ και παρατηρείται προς 
      την κατεύθυνση που δείχνει η κλίση, και ο \textbf{ελάχιστος ρυθμός} μεταβολής
      αντίστοιχα, είναι $ D_{\mathbf{u}}f(P_{0})_{\min} = -\norm{\grad f(P_{0})} $ 
      και παρατηρείται προς την αντίθετη κατεύθυνση από αυτή της κλίσης.
    \item Έχουμε ότι 
      \[ 
        D_{\mathbf{u}}f(P_{0}) = 0 \Leftrightarrow \norm{\grad f (P_{0})}
        \norm{\mathbf{\widehat{u}}} \cos{\theta} = 0 \Leftrightarrow 
        \cos{\theta} = 0 \Leftrightarrow \theta = \pi/2 
      \]
      Δηλαδή, προς κατεύθυνση \textbf{κάθετη} προς την κατεύθυνση που μας δείχνει η 
      κλίση της $ f $ στο σημείο $ P_{0} $, ο ρυθμός μεταβολής της συνάρτησης είναι 
      μηδέν, και επομένως η $f$ είναι σταθερή. Για να προσδιορίσουμε το διάνυσμα $
      \mathbf{u} $, έχουμε: Έστω $ \mathbf{u} = (u_{1}, u_{2}) $. 
      \[
        \mathbf{u} \perp \grad f(P_{0}) \Leftrightarrow \mathbf{u} \cdot 
        \grad f (P_{0}) = 0 \Leftrightarrow (u_1,u_{2}) \cdot (-6,-8) =
        0 \Leftrightarrow \inlineequation[eq:perp]{-6u_{1}-8u_{2}=0}
      \]
      Επίσης, επειδή $ \mathbf{u} $ είναι μοναδιαίο, έχουμε: 
      \[
        \norm{\mathbf{u}} =1 \Leftrightarrow \sqrt{u_{1}^{2}+u_{2}^{2}} = 1
        \Leftrightarrow \inlineequation[eq:unit]{u_{1}^{2}+u_{2}^{2}=1}
      \] 
      Από τις παραπάνω σχέσεις, για τα $ u_{1} $ και $ u_{2} $, εύκολα τα υπολογίζουμε,
      και βρίσκουμε ότι $ \mathbf{u} = \left(- \frac{4}{5}
      , \frac{3}{5}\right) $ ή $ \mathbf{u} = 
      \left( \frac{4}{5} , - \frac{3}{5}\right) $. 
  \end{enumerate}
\end{solution}



\section{Εφαπτόμενο Επίπεδο}

\begin{myitemize}
  \item Αν $S: F(x,y,z)=0$ τότε η διανυσματική εξίσωση του εφαπτόμενου επιπέδου της $S$ 
    στο σημείο της $ P(x_{0}, y_{0}, z_{0}) $ είναι
    \begin{equation}\label{eq:tan}
      (\mathbf{r} - \mathbf{r}_{0})\cdot \grad F(P_{0})= \mathbf{0} 
    \end{equation} 
    όπου $ \mathbf{r}=x \mathbf{i}+y \mathbf{j}+z \mathbf{k} $ και 
    $ r_{0}=x_{0} \mathbf{i}+ y _{0} \mathbf{j}+ z_{0} \mathbf{k} $.

    Σε καρτεσιανές συντεταγμένες η εξίσωση ~\eqref{eq:tan} γράφεται:
    \begin{equation}\label{eq:tan2}
      \boxed{		
        \pdv{F}{x}\eval_{P_{0}} (x-x_{0}) + \pdv{F}{y}\eval_{P_{0}} (y-y_{0}) +
        \pdv{F}{z}\eval_{P_{0}} (z-z_{0}) =0 
      } 
    \end{equation}

  \item Αν $ S: z=f(x,y) $ τότε θέτουμε $ F(x,y,z) =  f(x,y) - z = 0 $. Οπότε σε
    αυτήν την περίπτωση η εξίσωση ~\eqref{eq:tan2} γράφεται
    \[
      \boxed{
        \pdv{f}{x}\eval_{P_{0}} (x - x_{0}) + 
        \pdv{f}{y}\eval_{P_{0}} (y - y_{0}) - (z - z_{0}) = 0 
      }
    \] 
\end{myitemize}

\begin{rem}
  Αν $ z=f(x,y) $ και $ \grad f(x_{0}, y_{0}) = 0 $, τότε το εφαπτόμενο επίπεδο 
  της επιφάνειας $S$ στο σημείο $ (x_{0}, y_{0}, f(x_{0}, y_{0})) $ είναι 
  παράλληλο στο επίπεδο $ xy $, γιατί τότε ένα κάθετο διάνυσμα στο επίπεδο είναι 
  \[
    \mathbf{n} = \pdv{f}{x} \left(x_{0}, y_{0}\right)  \mathbf{i} + 
    \pdv{f}{y} \left(x_{0}, y_{0}\right) \mathbf{j} - \mathbf{k} = - \mathbf{k}
  \] 
  το οποίο είναι παράλληλο στον άξονα $z$. Επομένως όταν 
  $ \grad f(x_{0}, y_{0}) = 0 $ το εφαπτόμενο επίπεδο επίπεδο της επιφάνειας στο 
  $ (x_{0}, y_{0}, f(x_{0}, y_{0})) $ είναι το οριζόντιο επίπεδο με εξίσωση 
  $ z = z_{0} $.
\end{rem}
%todo παραδειγμα εφαπτομενο επιπεδο


\section{Κάθετη Ευθεία}

\begin{myitemize}
  \item Αν $S: F(x,y,z)=0$ τότε ένα κάθετο διάνυσμα στην επιφάνεια $S$ 
    στο σημείο $ P(x_{0}, y_{0}, z_{0}) $ είναι
    \[
      \grad F{(P_{0})} = \left(\pdv{F}{x}, \pdv{F}{y}, \pdv{F}{z}\right)_{P_{0}}
    \]
  \item Αν $S: F(x,y,z)=0$ τότε η διανυσματική εξίσωση της κάθετης ευθείας της 
    επιφάνειας $S$ στο σημείο της $ P(x_{0}, y_{0}, z_{0}) $ είναι
    \begin{equation}\label{eq:kath}
      (\mathbf{r} - \mathbf{r_{0}}) \times \grad F(P_{0}) = \mathbf{0}
    \end{equation} 
    όπου $ \mathbf{r}=x \mathbf{i}+y \mathbf{j}+z \mathbf{k} $ και 
    $ r_{0}=x_{0} \mathbf{i}+ y _{0} \mathbf{j}+ z_{0} \mathbf{k} $.

    Σε καρτεσιανές συντεταγμένες η εξίσωση ~\eqref{eq:kath} γράφεται:
    \begin{equation}\label{eq:kath2}
      \boxed{ 
        \frac{x- x_{0}}{\eval{\pdv{F}{x}} _{P_{0}}} = 
        \frac{y- y_{0}}{\eval{\pdv{F}{y}} _{P_{0}}} = 
        \frac{z- z_{0}}{\eval{\pdv{F}{z}} _{P_{0}}}  
      }
    \end{equation}

  \item Αν $ S: z=f(x,y) $ τότε θέτουμε $ F(x,y,z) =  f(x,y) - z = 0 $. Οπότε σε
    αυτήν την περίπτωση η εξίσωση ~\eqref{eq:kath2} γράφεται
    \[
      \boxed{
        \frac{x- x_{0}}{\eval{\pdv{f}{x}} _{P_{0}}} = 
        \frac{y- y_{0}}{\eval{\pdv{f}{y}} _{P_{0}}} = 
        \frac{z- z_{0}}{-1} 
      }
    \] 
\end{myitemize}
%todo παραδειγμα κάθετη ευθεία

\chapter{Παράγωγος Σύνθετων Συναρτήσεων}

\section{1η Περίπτωση: \ensuremath{z=f(x,y),  x=x(t),  y=y(t)}} 

\begin{thm}
  Αν η συνάρτηση $ f(x,y) $ είναι ορισμένη στο ανοιχτό σύνολο 
  $ A \subseteq \mathbb{R}^{2} $ και $ x = x(t) $, $ y=y(t) $, με 
  $ t \in [a,b] $ και η $f$ έχει συνεχείς μερικές 
  παραγώγους στο $A$ και οι $ x(t) $ και $ y(t) $ είναι παραγωγίσιμες στο 
  $ [a,b] $, τότε η παράγωγος της σύνθετης συνάρτησης $f$ ως προς $t$ δίνεται από 
  τον τύπο:

  \twocolumnsidel{
    \[\xymatrix{ & f \ar@{-}[dl]_{\mathlarger{\pdv{f}{x}}}
        \ar@{-}[dr]^{\mathlarger{\pdv{f}{y}}} &  \\
        x \ar@{-}[dr]_{\mathlarger{\dv{x}{t}}} & & y 
    \ar@{-}[dl]^{\mathlarger{\dv{y}{t}}} \\ & t & }\]
    }{
    \begin{equation}\label{eq:deriv1}
      \dv{f}{t} = \pdv{f}{x} \dv{x}{t} + \pdv{f}{y} \dv{y}{t} 
    \end{equation}
    Ενώ η 2η παράγωγος από τον τύπο:
    \[
      \dv[2]{f}{t} =  \pdv[2]{f}{x} \left(\dv{x}{t}\right)^{2} + 
      2 \pdv[2]{f}{x}{y} \dv{x}{t} \dv{y}{t} + \pdv[2]{f}{y} 
      \left(\dv{y}{t}\right)^{2} + \pdv{f}{x} \dv[2]{x}{t} + \pdv{f}{y} \dv[2]{y}{t}
    \]
  }
\end{thm}

\section{2η Περίπτωση: \ensuremath{z=f(x,y),  x=x(u,v),  y=y(u,v)}} 

\begin{thm}
  Αν η συνάρτηση $ f(x,y) $ είναι ορισμένη στο ανοιχτό σύνολο 
  $ A \subseteq \mathbb{R}^{2} $ και $ x = x(u,v) $, $ y=y(u,v) $, με 
  και η $f$ έχει συνεχείς μερικές παραγώγους στο $A$ και οι $ x $ και $ y $, έχουν 
  συνεχείς μερικές παραγώγους στο $ E \subseteq \mathbb{R}^{2} $,
  τότε οι μερικές παράγωγοι της $f$, υπάρχουν και δίνονται από τους τύπους:
\end{thm}

\twocolumnsidel{
  \[
    \xymatrix@C-6pt{ & & f \ar@{-}[dl] 
      _{\mathlarger{\pdv{f}{x}}}^{\mathlarger{f_{x}}} 
      \ar@{-}[dr]^{\mathlarger{\pdv{f}{y}}}_{\mathlarger{f_{y}}}
                     & &  \\
                     & x \ar@{-}[dl]_{\mathlarger{\pdv{x}{u}}}
      \ar@{-}[d]^{\mathlarger{\pdv{x}{v}}} &  
                                           & y \ar@{-}[d]_{\mathlarger{\pdv{y}{u}}} 
      \ar@{-}[dr]^{\mathlarger{\pdv{y}{v}}} & \\
    u & v & & u & v \\ } 
  \]
  }{
  \begin{equation}\label{eq:deriv2}
    \pdv{f}{u} = \pdv{f}{x} \pdv{x}{u} + \pdv{f}{y} \pdv{y}{u} 
    \quad \text{και} \quad
    \pdv{f}{v} = \pdv{f}{x} \pdv{x}{v} + \pdv{f}{y} \pdv{y}{v} 
  \end{equation}
  Ενώ οι μερικές παράγωγοι 2ης τάξης, δίνονται από τους τύπους:
  \[
    \pdv[2]{f}{u} =  \pdv[2]{f}{x} \left(\pdv{x}{u}\right)^{2} + 
    2 \pdv[2]{f}{x}{y} \pdv{x}{u} \pdv{y}{u} + \pdv[2]{f}{y} 
    \left(\pdv{y}{u}\right)^{2} + \pdv{f}{x} \pdv[2]{x}{u} + \pdv{f}{y} 
    \pdv[2]{y}{u}
  \]
  \[
    \pdv[2]{f}{v} =  \pdv[2]{f}{x} \left(\pdv{x}{v}\right)^{2} + 
    2 \pdv[2]{f}{x}{y} \pdv{x}{u} \pdv{y}{v} + \pdv[2]{f}{y} 
    \left(\pdv{y}{v}\right)^{2} + \pdv{f}{x} \pdv[2]{x}{v} + \pdv{f}{y} 
    \pdv[2]{y}{v}
  \]
  \[
    \pdv[2]{f}{v}{u} = \pdv[2]{f}{x} \pdv{x}{u} \pdv{x}{v} + \pdv[2]{f}{x}{y}
    \left(\pdv{x}{u} \pdv{y}{v}+ \pdv{x}{v} \pdv{y}{u} \right) + \pdv[2]{f}{y} 
    \pdv{y} {u} \pdv{y}{v} + \pdv{f}{x} \pdv[2]{x}{u}{v} + \pdv{f}{y} 
    \pdv[2]{y}{u}{v} 
  \]
}


\begin{rem}
  Οι τύποι~\eqref{eq:deriv1} και~\eqref{eq:deriv2} προέκυψαν αθροίζοντας κάθε φορά, 
  τα μονοπάτια που ξεκινούν από τη μεταβλητή $f$ και καταλήγουν στη μεταβλητή ως 
  προς την οποία παραγωγίζουμε, όπου κάθε μονοπάτι αποτελείται από το γινόμενο των 
  παραγώγων που συναντούμε "διασχίζοντάς" το.
\end{rem}



\section{Αποδείξεις των τύπων των μερικών Παραγώγων 2ης τάξης}

\subsection{Απόδειξη με το συμβολισμό του Leibnitz}

Αποδεικνύουμε τον τύπο για την $ \pdv[2]{f}{u} $ και ομοίως προκύπτουν και οι τύποι 
για τις $ \pdv[2]{f}{v}  $ και $ \pdv[2]{f}{u}{v} $.

\vspace{\baselineskip}

\twocolumnsidel{
  \[ 
    \xymatrix@C-6pt{ & & \pdv{f}{x} 
      \ar@{-}[dl]_{\mathlarger{\pdv[2]{f}{x}}} 
      \ar@{-}[dr]^{\mathlarger{\pdv[2]{f}{x}{y}}} & &  \\
                                                  & x \ar@{-}[dl]
                                                  _{\mathlarger{\pdv{x}{u}}}
      \ar@{-}[d]^{\mathlarger{\pdv{x}{v}}} &  
                                           & y \ar@{-}[d]_{\mathlarger{\pdv{y}{u}}} 
      \ar@{-}[dr]^{\mathlarger{\pdv{y}{v}}} & \\
      u & v & & u & v \\ 
    } 
  \] 
  \[\xymatrix@C-6pt{ & & \pdv{f}{y} 
      \ar@{-}[dl]_{\mathlarger{\pdv[2]{f}{y}{x}}} 
      \ar@{-}[dr]^{\mathlarger{\pdv[2]{f}{y}}} & &  \\
                                               & x \ar@{-}[dl]
                                               _{\mathlarger{\pdv{x}{u}}}
      \ar@{-}[d]^{\mathlarger{\pdv{x}{v}}} &  
                                           & y \ar@{-}[d]_{\mathlarger{\pdv{y}{u}}} 
      \ar@{-}[dr]^{\mathlarger{\pdv{y}{v}}} & \\
    u & v & & u & v \\ }
  \] 
  }{
  \begin{proof}
    \[\begin{aligned}
      \pdv[2]{f}{u} 
  &= \pdv{}{u}\left(\pdv{f}{u}\right) = \pdv{}{u} 
  \left( \pdv{f}{x} \pdv{x}{u} + \pdv{f}{y} \pdv{y}{u}\right) = 
  \pdv{}{u} \left(\pdv{f}{x} \pdv{x}{u}\right) + \pdv{}{u} 
  \left(\pdv{f}{y} \pdv{y}{u}\right) \\
  &= \pdv{}{u} \left( \pdv{f}{x}\right) \pdv{x}{u} + \pdv{f}{x} \pdv{}{u} \left(
  \pdv{x}{u} \right) + \pdv{}{u} \left(\pdv{f}{y} \right) \pdv{y}{u} + \pdv{f}{y} 
  \pdv{}{u} \left(\pdv{y}{u}\right) \\
  &=\left[ \pdv[2]{f}{x} \pdv{x}{u} + \pdv[2]{f}{x}{y} \pdv{y}{u} \right] \pdv{x}{u} +
  \pdv{f}{x} \pdv[2]{x}{u} + 
  \left[ \pdv[2]{f}{y}{x} \pdv{x}{u} + \pdv[2]{f}{y} \pdv{y}{u} \right] \pdv{y}{u} +
  \pdv{f}{y} \pdv[2]{y}{u} \\
  &= \pdv[2]{f}{x} \left(\pdv{x}{u}\right)^{2} + \pdv[2]{f}{x}{y} \pdv{y}{u}
  \pdv{x}{u} + \pdv{f}{x} \pdv[2]{x}{u} + \pdv[2]{f}{y}{x} \pdv{x}{u} \pdv{y}{u} + 
  \pdv[2]{f}{y} \left(\pdv{y}{u}\right)^{2} +
  \pdv{f}{y} \pdv[2]{y}{u} \\
  &= \pdv[2]{f}{x} \left(\pdv{x}{u}\right)^{2} + 2\pdv[2]{f}{x}{y} \pdv{x}{u}
  \pdv{y}{u} + \pdv[2]{f}{y} \left(\pdv{y}{u}\right)^{2} + \pdv{f}{x} \pdv[2]{x}{u} +
  \pdv{f}{y} \pdv[2]{y}{u} \\
    \end{aligned}
  \]
\end{proof}
  }


  \subsection{Απόδειξη με το συμβολισμό των δεικτών}

  Αποδεικνύουμε τον τύπο για την $ f_{uu} $ και ομοίως προκύπτουν και οι τύποι για τις  
  $ f_{vv} $ και $ f_{uv} $.

  \vspace{\baselineskip}

  \twocolumnsidel{
    {
      \[ 
        \xymatrix@C-6pt{ & & f_{x}
          \ar@{-}[dl]_{\mathlarger{f_{xx}}} 
          \ar@{-}[dr]^{\mathlarger{f_{xy}}} & &  \\
                                            & x \ar@{-}[dl]_{\mathlarger{x_{u}}}
          \ar@{-}[d]^{\mathlarger{x_{v}}} &  
                                          & y \ar@{-}[d]_{\mathlarger{y_{u}}} 
          \ar@{-}[dr]^{\mathlarger{y_{v}}} & \\
          u & v & & u & v \\ 
        } 
      \] 
      \[\xymatrix@C-6pt{ & & f_{y}
          \ar@{-}[dl]_{\mathlarger{f_{yx}}} 
          \ar@{-}[dr]^{\mathlarger{f_{yy}}} & &  \\
                                            & x \ar@{-}[dl]_{\mathlarger{x_{u}}}
          \ar@{-}[d]^{\mathlarger{x_{v}}} &  
                                          & y \ar@{-}[d]_{\mathlarger{y_{u}}} 
          \ar@{-}[dr]^{\mathlarger{y_{v}}} & \\
        u & v & & u & v \\ }
      \] 
    }
    }{
    \begin{proof}
      \[
        \begin{aligned}
          f_{uu} &= (f_{u})_{u} = (f_{x}x_{u}+f_{y}y_{u})_{u} \\
                 &=(f_{x}x_{u})_{u}+ (f_{y}y_{u})_{u} \\
                 &=(f_{x})_{u}x_{u} + f_{x}(x_{u})_{u} + (f_{y})_{u}y_{u}+ 
                 f_{y}(y_{u})_{u} \\
                 &= (f_{xx}x_{u}+f_{xy}{y_{u}})x_{u} + f_{x} x_{uu} + 
                 (f_{yx}x_{u}+f_{yy}y_{u})y_{u} + f_{y}y_{uu} \\
                 &= f_{xx}(x_{u})^{2} + f_{xy}y_{u}x_{u}+ f_{x}x_{uu} + 
                 f_{yx}x_{u}y_{u}+f_{yy}(y_{u})^{2}+ f_{y}y_{uu} \\
                 &= f_{xx}(x_{u})^{2}+ 2f_{xy}x_{u}y_{u} + f_{yy}(y_{u})^{2} + 
                 f_{x}x_{uu} + f_{y}y_{uu}
        \end{aligned}
      \] 
    \end{proof}
  }

  \begin{rem}
    Συγκεντρωτικά οι τύποι για τις παραγώγους 2ης ταξής της συνάρτησης 

    \[
      f_{uu}= f_{xx}(x_{u})^{2}+ 2f_{xy}x_{u}y_{u} + f_{yy}(y_{u})^{2} + f_{x}x_{uu} + 
      f_{y}y_{uu} 
    \] 
    \[
      f_{vv}= f_{xx}(x_{v})^{2}+ 2f_{xy}x_{v}y_{v} + f_{yy}(y_{v})^{2} + f_{x}x_{vv} + 
      f_{y}y_{vv} 
    \]
    \[
      f_{uv}= f_{xx}x_{u}x_{v}+ f_{xy}(x_{u}y_{v} + x_{v}y_{u}) + 
      f_{yy}y_{u}y_{v} + f_{x}x_{uv} + f_{y}y_{v} 
      f_{y}y_{vv} 
    \]
  \end{rem}


\end{document}

