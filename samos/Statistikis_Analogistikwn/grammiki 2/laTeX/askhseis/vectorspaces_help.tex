\documentclass[a4paper,12pt]{article}
\usepackage{etex}
%%%%%%%%%%%%%%%%%%%%%%%%%%%%%%%%%%%%%%
% Babel language package
\usepackage[english,greek]{babel}
% Inputenc font encoding
\usepackage[utf8]{inputenc}
%%%%%%%%%%%%%%%%%%%%%%%%%%%%%%%%%%%%%%

%%%%% math packages %%%%%%%%%%%%%%%%%%
\usepackage{amsmath}
\usepackage{amssymb}
\usepackage{amsfonts}
\usepackage{amsthm}
\usepackage{proof}

\usepackage{physics}

%%%%%%% symbols packages %%%%%%%%%%%%%%
\usepackage{bm} %for use \bm instead \boldsymbol in math mode 
\usepackage{dsfont}
\usepackage{stmaryrd}
%%%%%%%%%%%%%%%%%%%%%%%%%%%%%%%%%%%%%%%


%%%%%% graphicx %%%%%%%%%%%%%%%%%%%%%%%
\usepackage{graphicx}
\usepackage{color}
%\usepackage{xypic}
\usepackage[all]{xy}
\usepackage{calc}
\usepackage{booktabs}
\usepackage{minibox}
%%%%%%%%%%%%%%%%%%%%%%%%%%%%%%%%%%%%%%%

\usepackage{enumerate}

\usepackage{fancyhdr}
%%%%% header and footer rule %%%%%%%%%
\setlength{\headheight}{14pt}
\renewcommand{\headrulewidth}{0pt}
\renewcommand{\footrulewidth}{0pt}
\fancypagestyle{plain}{\fancyhf{}
\fancyhead{}
\lfoot{}
\rfoot{\small \thepage}}
\fancypagestyle{vangelis}{\fancyhf{}
\rhead{\small \leftmark}
\lhead{\small }
\lfoot{}
\rfoot{\small \thepage}}
%%%%%%%%%%%%%%%%%%%%%%%%%%%%%%%%%%%%%%%

\usepackage{hyperref}
\usepackage{url}
%%%%%%% hyperref settings %%%%%%%%%%%%
\hypersetup{pdfpagemode=UseOutlines,hidelinks,
bookmarksopen=true,
pdfdisplaydoctitle=true,
pdfstartview=Fit,
unicode=true,
pdfpagelayout=OneColumn,
}
%%%%%%%%%%%%%%%%%%%%%%%%%%%%%%%%%%%%%%

\usepackage[space]{grffile}

\usepackage{geometry}
\geometry{left=25.63mm,right=25.63mm,top=36.25mm,bottom=36.25mm,footskip=24.16mm,headsep=24.16mm}

%\usepackage[explicit]{titlesec}
%%%%%% titlesec settings %%%%%%%%%%%%%
%\titleformat{\chapter}[block]{\LARGE\sc\bfseries}{\thechapter.}{1ex}{#1}
%\titlespacing*{\chapter}{0cm}{0cm}{36pt}[0ex]
%\titleformat{\section}[block]{\Large\bfseries}{\thesection.}{1ex}{#1}
%\titlespacing*{\section}{0cm}{34.56pt}{17.28pt}[0ex]
%\titleformat{\subsection}[block]{\large\bfseries{\thesubsection.}{1ex}{#1}
%\titlespacing*{\subsection}{0pt}{28.80pt}{14.40pt}[0ex]
%%%%%%%%%%%%%%%%%%%%%%%%%%%%%%%%%%%%%%

%%%%%%%%% My Theorems %%%%%%%%%%%%%%%%%%
\newtheorem{thm}{Θεώρημα}[section]
\newtheorem{cor}[thm]{Πόρισμα}
\newtheorem{lem}[thm]{λήμμα}
\theoremstyle{definition}
\newtheorem{dfn}{Ορισμός}[section]
\newtheorem{dfns}[dfn]{Ορισμοί}
\theoremstyle{remark}
\newtheorem{remark}{Παρατήρηση}[section]
\newtheorem{remarks}[remark]{Παρατηρήσεις}
%%%%%%%%%%%%%%%%%%%%%%%%%%%%%%%%%%%%%%%




\newcommand{\vect}[2]{(#1_1,\ldots, #1_#2)}
%%%%%%% nesting newcommands $$$$$$$$$$$$$$$$$$$
\newcommand{\function}[1]{\newcommand{\nvec}[2]{#1(##1_1,\ldots, ##1_##2)}}

\newcommand{\linode}[2]{#1_n(x)#2^{(n)}+#1_{n-1}(x)#2^{(n-1)}+\cdots +#1_0(x)#2=g(x)}

\newcommand{\vecoffun}[3]{#1_0(#2),\ldots ,#1_#3(#2)}

\newcommand{\mysum}[1]{\sum_{n=#1}^{\infty}


\begin{document}





\begin{enumerate}
    \item Έστω τα διανύσματα $ \mathbf{u_{1}} = (1,2), \mathbf{u_{2}} = (0,1) $ και 
        $ \mathbf{u_{3}} = (-1,0) $ του $ \mathbb{R}^{2} $.
        \begin{enumerate}[i)]
            \item Να γραφεί το $ \mathbf{u_{1}} $ ως γραμμικός συνδυασμός των $
                \mathbf{u_{2}} $ και $ \mathbf{u_{3}} $.
            \item Να δείξετε ότι τα $ \mathbf{u_{2}} $ και $ \mathbf{u_{3}} $ παράγουν 
                τον $ \mathbb{R}^{2} $.

                \hfill Απ: 
                \begin{tabular}{l}
                    $ \mathrm{i)} \quad \mathbf{u_{1}} = 2 \mathbf{u_{2}} - 
                    \mathbf{u_{3}} $ \\
                    $ \mathrm{ii)} \quad (x,y) = y(0,1) - x(-1,0) $
                \end{tabular} 
        \end{enumerate}


    \item Έστω $ W_{1} = \{(x,y,z,w)\in \mathbb{R}^{4} \; : \; y+z+w=0 \} $ και 
        $ W_{2} = \{(x,y,z,w)\in \mathbb{R}^{4} \; : \; x+y=0 \; 
        \text{και} \; z=2w \} $. Να βρεθούν βάσεις των υπόχωρων $ W_{1}, W_{2} $ και 
        $ W_{1} \cap W_{2} $. 

        \hfill Απ: 
        \begin{tabular}{l}
            $ B_{W_{1}} = \{ \begin{pmatrix} 1 & 0 & 0 & 0  \end{pmatrix}^{T}, 
                \begin{pmatrix} 0 & -1 & 1 & 0 \end{pmatrix}^{T}, 
            \begin{pmatrix} 0 & -1 & 0 & 1 \end{pmatrix}^{T} \} $ \\
            $ B_{W_{2}} = \{ 
                \begin{pmatrix}-1 & 1 & 0 & 0\end{pmatrix}^{T}, 
            \begin{pmatrix}0 & 0 & 2 & 1\end{pmatrix}^{T} \} $ \\
            $ B_{W_{1} \cap W_{2}} = \{ 
            \begin{pmatrix}3 & -3 & 2 & 1\end{pmatrix}^{T} \} $
        \end{tabular}

    \item Έστω $ \mathbf{u}_{1} = 
            \begin{pmatrix}1 & 2 & -2 & 1\end{pmatrix}^{T}, \mathbf{u_{2}} = 
            \begin{pmatrix}1 & 3 & -1 &4\end{pmatrix}^{T}, \mathbf{u_{3}} = 
            \begin{pmatrix}2 & 1 & -7 & -7\end{pmatrix}^{T} $, 
            διανύσματα του $ \mathbb{R}^{4} $. Να δείξετε ότι είναι 
            γραμμικώς εξαρτημένα και να βρεθεί μια σχέση που τα συνδέει.

            \hfill Απ: $ -5 \mathbf{u_{1}}+ 3 \mathbf{u_{2}} + \mathbf{u_{3}}=0 $ 

        \item Έστω $ W_{1} = < 
            \begin{pmatrix}1 & 2 & 3\end{pmatrix}^{T}, 
            \begin{pmatrix}2 & -1 & 1\end{pmatrix}^{T} > $ και $ W_{2} = < 
            \begin{pmatrix}1 & 0 & 1\end{pmatrix}^{T}, 
            \begin{pmatrix}0 & 1 & 1\end{pmatrix}^{T}> $ υπόχωροι του 
            $ \mathbb{R}^{3} $.  Να δείξετε ότι $ W_{1} = W_{2} $.

        \item Έστω $ W_{1} = < 
            \begin{pmatrix}1 & 2 & 5\end{pmatrix}^{T}, 
            \begin{pmatrix}1 & 3 & 7\end{pmatrix}^{T} > $ και $ W_{2} = < 
            \begin{pmatrix}1 & 0 & 1\end{pmatrix}^{T}, 
            \begin{pmatrix}0 & 1 & 2\end{pmatrix}^{T}> $ υπόχωροι του 
            $ \mathbb{R}^{3} $.  Να δείξετε ότι $ W_{1} = W_{2} $.

        \item Να βρεθεί μια βάση του $ \mathbb{R}^{4} $ που να 
            περιέχει το διάνυσμα $ \mathbf{u} = 
            \begin{pmatrix}1 & 1 & 2 & 3\end{pmatrix}^{T} $.

            \hfill Απ: $ B = 
            \{ 
                \mathbf{u}, \mathbf{e_{2}}, \mathbf{e_{3}}, 
                \mathbf{e_{4}} 
            \} $  

        \item 
            \begin{enumerate}[i)]
                \item Να βρεθεί μια βάση, καθώς και η διάσταση του υπόχωρου $W$ του 
                    $ \mathbb{R}^{4} $ που παράγεται από τα διανύσματα 
                    $ \mathbf{u_{1}} = \begin{pmatrix}1 & 2 & -2 & 3\end{pmatrix}^{T} $,
                    $ \mathbf{u_{2}} = \begin{pmatrix}5 & 6 & -4 & 13\end{pmatrix}^{T}, 
                    \mathbf{u_{3}} = \begin{pmatrix}2 & 0 & 2 & 4\end{pmatrix}^{T} $.
                \item Να επεκτείνετε τη βάση αυτή σε μια βάση του $\mathbb{R}^{4} $ 
            \end{enumerate}

            \hfill Απ: \begin{tabular}{l}
                $ \mathrm{i)} \quad B_{W} = 
                \{ 
                    \begin{pmatrix}1 & 2 & -2 & 3\end{pmatrix}^{T}, 
                    \begin{pmatrix}0 & -4 & 6 & -2\end{pmatrix}^{T} 
                \} $ \\
                $ \mathrm{ii)} \quad B_{\mathbb{R}^{4}} = 
                \{ 
                    \begin{pmatrix}1 & 2 & -2 & 3\end{pmatrix}^{T}, 
                    \begin{pmatrix}0 & -4 & 6 & -2 \end{pmatrix}^{T}, 
                    \begin{pmatrix}0 & 0 & 1 & 0\end{pmatrix}^{T}, 
                    \begin{pmatrix}0 & 0 & 0 & 1\end{pmatrix}^{T} 
                \} $ 
            \end{tabular}

        \item Να εξετάσετε αν τα διανύσματα $ 
                    \begin{pmatrix}1 & 2 & 3\end{pmatrix}^{T}, 
                    \begin{pmatrix}0 & 1 & 2\end{pmatrix}^{T} , 
                    \begin{pmatrix}0 & 0 & 1\end{pmatrix}^{T}$ 
                    \begin{enumerate}[i)]
                        \item είναι γραμμικώς ανεξάρτητα 
                        \item παράγουν τον $ \mathbb{R}^{3} $.
                    \end{enumerate}

                \item Να βρεθεί μια βάση του υπόχωρου που παράγουν τα $ 
                    \begin{pmatrix}1 & 3 & -1\end{pmatrix}^{T} , 
                    \begin{pmatrix}0 & 3 & -3\end{pmatrix}^{T} , 
                    \begin{pmatrix}-1 & -4 & 2\end{pmatrix}^{T}$.

                    \hfill Απ: $ B_{W} = 
                    \{ 
                        \begin{pmatrix}1 & 3 & -1\end{pmatrix}^{T} , 
                        \begin{pmatrix}0 & 3 & -3 \end{pmatrix}^{T}
                    \} $. 

                \item Έστω $ V $ ένας διανυσματικός χώρος επί του $\mathbb{R}$ και έστω 
                    $ \mathbf{u}, \mathbf{v} \in V $. Να δείξετε ότι 
                    $ < \mathbf{u}, \mathbf{v} > = < \mathbf{u} - \mathbf{v}, 
                    \mathbf{u} + \mathbf{v} > $

                \item Να εξετάσετε αν τα παρακάτω υποσύνολα είναι βάσεις του 
                    $ \mathbb{R}^{3} $.
                    \begin{enumerate}[i)]
                        \item $ B_{1} = 
                            \{ 
                                \begin{pmatrix}1 & 0 & 0\end{pmatrix}^{T} , 
                                \begin{pmatrix}0 & 1 & 0\end{pmatrix}^{T} , 
                                \begin{pmatrix}0 & 0 & 0\end{pmatrix}^{T}
                            \} $ \hfill Απ: όχι 
                        \item $ B_{2} = 
                            \{ 
                                \begin{pmatrix}1 & 1 & 1\end{pmatrix}^{T}, 
                                \begin{pmatrix}1 & 2 & 1\end{pmatrix}^{T} , 
                                \begin{pmatrix}2 & 1 & 1\end{pmatrix}^{T}
                            \} $ \hfill Απ:  
                    \end{enumerate}
                        \end{enumerate}





                        %%%%%%%%%%%%%%%%%%%%% Γραμμική Ανεξαρτησία %%%%%%%%%%%%%%%%%%%%%%%%%%%%%%
                        \begin{enumerate}
                            \item Να εξετάσετε αν το διάνυσμα $ \mathbf{b}= 
                                \begin{pmatrix}0 & 1 & 1\end{pmatrix}^{T}$ γράφεται ως γραμμικός 
                                συνδυασμός των $ 
                                \begin{pmatrix}1 & 1 & 0\end{pmatrix}^{T}$ και $ 
                                \begin{pmatrix}1 & -1 & -1\end{pmatrix}^{T} $.

                                \hfill Απ: όχι 

                            \item Αν τα διανύσματα $ \mathbf{u}, \mathbf{v}, \mathbf{w} $ είναι 
                                γραμμικώς ανεξάρτητα, τότε να δείξετε ότι και τα διανύσματα 
                                $ \mathbf{u}+ \mathbf{v} $, $ \mathbf{u}- \mathbf{v} $ και 
                                $ \mathbf{u}- 2 \mathbf{v}+ \mathbf{w} $ είναι επίσης γραμμικώς ανεξάρτητα.

                            \item Δινονται τα διανύσματα $ \mathbf{u_{1}} = 
                                \begin{pmatrix}2 & 0 & 1 & 3\end{pmatrix}^{T} , 
                                \mathbf{u_{2}} = \begin{pmatrix}0 & 1 & 2 & 3\end{pmatrix}^{T} , 
                                \mathbf{u_{3}} = \begin{pmatrix}1 & -1 & 0 & 0\end{pmatrix}^{T}, 
                                \mathbf{u_{4}} = \begin{pmatrix}0 & -2 & 1 & 0\end{pmatrix}^{T}$ και 
                                $ \mathbf{b} = \begin{pmatrix}0 & 5 & 2 & 6\end{pmatrix}^{T}$.
                                \begin{enumerate}[i)]
                                    \item Να δείξετε ότι τα διανύσματα $ \mathbf{u_{1}}, \mathbf{u_{2}}, 
                                        \mathbf{u_{3}}, \mathbf{u_{4}} $ είναι γραμμικώς ανεξάρτητα.
                                    \item Να εκφράσετε το $ \mathbf{b} $ ως γραμμικό συνδυασμό των 
                                        $ \mathbf{u_{1}}, \mathbf{u_{2}}, \mathbf{u_{3}}, \mathbf{u_{4}} $.
                                \end{enumerate}

                            \item Αν $ \mathbf{u} = 
                                \begin{pmatrix}1 & 0 & 1\end{pmatrix}^{T} , \mathbf{v} = 
                                \begin{pmatrix}0 & 1 & -1\end{pmatrix}^{T}$ διανύσματα του 
                                $ \mathbb{R}^{3} $, τότε ποια σχέση πρέπει να ικανοποιούν τα 
                                $ a,b,c \in \mathbb{R} $, ώστε $ 
                                \begin{pmatrix}a & b & c\end{pmatrix}^{T} \in < \mathbf{u}, \mathbf{v} > $;

                                \hfill Απ: $ a-b=c $ 
                        \end{enumerate}

                        %%%%%%%%%%%%%%%%%%%%%%%%%%%%%%%%%%%%%%%%%%%%%%%%%%%%%%

                        



\end{document}
