\documentclass[a4paper,table]{report}
\input{preamble_themata}
\newcommand{\vect}[2]{(#1_1,\ldots, #1_#2)}
%%%%%%% nesting newcommands $$$$$$$$$$$$$$$$$$$
\newcommand{\function}[1]{\newcommand{\nvec}[2]{#1(##1_1,\ldots, ##1_##2)}}

\newcommand{\linode}[2]{#1_n(x)#2^{(n)}+#1_{n-1}(x)#2^{(n-1)}+\cdots +#1_0(x)#2=g(x)}

\newcommand{\vecoffun}[3]{#1_0(#2),\ldots ,#1_#3(#2)}

\newcommand{\suma}{\sum_{n=0}^{\infty}a_n x^n}

\newcommand{\sumb}{\sum_{n=1}^{\infty}a_n n x^{n-1}}

\newcommand{\sumc}{\sum_{n=2}^{\infty}a_n n (n-1) x^{n-2}}

\newcommand{\varsum}[2]{\sum_{n=#1}^{#2}}
\input{myboxes}

\everymath{\displaystyle}
\pagestyle{vangelis}


\begin{document}

\begin{mybox3}
  \begin{thema}
    Να δείξετε ότι κάθε ακμή του συνοδεύοντος τριέδρου της καμπύλης
    \[
      (c)\colon
      \mathbf{r}(t)=(\mathrm{e}^{t} \sin{t})\, \mathbf{i} + (\mathrm{e}^{t} \cos{t}
      )\, \mathbf{j} + \mathrm{e}^{t} \, \mathbf{k}
    \]
    στο τυχαίο σημείο της, σχηματίζει
    σταθερή γωνία με τον άξονα $ z $.
  \end{thema}
\end{mybox3}
\begin{solution}
  Θα βρούμε τα διανύσματα $ \mathbf{T}, \mathbf{N} $ και $ \mathbf{B} $ του συνοδεύοντος
  τριέδρου και θα βρούμε τη γωνία που σχηματίζουν με τον άξονα $ z $:
  \begin{align*}
    \mathbf{r'}(t)
    &= \mathrm{e}^{t} (\sin{t} + \cos{t})\, \mathbf{i} +
    \mathrm{e}^{t} (\cos{t} - \sin{t})\, \mathbf{j} + \mathrm{e}^{t} \, \mathbf{k} \\
    \norm{\mathbf{r'}(t)} 
    &= \sqrt{[\mathrm{e}^{t} (\sin{t} + \cos{t})]^{2}+ [\mathrm{e}^{t} (\cos{t} - \sin{t})]^{2}+ \mathrm{e}^{2t}} \\
    &= \sqrt{\mathrm{e}^{2t} (\sin^{2}{t} + 2 \cancel{\sin{t} \cos{t}} + \cos^{2}{t} + \cos^{2}{t} - 2 \cancel{\sin{t} \cos{t}} + \sin^{2}{t} +1)} = \sqrt{3} \,
    \mathrm{e}^{t} \\
    \mathbf{T} 
    &= \frac{\mathbf{r'}(t)}{\norm{\mathbf{r'}(t)}} = \frac{\sin{t} +
    \cos{t}}{\sqrt{3}} \, \mathbf{i} + \frac{\cos{t} - \sin{t}}{\sqrt{3}} \, \mathbf{j} +
    \frac{1}{\sqrt{3}} \, \mathbf{k}
  \end{align*}
  Η γωνία που σχηματίζει το διάνυσμα $ \mathbf{T} $ με τον άξονα $z$ είναι: $\cos{\theta} = \mathbf{T}\cdot \mathbf{k} = \frac{1}{\sqrt{3}} = \;\text{σταθ.}
    \Rightarrow \theta = \; \text{σταθ.}$ \begin{align*}
    \dv{\mathbf{T}}{t} 
     &= \frac{\cos{t} - \sin{t}}{\sqrt{3}} \, \mathbf{i} +
     \frac{- \sin{t} - \cos{t}}{\sqrt{3}} \, \mathbf{j} + 0 \mathbf{k} \\
     \norm{\dv{\mathbf{T}}{t}} 
     &= \sqrt{\Bigl[\frac{\cos{t} - \sin{t}}{\sqrt{3}} \Bigr]^{2}+
     \Bigl[\frac{- \sin{t} - \cos{t}}{\sqrt{3}}\Bigr]^{2} +0^{2}} \\
     &= \sqrt{\frac{\cos^{2}{t} - 2 \cancel{\sin{t} \cos{t}} + \sin^{2}{t} + \sin^{2}{t} + 2
     \cancel{\sin{t} \cos{t}} + \cos^{2}{t}}{3}} = \frac{\sqrt{2}}{\sqrt{3}} \\
     \mathbf{N} 
     &= \frac{d \mathbf{T}/dt}{\norm{d \mathbf{T}/dt}} = \frac{\cos{t} - \sin{t}
     }{\sqrt{2}} \, \mathbf{i} - \frac{\sin{t} + \cos{t}}{\sqrt{2}} \, \mathbf{j} +0
     \mathbf{k}
   \end{align*}
   Η γωνία που σχηματίζει το διάνυσμα $ \mathbf{N} $ με τον άξονα $z$ είναι: $\cos{\phi} = \mathbf{N}\cdot \mathbf{k} = 0 =\;\text{σταθ.} \Rightarrow \phi =
     \; \text{σταθ.}$ \begin{align*}
     \mathbf{B} = \mathbf{T} \times \mathbf{N} 
     &= \begin{vmatrix*}[c]
       \mathbf{i} & \mathbf{j} & \mathbf{k} \\[10pt]
        \frac{\sin{t} + \cos{t}}{\sqrt{3}} & \frac{\cos{t} - \sin{t}}{\sqrt{3}} &
        \frac{1}{\sqrt{3}} \\[10pt]
        \frac{\cos{t} - \sin{t}}{\sqrt{2}} & - \frac{\sin{t} + \cos{t}}{\sqrt{2}} & 0
      \end{vmatrix*} 
    = \frac{1}{\sqrt{3}} \cdot \frac{1}{\sqrt{2}} \cdot 
    \begin{vmatrix*}[c]
      \mathbf{i} & \mathbf{j} & \mathbf{k} \\[5pt]
      \sin{t} + \cos{t} & \cos{t} - \sin{t} & 1 \\[5pt]
       \cos{t} - \sin{t} & - (\sin{t} + \cos{t}) & 0
     \end{vmatrix*} \\
     &= \frac{1}{\sqrt{6}} \Bigl[(\sin{t} + \cos{t}) \, \mathbf{i} + (\cos{t}-
     \sin{t}) \, \mathbf{j} + \bigl[-(\sin{t} + \cos{t} )^{2}-(\cos{t} - \sin{t} )^{2}\bigr]\,
     \mathbf{k}\Bigr] \\
     &= \frac{1}{\sqrt{6}} \Bigl[(\sin{t} + \cos{t}) \, \mathbf{i} + (\cos{t} - \sin{t}) \,
     \mathbf{j} - 2 \, \mathbf{k}\Bigr] \\
     &= \frac{\sin{t} + \cos{t}}{\sqrt{6}} \, \mathbf{i} + \frac{\cos{t} -
     \sin{t}}{\sqrt{6}} \, \mathbf{j} - \frac{2}{\sqrt{6}} \, \mathbf{k}  
   \end{align*}
   Η γωνία που σχηματίζει το διάνυσμα $ \mathbf{B} $ με τον άξονα $z$ είναι: $\cos{\omega} = \mathbf{B}\cdot \mathbf{k} = - \frac{2}{\sqrt{6}} =\;\text{σταθ.} \Rightarrow \omega =
     \; \text{σταθ.}$ \end{solution}


\end{document}
