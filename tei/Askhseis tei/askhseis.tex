\documentclass[a4paper,12pt]{article}
\usepackage{etex}
%%%%%%%%%%%%%%%%%%%%%%%%%%%%%%%%%%%%%%
% Babel language package
\usepackage[english,greek]{babel}
% Inputenc font encoding
\usepackage[utf8]{inputenc}
%%%%%%%%%%%%%%%%%%%%%%%%%%%%%%%%%%%%%%

%%%%% math packages %%%%%%%%%%%%%%%%%%
\usepackage{amsmath}
\usepackage{amssymb}
\usepackage{amsfonts}
\usepackage{amsthm}
\usepackage{proof}

\usepackage{physics}

%%%%%%% symbols packages %%%%%%%%%%%%%%
\usepackage{bm} %for use \bm instead \boldsymbol in math mode 
\usepackage{dsfont}
\usepackage{stmaryrd}
%%%%%%%%%%%%%%%%%%%%%%%%%%%%%%%%%%%%%%%


%%%%%% graphicx %%%%%%%%%%%%%%%%%%%%%%%
\usepackage{graphicx}
\usepackage{color}
%\usepackage{xypic}
\usepackage[all]{xy}
\usepackage{calc}
\usepackage{booktabs}
\usepackage{minibox}
%%%%%%%%%%%%%%%%%%%%%%%%%%%%%%%%%%%%%%%

\usepackage{enumerate}

\usepackage{fancyhdr}
%%%%% header and footer rule %%%%%%%%%
\setlength{\headheight}{14pt}
\renewcommand{\headrulewidth}{0pt}
\renewcommand{\footrulewidth}{0pt}
\fancypagestyle{plain}{\fancyhf{}
\fancyhead{}
\lfoot{}
\rfoot{\small \thepage}}
\fancypagestyle{vangelis}{\fancyhf{}
\rhead{\small \leftmark}
\lhead{\small }
\lfoot{}
\rfoot{\small \thepage}}
%%%%%%%%%%%%%%%%%%%%%%%%%%%%%%%%%%%%%%%

\usepackage{hyperref}
\usepackage{url}
%%%%%%% hyperref settings %%%%%%%%%%%%
\hypersetup{pdfpagemode=UseOutlines,hidelinks,
bookmarksopen=true,
pdfdisplaydoctitle=true,
pdfstartview=Fit,
unicode=true,
pdfpagelayout=OneColumn,
}
%%%%%%%%%%%%%%%%%%%%%%%%%%%%%%%%%%%%%%

\usepackage[space]{grffile}

\usepackage{geometry}
\geometry{left=25.63mm,right=25.63mm,top=36.25mm,bottom=36.25mm,footskip=24.16mm,headsep=24.16mm}

%\usepackage[explicit]{titlesec}
%%%%%% titlesec settings %%%%%%%%%%%%%
%\titleformat{\chapter}[block]{\LARGE\sc\bfseries}{\thechapter.}{1ex}{#1}
%\titlespacing*{\chapter}{0cm}{0cm}{36pt}[0ex]
%\titleformat{\section}[block]{\Large\bfseries}{\thesection.}{1ex}{#1}
%\titlespacing*{\section}{0cm}{34.56pt}{17.28pt}[0ex]
%\titleformat{\subsection}[block]{\large\bfseries{\thesubsection.}{1ex}{#1}
%\titlespacing*{\subsection}{0pt}{28.80pt}{14.40pt}[0ex]
%%%%%%%%%%%%%%%%%%%%%%%%%%%%%%%%%%%%%%

%%%%%%%%% My Theorems %%%%%%%%%%%%%%%%%%
\newtheorem{thm}{Θεώρημα}[section]
\newtheorem{cor}[thm]{Πόρισμα}
\newtheorem{lem}[thm]{λήμμα}
\theoremstyle{definition}
\newtheorem{dfn}{Ορισμός}[section]
\newtheorem{dfns}[dfn]{Ορισμοί}
\theoremstyle{remark}
\newtheorem{remark}{Παρατήρηση}[section]
\newtheorem{remarks}[remark]{Παρατηρήσεις}
%%%%%%%%%%%%%%%%%%%%%%%%%%%%%%%%%%%%%%%




\newcommand{\vect}[2]{(#1_1,\ldots, #1_#2)}
%%%%%%% nesting newcommands $$$$$$$$$$$$$$$$$$$
\newcommand{\function}[1]{\newcommand{\nvec}[2]{#1(##1_1,\ldots, ##1_##2)}}

\newcommand{\linode}[2]{#1_n(x)#2^{(n)}+#1_{n-1}(x)#2^{(n-1)}+\cdots +#1_0(x)#2=g(x)}

\newcommand{\vecoffun}[3]{#1_0(#2),\ldots ,#1_#3(#2)}

\newcommand{\mysum}[1]{\sum_{n=#1}^{\infty}



\pagestyle{askhseis}
\everymath{\displaystyle}


\begin{document}

\begin{center}
  \minibox{\large\bfseries \textcolor{Col1}{Ασκήσεις}}
\end{center}

\vspace{\baselineskip}

\begin{enumerate}

  \item {\bfseries Δίνονται τα σημεία  \boldmath A$(-3,4) $, B$(1,2) $ και 
    $\Gamma(4,-6) $.}
    \begin{enumerate}[i)]
      \item {\bfseries Να δειχθεί ότι τα σημεία δεν είναι συνευθειακά.}
      \item {\bfseries Να βρεθεί η εξίσωση της διαμέσου  BM  του τριγώνου  AB$\Gamma$ }
    \end{enumerate}
    \begin{solution}
    \item {}
      \begin{enumerate}[i)]
        \item Έστω ότι τα σημεία A$(-3,4) $, B$(1,2) $ και $\Gamma(4,-6) $ είναι 
          συνευθειακά.

          Αυτό σημαίνει ότι υπάρχει $\lambda \in \mathbb{R}$ τέτοιο ώστε 
          \begin{align*}
          AB &= \lambda B\Gamma \\
            (1-(-3),2-4) &= \lambda (4-1,-6-2) \\
            (4,-2) &= \lambda (3,-8) \\
            (4,-2) &= (3\lambda, -8\lambda) \\
          \end{align*}
          \renewcommand{\arraystretch}{2}
          \[
            \left.
              \begin{tabular}{c}
                $ 4 = 3 \lambda $ \\
                $ -2 = -8 \lambda $
              \end{tabular}  
            \right\} \Leftrightarrow 
            \left.
              \begin{tabular}{c}
                $  \lambda = \frac{4}{3} $ \\
                $  \lambda = \frac{-2}{-8} $
              \end{tabular}  
            \right\} \Leftrightarrow 
            \left.
              \begin{tabular}{c}
                $  \lambda = \frac{4}{3} $ \\
                $  \lambda = \frac{1}{4} $
              \end{tabular}  
            \right\} \quad \text{Άτοπο!}
          \] 

        \item Εφόσον τα σημεία δεν είναι συνευθειακά ορίζουν τρίγωνο AB$\Gamma$.

          Έστω $M$ το μέσο της πλευράς A$\Gamma$.

          Οι συντεταγμένες του σημείου $ M $ δίνονται από τις σχέσεις:
          \[
            x_{M} = \frac{x_{A}+x_{\Gamma}}{2} \quad \text{και} \quad y_{M} = \frac{y_{A}+y_{\Gamma}}{2} 
          \] 
          Επομένως
          \[
            x_{M}= \frac{-3 + 4}{2} = \frac{1}{2} \quad \text{και} \quad y_{M}= \frac{4 - 6}{2} =
            \frac{-2}{2} =-1 
          \] 
          Η εξίσωση της διαμέσου BM είναι
          \[
            y - y_{B} = \frac{y_{M}-y_{B}}{x_{M}-x_{B}} (x-x_{B}) \Leftrightarrow \\
          \]
          \[
            y - 2 = \frac{-1 - 2}{\frac{1}{2} - 1 } (x-1) \Leftrightarrow 
          \] 
          \[
            y-2= \frac{-3}{- \frac{1}{2}} (x-1) \Leftrightarrow 
          \]
          \[
            y-2=6(x-1) \Leftrightarrow 
          \]
          \[
            y-2=6x-6 
          \] 
          \[
            \boxed{    y=6x-4 }
          \]
      \end{enumerate}
    \end{solution}

    \pagebreak

  \item {\bfseries Να βρεθούν οι εξισώσεις των εφαπτόμενων ευθειών της έλλειψης \boldmath $
    x^{2}+3y^{2}=4 $ που είναι παράλληλες στην ευθεία $ x-3y-2=0 $.}

    \begin{description}
      \item[Λύση] 
    \end{description}

    Η εξίσωση της εφαπτομένης της έλλειψης στο τυχαίο σημείο $ (x_{0}, y_{0}) $ είναι
    \[
      x x_{0} + 3y y_{0}= 4 \Leftrightarrow 3y y_{0} = 4 - x x_{0}  \Leftrightarrow y =
      -\frac{x_{0}}{3 y_{0}}x + \frac{4}{3 y_{0}} 
    \]
    και θέλουμε να είναι παράλληλη στην ευθεία 
    \[
      x-3y-2=0 \Leftrightarrow 3y=x-2 \Leftrightarrow y = \frac{1}{3} x - \frac{2}{3}  
    \]
    Άρα πρέπει η εφαπτομένη και η ευθεία να έχουν ίδιο συντελεστή διεύθυνσης, δηλ πρέπει
    \[
      -\frac{x_{0}}{3 y_{0}} = \frac{1}{3} \Leftrightarrow 3 y_{0} = -3 x_{0} \Leftrightarrow 
    \]
    \begin{equation}\label{eq:3}
      y_{0}=- x_{0} 
    \end{equation} 
    Το σημείο $ (x_{0}, y_{0}) $ είναι σημείο της έλλειψης, επομένως επαληθεύει την εξίσωσή της:
    \begin{equation}\label{eq:4}
      x_{0}^{2}+3 y_{0}^{2}=4 
    \end{equation} 
    Κάνοντας αντικατάσταση της εξίσωσης~\eqref{eq:3} στην~\eqref{eq:4} προκύπτει
    \[
      x_{0}^2+3(- x_{0})^{2}=4 \Leftrightarrow x_{0}^{2} + 3 x_{0}^{2} = 4 \Leftrightarrow 4
      x_{0}^{2} = 4 \Leftrightarrow x_{0}^{2} = 1 \Leftrightarrow x_{0} = \pm 1
    \] 
    και αντικαθιστώντας στην σχέση~\eqref{eq:3} τις τιμές του $ x_{0} $ που μόλις βρήκαμε έχουμε
    \[
      x_{0}=1 \Rightarrow y_{0}=-1 \quad \text{και} \quad x_{0}=-1 \Rightarrow y_{0}=1 
    \] 
    Επομένως οι εξισώσεις της έλλειψης που είναι παράλληλες στη δοσμένη ευθεία είναι οι
    \[
      \varepsilon_{1}: y = - \frac{1}{3\cdot (-1)} x + \frac{4}{3\cdot (-1)} \Rightarrow  \boxed {y =
      \frac{1}{3} x - \frac{4}{3}}
    \] 

    και 

    \[
      \varepsilon_{2}: y = - \frac{-1}{3 \cdot 1} x + \frac{4}{3 \cdot 1} \Rightarrow \boxed{y=
      \frac{1}{3} x + \frac{4}{3}}
    \] 

  \item  {\bfseries Δίνεται η συνάρτηση \boldmath $ f(x,y) = y^{3} - x^{3}
      +3x^{2}y - 3xy^{2} + 2xy - \sin^{10}{y} $.
    Να δειχθεί ότι \boldmath $ f_{xy}+f_{xx} = 2 $.}

    \begin{description}
      \item[Λύση]
    \end{description}

    Βρίσκω τις μερικές παραγώγους 1ης τάξης:
    \[ f_{x} = -3x^{2} + 6xy - 3y^{2} + 2y \]
    \[ f_{y} = 3y^{2}+3x^{2}-6xy +2x - 10 \sin^{9}{y}  \]
    Βρίσκω τις ζητούμενες μερικές παραγώγους 2ης τάξης:
    \[
      f_{xx} =  -6x+6y
    \] 
    \[
      f_{xy} = 6x -6y +2
    \] 
    Επομένως πράγματι ισχύει ότι
    \[
      f_{xy}+ f_{xx} = -6x+6y+6x-6y+2 = 2
    \] 


  \item {\bfseries Να λυθεί το πρόβλημα αρχικών τιμών \boldmath $  y''(x) + 5y'(x) +4y(x)=0
    $, \boldmath $ y(0)=2 $, \boldmath $ y'(0)=3 $.}

    \begin{description}
      \item[Λύση] 
    \end{description}

    Λύνω την χαρακτηριστική εξίσωση:

    \[
      \lambda ^{2} +5 \lambda + 4 = 0 
    \] 

    Βρίσκω τη διακρίνουσα του τριωνύμου:
    \[
      \Delta = 5^{2}-4 \cdot 1 \cdot 4 = 25 - 16 = 9 > 0
    \] 
    Επομένως έχω δυο λύσεις ως προς $\lambda$, τις: 
    \[
      \lambda _{1,2} = \frac{-5 \pm \sqrt{9}}{2} = \frac{-5 \pm 3}{2} \Rightarrow  
    \]
    \[
      \lambda _{1} = \frac{-5+3}{2} = \frac{-2}{2} = -1 \quad \text{και} \quad
      \lambda _{2} = \frac{-5-3}{2} = \frac{-8}{2} = -4   
    \] 

    Άρα η γενική λύση δίνεται από τον τύπο:
    \[
      y(x)= c_{1}e^{-x}+ c_{2}e^{-4x} 
    \] 
    Βρίσκουμε την παράγωγο της $ y(x) $:
    \[
      y'(x) = -c_{1}e^{-x}-4 c_{2}e^{-4x} 
    \]
    Εφαρμόζοντας τις αρχικές συνθήκες έχουμε:

    \begin{minipage}{0.4\textwidth}
      \begin{align}
        y(0)&=2 \notag \Leftrightarrow  \\ 
        c_{1}e^{0}+ c_{2}e^{0} &= 2 \notag \Leftrightarrow \\ 
        c_{1} + c_{2} &= 2 \label{eq:1} 
      \end{align} 
    \end{minipage} 
    \hspace{\baselineskip} 
    \begin{minipage}{0.4\textwidth}
      \begin{align}
        y'(0)&=3 \notag \Leftrightarrow \\
        -c_{1}e^{0}-4 c_{2}e^{0} &= 3 \notag \Leftrightarrow \\
        -c_{1} - 4 c_{2} &= 3 \label{eq:2}
      \end{align}
    \end{minipage}

    Λύνουμε το σύστημα των σχέσεων~\eqref{eq:1} και~\eqref{eq:2}:
    \[
      \left.
        \begin{tabular}{l}
          $c_{1}+ c_{2}=2 $ \\
          $-c_{1}- 4 c_{2}=3$
        \end{tabular}
      \right\} \Leftrightarrow 
      \left.
        \begin{tabular}{l}
          $c_{1}=2- c_{2} $ \\
          $-c_{1}- 4 c_{2}=3$
        \end{tabular}  
      \right\} \Leftrightarrow 
      \left.
        \begin{tabular}{l}
          $c_{1}=2- c_{2} $ \\
          $-(2- c_{2})- 4 c_{2}=3$
        \end{tabular}  
      \right\} \Leftrightarrow 
      \left.
        \begin{tabular}{l}
          $c_{1}=2- c_{2} $ \\
          $-2- 3c_{2}=3$
        \end{tabular}  
      \right\} \Leftrightarrow 
    \]
    \[
      \left.
        \begin{tabular}{l}
          $c_{1}=2- c_{2} $ \\
          $ -3c_{2}=5$
        \end{tabular}  
      \right\} \Leftrightarrow 
      \left.
        \begin{tabular}{l}
          $c_{1}=2- c_{2} $ \\
          $ c_{2}= -\frac{5}{3} $
        \end{tabular}  
      \right\} \Leftrightarrow 
      \left.
        \begin{tabular}{l}
          $c_{1}=2 + \frac{5}{3}  $ \\
          $ c_{2}= -\frac{5}{3} $
        \end{tabular}  
      \right\} \Leftrightarrow 
      \left.
        \begin{tabular}{l}
          $c_{1}= \frac{11}{3}  $ \\
          $ c_{2}= -\frac{5}{3} $
        \end{tabular}  
      \right\} \Leftrightarrow 
    \] 
    Άρα η λύση του προβλήματος αρχικών τιμών είναι η 
    \[
      \boxed {  y(x)= \frac{11}{3} e^{-x} - \frac{5}{3} e^{-4x} }
    \]




\end{enumerate}

\end{document}
