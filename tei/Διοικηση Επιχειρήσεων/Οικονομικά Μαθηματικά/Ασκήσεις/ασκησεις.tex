\documentclass[a4paper,12pt]{article}


\usepackage[english,greek]{babel}
\usepackage[utf8]{inputenc}

\usepackage{amsmath}
\usepackage{amssymb}
\usepackage{amsthm}
\usepackage{amsfonts}

\usepackage{eurosym}

\usepackage[margin=2cm]{geometry}


\begin{document}


\begin{center}
\fbox{\Large\bfseries{Ασκήσεις στον Απλό Τόκο}}
\end{center}

\begin{enumerate}

\item Αρχικό κεφάλαιο $5000$ \euro\ τοκίζεται σήμερα στην Τράπεζα Δ με απλό τόκο για 5 έτη και επιτόκιο 2,5\%. Ζητείται ο τόκος και η μέλλουσα αξία του κεφαλαίου.

 \hfill Απ: ($I=625$\euro, $FV=5625$\euro)

\item Αρχικό κεφάλαιο $7000$ \euro\ τοκίζεται σήμερα στην Τράπεζα H με απλό τόκο για 6 μήνες και επιτόκιο 3\%. Ζητείται ο τόκος του αρχικού κεφαλαίου.

 \hfill Απ: ($I=105$\euro)
 
  \item Αρχικό κεφάλαιο $80000$ \euro\ τοκίζεται σήμερα στην Τράπεζα Κ με απλό τόκο για 6 μήνες και 15 ημέρες και επιτόκιο 3\%. Ζητείται ο τόκος του αρχικού κεφαλαίου. (εμπορικό έτος) 

 \hfill Απ: ($I=1300$\euro)

 \item Αρχικό κεφάλαιο $20000$ \euro\ τοκίζεται την 10/02/2015 στην Τράπεζα Ε με απλό τόκο και ημερομηνία λήξης την 22/12/2015 και επιτόκιο 2\%. Ζητείται ο τόκος του αρχικού κεφαλαίου. (μεικτό έτος) (10/02/2015: 41η ημέρα, 22/12/2015: 356η ημέρα)

 \hfill Απ: ($I=350$\euro)
 
  \item Αρχικό κεφάλαιο $100000$ \euro\ τοκίζεται την 06/04/2015 στην Τράπεζα Ω με απλό τόκο και ημερομηνία λήξης την 19/08/2015 και επιτόκιο 3\%. Ζητείται ο τόκος του αρχικού κεφαλαίου μετά την επιβολή φόρου $15\%$. (μεικτό έτος) (06/04/2015: 96η ημέρα, 19/08/2015: 231η ημέρα)

 \hfill Απ: ($I=1125$\euro)
 
 \item Καταθέτης διαπραγματεύεται με την Τράπεζα Α την κατάθεση ποσού 80000\euro\ για 4 χρόνια. Η τράπεζα του προσφέρει το πρόγραμμα απλού τόκου ως εξής: Για τα πρώτα 20000\euro\ επιτόκιο 2\%, για το ποσό από 20000 έως 60000 επιτόκιο 3\% και για το υπόλοιπο ποσό επιτόκιο 3,5\%. Ζητείται το καθαρό ποσό των τόκων που θα εισπράξει ο καταθέτης μετά την κράτηση φόρου καταθέσεων 15\%.
 
 \hfill Απ: ($Ι_k=7820$\euro)
 
 \item Ο πελάτης $A$ έχει προυπολογίσει τις καταθέσεις που θα κάνει για το επόμενο έτος (δίσεκτο) στην τράπεζα B. Την 10/03 40000\euro\ , την 25/05 60000\euro\ και την 18/08 40000 \euro\ . Το τραπεζικό επιτόκιο που πέτυχε στη βάση πολιτικού έτους, ήταν 8\%. Στις 24/12, παραμονή Χριστουγέννων θα αποσύρει το ποσό και τους τόκους προκειμένου να αγοράσει μια κατοικία αξίας 142000 \euro\ . Απευθύνεται στο γραφείο σας για να τον συμβουλέψετε αν το κεφάλαιο μαζί με τους τόκους και μετά τη φορολόγησή του κατά 15\% επαρκούν για την αγορά που έχει προγραμματίσει. Να χρησιμοποιήσετε τη μέθοδο του τοκαρίθμου.

\hfill Απ: ($I_k=1851.1$\euro, όχι)

\item Η επιχείρηση ΑΝΕΜΟΣ υπογράφει δανειακή σύμβαση με την τράπεζα Ζ για το ποσό των 200000\euro\ με τη συμφωνία να προκαταβληθούν οι τόκοι του δανείου. Το επιτόκιο της δανειακής σύμβασης είναι 6\% με απλό τόκο και η διάρκεια αποπληρωμής του δανείου 2,6 έτη. Τα έξοδα φακέλου είναι 1\textperthousand . Ποιο ποσό θα εισπράξει η επιχείρηση με την υπογραφή της δανειακής σύμβασης? (εμπορικό έτος)

\hfill Απ: ($K=169830$\euro)

\item Η Τράπεζα Ε δέχεται από τον πελάτη Νικολάου κατάθεση όψης 50000\euro\ , τοκοφόρα για 180 ημέρες, απλού τόκου, με επιτόκιο χορήγησης 2,2\% και κατάθεση προθεσμιακού λογαριασμού 40000\euro\ , τοκοφόρα για 630 ημέρες, με απλό τόκο και προνομιακό επιτόκιο 3,6\% . Ποιο είναι το μέσο επιτόκιο χορηγήσεων για τον πελάτη; (μεικτό έτος).

\hfill Απ: (μέσο επιτόκιο=3,2\%)

\end{enumerate}


\end{document}