\documentclass[a4paper,12pt]{article}
\usepackage{etex}
%%%%%%%%%%%%%%%%%%%%%%%%%%%%%%%%%%%%%%
% Babel language package
\usepackage[english,greek]{babel}
% Inputenc font encoding
\usepackage[utf8]{inputenc}
%%%%%%%%%%%%%%%%%%%%%%%%%%%%%%%%%%%%%%

%%%%% math packages %%%%%%%%%%%%%%%%%%
\usepackage{amsmath}
\usepackage{amssymb}
\usepackage{amsfonts}
\usepackage{amsthm}
\usepackage{proof}

\usepackage{physics}

%%%%%%% symbols packages %%%%%%%%%%%%%%
\usepackage{bm} %for use \bm instead \boldsymbol in math mode 
\usepackage{dsfont}
\usepackage{stmaryrd}
%%%%%%%%%%%%%%%%%%%%%%%%%%%%%%%%%%%%%%%


%%%%%% graphicx %%%%%%%%%%%%%%%%%%%%%%%
\usepackage{graphicx}
\usepackage{color}
%\usepackage{xypic}
\usepackage[all]{xy}
\usepackage{calc}
\usepackage{booktabs}
\usepackage{minibox}
%%%%%%%%%%%%%%%%%%%%%%%%%%%%%%%%%%%%%%%

\usepackage{enumerate}

\usepackage{fancyhdr}
%%%%% header and footer rule %%%%%%%%%
\setlength{\headheight}{14pt}
\renewcommand{\headrulewidth}{0pt}
\renewcommand{\footrulewidth}{0pt}
\fancypagestyle{plain}{\fancyhf{}
\fancyhead{}
\lfoot{}
\rfoot{\small \thepage}}
\fancypagestyle{vangelis}{\fancyhf{}
\rhead{\small \leftmark}
\lhead{\small }
\lfoot{}
\rfoot{\small \thepage}}
%%%%%%%%%%%%%%%%%%%%%%%%%%%%%%%%%%%%%%%

\usepackage{hyperref}
\usepackage{url}
%%%%%%% hyperref settings %%%%%%%%%%%%
\hypersetup{pdfpagemode=UseOutlines,hidelinks,
bookmarksopen=true,
pdfdisplaydoctitle=true,
pdfstartview=Fit,
unicode=true,
pdfpagelayout=OneColumn,
}
%%%%%%%%%%%%%%%%%%%%%%%%%%%%%%%%%%%%%%

\usepackage[space]{grffile}

\usepackage{geometry}
\geometry{left=25.63mm,right=25.63mm,top=36.25mm,bottom=36.25mm,footskip=24.16mm,headsep=24.16mm}

%\usepackage[explicit]{titlesec}
%%%%%% titlesec settings %%%%%%%%%%%%%
%\titleformat{\chapter}[block]{\LARGE\sc\bfseries}{\thechapter.}{1ex}{#1}
%\titlespacing*{\chapter}{0cm}{0cm}{36pt}[0ex]
%\titleformat{\section}[block]{\Large\bfseries}{\thesection.}{1ex}{#1}
%\titlespacing*{\section}{0cm}{34.56pt}{17.28pt}[0ex]
%\titleformat{\subsection}[block]{\large\bfseries{\thesubsection.}{1ex}{#1}
%\titlespacing*{\subsection}{0pt}{28.80pt}{14.40pt}[0ex]
%%%%%%%%%%%%%%%%%%%%%%%%%%%%%%%%%%%%%%

%%%%%%%%% My Theorems %%%%%%%%%%%%%%%%%%
\newtheorem{thm}{Θεώρημα}[section]
\newtheorem{cor}[thm]{Πόρισμα}
\newtheorem{lem}[thm]{λήμμα}
\theoremstyle{definition}
\newtheorem{dfn}{Ορισμός}[section]
\newtheorem{dfns}[dfn]{Ορισμοί}
\theoremstyle{remark}
\newtheorem{remark}{Παρατήρηση}[section]
\newtheorem{remarks}[remark]{Παρατηρήσεις}
%%%%%%%%%%%%%%%%%%%%%%%%%%%%%%%%%%%%%%%




\newcommand{\vect}[2]{(#1_1,\ldots, #1_#2)}
%%%%%%% nesting newcommands $$$$$$$$$$$$$$$$$$$
\newcommand{\function}[1]{\newcommand{\nvec}[2]{#1(##1_1,\ldots, ##1_##2)}}

\newcommand{\linode}[2]{#1_n(x)#2^{(n)}+#1_{n-1}(x)#2^{(n-1)}+\cdots +#1_0(x)#2=g(x)}

\newcommand{\vecoffun}[3]{#1_0(#2),\ldots ,#1_#3(#2)}

\newcommand{\mysum}[1]{\sum_{n=#1}^{\infty}


\everymath{\displaystyle}


\begin{document}

%\thispagestyle{empty}

\begin{center}
  \fbox{\Large\bfseries Ασκήσεις ΣΔΕ $1$ης Τάξης}
\end{center}

\vspace{\baselineskip}

\begin{enumerate}
  \item Να λυθούν οι παρακάτω σδε 1ης τάξης.

\begin{description}
  \setlength{\itemsep}{\baselineskip}
  \item[(Θέμα Ιουν 2018)]
  \begin{enumerate}[i)]
    \item
      Επιλύστε ως προς την ένταση του ρεύματος $i(t)$, ένα απλό κύκλωμα $RL$ που λειτουργεί υπό ηλεκτρεγερτική δύναμη $E=E_{0}\sin 2t$, αν $i(0)=0$.
    \item Σχεδιάστε γραφικά την $i(t)$, αν $E=$σταθερά, με την ίδια αρχική συνθήκη.
  \end{enumerate}

\item[(Θέμα Σεπ 2016)]
   Δίνεται το σύστημα $\begin{pmatrix}
  x'(t) \\ y'(t)
\end{pmatrix}=\begin{pmatrix*}[r]
    -2 & -4 \\
    -1 & 1
\end{pmatrix*}\cdot \begin{pmatrix*}
  x(t)\\y(t)
\end{pmatrix*}+\begin{pmatrix*}
4t+1\\
\frac{3t^{2}}{2}
\end{pmatrix*}$
\begin{enumerate}[i)]
\item Βρείτε την ευστάθεια και το είδος του σημείου ισορροπίας $(0,0)$.
\item Υπολογίστε τον θεμελιώδη πίνακα του ομογενούς συστήματος.
\item Βρείτε τη γενική λύση του συστήματος με τη μέθοδο μεταβολής των παραμέτρων.
\end{enumerate}

\item[(Θέμα Ιουν 2016)]
   Δίνεται το σύστημα $\begin{pmatrix}
  x'(t) \\ y'(t)
\end{pmatrix}=\begin{pmatrix*}[r]
    -3 & 4 \\
    -2 & 3
\end{pmatrix*}\cdot \begin{pmatrix*}
  x(t)\\y(t)
\end{pmatrix*}+\begin{pmatrix*}
\sin t\\
t
\end{pmatrix*}$

Να βρείτε τη γενική του λύση.
\item[(Θέμα Σεπ 2016)] Με τη μέθοδο των ιδιοτιμών και ιδιοδιανυσμάτων, να βρεθεί ο θεμελιώδης πίνακας λύσεων του παρακάτω γραμμικού συστήματος διαφορικών εξισώσεων:
\[
\begin{pmatrix*}
  {x_{1}}'(t)\\ {x_{2}}'(t)\\ {x_{3}}'(t)
\end{pmatrix*}=\begin{pmatrix*}[r]
  -1 & 1 & -1 \\
  -2 & 0 & 2 \\
  -1 & 3 & -1
\end{pmatrix*}\cdot \begin{pmatrix*}[c]
  x_{1}(t) \\ x_{2}(t)\\x_{3}(t)
\end{pmatrix*}
\]

\item[(Θέμα )] Δίνεται το μη γραμμικό σύστημα σδε:
\[
\left.
\begin{aligned}
  x'&=\phantom{-}y+x(x^{2}+y^{2}) \\
  y'&=-x+y(x^{2}+y^{2})
\end{aligned}
\right\} \quad x=x(t), y=y(t)
\]
\begin{enumerate}[i)]
  \item Να βρείτε και να χαρακτηρίσετε τα σημεία ισορροπίας $(x^{0},y^{0})$, δικαιολογώντας την απάντησή σας.
  \item Να δείξετε ότι το $(0,0)$ είναι ασταθής εστία.
\end{enumerate}

\item[(Θέμα Σεπ 2017)]
Δίνεται το μη γραμμικό σύστημα σδε:
\[
\left.
\begin{aligned}
  x'&=e^{x+y}-y \\
  y'&=-x+xy
\end{aligned}
\right\} \quad x=x(t), y=y(t)
\]

\begin{enumerate}[i)]
  \item Να βρείτε και να χαρακτηρίσετε τα σημεία ισορροπίας $(x^{0},y^{0})$, δικαιολογώντας την απάντησή σας.
  \item Σε μια μικρή περιοχή γύρω από το κάθε σημείο ισορροπίας να σχεδιασετε ποιοτικά το διάγραμμα φάσεων του συστήματος.
\end{enumerate}


\item[(Θέμα Ιουν 2017)]
Δίνεται το μη γραμμικό σύστημα σδε:
\[
\left.
\begin{aligned}
  x'&=x+y-3 \\
  y'&=2x-x^{2}
\end{aligned}
\right\} \quad x=x(t), y=y(t)
\]

\begin{enumerate}[i)]
  \item Να βρείτε και να χαρακτηρίσετε τα σημεία ισορροπίας $(x^{0},y^{0})$, δικαιολογώντας την απάντησή σας.
  \item Σε μια μικρή περιοχή γύρω από το κάθε σημείο ισορροπίας να σχεδιασετε ποιοτικά το διάγραμμα φάσεων του συστήματος.
\end{enumerate}

\item[(Θέμα Ιουν 2016)]
Δίνεται το μη γραμμικό σύστημα σδε:
\[
\left.
\begin{aligned}
  x'&=-x+2xy \\
  y'&=y-x^{2}-y^{2}
\end{aligned}
\right\} \quad x=x(t), y=y(t)
\]

\begin{enumerate}[i)]
  \item Να βρείτε τα σημεία ισορροπίας $(x^{0},y^{0})$.
  \item Να χαρακτηρίσετε τα σημεία ισορροπίας $(x^{0},y^{0})$, $x^{0}\geq 0$, $y^{0}\geq 0$, δικαιολογώντας πλήρως την απάντησή σας.
  \item Σε περίπτωση εστιακού σημείου ή κέντρου, να υπολογίσετε τοπικά την λύση του συστήματος.
\end{enumerate}

\end{description}

\end{enumerate}

\end{document}
