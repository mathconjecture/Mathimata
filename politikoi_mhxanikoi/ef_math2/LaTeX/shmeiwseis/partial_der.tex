\input{preamble.tex}
\newcommand{\vect}[2]{(#1_1,\ldots, #1_#2)}
%%%%%%% nesting newcommands $$$$$$$$$$$$$$$$$$$
\newcommand{\function}[1]{\newcommand{\nvec}[2]{#1(##1_1,\ldots, ##1_##2)}}

\newcommand{\linode}[2]{#1_n(x)#2^{(n)}+#1_{n-1}(x)#2^{(n-1)}+\cdots +#1_0(x)#2=g(x)}

\newcommand{\vecoffun}[3]{#1_0(#2),\ldots ,#1_#3(#2)}

\newcommand{\suma}{\sum_{n=0}^{\infty}a_n x^n}

\newcommand{\sumb}{\sum_{n=1}^{\infty}a_n n x^{n-1}}

\newcommand{\sumc}{\sum_{n=2}^{\infty}a_n n (n-1) x^{n-2}}

\newcommand{\varsum}[2]{\sum_{n=#1}^{#2}}
\input{tikz.tex}
\input{myboxes.tex}

\geometry{top=2.5cm,left=1cm,right=1cm,bottom=2.5cm}

\usepackage[all]{xy}
\everymath{\displaystyle}
\pagestyle{vangelis}

\begin{document}


\chapter{Μερική Παραγώγιση}

\section{Συναρτήσεις Μερικών Παραγώγων}

Έστω $ f(x,y) $ συνάρτηση δύο μεταβλητών. 
\begin{myitemize}
  \item Η \textcolor{Col1}{μερική παράγωγος της $f$ ως προς $x$} 
    υπολογίζεται παραγωγίζοντας την συνάρτηση $ f(x,y) $ ως προς $x$, 
    θεωρώντας το $y$ σταθερό. 
  \item Η \textcolor{Col1}{μερική παράγωγος της $f$} ως προς $y$ 
    υπολογίζεται παραγωγίζοντας την συνάρτηση $ f(x,y) $ ως προς $y$, 
    θεωρώντας το $x$ σταθερό. 
\end{myitemize}

\begin{rem}
  Γενικότερα η \textcolor{Col1}{μερική παράγωγος της $f$ ως προς $ x_{i} $} 
  υπολογίζεται παραγωγίζοντας τη συνάρτηση $ f(x_{1}, \ldots, x_{n}) $ ως προς 
  $ x_{i} $, θεωρώντας \textbf{όλες} τις υπόλοιπες μεταβλητές σταθερές.
\end{rem}

\section{Συμβολισμός}

Συμβολισμοί για τις μερικές παραγώγους της $f(x,y)$ ως 
προς $x$ και ως προς $y$ είναι:
\begin{align*}
  \eval{\pdv{f}{x} }_{(x_{0}, y_{0})} = \pdv{f(x_{0}, y_{0})}{x} = 
  f_{x}(x_{0}, y_{0}) = f'_{x}(x_{0}, y_{0} ) \quad \text{και} \quad
  \eval{\pdv{f}{y} }_{(x_{0}, y_{0})} = \pdv{f(x_{0}, y_{0})}{y} = 
  f_{y}(x_{0}, y_{0}) = f'_{y}(x_{0}, y_{0} ) 
\end{align*} 

\section{Κανόνες Παραγώγισης}

Έστω $ f(x,y) $ και $ g(x,y) $ συναρτήσεις δύο μεταβλητών, που ορίζονται στο 
ανοικτό υποσύνολο $ A $ του $ \mathbb{R}^{2} $, και έστω ότι οι 
μερικές παράγωγοι 
$ \pdv{f}{x} $ και $ \pdv{g}{x} $ υπάρχουν για κάθε $ (x,y) \in A $, τότε:

\vspace{\baselineskip}

\twocolumnsides{
  \begin{myitemize}
    \item $ \pdv{x}(f+g) = \pdv{f}{x} + \pdv{g}{x} $
    \item $ \pdv{x}(af) = a \pdv{f}{x} $ 
  \end{myitemize}}{
  \begin{myitemize}
    \item $ \pdv{x}(f\cdot g) = \pdv{f}{x} \cdot g + f \cdot \pdv{g}{x} $
    \item $ \pdv{x}(\frac{f}{g}) = \frac{\pdv{f}{x} \cdot g - f \cdot 
      \pdv{g}{x}}{g^{2}} $
\end{myitemize}}

\begin{examples}
\item {}
  \begin{enumerate}
    \item Έστω $ f(x,y)=x^{2}y^{3}+4xy^{2}+4y+5 $. Να 
      υπολογιστούν οι μερικές παράγωγοι $ f_{x} $ και 
      $ f_{y} $.
      \begin{solution}
        \begin{align*}
          f_{x} &= (x^{2}y^{3}+4xy^{2}+4y+5)_{x} =
          (x^{2}y^{3})_{x}+(4xy^{2})_{x}+(4y)_{x}+(5)_{x} = 2xy^{3} + 4y^{2}
          \intertext{και}
          f_{y}&=(x^{2}y^{3}+4xy^{2}+4y+5)_{y} = 
          (x^{2}y^{3})_{y}+(4xy^{2})_{y}+(4y)_{y}+(5)_{y} = 3x^{2}y^{2} + 
          8xy + 4
        \end{align*} 
      \end{solution}
    \item Έστω $ f(x,y)=2x^{2}y+3 \cos{3y} +1 $. Να υπολογιστούν οι 
      μερικές παράγωγοι $ f_{x}$ και $ f_{y} $.
      \begin{solution}
        \[
          f_{x}=4xy \quad \text{και} \quad f_{y}=2x^{2}-3 \sin{3y} (3y)_{y} 
          = 2x^{2}-9 \sin{3y}
        \] 
      \end{solution}
    \item Έστω $ f(x,y,z)=x^{2}yz - y \cos{(xy)} $. Να υπολογιστούν οι 
      μερικές παράγωγοι $ f_{x}, f_{y}, f_{z} $. 
      \begin{solution}
      \item {}
        \begin{align*}
          f_{x}&=2xyz- \cos{(xy)}(xy)_{x} = 2xyz-y \cos{xy} \\
          f_{y}&=x^{2}z- \cos{xy}(xy)_{y}=x^{2}z - x \cos{xy} \\
          f_{z}&=x^{2}z
        \end{align*}
      \end{solution}
  \end{enumerate}
\end{examples}

\section{Μερικές Παράγωγοι Ανώτερης Τάξης}

\begin{example}
\item {}
  Έστω $ f(x,y,z) = 3x^{2}y^{2} + xy^{3} + 3x +1 $. 
  Να υπολογιστούν οι μερικές παράγωγοι 1ης και 2ης τάξης.
  \begin{solution}
  \item {} 
    \begin{align*}
      f_{x} &= 6xy^{2}+y^{3}+3 \quad \text{και} \quad 
      f_{y} = 6x^{2}y+3xy^{2} \\
      f_{xx} &= (f_{x})_{x} = (6xy^{2}+y^{3}+3)_{x} =
      6y^{2} \\
      f_{yy} &= (f_{y})_{y} = (6x^{2}y+3xy^{2})_{y} = 
      6x^{2}+6xy \\
      f_{xy} &= (f_{x})_{y} = (6xy^{2}+y^{3}+3)_{y} = 
      12xy = 3y^{2} \\
      f_{yx} &= (f_{y})_{x} = (6x^{2}y+3xy^{2})_{x} = 
      12xy+3y^{2} \; \; \,
    \end{align*}
  \end{solution}
\end{example}

\begin{rem}
\item {}
  Για τις μικτές παραγώγους $ f_{xy} $ και $ f_{yx} $ 
  ισχύει:
  \begin{align*}
    \pdv[2]{f}{x}{y} = \pdv{}{x} \left(\pdv{f}{y}\right) = \pdv{x} (f_{y}) 
    = (f_{y})_{x} = f_{yx}
    \intertext{και}
    \pdv[2]{f}{y}{x} = \pdv{}{y} \left(\pdv{f}{x}\right) = \pdv{y} (f_{x}) 
    = (f_{x})_{y} = f_{xy}
  \end{align*} 
\end{rem}

\begin{rem}
\item {}
  Οι πολυωνυμικές συναρτήσεις δύο (ή περισσότερων) μεταβλητών, 
  έχουν συνεχείς μερικές παραγώγους σε κάθε σημείο του $ \mathbb{R}^{2} $ 
  (ή $\mathbb{R}^{n}$).
  Οι λοιπές στοιχειώδεις συναρτήσεις $ \sin{f(x,y)}, \cos{f(x,y)}, a^{f(x,y)}, 
  \ln{f(x,y)} $ κ.λ.π. όπου $ f(x,y) $ πολυωνυμική συνάρτηση, έχουν 
  συνεχείς μερικές παραγώγους σε κάθε σημείο του πεδίου ορισμού τους.
  Για αυτές τις συναρτήσεις ισχύει $ f_{xy}=f_{yx} $.
\end{rem}

\begin{thm}[Schwartz]
  Αν για τη συνάρτηση $ f(x,y) $ υπάρχουν οι μερικές παράγωγοι 
  $ f_{xy} $ και $ f_{yx} $ και είναι συνεχείς σε μια περιοχή του σημείου 
  $ (x_{0}, y_{0}) $, τότε $ f_{xy}=f_{yx} $ στην περιοχή αυτή.
\end{thm}

\section{Μερική Ολοκλήρωση}

\begin{rem}
\item {}
  Αν $ f_{x}(x,y) = g(x,y)$ και $ f_{y}(x,y)=h(x,y) $ τότε ισχύει:
  \begin{align*}
    f(x,y) = \int g(x,y) \,{dx} + c(y) \quad \text{και} \quad f(x,y) = 
    \int h(x,y) \,{dy} + c(x) 
  \end{align*} 
\end{rem}

\begin{example}
\item {}
  Έστω $ f(x,y)$ με $ f_{x}=e^{x+y} $ και $ f(0,y)=e^{y} $. 
  Να βρεθεί ο τύπος της $f$.
  \begin{solution}
    \begin{align*}
      f(x,y) = \int e^{x+y} \,{dx} = e^{x+y} + c(y) \; \tikzmark{a} \\ 
      f(0,y) = e^{y} \Leftrightarrow e^{y}+ c(y) = e^{y} \Rightarrow c(y) = 0 
      \; \tikzmark{b}
    \end{align*}
    \mybrace{a}{b}[ $f(x,y) = e^{x+y}$ ]
  \end{solution}
\end{example}

\begin{rem}
  Όταν ολοκληρώνουμε μια συνάρτηση πολλών μεταβλητών, ως προς κάποια από τις 
  μεταβλητές, τότε η σταθερά ολοκλήρωσης είναι συνάρτηση των υπολοίπων μεταβλητών, 
  οι οποίες θεωρούνται σταθερές κατά την ολοκλήρωση.
\end{rem}

\section{Ολικό Διαφορικό}

\begin{dfn}
  Έστω η συνάρτηση $ f(x,y) $. Τα \textcolor{Col1}{ολικά διαφορικά 1ης και 
  2ης τάξης} της συνάρτηση $f$ συμβολίζονται με $ df $ και $ d^{2}f $, αντίστοιχα 
  και ισχύει:
  \[
    \boxed{df = f_{x}dx + f_{y}dy} \quad \text{και} \quad 
    \boxed{d^{2}f = f_{xx}dx^{2}+2f_{xy}dxdy+f_{yy}dy^{2}}
  \] 
  Στην περίπτωση όπου $ f= f(x_{1}, x_{2}, \ldots, x_{n}) $, το ολικό 
  διαφορικό 1ης τάξης γίνεται: 
  \[
    df = f_{x_{1}}d{x_{1}} + f_{x_{2}}d{x_{2}} + \cdots f_{x_{n}} dx_{n}
  \]
\end{dfn}


\begin{example}
  Να βρείτε το ολικό διαφορικό της συνάρτησης $ f(x,y) = xye^{x+2y} $ 
  \begin{solution}
  \item {}
    Το ολικό διαφορικό δίνεται από τη σχέση $ df = f_{x} dx + f_{y} dy $.  
    Για τις μερικές παραγώγους έχουμε ότι: 
    \[
      f_{x} = ye^{x+2y}+xye^{x+2y} \quad \text{και} \quad f_{y} = xe^{x+2y} +
      2xye^{x+2y}
    \] 
    Επομένως
    \[
      df = y(1+x)e^{x+2y} dx + x(1+2y)e^{x+2y}dy
    \]
  \end{solution}
\end{example}

\section{Εφαρμογές του Διαφορικού}

\begin{example}
  Να υπολογίσετε το $ \Delta f $ και το $ df $ στο σημείο $ (x,y) = (1,1) $, 
  για τη συνάρτηση $ f(x,y) = x^{3}+y^{2} $ και για 
  \begin{enumerate}[i)]
    \item $ \Delta x = 0,1, \; \Delta y = 0,1 $
    \item $ \Delta x = 0,01, \; \Delta y = 0,01 $
  \end{enumerate}
  Τι παρατηρείτε?
\end{example}
\begin{solution}
\item {}
  \begin{enumerate}[i)]
    \item 
      \begin{align*} 
        \Delta f &= f(x+ \Delta x, y + \Delta y) - f(x,y) = f(1+0.1,1+0.1) - 
        f(1,1) = f(1.1,1.1) - f(1,1) \\ 
                 &= [(1.1)^{3}+(1.1)^{2}] - (1^{3}+1^{2}) = 0.541 \\
        df &= f_{x}dx + f_{y}dy  = 3x^{2} dx + 2y dy \Rightarrow df (1,1) = 
        3\cdot 1^{2} \cdot 0.1 + 2 \cdot 1 \cdot 0.1 = 0.5 
      \end{align*}
    \item 
      \begin{align*} 
        \Delta f &= f(x+ \Delta x, y + \Delta y) - f(x,y) = f(1+0.01,1+0.01) - 
        f(1,1) = f(1.01,1.01) - f(1,1) \\ 
                 &= [(1.01)^{3}+(1.01)^{2}] - (1^{3}+1^{2}) = 0.0504 \\
        df &= f_{x}dx + f_{y}dy  = 3x^{2} dx + 2y dy \Rightarrow df (1,1) = 
        3\cdot 1^{2} \cdot 0.01 + 2 \cdot 1 \cdot 0.01 = 0.05 
      \end{align*}
  \end{enumerate}
  Παρατηρούμε ότι στην 1η περίπτωση, έχουμε
  $ \Delta f - df \approx 0.04 $, ενώ στη 2η περίπτωση έχουμε $ \Delta f - df \approx 
  0,0004 $, δηλαδή στη 2η περίπτωση, όπου τα $ \Delta x $ και $ \Delta y $ επιλέχθηκαν 
  να είναι μικρότερα, το διαφορικό προσεγγίζει με μεγαλύτερη ακρίβεια τη διαφορά 
  $ \Delta f $.
\end{solution}

\begin{example}
  Κατά τη μέτρηση των στοιχείων μιας τριγωνικής επιφάνειας, βρέθηκαν οι πλευρές να 
  έχουν μήκος \SI{100}{cm} και \SI{125}{cm} αντίστοιχα, ενώ η περιεχόμενη γωνία βρέθηκε 
  να είναι \SI{60}{\degree}. Αν το πιθανό σφάλμα κατά την μέτρηση των πλευρών είναι 
  $ \SI{0.2}{cm} $ και για τις γωνίες είναι $ \SI{1}{\degree} $, να υπολογίσετε κατά 
  προσέγγιση το σφάλμα στο εμβαδό της επιφάνειας.
\end{example}

\begin{solution}
\item {}

  \twocolumnsidelc{
    \begin{tikzpicture}[scale=0.7]
      \node  (B) at (0, 0) {};
      \node  (A) at (6, 0) {};
      \node  (C) at (2, 2) {};
      \node  (3) at (2, 0) {};
      \draw (B.center) to node[left=3pt]{$x$} (C.center);
      \draw (C.center) to (A.center);
      \draw (B.center) to node[below]{$y$} (A.center) 
        pic[draw,fill=magenta!25,"$\phi$",angle eccentricity=1.5]{angle} ;
      \draw [dashed, in=90, out=-90] (C.center) to node[right]{$z$} (3.center);
    \end{tikzpicture}
    }{
    Έχουμε ότι $ E = \frac{y \cdot z}{2} $. Όμως από το ορθογώνιο τρίγωνο έχουμε ότι 
    $ z = x \sin{\phi} $, άρα 
    \[
      E(x,y,\phi) = \frac{y \cdot x \cdot \sin{\phi}}{2} 
    \] 
  }
  Παρατηρούμε ότι $ \Delta x = dx = 0.2, \; \Delta y = dy = 0.2 $ και 
  $ \Delta \phi = d\phi = \SI{1}{\degree} \frac{\pi}{180} $. Οπότε,ισχύει ότι
  \begin{equation*}
    \Delta E \approx dE = E_{x} dx + E_{y} dy + E_{\phi} d\phi = \frac{y \cdot
    \sin{\phi}}{2} dx + \frac{x \cdot \sin{\phi}}{2} dy + 
    \frac{y \cdot x \cdot \cos{\phi}}{2} d\phi
  \end{equation*} 
  και πιο συγκεκριμένα ισχύει ότι 
  \begin{align*}
    \Delta E(100,125,60) \approx dE(100,125,60) 
  &= E_{x} dx + E_{y} dy + E_{\phi} d\phi \\
  &= E_{x}(100,125,60) dx + E_{y}(100,125,60) dy + E_{\phi}(100,125,60) d\phi \\
  &= \frac{1}{2} \left(125 \cdot \sin{\frac{\pi}{3}} \cdot 0.2 + 100 \cdot
    \sin{\frac{\pi}{3}} \cdot 0.2 + 100 \cdot 125 \cdot \cos{\frac{\pi}{3}} \cdot
  \frac{\pi}{180}\right) \\
  &= \SI{74.03}{cm^{2}}
  \end{align*}
  Άρα το σφάλμα μέτρησης του εμβαδού της τριγωνικής επιφάνειας είναι 
  $ \SI{74.03}{cm^{2}} $
\end{solution}

\begin{example}
  Κατά την εκτίμηση των συντεταγμένων ενός σημείου $ P(x,y) $ γίνονται σφάλματα 
  $ dx $ και $ dy $ αντίστοιχα. Να υπολογίσετε την επίδραση αυτών των σφαλμάτων 
  στη γωνία $ \theta $ και να γίνει αριθμητική εφαρμογή για $ x=1 $, $ y=2 $ και για 
  $ dx = 0.01 $ και $ dy =-0.02 $.
\end{example}
\begin{solution}
\item {}

  \twocolumnsidelc{
    \begin{tikzpicture}[scale=0.7]
      \node (0) at (0, 0) {};
      \node (1) at (4.5, 0) {};
      \node (2) at (0, 4) {};
      \node (3) at (3, 2) {};
      \draw [fill] (3,2) node[right] {$P(x,y)$} circle (2pt);
      \node (4) at (2, 3) {};
      \draw [fill] (2,3) node[right] {$P'(x+dx, y+dy)$} circle (2pt);
      \node (5) at (2, 0) {};
      \node (6) at (3, 0) {};
      \draw (0.center) to (3.center);
      \draw (0.center) to (4.center);
      \draw [dashed] (4.center) to (5.center);
      \draw [dashed] (3.center) to (6.center);
      \draw [-stealth] (0.center) to (1.center);
      \draw [-stealth] (0.center) to (2.center);
      \draw pic[draw,fill=magenta!25,"$\theta$",angle eccentricity=1.5]{angle=1--0--3};
    \end{tikzpicture}}{
    Έστω οι συντεταγμένες του σημείου $ P'(x+ dx, y + dy) $, όπως φαίνονται στο σχήμα.
    Έχουμε ότι 
    \[
      \tan{\theta} = \frac{y}{x} \Rightarrow \theta = \arctan{\frac{y}{x}}
    \] 
    Αν η γωνία μετά το σφάλμα της εκτίμησης είναι $ \theta'$, έχουμε ότι 
    \[
      \theta ' = \arctan{\frac{y+ dy}{x + dx}} 
  \]} 
  Παρατηρούμε ότι $ \Delta x = dx = 0.01 $, $ \Delta y = dy = -0.02 $. Οπότε ισχύει ότι 
  \[
    \Delta \theta \approx d \theta = \theta _{x} dx + \theta _{y} dy =
    \frac{-y}{x^{2}+y^{2}} dx + \frac{x}{x^{2}+y^{2}} dy = \frac{xdy-ydx}{x^{2}+y^{2}}
  \]
  και μάλιστα υπολογίζουμε την απόλυτη τιμή και έχουμε
  \[
    \abs{\Delta \theta} \approx \frac{\abs{xdy-ydx}}{x^{2}+y^{2}} \leq \frac{\abs{x dy}
    + \abs{y dx}}{x^{2}+y^{2}} 
  \]
  και πιο συγκεκριμένα, για το σημείο $ P(1,2) $ έχουμε
  \[
    \abs{\Delta \theta(1,2)} \leq \frac{\abs{1(-0.02)}+\abs{2\cdot 0.01}}{1^{2}+2^{2}} =
    \frac{0.04}{5} = 0.008
  \]
\end{solution}


\section{Τέλειο Διαφορικό}

\begin{dfn}
  Θεωρούμε τις συναρτήσεις $ P(x,y) $ και $ Q(x,y) $ με πεδίο ορισμού $ A \subseteq
  \mathbb{R}^{2} $.
  Η παράσταση $ P(x,y) dx + Q(x,y) dy $ λέγεται \textcolor{Col2}{τέλειο 
  διαφορικό} αν υπάρχει συνάρτηση $ f(x,y) $ με πεδίο ορισμού το $A$, ώστε 
  \begin{gather*}
    df = P(x,y)dx + Q(x,y)dy \Leftrightarrow \pdv{f}{x} dx + \pdv{f}{y} dy = 
    P(x,y)dx + Q(x,y)dy \Leftrightarrow \\
    \boxed{\pdv{f}{x} = P(x,y) \quad \text{και} \quad \pdv{f}{y} = Q(x,y)}
  \end{gather*}
\end{dfn}


\begin{prop}
  Αν οι  $ P(x,y) $  και  $ Q(x,y) $  είναι συνεχείς συναρτήσεις και έχουν συνεχείς 
  παραγώγους πρώτης τάξης, σε μια ορθογώνια περιοχή $A$ του $ \mathbb{R}^{2} $,  
  τότε η  παράσταση  $ P(x,y)dx + Q(x,y)dy $ είναι τέλειο διαφορικό αν 
  \[
    \boxed{\pdv{p}{y} = \pdv{q}{x}} \quad \forall (x,y) \in A
  \]
\end{prop}

\begin{dfn}
  Η παράσταση  $ P(x,y,z)dx + Q(x,y,z)dy + R(x,y,z)dz $ είναι τέλειο διαφορικό 
  αν υπάρχει συνάρτηση  $ f(x,y,z) $  τέτοια ώστε  $ df = P(x,y,z)dx + Q(x,y,z)dy 
  + R(x,y,z)dz $.  Τότε ισχύουν οι παρακάτω σχέσεις:
  \[
    \boxed{\pdv{f}{x} = P(x,y,z) \quad \text{και} \quad \pdv{f}{y} = Q(x,y,z) 
    \quad \text{και} \quad \pdv{f}{z} = R(x,y,z)} 
  \] 
\end{dfn}

\begin{prop}
  Αν οι  $ P(x,y,z) $, $ Q(x,y,z) $  και  $ R(x,y,z) $ είναι συνεχείς συναρτήσεις 
  και έχουν συνεχείς παραγώγους πρώτης τάξης, σε μια ορθογώνια περιοχή Α του 
  $ \mathbb{R}^{3} $ τότε η  παράσταση 
  $ P(x,y,z)dx + Q(x,y,z)dy + R(x,y,z)dz $   είναι τέλειο διαφορικό αν 
  \[
    \boxed{\pdv{P}{y} = \pdv{Q}{x}} \quad \text{και} \quad \boxed{\pdv{Q}{z} = 
    \pdv{R}{y}} \quad \text{και} \quad  \boxed{\pdv{P}{z} = \pdv{R}{x}} 
    \quad \forall (x,y,z) \in A 
  \] 
\end{prop}

\begin{rem}\label{olokl}
  Οι συναρτήσεις  $ f(x,y) $  και  $ f(x,y,z) $ υπολογίζονται επίσης από τις 
  παρακάτω σχέσεις:
  \begin{align*}
    f(x,y) &= \int_{x_{0}}^{x} P(t,y) \,{dt} + \int_{y_{0}}^{y} Q(x_{0},t) \,{dt} \\
    f(x,y,z) &= \int_{x_{0}}^{x} P(t,y,z) \,{dt} + \int_{y_{0}}^{y} Q(x_{0},t,z) 
    \,{dt} + \int _{z_{0}}^{z} R(x_{0},y_{0},t) \,{dt}  
  \end{align*}
  όπου τα $ x_{0} $, $ y_{0} $  και  $ z_{0} $ επιλέγονται \textbf{αυθαίρετα} στο πεδίο 
  ορισμού των  $ P $, $ Q $  και  $ R $.
\end{rem}

\begin{rem}
  Η συνάρτηση $ f(x,y) $ ή η συνάρτηση $ f(x,y,z) $ του ορισμού του τέλειου διαφορικού
  λέγεται \textcolor{Col2}{συνάρτηση δυναμικού}.
\end{rem}

\begin{example}
  Να εξετάσετε αν η παράσταση $ \left(1+x- \frac{2}{y}\right)dx + 
  \left(1+ \frac{2x}{y^{2}} \right)dy $ είναι τέλειο διαφορικό και αν ναι, να 
  υπολογίσετε τη συνάρτηση δυναμικού.
\end{example}
\begin{solution}
  Ελέγχουμε με το κριτήριο:
  \[ 
    \pdv{P}{y} = \frac{2}{y^{2}} = \pdv{Q}{x} 
  \]
  Άρα η παράσταση είναι τέλειο διαφορικό. Επομένως υπάρχει 
  συνάρτηση, $ f(x,y) $ τέτοια ώστε: 
  \begin{align}
    \pdv{f}{x} &= 1 + x - \frac{2}{y} \label{fx1} \\
    \pdv{f}{y} &= 1+ \frac{2x}{y^{2}} \label{fy1}
  \end{align}
  Ολοκληρώνουμε μερικώς ως προς $x$ την~\eqref{fx1} και έχουμε
  \[
    f(x,y) = \int \left(1+x- \frac{2}{y}\right) \,{dx} = x + 
    \frac{x^{2}}{2} - \frac{2x}{y} + c(y) 
  \] 
  Άρα  
  \begin{equation}
    f(x,y) = x + \frac{x^{2}}{2} - \frac{2x}{y} + c(y) \label{fxy}
  \end{equation}
  Στη συνέχεια παραγωγίζουμε μερικώς ως προς $y$ τη συνάρτηση $ f(x,y) $ που μόλις 
  βρήκαμε και έχουμε:
  \begin{equation}
    \pdv{f}{y} = \frac{2x}{y^{2}} + c'(y) \label{ffy}
  \end{equation} 
  Στη συνέχεια εξισώνουμε την~\eqref{ffy} με την~\eqref{fy1} και προκύπτει
  \[
    c'(y) = 1 \Rightarrow c(y) = y + k 
  \] 
  Άρα, τελικά η συνάρτηση δυναμικού είναι 
  \[
    f(x,y) = x + \frac{x^{2}}{2} - \frac{2x}{y} + y + k 
  \] 
\end{solution}

\begin{example}
  Να εξετάσετε αν η παράσταση $ (3x^{2}+3y-1)dx + (z^{2}+3x)dy + (2yz+1)dz $ είναι 
  τέλειο διαφορικό και αν ναι, να υπολογίσετε τη συνάρτηση δυναμικού.
\end{example}
\begin{solution}
  Ελέγχουμε με το κριτήριο:
  \[
    \pdv{P}{y} = 3 = \pdv{Q}{x} \quad \text{και} \quad \pdv{Q}{z} = 2z = \pdv{R}{y}
    \quad \text{και} \quad \pdv{P}{z} = 0 = \pdv{R}{x}
  \] 
  Άρα η παράσταση είναι τέλειο διαφορικό. Για να βρούμε τη συνάρτηση δυναμικού, έχουμε
  \begin{description}
    \item [A᾽ Τρόπος: (Με Τύπο)]
      Θα χρησιμοποιήσουμε τον τύπο της παρατήρησης~\ref{olokl} για να υπολογίσουμε 
      τη συνάρτηση δυναμικού. 
      \begin{align*}
        f(x,y,z) &= \int _{x_{0}}^{x} P(t,y,z) \,{dt} + \int _{y_{0}}^{y} Q(x_{0},t,z) 
        \,{dt} + \int _{z_{0}}^{z} R(x_{0}, y_{0}, t) \,{dt} \\
                 &= \int _{0}^{x} (3t^{2}+3y-1) \,{dt} + \int _{0}^{y} (z^{2}+3\cdot 0) 
                 \,{dt} + \int _{0}^{z} (2\cdot 0\cdot t + 1) \,{dt} \\ 
                 &= \int _{0}^{x} (3t^{2}+3y-1) \,{dt} + \int _{0}^{y} z^{2} \,{dt} + 
                 \int _{0}^{z} 1 \,{dt} \\
                 &= \left[t^{3}+3yt-t\right]_{0}^{x} + \left[z^{2}t\right]_{0}^{y} + 
                 \bigl[t\bigr]_{0}^{z} = x^{3}+3xy-x + z^{2}y+z
      \end{align*}
    \item [B᾽ Τρόπος: (Με Ολοκλήρωση)] Αφού η παράσταση είναι τέλειο διαφορικό,  
      υπάρχει συνάρτηση, $ f(x,y,z) $ τέτοια ώστε: 
      \begin{align}
        \pdv{f}{x} &= 3x^{2}+3y-1 \label{fx11} \\
        \pdv{f}{y} &= z^{2}+3x \label{fy11} \\
        \pdv{f}{z} &= 2yz+1 \label{fz11}
      \end{align} 
      Ολοκληρώνουμε μερικώς την~\eqref{fx11} και έχουμε
      \begin{equation*}
        f(x,y,z) = \int (3x^{2}+3y-1) \,{dx} = x^{3} + 3xy -x + c(y,z) 
      \end{equation*} 
      Άρα 
      \begin{equation}
        f(x,y,z) =  x^{3} + 3xy -x + c(y,z) \label{fxyz}
      \end{equation}
      Στη συνέχεια παραγωγίζουμε μερικώς ως προς $y$ και ως προς $z$  
      τη συνάρτηση $f(x,y,z)$ που μόλις βρήκαμε και έχουμε:
      \begin{align}
        \pdv{f}{y} &= 3x + \pdv{c}{y} \label{cy} \\
        \pdv{f}{z} &= \pdv{c}{z} \label{cz}
      \end{align}
      Στη συνέχεια εξισώνοντας την~\eqref{cy} με την~\eqref{fy11} προκύπτει 
      \begin{equation}
        \pdv{c}{y} = z^{2} \label{last1}
      \end{equation}
      και εξισώνοντας την~\eqref{cz} με την~\eqref{fz11} προκύπτει
      \begin{equation}
        \pdv{c}{z} = 2yz+1 \label{last2}
      \end{equation}
      Από τις σχέσεις~\eqref{last1} και~\eqref{last2} μπορούμε να βρούμε τη συνάρτηση 
      $ c(y,z) $ ακολουθώντας τη διαδικασία του προηγούμενου παραδείγματος, αφού 
      ουσιαστικά έχουμε τις μερικές παραγώγους της συνάρτησης και ζητάμε 
      να βρούμε την ίδια τη συνάρτηση. Οπότε ολοκληρώνουμε μερικώς ως προς 
      $y$ την~\eqref{last1} και έχουμε
      \begin{equation}
        c(y,z) = \int z^{2} \,{dy} = z^{2}y + k(z) \label{kz} 
      \end{equation} 
      Στη συνέχεια παραγωγίζουμε ως προς $z$ τη συνάρτηση $ c(y,z) $ που μόλις βρήκαμε 
      και έχουμε
      \begin{equation}
        \pdv{c}{z} = 2yz + k'(z) \label{kz1}
      \end{equation} 
      Έπειτα εξισώνουμε τις σχέσεις~\eqref{last2} και~\eqref{kz1} και προκύπτει 
      \[
        k'(z) = 1 \Rightarrow k(z) = z + c_{1} \label{finally}
      \] 
      Οπότε με αντικατάσταση της~\eqref{finally} στην~\eqref{kz} βρίσκουμε 
      \begin{equation}
        c(y,z) = z^{2}y+z + c_{1} \label{cyz}
      \end{equation}
      Τέλος με αντικατάσταση της~\eqref{cyz} στη συνάρτηση $ f(x,y,z) $ από τη 
      σχέση~\eqref{fxyz} βρίσκουμε
      \[
        f(x,y,z) = x^{3}+3xy-x+z^{2}y+z+ c_{1} 
      \] 
  \end{description}
\end{solution}


\section{Τύπος Taylor και Maclaurin}

\begin{example}
  Να υπολογιστεί το ανάπτυγμα της συνάρτησης $f(x,y)=x^3+y^3+xy^2$ γύρω από το 
  σημείο $ (1,2) $ (ή ισοδύναμα σε δυνάμεις του $(x-1)$ και $(y-2)$).
\end{example}
\begin{solution}
  Για να υπολογίσουμε το ανάπτυγμα της συνάρτησης σε δυνάμεις του $(x-1)$ και $(y-2)$ 
  αρκεί ισοδύναμα να υπολογίσουμε το ανάπτυγμα της συνάρτησης γύρω από το 
  σημείο $(x_0,y_0)=(1,2)$.
  Επειδή στην εκφώνηση δεν αναφέρεται μέχρι τους όρους ποιας τάξης 
  πρέπει να βρω το ανάπτυγμα, και η συνάρτηση είναι πολυωνυμική,
  βρίσκω μέχρι την τάξη όπου μηδενίζονται οι μερικές παράγωγοι: 
  (Δηλαδή μέχρι $3$ης τάξης, αφού όλες οι παράγωγοι $4$ης και ανώτερης τάξης, 
  είναι όλες μηδέν)

  \vspace{\baselineskip}

  \twocolumnsides{\begin{itemize}
      \item $f_x=3x^2+y^2\Rightarrow f_x(1,2)=7$
      \item $f_y=3y^2+2xy\Rightarrow f_y(1,2)=16$
      \item $f_{xx}=6x\Rightarrow f_{xx}(1,2)=6$
      \item $f_{xy}=2y\Rightarrow f_{xy}(1,2)=4$
      \item $f_{yy}=6y+2x\Rightarrow f_{yy}(1,2)=14$
      \end{itemize}}{\begin{itemize}
      \item $f_{xxx}=6$
      \item $f_{xxy}=f_{xyx}=0$
      \item $f_{xyy}=f_{yxy}=2$
      \item $f_{yyy}=6$ 
  \end{itemize}}

  \vspace{\baselineskip}

  Με αντικατάσταση των μερικών παραγώγων στον τύπο Taylor, έχουμε:
  \begin{align*}
    f(x,y)&=13+\Bigl(7(x-1)+16(y-2)\Bigr)+ \\ 
          &\quad +\frac{1}{2!}\Bigl(6(x-1)^2 +2\cdot 4(x-1)(y-2)+14(y-2)^2\Bigr)+ \\
          &\quad +\frac{1}{3!}\Bigl(6(x-1)^3+3\cdot 0(x-1)^2(y-2)+3
          \cdot 2(x-1)(y-2)^2+6(y-2)^3\Bigr).
  \end{align*}
  Και μετά τις πράξεις, έχουμε: 
  \begin{align*}
    f(x,y)&=13+7(x-1)(y-2)+16(y-2)+3(x-1)^2+4(x-1)(y-2) \\
          &\quad +7(y-2)^2+(x-1)^3+(x-1)(y-2)^2+(y-2)^3.
  \end{align*}
  Δεν κάνουμε άλλες πράξεις.

  Έχουμε το ανάπτυγμα της $f(x,y)$ σε δυνάμεις του $(x-1)$ και $(y-2)$ όπως ζητήθηκε.
\end{solution}


\chapter{Πεπλεγμένες Συναρτήσεις}


\section{Θεωρήματα Πεπλεγμένων συναρτήσεων}

\vspace{\baselineskip}

\subsection{Η εξίσωση \ensuremath{F(x,y) = 0}}

\subsubsection{Πεπλεγμένη συνάρτηση της μορφής \ensuremath{y=f(x)}}

Έστω $ F(x,y) = 0 $, όπου $ F\colon D \to \mathbb{R} $ μια συνάρτηση με πεδίο
ορισμού ένα ανοιχτό υποσύνολο $D$ του $\mathbb{R}^{2}$ και $ (x_0,y_0) $ ένα 
εσωτερικό σημείο του $D$.  Αν 
\begin{enumerate}[(i)]
  \item $F(x_0,y_0) = 0$ 
  \item $ F_x, F_y$ συνεχείς σε περιοχή του σημείου $ (x_0,y_0) $ 
  \item $ F_{\textcolor{Col1}{y}}(x_0,y_0) \neq 0 $
\end{enumerate}
τότε υπάρχει μοναδική συνάρτηση $\textcolor{Col1}{y=y(x)} $ ορισμένη στο $ I_0 \subseteq \mathbb{R} $ 
τέτοια ώστε:
\begin{myitemize}
  \item $y_0 = y(x_0)$
  \item $F(x,y(x)) = 0 \quad \forall x \in I_0$
  \item $ \dv{y}{x} = - \frac{F_x}{F_y} \quad \forall x \in I_0  $
\end{myitemize}

\begin{rem}
  Ο παραπάνω τύπος για την παράγωγο $ \dv{y}{x} $ προκύπτει ως λύση της εξίσωσης
  \[
    \pdv{F}{x} + \pdv{F}{y}\dv{y}{x} = 0 
  \] 
  η οποία προκύπτει με παραγώγιση της $ F(x,y) = 0$, αν θεωρήσουμε ότι $ y=y(x) $.
\end{rem}

\subsubsection{Πεπλεγμένη συνάρτηση της μορφής \ensuremath{x=f(y)}}

Έστω $ F(x,y) = 0 $, όπου $ F\colon D \to \mathbb{R} $ μια συνάρτηση με πεδίο
ορισμού ένα ανοιχτό υποσύνολο $D$ του $\mathbb{R}^{2}$ και $ (x_0,y_0) $ ένα 
εσωτερικό σημείο του $D$.  Αν 
\begin{enumerate}[(i)]
  \item $F(x_0,y_0) = 0$ 
  \item $ F_x, F_y$ συνεχείς σε περιοχή του σημείου $ (x_0,y_0) $ 
  \item $ F_{\textcolor{Col1}{x}}(x_0,y_0) \neq 0 $
\end{enumerate}
τότε υπάρχει μοναδική συνάρτηση $ \textcolor{Col1}{x=x(y)} $ ορισμένη στο $ I_0 \subseteq \mathbb{R} $ 
τέτοια ώστε:
\begin{myitemize}
  \item $x_0 = x(y_0)$
  \item $F(x(y),y) = 0 \quad \forall y \in I_0$
  \item $ \dv{x}{y} = - \frac{F_y}{F_x} \quad \forall y \in I_0  $
\end{myitemize}

\begin{rem}
  Ο παραπάνω τύπος για την παράγωγο $ \dv{x}{y} $ προκύπτει ως λύση της εξίσωσης
  \[
    \pdv{F}{y} + \pdv{F}{x}\dv{x}{y} = 0 
  \] 
  η οποία προκύπτει με παραγώγιση της $ F(x,y) = 0$, αν θεωρήσουμε ότι $ x=x(y) $.
\end{rem}
\begin{example}
  Να υπολογίσετε τα σημεία της καμπύλης $ F(x,y) = x^{3}+y^{3}-6xy=0 $ για τα οποία 
  ισχύει το θεώρημα πεπλεγμένης συνάρτησης.
\end{example}
\begin{solution}
  Αρχικά εξετάζουμε αν το $ y $ μπορεί να είναι πεπλεγμένη συνάρτηση του $x$.   
  Βρίσκουμε τα σημεία που επαληθεύουν την εξίσωση και για τα οποία $ F_{y}=0 $.
  \[
    F_{y} = 0 \Leftrightarrow 3y^{2}-6x=0 \Leftrightarrow \inlineequation[xsom]
    {x = \frac{1}{2} y^{2}}
  \] 
  Βρίσκουμε ποια από αυτά τα σημεία επαληθεύουν την εξίσωση και έχουμε:
  \[
    \left(\frac{1}{2} y^{2}\right)^{3}+y^{3}-6 \frac{1}{2} y^{2}y=0 
    \Leftrightarrow \frac{1}{8} y^{6}+y^{3} -3y^{3}=0 
    \Leftrightarrow y^{3}\left(\frac{1}{8} y^{3}-2\right)=0 \Leftrightarrow y=0 \quad
    \text{ή} \quad y= \sqrt[3]{16} = \sqrt[3]{8\cdot2} = 2 \sqrt[3]{2}
  \] 
  Άρα, από την εξίσωση~\eqref{xsom} έχουμε ότι 
  \[ 
    x=0  \quad \text{ή} \quad x = \frac{1}{2} \left(2 \sqrt[3]{2}\right)^{2} = 
    \frac{1}{2} \left(2 \cdot 2^{\frac{1}{3}}\right)^{2} = 2 \cdot 2^{\frac{2}{3}} 
    = 2 \sqrt[3]{2^{2}} = 2 \sqrt[3]{4}  
  \]
  Επομέvως το θεώρημα δεν εφαρμόζεται για τα σημεία $ (0,0) $ και $ (2 \sqrt[3]{4}
  , 2 \sqrt[3]{2}) $.

  Στη συνέχεια εξετάζουμε αν το $ x $ μπορεί να είναι πεπλεγμένη συνάρτηση του $y$.
  Βρίσκουμε τα σημεία που επαληθεύουν την εξίσωση και για τα οποία $ F_{x}=0 $.
  \[
    F_{x}=0 \Leftrightarrow 3x^{2}-6y=0 \Leftrightarrow \inlineequation[ysom]
    {y= \frac{1}{2} x^{2}}
  \] 
  Βρίσκουμε ποια από αυτά τα σημεία επαληθεύουν την εξίσωση και έχουμε:
  \[
    x^{3}+\left(\frac{1}{2} x^{2}\right)^{3}-6x \frac{1}{2} x^{2}=0 \Leftrightarrow 
    x^{3}+ \frac{1}{8} x^{6}-3x^{3}=0 \Leftrightarrow \frac{1}{8} x^{6}-3x^{3}=0
    \Leftrightarrow x^{3}\left(\frac{1}{8} x^{3}-2\right)=0 \Leftrightarrow x = 0 \quad
    \text{ή} \quad x=2 \sqrt[3]{2} 
  \] 
  Άρα, από την εξίσωση~\eqref{ysom} έχουμε ότι 
  \[ 
    y=0  \quad \text{ή} \quad y = 2 \sqrt[3]{4} 
  \]
  Επομέvως το θεώρημα δεν εφαρμόζεται για τα σημεία $ (0,0) $ και $ (2 \sqrt[3]{2}
  , 2 \sqrt[3]{4}) $.

  Οπότε συνολικά, το θεώρημα δεν εφαρμόζεται για τα σημεία 
  \[
    \boxed{(0,0), \quad (2 \sqrt[3]{4} , 2 \sqrt[3]{2}), \quad (2 \sqrt[3]{2} , 2
    \sqrt[3]{4})}
  \] 
\end{solution}

\subsection{Η εξίσωση \ensuremath{F(x,y,z) = 0}}

\subsubsection{Πεπλεγμένη συνάρτηση της μορφής \ensuremath{z=z(x,y)}}

Έστω $ F(x,y,z) = 0 $, όπου $F\colon D \to \mathbb{R}$ μια συνάρτηση με πεδίο ορισμού 
ένα ανοικτό υποσύνολο $ D $ του $ \mathbb{R}^{3}  $ και $ (x_0,y_0,z_0) $ ένα 
εσωτερικό σημείο του $ D $. Αν
\begin{enumerate}[(i)]
  \item $ F(x_0,y_0,z_0) $
  \item $ F_x, F_y, F_z $ συνεχείς σε περιοχή του σημείου $ (x_0,y_0,z_0) $
  \item $ F_{\textcolor{Col1}{z}}(x_0,y_0,z_0) \neq 0 $
\end{enumerate}
τότε υπάρχει μοναδική συνάρτηση, $ \textcolor{Col1}{z=z(x,y)} $ ορισμένη στο 
$ D_0 \subseteq \mathbb{R}^{2} $ τέτοια ώστε:
\begin{myitemize}
  \item $ z_0 = z(x_0,y_0) $
  \item $ F(x,y,z(x,y)) = 0  \quad \forall (x,y)\in  D_0 $
  \item $ \pdv{z}{x} = - \frac{F_x}{F_z} $, $ \pdv{z}{y} = - \frac{F_y}{F_z}\quad 
    \forall (x,y) \in D_0$
\end{myitemize}

\begin{rem}
  Οι παραπάνω τύποι για τις παραγώγους $ \pdv{z}{x}, \pdv{z}{y} $ προκύπτουν 
  ως λύσεις των εξισώσεων  
  \begin{align*}	
    \pdv{F}{x} + \pdv{F}{z}\pdv{z}{x} &= 0 \\
    \pdv{F}{y} + \pdv{F}{z}\pdv{z}{y} &= 0 
  \end{align*}
  οι οποίες προκύπτουν με παραγώγιση της $ F(x,y,z) = 0 $, ως προς $x$ και $y$ 
  αντίστοιχα και  αν θεωρήσουμε ότι $ z=z(x,y) $.
\end{rem}

\subsubsection{Πεπλεγμένη συνάρτηση της μορφής \ensuremath{y=y(x,z)}}

Έστω $ F(x,y,z) = 0 $, όπου $F\colon D \to \mathbb{R}$ μια συνάρτηση με πεδίο ορισμού 
ένα ανοικτό υποσύνολο $ D $ του $ \mathbb{R}^{3}  $ και $ (x_0,y_0,z_0) $ ένα 
εσωτερικό σημείο του $ D $. Αν
\begin{enumerate}[(i)]
  \item $ F(x_0,y_0,z_0) $
  \item $ F_x, F_y, F_z $ συνεχείς σε περιοχή του σημείου $ (x_0,y_0,z_0) $
  \item $ F_{\textcolor{Col1}{y}}(x_0,y_0,z_0) \neq 0 $
\end{enumerate}
τότε υπάρχει μοναδική συνάρτηση, $ \textcolor{Col1}{y=y(x,z)} $ ορισμένη στο 
$ D_0 \subseteq \mathbb{R}^{2} $ τέτοια ώστε:
\begin{myitemize}
  \item $ y_0 = y(x_0,z_0) $
  \item $ F(x,y(x,z),z)) = 0  \quad \forall (x,z) \in  D_0 $
  \item $ \pdv{y}{x} = - \frac{F_x}{F_y} $, $ \pdv{y}{z} = - \frac{F_z}{F_y}\quad 
    \forall (x,z) \in D_0$
\end{myitemize}

\begin{rem}
  Οι παραπάνω τύποι για τις παραγώγους $ \pdv{y}{x}, \pdv{y}{z} $ προκύπτουν 
  ως λύσεις των εξισώσεων  
  \begin{align*}	
    \pdv{F}{x} + \pdv{F}{y}\pdv{y}{x} &= 0 \\
    \pdv{F}{z} + \pdv{F}{y}\pdv{y}{z} &= 0 
  \end{align*}
  οι οποίες προκύπτουν με παραγώγιση της $ F(x,y,z) = 0 $, ως προς $x$ και $z$ 
  αντίστοιχα και  αν θεωρήσουμε ότι $ y=y(x,z) $.
\end{rem}

\subsubsection{Πεπλεγμένη συνάρτηση της μορφής \ensuremath{x=x(y,z)}}

Έστω $ F(x,y,z) = 0 $, όπου $F\colon D \to \mathbb{R}$ μια συνάρτηση με πεδίο ορισμού 
ένα ανοικτό υποσύνολο $ D $ του $ \mathbb{R}^{3}  $ και $ (x_0,y_0,z_0) $ ένα 
εσωτερικό σημείο του $ D $. Αν
\begin{enumerate}[(i)]
  \item $ F(x_0,y_0,z_0) $
  \item $ F_x, F_y, F_z $ συνεχείς σε περιοχή του σημείου $ (x_0,y_0,z_0) $
  \item $ F_{\textcolor{Col1}{x}}(x_0,y_0,z_0) \neq 0 $
\end{enumerate}
τότε υπάρχει μοναδική συνάρτηση, $ \textcolor{Col1}{x=x(y,z)} $ ορισμένη στο 
$ D_0 \subseteq \mathbb{R}^{2} $ τέτοια ώστε:
\begin{myitemize}
  \item $ x_0 = x(y_0,z_0) $
  \item $ F(x(y,z),y,z)) = 0  \quad \forall (y,z) \in  D_0 $
  \item $ \pdv{x}{y} = - \frac{F_y}{F_x} $, $ \pdv{x}{z} = - \frac{F_z}{F_x}\quad 
    \forall (y,z) \in D_0$
\end{myitemize}

\begin{rem}
  Οι παραπάνω τύποι για τις παραγώγους $ \pdv{x}{y}, \pdv{x}{z} $ προκύπτουν 
  ως λύσεις των εξισώσεων  
  \begin{align*}	
    \pdv{F}{y} + \pdv{F}{x}\pdv{x}{y} &= 0 \\
    \pdv{F}{z} + \pdv{F}{x}\pdv{x}{z} &= 0 
  \end{align*}
  οι οποίες προκύπτουν με παραγώγιση της $ F(x,y,z) = 0 $, ως προς $y$ και $z$ 
  αντίστοιχα και  αν θεωρήσουμε ότι $ x=x(x,z) $.
\end{rem}
\begin{example}
  Έστω η εξίσωση $ F(x,y,z) = y^{2}+xz+z^{2}- \mathrm{e}^{z}-4 = 0$. 
  Να εξετάσετε αν ισχύει το θεώρημα πεπλεγμένης συνάρτησης, για την 
  $ z = z(x,y) $, στο σημείο $P(0,\mathrm{e},2) $ και να βρεθούν οι μερικές παράγωγοι 
  1ης και 2ης τάξης της συνάρτησης $ z = z(x,y) $ στο σημείο $ (0, \mathrm{e}) $.
\end{example}
\begin{solution}
\item {}
  \begin{enumerate}[i)]
    \item $ F(0, \mathrm{e}, 2) = \mathrm{e}^{2} + 0\cdot 2 + 2^{2} - 
      \mathrm{e}^{2} - 4 = 0  $ 
    \item \begin{tabular}{l}
        $ F_{x} = z \phantom{\ +2z- \mathrm{e}^{z}} $ \tikzmark{a} \\
        $ F_{y} = 2y $ \\
        $ F_{z} = x+2z- \mathrm{e}^{z} \tikzmark{b} $
      \end{tabular}
      \mybrace{a}{b}[είναι συνεχείς ως πολυωνυμικές]
    \item $ F_{z}(0, \mathrm{e},2) = 0+2\cdot 2- \mathrm{e}^{2} 
      = 4- \mathrm{e}^{2} \neq 0 $
  \end{enumerate}
  Επομένως ικανοποιούνται οι προυποθέσεις του θεωρήματος Πεπλεγμένης συνάρτησης 
  και άρα υπάρχει μοναδική συνάρτηση $ z=z(x,y) $, για την οποία ισχύουν:
  \begin{myitemize}
    \item $ \inlineequation[eq:perplex1]{2=z(0, \mathrm{e})} $
    \item $ z_{x} = - \frac{F_{x}}{F_{z}} = - \frac{z}{x+2z- \mathrm{e}^{z}} $ και 
      $ z_{y} = - \frac{F_{y}}{F_{z}} - \frac{2y}{x+2z- \mathrm{e}^{z}} $ σε μια 
      περιοχή του σημείου $ (0,\mathrm{e}) $.
  \end{myitemize}
  Επομένως στο σημείο $ (0, \mathrm{e}) $ έχουμε ότι $ z= z(0, \mathrm{e}) = 2 $ και 
  άρα:
  \[
    z_{x}(0, \mathrm{e}) \overset{\eqref{eq:perplex1}}{=}  
    - \frac{2}{0+2\cdot 2- e^{2}} = - \frac{2}{4-\mathrm{e}^{2}} \quad \text{και} 
    \quad z_{y}(0, \mathrm{e}) \overset{\eqref{eq:perplex1}}{=}  
    - \frac{2 \cdot \mathrm{e}}{0+2\cdot 2 - 
    \mathrm{e}^{2}} = - \frac{2 \mathrm{e}}{4- \mathrm{e}^{2}} 
  \]
  Για να βρούμε τις μερικές παραγώγους 2ης τάξης της συνάρτησης $ z(x,y) $, 
  παραγωγίζουμε ξανά, τις μερικές παραγώγους 1ης τάξης, θεωρούμε όμως ότι 
  $z=z(x,y)$ είναι συνάρτηση.
  \begin{align*}
    z_{xx} &= \left(\frac{-z}{x+2z- \mathrm{e}^{z}}\right) _{x} =
    \frac{(-z)_{x}(x+2z- \mathrm{e}^{z})-(-z)(x+2z- \mathrm{e}^{z} )_{x}}{(x+2z-
      \mathrm{e}^{z})^{2}} = \frac{-z_{x}(x+2z- \mathrm{e}^{z})+z(1+2z_{x}- 
    \mathrm{e}^{z} z_{x})}{(x+2z- \mathrm{e}^{z})^{2}}  \\ 
    z_{yy} &= \left(\frac{-2y}{x+2z- \mathrm{e}^{z}}\right)_{y} = 
    \frac{(-2y)_{y}(x+2z- \mathrm{e}^{z})- (-2y)(x+2z- \mathrm{e}^{z} )_{y}}{(x+2z-
      \mathrm{e}^{z} )^{2}} = \frac{(-2)(x+2z- \mathrm{e}^{z} )+2y(2z_{y}-
    \mathrm{e}^{z} z_{y})}{(x+2z- \mathrm{e}^{z})^{2}} \\  
    z_{xy}&= \left(\frac{-z}{x+2z+ \mathrm{e}^{z}}\right)_{y} = 
    \frac{(-z)_{y}(x+2z+ \mathrm{e}^{z})-(-z)(x+2z+ \mathrm{e}^{z} )_{y}}{(x+2z+
      \mathrm{e}^{z})^{2}} = \frac{-z_{y}(x+2z+ \mathrm{e}^{z})+z(+2z_{y}+ 
    \mathrm{e}^{z} z_{y})}{(x+2z+ \mathrm{e}^{z})^{2}}  \\ 
  \end{align*} 
  Για να υπολογίσουμε τις μερικές παραγώγους 2ης τάξης στο σημείο $ (0,e) $ το 
  μόνο που απομένει είναι να αντικαταστήσουμε όπουμε $ x=0 $, $ y=e $, $ z=2 $ και 
  όπου $ z_{x} = - \frac{2}{4-e^{2}} $ και $ z_{y} = - \frac{2e}{4-e^{2}} $
\end{solution}

\chapter{Παράγωγος Σύνθετων Συναρτήσεων}

\section{1η Περίπτωση: \ensuremath{z=f(x,y),  x=x(t),  y=y(t)}} 

\begin{thm}
  Αν η συνάρτηση $ f(x,y) $ είναι ορισμένη στο ανοιχτό σύνολο 
  $ A \subseteq \mathbb{R}^{2} $ και $ x = x(t) $, $ y=y(t) $, με 
  $ t \in [a,b] $ και η $f$ έχει συνεχείς μερικές 
  παραγώγους στο $A$ και οι $ x(t) $ και $ y(t) $ είναι παραγωγίσιμες στο 
  $ [a,b] $, τότε η παράγωγος της σύνθετης συνάρτησης $f$ ως προς $t$ δίνεται από 
  τον τύπο:

  \twocolumnsidel{
    \[\xymatrix{ & f \ar@{-}[dl]_{\mathlarger{\pdv{f}{x}}}
        \ar@{-}[dr]^{\mathlarger{\pdv{f}{y}}} &  \\
        x \ar@{-}[dr]_{\mathlarger{\dv{x}{t}}} & & y 
    \ar@{-}[dl]^{\mathlarger{\dv{y}{t}}} \\ & t & }\]
    }{
    \begin{equation}\label{eq:deriv1}
      \dv{f}{t} = \pdv{f}{x} \dv{x}{t} + \pdv{f}{y} \dv{y}{t} 
    \end{equation}
    Ενώ η 2η παράγωγος από τον τύπο:
    \[
      \dv[2]{f}{t} =  \pdv[2]{f}{x} \left(\dv{x}{t}\right)^{2} + 
      2 \pdv[2]{f}{x}{y} \dv{x}{t} \dv{y}{t} + \pdv[2]{f}{y} 
      \left(\dv{y}{t}\right)^{2} + \pdv{f}{x} \dv[2]{x}{t} + \pdv{f}{y} \dv[2]{y}{t}
    \]
  }
\end{thm}

\section{2η Περίπτωση: \ensuremath{z=f(x,y),  x=x(u,v),  y=y(u,v)}} 

\begin{thm}
  Αν η συνάρτηση $ f(x,y) $ είναι ορισμένη στο ανοιχτό σύνολο 
  $ A \subseteq \mathbb{R}^{2} $ και $ x = x(u,v) $, $ y=y(u,v) $, με 
  και η $f$ έχει συνεχείς μερικές παραγώγους στο $A$ και οι $ x $ και $ y $, έχουν 
  συνεχείς μερικές παραγώγους στο $ E \subseteq \mathbb{R}^{2} $,
  τότε οι μερικές παράγωγοι της $f$, υπάρχουν και δίνονται από τους τύπους:
\end{thm}

\twocolumnsidel{
  \[
    \xymatrix@C-6pt{ & & f \ar@{-}[dl] 
      _{\mathlarger{\pdv{f}{x}}}^{\mathlarger{f_{x}}} 
      \ar@{-}[dr]^{\mathlarger{\pdv{f}{y}}}_{\mathlarger{f_{y}}}
                     & &  \\
                     & x \ar@{-}[dl]_{\mathlarger{\pdv{x}{u}}}
      \ar@{-}[d]^{\mathlarger{\pdv{x}{v}}} &  
                                           & y \ar@{-}[d]_{\mathlarger{\pdv{y}{u}}} 
      \ar@{-}[dr]^{\mathlarger{\pdv{y}{v}}} & \\
    u & v & & u & v \\ } 
  \]
  }{
  \begin{equation}\label{eq:deriv2}
    \pdv{f}{u} = \pdv{f}{x} \pdv{x}{u} + \pdv{f}{y} \pdv{y}{u} 
    \quad \text{και} \quad
    \pdv{f}{v} = \pdv{f}{x} \pdv{x}{v} + \pdv{f}{y} \pdv{y}{v} 
  \end{equation}
  Ενώ οι μερικές παράγωγοι 2ης τάξης, δίνονται από τους τύπους:
  \[
    \pdv[2]{f}{u} =  \pdv[2]{f}{x} \left(\pdv{x}{u}\right)^{2} + 
    2 \pdv[2]{f}{x}{y} \pdv{x}{u} \pdv{y}{u} + \pdv[2]{f}{y} 
    \left(\pdv{y}{u}\right)^{2} + \pdv{f}{x} \pdv[2]{x}{u} + \pdv{f}{y} 
    \pdv[2]{y}{u}
  \]
  \[
    \pdv[2]{f}{v} =  \pdv[2]{f}{x} \left(\pdv{x}{v}\right)^{2} + 
    2 \pdv[2]{f}{x}{y} \pdv{x}{u} \pdv{y}{v} + \pdv[2]{f}{y} 
    \left(\pdv{y}{v}\right)^{2} + \pdv{f}{x} \pdv[2]{x}{v} + \pdv{f}{y} 
    \pdv[2]{y}{v}
  \]
  \[
    \pdv[2]{f}{v}{u} = \pdv[2]{f}{x} \pdv{x}{u} \pdv{x}{v} + \pdv[2]{f}{x}{y}
    \left(\pdv{x}{u} \pdv{y}{v}+ \pdv{x}{v} \pdv{y}{u} \right) + \pdv[2]{f}{y} 
    \pdv{y} {u} \pdv{y}{v} + \pdv{f}{x} \pdv[2]{x}{u}{v} + \pdv{f}{y} 
    \pdv[2]{y}{u}{v} 
  \]
}

\begin{rem}
  Οι τύποι~\eqref{eq:deriv1} και~\eqref{eq:deriv2} προέκυψαν αθροίζοντας κάθε φορά, 
  τα μονοπάτια που ξεκινούν από τη μεταβλητή $f$ και καταλήγουν στη μεταβλητή ως 
  προς την οποία παραγωγίζουμε, όπου κάθε μονοπάτι αποτελείται από το γινόμενο των 
  παραγώγων που συναντούμε "διασχίζοντάς" το.
\end{rem}


\section{Αποδείξεις των τύπων των μερικών Παραγώγων 2ης τάξης}

\subsection{Απόδειξη με το συμβολισμό του Leibnitz}

Αποδεικνύουμε τον τύπο για την $ \pdv[2]{f}{u} $ και ομοίως προκύπτουν και οι τύποι 
για τις $ \pdv[2]{f}{v}  $ και $ \pdv[2]{f}{u}{v} $.

\vspace{\baselineskip}

\twocolumnsidel{
  \[ 
    \xymatrix@C-6pt{ & & \pdv{f}{x} 
      \ar@{-}[dl]_{\mathlarger{\pdv[2]{f}{x}}} 
      \ar@{-}[dr]^{\mathlarger{\pdv[2]{f}{x}{y}}} & &  \\
                                                  & x \ar@{-}[dl]
                                                  _{\mathlarger{\pdv{x}{u}}}
      \ar@{-}[d]^{\mathlarger{\pdv{x}{v}}} &  
                                           & y \ar@{-}[d]_{\mathlarger{\pdv{y}{u}}} 
      \ar@{-}[dr]^{\mathlarger{\pdv{y}{v}}} & \\
      u & v & & u & v \\ 
    } 
  \] 
  \[\xymatrix@C-6pt{ & & \pdv{f}{y} 
      \ar@{-}[dl]_{\mathlarger{\pdv[2]{f}{y}{x}}} 
      \ar@{-}[dr]^{\mathlarger{\pdv[2]{f}{y}}} & &  \\
                                               & x \ar@{-}[dl]
                                               _{\mathlarger{\pdv{x}{u}}}
      \ar@{-}[d]^{\mathlarger{\pdv{x}{v}}} &  
                                           & y \ar@{-}[d]_{\mathlarger{\pdv{y}{u}}} 
      \ar@{-}[dr]^{\mathlarger{\pdv{y}{v}}} & \\
    u & v & & u & v \\ }
  \] 
  }{
  \begin{proof}
    \[\begin{aligned}
      \pdv[2]{f}{u} 
  &= \pdv{}{u}\left(\pdv{f}{u}\right) = \pdv{}{u} 
  \left( \pdv{f}{x} \pdv{x}{u} + \pdv{f}{y} \pdv{y}{u}\right) = 
  \pdv{}{u} \left(\pdv{f}{x} \pdv{x}{u}\right) + \pdv{}{u} 
  \left(\pdv{f}{y} \pdv{y}{u}\right) \\
  &= \pdv{}{u} \left( \pdv{f}{x}\right) \pdv{x}{u} + \pdv{f}{x} \pdv{}{u} \left(
  \pdv{x}{u} \right) + \pdv{}{u} \left(\pdv{f}{y} \right) \pdv{y}{u} + \pdv{f}{y} 
  \pdv{}{u} \left(\pdv{y}{u}\right) \\
  &=\left[ \pdv[2]{f}{x} \pdv{x}{u} + \pdv[2]{f}{x}{y} \pdv{y}{u} \right] \pdv{x}{u} +
  \pdv{f}{x} \pdv[2]{x}{u} + 
  \left[ \pdv[2]{f}{y}{x} \pdv{x}{u} + \pdv[2]{f}{y} \pdv{y}{u} \right] \pdv{y}{u} +
  \pdv{f}{y} \pdv[2]{y}{u} \\
  &= \pdv[2]{f}{x} \left(\pdv{x}{u}\right)^{2} + \pdv[2]{f}{x}{y} \pdv{y}{u}
  \pdv{x}{u} + \pdv{f}{x} \pdv[2]{x}{u} + \pdv[2]{f}{y}{x} \pdv{x}{u} \pdv{y}{u} + 
  \pdv[2]{f}{y} \left(\pdv{y}{u}\right)^{2} + 
  \pdv{f}{y} \pdv[2]{y}{u} \\
  &= \pdv[2]{f}{x} \left(\pdv{x}{u}\right)^{2} + 2\pdv[2]{f}{x}{y} \pdv{x}{u}
  \pdv{y}{u} + \pdv[2]{f}{y} \left(\pdv{y}{u}\right)^{2} + \pdv{f}{x} \pdv[2]{x}{u} +
  \pdv{f}{y} \pdv[2]{y}{u} \\
    \end{aligned}
  \]
\end{proof}
  }


  \subsection{Απόδειξη με το συμβολισμό των δεικτών}

  Αποδεικνύουμε τον τύπο για την $ f_{uu} $ και ομοίως προκύπτουν και οι τύποι για τις  
  $ f_{vv} $ και $ f_{uv} $.

  \vspace{\baselineskip}

  \twocolumnsidel{
    {
      \[ 
        \xymatrix@C-6pt{ & & f_{x}
          \ar@{-}[dl]_{\mathlarger{f_{xx}}} 
          \ar@{-}[dr]^{\mathlarger{f_{xy}}} & &  \\
                                            & x \ar@{-}[dl]_{\mathlarger{x_{u}}}
          \ar@{-}[d]^{\mathlarger{x_{v}}} &  
                                          & y \ar@{-}[d]_{\mathlarger{y_{u}}} 
          \ar@{-}[dr]^{\mathlarger{y_{v}}} & \\
          u & v & & u & v \\ 
        } 
      \] 
      \[\xymatrix@C-6pt{ & & f_{y}
          \ar@{-}[dl]_{\mathlarger{f_{yx}}} 
          \ar@{-}[dr]^{\mathlarger{f_{yy}}} & &  \\
                                            & x \ar@{-}[dl]_{\mathlarger{x_{u}}}
          \ar@{-}[d]^{\mathlarger{x_{v}}} &  
                                          & y \ar@{-}[d]_{\mathlarger{y_{u}}} 
          \ar@{-}[dr]^{\mathlarger{y_{v}}} & \\
        u & v & & u & v \\ }
      \] 
    }
    }{
    \begin{proof}
      \[
        \begin{aligned}
          f_{uu} &= (f_{u})_{u} = (f_{x}x_{u}+f_{y}y_{u})_{u} \\
                 &=(f_{x}x_{u})_{u}+ (f_{y}y_{u})_{u} \\
                 &=(f_{x})_{u}x_{u} + f_{x}(x_{u})_{u} + (f_{y})_{u}y_{u}+ 
                 f_{y}(y_{u})_{u} \\
                 &= (f_{xx}x_{u}+f_{xy}{y_{u}})x_{u} + f_{x} x_{uu} + 
                 (f_{yx}x_{u}+f_{yy}y_{u})y_{u} + f_{y}y_{uu} \\
                 &= f_{xx}(x_{u})^{2} + f_{xy}y_{u}x_{u}+ f_{x}x_{uu} + 
                 f_{yx}x_{u}y_{u}+f_{yy}(y_{u})^{2}+ f_{y}y_{uu} \\
                 &= f_{xx}(x_{u})^{2}+ 2f_{xy}x_{u}y_{u} + f_{yy}(y_{u})^{2} + 
                 f_{x}x_{uu} + f_{y}y_{uu}
        \end{aligned}
      \] 
    \end{proof}
  }

  \begin{rem}
    Συγκεντρωτικά οι τύποι για τις παραγώγους 2ης ταξής της συνάρτησης 

    \[
      f_{uu}= f_{xx}(x_{u})^{2}+ 2f_{xy}x_{u}y_{u} + f_{yy}(y_{u})^{2} + f_{x}x_{uu} + 
      f_{y}y_{uu} 
    \] 
    \[
      f_{vv}= f_{xx}(x_{v})^{2}+ 2f_{xy}x_{v}y_{v} + f_{yy}(y_{v})^{2} + f_{x}x_{vv} + 
      f_{y}y_{vv} 
    \]
    \[
      f_{uv}= f_{xx}x_{u}x_{v}+ f_{xy}(x_{u}y_{v} + x_{v}y_{u}) + 
      f_{yy}y_{u}y_{v} + f_{x}x_{uv} + f_{y}y_{v} 
      f_{y}y_{vv} 
    \]
  \end{rem}


  \end{document}

