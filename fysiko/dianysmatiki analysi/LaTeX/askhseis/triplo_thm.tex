\input{preamble_ask.tex}
\input{definitions_ask.tex}

\pagestyle{askhseis}

% \linespread{1.3}
\renewcommand{\vec}{\mathbf}

\begin{document}

\begin{center}
  \minibox{\large \bfseries \textcolor{Col1}{θέματα στo Τριπλό Ολοκλήρωμα}}
\end{center}

\vspace{\baselineskip}

\section*{Όγκος}

\begin{enumerate}
  \item \textbf{(Σεπ. 2019)} Να υπολογίσετε τον όγκο του στερεού που περικλείεται 
    από τις επιφάνειες $ z=0 $, $ x^{2}+y^{2}=a^{2} $ και $ y^{2}=mz $ ($ m>0$, $ z>0 $ )

    \hfill Απ: $ \frac{\pi a^{4}}{4m} $

  \item \textbf{(Ιουν. 2019)} Να υπολογίσετε τον όγκο του στερεού που περικλείεται 
    από τις επιφάνειες $ x^{2}+y^{2}=1 $ και $ y+z=2 $.

    \hfill Απ:  

  \item \textbf{(Σεπ. 2015)} Δίνεται ο κύλινδρος $ x^{2}+y^{2}=R^{2} $ με $ 2 \leq z \leq
    5 $. Να υπολογιστεί ο όγκος του μέρους του κυλίνδρου που αποκόπτεται από τα επίπεδα 
    $ y=ax $ και $ y=bx $ και που βρίσκεται στο 1ο ογδοημόριο και στο 3ο.

    \hfill Απ:  

  \item \textbf{(Σεπ. 2014)} Έστω $V$ το στερεό που βρίσκεται κάτω από την επιφάνεια 
    με εξίσωση $ z = \mathrm{e}^{x^{2}+y^{2}} $ και πάνω από την επίπεδη περιοχή που 
    ορίζεται από την διπλή ανισότητα $ 1 \leq x^{2}+y^{2} \leq 2 $. Να υπολογιστεί 
    ο όγκος του στερεού $V$.
\end{enumerate}


\section*{Τριπλά Ολοκληρώματα}

\begin{enumerate}
  \item \textbf{(Ιαν. 2018)} Να υπολογιστεί το τριπλό ολοκλήρωμα $ \iiint_{V}
    \mathrm{e}^{\sqrt{x^{2}+y^{2}+z^{2}}} \, dxdydz $, όπου $ V $ η στερεά περιοχή, 
    που ορίζεται από τις εξισώσεις $ 1 \leq x^{2}+y^{2}+z^{2} \leq 2 $ και $ z \geq 0 $.

\end{enumerate}


\section*{Εφαρμογές}

\begin{enumerate}
  \item \textbf{(Ιαν. 2016)} Να βρεθεί το κέντρο μάζας του τμήματος της ομογενούς 
    σφαιρικής επιφάνειας $ x^{2}+y^{2}+z^{2}=a^{2} $, που βρίσκεται στο 1ο ογδοημόριο.

    \hfill Απ:  

  \item \textbf{(Ιουν. 2013)} Θεωρούμε μια ηλεκτρικά φορτισμένη υλική πλάκα, της οποίας 
    η περίμετρος ορίζεται από την εξίσωση $ \frac{x^{2}}{3} + y^{2} = 1 $. Η 
    πυκνότητα του ηλεκτρικού φορτίου δίνεται από την σχέση
    \[
      p(x,y) = 10 - (x-1)^{2}-y^{2}. 
    \]  
    Να βρεθούν τα σημεία της υλικής πλάκας όπου η πυκνότητα $ p(x,y) $ του φορτίου 
    είναι μέγιστη ή ελάχιστη.
\end{enumerate}

\end{document}


