\chapter{Ομογενείς Συναρτήσεις}

\section{Ορισμός}

\begin{dfn}
\item {}
  \begin{enumerate}[i)]
    \item $ f(x,y) $ \textcolor{Col1}{ομογενής βαθμού} $ \textcolor{Col1}{\rho} 
      \Leftrightarrow f(\lambda x, \lambda y), \; \forall \lambda \in \mathbb{R} $ 
    \item $ f(x,y,z) $ \textcolor{Col1}{ομογενής βαθμού} $ \textcolor{Col1}{\rho} 
      \Leftrightarrow f(\lambda x, \lambda y, \lambda z), \; \forall \lambda \in 
      \mathbb{R} $ 
  \end{enumerate}
\end{dfn}

\begin{thm}[Euler]
  Αν $ f(x,y) $ είναι ομογενής βαθμού $ \rho $ τότε $x f_{x} + y f_{y} = \rho f $.
\end{thm}
\begin{prop}
  Αν $ f(x,y) $ είναι ομογενής βαθμού $ \rho $ τότε οι συναρτήσεις 
  $f_{x}, f_{y} $ είναι ομογενείς βαθμού $ \rho -1 $.
\end{prop}
\begin{proof}
\item {}
  $ f $ ομογενής βαθμού $ \rho \overset{\text{(Euler)}}{\Rightarrow} xf_{x}+yf_{y}= 
  \rho f \xRightarrow[\text{ως προς $x$}]{\text{παρ/ζω}} f_{x} + x f_{xx} + y f_{yx} =
  \rho f_{x} \Rightarrow xf_{xx} + yf_{xy} = (\rho -1)f_{x} $

  $ f $ ομογενής βαθμού $ \rho \overset{\text{(Euler)}}{\Rightarrow} xf_{x}+yf_{y}= 
  \rho f \xRightarrow[\text{ως προς $y$}]{\text{παρ/ζω}} xf_{xy} + f_{y} + y f_{yy} =
  \rho f_{y} \Rightarrow xf_{yx} + yf_{yy} = (\rho -1)f_{y} $
\end{proof}

\begin{exercise}
  Έστω $ f(x,y) $, συνεχής, με συνεχείς μερικές παραγώγους 2ης τάξης, 
  ομογενής βαθμού $\rho$. Να δείξετε ότι ισχύει \[ x^{2}f_{xx}+2xyf_{xy}+y^{2}f_{yy} =
  \rho (\rho -1)f \]
\end{exercise}
\begin{solution}
\item {}
  $f$ ομογενής βαθμού $\rho \Rightarrow f_{x}, f_{y} $ ομογενείς βαθμού $\rho -1$.
  Οπότε ισχύει το θεώρημα Euler για αυτές, άρα:
  \begin{align*}
    \sysdelim.\}\systeme{x f_{xx}+yf_{xy}=(\rho -1)f_{x}, x f_{yx}+yf_{yy}=
  (\rho -1)f_{y}} \Rightarrow \sysdelim.\}
  \systeme*{x^{2} f_{xx} \+ xyf_{xy}=(\rho-1)xf_{x},yxf_{yx} \+ y^{2}f_{yy}=
  (\rho -1)yf_{y}} \xRightarrow{(+)} x^{2}f_{xx}+2xyf_{xy}+y^{2}f_{yy}=
  (\rho -1)(xf_{x}+yf_{y})
\end{align*} 
Όμως επειδή $f$ ομογενής έχουμε ότι $ xf_{x}+yf_{y}= \rho f $. Οπότε
\[
  x^{2}f_{xx}+2xyf_{xy}+y^{2}f_{yy} = \rho (\rho -1)f
\] 
\end{solution}


\begin{exercise}
  Αν $ u = u(x,y) $ και $ v=v(x,y) $ ομογενείς βαθμού $ \rho $, 
  τότε να δείξετε ότι $ \forall f(u,v) $ με συνεχείς μερικές παραγώγους 1ης τάξης
  ισχύει 
  \[
    xf_{x}+yf_{y}= \rho (u f_{u}+vf_{v}) 
  \] 
\end{exercise}
\begin{solution}
\item {}
  Έχουμε ότι για τις συναρτήσεις $u(x,y) $ και $v(x,y)$ είναι ομογενείς 
  βαθμού $\rho$, οπότε:
  \[
  \sysdelim.\}\systeme{xu_{x}+yu_{y}= \rho u, xv_{x}+yv_{y}= \rho v} 
\] 
Για την συνάρτηση $f$ έχουμε ότι είναι σύνθετη με $ f=f(u,v) $ και 
$ u = u(x,y) $ και $ v=v(x,y) $, οπότε οι μερικές παράγωγοί της δίνονται από
το δέντρο της και είναι:
\[
  \left.
    \begin{tabular}{l}
      $f_{x}=f_{u}u_{x}+f_{v}v_{x}$ \\
      $f_{y}=f_{u}u_{y}+f_{v}v_{y}$
    \end{tabular}
  \right\}
  \Rightarrow xf_{x}+yf_{y} = \underbrace{(xu_{x}+yu_{y})}_{\rho
  u}f_{u}+\underbrace{(xv_{x}+yv_{y})}_{\rho v}f_{v}
\] 
Άρα 
\[
  xf_{x}+yf_{y} = \rho (uf_{u}+vf_{v}) 
\] 
\end{solution}

\begin{exercise}
  Έστω $ u = u(x,y) $ και $ v = v(x,y) $, ομογενείς βαθμού $\rho$, με 
  $ u(x,y) \neq 0, v(x,y) \neq 0, \; \forall (x,y) \in \mathbb{R}^{2} $. 
  Να δείξετε ότι 
  \[
    udv -v du = \frac{1}{\rho} \pdv{(u,v)}{(x,y)} (xdy-ydx) 
    \quad \text{όπου} \quad  \pdv{(u,v)}{(x,y)} = 
    \begin{vmatrix}
      u_{x} & u_{y} \\ v_{x} & v_{y} 
    \end{vmatrix} 
  \] 
\end{exercise}
\begin{solution}
  \[
    \left.
      \begin{matrix}
        du=u_{x}dx+u_{y}dy \\
        dv=v_{x}dx+v_{y}dy
      \end{matrix} 
    \right\} \Rightarrow 
    \left.
      \begin{matrix}
        vdu=vu_{x}dx+vu_{y}dy \\ 
        udv=uv_{x}dx+uv_{y}dy
      \end{matrix} 
    \right\} \overset{(-)}{\Rightarrow} 
    udv -vdu = (uv_{x}-vu_{x})dx+(uv_{y}-vu_{y})dy 
  \] 

  \begin{align*}
    u \quad \text{ομογενής βαθμού $\rho$} \Rightarrow xu_{x}+yu_{y}= 
    \rho u \Rightarrow u &= \frac{1}{\rho} (xu_{x}+yu_{y})   \\
    v \quad \text{ομογενής βαθμού $\rho$} \Rightarrow xv_{x}+yv_{y}= 
    \rho v \Rightarrow u &= \frac{1}{\rho} (xv_{x}+yv_{y})   
  \end{align*} 
  Οπότε με αντικατάσταση, έχουμε:
  \begin{align*}
    udv - vdu &= \frac{1}{\rho} (u_{y}v_{x}-u_{x}v_{y})ydx+ \frac{1}{\rho}
    (u_{x}v_{y}-u_{y}v_{x})xdy \\ 
              &=- \frac{1}{\rho} (u_{x}v_{y}-u_{y}v_{x})ydx+ \frac{1}{\rho}
              (u_{x}v_{y}-u_{y}v_{x})xdy \\
              &= \frac{1}{\rho} (u_{x}v_{y}-u_{y}v_{x})(xdy-ydx) \\
              &= \frac{1}{\rho} \cdot \pdv{(u,v)}{(x,y)} (xdy-ydx)
  \end{align*}
\end{solution}




