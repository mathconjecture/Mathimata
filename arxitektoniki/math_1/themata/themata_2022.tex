\input{preamble_ask.tex}
\input{definitions_ask.tex}
\input{tikz}



\begin{document}

\begin{center}
\textcolor{Col1}{\minibox{\large\bfseries{Θέματα} 2022}}
\end{center}

\vspace{\baselineskip}

\begin{description}
  \item [Θέμα 1ο]
    Να βρεθεί το μήκος του τόξου ΑΒ της καμπύλης $ y^{2}=x(1- \frac{x}{3})^{2} $, 
    όπως φαίνεται στο παρακάτω σχήμα.
    \begin{tikzpicture}
        \node  (0) at (0, 0) {};
        \node at (0) [below] {$A$} ;
        \node  (1) at (2.75, 0) {};
        \node  (2) at (0, 2) {};
        \node at (1) [below] {$x$} ;
        \node at (2) [left] {$y$} ;
        \node  (3) at (2, 0) {};
        \node at (3) [below] {$B$} ;
        \draw [graph,bend left=90, looseness=1.25] (0.center) to 
          node[pos=0.7,above] {$y^{2}=x(1- \frac{x}{3})^{2}$} (3.center);
        \draw[-latex] (0.center) to (1.center);
        \draw[-latex] (0.center) to (2.center);
    \end{tikzpicture}
    \hfill Απ: $ 2 \sqrt{3} $ 

  \item [Θέμα 2ο] 
    Να βρεθεί ο όγκος του στερεού που προκύπτει από την περιστροφή της επίπεδης περιοχής 
    $R$ που περικλείεται από τις καμπύλες $ y^{2}=16-x, x-y=4 $ και $ y=4 $ για 
    $ x,y \geq 0 $, γύρω από τον οριζόντιο άξονα.

    \hfill Απ: $ V = \frac{169 \pi }{6} $ 

  \item [Θέμα 3ο] Να λυθεί το παρακάτω σύστημα:
    \[
      \left.
        \begin{matrix}
          2x+z=5 \\
          x+y-2z=6 \\
          6y-15z=21
        \end{matrix} 
      \right\} 
    \] 

    \hfill Απ: $ x = \frac{5-z}{2} , y = \frac{7+5z}{2} , z \in \mathbb{R} $  

  \item [Θέμα 4ο] 
    Να υπολογισtούν τα παρακάτω ολοκληρώματα:
    \[
      I_{1} = \int \frac{x^{4}-x^{3}-x-1}{x^3-x^{2}} \,{dx} \quad 
      \text{και} \quad I_{2} = \int _{- \infty}^{+\infty} 
      \frac{dx}{\mathrm{e}^{x} + \mathrm{e}^{-x}} \,{dx} 
    \] 

    \hfill Απ: $ I_{1} = \frac{x^{2}}{2} - 
    \frac{1}{x} + 2 \ln{\abs{\frac{x}{x-1} }} $ , $ I_{2}= \frac{\pi}{2} $  

  \item [Θέμα 5ο] 
Να βρεθεί η κλίση της καμπύλης $ x=2t^{3}+ \mathrm{e}^{t} $, $ y=3t^{2}-
\mathrm{e}^{t} $ στο σημείο $(1,-1)$.

\hfill Απ: $ \dv{y}{x} = 1 $  

\item [Θέμα 6ο] 
  4 μέτρα σύρμα πρόκειται να χρησιμοποιηθούν για την κατασκευή ενός τετραγώνου και ενός 
  κύκλου.  Πόσο σύρμα πρέπει να χρησιμoποιηθεί για το τετράγωνο και τον κύκλο, ώστε το 
  συνολικό εμβαδό να είναι μέγιστο.

  \hfill Απ: $ r = \frac{2}{4+ \pi} $, $ a = \frac{4}{4+ \pi} $  
      

\end{description}

\end{document}

