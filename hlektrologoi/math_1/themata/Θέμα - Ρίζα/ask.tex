\documentclass[a4paper,12pt]{article}

\usepackage[english,greek]{babel}
\usepackage[utf8]{inputenc}

\usepackage{amsmath,amssymb,amsthm,amsfonts}

\usepackage[margin=1in]{geometry}
\usepackage{enumerate}

\everymath{\displaystyle}

\thispagestyle{empty}
\begin{document}


\vspace{1cm}



\begin{enumerate}
\item Να υπολογισθεί η $\sqrt[6]{e}$ κατά προσέγγιση $0,0001$.
\end{enumerate}


Ισχύει: $\sqrt[6]{e}=e^{\frac{1}{6}}$

Επίσης: \[e^x=1+\frac{x}{1!}+\frac{x^2}{2!}+\frac{x^3}{3!}+\cdots + \frac{x^n}{n!}+\cdots\]

για κάθε $x\in \mathbb{R}$.

Επομένως και για $x=\frac{1}{6}$ έχουμε:

\[
e^{\frac{1}{6}}=1+\frac{1}{6}+\frac{1}{2!}\left(\frac{1}{6}\right)^2+\frac{1}{3!}\left(\frac{1}{6}\right)^3+\cdots+\frac{1}{n!}\left(\frac{1}{6}\right)^n+\cdots
\]

Παρατηρούμε ότι $\frac{a_{n+1}}{a_n}=\ldots \frac{1}{n+1}\cdot \frac{1}{6}<\frac{1}{2}, \forall n\in \mathbb{N}.$

Δηλαδή, ο λόγος κάθε όρου προς τον προηγούμενό του είναι μικρότερος του $\frac{1}{2}$.

 Επομένως αν πάρουμε μόνο τους $n$ πρώτους όρους της σειράς κάνουμε λάθος μικρότερο του $\frac{a_{n+1}}{1-\frac{1}{2}}=2a_{n+1}.$ Αν λοιπόν αρκεστούμε στους $5$ πρώτους όρους το λάθος είναι μικρότερο του διπλάσιου του $6$ου, δηλαδή του $2\cdot \frac{1}{5!}\left(\frac{1}{6}\right)^5=\frac{2}{5!6^5}<\frac{4}{10^6}$. Υπολογίζουμε τους 5 πρώτους όρους με προσέγγιση $\frac{1}{10^5}$ καθένα: 

\begin{itemize}
\item $1=1$ 
\item $\frac{1}{6}=0,16666$ 
\item $\frac{1}{2!}\left(\frac{1}{6}\right)^2=0,01388$
\item $\frac{1}{3!}\left(\frac{1}{6}\right)^3=0,00077$
\item $\frac{1}{4!}\left(\frac{1}{6}\right)^4=0,00003$
\end{itemize}

Σε καθέναν από τους προηγούμενους όρους το λάθος υπολογισμού είναι μικρότερο του $\frac{1}{10^5}$. Επομένως για το $S_5$ το λάθος είναι μικρότερο του $\frac{4}{10^5}$. Άρα το ολικό λάθος που κάνουμε όταν πάρουμε ως άθροισμα $S$ το $1,18134$ είναι μικρότερο του $\frac{4}{10^3}+\frac{4}{10^4}<\frac{4}{10^5}+\frac{4}{10^5}=0,00005$. Το σφάλμα αυτό αν το προσθέσουμε στο $1,18134$ δεν επηρεάζει το $4$ο δεκαδικό ψηφίο. Επομένως μετά από τα τέσσερα δεκαδικά ψηφία έχουμε $S=1,18134$ (με προσέγγιση 0,00001).
\end{document}