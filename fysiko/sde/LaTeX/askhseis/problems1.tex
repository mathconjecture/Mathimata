\input{preamble_ask.tex}
\input{definitions_ask.tex}


\pagestyle{askhseis}


\begin{document}

\begin{center}
  \minibox{\large\bfseries \textcolor{Col1}{Εφαρμογές των συνήθων διαφορικών εξισώσεων}}
\end{center}

\vspace{\baselineskip}

\section*{Μόρφωση σδε}

\begin{enumerate}
  \item Να βρεθεί η σδε που ικανοποιεί η οικογένεια των ομόκεντρων κύκλων ακτίνας $c>0$ 
    με κέντρο στο  $ (0,0) $ που δίνεται από τη σχέση:  $ x^{2} + y^{2} = c^{2} $,  
    όπου  $ y=y(x) $.
    \hfill Απ: $ x+yy'=0 $ 

  \item Να βρεθεί η σδε της οικογένειας των παραβολών $ y=cx^{2} $.
    \hfill Απ: $ y'=2 \frac{y}{x} $ 

  \item  Να βρεθεί η σδε που ικανοποιεί η μονοπαραμετρική οικογένεια καμπυλών  $ y(x) $ 
    που δίνεται από τη σχέση  $ cy+ \sin{y} = x $, $ y=y(x) $.
    \hfill Απ: $ \frac{1-y' \cos{y}}{y'} y + \sin{y} = x $ 

  \item  Να βρεθεί η σδε που ικανοποιεί η διπαραμετρική οικογένεια καμπυλών $ y(x) $, 
    που δίνεται από τη σχέση  $ y = c_{1}x + c_{2}x^{2} $, $y=y(x)$.
    \hfill Απ: $ x^{2}y''-2xy'+2y=0 $ 
\end{enumerate}

\section*{Ορθογώνιες τροχιές}

\begin{enumerate}
  \item Να βρεθούν οι ορθογώνιες τροχιές των καμπυλών:

    \begin{enumerate}[i)]
      \item $ y=c \mathrm{e}^{x} $ \hfill Απ: $ y^{2}+2x=c $ %Argyriou
      \item $ x^{2}+y^{2}=c^{2} $ \hfill Απ: $ y=cx $  %Argyriou
      \item $ y^{2}=cx^{3} $ \hfill Απ: $ 3y^{2}+2x^{2}=c $ 
      \item $ 2x^{2}+xy-2y^{2}=c $ \hfill Απ: $ y^{2}+8xy-x^{2}=c $ 
      \item $ x^{3}-y^{3}=c $ \hfill Απ: $ \frac{1}{y} + \frac{1}{x} =c $ 
    \end{enumerate}
\end{enumerate}


\section*{Μεταφορά θερμότητας}

\begin{enumerate}

  %Petropoulou p573
  \item Μια ζεστή καλοκαιρινή μέρα, η θερμοκρασία περιβάλλοντος είναι 
    $ \SI{36}{\degree C} $, ενώ η θερμοκρασία εντός ενός δωματίου είναι 
    $ \SI{26}{\degree C} $. Το δωμάτιο διαθέτει κλιματιστικό το οποίο τίθεται 
    σε λειτουργία και μέσα σε 10 λεπτά, η θερμοκρασία του δωματίου μειώνεται ένα βαθμό. 
    Σε πόση ώρα η θερμοκρασία του δωματίου θα φτάσει τους $ \SI{22}{\degree C} $; 
    \hfill Απ: $ 35,3 $ λεπτά 

    %Petropoulou p574
  \item Η θερμοκρασία στο εσωτερικό ενός δωματίου είναι $ \SI{21}{\degree C} $. 
    Τοποθετούμε στο μπαλκόνι έξω από το δωμάτιο ένα θερμόμετρο που προηγουμένως 
    βρισκόταν μέσα στο συγκεκριμένο δωμάτιο και ελέγχουμε τακτικά τις ενδείξεις του 
    θερμομέτρου. Μετά από 5 λεπτά, το θερμόμετρο δείχνει $ \SI{16}{\degree C} $, 
    ενώ μετά από άλλα 5 λεπτά δείχνει $ \SI{13}{\degree C} $. 
    Πόσο είναι η εξωτερική θερμοκρασία;
    \hfill Απ: $ \SI{8,5}{\degree C} $ 

    %Siafarikas p57
  \item Η θερμοκρασία ενός φλιτζανιού με τσάι είναι αρχικά $ \SI{100}{\degree C} $ και 
    η θερμοκρασία του δωματίου μέσα στο οποίο βρίσκεται είναι σταθερή και ίση με 
    $ \SI{30}{\degree C} $. 
    \begin{enumerate}[i)]
      \item Αν μετά από 15 λεπτά η θερμοκρασία του τσαγιού είναι $ \SI{80}{\degree C} $ 
        να υπολογιστεί ο χρόνος που απαιτείται ώστε η θερμοκρασία του να φτάσει τους 
        $ \SI{50}{\degree C} $. 
      \item Υποθέτουμε, τώρα, ότι το ίδιο φλιτζάνι τσάι, πρώτα το αφήνουμε να κρυώσει 
        για 20 λεπτά μέσα στο δωμάτιο. Στη συνέχεια το βάζουμε σε ένα ψυγείο 
        με θερμοκρασία $ \SI{15}{\degree C} $. Αν μετά από 10 λεπτά στο ψυγείο, 
        η θερμοκρασία του τσαγιού έχει πέσει στους $ \SI{60}{\degree C} $, να 
        βρεθεί η θερμοκρασία του μετά από 1 ώρα. 
        \hfill Απ: 
        \begin{enumerate*}[i)] 
          \item 56 λεπτά \item $ \SI{26}{\degree C} $  
        \end{enumerate*}
    \end{enumerate}
\end{enumerate}

\section*{Διάσπαση Ραδιενεργών Πυρήνων}

\begin{enumerate}
  %Siafarikas p204
  \item Έστω ότι ο ρυθμός με τον οποίο διασπάται μια ραδιενεργός ουσία είναι ανάλογος 
    προς την υπάρχουσα ποσότητά της. Ένα συγκεκριμένο δείγμα με περιεκτικότητα $ 50 \% $ 
    της ουσίας διασπάται πλήρως σε μια περίοδο $ 1600 $ ετών (πχ. ράδιο-226). Ποιο 
    ποσοστό του αρχικού δείγματος θα διασπαστεί σε $ 800 $ έτη; Σε πόσα χρόνια 
    θα απομείνει μόνο το 1/5 της αρχικής ποσότητας; 
    \hfill Απ: $ k\approx 0,000433 $, $ N(800) = 0,70723 $, $ t_{1}=3717 $ 


    %Siafarikas p204
  \item Το Θόριο-234 είναι μια ραδιενεργή ουσία που διασπάται με ρυθμό ανάλογο της 
    αρχικής ποσότητας. Υποθέστε ότι $ \SI{1}{gr} $ αυτού του υλικού ελαττώνεται στα 
    $ \SI{0,8}{gr} $ σε μια εβδομάδα. Βρείτε το χρόνο υποδιπλασιασμού του Θορίου-234. 
    Πόση ποσότητα Θορίου-234 θα έχει απομείνει μετά από 10 βδομάδες;
    \hfill Απ: $ t_{1/2} = 3,1$, $ N(10)=0,107528 $  

    %Siafarikas p214
  \item Με τη βοήθεια της χημικής ανάλυσης, υπολογίστηκε ότι η εναπομείνουσα ποσότητα 
    άνθρακα-14 που βρέθηκε στα δείγματα του ξυλάνθρακα που πάρθηκαν από το σπήλαιο 
    Lascaux ήταν $ 15 \% $ της αρχικής ποσότητας, τη στιγμή που το δέντρο πέθανε. Αν 
    είναι γνωστό ότι ο χρόνος ημίσειας ζωής του άνθρακα-14 είναι περίπου 5600 έτη και 
    ότι η ποσότητα $ N(t) $, του άνθρακα-14 σε ένα δείγμα ξυλάνθρακα ικανοποιεί την 
    εξίσωση $ N'(t)=-kN(t) $, να βρεθεί η ηλικία των τοιχογραφιών του σπηλαίου Lascaux.
    \hfill Απ: $ t_{1} = 15327 $ 

    %Siafarikas p215
  \item Σε μια αρχαιολογική έρευνα οι επιστήμονες βρήκαν ένα αρχαίο εργαλείο κοντά 
    σε ένα απολιθωμένο ανθρώπινο οστό. Αν το εργαλείο και το απολίθωμα περιέχουν 
    $ 65 \% $ και $ 60 \% $ της αρχικής ποσότητας του άνθρακα-14, αντίστοιχα, 
    να προσδιορισθεί αν είναι δυνατόν το εργαλείο να είχε χρησιμοποιηθεί από αυτόν 
    τον άνθρωπο.  Δίνεται ότι ο χρόνος ημίσειας ζωής του άνθρακα-14 είναι περίπου 
    5600 έτη.  
    \hfill Απ: $ t_{\text{εργ.}} = 3480 $, $ t_{\text{ανθρ.}} = 4127 $  

    %Siafarikas p216
  % \item Το 1960 οι New York Times με άρθρο τους ανακοίνωσαν ότι "οι αρχαιολόγοι 
  %   υποστηρίζουν ότι η κοινωνία των Σουμέριων κατοικούσε στην κοιλάδα του Τίγρη πριν από 
  %   5000 χρόνια". Υποθέτωντας ότι οι αρχαιολόγοι χρησιμοποίησαν την μέθοδο του 
  %   άνθρακα-14 για να χρονολογήσουν την περιοχή, υπολογίστε το ποσοστό του 
  %   άνθρακα-14 που βρέθηκε στα δείγματά τους. Δίνεται ότι ο χρόνος ημίσειας ζωής 
  %   του άνθρακα-14 είναι περίπου 5600 έτη.
  %   \hfill Απ: $ 53,8 \% $ 
\end{enumerate}


\section*{Προβλήματα Μίξης}

%Siafarikas p78
\begin{enumerate}
  \item Αρχικά $ \SI{50}{kg} $ αλατιού διαλύονται σε δεξαμενή όγκου 
    $ \SI{300}{lt} $.  Αργότερα ένα διάλυμα αλατιού με συγκέντρωση 
    $ \SI{2}{kg} $ ανά λίτρο διαλύτη, εισέρχεται στη δεξαμενή με ρυθμό 
    $ \SI{3}{lt/min} $. Το διάλυμα της δεξαμενής αναδεύεται καλά και εξέρχεται από 
    αυτή με με τον ίδιο ρυθμό.
    \begin{enumerate}[i)]
      \item Βρείτε την ποσότητα του αλατιού στη δεξαμενή για οποιαδήποτε χρονική 
        στιγμή, καθώς και την οριακή της τιμή.
      \item Βρείτε τη συγκέντρωση του αλατιού στη δεξαμενή για οποιαδήποτε χρονική 
        στιγμή, καθώς και την οριακή της τιμή.
    \hfill Απ: 
    \begin{enumerate*}[i),itemjoin=\hspace{10pt}]
      \item $ y(t)=600-550 \mathrm{e}^{- \frac{t}{110}} $, $y_{\text{ορ.}}=600$
      \item $ C(t)=2- \frac{11}{6} \mathrm{e}^{- \frac{t}{110}} $, $C_{\text{ορ.}}=2$
    \end{enumerate*}
    \end{enumerate}

    %Siafarikas p79
  \item Μια δεξαμενή αρχικά περιέχει $ \SI{200}{lt} $ καθαρό νερό. Αργότερα ένα διάλυμα 
    ζάχαρης, άγνωστης συγκέντρωσης, εισέρχεται στη δεξαμενή με ρυθμό $ \SI{2}{lt/min} $. 
    Αφού αναδεύεται καλά εξέρχεται από τη δεξαμενή με τον ίδιο ρυθμό. 
    Μετά από 120 λεπτά η συγκέντρωση της ζάχαρης μέσα στη δεξαμενή είναι 
    $ \SI{1,4}{kg/lt} $. Ποια η συγκέντρωση του εισερχόμενου διαλύματος; 
    \hfill Απ: $ C=2 $

    %Siafarikas p80
  \item Σε μια δεξαμενή που περιέχεi $ \SI{100}{lt} $ καθαρό νερό, ένας εργάτης πρόσθεσε 
    $ \SI{20}{kg} $ αλάτι ενώ έπρεπε να προσθέσει μόνο $ \SI{10}{kg} $. Για να διορθώσει 
    το λάθος του ο εργάτης άρχισε να προσθέτει καθαρό νερό με ρυθμό $ \SI{3}{lt/mi} $ 
    ενώ διάλυμα αλατιού εξερχόταν από τη δεξαμενή με τον ίδιο ρυθμό. Πόσος χρόνος θα
    χρειαστεί μέχρι η δεξαμενή να περιέχει τη σωστή ποσότητα αλατιού;
    \hfill Απ: $ 23,1 $ 
\end{enumerate}

\end{document}
