\documentclass[a4paper,12pt]{article}
\usepackage{etex}
%%%%%%%%%%%%%%%%%%%%%%%%%%%%%%%%%%%%%%
% Babel language package
\usepackage[english,greek]{babel}
% Inputenc font encoding
\usepackage[utf8]{inputenc}
%%%%%%%%%%%%%%%%%%%%%%%%%%%%%%%%%%%%%%

%%%%% math packages %%%%%%%%%%%%%%%%%%
\usepackage{amsmath}
\usepackage{amssymb}
\usepackage{amsfonts}
\usepackage{amsthm}
\usepackage{proof}

\usepackage{physics}

%%%%%%% symbols packages %%%%%%%%%%%%%%
\usepackage{bm} %for use \bm instead \boldsymbol in math mode 
\usepackage{dsfont}
\usepackage{stmaryrd}
%%%%%%%%%%%%%%%%%%%%%%%%%%%%%%%%%%%%%%%


%%%%%% graphicx %%%%%%%%%%%%%%%%%%%%%%%
\usepackage{graphicx}
\usepackage{color}
%\usepackage{xypic}
\usepackage[all]{xy}
\usepackage{calc}
\usepackage{booktabs}
\usepackage{minibox}
%%%%%%%%%%%%%%%%%%%%%%%%%%%%%%%%%%%%%%%

\usepackage{enumerate}

\usepackage{fancyhdr}
%%%%% header and footer rule %%%%%%%%%
\setlength{\headheight}{14pt}
\renewcommand{\headrulewidth}{0pt}
\renewcommand{\footrulewidth}{0pt}
\fancypagestyle{plain}{\fancyhf{}
\fancyhead{}
\lfoot{}
\rfoot{\small \thepage}}
\fancypagestyle{vangelis}{\fancyhf{}
\rhead{\small \leftmark}
\lhead{\small }
\lfoot{}
\rfoot{\small \thepage}}
%%%%%%%%%%%%%%%%%%%%%%%%%%%%%%%%%%%%%%%

\usepackage{hyperref}
\usepackage{url}
%%%%%%% hyperref settings %%%%%%%%%%%%
\hypersetup{pdfpagemode=UseOutlines,hidelinks,
bookmarksopen=true,
pdfdisplaydoctitle=true,
pdfstartview=Fit,
unicode=true,
pdfpagelayout=OneColumn,
}
%%%%%%%%%%%%%%%%%%%%%%%%%%%%%%%%%%%%%%

\usepackage[space]{grffile}

\usepackage{geometry}
\geometry{left=25.63mm,right=25.63mm,top=36.25mm,bottom=36.25mm,footskip=24.16mm,headsep=24.16mm}

%\usepackage[explicit]{titlesec}
%%%%%% titlesec settings %%%%%%%%%%%%%
%\titleformat{\chapter}[block]{\LARGE\sc\bfseries}{\thechapter.}{1ex}{#1}
%\titlespacing*{\chapter}{0cm}{0cm}{36pt}[0ex]
%\titleformat{\section}[block]{\Large\bfseries}{\thesection.}{1ex}{#1}
%\titlespacing*{\section}{0cm}{34.56pt}{17.28pt}[0ex]
%\titleformat{\subsection}[block]{\large\bfseries{\thesubsection.}{1ex}{#1}
%\titlespacing*{\subsection}{0pt}{28.80pt}{14.40pt}[0ex]
%%%%%%%%%%%%%%%%%%%%%%%%%%%%%%%%%%%%%%

%%%%%%%%% My Theorems %%%%%%%%%%%%%%%%%%
\newtheorem{thm}{Θεώρημα}[section]
\newtheorem{cor}[thm]{Πόρισμα}
\newtheorem{lem}[thm]{λήμμα}
\theoremstyle{definition}
\newtheorem{dfn}{Ορισμός}[section]
\newtheorem{dfns}[dfn]{Ορισμοί}
\theoremstyle{remark}
\newtheorem{remark}{Παρατήρηση}[section]
\newtheorem{remarks}[remark]{Παρατηρήσεις}
%%%%%%%%%%%%%%%%%%%%%%%%%%%%%%%%%%%%%%%




\newcommand{\vect}[2]{(#1_1,\ldots, #1_#2)}
%%%%%%% nesting newcommands $$$$$$$$$$$$$$$$$$$
\newcommand{\function}[1]{\newcommand{\nvec}[2]{#1(##1_1,\ldots, ##1_##2)}}

\newcommand{\linode}[2]{#1_n(x)#2^{(n)}+#1_{n-1}(x)#2^{(n-1)}+\cdots +#1_0(x)#2=g(x)}

\newcommand{\vecoffun}[3]{#1_0(#2),\ldots ,#1_#3(#2)}

\newcommand{\mysum}[1]{\sum_{n=#1}^{\infty}


\everymath{\displaystyle}
\pagestyle{askhseis}


\begin{document}


\begin{center}
  \minibox{\large\bfseries \textcolor{Col1}{Ασκήσεις Στo Μετασχηματισμό Laplace}}
\end{center}

\vspace{\baselineskip}

\begin{enumerate}
\item Να λυθούν οι ολοκληρoδιαφορικές εξισώσεις:
\begin{enumerate}[i)]

\item $y'(t)=\sin t+\int_{0}^{t}y(t-\xi)\cos\xi\, d\xi,\quad y(0)=0$ \hfill Απ: $y(t)=\frac{1}{2}t$
\item $\phi''(t)+\int_{0}^{t}e^{2(t-\xi)}\phi'(\xi)\,d\xi=e^{2t},\quad \phi(0)=0, \phi'(0)=1$ \hfill Απ: $\phi(t)=e^t-1$
\item $g'(t)-g(t)+\int_{0}^{t}(t-x)g'(x)\,dx - \int_{0}^{t}g(x)\,dx = t \quad, g(0)=1$\hfill Απ: $g'(t)=-e^t$

\end{enumerate}

\item Να λυθεί το σύστημα των ολοκληρωδιαφορικών εξισώσεων:

\begin{enumerate}[i)]

\item $\begin{cases}
y_1(t)=1-\cos t+\int_{0}^{t}y_2(x)\,dx \\
y_2(t)=\sin t+\int_{0}^{t}y_1(x)\,dx
\end{cases}$ \hfill Απ: \begin{tabular}[t]{l}$y_1(t)=\frac{1}{2}e^t-\frac{1}{2}e^{-t}-\cos t$ \\
$y_2(t)=\frac{1}{2}e^t-\frac{1}{2}e^{-t}$
\end{tabular}

\vspace{\baselineskip}

\item $\begin{cases}
y_1(t)=\cos t +\int_{0}^{t}y_2(x)\,dx \\
y_2(t)=\sin t +\int_{0}^{t}y_1(x)\,dx
\end{cases}$ \hfill Απ: \begin{tabular}[t]{l}$y_1(t)=\cos t$ \\
$y_2(t)=0$
\end{tabular}

\vspace{\baselineskip}

\item $\begin{cases}
\phi_1(t)=t+\int_{0}^{t}\phi_2(x)\,dx \\
\phi_2(t)=1-\int_{0}^{t}\phi_1(x)\,dx \\
\phi_3(t)=\sin t-\frac{1}{2}\int_{0}^{t}(t-x)\phi_1(x)\,dx
\end{cases}$ \hfill Απ: $\begin{tabular}[t]{l}$\phi_1(t)=2\sin t$ \\
$\phi_2(t)=2\cos t -1$ \\
$\phi_3(t)=t$
\end{tabular}$

\end{enumerate}

\item Εξετάστε αν οι παρακάτω συναρτήσεις, μπορεί να είναι ο μετασχηματισμός Laplace κάποιων συνεχών συναρτήσεων $f(t)$ εκθετικής τάξης. Δικαιολογήστε την απάντησή σας.

\begin{enumerate}[i)]

\item $F(s) = \frac{e^{-s}}{(s-2)^4}$ \hfill Απ: Ναι \; $f(t)=\frac{1}{3!}e^{2(t-1)}(t-1)^3v_1(t)$
\item $F(s)=\frac{1}{s^2(s+2)}$ \hfill Απ: Ναι \; $f(t)=te^{-t}+e^{-2t}-e^{-t}$
\item $F(s)=\frac{s^3}{s^3+2s^2+s+1}$ \hfill Απ: Οχι


\end{enumerate}

\item Αποδείξτε ότι ο μετασχηματισμός {Laplace} της μεταφοράς της συνάρτησης $f(t), t\geq 0$ στο $c>0$ είναι:
\[
\mathcal{L}[v_c(t)f(t-c)]=e^{-s}F(s)
\]
όπου $v_c(t)$ η βηματική συνάρτηση (Heaviside) και $F$ ο μετασχηματισμός Laplace της $f$.

\newpage

\item Να βρείτε το μετασχηματισμό Laplace των συναρτήσεων:

\begin{enumerate}[i)]

\item $f(t)=\begin{cases}
1 & ,0\leq t<1 \\
t-1 & ,1\leq t<2 \\
0 & ,2\leq t <3 \\
\sin(t-3) & ,t\geq 0
\end{cases}$ \hfill Απ: ${\scriptstyle{F(s)=\frac{1}{s}+(\frac{1}{s^2}-\frac{1}{s})e^{-s}-(\frac{1}{s^2}+\frac{1}{s})e^{-2s}+\frac{1}{s^2+1}e^{-3s}}}$


\item $f(t)=\begin{cases}
e^{2t} & ,0\leq t<1 \\
4 & , t\geq 1
\end{cases}$\hfill Απ: $f(t)=\frac{1}{s-2}-(\frac{e^2}{S-2}-\frac{4}{s})e^{-s}$


\end{enumerate}

\item Αποδείξτε ότι ο μετασχηματισμός Laplace της συνάρτησης $t^2f(t)$ είναι ίσος με $F''(s)$, όπου $F(s)$ ο μετασχηματισμός Laplace της $f$.

Να λυθούν τα παρακάτω προβλήματα αρχικών τιμών:

\begin{enumerate}[i)]

\item $y''(t)-3y'(t)+2y(t)=\begin{cases}
0 & , 0\leq t <2 \\
1 & , 2\geq t
\end{cases}
\quad , y(0)=1, y'(0)=0$.

\hfill Απ: $y(t)=v_2(t)(\frac{1}{2}(t-2)-e^{t-2}+\frac{1}{2}e^{2(t-2)})+2e^t-e^{2t}$

\end{enumerate}

\item Να βρεθεί ο μετασχηματισμός Laplace των συναρτήσεων: 

\begin{enumerate}[i)]

\item $f(t)=e^t\int_{0}^{t}e^{-x}\cos t\,dt$ \hfill Απ: $F(s)=\frac{1}{s-1}\cdot \frac{s}{s^2+1}$

\end{enumerate}

\item Αν υπάρχει ο μετασχηματισμός Laplace $F(s)$ της συνάρτησης $f(t)$ και ισχύει ότι $f(0)=0$, $\lim\limits_{s\to+\infty}s^2F(s)=1$ και υπάρχει ο μετασχηματισμός Laplace της $f'(t)$, να δειχθει ότι $f'(0)=1$.

\end{enumerate}





\end{document}
