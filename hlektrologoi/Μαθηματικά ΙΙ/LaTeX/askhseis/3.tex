\documentclass[a4paper,12pt]{article}
\usepackage{etex}
%%%%%%%%%%%%%%%%%%%%%%%%%%%%%%%%%%%%%%
% Babel language package
\usepackage[english,greek]{babel}
% Inputenc font encoding
\usepackage[utf8]{inputenc}
%%%%%%%%%%%%%%%%%%%%%%%%%%%%%%%%%%%%%%

%%%%% math packages %%%%%%%%%%%%%%%%%%
\usepackage{amsmath}
\usepackage{amssymb}
\usepackage{amsfonts}
\usepackage{amsthm}
\usepackage{proof}

\usepackage{physics}

%%%%%%% symbols packages %%%%%%%%%%%%%%
\usepackage{bm} %for use \bm instead \boldsymbol in math mode 
\usepackage{dsfont}
\usepackage{stmaryrd}
%%%%%%%%%%%%%%%%%%%%%%%%%%%%%%%%%%%%%%%


%%%%%% graphicx %%%%%%%%%%%%%%%%%%%%%%%
\usepackage{graphicx}
\usepackage{color}
%\usepackage{xypic}
\usepackage[all]{xy}
\usepackage{calc}
\usepackage{booktabs}
\usepackage{minibox}
%%%%%%%%%%%%%%%%%%%%%%%%%%%%%%%%%%%%%%%

\usepackage{enumerate}

\usepackage{fancyhdr}
%%%%% header and footer rule %%%%%%%%%
\setlength{\headheight}{14pt}
\renewcommand{\headrulewidth}{0pt}
\renewcommand{\footrulewidth}{0pt}
\fancypagestyle{plain}{\fancyhf{}
\fancyhead{}
\lfoot{}
\rfoot{\small \thepage}}
\fancypagestyle{vangelis}{\fancyhf{}
\rhead{\small \leftmark}
\lhead{\small }
\lfoot{}
\rfoot{\small \thepage}}
%%%%%%%%%%%%%%%%%%%%%%%%%%%%%%%%%%%%%%%

\usepackage{hyperref}
\usepackage{url}
%%%%%%% hyperref settings %%%%%%%%%%%%
\hypersetup{pdfpagemode=UseOutlines,hidelinks,
bookmarksopen=true,
pdfdisplaydoctitle=true,
pdfstartview=Fit,
unicode=true,
pdfpagelayout=OneColumn,
}
%%%%%%%%%%%%%%%%%%%%%%%%%%%%%%%%%%%%%%

\usepackage[space]{grffile}

\usepackage{geometry}
\geometry{left=25.63mm,right=25.63mm,top=36.25mm,bottom=36.25mm,footskip=24.16mm,headsep=24.16mm}

%\usepackage[explicit]{titlesec}
%%%%%% titlesec settings %%%%%%%%%%%%%
%\titleformat{\chapter}[block]{\LARGE\sc\bfseries}{\thechapter.}{1ex}{#1}
%\titlespacing*{\chapter}{0cm}{0cm}{36pt}[0ex]
%\titleformat{\section}[block]{\Large\bfseries}{\thesection.}{1ex}{#1}
%\titlespacing*{\section}{0cm}{34.56pt}{17.28pt}[0ex]
%\titleformat{\subsection}[block]{\large\bfseries{\thesubsection.}{1ex}{#1}
%\titlespacing*{\subsection}{0pt}{28.80pt}{14.40pt}[0ex]
%%%%%%%%%%%%%%%%%%%%%%%%%%%%%%%%%%%%%%

%%%%%%%%% My Theorems %%%%%%%%%%%%%%%%%%
\newtheorem{thm}{Θεώρημα}[section]
\newtheorem{cor}[thm]{Πόρισμα}
\newtheorem{lem}[thm]{λήμμα}
\theoremstyle{definition}
\newtheorem{dfn}{Ορισμός}[section]
\newtheorem{dfns}[dfn]{Ορισμοί}
\theoremstyle{remark}
\newtheorem{remark}{Παρατήρηση}[section]
\newtheorem{remarks}[remark]{Παρατηρήσεις}
%%%%%%%%%%%%%%%%%%%%%%%%%%%%%%%%%%%%%%%




\input{definitions_ask.tex}

\pagestyle{askhseis}
\everymath{\displaystyle}
\renewcommand{\vec}{\mathbf}

\begin{document}

\begin{center}
  \minibox{\large\bf \textcolor{Col1}{Ασκήσεις Κλίση, Απόκλιση, Στροβιλισμός}}
\end{center}

\vspace{\baselineskip}

\begin{enumerate}

\item Να αποδείξετε τις παρακάτω σχέσεις για τους διαφορικούς τελεστές

\begin{enumerate}[i)]
\item $\curl(\grad f)=0$
\item $\div(\curl F)=0$
\end{enumerate}

\item Δίνεται το βαθμωτό πεδίο 
  $f(x,y,z)=3y^4z^2\vec{i}+4x^3z^2\vec{j}-3x^2y^2\vec{k}$. Να δείξετε ότι το 
  πεδίο είναι σωληνοειδές.

\item Δίνεται το διανυσματικό πεδίο 
  $\boldsymbol{F}(x,y,z)=(x^4+yz^3)\vec{i}+x^3\vec{j}+(-1+x^2+y^2)\vec{k}$. 
  Να υπολογιστούν:
\begin{enumerate}[i)]
\item $\div F$
\item $\grad(\div F)$
\item $\curl(\curl F)$

\hfill Απ: \begin{tabular}{l}
  $\mathrm{i)}\; \div F=4x^3$ \\
  $\mathrm{ii)}\; \grad(\div F)=12x^2\vec{i}+0\vec{j}+0\vec{k}$ \\
  $\mathrm{iii)}\; \curl(\curl F)=(-6yz)\vec{i}+(-6x)\vec{j}+(-2y^2-2x^2)\vec{k}$
\end{tabular}

\end{enumerate}

\item Δίνεται το διανυσματικό πεδίο $\boldsymbol{F}=x\vec{i}+y\vec{j}+z\vec{k}$. 
\begin{enumerate}[i)]
\item Να δείξετε ότι το πεδίο είναι αστρόβιλο.
\item Να βρειτε το $\div F$.
\item Να βρειτε το $\grad(\div F)$.
\end{enumerate}

\hfill Απ: \begin{tabular}{l}
  $\mathrm{ii)}\; \div F=3$ \\
  $\mathrm{iii)}\; \grad(\div F)=0$

\end{tabular}

\item Θεωρούμε το διανυσματικό πεδίο
\[
\boldsymbol{F}(x,y,z)=e^{2x}\vec{i}+f(x,z)\vec{j}+e^z\vec{k}
\]
\begin{enumerate}[i)]
\item Να βρεθεί η συνάρτηση $f(x,z)$ ώστε το πεδίο να είναι συντηρητικό ($\curl F=0$).
\item Να βρείτε τη συνάρτηση δυναμικού.

\hfill Απ: \begin{tabular}{l}
  $\mathrm{i)}\; f(x,z)=c$ (σταθ.) \\
  $\mathrm{ii)}\;f(x,y,z)=\frac{e^{2x}}{2}+cy+e^z-\frac{3}{2}$
\end{tabular}

\end{enumerate}
\end{enumerate}

\end{document}
