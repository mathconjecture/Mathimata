\input{preamble_ask.tex}
\input{definitions_ask.tex}


\pagestyle{askhseis}


\begin{document}

\begin{center}
  \minibox{\large\bfseries \textcolor{Col1}{Εφαρμογές των συνήθων διαφορικών εξισώσεων}}
\end{center}

\vspace{\baselineskip}

\textcolor{Col1}{\textbf{Πρόβλημα 2}}

Έστω $ T_{\pi } $ η θερμοκρασία περιβάλλοντος. Από το νόμο του Νεύτωνα, έχουμε: 
\[
  - \dv{T}{t} =  k (T - T_{\pi}) \quad \text{με λύση προκύπτει} \quad λύση 
  T(t) = T_{\pi} + c \mathrm{e}^{-kt} 
\]
\[
  T(0) = 21 \Rightarrow T_{\pi} + c = 21 \Rightarrow c = 21 - T_{\pi} 
  \quad \text{άρα} \quad  T(t) = T_{\pi} + (21-T_{\pi }) \mathrm{e}^{-kt} 
\] 
\[
  T(5) = 16 \Rightarrow
  T_{\pi} + (21-T_{pi}) \mathrm{e}^{-5k} = 16 \Rightarrow \mathrm{e}^{-5k} =
  \frac{16-T_{\pi}}{21 - T_{\pi}} 
\] 
\[
  T(0) = 13 \Rightarrow T_{\pi} + c \mathrm{e}^{-10k} = 13 \Rightarrow T_{\pi} +
  (21-T_{\pi}) (\mathrm{e}^{-5k} )^{2} = 13 \overset{}{\Rightarrow} 
  T_{\pi} + (21-T_{\pi}) \left(\frac{16-T_{\pi}}{21-T_{\pi}} \right)^{2} = 13
\] 
Από όπου μπορούμε να λύσουμε ως προς $ T_{\pi} $ και προκύπτει 
\[
  T_{\pi} + \frac{256 - 32 T_{\pi } + T_{\pi }^{2}}{21-T_{\pi}} = 13 \Rightarrow 2
  T_{\pi } = 17 \Rightarrow T_{\pi } 8.5
\] 

\textcolor{Col1}{\textbf{Πρόβλημα 3}}

Έστω $ T_{\pi } $ η θερμοκρασία περιβάλλοντος. Από το νόμο του Νεύτωνα, έχουμε: 
\[
  - \dv{T}{t} =  k (T - 30) \quad \text{με λύση προκύπτει} \quad λύση 
  T(t) = 30 + c \mathrm{e}^{-kt} 
\]
\[
  T(0) = 100 \Rightarrow 30 + c = 100 \Rightarrow c = 70
  \quad \text{άρα} \quad  T(t) = 30 + 70 \mathrm{e}^{-kt} 
  Τ(15) = 80 \Rightarrow \mathrm{e}^{-15k} = \frac{5}{7} \Rightarrow \mathrm{e}^{-k}
  = \Bigl(\frac{5}{7}\Bigr)^{1/15}
\] 
Άρα
\[
  T(t) = 30 + 70 \Bigl(\frac{5}{7} \Bigr)^{t/15}
\]
Ο ζητούμενος χρόνος προκύπτει από τη λύση της εξίσωσης 
\[
  T(t) = 50 \Leftrightarrow 30 + 70 \Bigl(\frac{5}{7} \Bigr)^{t/15} = 50 
  \Leftrightarrow 
  \Bigl(\frac{5}{7}\Bigr)^{t/15} = \frac{2}{7} \Rightarrow t = 
  \frac{15 \ln{(2/7)}}{\ln{(5/7)}} 
  \approx 56
\] 
Μετά από 20 λεπτά στο δωμάτιο, η θερμοκρασία του τσαγιού είναι
\[
  T(20) = 30 + 70 \Bigl(\frac{5}{7} \Bigr)^{20/15} \approx 75
\] 
Η θερμοκρασία μέσα στο ψυγείο είναι $ \SI{15}{\celsius} $, οπότε το νέο πρόβλημα αρχικών
τιμών είναι 
\begin{align*}
  \dv{T}{t} = - k (T-15) \quad \text{με Α.Σ.} \quad T(0) = 75
\end{align*}
με λύση, όπως προηγουμένως 
\[
  T(t) = 15 + 60 \mathrm{e}^{-kt} 
\] 
Επιπλέον, ισχύει ότι 
\[
  T(10) = 60 \Rightarrow \cdots \Rightarrow \mathrm{e}^{-k} = 
  \Bigl(\frac{3}{4}\Bigr)^{1/10} 
\] 
Άρα 
\[
  T(t) = 15 + 60 \Bigl(\frac{3}{4} \Bigr)^{t/10} 
\]
και άρα η θερμοκρασία του τσαγιού μετά από 60 λεπτά θα είναι
\[
  T(60) = 15 + 60 \Bigl(\frac{3}{4}\Bigr)^{60/10} \approx 26
\]
\end{document}
