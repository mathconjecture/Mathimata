\input{preamble_ask.tex}
\newcommand{\vect}[2]{(#1_1,\ldots, #1_#2)}
%%%%%%% nesting newcommands $$$$$$$$$$$$$$$$$$$
\newcommand{\function}[1]{\newcommand{\nvec}[2]{#1(##1_1,\ldots, ##1_##2)}}

\newcommand{\linode}[2]{#1_n(x)#2^{(n)}+#1_{n-1}(x)#2^{(n-1)}+\cdots +#1_0(x)#2=g(x)}

\newcommand{\vecoffun}[3]{#1_0(#2),\ldots ,#1_#3(#2)}

\newcommand{\mysum}[1]{\sum_{n=#1}^{\infty}



% \DeclareMathOperator{\r}{r}

\begin{document}

\chapter*{Γραμμικά Συστήματα}

\begin{dfn}
  Έστω $ A,B \in \textbf{M}_{m \times n}(\mathbb{K}) $. Λέμε ότι ο πίνακας $B$ είναι 
  \textcolor{Col1}{γραμμοϊσοδύναμος} με τον πίνακα $A$ και γράφουμε $ B \sim A $, 
  αν ο $B$ μπορεί να προκύψει από τον $A$, εφαρμόζοντας μια πεπερασμένη ακολουθία 
  στοιχειωδών μετασχηματισμών γραμμών (γραμμοπράξεων).
\end{dfn}

\begin{rem}
\item {}
  \begin{enumerate}[i)]
    \item $ A \sim A $
    \item $ B \sim A \Rightarrow A \sim B$
    \item $ A \sim B \Rightarrow B \sim C \Rightarrow A \sim C$
  \end{enumerate}
\end{rem}

\begin{dfn}
  Έστω $ A \in \textbf{M}_{m \times n}(\mathbb{K}) $ και έστω ο $A$ έχει τουλάχιστον μια
  μη μηδενική γραμμή. Το πρώτο μη μηδενικό στοιχείο κάθε μη μηδενικής γραμμής του 
  $A$ λέγεται \textcolor{Col1}{ηγετικό} στοιχείο (ή \textcolor{Col1}{οδηγό} στοιχείο) 
  της γραμμής.
\end{dfn}

\begin{dfn}
  Ένας $ m \times n $ πίνακας $A$ (με στοιχεία στο $\mathbb{K}$, όπου $ \mathbb{K} \in
  \{ \mathbb{Q}, \mathbb{R}, \mathbb{C} \} $) ονομάζεται \textcolor{Col1}{κλιμακωτός} 
  πίνακας αν ικανοποιεί τις παρακάτω τρεις συνθήκες:
  \begin{enumerate}[i)]
    \item Το ηγετικό στοιχείο σε κάθε μη μηδενική γραμμή, είναι 1.
    \item Το ηγετικό 1 σε κάθε μη μηδενική γραμμή βρίσκεται στα δεξιά του ηγετικού 1 
      της κάθε προηγούμενης γραμμής
    \item Οι μη μηδενικές γραμμές βρίσκονται πιο κάτω από τις μηδενικές γραμμές.
  \end{enumerate}
  Ένας κλιμακωτός πίνακας λέγεται \textcolor{Col1}{ανηγμένος κλιμακωτός} πίνακας αν 
  \begin{enumerate}[i), start=4]
    \item Το ηγετικό 1 σε κάθε μη μηδενική γραμμή είναι το μόνο μη μηδενικό στοιχείο στη
      στήλη στην οποία βρίσκεται το 1.
  \end{enumerate}
\end{dfn}

\begin{thm}
  Για κάθε πίνακα $ A \in \textbf{M}_{m \times n}(\mathbb{K}) $ υπάρχει ακριβώς ένας 
  $ m \times n $ ανηγμένος κλιμακωτός πίνακας με στοιχεία στο $ \mathbb{K} $, ο οποίος 
  είναι γραμμοϊσοδύναμος με τον $A$.
\end{thm}

\begin{rem}
  Αν $ A \in \textbf{M}_{m \times n}(\mathbb{K}) $, τότε με $ R_{A} $ συμβολίζουμε τον 
  (μοναδικό) $ m \times n $ ανηγμένο κλιμακωτό πίνακα, ο οποίος είναι γραμμοϊσοδύναμος 
  με τον $A$.
\end{rem}

\subsection*{Μέθοδος Gauss-Jordan}

Τα βήματα της μεθόδου αυτής είναι τα ακόλουθα:
\setlist[description]{labelindent=0em,widest=ΒΗΜΑ 4,labelsep*=1em,leftmargin=*}
\begin{description}
  \item [ΒΗΜΑ 1] Ξεκίνησε με τη θέση $ (1,1) $ του δοθέντος πίνακα
  \item [ΒΗΜΑ 2] Αν αυτή η θέση είναι μη μηδενική, τότε πήγαινε στο ΒΗΜΑ 4. 
  \item [ΒΗΜΑ 3] Αν, αυτή η θέση είναι μηδενική, τότε φέρε ένα μη μηδενικό στοιχείο 
    στη θέση αυτή, εναλλάσσοντας αυτή τη γραμμή με κάποια γραμμή \textbf{παρακάτω}. 
    Αν δεν μπορείς να φέρεις μη μηδενικό στοιχείο στη θέση αυτή, τότε μετακινήσου μια 
    θέση προς τα δεξιά και πήγαινε στο ΒΗΜΑ 2. Αν δεν μπορείς να μετακινηθείς μια 
    θέση προς τα δεξιά, τότε η διαδικασία έχει τερματιστεί.
  \item [ΒΗΜΑ 4] Δημιούργησε μονάδα στη θέση αυτή πολλαπλασιάζοντας τη γραμμή επί τον 
    αντίστροφο του αριθμού αυτής της θέσης.
  \item [ΒΗΜΑ 5] Δημιούργησε μηδενικά σε όλες τις υπόλοιπες θέσεις αυτής της στήλης, 
    προσθέτοντας κατάλληλα πολλαπλάσια αυτής της γραμμής στις υπόλοιπες γραμμές.
  \item [ΒΗΜΑ 6] Μετακινήσου μια θέση προς τα δεξιά και μια θέση προς τα κάτω και πήγαινε
    στο ΒΗΜΑ 2. Αν δε μπορείς να μετακινηθείς μια θέση προς τα δεξιά και μια θέση 
    προς τα κάτω, τότε η διαδικασία έχει τερματιστεί.
\end{description}

\begin{dfn}[Ισότητα Πινάκων]
  Δυο πίνακες $ A = (a_{ij}) \in \textbf{M}_{m \times n}(\mathbb{K}) $ και 
  $ B \in \textbf{M}_{p \times q}(\mathbb{K}) $ λέγονται \textcolor{Col1}{ίσοι} και 
  συμβολικά γράφουμε 
  $ A=B $, αν $ m=p $, $ n=q $ και $ a_{ij} = b_{ij} $ για όλα τα $ i=1,\ldots m $ 
  και $ j=1,\ldots, n $.
\end{dfn}

\begin{prop}
  Έστω $ A, B \in \textbf{M}_{m \times n}(\mathbb{K}) $. Τότε ο $B$ είναι
  γραμμοϊσοδύναμος με τον $A$ αν και μόνον αν $ R_{A}=R_{B} $, όπου $ R_{A} $ και $ R_{B}
  $ είναι οι ανηγμένοι κλιμακωτοί πίνακες οι οποίοι είναι γραμμοϊσοδύναμοι με τους 
  $A$ και $B$, αντίστοιχα.
\end{prop}

\section*{Συστήματα Γραμμικών Εξισώσεων}

\begin{dfn}
  Κάθε έκφραση της μορφής 
  \begin{equation}\label{eq:gram_eq}
    a_{1} x_{1}+ a_{2} x_{2} + \cdots + a_{n} x_{n} = b 
  \end{equation} 
   όπου τα $ a_{i}, b \in \mathbb{R} $ και τα $ x_{i} $ είναι άγνωστοι, λέγεται 
   \textcolor{Col1}{γραμμική εξίσωση} επί του $ \mathbb{R} $. Οι αριθμοί $ a_{i} $ 
   λέγονται \textcolor{Col1}{συντελεστές} των $ x_{i} $ αντίστοιχα και ο $b$ λέγεται 
   \textcolor{Col1}{σταθερός όρος} ή απλά σταθερά της εξίσωσης. 
   Μια $n$-άδα πραγματικών αριθμών, $ (k_{1},k_{2}, \ldots,
   k_{n})$, λέγεται \textcolor{Col1}{λύση} της εξίσωσης~\eqref{eq:gram_eq}, 
   αν η σχέση που προκύπτει από την~\eqref{eq:gram_eq} αντικαθιστώντας $ x_{1} = k_{i} $
   \[
     a_{1} k_{1} + a_{2} k_{2} + \ldots a_{n}k_{n} =b 
    \] 
    είναι αληθής. Λέμε επίσης, ότι η $n$-άδα $ (k_{1},k_{2},\ldots,k_{n}) $
    \textbf{ικανοποιεί} (ή επαληθεύει) την εξίσωση.
\end{dfn}

\begin{dfn}
  Μια συλλογή $ m $ γραμμικών εξισώσεων με $n$ αγνώστους 
  $ x_{1}, x_{2}, \ldots, x_{n} $, 
  \begin{equation*}
    \setlength\arraycolsep{1.5pt} 
    \left.
      \begin{array}{ccc ccc c @{\extracolsep{2.5pt}}c
        @{\extracolsep{2.5pt}}c}
        a_{11}x_{1} & + & a_{12}x_{2} & + & \cdots & + & a_{1n}x_{n} & =
                    & b_{1} \\
        a_{21}x_{1} & + & a_{22}x_{2} & + & \cdots & + & a_{2n}x_{n} & =
                    & b_{2} \\
        \vdots & & \vdots & & \ddots & &  \vdots & &  \vdots \\
        a_{m1}x_{1} & + & a_{m2}x_{2} & + & \cdots & + & a_{mn}x_{n} & =
                    & b_{m} \\
      \end{array}
    \right\} \Leftrightarrow
    \underbrace{\begin{pmatrix}
        a_{11} & a_{12} & \cdots & a_{1n} \\
        a_{21} & a_{22} & \cdots & a_{2n} \\
        \vdots & \vdots & \ddots & \vdots \\
        a_{m1} & a_{m2} & \cdots & a_{mn} 
    \end{pmatrix}}_{A_{m \times n}}
    \cdot 
    \underbrace{\begin{pmatrix*}
        x_{1} \\
        x_{2} \\
        \vdots \\
        x_{n}
    \end{pmatrix*}}_{X} = 
    \begin{pmatrix*}[r]
      b_{1} \\
      b_{2} \\
      \vdots \\
      b_{m}
    \end{pmatrix*}
  \end{equation*}
  όπου $ a_{ij} \in \mathbb{R} $, $(i=1,\ldots,m, \; j=1,\ldots,n) $ και $ b_{i} \in
  \mathbb{R}$, $ (i=1,\ldots, m) $, καλείται \textcolor{Col1}{γραμμικό σύστημα} $ m $ 
  εξισώσεων με $n$ αγνώστους επί του $ \mathbb{R} $. Ο $ m \times n $ πίνακας $ A
  $ λέγεται \textcolor{Col1}{πίνακας των συντελεστών} των αγνώστων του συστήματος, 
  ενώ ο $ n \times 1 $ πίνακας $ X $, λέγεται \textcolor{Col1}{πίνακας των αγνώστων} του 
  συστήματος. Ο $ n \times 1 $ πίνακας $ B $, λέγεται \textcolor{Col1}{πίνακας των
  σταθερών} όρων του συστήματος. Ο $ m \times (n+1) $ πίνακας 
  \[ 
    E = (A|B) = 
    \begin{pmatrix*}[r]
      a_{11} & a_{12} & \cdots & a_{1n} & \vrule &  b_{1} \\
      a_{21} & a_{22} & \cdots & a_{2n} &\vrule &  b_{2} \\
      \vdots & \vdots & & \vdots & \vrule & \vdots \\
      a_{m1} & a_{mm} & \cdots & a_{mn} & \vrule &  b_{m} \\
  \end{pmatrix*}
  \] 
  λέγεται \textcolor{Col1}{επαυξημένος} πίνακας του συστήματος. Το σύστημα, 
  γράφεται σε πιο συντετμημένη μορφή, ως εξής:
  \[
    \sum_{j=1}^{n} a_{ij} x_{i} = b_{i}, \quad (i=1,2,\ldots, m)
  \]
\end{dfn}

\begin{dfn}
\item {}
  \begin{enumerate}[i)]
    \item Μια $n$-άδα πραγματικών αριθμών $ (s_{1},s_{2}, \ldots, s_{n}) $ η οποία 
      ικανοποιεί όλες τις εξισώσεις του συστήματος, καλείται \textcolor{Col1}{λύση} 
      του συστήματος.
    \item Το σύστημα καλείται \textcolor{Col1}{συμβιβαστό}, αν έχει τουλάχιστον μια 
      λύση. Διαφορετικά (δηλαδή αν το σύστημα δεν έχει λύση), το σύστημα καλείται
      \textcolor{Col1}{αδύνατο}.
    \item Το σύστημα καλείται \textcolor{Col1}{ομογενές} αν οι σταθεροί όροι 
      $ b_{1}, b_{2}, \ldots, b_{m} $ είναι όλοι μηδέν. Η $n$-άδα $ (0,0,\ldots, 0) $ 
      είναι λύση ενός ομογενούς συστήματος ($m$ εξισώσεων με $n$ αγνώστους). 
      Καλείται \textcolor{Col1}{τετριμμένη} λύση του ομογενούς συστήματος. 
      Δηλαδή κάθε ομογενές σύστημα είναι \textbf{συμβιβαστό}.
    \item Δυο γραμμικά συστήματα $m$ εξισώσεων με $n$ αγνώστους καλούνται 
      \textcolor{Col1}{ισοδύναμα} αν κάθε λύση του ενός συστήματος είναι μια λύση του 
      άλλου συστήματος και αντίστροφα (με άλλα λόγια, δυο γραμμικά συστήματα $m$ 
      εξισώσεων με $n$ αγνώστους είναι ισοδύναμα αν έχουν ακριβώς τις \textbf{ίδιες
      λύσεις}).
  \end{enumerate}
\end{dfn}

\begin{thm}
  Έστω δυο γραμμικά συστήματα $m$ εξισώσεων με $n$ αγνώστους 
  $ x_{1}, x_{2}, \ldots, x_{n} $ επί του $ \mathbb{R} $,
  \begin{equation*}\label{eq:sys1}
    \setlength\arraycolsep{1.5pt} 
    \left.
      \begin{array}{ccc ccc c @{\extracolsep{2.5pt}}c
        @{\extracolsep{2.5pt}}c}
        a_{11}x_{1} & + & a_{12}x_{2} & + & \cdots & + & a_{1n}x_{n} & =
                    & b_{1} \\
        a_{21}x_{1} & + & a_{22}x_{2} & + & \cdots & + & a_{2n}x_{n} & =
                    & b_{2} \\
                    & & & & \cdots & &  & &  \\
        a_{m1}x_{1} & + & a_{m2}x_{2} & + & \cdots & + & a_{mn}x_{n} & =
                    & b_{m} \tag{$*$}
      \end{array}
    \right\} 
  \end{equation*} 
  και 
  \begin{equation*}\label{eq:sys2}
    \setlength\arraycolsep{1.5pt} 
    \left.
      \begin{array}{ccc ccc c @{\extracolsep{2.5pt}}c
        @{\extracolsep{2.5pt}}c}
        a'_{11}x_{1} & + & a'_{12}x_{2} & + & \cdots & + & a'_{1n}x_{n} & =
                     & b'_{1} \\
        a'_{21}x_{1} & + & a'_{22}x_{2} & + & \cdots & + & a'_{2n}x_{n} & =
                     & b'_{2} \\
                     & & & & \cdots & &  & &  \\
        a'_{m1}x_{1} & + & a'_{m2}x_{2} & + & \cdots & + & a'_{mn}x_{n} & =
                     & b'_{m} \tag{$**$}
      \end{array}
    \right\} 
  \end{equation*}
  Αν ο επαυξημένος πίνακας του συστήματος $ (A'|B') = (a'_{ij}|b'_{i}) $ του 
  συστήματος~\eqref{eq:sys2} προκύπτει από τον επαυξημένο πίνακα 
  $ (A|B) = (a_{ij}|b_{i}) $ του συστήματος~\eqref{eq:sys1} εφαρμόζοντας ένα στοιχειώδη 
  μετασχηματισμό γραμμών, τότε τα συστήματα~\eqref{eq:sys1} και~\eqref{eq:sys2} είναι 
  ισοδύναμα.
\end{thm}

\begin{cor}\label{cor:1}
  Έστω ένα γραμμικό σύστημα $m$ εξισώσεων με $n$ αγνώστους $ x_{1}, x_{2}, \ldots, x_{n} $
  επί του $ \mathbb{R} $, 
  \[
    \sum_{j=1}^{n} a_{ij}x_{j}=b_{i}, \quad (i=1,\ldots, m)  
   \] 
   με επαυξημένο πίνακα $ (A|B) = (a_{ij}|b_{i}) $. Αν $ (R|S) = (r_{ij}|s_{i}) $ 
   είναι ο ανηγμένος κλιμακωτός πίνακας του $ (A|B) $, τότε τα συστήματα 
   $ \sum_{j=1}^{n} a_{ij}x_{}j=b_{i} $, $ (i=1,\dots, m) $ και $ \sum_{j=1}^{n}
   r_{ij}x_{j}=s_{i}  $, $(i=1,\dots, m) $, είναι \textbf{ισοδύναμα}.
\end{cor}

\begin{rem}
  Το αποτέλεσμα του Πορίσματος~\ref{cor:1} υποδεικνύει έμμεσα τη μέθοδο που θα 
  ακολουθήσουμε για να λύνουμε γραμμικά συστήματα $m$ εξισώσεων με $ n $ αγνώστους. 
  Πράγματι, αν $ \sum_{j=1}^{n} a_{ij}x_{j} = b_{i}  $, $ (i=1,\ldots,m) $ είναι ένα 
  γραμμικό σύστημα με επαυξημένο πίνακα $ (A|B)=(a_{ij}|b_{i}) $ τότε πρώτα θα βρίσκουμε 
  τον ανηγμένο κλιμακωτό πίνακα $ (R|S)=(r_{ij}|s_{i}) $ του $ (A|B) $, χρησιμοποιώντας 
  τη μέθοδο απαλοιφής των Gauss-Jordan και στη συνέχεια θα λύνουμε το σύστημα 
  $ \sum_{j=1}^{n} r_{ij}x_{j}=s_{i}  $, $ (i=1,\ldots,m) $ το οποίο από το
  Πόρισμα~\ref{cor:1} είναι ισοδύναμο με το αρχικό, και είναι πολύ πιο \textbf{απλό} 
  από το αρχικό. Ο λόγος, γι᾽ αυτό είναι ότι στο τελευταίο σύστημα έχουν απαλειφθεί 
  πολλοί άγνωστοι (αφού αρκετοί συντελεστές είναι πλέον μηδέν) λόγω της μορφής 
  ενός ανηγμένου κλιμακωτού πίνακα.
\end{rem}

\begin{thm}\label{thm:sys_r}
  Ένα γραμμικό σύστημα $ m $ εξισώσεων με $n$ αγνώστους επί  του $ \mathbb{R} $ είναι 
  συμβιβαστό αν και μόνον αν ο ανηγμένος κλιμακωτός πίνακας του επαυξημένου πίνακα του 
  συστήματος δεν έχει ηγετικό 1 στη τελευταία στήλη του.
\end{thm}

\begin{cor}
  Ένα συμβιβαστό γραμμικό σύστημα $m$ εξισώσεων με $n$ αγνώστους επί του $ \mathbb{R} $ 
  έχει: 
  \begin{enumerate}[i)]
    \item \label{it:1} Μοναδική λύση αν $ r=n $
    \item \label{it:2} Άπειρο πλήθος λύσεων αν $ r<n $
  \end{enumerate}
  όπου $r$ το πλήθος των ηγετικών μονάδων του ανηγμένου κλιμακωτού πίνακα του 
  επαυξημένου πίνακα του συστήματος.
\end{cor}

\begin{rem}
  Σημειώστε ότι οι περιπτώσεις~\ref{it:1} και~\ref{it:2} είναι όλες οι δυνατές 
  περιπτώσεις για ένα συμβιβαστό σύστημα, διότι από το θεώρημα~\ref{thm:sys_r} 
  έχουμε ότι $ r \leq n $ για συμβιβαστά συστήματα $m$ γραμμικών εξισώσεων με $n$ 
  αγνώστους.
\end{rem}

\begin{cor}
  Ένα ομογενές γραμμικό σύστημα $m$ εξισώσεων με $n$ αγνώστους, για το οποίο ισχύει
  $ m < n $, έχει άπειρο πλήθος λύσεων.
\end{cor}

\begin{rem}
  Πράγματι, εφόσον $ m<n $ και $ r \leq m $, έχουμε ότι $ r<n $.
\end{rem}

\begin{cor}
  Ένα ομογενές γραμμικό σύστημα $n$ εξισώσεων με $n$ αγνώστους έχει μοναδική λύση την 
  τετριμμένη αν και μόνον αν $ R_{A}=I_{n} $, όπου $ R_{A} $ ο ανηγμένος κλιμακωτός 
  πίνακας του πίνακα $ A $ των συντελεστών.
\end{cor}

\begin{rem}
  Ισοδύναμα, ένα ομογενές γραμμικό σύστημα $n$ εξισώσεων με $n$ αγνώστους έχει μια 
  μη τετριμμένη λύση αν και μόνον αν $ R_{A} \neq I_{n} $.
\end{rem}

\section*{Βαθμός Πίνακα}

\begin{dfn}
  Έστω $ A \in \textbf{M}_{m \times n}(\mathbb{R}) $. Το πλήθος των ηγετικών μονάδων 
  του ανηγμένου κλιμακωτού πίνακα $ R_{A} $ του $A$ καλείται \textcolor{Col1}{βαθμός του
  $A$} και συμβολίζεται με $ r(A) $.
\end{dfn}
\begin{rem}
  Από τον ορισμό του $ r(A) $, έχουμε αμέσως ότι: 
  \begin{myitemize}
    \item $ r(A) = $ πλήθος των μη μηδενικών γραμμών του $ R_{A} $ 
    \item $ r(A) = $ πλήθος των στηλών του $ R_{A} $ στις οποίες εμφανίζονται τα 
      ηγετικά 1.
  \end{myitemize}
\end{rem}

\begin{thm}
  Ένα γραμμικό σύστημα $m$ εξισώσεων με $n$ αγνώστους είναι συμβιβαστό αν και μόνον αν 
  $ r(A)=r(A|B) $, όπου $A$ ο πίνακας των συντελεστώ και $ (A|B) $ ο επαυξημένος πίνακας
  του συστήματος (Σημειώστε ότι πάντα έχουμε $ r(A) \leq r(A|B) $).
\end{thm}

\begin{thm}
  Ένα ομογενές γραμμικό σύστημα $m$ εξισώσεων με $n$ αγνώστους έχει:
  \begin{enumerate}[i)]
    \item Μοναδική λύση την τετριμμένη αν και μόνον αν $ r(A)=n $
    \item Άπειρο πλήθος λύσεων αν και μόνον αν $ r(A)<n $, όπου $A$ ο πίνακας των 
      συντελεστών των αγνώστων.
  \end{enumerate}
\end{thm}

\begin{thm}
  Έστω $ A\cdot X=B $ ένα γραμμικό σύστημα $m$ εξισώσεων με $n$ αγνώστους επί του 
  $ \mathbb{R} $. Αν $s$ είναι λύση του συστήματος $ A\cdot X=B $ (το $s$ είναι $n$-άδα 
  στήλη και $ A\cdot s =B $), τότε το σύνολο λύσεων του $ A\cdot X=B $ είναι το σύνολο
  \[
    T = \{ s+u \; : \; u \; \text{λύση του αντίστοιχου ομογενούς } A\cdot X=0 \} 
  \] 
\end{thm}

\begin{cor}
  Αν ένα ομογενές γραμμικό σύστημα $m$ εξισώσεων με $n$ αγνώστους έχει περισσότερο από 
  μία λύσεις, τότε έχει άπειρο πλήθος λύσεων.
\end{cor}

\end{document}

