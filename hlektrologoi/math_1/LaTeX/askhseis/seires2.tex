\input{preamble_ask.tex}
\input{definitions_ask.tex}


\geometry{top=1.6cm}
\everymath{\displaystyle}
\pagestyle{askhseis}

\begin{document}


\begin{center}
  \minibox{\large \bfseries \textcolor{Col1}{Ασκήσεις στις Σειρές}}
\end{center}

\vspace{\baselineskip}

\begin{enumerate}
  \item Να εξεταστούν ως προς τη σύγκλιση, με το γενικευμένο κριτήριο σύγκρισης, 
    οι παρακάτω σειρές.
    \begin{enumerate}[i)]
      \item $ \sum_{n=1}^{\infty} \frac{n^{4}+n^{2}}{3n^{6}+n} $ \hfill Απ: συγκλίνει 
      \item $ \sum_{n=1}^{\infty} \left(\frac{1}{n} + \frac{1}{n^{2}}\right) $ 
        \hfill Απ: αποκλίνει 
      \item $ \sum_{n=1}^{\infty} \frac{3^{n}+4^{n}}{4^{n}+5^{n}} $ \hfill Απ: συγκλίνει 
    \end{enumerate}

  \item Να βρεθεί το υπόλοιπο $ R_{n} $ για τη σειρά 
    $ \sum_{n=1}^{\infty} \frac{n!}{n^{n}} $. 
    \hfill Απ: $ R_{n} = \frac{2(n+1)!}{(n+1)^{n+1}} $  

  \item Πόσους όρους των παρακάτω σειρών, πρέπει να αθροίσουμε ώστε να προσεγγίσουμε 
    το άθροισμά τους με τη ζητούμενη ακρίβεια δεκαδικών ψηφίων; 
    \begin{enumerate}[i)]
      \item $ \sum_{n=1}^{\infty} \frac{1}{n^{2}} $ \quad 
        \textcolor{Col1}{ακρίβεια:} 2 δεκαδικά ψηφία \hfill Απ: $ n>200 $  
      \item $ \sum_{n=1}^{\infty} (-1)^{n} \frac{1}{n^{4}} $ \quad
        \textcolor{Col1}{ακρίβεια:} 5 δεκαδικά ψηφία \hfill Απ: $ n>20 $ 
    \end{enumerate}

  \item Να εξεταστεί ως προς τη σύγκλιση, με το γενικευμένο κριτήριο λόγου, η σειρά 
    $ \sum_{n=1}^{\infty} \frac{(2n)!}{4^{n} (n!)^{2}}$ 

    \hfill Απ: $ \lambda ' = 1/2 < 1 $ αποκλίνει

\end{enumerate}
  \end{document}
