\input{preamble_ask.tex}
\input{definitions_ask.tex}
\input{tikz.tex}


\pagestyle{askhseis}
\everymath{\displaystyle}

\begin{document}

\begin{center}
\minibox{\large\bfseries \textcolor{Col1}{Επικαμπύλιο Ολοκλήρωμα Ιου είδους}}
\end{center}

\vspace{\baselineskip}


\begin{enumerate}
  \item Να υπολογιστούν τα παρακάτω επικαμπύλια ολοκληρώματα κατά μήκος των 
    δοσμένων καμπυλών:
    \begin{enumerate}[i)]

      %%thomas ex.11 p. 906
      %\item $ \int _{c} (xy+y+z) \,{ds} $, όπου $ c \colon $ καμπύλη 
      %  $ \mathbf{r}(t)=2t\, \mathbf{i} + t\, \mathbf{j} + (2-2t) \, \mathbf{k} $, 
      %  με $ 0 \leq t \leq 1 $.  
      %  \hfill Απ: $13/2$ 

      %thomas ex.12 p. 906
      \item $ \int _{c} \sqrt{x^{2}+y^{2}} \,{ds} $, όπου $ c \colon $ έλικα με 
        $ \mathbf{r}(t)=(4 \cos{t} )\, \mathbf{i} + (4 \sin{t})\, \mathbf{j} + 3t \, 
        \mathbf{k} $, με $ -2 \pi \leq t \leq 2 \pi $.  
        \hfill Απ: $80 \pi$ 
      %thomas ex.10 p. 906
      \item $ \int _{c} (x-y+z-2) \,{ds} $, όπου $ c \colon $ το ευθύγραμμο τμήμα 
        από το σημείο $ A(0,1,1) $ έως $ B(1,0,1) $.
        \hfill Απ: $ -\sqrt{2} $ 

      %%thomas ex.14 p. 906
      %\item $ \int _{c} \frac{\sqrt{3}}{x^{2}+y^{2}+z^{2}}\,{ds} $, όπου $ c \colon $ 
      %  $ \mathbf{r}(t)= t\, \mathbf{i} + t\, \mathbf{j} + t\, 
      %  \mathbf{k} $, με $1 \leq t \leq \infty$.  
      %  \hfill Απ: $1$ 

      %thomas ex.29 p. 906
      \item $ \int _{c} (x+y) \,{ds} $, όπου $ c \colon x^{2}+y^{2}=4 $ στο 1ο
        τεταρτημόριο.  \hfill Απ: $8$ 
        %spand
      %thomas ex.27 p. 906
      \item $ \int _{c} x^{3}/y \,{ds} $, όπου $ c \colon y=x^{2}/2 $ με 
        $0 \leq x \leq 2$.  
        \hfill Απ: $ \frac{10 \sqrt{5} -2}{3} $ 
      \item $ \int _{c} (x+y) \,{ds} $, όπου $ c \colon $ η περίμετρος τριγώνου 
        με κορυφές $ A(1,0) $, $ B(0,1) $, $ O(0,0) $.
        \hfill Απ: $ \sqrt{2} +1 $ 

    \end{enumerate}

  \item Να υπολογιστεί το επικαμπύλιο ολοκλήρωμα $ \int _{c} (x+ \sqrt{y}) \,{ds} $, 
    όπου $ c $ η καμπύλη του σχήματος.
        \hfill Απ: $ \frac{5}{6} \sqrt{5} + \frac{3}{2} $ 

\begin{tikzpicture}[scale=0.8]
  \node (0) at (0, 0) {};
  \node (1) at (3, 0) {};
  \node (2) at (0, 3) {};
  \coordinate [point,Col2] (3) at (45:3) ;
  \node[right] at (3) {$(1,1)$} ;
  \draw[-latex] (0.center) to (1.center) node[below]{$x$};
  \draw[-latex] (0.center) to (2.center) node[left]{$y$};
  \draw [graph] (3.center) to node[pos=0.3,left,graph label] {\small$y=x$} 
    pic[pos=0.5,rotate=225]{arrow} (0.center);
  \draw [graph,bend right] (0.center) to node[right,graph label] {\small$y=x^{2}$} 
    pic[pos=0.5,rotate=45]{arrow} (3.center);
\end{tikzpicture}


\end{enumerate}

\begin{center}
\minibox{\large\bfseries \textcolor{Col1}{Επικαμπύλιο Ολοκλήρωμα ΙΙου είδους}}
\end{center}

\vspace{\baselineskip}

\begin{enumerate}
  \item Να υπολογιστεί το επικαμπύλιο ολοκλήρωμα των διανυσματικών πεδίων, κατά μήκος των
    καμπυλών 
    \begin{enumerate}[i)]
      \item $ c_{1} \colon $ ευθύγραμμο τμήμα από $ A(0,0,0) $ έως $ B(1,1,1) $.
      \item $ c_{2} \colon $ παραβολική τροχιά $ \mathbf{r}(t)=t\, \mathbf{i} +
        t^{2}\, \mathbf{j} + t^{4} \, \mathbf{k} $ από $ A(0,0,0) $ έως $ B(1,1,1) $.
      \item $ c_{3} \colon $ η ένωση των ευθυγράμμων τμημάτων από 
        $ A(0,0,0) $ έως $ B(1,1,0) $ και από $ B(1,1,0) $ έως $ C(1,1,1) $.
    \end{enumerate}
    \begin{myitemize}
      \item $ \mathbf{F}(x,y,z) = 3y \mathbf{i} + x \mathbf{j} + 4z \mathbf{k}
        $ \hfill Απ: 
        \begin{enumerate*}[i),itemjoin=\hspace{5pt}] 
          \item 9/2  
          \item 13/3
          \item 9/2
        \end{enumerate*}
      \item $ \mathbf{F}(x,y,z) = (y+z) \mathbf{i} + (z+x) \mathbf{j} + (x+y) \mathbf{k}
        $ \hfill Απ: 
        \begin{enumerate*}[i),itemjoin=\hspace{5pt}] 
          \item 3
          \item 3
          \item 3
        \end{enumerate*}
    \end{myitemize}

\end{enumerate}

\section*{Έργο}

\begin{enumerate}
  \item Να βρείτε το \textbf{έργο} που παράγεται από το διανυσματικό πεδίο 
    $ \mathbf{F} $ κατά μήκος της καμπύλης $c$ προς τη θετική φορά.
    \begin{enumerate}[i)]
      \item $ \mathbf{F}(x,y,z) = xy \mathbf{i} + y \mathbf{j} + -yz \mathbf{k}
        $ όπου $ c \colon \mathbf{r}(t)=t\, \mathbf{i} + t^{2}\, \mathbf{j} + t \,
        \mathbf{k}$, με $ 0 \leq t \leq 1 $. 
        \hfill Απ: $ 1/2 $ 
      \item $ \mathbf{F}(x,y,z) = 2y \mathbf{i} + 3x \mathbf{j} + (x+y) \mathbf{k}
        $ όπου $ c \colon \mathbf{r}(t)= \cos{t}\, \mathbf{i} + \sin{t}\, 
        \mathbf{j} + t/6 \, \mathbf{k} $, με $ 0 \leq t \leq 2 \pi $ 
        \hfill Απ: $ \pi $ 
      \item $ \mathbf{F}(x,y,z) = z \mathbf{i} + x \mathbf{j} + y \mathbf{k}
        $, όπου $ c \colon \mathbf{r}(t)= \sin{t}\, \mathbf{i} + \cos{t}\, 
        \mathbf{j} + t \, \mathbf{k}$, με $ 0 \leq t \leq 2 \pi $ 
        \hfill Απ: $ - \pi $  
      \item $ \mathbf{F}(x,y,z) = 6z \mathbf{i} + y^{2} \mathbf{j} + 12 x \mathbf{k}
        $, όπου $ c \colon \mathbf{r}(t)= \sin{t}\, \mathbf{i} + \cos{t}\, 
        \mathbf{j} + t/6 \, \mathbf{k}$, με $ 0 \leq t \leq 2 \pi $ 
        \hfill Απ: 0  
    \end{enumerate}

  \item Να βρείτε το \textbf{έργο} που παράγεται από την κλίση της συνάρτησης 
    $ f(x,y) = (x+y)^{2} $ κατά μήκος του κύκλου $ x^{2}+y^{2}=4 $, προς τη θετική φορά. 
    \hfill Απ: 0  
\end{enumerate}


\section*{Ροή κατά μήκος, Κυκλοφορία}

\begin{enumerate}
  \item Για τα διανυσματικά πεδία, 
    $ \mathbf{F_{1}}(x,y) = x \mathbf{i} + y \mathbf{j} $ και $ \mathbf{F_{2}}(x,y) = 
    -y\mathbf{i} + x \mathbf{j} $, να βρείτε την \textbf{κυκλοφορία}.
    \begin{enumerate}[i)]
      \item κύκλος $ \mathbf{r}(t)= \cos{t}\, \mathbf{i} + \sin{t}\, \mathbf{j}$, με 
        $ 0 \leq t \leq 2 \pi $ 
        \hfill Απ: 
        \begin{enumerate*}[i)]
          \item $ circ_{1} = 2 \pi, \quad circ_{2}=0 $
        \end{enumerate*}
      \item έλλειψη $ \mathbf{r}(t)= \cos{t}\, \mathbf{i} + 4\sin{t}\, \mathbf{j}$, με 
        $ 0 \leq t \leq 2 \pi $ 
        \hfill Απ:  
        \begin{enumerate*}[i)]
          \item $ circ_{1} = 0, \quad circ_{2}=8 \pi $
        \end{enumerate*}
    \end{enumerate}

  \item Να βρείτε τη \textbf{ροή} του διανυσματικού πεδίου $ \mathbf{F}(x,y) = (x+y) 
    \mathbf{i} - (x^{2}+y^{2}) \mathbf{j} $ από το σημείο $ (1,0) $ προς το $ (-1,0) $ 
    κατά μήκος των παρακάτω καμπυλών:
    \begin{enumerate}[i)]
      \item το πάνω ημικύκλιο του κύκλου $ x^{2}+y^{2}=1 $.
      \item το ευθύγραμμο τμήμα από το $ (1,0) $ έως το $ (-1,0) $.
      \item την ένωση των ευθυγράμμων τμημάτων από το $ (1,0) $ έως το $ (0,-1) $ και 
        από το $ (0,-1) $ έως το $ (-1,0) $.
    \end{enumerate}
    \hfill Απ: 
    \begin{enumerate*}[i)]
      \item $ - \pi /2 $
      \item $ 0 $
      \item $ 1 $
    \end{enumerate*}

\end{enumerate}




\end{document}


