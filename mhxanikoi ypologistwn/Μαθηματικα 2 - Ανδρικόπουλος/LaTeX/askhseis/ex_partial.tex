\input{preamble_ask.tex}
\input{definitions_ask.tex}


\pagestyle{askhseis}
\everymath{\displaystyle}

\begin{document}

\begin{center}
  {\color{Col2}\minibox[frame,rule=1pt]{\large\bfseries Ασκήσεις στις 
  Μερικές Παραγώγους}}
\end{center}

\section*{Κανόνες Παραγώγισης}

\begin{enumerate}
  \item Με τη βοήθεια των κανόνων παραγώγισης να υπολογιστούν οι μερικές 
    παράγωγοι των συναρτήσεων:

    \begin{enumerate}[i)]
      \item $f(x,y)=y\sin (xy)$ \hfill Απ: \begin{tabular}{l}
          $f_x=y^2\cos(xy)$ \\ 
          $f_y=\sin(xy)+yx\cos(xy)$
        \end{tabular}

      \item $f(x,y)=y\ln(x+y)$\hfill Απ: \begin{tabular}{l}
          $f_x=\frac{y}{x+y}$ \\ 
          $f_y=\ln(x+y)+\frac{y}{x+y}$
        \end{tabular}

      \item $f(x,y)=\arcsin(\frac{x}{y})$\hfill Απ: \begin{tabular}{l}
          $f_x=\frac{1}{y^2-x^2}$ \\ 
          $f_y=-\frac{x}{y\sqrt{y^2-x^2}}$
        \end{tabular}
      \item $ f(x,y,z) = (x+y^{2}) \sin{(xz)} $ \hfill Απ: \begin{tabular}{l}
          $ f_{x} = \sin{(xz)} + z(x+y^{2}) \sin{(xz)} $ \\
          $ f_{y} = 2y \sin{(xz)} $ \\
          $ f_{z} = x(x+y^{2}) \sin{(xz)} $
        \end{tabular} 
    \end{enumerate}

  \item Έστω η συνάρτηση $ f(x,y) = x^{2} \sin{(x+y)} $. Να υπολογίσετε τις μερικές 
    παραγώγους 1ης και 2ης τάξης.

    \hfill Απ: \begin{tabular}{l}
      $ f_{x} = 2x \sin{(x+y)} + x^{2} \cos{(x+y)} $ \\
      $ f_{y} = x^{2} \cos{(x+y)} $ \\
      $ f_{xx} =  -x^{2} \sin{(x+y)} $ \\
      $ f_{yy} = 2 \sin{(x+y)} + 4x \cos{(x+y)} -x^{2} \sin{(x+y)} $ \\
      $ f_{xy}=f_{yx} =  2x \cos{(x+y)} -x^{2} \sin{(x+y)} $
    \end{tabular}

  \item Έστω η συνάρτηση $f(x,y)=\ln\left(\cos y+x\cos x\right)$.  Να υπολογισθεί 
    η $ f_{xy} $ στο σημείο $(\pi,-\pi/2)$.  \hfill Απ: $\frac{1}{\pi^2}$

    % \item Έστω η συνάρτηση $ f(x,y) = x^{y} $. Να δείξετε ότι ισχύει ότι το θεώρημα 
    % Schwartz, δηλαδή ότι $ f_{xy} = f_{yx} $.

    %hlektr
    %\item Να δείξετε ότι η συνάρτηση $ f(x,y) = (y+3x)^{1/2} - 
    %    (y-3x)^{2} $ ικανοποιεί τη σχέση $ f_{xx} - 9 f_{yy} = 0 $.

    %    %span
  \item  Δίνεται η συνάρτηση $ f(x,y) = \cos{(x+y)} + \cos{(x-y)} $. Να δείξετε ότι  
    $ z_{xx} - z_{yy} = 0 $.

  \item Να δείξετε ότι για τη συνάρτηση $ f(x,y) = x \arctan{\frac{y}{x}} $ 
    ισχύει η σχέση $ x^{2} f_{xx} + 2xyf_{xy} + y^{2} f_{yy} = 0 $ 
\end{enumerate}


\section*{Σύνθετη Παραγώγιση}

\begin{enumerate}

  \item Να βρεθεί η παράγωγος $\dv{f}{t}$ της συνάρτησης 
    $f(x,y)=\ln(y^2-x^2)$, όταν $x=\sin t, y=\cos t$, για $t=\frac{\pi}{8}$.

    \hfill Απ: $-2$

  \item Να βρεθεί η παράγωγος $\dv{w}{t}$ της συνάρτησης 
    $ w = xy+z $, όταν $ x = \cos{t}, y = \sin{t}$ και $ z = t $, για 
    $ t = \frac{ \pi }{ 4 } $.

    \hfill Απ: 1

  \item Δίνεται η συνάρτηση $ z=f(x,y) $, όπου $ x=u+v $ και $ y = u-v $. 
    Να αποδείξετε ότι $ z_{xx}-z_{yy} = z_{u}\cdot z_{v} $ 

  \item Έστω η συνάρτηση  $ f(x,y) = x^{2} + xy $, όπου $ x=r \cos{\theta} $ και 
    $ y= r \sin{\theta} $. 
    Να υπολογίσετε τις μερικές παραγώγους $ \pdv{f}{r} $ και $ \pdv{f}{\theta} $.  

    \hfill Απ: 
      $ \pdv{f}{r} = 2r(\cos{\theta} )(\cos{\theta} + \sin{\theta}) $ \; και \; 
      $ \pdv{f}{\theta}=r^{2}(\cos{2\theta} - \sin{2 \theta}) $

  \item Δίνεται η συνάρτηση $ f = f(x,y) $, όπου $ x=r \cos{\theta} $ και 
    $ y= r \sin{\theta} $. Να δείξετε ότι 
    \[
      \left(\pdv[2]{f}{x}\right)^{2} + \left(\pdv[2]{f}{y}\right)^{2} = 
      \left(\pdv[2]{f}{r}\right)^{2} + \frac{1}{r^{2}} 
      \left(\pdv[2]{f}{\theta}\right)^{2}
    \] 

    % \item Να υπολογιστούν οι μερικές παράγωγοι, ως προς $u$ και $v$, 
    %   της συνάρτησης $ f(x,y,z) = x + 2y + z^{2}$, όπου $ x = \frac{ u }{ v } $, 
    %   $y = u^{2} + \ln{v} $ και $ z = 2u $.
    %   \hfill Απ: $ \pdv{f}{u} = \frac{1}{ v } + 12u $, 
    %   $\pdv{f}{v} = -\frac{ u }{ v^{2} } + \frac{ 2 }{ v } $

\end{enumerate}

\section*{Διαφορικό}

\begin{enumerate}
  \item Να υπολογίσετε το ολικό διαφορικό 1ης τάξης, της συνάρτησης 
    $f(x,y)=\ln(xy)+\cos(y^2)$ 

    \hfill Απ: $df=\frac{dx}{x}+\left(\frac{1}{y}-2y\sin(y^2)\right)dy$

  \item Να υπολογίσετε τη διαφορά $ \Delta f $ και το ολικό διαφορικό $ df $ της 
    συνάρτησης $ f(x,y) = \sin{(x+y)} $, στο σημείο $ (0,0) $, όταν $ \Delta x = 0,1 $ 
    και $ \Delta y = 0,2 $.

    \hfill Απ: $ \Delta f = 0.29552, \; df = 0.3 $
\end{enumerate}


\section*{Πολυώνυμο Taylor}

\begin{enumerate}
  \item Να βρεθούν τα αναπτύγματα Taylor, μέχρι και όρους 
    \textbf{2ης τάξης}, των συναρτήσεων:

    \begin{enumerate}[i)]
      \item  $f(x,y)=y\cos{xy} $, γύρω από το σημείο 
        $ \left(1, \frac{ \pi }{ 2 }\right) $.

        \hfill Απ: $f(x,y)=-\frac{\pi^{2}}{4}(x-1) - \frac{ \pi }{ 2 } 
        \left(y - \frac{ \pi }{2 }\right) - \pi(x-1)
        \left(y-\frac{\pi}{2}\right)- \left(y- \frac{ \pi }{ 2} \right)^{2} $

      \item $ f(x,y)=e^{x}\tan{y} $ σε δυνάμεις των $ (x-1) $ και 
        $ \left(y - \frac{ \pi }{ 4 }\right) $

        \hfill Απ: $ f(x,y) = e + e(x-1) + 2e\left(y- \frac{ \pi }{ 4 }\right)
        + \frac{1}{ 2 } \left(e(x-1)^{2}+4e(x-1)\left(y- \frac{ \pi }{ 4 }
        \right) + 4e\left(y- \frac{ \pi }{ 4 } \right)^{2}\right) $
    \end{enumerate}

  \item Να βρεθεί το ανάπτυγμα Maclaurin, μέχρι όρους 2ης τάξης, 
    της συνάρτησης $ f(x,y) = e^{x}\ln(1+y)$.

    \hfill Απ: $ f(x,y)=y + xy - \frac{1}{ 2 } y^{2} + \frac{1}{ 2 } x^{2}y - 
    \frac{1}{ 2 } xy^{2} + \frac{1}{ 3 } y^{3} $
\end{enumerate}

\section*{Ακρότατα}

\begin{enumerate}
  \item Να βρεθούν και να χαρακτηριστούν τα κρίσιμα σημεία  των παρακάτω συναρτήσεων:
    \begin{enumerate}[(i)]
      \item $ f(x,y) = x^{3} + y^{3} + 3xy $ 
        \hfill Απ: max: $(-1,-1)  $, σάγμα: $ (0,0) $
      \item $ f(x,y) = x^{2}+y^{4} $ 
        \hfill Απ: min: $ (0,0) $ 
      \item $ f(x,y) = x^{3} + y^{3} - 3x -12y + 50 $ 
        \hfill Απ: max: $ (-1,-2)$, min: $ (1,2) $, 
        σάγμα: $ (1,-2), (-1,2) $
      \item $ f(x,y) = x^{3} + y^{3} -3x -3y + 1 $ 
        \hfill Απ: max: $(-1,-1)  $, min: $ (1,1) $
      \item $ f(x,y) = x^{3} + 4xy -4y^{2} $ 
        \hfill Απ: max: $ (-2/3, -1/3)  $
      \item $ f(x,y) = (x^{2}-3y^{2})e^{1-x^{2}-y^{2}} $ 
        \hfill Απ: max: $ (1,0)$, min: $ (0,1), (0,-1) $, 
        σαγμα: $ (0,0) $
    \end{enumerate}
\end{enumerate}
\end{document}
