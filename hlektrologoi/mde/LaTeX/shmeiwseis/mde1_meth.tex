\input{preamble_ask.tex}
\input{definitions_ask.tex}


\everymath{\displaystyle}
\thispagestyle{askhseis}


\begin{document}

\chapter{ΜΔΕ 1ης τάξης}

\section{Ημιγραμμικές ΜΔΕ}

\begin{thm}
  Η γενική λύση της μερικής διαφορικής εξίσωσης 
  \begin{equation} \label{eq:hmigram}
    P(x,y,u)u_{x} + Q(x,y,u)u_{y} = R(x,y,u) 
  \end{equation} 
  όπου $ u=u(x,y) $ και οι συναρτήσεις $ P,Q,R $ έχουν συνεχείς μερικές παραγώγους 
  1ης τάξης, είναι η αυθαίρετη συνάρτηση $ F $ με $ F(u_{1},u_{2})=0 $, όπου οι 
  συναρτήσεις $ u_{1}(x,y,u)= c_{1} $ και $ u_{2}(x,y,u)= c_{2} $ αποτελούν γραμμικώς 
  ανεξάρτητες λύσεις του συστήματος:
  \begin{equation} \label{eq:systhm}
    \frac{dx}{P(x,y,u)} = \frac{dy}{Q(x,y,u)} = \frac{du}{R(x,y,u)} 
  \end{equation}
\end{thm}

\begin{rem}
  Οι συναρτήσεις $ u_{1}(x,y,u)= c_{1} $ και $ u_{2}(x,y,u)= c_{2} $ παριστάνουν μια 
  διπαραμετρική οικογένεια ολοκληρωτικών καμπυλών. Οι καμπύλες αυτές ονομάζονται 
  χαρακτηριστικές καμπύλες. Η $ F(u_{1},u_{2})=0 $ είναι η εξίσωση των ολοκληρωτικών 
  επιφανειών της εξίσωσης~\eqref{eq:hmigram}
\end{rem}

\begin{rem}
  Τα δύο πρώτα ολοκληρώματα $ u_{1}(x,y,u)= c_{1} $ και $ u_{2}(x,y,u)= c_{2} $ με 
  τη βοήθεια των οποίων βρίσκουμε τη γενική λύση της~\eqref{eq:hmigram} πρέπει να 
  είναι γραμμικώς ανεξάρτητα. Αυτό συμβαίνει όταν η τάξη του πίνακα 
  \[
    \begin{pmatrix}
      u_{{1}_{x}} & u_{{1}_{y}} & u_{{1}_{u}} \\
      u_{{2}_{x}} & u_{{2}_{y}} & u_{{2}_{u}} \\
    \end{pmatrix}
  \]
  είναι ίση με 2.
\end{rem}

\section{Μέθοδοι Επίλυσης του Χαρακτηριστικού Συστήματος}

\subsection{Μία ισότητα εξαρτάται μόνο από δυο μεταβλητές}

Αν κάποια από τις ισότητες του συστήματος~\eqref{eq:systhm} εξαρτάται μόνο από 
δύο μεταβλητές, τότε πρόκειται για μία συνήθη διαφορική εξίσωση.
Λύνοντας αυτή την εξίσωση προκύπτει η 1η λύση της μορφής $ u_{1}(x,y)= c_{1} $. 
Στη συνέχεια μπορούμε να λύσουμε τη σχέση που βρήκαμε ως προς $x$ ή ως προς $y$ και 
να την αντικαταστήσουμε σε οποιοδήποτε από τους λόγους του συστήματος, στον οποίο 
θέλουμε να απαλείψουμε μία μεταβλητή, ώστε να προκύψει μια καινούρια ισότητα 
η οποία να  εξαρτάται μόνο από δύο μεταβλητές.

\begin{example}
  Να λυθεί η εξίσωση $ yu_{x}-xu_{y}=2xyu $  
\end{example}
\begin{solution}
  Έχουμε το χαρακτηριστικό σύστημα
  \[
    \frac{dx}{y} = \frac{dy}{-x} = \frac{du}{2xyu} 
  \] 
  Από την 1η ισότητα προκύπτει η συνήθης διαφορική εξίσωση $ -x dx = y dy $, η 
  οποία είναι χωριζομένων μεταβλητών και με μια απλή ολοκλήρωση παίρνουμε 
  \[
    - \frac{x^{2}}{2} = \frac{y^{2}}{2} + c \Rightarrow x^{2}+y^{2}=-2c \Rightarrow
    \boxed{x^{2}+y^{2}= c_{1}}
  \] 
  όπου $ c_{1} $ είναι μια αυθαίρετη σταθερά. Από τη 2η ισότητα παρατηρούμε ότι 
  μπορούμε να απαλείψουμε τη μεταβλητή $y$, και προκύπτει μια ισότητα λόγων με δύο 
  μεταβλητές
  \[
    \frac{dx}{\cancel{y}} = \frac{du}{2x\cancel{y}u} \Leftrightarrow \frac{dx}{1} =
    \frac{du}{2xu} 
  \] 
  Από την ολοκλήρωση της διαφορικής εξίσωσης $2x dx = \frac{1}{u} du$, παίρνουμε
  \[
    x^{2}= \ln{\abs{u}} + c \Rightarrow x^{2}= \ln{\abs{u}} + \ln{\abs{c'}}
    \Rightarrow x^{2}= \ln{\abs{uc'}} \Rightarrow \abs{uc'} = \mathrm{e}^{x^{2}} 
    \Rightarrow uc' = \pm \mathrm{e}^{x^{2}} 
    \Rightarrow \frac{\mathrm{e}^{x^{2}}}{u} = \pm c' \Rightarrow
    \boxed{\frac{\mathrm{e}^{x^{2}}}{u} = c_{2}}
  \] 
  Επομένως η γενική λύση της μδε θα είναι 
  \[
    F\Bigl(x^{2}+y^{2}, \frac{\mathrm{e}^{x^{2}}}{u}\Bigr) = 0
  \] 
\end{solution}

%todo να συνεχίσω τη μεθοδολογια επιλυσης του χαρ. συστυματος

\end{document}
