\input{preamble_ask.tex}
\input{definitions_ask.tex}

\usepackage[RPvoltages]{circuitikz}


\geometry{left=15.63mm,right=15.63mm,top=30.25mm,bottom=33.25mm,
footskip=24.16mm,headsep=24.16mm}

\pagestyle{vangelis}
% \everymath{\displaystyle}



\begin{document}

\begin{center}
  \minibox{\large\bfseries \textcolor{Col1}{Μηχανικές και Ηλεκτρικές ταλαντώσεις}}
\end{center}

\vspace{\baselineskip}

\begin{problem}
  Ένα σώμα μάζας $ m= \SI{4}{kg} $ κρέμεται από ένα ελατήριο, σταθεράς 
  $ k= \SI{4}{kg/s^{2}} $ και ισορροπεί. Μετατοπίζουμε το σώμα κατά 
  επιπλέον $ \SI{8}{cm} $ προς τη θετική κατεύθυνση, και το αφήνουμε ελεύθερο. 
  Το σύστημα ελατήριο-μάζα, βρίσκεται μέσα σε ρευστό που ασκεί αντίσταση (ανάλογη της 
  ταχύτητας) ίση με $ \SI{6}{\newton} $ όταν η ταχύτητα είναι είναι $\SI{3}{cm/s}$. 
  Να διατυπώσετε το πρόβλημα αρχικών τιμών που περιγράφει την κίνηση του σώματος.
\end{problem}

\hfill Απ: $ \ddot{x}(t)+ 50\dot{x}(t)+ x(t)=0, \; x(0)=0.06, \; \dot{x}(0)=0 $ 

\begin{problem}
  Ένα σώμα μάζας $ m= \SI{5}{kg} $ κρέμεται από ένα ελατήριο, και ισορροπεί, ενώ η 
  επιμήκυνση του ελατηρίου είναι $ \SI{10}{cm} $. Στο σώμα δρα εξωτερική δύναμη ίση με 
  $ 10 \sin{(t/2)} \, \si{N} $. Το σύστημα ελατήριο-μάζα, βρίσκεται μέσα σε ρευστό που 
  ασκεί αντίσταση (ανάλογη της ταχύτητας) ίση με $ \SI{2}{\newton} $ όταν η ταχύτητα 
  είναι είναι $\SI{4}{cm/s}$. Αν το σώμα αρχικά, $ t=0 $ περνά από τη θέση 
  ισορροπίας με ταχύτητα $ \SI{3}{cm/s} $, να διατυπώσετε το πρόβλημα αρχικών τιμών 
  που περιγράφει την κίνηση του σώματος.
  
  \hfill (δίνεται: $ g= \SI{9.8}{m/s^{2}} $)
\end{problem}

\hfill Απ: $ \ddot{x}(t)+10 \dot{x}(t)+98x= 2 \sin{(t/2)}, \; x(0)=0, \; 
\dot{x}(0)=0.03 $  

%xatzikonstantinou p.214
\begin{problem}
  Να υπολογιστεί το ρεύμα $ i(t) $ που διαρρέει ένα κύκλωμα $ RLC $, 
  όπως στο παρακάτω σχήμα, 
  με ωμική αντίσταση $ R= \SI{100}{\ohm} $, συντελεστή αυτεπαγωγής $ L= \SI{0,1}{\henry}
  $ και πυκνωτή χωρητικότητας $ C= \SI{d-3}{\farad} $ τα οποία συνδέονται σε σειρά με 
  μια πηγή τάσης $ E(t) = 155 \sin{377t} $ 
  (ουσιαστικά συχνότητας $ \nu = \omega / 2 \pi = 377/2 \pi = \SI{60}{\hertz} $). 
  Η αρχική συνθήκη είναι ότι τα φορτία και τα ρεύματα για $ t=0 $ είναι μηδέν, δηλαδή  
  $ i(0)=0, \; i'(0)=0 $.
\end{problem}
\begin{center}
  \begin{circuitikz}[american,cute inductors]
    \draw (0,0) to[short,i=$i$,battery1,l=$E$] (0,3) to[L=$L$]  (4,3) to[C=$C$]
    (4,0) to[R=$R$]  (0,0) ;
  \end{circuitikz}
\end{center}

\hfill Απ: $ i(t) = -0,043 \mathrm{e}^{-10t} + 0,526 \mathrm{e}^{-990t} - 0,483
\cos{(377t)} + 1,381 \sin{(377t)} $ 

%Boyce_Diprima 11th edition p. 168
\begin{problem}
  Σε ένα κύκλωμα $ RLC $, όπως στο παρακάτω σχήμα, ή ωμική αντίσταση είναι 
  $ R= \SI{5e3}{\ohm} $, ο συντελεστή αυτεπαγωγής $ L= \SI{1}{\henry} $ και ο πυκνωτής 
  έχει χωρητικότητα $ C= \SI{0.35d-6}{\farad} $. Τα στοιχεία αυτά, συνδέονται σε σειρά 
  με μια πηγή τάσης $ E = \SI{12}{\volt} $. Να υπολογιστεί το φορτίο που διαρρέει 
  τον πυκνωτή κατά τις χρονικές στιγμές $ t_{1}= \SI{0.001}{\sec} $, 
  $ t_{2}= \SI{0.01}{\sec}$, αλλά και κάθε χρονική στιγμή $t$. Επίσης, να υπολογιστεί 
  το οριακό φορτίο καθώς $ t \to \infty $.
\end{problem}
\begin{center}
  \begin{circuitikz}[american,cute inductors]
    \draw (0,0) to[short,i=$i$,battery1,l=$E$] (0,3) to[L=$L$]  (4,3) to[C=$C$]
    (4,0) to[R=$R$]  (0,0) ;
  \end{circuitikz}
\end{center}



\end{document}
