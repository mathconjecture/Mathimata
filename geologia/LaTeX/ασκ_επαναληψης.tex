\documentclass[a4paper,12pt]{article}


\usepackage[english,greek]{babel}
\usepackage[utf8]{inputenc}
\usepackage{amsmath}
\usepackage{amssymb}
\usepackage{amsfonts}
\usepackage{enumerate}

\usepackage{physics}
\usepackage{graphicx}
\usepackage{geometry}

\geometry{left=25.63mm,right=25.63mm,top=36.25mm,bottom=36.25mm,footskip=24.16mm,headsep=24.16mm}

\pagestyle{empty}


\begin{document}

\begin{center}
\fbox{\large \bfseries Ασκήσεις Επανάληψης}
\end{center}

\vspace{2\baselineskip}

\begin{enumerate}

\item Να βρεθεί ο $(A^T \cdot A)^{-1}$ όταν:
\(
A=\begin{pmatrix}
0 & 0 \\
1 & 0 \\
2 & -1 \\
0 & 3 
\end{pmatrix}
\)\hfill Απ: $\frac{1}{46}\begin{pmatrix}
10 & 2 \\
2 & 5
\end{pmatrix}$

\item Να βρεθεί ο $(A^T \cdot A)^{-1}$ όταν:
\(
Α=\begin{pmatrix}
0 & 0 & 1 \\
2 & 0 & 0 \\
2 & 2 & 0 \\
0 & 1 & 1
\end{pmatrix}
\)\hfill Απ: $\frac{1}{40}\begin{pmatrix}
9 & -8 & 4 \\
-8 & 16 & -8 \\
4 & -8 & 24
\end{pmatrix}$


\item Να βρεθεί ο $x$ ώστε ο βαθμός του πίνακα να είναι $1$.

$A=\begin{pmatrix}
\phantom{-}2 & -3 \\
-6 & \phantom{-}x
\end{pmatrix}$\hfill Απ: $x=9$

\item Να βρεθεί ο $x$ ώστε ο βαθμός του πίνακα να είναι $2$.

$Α=\begin{pmatrix}
1 & 2 & 1\\
2 & x & 1 \\
1 & 1 & 0
\end{pmatrix}$\hfill Απ: $x=3$

\item Να υπολογιστεί ο βαθμός των πινάκων:

\begin{enumerate}[i)]

\item $A=\begin{pmatrix}
\phantom{-}3 & \phantom{-}2 & 4 \\
\phantom{-}1 & -2 & 3 \\
-3 & -10 & 1 
\end{pmatrix}$\hfill Απ: $\rank(A)=2$

\item $B=\begin{pmatrix}
1 & 1 & \phantom{-}1 \\
2 & 0 & \phantom{-}4 \\
3 & 2 & \phantom{-}4 \\
0 & 5 & -5
\end{pmatrix}$\hfill Απ: $\rank(B)=2$

\end{enumerate}

\item Δίνεται η συνάρτηση: \[f(x,y,z)=e^{5x}\sin(3y)\cos(4z)\] Να δείξετε ότι \[f_{xx}+f_{yy}+f_{zz}=0\] (ή με άλλα λόγια ότι η συνάρτηση ικανοποιεί την εξίσωση \textlatin{Laplace})

\item Να βρεθεί η $f(x,y)$ αν είναι γνωστές οι μερικές παράγωγοί της:
\[
f_x=2\sin y + 3x^2 \quad \text{και} \quad f_y=2x\cos y 
\]

\hfill Απ: $f(x,y)=2x\sin y + x^3 + 3$

\item Να βρεθεί η $f(x,y)$ αν είναι γνωστές οι παρακάτω σχέσεις: 
\[
f_x=2xy^2+4x^3e^y \quad \text{και} \quad f(-1,y)=y^2+e^y+2y+1
\]

\hfill Απ: $f(x,y)= y^2x^2 +e^yx^4+2y+1$

\item Να βρεθεί η κλίση των συναρτήσεων:

\begin{enumerate}[i)]

\item $f(x,y,z)=x^2-y^2+2xy$ 

\hfill Απ: $\grad f = (2x+2y,-2y+2x,0)$

\item $f(x,y,z)=xz\cos y+z^2\sin y$ 

\hfill Απ: $\grad f=(z\cos y, -xz\sin y + z^2\cos y, x\cos y + 2z\sin y)$

\end{enumerate} 

\item Να βρεθούν η απόκλιση και ο στροβιλισμός για τις παρακάτω διανυσματικές συναρτήσεις (Διανυσματικά Πεδία):
\begin{enumerate}[i)]

\item $\vec{F}(x,y,z)=(2xy+z^2,x^2-2yz,-y^2+2xz)$

\hfill Απ: $\div F = 2y-2z+2x,\quad  \curl F=0$

\item $\vec{F}(x,y,z)=(xz\sin y, z^2-y^2+x^2, 2xyz)$

\hfill Απ: $\div F = z\sin y-2y+2xy,  \quad \curl F=(2xz-2z,x\sin y-2yz,2x-xz\cos y)$
\end{enumerate}

\end{enumerate}



\end{document}