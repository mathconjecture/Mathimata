\documentclass[a4paper,12pt]{article}
\usepackage{etex}
%%%%%%%%%%%%%%%%%%%%%%%%%%%%%%%%%%%%%%
% Babel language package
\usepackage[english,greek]{babel}
% Inputenc font encoding
\usepackage[utf8]{inputenc}
%%%%%%%%%%%%%%%%%%%%%%%%%%%%%%%%%%%%%%

%%%%% math packages %%%%%%%%%%%%%%%%%%
\usepackage{amsmath}
\usepackage{amssymb}
\usepackage{amsfonts}
\usepackage{amsthm}
\usepackage{proof}

\usepackage{physics}

%%%%%%% symbols packages %%%%%%%%%%%%%%
\usepackage{bm} %for use \bm instead \boldsymbol in math mode 
\usepackage{dsfont}
\usepackage{stmaryrd}
%%%%%%%%%%%%%%%%%%%%%%%%%%%%%%%%%%%%%%%


%%%%%% graphicx %%%%%%%%%%%%%%%%%%%%%%%
\usepackage{graphicx}
\usepackage{color}
%\usepackage{xypic}
\usepackage[all]{xy}
\usepackage{calc}
\usepackage{booktabs}
\usepackage{minibox}
%%%%%%%%%%%%%%%%%%%%%%%%%%%%%%%%%%%%%%%

\usepackage{enumerate}

\usepackage{fancyhdr}
%%%%% header and footer rule %%%%%%%%%
\setlength{\headheight}{14pt}
\renewcommand{\headrulewidth}{0pt}
\renewcommand{\footrulewidth}{0pt}
\fancypagestyle{plain}{\fancyhf{}
\fancyhead{}
\lfoot{}
\rfoot{\small \thepage}}
\fancypagestyle{vangelis}{\fancyhf{}
\rhead{\small \leftmark}
\lhead{\small }
\lfoot{}
\rfoot{\small \thepage}}
%%%%%%%%%%%%%%%%%%%%%%%%%%%%%%%%%%%%%%%

\usepackage{hyperref}
\usepackage{url}
%%%%%%% hyperref settings %%%%%%%%%%%%
\hypersetup{pdfpagemode=UseOutlines,hidelinks,
bookmarksopen=true,
pdfdisplaydoctitle=true,
pdfstartview=Fit,
unicode=true,
pdfpagelayout=OneColumn,
}
%%%%%%%%%%%%%%%%%%%%%%%%%%%%%%%%%%%%%%

\usepackage[space]{grffile}

\usepackage{geometry}
\geometry{left=25.63mm,right=25.63mm,top=36.25mm,bottom=36.25mm,footskip=24.16mm,headsep=24.16mm}

%\usepackage[explicit]{titlesec}
%%%%%% titlesec settings %%%%%%%%%%%%%
%\titleformat{\chapter}[block]{\LARGE\sc\bfseries}{\thechapter.}{1ex}{#1}
%\titlespacing*{\chapter}{0cm}{0cm}{36pt}[0ex]
%\titleformat{\section}[block]{\Large\bfseries}{\thesection.}{1ex}{#1}
%\titlespacing*{\section}{0cm}{34.56pt}{17.28pt}[0ex]
%\titleformat{\subsection}[block]{\large\bfseries{\thesubsection.}{1ex}{#1}
%\titlespacing*{\subsection}{0pt}{28.80pt}{14.40pt}[0ex]
%%%%%%%%%%%%%%%%%%%%%%%%%%%%%%%%%%%%%%

%%%%%%%%% My Theorems %%%%%%%%%%%%%%%%%%
\newtheorem{thm}{Θεώρημα}[section]
\newtheorem{cor}[thm]{Πόρισμα}
\newtheorem{lem}[thm]{λήμμα}
\theoremstyle{definition}
\newtheorem{dfn}{Ορισμός}[section]
\newtheorem{dfns}[dfn]{Ορισμοί}
\theoremstyle{remark}
\newtheorem{remark}{Παρατήρηση}[section]
\newtheorem{remarks}[remark]{Παρατηρήσεις}
%%%%%%%%%%%%%%%%%%%%%%%%%%%%%%%%%%%%%%%




\newcommand{\vect}[2]{(#1_1,\ldots, #1_#2)}
%%%%%%% nesting newcommands $$$$$$$$$$$$$$$$$$$
\newcommand{\function}[1]{\newcommand{\nvec}[2]{#1(##1_1,\ldots, ##1_##2)}}

\newcommand{\linode}[2]{#1_n(x)#2^{(n)}+#1_{n-1}(x)#2^{(n-1)}+\cdots +#1_0(x)#2=g(x)}

\newcommand{\vecoffun}[3]{#1_0(#2),\ldots ,#1_#3(#2)}

\newcommand{\mysum}[1]{\sum_{n=#1}^{\infty}



\everymath{\displaystyle}
\pagestyle{vangelis}



\begin{document}


\chapter{Όρια Συναρτήσεων Πολλών Μεταβλητών}

\section{Ορίο Συνάρτησης δύο Μεταβλητών}

\vspace{\baselineskip}

\mydfn{%
  Έστω $ f \colon A \subseteq \mathbb{R}^{2} \to \mathbb{R} $.

  Λέμε ότι η $f$ έχει \textcolor{Col1}{όριο} τον αριθμό $ L \in \mathbb{R} $ στο σημείο 
  $ (x_{0}, y_{0}) $, και το συμβολίζουμε με 
  \[ \lim\limits_{(x,y)\to (x_{0},y_{0})}
    f(x,y) = L \quad \text{ή} \quad \lim_{\substack{x\to x_{0} \\ y\to y_{0}}} 
  f(x,y)=L, \; \text{αν} \] 
  \[
    \forall \varepsilon > 0, \; \exists \delta > 0 \; : \; \forall (x,y) \in A 
    \; \text{με} \; 0 < \sqrt{(x- x_{0})^{2}+(y- y_{0})^{2}} < 
    \delta \Rightarrow \abs{f(x,y)-L} < \varepsilon 
\]}


\section{Βασικά Θεωρήματα}

\mythm{%
  Έστω $ f,g \colon A \subseteq \mathbb{R}^{2} \to \mathbb{R} $ με 
  $ \lim\limits_{\substack{x\to x_{0} \\y \to y_{0}}}  f(x,y) = L \in \mathbb{R} $ και 
  $ \lim\limits_{\substack{x\to x_{0} \\y \to y_{0}}} g(x,y) = M \in \mathbb{R} $. Τότε:

  \begin{minipage}[t]{0.48\textwidth}
    \begin{enumerate}
      \item $ \lim\limits_{\substack{x\to x_{0} \\y \to y_{0}}}  
        [f(x,y)\pm g(x,y)] = L\pm M $
      \item $ \lim\limits_{\substack{x\to x_{0} \\y \to y_{0}}}  
        [a f(x,y)] = aL $ 
      \item $ \lim\limits_{\substack{x\to x_{0} \\y \to y_{0}}}  
        [f(x,y)\cdot g(x,y)] = L\cdot M $
      \item $ \lim\limits_{\substack{x\to x_{0} \\y \to y_{0}}}  
        \frac{f(x,y)}{g(x,y)} = \frac{L}{M}, \; M \neq 0, \; g(x,y) \neq 0 $
      \item $ \lim\limits_{\substack{x\to x_{0} \\y \to y_{0}}}  
        [f(x,y)]^{a} = {L}^{a}, \; a \in \mathbb{R}^{*}_{+}, \; f(x,y) \geq 0 $
      \item $ \lim\limits_{\substack{x\to x_{0} \\y \to y_{0}}} \sqrt[n]{f(x,y)} =
        \sqrt[n]{L}, \; n \in \mathbb{N} \; f(x,y) \geq 0  $
    \end{enumerate}
  \end{minipage} \hfill 
  \begin{minipage}[t]{0.48\textwidth}
    \begin{enumerate}
      \setcounter{enumi}{6}
    \item $ \lim\limits_{\substack{x\to x_{0} \\y \to y_{0}}}  \abs{f(x,y)} = \abs{L} $
    \item $ \lim\limits_{\substack{x\to x_{0} \\y \to y_{0}}} a^{f(x,y)} = 
      a^{L}, \; a >0 $
    \item $ f(x,y) \leq g(x,y), \; \forall (x,y) \in A \Rightarrow L \leq M $
  \end{enumerate}
\end{minipage}
}

\begin{example}
  \begin{align*}
    \lim\limits_{(x,y)\to (0, 0)} \frac{x^{2}-xy}{\sqrt{x} - \sqrt{y}} 
    &= \lim\limits_{(x,y)\to (0, 0)} \frac{x(x-y)(\sqrt{x} + \sqrt{y})}{(\sqrt{x} -
    \sqrt{y} )(\sqrt{x} + \sqrt{y})} = \lim\limits_{(x,y)\to (0, 0)}
    \frac{x(x-y)(\sqrt{x} - \sqrt{y} )}{x-y} \\
    &= \lim\limits_{(x,y)\to (0,0)} [x(\sqrt{x} + \sqrt{y})] = \lim\limits_{(x,y)\to
    (0, 0)} x \cdot \lim\limits_{(x,y)\to (0, 0)} (\sqrt{x} + \sqrt{y}) = 0 \cdot 0 = 0 
  \end{align*}
\end{example}

\mythm{%
  \begin{minipage}{0.3\textwidth}
    \begin{enumerate}[i)]
      \item $ \abs{f(x,y)} \leq \abs{g(x,y)} $ \hfill \tikzmark{a}
      \item $ \lim\limits_{(x,y)\to (x_{0}, y_{0})} g(x,y) = 0 $ \hfill \tikzmark{b}
    \end{enumerate}
    \mybrace{a}{b}[$\lim\limits_{(x,y)\to (x_{0}, y_{0})} f(x,y) = 0 $]
\end{minipage}}

\begin{example}
  Να υπολογίσετε το όριο $ \lim\limits_{(x,y)\to (0, 0)} xy \sin{\frac{1}{x}} $. 

  \begin{solution}
  \item {}
    \begin{minipage}{0.42\textwidth}
      \begin{enumerate}[i)]
        \item $ \abs{xy \sin{\frac{1}{x}}} = \abs{xy} 
          \abs{\sin{\frac{1}{x}}} \leq \abs{xy} \cdot 1 = \abs{xy} $ 
          \hfill \tikzmark{a}
        \item $ \lim\limits_{(x,y)\to (0, 0)} xy = 0$ \hfill \tikzmark{b}
      \end{enumerate}
    \end{minipage}

    \mybrace{a}{b}[$ \lim\limits_{(x,y)\to (0, 0)} xy 
    \sin{\frac{1}{x}} = 0$]
  \end{solution}
\end{example}

\mythm{%
  \begin{minipage}{0.3\textwidth}
    \begin{enumerate}[i)]
      \item $ \abs{g(x,y)} \leq M \in \mathbb{R} $ \hfill \tikzmark{a}
      \item $ \lim\limits_{(x,y)\to (x_{0}, y_{0})} f(x,y) = 0 $ \hfill \tikzmark{b}
    \end{enumerate}
    \mybrace{a}{b}[$ \lim\limits_{(x,y)\to (x_{0}, y_{0})} [f(x,y)\cdot g(x,y)] = 0 $]
\end{minipage}}

\begin{example}
  Να υπολογίσετε το όριο $ \lim\limits_{(x,y)\to (0, 0)} (x^{2}+y^{2}) 
  \sin{\frac{1}{y}} $.  

  \begin{solution}
  \item {}
    \begin{minipage}{0.35\textwidth}
      \begin{enumerate}[i)]
        \item $ \abs{\sin{\frac{1}{y}}} \leq 1 $ \hfill \tikzmark{a}
        \item $ \lim\limits_{(x,y)\to (0, 0)} (x^{2}+y^{2}) = 0+0=0 $ 
          \hfill \tikzmark{b}
      \end{enumerate}
    \end{minipage}
    \mybrace{a}{b}[$ \lim\limits_{(x,y)\to (0, 0)} (x^{2}+y^{2}) \sin{\frac{1}{y}} 
    = 0$]
  \end{solution}
\end{example}


\section{Διαδοχικά ή Επάλληλα όρια}

\mydfn{Ορίζουμε, τα \textcolor{Col1}{διαδοχικά} ή 
  \textcolor{Col1}{επάλληλα} όρια, ως εξής:
  \[
    L_{1} = \lim_{x \to x_{0}} \left(\lim_{y \to y_{0}} f(x,y)\right) 
    \quad \text{και ομοίως} \quad
    L_{2} = \lim_{y \to y_{0}} \left(\lim_{x \to x_{0}} f(x,y)\right) 
\]} 

\begin{rem}
  Αν συμβολίσουμε το όριο $ \lim\limits_{(x,y)\to (x_{0}, y_{0})} f(x,y) = L $,
  τότε ισχύουν οι παρακάτω περιπτώσεις:
  \begin{myitemize}
    \item Αν υπάρχουν τα όρια $ L_{1} $ και $ L_{2} $ και $ L_{1} \neq L_{2} $, τότε
      \textbf{δεν} \textbf{υπάρχει} το $ L $
    \item Αν υπάρχουν τα $ L_{1} $ και $ L_{2} $, τότε 
      \textbf{δεν} \textbf{εξασφαλίζεται} η ύπαρξη του $ L $ ακόμη και αν 
      $ L_{1}=L_{2} $.
    \item Αν υπάρχουν τα όρια $ L_{1} $ και $ L_{2} $ \textbf{και} \textbf{υπάρχει} το 
      $ L $, τότε $ L_{1}=L_{2}=L $
    \item Η ύπαρξη του $ L_{1} $ δεν εξασφαλίζει την ύπαρξη του $ L_{2} $ 
      και αντιστρόφως.
    \item Η ύπαρξη του $ L $ δεν εξασφαλίζει την ύπαρξη των $ L_{1} $ και $ L_{2} $.
  \end{myitemize}
\end{rem}

\begin{example}
  Να υπολογιστεί το όριο $ \lim\limits_{(x,y)\to (0, 0)} \frac{x-y}{x+y} $
  \begin{solution}
    \begin{align*}
      L_{1}&= \lim_{x \to 0} \left(\lim_{y \to 0} \frac{x-y}{x+y}\right) = 
      \lim_{x \to 0} \frac{x}{x} = \lim_{x \to 0} 1 = 1 
      \intertext{και}
      L_{2}&= \lim_{y \to 0} \left(\lim_{x \to 0} \frac{x-y}{x+y} \right) = 
      \lim_{y \to 0} \frac{-y}{y} = \lim_{y \to 0}(-1) = -1
    \end{align*}
    Επομένως αφού $ L_{1} \neq L_{2} $ δεν υπάρχει το ζητούμενο όριο.
  \end{solution}
\end{example}

\begin{example}
  Να υπολογιστεί το όριο $ \lim\limits_{(x,y)\to (0, 0)} \frac{x^{3}+y}{x+y^{3}} $
  \begin{solution}
    \begin{align*}
      L_{1} &= \lim_{x \to 0} \left(\lim_{y \to 0} \frac{x^{3}+y}{x+y^{3}}\right) = 
      \lim_{x \to 0} \left(\frac{x^{3}}{x}\right) = \lim_{x \to 0} \left(x^{2}\right) = 0
      \intertext{και}
      L_{2} &= \lim_{y \to 0} \left( \lim_{x \to 0} \frac{x^{3}+y}{x+y^{3}}\right) = 
      \lim_{y \to 0} \frac{y}{y^{3}} = \lim_{y \to 0} \left(\frac{1}{y^{2}}\right) = 
      +\infty
    \end{align*} 
    Επομένως αφού $ L_{1} \neq L_{2} $ δεν υπάρχει το ζητούμενο όριο.
  \end{solution}
\end{example}

\begin{example}
  Να υπολογιστεί το όριο $ \lim\limits_{(x,y)\to (0, 0)} \frac{xy}{x^{2}+y^{2}} $.
  \begin{solution}
    \begin{align*}
      L_{1} &= \lim_{x \to 0} \left(\lim_{y \to 0} \frac{xy}{x^{2}+y^{2}}\right) = 
      \lim_{x \to 0} \frac{0}{x^{2}} = \lim_{x \to 0} 0 = 0
      \intertext{και}
      L_{2} &= \lim_{y \to 0} \left(\lim_{x \to 0} \frac{xy}{x^{2}+y^{2}}\right) = 
      \lim_{y \to 0} \frac{0}{y^{2}} = \lim_{y \to 0} 0 = 0
    \end{align*}
  \end{solution}
  Παρατηρούμε ότι $ L_{1} = L_{2} $, όμως δεν γνωρίζουμε αν υπάρχει το $L$, 
  οπότε δεν μπορούμε να ισχυριστούμε ότι $ L=0 $.
\end{example}


\section{Συνέχεια Συναρτήσεων Πολλών Μεταβλητών}

\mydfn{Η συνάρτηση $ f \colon A \subseteq \mathbb{R}^{2} \to \mathbb{R} $ 
  λέγεται \textcolor{Col1}{συνεχής στο $ (x_{0}, y_{0}) \in A $} αν 
$ \lim\limits_{(x,y)\to (x_{0}, y_{0})} f(x,y) = f(x_{0}, y_{0}) $.}

\mydfn{Η συνάρτηση $ f \colon A \subseteq \mathbb{R}^{2} \to \mathbb{R} $ 
  λέγεται \textcolor{Col1}{συνεχής στο Α}, αν είναι συνεχής σε \textbf{κάθε} σημείο 
$ (x_{0}, y_{0}) \in A $.}

\begin{rem}
  Από τον παραπάνω ορισμό, καταλαβαίνουμε ότι μια συνάρτηση $ f(x,y) $ 
  είναι συνεχής στο σημείο $ (x_{0}, y_{0}) $, αν
  \begin{myitemize}
    \item Υπάρχει το $ \lim\limits_{\substack{x\to x_{0} \\y \to y_{0}}} f(x,y) $
    \item Υπάρχει το $ f(x_{0}, y_{0}) \in \mathbb{R} $
    \item $  \lim\limits_{\substack{x\to x_{0} \\y \to y_{0}}} f(x,y)= 
      f(x_{0}, y_{0}) $
  \end{myitemize}
\end{rem}

\begin{example}
  Να μελετηθεί πλήρως ως προς τη συνέχεια η συνάρτηση 
  $ f(x,y) = \frac{x^{2}+y^{4}}{x^{4}+y^{2}} $.
  \begin{solution}
  \item {}
    Παρατηρούμε ότι η συνάρτηση δεν ορίζεται στο σημείο $ (0,0) $. Οπότε:

    Αν $ (x,y) \neq (0,0) $ τότε η $f= \frac{x^{4}+y^{2}}{x^{4}+y^{2}} $ είναι συνεχής 
    ως ρητή συνάρτηση.

    Αν $ (x,y)=(0,0) $, εξετάζουμε το όριο της $f$: 

    Εξετάζουμε τα επάλληλα όρια της $f$ και έχουμε
    \begin{myitemize}
      \item $ L_{1} = \lim_{x \to 0} 
        \left( 
          \lim_{y \to 0} \frac{x^{2}+y^{4}}{x^{4}+y^{2}} 
        \right) = \lim_{x \to 0} \frac{x^{2}}{x^{4}} = \lim_{x \to 0} \frac{1}{x^{2}}
        \overset{(\frac{1}{0^{+}})}{=} +\infty $
      \item $ L_{2} = \lim_{y \to 0} 
        \left(
          \lim_{x \to 0} 
          \frac{x^{2}+y^{4}}{x^{4}+y^{2}}
        \right) = \lim_{y \to 0} \frac{y^{4}}{y^{2}} = \lim_{y \to 0} y^{2} = 0 $
    \end{myitemize}
    Επομένως δεν υπάρχει το όριο της $f$ στο $ (0,0) $, γιατί τα επάλληλα όρια δεν 
    είναι ίσα.
    Άρα η $f$ δεν είναι συνεχής στο σημείο $(0,0)$.
  \end{solution}
\end{example}


\end{document}
