\input{preamble.tex}
\newcommand{\vect}[2]{(#1_1,\ldots, #1_#2)}
%%%%%%% nesting newcommands $$$$$$$$$$$$$$$$$$$
\newcommand{\function}[1]{\newcommand{\nvec}[2]{#1(##1_1,\ldots, ##1_##2)}}

\newcommand{\linode}[2]{#1_n(x)#2^{(n)}+#1_{n-1}(x)#2^{(n-1)}+\cdots +#1_0(x)#2=g(x)}

\newcommand{\vecoffun}[3]{#1_0(#2),\ldots ,#1_#3(#2)}

\newcommand{\suma}{\sum_{n=0}^{\infty}a_n x^n}

\newcommand{\sumb}{\sum_{n=1}^{\infty}a_n n x^{n-1}}

\newcommand{\sumc}{\sum_{n=2}^{\infty}a_n n (n-1) x^{n-2}}

\newcommand{\varsum}[2]{\sum_{n=#1}^{#2}}

\pagestyle{empty}

\begin{document}

\begin{center}
\minibox{\large\bfseries Ασκήσεις Στη Μέθοδο Σειρών}
\end{center}

\vspace{\baselineskip}

\begin{enumerate}
\item Να εξετάσετε τι είδους σημείο είναι το $x_0=0$ για τις παρακάτω εξισώσεις:
\begin{enumerate}[i)]
\item $x^3y''+\sin(2x)y=0, \quad y=y(x)$\hfill Απ: κανονικό ιδιάζον
\item $x^4y'' + (\cos x-1)y'=0, \quad y=y(x)$\hfill Απ: μη-κανονικό ιδιάζον
\item $(x-x^2)y''-(1+2x)y'+2y=0, \quad y=y(x)$\hfill Απ: κανονικό ιδιάζον
\item $x^{a}y''+\sin x y=0, y=y(x), \quad a=1, \; a=4$\hfill Απ: \minibox[]{$a=1$, κανονικό ιδιάζον\\ 
$a=4$, μη-κανονικό ιδιάζον}
\end{enumerate}

\item Για τις παρακάτω εξισώσεις να βρεθούν δύο γραμμικώς ανεξάρτητες λύσεις, υπό μορφή σειράς γύρω από το σημείο $x_0=0$.
\begin{enumerate}[i)]
\item $y''-xy'+y=0$\hfill Απ: \begin{tabular}[t]{l}
$y_1=1-\frac{1}{2}x^2+\sum\limits_{n=2}^{\infty}(-1)\frac{(2n-3)!!}{(2n)!}x^{2n}$\\
$y_2=x$
\end{tabular}
\item $y''+x^2y'-4xy=0$\hfill Απ:\begin{tabular}[t]{l}$y_1=1+\frac{2}{3}x^3+\frac{1}{4\cdot 5}x^6-\frac{1}{16\cdot 20}x^9+\cdots$ \\[5pt]
$y_2=x+\frac{1}{4}x^4$
\end{tabular}
\item $y''-xy'+y=5, y(0)=5, y'(0)=3$\hfill Απ: \begin{tabular}[t]{l}
$y(x)=5+3x$
\end{tabular}
\item $(x-x^2)y''-(1+2x)y'+2y=0$\hfill Απ: \begin{tabular}[t]{l}
$y_1=1+2x$\\
$y_2=\sum\limits_{n=2}^{\infty}\frac{n+1}{3}x^n$
\end{tabular}
\item $xy''+y=0$\hfill Απ: \begin{tabular}[t]{l}
$y_1=-x+\sum\limits_{n=2}^{\infty}(-1)^n\frac{n}{(n!)^2}x^n$\\
$y_2=y_1\ln x+1+x+\sum\limits_{n=2}^{\infty}(-1)^{n+1}\frac{n}{(n!)^2}(2H_{n-1}+\frac{1}{n})x^n$
\end{tabular}
\item $xy''+2y'-xy=0$\hfill Απ: \begin{tabular}[t]{l}
$y_1=\frac{\cosh x}{x}$\\
$y_2=\frac{\sinh x}{x}$
\end{tabular}
\item $xy''-2y'+xy=0$\hfill Απ: \begin{tabular}[t]{l}
$y_1=1+\sum\limits_{n=1}^{\infty}\frac{(-1)^{n+1}}{2n(2n-2)!}x^{2n}$\\
$y_2=\sum\limits_{n=2}^{\infty}\frac{(-1)^n}{(2n-1)(2n-3)!}x^{2n-1}$
\end{tabular}


\end{enumerate}

\end{enumerate}


\end{document}
