\documentclass[a4paper,12pt]{article}
\usepackage{etex}
%%%%%%%%%%%%%%%%%%%%%%%%%%%%%%%%%%%%%%
% Babel language package
\usepackage[english,greek]{babel}
% Inputenc font encoding
\usepackage[utf8]{inputenc}
%%%%%%%%%%%%%%%%%%%%%%%%%%%%%%%%%%%%%%

%%%%% math packages %%%%%%%%%%%%%%%%%%
\usepackage{amsmath}
\usepackage{amssymb}
\usepackage{amsfonts}
\usepackage{amsthm}
\usepackage{proof}

\usepackage{physics}

%%%%%%% symbols packages %%%%%%%%%%%%%%
\usepackage{bm} %for use \bm instead \boldsymbol in math mode 
\usepackage{dsfont}
\usepackage{stmaryrd}
%%%%%%%%%%%%%%%%%%%%%%%%%%%%%%%%%%%%%%%


%%%%%% graphicx %%%%%%%%%%%%%%%%%%%%%%%
\usepackage{graphicx}
\usepackage{color}
%\usepackage{xypic}
\usepackage[all]{xy}
\usepackage{calc}
\usepackage{booktabs}
\usepackage{minibox}
%%%%%%%%%%%%%%%%%%%%%%%%%%%%%%%%%%%%%%%

\usepackage{enumerate}

\usepackage{fancyhdr}
%%%%% header and footer rule %%%%%%%%%
\setlength{\headheight}{14pt}
\renewcommand{\headrulewidth}{0pt}
\renewcommand{\footrulewidth}{0pt}
\fancypagestyle{plain}{\fancyhf{}
\fancyhead{}
\lfoot{}
\rfoot{\small \thepage}}
\fancypagestyle{vangelis}{\fancyhf{}
\rhead{\small \leftmark}
\lhead{\small }
\lfoot{}
\rfoot{\small \thepage}}
%%%%%%%%%%%%%%%%%%%%%%%%%%%%%%%%%%%%%%%

\usepackage{hyperref}
\usepackage{url}
%%%%%%% hyperref settings %%%%%%%%%%%%
\hypersetup{pdfpagemode=UseOutlines,hidelinks,
bookmarksopen=true,
pdfdisplaydoctitle=true,
pdfstartview=Fit,
unicode=true,
pdfpagelayout=OneColumn,
}
%%%%%%%%%%%%%%%%%%%%%%%%%%%%%%%%%%%%%%

\usepackage[space]{grffile}

\usepackage{geometry}
\geometry{left=25.63mm,right=25.63mm,top=36.25mm,bottom=36.25mm,footskip=24.16mm,headsep=24.16mm}

%\usepackage[explicit]{titlesec}
%%%%%% titlesec settings %%%%%%%%%%%%%
%\titleformat{\chapter}[block]{\LARGE\sc\bfseries}{\thechapter.}{1ex}{#1}
%\titlespacing*{\chapter}{0cm}{0cm}{36pt}[0ex]
%\titleformat{\section}[block]{\Large\bfseries}{\thesection.}{1ex}{#1}
%\titlespacing*{\section}{0cm}{34.56pt}{17.28pt}[0ex]
%\titleformat{\subsection}[block]{\large\bfseries{\thesubsection.}{1ex}{#1}
%\titlespacing*{\subsection}{0pt}{28.80pt}{14.40pt}[0ex]
%%%%%%%%%%%%%%%%%%%%%%%%%%%%%%%%%%%%%%

%%%%%%%%% My Theorems %%%%%%%%%%%%%%%%%%
\newtheorem{thm}{Θεώρημα}[section]
\newtheorem{cor}[thm]{Πόρισμα}
\newtheorem{lem}[thm]{λήμμα}
\theoremstyle{definition}
\newtheorem{dfn}{Ορισμός}[section]
\newtheorem{dfns}[dfn]{Ορισμοί}
\theoremstyle{remark}
\newtheorem{remark}{Παρατήρηση}[section]
\newtheorem{remarks}[remark]{Παρατηρήσεις}
%%%%%%%%%%%%%%%%%%%%%%%%%%%%%%%%%%%%%%%




\newcommand{\vect}[2]{(#1_1,\ldots, #1_#2)}
%%%%%%% nesting newcommands $$$$$$$$$$$$$$$$$$$
\newcommand{\function}[1]{\newcommand{\nvec}[2]{#1(##1_1,\ldots, ##1_##2)}}

\newcommand{\linode}[2]{#1_n(x)#2^{(n)}+#1_{n-1}(x)#2^{(n-1)}+\cdots +#1_0(x)#2=g(x)}

\newcommand{\vecoffun}[3]{#1_0(#2),\ldots ,#1_#3(#2)}

\newcommand{\mysum}[1]{\sum_{n=#1}^{\infty}



\thispagestyle{empty}



\begin{document}

\begin{center}
\fbox{\large\bfseries Ασκήσεις στις Ιδιοτιμές - Ιδιοδιανύσματα Πίνακα}
\end{center}

\vspace{\baselineskip}

\begin{enumerate}

\item Να βρεθούν οι ιδιοτιμές και τα ιδιοδιανύσματα των παρακάτω πινάκων:

\begin{enumerate}[i)]

\item $\begin{pmatrix}
3 & 2 \\
2 & 3
\end{pmatrix}$\hfill Απ: \begin{tabular}{l}
$\lambda_1=1$, $\lambda_2=5$ \\
$\bm{\delta}_1=(-1,1)^T$ \\
$\bm{\delta}_2=(1,1)^T$
\end{tabular}

\item $\begin{pmatrix}
2 & -3 \\
3 & \phantom{-}2
\end{pmatrix}$\hfill Απ: \begin{tabular}{l}
$\lambda_{1,2}=2\pm 3i$ \\
$\bm{\delta}_1=(i,1)^T$ \\
$\bm{\delta}_2=(-i,1)^T$
\end{tabular}

\item $\begin{pmatrix*}[r]
\phantom{-}13 & -3 & \phantom{-}5 \\
\phantom{-}0 & \phantom{-}4 & \phantom{-}0 \\
-15 & \phantom{-}9 & -7
\end{pmatrix*}$ \hfill Απ: \begin{tabular}{l}
$\lambda_1=4$, $\lambda_2=8$, $\lambda_3=-2$ \\ 
$\bm{\delta}_1=(1,-2,-3)^T$ \\
$\bm{\delta}_2=(1,0,-1)^T$ \\
$\bm{\delta}_3=(1,0,-3)^T$
\end{tabular}

\end{enumerate}

\item Να βρεθούν οι ιδιοτιμές και τα ιδιοδιανύσματα των παρακάτω πινάκων και να εξετάσετε αν οι πίνακες διαγωνιοποιούνται:

\begin{enumerate}[i)]

\item $\begin{pmatrix}
2 & 1 \\
0 & 2
\end{pmatrix}$\hfill Απ: \begin{tabular}{l}
$\lambda_{1,2}=2 \; (\text{διπλή})$ \\
$\bm{\delta}_1=(1,0)^T$ \\
\end{tabular}

\item $\begin{pmatrix}
-2 & 0 & \phantom{-}0 \\
-2 & 2 & \phantom{-}2 \\
\phantom{-}0 & 0 & -2
\end{pmatrix}$ \hfill Απ: \begin{tabular}{l}
$\lambda_1=2$, $\lambda_{2,3}=-2 \; (\text{διπλή})$ \\ 
$\bm{\delta}_1=(0,1,0)^T$ \\
$\bm{\delta}_2=(2,1,0)^T$ \\
$\bm{\delta}_3=(1,0,1)^T$
\end{tabular}



\item $\begin{pmatrix}
\phantom{-}0 & \phantom{-}1 & -2 \\
-2 & \phantom{-}2 & -2 \\
\phantom{-}1 & -1 & \phantom{-}3
\end{pmatrix}$ \hfill Απ: \begin{tabular}{l}
$\lambda_1=1$, $\lambda_{2,3}=2 \; (\text{διπλή})$ \\ 
$\bm{\delta}_1=(0,2,1)^T$ \\
$\bm{\delta}_2=(-1,0,1)^T$ \\
\end{tabular}

\item $ \begin{pmatrix*}[r]
		-3 & -2 & -4 \\
		0 & -1 & 0 \\
		1 & 1 & 1 
\end{pmatrix*}$ \hfill Απ: \begin{tabular}{l}
$ \lambda_{1} = \lambda_{2} = \lambda_{3} = -1 $ \\
$ \bm{\delta}_{1} = (-1,1,0)^T$ \\
$ \bm{\delta}_{2} = (-2,0,1)^{T} $
\end{tabular}

\item $ \begin{pmatrix*}[r]
		-4 & 0  & 1 \\
		4 & -3 & -3 \\
		-1 & 0 & -2 
\end{pmatrix*}$ \hfill Απ: \begin{tabular}{l}
$ \lambda_{1} = \lambda_{2} = \lambda_{3} = -3 $ \\
$ \bm{\delta}_{1} = (0,1,0)^{T} $
\end{tabular}

\end{enumerate}

\item Αν $A=\begin{pmatrix}
1 & 4 \\
2 & 3 
\end{pmatrix}$ τότε να υπολογιστεί ο $A^{10}$.\hfill Απ: $A^{10}=\frac{1}{3}\begin{pmatrix}
5^{10}+2 & 2\cdot 5^{10}-2 \\
5^{10}-1 & 2\cdot 5^{10}+1
\end{pmatrix}$

\item Για ποιες τιμές του $\lambda$ ο πίνακας $A=\begin{pmatrix}
2 & -1 & 1 \\
0 & \phantom{-}1 & 1 \\
0 & \phantom{-}0 & \lambda
\end{pmatrix}$
διαγωνιοποιείται?

\item Να βρεθεί ο ορθογώνιος πίνακας $Q$ που διαγωνιοποιεί τον πίνακα
	$A = \begin{pmatrix*}[r]
		1 & 2 & 2 \\
		2 & 1 & 2 \\
		2 & 2 & 5 
	\end{pmatrix*}.$
	
	\hfill Απ: $ Q = \begin{pmatrix*}[r]
		1/ \sqrt{6} & -1/ \sqrt{2} & -1/ \sqrt{3}	\\
		1/ \sqrt{6} & 1/ \sqrt{2} & -1\ \sqrt{3} \\
		2/ \sqrt{6} & 0 & 1/ \sqrt{3} 
	\end{pmatrix*} $

\end{enumerate}



\end{document}
