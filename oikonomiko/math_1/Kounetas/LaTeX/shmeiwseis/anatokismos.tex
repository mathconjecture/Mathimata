\documentclass[a4paper,table]{report}
\input{preamble_ask.tex}
\newcommand{\vect}[2]{(#1_1,\ldots, #1_#2)}
%%%%%%% nesting newcommands $$$$$$$$$$$$$$$$$$$
\newcommand{\function}[1]{\newcommand{\nvec}[2]{#1(##1_1,\ldots, ##1_##2)}}

\newcommand{\linode}[2]{#1_n(x)#2^{(n)}+#1_{n-1}(x)#2^{(n-1)}+\cdots +#1_0(x)#2=g(x)}

\newcommand{\vecoffun}[3]{#1_0(#2),\ldots ,#1_#3(#2)}

\newcommand{\mysum}[1]{\sum_{n=#1}^{\infty}


% \setcounter{chapter}{1}

\everymath{\displaystyle}

\pagestyle{askhseis}


\begin{document}

\chapter*{Ανατοκισμός}

\section*{Απλός Τόκος (Simple Interest)}

\begin{dfn}
  Ο \textcolor{Col1}{απλός τόκος} είναι ένα \textbf{σταθερό} ποσοστό του αρχικού 
  κεφαλαίου $ P_{0} $ που καταβάλλεται στον επενδυτή κάθε έτος. Αν $ i $ είναι το 
  επιτόκιο, τότε ο τόκος μετά από $t$ έτη, θα είναι:
  \[
    I=P_{0}\cdot i \cdot t 
  \] 
  Επομένως η μελλοντική αξία μετά από $t$ έτη, $ P_{t} $, θα βρεθεί αν στην παρούσα αξία
  $ P_{0} $, προσθέσουμε τον τόκο. Επομένως:
  \[
    P_{t} = P_{0} + I = P_{0} + P_{0}\cdot i\cdot t = P_{0}(1+i\cdot t)
  \] 
  Τότε, η παρούσα αξία θα είναι
  \[
    P_{0} = \frac{P_{t}}{1+ i \cdot t }  
  \] 
\end{dfn}



\section*{Σύνθετος Τόκος (Compound Interest)}

Με το \textcolor{Col1}{σύνθετο τόκο}, το επιτόκιο καταβάλλεται στο αρχικό κεφάλαιο αλλά 
και στο τοκισμένο κεφάλαιο, κάθε έτος. 
\begin{description}[widest=3ο Έτος,labelsep*=1ex,leftmargin=*]
  \item[1ο Έτος:] Έστω ότι στην αρχή του έτους, το κεφάλαιο είναι $ P_{0} $. Ο τόκος 
    που καταβάλλεται το 1ο έτος είναι $ I = P_{0} \cdot i $ και άρα στο τέλος του 
    1ου έτους, το κεφάλαιο θα είναι:
    \[
      P_{1} = P_{0} (1+i)  
    \] 
  \item [2ο Έτος:] Το κεφάλαιο, είναι τώρα $ P_{1}=P_{0}(1+i) $. Ο τόκος που 
    καταβάλλεται το 2ο έτος είναι $ I = P_{1}\cdot i $ και άρα στο τέλος του 
    2ου έτους, το κεφάλαιο θα είναι:
    \[
      P_{2} = P_{1}(1+i) = P_{0}(1+i)(1+i) = P_{0} (1+i)^{2} 
    \] 
  \item [3ο Έτος:] Το κεφάλαιο, είναι τώρα $ P_{2}=P_{0}(1+i)^{2} $. Ο τόκος που 
    καταβάλλεται το 3ο έτος είναι $ I = P_{2}\cdot i $ και άρα στο τέλος του 
    3ου έτους, το κεφάλαιο θα είναι:
    \[
      P_{3} = P_{2}(1+i) = P_{0}(1+i)^{2}(1+i) = P_{0} (1+i)^{3} 
    \] 
  \item [$t$ Έτος:] Το κεφάλαιο, είναι τώρα $ P_{t-1}=P_{0}(1+i)^{t-1} $. Ο τόκος που 
    καταβάλλεται το $t$ έτος είναι $ I = P_{t-1}\cdot i $ και άρα στο τέλος του 
    $t$ έτους, το κεφάλαιο θα είναι:
    \[
      P_{t} = P_{t-1}(1+i) = P_{0}(1+i)^{t-1}(1+i) = P_{0} (1+i)^{t} 
    \] 
\end{description}

\begin{rem}
  Παρατηρούμε ότι τα ποσά που έχουμε για κάθε έτος, είναι:
  \[
    P_{0}, \; P_{0}(1+i), \; P_{0}(1+i)^{2}, \; \ldots \;, P_{0}(1+i)^{t} 
  \] 
  και αποτελούν τους όρους \textbf{γεωμετρικής προόδου}, με 1ο όρο $ a_{1}=P_{0} $ και 
  λόγο $ \lambda =1+i $.
\end{rem}

\begin{example}
  Έστω ότι επενδύουμε $ 8000 $ ευρώ με $ 10 \% $ σύνθετο τόκο για 3 έτη.
  \begin{enumerate}[i)]
    \item Να υπολογιστεί η τιμή της επένδυσης μετά τα τρία έτη.
    \item Να υπολογιστεί η παρούσα αξία αν το κεφάλαιο που παίρνουμε μετά από 3 έτη 
      είναι $14641$ ευρώ.
  \end{enumerate}
\end{example}
\begin{solution}
\item {}
  \begin{enumerate}[i)]
    \item 
      Έχουμε ότι $ P_{0} = 8000  $ και $ i= \frac{10}{100} = 0.1 $, ενώ $ t=3 $ έτη. 
      Επομένως, η μέλλουσα αξία της επένδυσης, μετά από 3 έτη θα είναι:
      \[
        P_{3} = 8000(1+0.1)^{3} = 8000 \cdot 1.1^{3} = 8000 \cdot 1.331 = 10648.
      \] 
    \item Για να υπολογίσουμε την παρούσα αξία, δηλαδή το ποσό που πρέπει να 
      επενδύσουμε σήμερα, ώστε μετά από 3 έτη, να έχουμε $14641$, θα είναι:
      \[
        P_{0} = \frac{14641}{(1+i)^{3}} = \frac{14641}{1.1^{3}} = 11000 
      \] 
  \end{enumerate}
\end{solution}

\begin{example}
  Να βρεθεί το επιτόκιο ανατοκισμού που απαιτείται προκειμένου $10000$ ευρώ να αυξηθούν 
  σε $30000$ ευρώ σε 8 έτη.
\end{example}
\begin{solution}
  Έχουμε ότι, $P_{0}=10000$, $ P_{8}=30000 $ και $ t=8 $ έτη. Για να βρούμε το επιτόκιο,
  $ i $, από τον τύπο $ P_{t}=P_{0}(1+i) $, έχουμε:
  \[
    30000 = 10000(1+i)^{8} \Leftrightarrow \frac{30000}{10000} = (1+i)^{8} \Leftrightarrow
    3 = (1+i)^{8} \Leftrightarrow 1+i = 3^{1/8} 
  \] 
  άρα $ i = 3^{1/8} -1 = 1.15-1 = 0.15 $ ή $ 15\% $.
\end{solution}

\begin{example}
  Αν μια τράπεζα προσφέρει επιτόκιο $ 7\% $, με ετήσιο ανατοκισμό, να βρεθεί το χρονικό 
  διάστημα που απαιτείται προκειμένου ένα αρχικό κεφάλαιο $ 10000 $ ευρώ να δώσει ολική 
  αξία $ 20000 $ ευρώ.
\end{example}
\begin{solution}
  Έχουμε $ P_{0} = 10000$, $ P_{t}=20000 $ και $ i= \frac{7}{100} = 0.07 $. Ζητάμε 
  το χρόνο της επένδυσης, $t$, οπότε:
  \[
    20000 = 10000 (1+0.07)^{t} \Leftrightarrow \frac{20000}{10000} = (1.07)^{t}
    \Leftrightarrow 2 = (1.07)^{t} 
  \]
  οπότε λογαριθμίζοντας, παίρνουμε:
  \[ \ln{2} = \ln{(1.07)}^{t} \Leftrightarrow \ln{2} = t \ln{(1.07)} \Leftrightarrow t =
  \frac{\ln{2}}{\ln{(1.07)}} \approx \frac{0.69}{0.067} \approx 10.3 \; \text{έτη} \]
\end{solution}


\section*{Ανατοκισμός που γίνεται πολλές φορές Ετησίως}

Κάθε χρονική περίοδος, κατά την οποία γίνεται ο ανατοκισμός ονομάζεται
\textcolor{Col1}{περίοδος μετατροπής} (conversion period) ή τοκοφόρα περίοδος 
(interest period). Αν το πλήθος των περιόδων μετατροπής ετησίως, το συμβολίσουμε με 
$ m $, τότε το επιτόκιο που αντιστοιχεί σε κάθε περίοδο θα είναι $ {i}/{m} $. Αν 
συμβολίσουμε με $n$ το συνολικό αριθμό των περιόδων μετατροπής και με $t$ τον αριθμό 
των ετών, τότε η αξία της επένδυσης, μετά το τέλος της $n$ περιόδου, θα είναι:
\[
  P_{t} = P_{0} \left(1+ \frac{i}{m} \right)^{n} = 
  P_{0}\left(1+ \frac{i}{m} \right)^{mt} 
\] 

\begin{example}
  Επενδύονται $ 10000 $ ευρώ για 3 έτη με εξαμηνιαίο ανατοκισμό και ετήσιο επιτόκιο 
  $ 8\% $. Στο τέλος των 3 ετών:
  \begin{enumerate}[i)]
    \item Να υπολογιστεί η συνολική αξία της επένδυσης
    \item Να συγκριθεί η απόδοση (return) της επένδυσης στις περιπτώσεις του ετήσιου και
      του εξαμηνιαίου ανατοκισμού.
  \end{enumerate}
\end{example}
\begin{solution}
\item {}
  \begin{enumerate}[i)]
    \item Έχουμε $ P_{0} = 10000 $, $ i = {8}/{100} = 0.08 $, $ t=3 $ έτη, 
      και αφού ο ανατοκισμός είναι εξαμηνιαίος, δηλαδή 2 φορές το χρόνο, έχουμε ότι 
      $m=2 $ και άρα οι συνολικές περίοδοι ανατοκισμού στα 3 έτη, θα είναι $n=2\cdot 3 =
      6$. Επομένως
      \[
        P_{3} = 10000\left(1+ \frac{0.08}{2} \right)^{6} = 
        10000(1+0.04)^{6} = 10000 (1.04)^{6} = 12653.19
      \] 
    \item Για τον απλό τόκο, έχουμε $ P_{0}=10000 $, $ i=0.08 $ και $ t=3 $ έτη. Άρα
      \[
        P_{3} = 10000(1+0.08)^{3} = 10000(1.08)^{3} = 10000\cdot 1.259712 = 12597.12 
      \] 
      Επομένως το κέρδος που προκύπτει από τον εξαμηνιαίο ανατοκισμό, έναντι του 
      ετήσιου, είναι
      \[
        12653.19-12597.12 = 56.07, \; \text{ευρώ}
      \] 
  \end{enumerate}
\end{solution}

\begin{example}
  Επενδύονται $10000$ ευρώ για 3 έτη με ετήσιο επιτόκιο $ 8\% $. Στο τέλος 3 ετών 
  να υπολογιστεί η αξία της επένδυσης, όταν:
  \begin{enumerate}[i)]
    \item Ο ανατοκισμός είναι μηνιαίος
    \item Ο ανατοκισμός είναι ημερήσιος
  \end{enumerate}
\end{example}
\begin{solution}
\item {}
  \begin{enumerate}[i)]
    \item Έχουμε $ P_{0} = 10000 $, $ i=0.08 $, $ t=3 $ έτη και αφού ο ανατοκισμός 
      είναι μηνιαίος, τότε $ m=12 $, και οι συνολικές περίοδοι ανατοκισμού στα 3 έτη 
      θα είναι $ n=12\cdot 3 = 36 $. Επομένως
      \[
        P_{3} = 10000\left(1+ \frac{0.08}{12} \right)^{36} = 12702 
      \] 
    \item Έχουμε $ P_{0} = 10000 $, $ i=0.08 $, $ t=3 $ έτη και αφού ο ανατοκισμός 
      είναι ημερήσιος, τότε $ m=365 $, και οι συνολικές περίοδοι ανατοκισμού στα 3 έτη 
      θα είναι $ n=365\cdot 3 = 1095 $. Επομένως
      \[
        P_{3} = 10000\left(1+ \frac{0.08}{365} \right)^{1095} = 12712.37 
      \] 
  \end{enumerate}
\end{solution}



\chapter*{Ράντες}

\begin{dfn}
  Μια \textcolor{Col1}{ράντα} είναι μια σειρά από ισόποσες καταθέσεις ή αναλήψεις που 
  γίνονται σε ίσα χρονικά διαστήματα. Αν η κατάθεση γίνεται τη χρονική στιγμή του 
  ανατοκισμού τότε η ράντα λέγεται \textcolor{Col1}{συνήθης ράντα}.
\end{dfn}
Έστω $ A_{0} $ ότι είναι το αρχικό ποσό και $i$ το επιτόκιο. Τότε στο τέλος κάθε έτους, 
για $ t $ έτη, η τιμή της ράντας θα είναι:
\begin{description}[widest=[3o Έτος,labelsep*=1ex,leftmargin=*]
  \item [1ο έτος:] τιμή της ράντας: $ A_{0} $
  \item [2ο έτος:] τιμή της ράντας: $ A_{0}(1+i) + A_{0} $
  \item [3ο έτος:] τιμή της ράντας: $ A_{0}(1+i)^{2} + A_{0}(1+i) + A_{0} $
  \item [$t$ έτος:] τιμή της ράντας: $ A_{0}(1+i)^{t-1} + A_{0}(1+i)^{t-2} + \cdots + 
    A_{0}(1+i) + A_{0} $
\end{description}
Οπότε το συνολικό ποσό της ράντας, είναι το άθροισμα των $t$ πρώτων όρων της 
γεωμετρικής προόδου με 1ο όρο $ a_{1}=A_{0} $ και λόγο $ \lambda = 1+i $. Από τον 
γνωστό τύπο, για το άθροισμα των $n$ πρώτων όρων γεωμετρικής προόδου:
\[
  S_{n} = a_{1} \frac{\lambda ^{n}-1}{\lambda -1} 
\] 
έχουμε ότι η αξία της ράντας, μετά από $t$ έτη,  θα είναι:
\[
  V_{t} = A_{0} \frac{(1+i)^{t}-1}{1+i-1} = A_{0} \frac{(1+i)^{t}-1}{i}
\] 

\begin{example}
Έστω ότι αποφασίζετε να αποταμιεύετε ένα σταθερό ποσό ετησίως. Αν η αποταμίευση 
συνεχιστεί για 10 έτη με ετήσιο επιτόκιο $ 7\% $, τότε:
\begin{enumerate}[i)]
  \item Να υπολογιστεί η αξία του αποθέματος στο τέλος των 10 ετών αν η κατάθεση είναι 
    500 ευρώ ετησίως.
  \item Ποιο είναι το ποσό που πρέπει να αποταμιεύετε κάθε χρόνο προκειμένου μετά από 
    10 έτη να έχετε αποθεματικό αξίας 50000 ευρώ; 
  \item Τι ποσό πρέπει να αποταμιεύετε κάθε μήνα, προκειμένου μετά από 10 έτη να έχετε
    αποθεματικό αξίας 50000; 
\end{enumerate}
\begin{solution}
\item {}
  \begin{enumerate}[i)]
    \item Έχουμε $ A_{0} = 500 $, $ i=0,07 $, $ t=10 $ έτη. Άρα
      \[
        V_{t} = A_{0} \frac{(1+i)^{t}-1}{i} = 500 \frac{(1+0.07)^{10}-1}{0.07} = 
        6908.22
      \] 
    \item Έχουμε $ V_{10} = 50000 $, $ i=0,07 $, $ t=10 $ έτη. Ζητάμε το $ A_{0} $. Άρα
      \[
        V_{t} = A_{0} \frac{(1+i)^{t}-1}{i} \Leftrightarrow 50000 = A_{0}
        \frac{(1+0.07)^{10}-1}{0.07} \Leftrightarrow 50000 = A_{0} \cdot 0.967 
        \Leftrightarrow A_{0} = 3619.44
      \] 
    \item Έχουμε $ V_{10} = 50000 $, $ i=0,07 $, $ t=10 $ έτη, μηνιαία αποταμίευση,
      δηλαδή $ m=12 $ και άρα $n = 12\cdot 10 = 120$ συνολικές περίοδοι ανατοκισμού. 
      Ζητάμε το $ A_{0} $. Άρα
      \[
        V_{t} = A_{0} \frac{(1+i)^{t}-1}{i} \Leftrightarrow 50000 = A_{0}
        \frac{(1+ \frac{0.07}{12})^{120}-1}{\frac{0.07}{12}} \Leftrightarrow 291.66 =
        A_{0} \cdot 1.01 \Leftrightarrow A_{0} \approx 288.77
       \] 
  \end{enumerate}
\end{solution}
\end{example}

\end{document}
