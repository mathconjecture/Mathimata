\input{preamble_ask.tex}
\newcommand{\vect}[2]{(#1_1,\ldots, #1_#2)}
%%%%%%% nesting newcommands $$$$$$$$$$$$$$$$$$$
\newcommand{\function}[1]{\newcommand{\nvec}[2]{#1(##1_1,\ldots, ##1_##2)}}

\newcommand{\linode}[2]{#1_n(x)#2^{(n)}+#1_{n-1}(x)#2^{(n-1)}+\cdots +#1_0(x)#2=g(x)}

\newcommand{\vecoffun}[3]{#1_0(#2),\ldots ,#1_#3(#2)}

\newcommand{\mysum}[1]{\sum_{n=#1}^{\infty}

\input{tikz.tex}
\input{myboxes.tex}



\pagestyle{vangelis}
% \everymath{\displaystyle}
\setcounter{chapter}{1}

\input{insbox}



\begin{document}

\chapter*{Σημειώσεις Θεωρίας}

\section{Μερικές Διαφορικές Εξισώσεις}

\begin{mybox1}
  \begin{dfn}
    Ονομάζουμε \textcolor{Col1}{μερική διαφορική εξίσωση}, για συντομία μδε, κάθε 
    εξίσωση που περιέχει μια άγνωστη συνάρτηση $ u=u(\mathbf{x}) $, της διανυσματικής 
    μεταβλητής 
    $ \mathbf{x} = (x_{1}, x_{2}, \ldots, x_{n}) \in \Omega \subseteq \mathbb{R}^{n} $ 
    με $ n \geq 2 $, και πεπερασμένο πλήθος από μερικές παραγώγους αυτής, δηλαδή 
    \begin{equation}
      \label{eq:pde}
      F\left(\mathbf{x}, u, \pdv{u}{x_{1}} , \pdv{u}{x_{2}}, \ldots, 
        \pdv{u}{x_{n}}, \ldots, \frac{\partial^{(m)}{u}}{\partial x_{1}^{\lambda_ 1} 
      \partial x_{2}^{\lambda_ 2}\ldots \partial x_{n}^{\lambda _{n}}}\right) = 0
    \end{equation} 
    όπου $ \lambda _1+ \lambda _2 + \cdots \lambda _n = m $, με $ \lambda _{1}, \lambda
    _{2} \ldots, \lambda _{n}, m =1,2,\ldots$
  \end{dfn}
\end{mybox1}

\begin{rem}
  Οι μερικές διαφορικές εξισώσεις, με δύο ανεξάρτητες μεταβλητές $ x $ και $ y $, έχουν 
  τη γενική μορφή
  \[
    F\left(x,y,u, \pdv{u}{x} , \pdv{u}{y} , \ldots, 
      \frac{\partial ^{(m)}u}{\partial x^{\lambda _1} \partial
    y^{\lambda _2}} \right) = 0, \; \text{όπου} \; \lambda _{1}+ \lambda _{2}=m 
  \]
\end{rem}

\begin{mybox1}
  \begin{dfn}
    Η \textcolor{Col1}{τάξη} της μερικής διαφορικής εξίσωσης καθορίζεται 
    από την \textbf{μεγαλύτερη} τάξη των μερικών παραγώγων που εμφανίζονται στην εξίσωση.
  \end{dfn}
\end{mybox1}

\begin{rem}
  Συνδυάζοντας τους ορισμούς των μδε και της τάξης τους, συμπεραίνουμε, ότι:
  \begin{myitemize}
    \item Μια μερική διαφορική εξίσωση 1ης τάξης, δύο μεταβλητών, έχει τη γενική μορφή
      \[
        F(x,y,u,u_{x},u_{y}) = 0 \; \text{όπου} \; u=u(x,y)
      \] 
    \item Μια μερική διαφορική εξίσωση 2ης τάξης, δύο μεταβλητών, έχει τη γενική μορφή
      \[
        F(x,y,u,u_{x},u_{y},u_{xx},u_{xy},u_{yy}) = 0 \; \text{όπου} \; u=u(x,y)
      \] 
  \end{myitemize}
\end{rem}

\begin{mybox1}
  \begin{dfn}
    Ονομάζεται \textcolor{Col1}{βαθμός} μιας μερικής διαφορικής εξίσωσης, ο
    \textbf{εκθέτης} στον οποίο είναι υψωμένη η μεγαλύτερης τάξης μερική παράγωγος που 
    περιέχεται στην εξίσωση.
  \end{dfn}
\end{mybox1}

\begin{examples}
\item {}
  \begin{enumerate}
    \item Η μδε $ uu_{x}+2xyu_{y} = 0 $ είναι 1ης τάξης, 1ου βαθμού.
    \item Η μδε $ u_{xx}+xu_{y} = 0 $ είναι 2ης τάξης, 1ου βαθμού.
    \item Η μδε $ (\pdv{u}{x})^{3} + (\pdv[2]{u}{x}{y})^{2} + xy-u=0 $ είναι 2ης τάξης,
      2ου βαθμού.
    \item Η μδε$ \frac{\partial^{3}u}{\partial x^{2}\partial y} + y\pdv[2]{u}{y}=0 $ 
      είναι 3ης τάξης, 1ου βαθμού.
  \end{enumerate}
\end{examples}


\subsection*{Γραμμικές Μερικές Διαφορικές Εξισώσεις}

\begin{mybox1}
  \begin{dfn}
    Μια μερική διαφορική εξίσωση της μορφής~\eqref{eq:pde} ονομάζεται
    \textcolor{Col1}{γραμμική}, όταν η συνάρτηση $F$ είναι γραμμική ως προς τη μεταβλητή 
    $u$ και ως προς όλες τις μερικές παραγώγους της $u$ που εμφανίζονται στην εξίσωση.
  \end{dfn}
\end{mybox1}

\begin{rem}
  Γενικά, σε μια γραμμική διαφορική εξίσωση, δεν εμφανίζονται καθόλου δυνάμεις, οι άλλες 
  συναρτήσεις της άγνωστης συνάρτησης $ u $ και των μερικών παραγώγων της, αλλά ούτε και
  γινόμενα των μερικών παραγώγων μεταξύ τους. Αν μια μδε δεν είναι γραμμική, τότε 
  λέγεται \textcolor{Col1}{μη γραμμική}.
\end{rem}

\begin{rem} \item {}
  Συνδυάζοντας τους ορισμούς των γραμμικών μδε και της τάξης τους, συμπεραίνουμε, ότι:
  \begin{myitemize}
    \item Μια γραμμική μδε 1ης τάξης, δύο μεταβλητών, έχει τη μορφή
      \begin{equation} 
        a(x,y) u_{x} + b(x,y)u_{y} + c(x,y)u = d(x,y), \; \text{όπου} \; u=u(x,y) 
      \end{equation}
      Αν $ d(x,y)=0 $, τότε η εξίσωση λέγεται \textcolor{Col1}{ομογενής}.
    \item Μια γραμμική μδε 1ης τάξης, τριών μεταβλητών, έχει τη μορφή
      \begin{equation} 
        a(x,y,z) u_{x} + b(x,y,z)u_{y} + c(x,y,z)u_{z} + d(x,y,z)u = e(x,y,z), 
        \; \text{όπου} \; u=u(x,y,z)  
      \end{equation} 
      Αν $ e(x,y,z)=0 $, τότε η εξίσωση λέγεται \textcolor{Col1}{ομογενής}.
    \item Μια γραμμική μδε 2ης τάξης, δύο μεταβλητών, έχει τη μορφή
      \begin{equation} 
        a(x,y)u_{xx} + b(x,y)u_{xy} + c(x,y)u_{yy} + d(x,y)u_{x} + e(x,y)u_{y} +
        f(x,y)u =g(x,y), \; \text{όπου} \; u=u(x,y) 
      \end{equation} 
      Αν $ g(x,y)=0 $, τότε η εξίσωση λέγεται \textcolor{Col1}{ομογενής}.
  \end{myitemize}
\end{rem}

\begin{examples} \item {}
  \begin{enumerate}
    \item Η μδε $ yu_{x}-xu_{y}=xy $, όπου $ u=u(x,y) $  είναι γραμμική.
    \item Η μδε $ xu_{x}+yu_{y}+zu_{z}=2u $, όπου $ u=u(x,y,z) $  είναι γραμμική,
      ομογενής.
  \end{enumerate}
\end{examples}


\subsection*{Ημιγραμμικές Μερικές Διαφορικές Εξισώσεις 1ης τάξης}

\begin{mybox1}
  \begin{dfn}
    Μια μερική διαφορική εξίσωση 1ης τάξης, ονομάζεται \textcolor{Col1}{ημιγραμμική}, 
    όταν είναι γραμμική, μόνο ως προς τις μερικές παραγώγους μέγιστης τάξης που 
    εμφανίζονται στην εξίσωση.
  \end{dfn}
\end{mybox1}

\begin{rem}
  Μια ημιγραμμική μδε 1ης τάξης, δύο μεταβλητών, έχει τη μορφή
  \[ a(x,y,u)u_{x} + b(x,y,u)u_{y} = c(x,y,u), \; \text{όπου} \; u=u(x,y) \]
\end{rem}

\begin{examples} \item {}
  \begin{enumerate}
    \item Η μδε $uu_{x}+u_{y}=1$, όπου $ u=u(x,y) $  είναι ημιγραμμική.
    \item Η μδε $ (y-u)u_{x} + (u-x)u_{y} = x-y $, όπου $ u=u(x,y) $  είναι ημιγραμμική.
  \end{enumerate}
\end{examples}


\subsection*{Σχεδόν Γραμμικές Μερικές Διαφορικές Εξισώσεις 1ης τάξης}

\begin{mybox1}
  \begin{dfn}
    Μια μερική διαφορική εξίσωση 1ης τάξης, ονομάζεται 
    \textcolor{Col1}{σχεδόν γραμμική}, όταν οι συντελεστές των μερικών παραγώγων 
    μέγιστης τάξης, εξαρτώνται μόνο από τις ανεξάρτητες μεταβλητές.
  \end{dfn}
\end{mybox1}

\begin{rem}
  Μια σχεδόν γραμμική μδε 1ης τάξης, δύο μεταβλητών, έχει τη μορφή
  \[  a(x,y) u_{x} + b(x,y) u_{y} = c(x,y,u) \; \text{όπου} \; u=u(x,y) \]
\end{rem}

\begin{examples} \item {}
  \begin{enumerate}
    \item Η μδε $yu_{x}+xu_{y}=u^{2}$, όπου $ u=u(x,y) $  είναι σχεδόν γραμμική.
  \end{enumerate}
\end{examples}


\section*{Λύσεις μιας Μερικής Διαφορικής Εξίσωσης}

\begin{mybox1}
  \begin{dfn}
    Ονομάζουμε \textbf{λύση} της μδε~\eqref{eq:pde}, οποιαδήποτε συνάρτηση 
    $ u=u(x_{1}, \ldots, x_{n}) $ που μαζί με τις μερικές παραγώγους της, 
    ικανοποιεί την εξίσωση, σε μια ανοιχτή περιοχή $ \Omega \subseteq \mathbb{R}^{n} $ 
    του πεδίου ορισμού των ανεξάρτητων μεταβλητών  $ x_{1}, \ldots, x_{n} $.
    Η λύση μιας μδε $n$ μεταβλητών, παριστάνει μια \textbf{υπερεπιφάνεια} στο χώρο 
    $ \mathbb{R}^{n+1} $.
  \end{dfn}
\end{mybox1}

\begin{mybox1}
  \begin{dfn}
    Ονομάζουμε \textcolor{Col1}{γενική λύση (ή γενικό ολοκλήρωμα)} μιας μδε τάξης $m$, 
    και $n$ ανεξάρτητων μεταβλητών, μια συνάρτηση $n$ μεταβλητών, η οποία είναι λύση της 
    μδε, και περιέχει $m$ το πλήθος αυθαίρετες συναρτήσεις $ n-1 $ μεταβλητών.
  \end{dfn}
\end{mybox1}

\begin{mybox1}
  \begin{dfn}
    Ονομάζουμε \textcolor{Col1}{μερική ή ειδική λύση (ή μερικό ολοκλήρωμα)} μιας μδε, 
    οποιαδήποτε συνάρτηση προκύπτει από τη γενική της λύση, με τον προσδιορισμό των 
    αυθαίρετων συναρτήσεων που αυτή περιέχει, από ένα ισάριθμο πλήθος αρχικών συνθηκών 
    που επιβάλλονται στη γενική λύση.
  \end{dfn}
\end{mybox1}

\begin{mybox1}
  \begin{dfn}
    Μια λύση ονομάζεται \textcolor{Col1}{ιδιάζουσα λύση (ή ιδιάζων ολοκλήρωμα)} μιας 
    μδε, αν δεν προκύπτει από τη γενική λύση της.
  \end{dfn}
\end{mybox1}

\begin{mybox1}
  \begin{dfn}
    Μια λύση ονομάζεται πλήρης λύση, μιας μδε, όταν περιέχει τόσες αυθαίρετες 
    σταθερές όσες είναι οι ανεξάρτητες μεταβλητές της άγνωστης συνάρτησης.
  \end{dfn}
\end{mybox1}


\section*{Μέθοδοι Επίλυσης μιας ΜΔΕ}

\subsection*{Γενικές Μέθοδοι}

Με τις μεθόδους αυτές επιδιώκεται η άμεση λύση μιας μδε, 
η οποία θα ικανοποιεί τις συνοριακές συνθήκες.

\subsection*{Χωρισμός Μεταβλητών}

Η άγνωστη συνάρτηση εκφράζεται ως γινόμενο συναρτήσεων, 
κάθε μία από τις οποίες εξαρτάται μόνο από μία μεταβλητή και η μδε ανάγεται σε 
σύστημα συνήθων διαφορικών εξισώσεων. Η μέθοδος αυτή εφαρμόζεται σε ομογενείς μδε.


\subsection*{Αλλαγή Μεταβλητών} 

Η μέθοδος αυτή μετασχηματίζει μια μδε, σε μια άλλη μδε, η 
οποία είναι ευκολότερη ή σε μια συνήθη διαφορική εξίσωση. Εφαρμόζεται για γραμμικές
αλλά και μη γραμμικές μδε.

\subsection*{Ολοκληρωτικοί Μετασχηματισμοί} 

Η μέθοδος μετασχηματίζει μια μδε με $n$
ανεξάρτητες μεταβλητές σε μια άλλη με $ (n-1) $ ανεξάρτητες μεταβλητές. Για 
παράδειγμα ο Μετασχηματισμός Fourier.

\subsection*{Ολοκληρωτικές Εξισώσεις} 

Μια μδε μετασχηματίζεται σε μια ολοκληρωτική εξίσωση, 
όπου η άγνωστη συνάρτηση εμφανίζεται μέσα σε ένα ολοκλήρωμα, το οποίο στη συνέχεια 
επιλύεται με διάφορες μεθόδους. Για παράδειγμα Η συνάρτηση Green.

\subsection*{Μέθοδος Ιδιοτιμών} 

Με τη μέθοδο αυτή μια μδε μετασχηματίζεται σε ένα πρόβλημα 
ιδιοτιμών και οι ζητούμενες λύσεις εκφράζονται ως άθροισμα των αντίστοιχων
ιδιοσυναρτήσεων.

\section*{Συνοριακές Συνθήκες}

Σε κάθε φυσικό πρόβλημα, που μοντελοποιείται μέσω μιας μδε, συνήθως ζητείται η επίλυση 
της εξίσωσης σε ένα συγκεκριμένο χωρίο $ \Omega $, φραγμένο ή μη, στο σύνορο $ \partial
\Omega $ του οποίου καθορίζονται συγκεκριμένες συνθήκες που ισχύουν για την 
συνάρτηση $ u $. Ένα τέτοιο πρόβλημα, καλείται \textcolor{Col1}{Πρόβλημα Συνοριακών 
Τιμών} (ΠΣΤ).
Οι συνοριακές συνθήκες, ανήκουν σε έναν από τους παρακάτω τύπους, οι οποίοι για 
λόγους ευκολίας, παρουσιάζονται στο δισδιάστατο χώρο.

\subsection*{Τύπου Dirichlet}

Η λύση $u$ της μδε σε μια δισδιάστατη περιοχή $D$, απαιτείται να ικανοποιεί μια συνθήκη η
οποία καθορίζει την τιμή της συνάρτησης επί του συνόρου της περιοχής $D$, που συνήθως
είναι κάποια καμπύλη $C$
\[
  u= f(x,y), \quad \forall (x,y) \in C \; \text{με} \; C = \partial D
\]
όπου $ f(x,y) $ είναι γνωστή συνάρτηση.

\subsection*{Τύπου Newmann}

Η λύση $u$ της μδε σε μια δισδιάστατη περιοχή $D$, απαιτείται να ικανοποιεί μια συνθήκη η
οποία καθορίζει την τιμή παραγώγων της συνάρτησης επί του συνόρου της περιοχής $D$, που 
συνήθως είναι κάποια καμπύλη $C$
\[
  \pdv{u}{\mathbf{n}} = f(x,y), \quad \forall (x,y) \in C \; \text{με} \; C = \partial D
\] 
όπου $ f(x,y) $ είναι γνωστή συνάρτηση και $ \pdv{u}{\mathbf{n}} $ είναι η παράγωγος
κατά κατεύθυνση της συνάρτησης $ u $ προς τη διεύθυνση του διανύσματος $ \mathbf{n} $, 
που είναι κάθετο στη συνοριακή καμπύλη σε κάθε σημείο της και με προσανατολισμό, προς το
εξωτερικό της περιοχής $D$.

\subsection*{Τύπου Robin}

Η λύση $u$ της μδε σε μια δισδιάστατη περιοχή $D$, απαιτείται να ικανοποιεί δύο συνθήκες
της μορφής
\[
  u= f_{1}(x,y), \quad \forall (x,y) \in C_{1} \quad \text{και} \quad \pdv{u}{\mathbf{n}}
  = f_{2}(x,y), \; \quad \forall (x,y) \in C_{2} 
\]
όπου $ f_{1}(x,y) $ και $ f_{2}(x,y) $ είναι γνωστές συναρτήσεις και $ C_{1} $, $ C_{2}
$ είναι δυο διαφορετικά τμήματα τμήματα της συνοριακής καμπύλης $C$, τέτοια ώστε 
$ C=C_{1}+C_{2} $.

\subsection*{Μεικτού Τύπου}

Η λύση $u$ της μδε σε μια δισδιάστατη περιοχή $D$, απαιτείται να ικανοποιεί μια συνθήκη
της μορφής
\[
  \pdv{u}{\mathbf{n}} + h(x,y)u = f(x,y)
\]
όπου $ h(x,y) $ και $ f(x,y) $ είναι γνωστές \textbf{συνεχείς} συναρτήσεις.
όπου $ f_{1}(x,y) $ και $ f_{2}(x,y) $ είναι γνωστές συναρτήσεις και $ C_{1} $, $ C_{2}
$ είναι δυο διαφορετικά τμήματα τμήματα της συνοριακής καμπύλης $C$, τέτοια ώστε 
$ C=C_{1}+C_{2} $.



\section*{Καλώς Διατυπωμένο Πρόβλημα}

\begin{mybox1}
  \begin{dfn}
    Ένα πρόβλημα είναι \textcolor{Col1}{καλώς διατυπωμένο} αν πληρούνται οι ακόλουθες 
    απαιτήσεις:
    \begin{enumerate}[i)]
      \item \textbf{Υπάρχει} λύση του προβλήματος
      \item Η λύση είναι \textbf{μοναδική}
      \item Η λύση εξαρτάται \textbf{συνεχώς} από τις αρχικές και συνοριακές συνθήκες 
        του προβλήματος
    \end{enumerate}
  \end{dfn}
\end{mybox1}

\begin{rem}
  Η έννοια της συνέχειας συνδέεται με το πρόβλημα της ευστάθειας της λύσης και είναι
  ουσιώδους σημασίας γιατί πρέπει μια μικρή αλλαγή στις αρχικές ή συνοριακές συνθήκες του
  προβλήματος να παράγει μια μικρή μεταβολή της λύσης σε όλη την περιοχή ολοκλήρωσης.
\end{rem}





\end{document}
