\input{preamble/preamble.tex}
\newcommand{\vect}[2]{(#1_1,\ldots, #1_#2)}
%%%%%%% nesting newcommands $$$$$$$$$$$$$$$$$$$
\newcommand{\function}[1]{\newcommand{\nvec}[2]{#1(##1_1,\ldots, ##1_##2)}}

\newcommand{\linode}[2]{#1_n(x)#2^{(n)}+#1_{n-1}(x)#2^{(n-1)}+\cdots +#1_0(x)#2=g(x)}

\newcommand{\vecoffun}[3]{#1_0(#2),\ldots ,#1_#3(#2)}

\newcommand{\suma}{\sum_{n=0}^{\infty}a_n x^n}

\newcommand{\sumb}{\sum_{n=1}^{\infty}a_n n x^{n-1}}

\newcommand{\sumc}{\sum_{n=2}^{\infty}a_n n (n-1) x^{n-2}}

\newcommand{\varsum}[2]{\sum_{n=#1}^{#2}}


\begin{document}

\begin{center}
\minibox[frame,c]{\bfseries\large Ασκήσεις Επανάληψης}
\end{center}

\vspace{\baselineskip}

\begin{enumerate}
    \item Βρείτε τις συντεταγμένες των σημείων όπου η ευθεία $ x-2y=2 $ 
        τέμνει τους άξονες.

    \item Βρείτε το σημείο τομής των δύο ευθειών. 
        \begin{gather*}
            4x+3y=11 \\
            2x+y=5
        \end{gather*}
        \hfill Απ: $ x=2,\; y=1 $ 

    \item Προσδιορίστε ποιες δυο από τις παρακάτω ευθείες είναι παράλληλες.
        \begin{enumerate}[i)]
            \item $ 3x+5y=2 $
            \item $ 5x-3y=1 $
            \item $ 5x+3y=13 $
            \item $ 10x-6y=9 $
            \item $ y=0,6x+2 $
        \end{enumerate}

        \hfill Απ: ii) και iv) 

    \item Ένα αρτοποιείο ανακάλυψε ότι αν μειώσει την τιμή στις τούρτες
        γενεθλίων κατά 1 ευρώ, πουλάει 12 παραπάνω τούρτες τον μήνα.
        \begin{enumerate}[i)]
            \item Υποθέτωντας ότι οι μηνιαίες πωλήσεις, $M$ σχετίζονται
                με τις τιμές, $P$ , με μία γραμμική σχέση της μορφής
                $ M=aP+b $, υπολογίστε την τιμή $ a $.

            \item Αν το αρτοποιείο πουλάει 240 τούρτες τον μήνα, υπολογίστε
                την τιμή του $b$ όταν η τιμή της τούρτας είναι 14 ευρώ.

            \item Εκτιμήστε τις μηνιάιες πωλήσεις όταν η τιμή είναι 9 ευρώ.

            \item Αν το αρτοποιείο μπορεί να φτιάξει μόνο 168 τούρτες
                υπολογίστε την τιμή που χρειάζεται έτσι ώστε να τις πουλήσεις
                όλες.

        \end{enumerate}

        \hfill Απ: $ a=-12,\; b=408,\; M=300,\; P=20 $ 

    \item Μια τράπεζα προσφέρει $ 7\% $ ετήσιο ανατοκισμό. Υπολογίστε 
        την μελλοντική αξία ενός κεφαλαίου 4500 ευρώ μετά από 6 χρόνια. 
        Ποια η συνολική ποσοστιαία αύξηση κατά τη διάρκεια αυτής της 
        περιόδου; (υποδειξη: ποσοστιαία μεταβολή = 
        $ (\frac{x_{\text{τελ.}}}{ x_{\text{αρχ.}}} - 1)\cdot 100\% $)

        \hfill Απ: 6753,29 ευρώ, $ 50\% $ 

    \item Υπολογίστε την μελλοντική αξία που θα έχουν τα 20000 ευρώ σε δύο 
        χρόνια αν τοκίζονται τριμηνιαία με τόκο $ 8\% $.

        \hfill Απ: 23433,19 ευρώ 


    \item Πόσο καιρό θα χρειαστεί ένα χρηματικό ποσό για να διπλασιασθεί αν 
        επενδυθεί με $ 5\% $ ετήσιο ανατοκισμό;

        \hfill Απ: 15 χρόνια 

    \item Υπάρχει υποτίμηση στην αξία ενός μηχανήματος κατά $ 5\% $ κάθε 
        χρόνο. Καθορίστε την αξία του σε 3 χρόνια αν η τρέχουσα αξία του
        είναι 50000 ευρώ.
        (Υπόδειξη: για υποτίμηση παρε $ K=K_{0}(1- \frac{r}{100})^{n} $)

        \hfill Απ: 42868,75 ευρώ 

    \item Ένα κεφάλαιο, 7000 ευρώ, επενδύεται με τόκο $ 9\% $ για 8 
        χρόνια. Καθορίστε την μελλοντική του αξία αν ο ανατοκισμός είναι
        ετήσιος, εξαμηνιαίος, μηνιαίος και συνεχής.ο

        \hfill Απ: $ 13947,94,\; 14156,59,\; 14342,45,\; 14381,03 $

    \item Ποιος από τους ακόλουθους ταμιευτικούς λογαριασμούς προσφέρει 
        καλύτερη απόδοση?
        Λογαριασμός Α με ετήσιο επιτόκιο $ 8,05\% $ και εξαμηνιαίο 
        ανατοκισμό ή Λογαριασμός Β με ετήσιο επιτόκιο $ 7,95\% $ και 
        μηνιαίο ανατοκισμό;

        \hfill Απ: $B$ 

    \item Υπολογίστε την μελλοντική αξία του ποσού των 100 ευρώ με 
        συνεχή ανατοκισμό $ 6\% $ για 12 χρόνια.

        \hfill Απ: 205,44 ευρώ 

    \item Πόσος καιρός χρειάζεται για να τριπλασιαστεί σε αξία ένα 
        κεφάλαιο αν επενδυθεί με ετήσιο επιτόκιο $ 3\% $ και συνεχή
        ανατοκισμό;

        \hfill Απ: 36,6 χρόνια 

    \item Αν υπάρχει συνεχής υποτίμηση της τάξης του $ 4\% $ σε ένα
        μηχάνημα, πόσα χρόνια χρειάζονται για να υποδιπλασιαστεί η αξία του;
        (υποδειξη: για συνεχή υποτίμηση παρε $ K=K_{0}e^{-rt} $)

        \hfill Απ: 17 χρόνια 


\item Η συνάρτηση ζήτησης ενός αγαθού είναι $ P+Q = 30 $ και η συνάρτηση 
    συνολικού κόστους είναι $ TC = \frac{1}{2} Q^{2} + 6Q + 7 $.
    Υπολογίστε το επίπεδο παραγωγής $ Q $ που μεγιστοποιεί τα έσοδα.
    Υπολογίστε το επίπεδο παραγωγής $ Q $ που μεγιστοποιεί το κέρδος.
    (υποδειξη: (Εσοδα) $TR = PQ$, (Κερδος) $\text{Π} = TR-TC$)

    \hfill Απ: $ TR= 30Q-Q^{2},\; Q_{1}^{*}=15, \text{Π} = - \frac{3}{2} Q^{2} + 
    24Q -7,\; Q_{2}^{*}=8 $ 

\end{enumerate}

\end{document}
