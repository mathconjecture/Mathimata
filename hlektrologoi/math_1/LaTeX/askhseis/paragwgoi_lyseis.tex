\documentclass[a4paper,table]{report}
\documentclass[a4paper,12pt]{article}
\usepackage{etex}
%%%%%%%%%%%%%%%%%%%%%%%%%%%%%%%%%%%%%%
% Babel language package
\usepackage[english,greek]{babel}
% Inputenc font encoding
\usepackage[utf8]{inputenc}
%%%%%%%%%%%%%%%%%%%%%%%%%%%%%%%%%%%%%%

%%%%% math packages %%%%%%%%%%%%%%%%%%
\usepackage{amsmath}
\usepackage{amssymb}
\usepackage{amsfonts}
\usepackage{amsthm}
\usepackage{proof}

\usepackage{physics}

%%%%%%% symbols packages %%%%%%%%%%%%%%
\usepackage{bm} %for use \bm instead \boldsymbol in math mode 
\usepackage{dsfont}
\usepackage{stmaryrd}
%%%%%%%%%%%%%%%%%%%%%%%%%%%%%%%%%%%%%%%


%%%%%% graphicx %%%%%%%%%%%%%%%%%%%%%%%
\usepackage{graphicx}
\usepackage{color}
%\usepackage{xypic}
\usepackage[all]{xy}
\usepackage{calc}
\usepackage{booktabs}
\usepackage{minibox}
%%%%%%%%%%%%%%%%%%%%%%%%%%%%%%%%%%%%%%%

\usepackage{enumerate}

\usepackage{fancyhdr}
%%%%% header and footer rule %%%%%%%%%
\setlength{\headheight}{14pt}
\renewcommand{\headrulewidth}{0pt}
\renewcommand{\footrulewidth}{0pt}
\fancypagestyle{plain}{\fancyhf{}
\fancyhead{}
\lfoot{}
\rfoot{\small \thepage}}
\fancypagestyle{vangelis}{\fancyhf{}
\rhead{\small \leftmark}
\lhead{\small }
\lfoot{}
\rfoot{\small \thepage}}
%%%%%%%%%%%%%%%%%%%%%%%%%%%%%%%%%%%%%%%

\usepackage{hyperref}
\usepackage{url}
%%%%%%% hyperref settings %%%%%%%%%%%%
\hypersetup{pdfpagemode=UseOutlines,hidelinks,
bookmarksopen=true,
pdfdisplaydoctitle=true,
pdfstartview=Fit,
unicode=true,
pdfpagelayout=OneColumn,
}
%%%%%%%%%%%%%%%%%%%%%%%%%%%%%%%%%%%%%%

\usepackage[space]{grffile}

\usepackage{geometry}
\geometry{left=25.63mm,right=25.63mm,top=36.25mm,bottom=36.25mm,footskip=24.16mm,headsep=24.16mm}

%\usepackage[explicit]{titlesec}
%%%%%% titlesec settings %%%%%%%%%%%%%
%\titleformat{\chapter}[block]{\LARGE\sc\bfseries}{\thechapter.}{1ex}{#1}
%\titlespacing*{\chapter}{0cm}{0cm}{36pt}[0ex]
%\titleformat{\section}[block]{\Large\bfseries}{\thesection.}{1ex}{#1}
%\titlespacing*{\section}{0cm}{34.56pt}{17.28pt}[0ex]
%\titleformat{\subsection}[block]{\large\bfseries{\thesubsection.}{1ex}{#1}
%\titlespacing*{\subsection}{0pt}{28.80pt}{14.40pt}[0ex]
%%%%%%%%%%%%%%%%%%%%%%%%%%%%%%%%%%%%%%

%%%%%%%%% My Theorems %%%%%%%%%%%%%%%%%%
\newtheorem{thm}{Θεώρημα}[section]
\newtheorem{cor}[thm]{Πόρισμα}
\newtheorem{lem}[thm]{λήμμα}
\theoremstyle{definition}
\newtheorem{dfn}{Ορισμός}[section]
\newtheorem{dfns}[dfn]{Ορισμοί}
\theoremstyle{remark}
\newtheorem{remark}{Παρατήρηση}[section]
\newtheorem{remarks}[remark]{Παρατηρήσεις}
%%%%%%%%%%%%%%%%%%%%%%%%%%%%%%%%%%%%%%%




\newcommand{\vect}[2]{(#1_1,\ldots, #1_#2)}
%%%%%%% nesting newcommands $$$$$$$$$$$$$$$$$$$
\newcommand{\function}[1]{\newcommand{\nvec}[2]{#1(##1_1,\ldots, ##1_##2)}}

\newcommand{\linode}[2]{#1_n(x)#2^{(n)}+#1_{n-1}(x)#2^{(n-1)}+\cdots +#1_0(x)#2=g(x)}

\newcommand{\vecoffun}[3]{#1_0(#2),\ldots ,#1_#3(#2)}

\newcommand{\mysum}[1]{\sum_{n=#1}^{\infty}


\everymath{\displaystyle}
\pagestyle{askhseis}

\begin{document}

\begin{center}
  \minibox{\large \bfseries \textcolor{Col1}{Ασκήσεις στις Παραγώγους}}
\end{center}

\vspace{\baselineskip}

\begin{enumerate}
	\item {\bfseries \boldmath Να βρείτε τα $ a, b \in \mathbb{R} $ έτσι ώστε η ευθεία
		\inlineequation[eq:line]{ y = 2x + 5 } να είναι εφαπτομένη της συνάρτησης 
    $ f(x) = x^{2} + ax + b $ στο σημείο $ x_{0} = -1 $.} 
  \begin{solution}
		Για $ x_{0} = -1 $ από τη σχέση~\eqref{eq:line}  προκύπτει $ y = 2(-1) + 5 =
		3$. Άρα το σημείο είναι το $ (-1,3) $. Το σημείο αυτό επαληθεύει τον
		τύπο της συνάρτησης, αφού ανήκει στη γραφική της παράσταση
		Επομένως $ f(-1) = 3 \Leftrightarrow 1 - a + b = 3
		\Leftrightarrow \inlineequation[eq:line2]{b - a = 2} $. 
		Η παράγωγος της συνάρτησης είναι $ f'(-1) = 2 \Leftrightarrow -2 + a = 2
		\Leftrightarrow a = 4$. Τέλος με αντικατάσταση στη σχέση~\eqref{eq:line2} 
    βρίσκουμε ότι $ b = 2 + 4 = 6 \Leftrightarrow b=6 $.
  \end{solution}


	\item {\bfseries \boldmath Να εξεταστεί πλήρως ως προς τη συνέχεια και την 
      παραγωγισιμότητα η συνάρτηση $ f(x) = e^{\abs{x}} $.}
    \begin{solution}

		Η συνάρτηση γράφεται ως $ f(x) = e^{\abs{x}} = \begin{cases}
				e^{x}, & x\geq 0 \\
				e^{-x}, & x < 0 
			\end{cases}$	
		
		Για τη συνέχεια, έχουμε:

		Αν $ x\neq 0 $ τότε η $f$ είναι συνεχής, ως σύνθεση συνεχών συναρτήσεων.

		Αν $ x = 0 $ τότε εξετάζουμε τα πλευρικά όρια και έχουμε:
		\[
		\left.
		\begin{tabular}{l}
			$\lim_{x\to 0^{+}} f(x) = \lim_{x\to 0^{+}} e^{x} =  e^{0} = 1$ \\ 
			$\lim_{x\to 0^{-}} f(x) = \lim_{x\to 0^{-}} e^{x} = e^{-0} = 1$
		\end{tabular}
		\right\}
		\Rightarrow \lim_{x\to 0} f(x) = \lim_{x\to 0} e^{\abs{x}} = 1 = e^{\abs{0}} = f(0) 
		\]
		Επομένως η συνάρτηση είναι συνεχής στο 0 και άρα $ f(x) $ συνεχής στο $
		\mathbb{R} $.
		Για την παραγωγισιμότητα έχουμε:

		Αν $ x \neq 0 $ τότε η $f$ είναι παραγωγίσιμη ως σύνθεση παραγωγίσιμων
		συναρτήσεων και έχουμε:
		\[
			f'(x) = \begin{cases}
				e^{x}, & x\geq 0 \\
				-e^{-x}, & x < 0
			\end{cases}
		\]
		Αν $ x = 0 $ τότε εξετάζουμε τα πλευρικά όρια και έχουμε:
		\[
			\begin{tabular}{l}
			$f_{+}'(0) = \lim_{x\to 0^{+}} \frac{f(x) - f(0)}{x - 0} = \lim_{x \to
				0^{+}} \frac{e^{x} - 1}{x}
				\overset{(\frac{0}{0})}{\underset{L'H}{=}} \lim_{x\to 0^{+}}
			\frac{e^{x}}{1} = 1 $ \\
			$f_{-}'(0) = \lim_{x\to 0^{-}} \frac{f(x) - f(0)}{x - 0} =
			\lim_{x\to 0^{-}} \frac{e^{-x} - 1}{x} \overset{(\frac{0}{0})}{\underset{L'H}{=}}
			\lim_{x\to 0^{-}}
			\frac{-e^{-x}}{1} = -1 $
			\end{tabular}
		\]
		Επομένως η $f$ δεν είναι παραγωγίσιμη συνάρτηση γιατί δεν είναι
		παραγωγίσιμη στο 0.
    \end{solution}

	\item {\bfseries \boldmath Να βρεθούν τα $ a, b \in \mathbb{R} $ έτσι ώστε η συνάρτηση 
		\[
			f(x) = \begin{cases}
				x^{2}, & x\geq 2 \\
				ax+b , & x<2
			\end{cases}
		\]
	να είναι παραγωγίσιμη στο $ x_{0} = 2 $.}
  \begin{solution}
		Αναγκαία συνθήκη ώστε να είναι η συνάρτηση παραγωγίσιμη στο $ x_{0} = 2
		$ είναι να είναι συνεχής στο $ x_{0} = 2 $. Άρα
		\begin{equation}\label{eq:cont}
			\lim_{x\to 2^{+}} f(x) = \lim_{x\to 2^{-}} f(x) = f(2)
			\Leftrightarrow 2^{2} = 2a + b = 4 \Leftrightarrow 4 = 2a + b
			\Leftrightarrow b = 4 - 2a
		\end{equation}
		Για να είναι παραγωγίσιμη στο $ x_{0} = 2 $ η συνάρτηση θα πρέπει τα
		πλευρικά όρια να ταυτίζονται δηλαδή \inlineequation[eq:eqlim]{f_{+}'(2) =
		f_{-}'(2)} Άρα
		\[
			\begin{tabular}{l}
			$f_{+}'(2) = \lim_{x\to 2^{+}} \frac{f(x) - f(2)}{x - 2} =
			\lim_{x\to 2^{+}} \frac{x^{2} - 4}{x - 2} = \lim_{x\to 2^{+}} (x+2)
			= 4$ \\
			$f_{+}'(2) = \lim_{x\to 2^{-}} \frac{f(x)- f(2)}{x-2} = \lim_{x\to
			2^{-}} \frac{(ax+b) - 4}{x - 2}\overset{\eqref{eq:cont}}{=} \lim_{x\to 2^{-}}
				\frac{ax+4 - 2a - 4}{x - 2}
				\overset{(\frac{0}{0})}{\underset{L'H}{=}} \lim_{x\to 2^{-}} a = a$
			\end{tabular}
		\]
		Επομένως από τη σχέση~\eqref{eq:eqlim}, την απαίτηση ισότητας των
		πλευρικών ορίων έχουμε $ a = 4 $ και με αντικατάσταση στην
		σχέση~\eqref{eq:cont} έχουμε ότι $ b = 4 - 2\cdot 4 = 4 \Leftrightarrow
		b = -4$ 
  \end{solution}

	\item {\bfseries\boldmath Δίνεται η συνεχής συνάρτηση 
      $ f(x) \colon\mathbb{R}\to \mathbb{R} $ τέτοια ώστε 
      \inlineequation[eq:ineq]{x - x^{2} \leq f(x) \leq x + x^{2}}, 
    $ \forall x \in \mathbb{R} $. Να δείξετε ότι η $ f(x) $ έχει παράγωγο στο 
    $ x=0 $ η οποία ισούται με $ f'(0)=1 $.}
  \begin{solution}
		Έχουμε ότι \inlineequation[eq:der]{f'(0) = \lim_{x\to 0} \frac{f(x) -
		f(0)}{x}} 

		Αν $ x=0 $ τότε από τη σχέση~\eqref{eq:ineq} προκύπτει $ 0 \leq f(0) \leq 0
		\Leftrightarrow f(0) = 0$. Με αντικατάσταση της τιμής αυτής στη
		σχέση~\eqref{eq:der} έχουμε ότι \inlineequation[eq:der2]{ f'(0) = \lim_{x\to 0}
		\frac{f(x)}{x}}.
		
		\begin{minipage}{0.45\textwidth}
		Αν $ x>0 $ τότε από σχέση~\eqref{eq:ineq} διαιρώντας με $x$, προκύπτει:
		\begin{gather*}
			1-x \leq \frac{f(x)}{x} \leq 1+x \Leftrightarrow \\ 
			\lim_{x\to 0^{+}}(1-x) \leq \lim_{x\to 0^{+}} \frac{f(x)}{x}
			\leq \lim_{x\to 0^{+}} (1+x)\overset{\eqref{eq:der2}}\Leftrightarrow
			\\
			1 \leq f_{+}'(0) \leq 1 \Leftrightarrow \\
			f_{+}'(0) = 1
		\end{gather*}
		\end{minipage} \hfill 
		\begin{minipage}{0.45\textwidth}
		Αν $ x<0 $ τότε από σχέση~\eqref{eq:ineq} διαιρώντας με $x$, προκύπτει:
			\begin{gather*}
			1-x \geq \frac{f(x)}{x} \geq 1+x \Leftrightarrow \\ 
			\lim_{x\to 0^{-}}(1-x) \geq \lim_{x\to 0^{-}} \frac{f(x)}{x}
			\geq \lim_{x\to 0^{-}} (1+x)\overset{\eqref{eq:der2}}\Leftrightarrow
			\\
			1 \geq f_{-}'(0) \geq 1 \Leftrightarrow \\
			f_{-}'(0) = 1
			\end{gather*}
		\end{minipage} 

		Επομένως από την ισότητα των πλευρικών ορίων έχουμε ότι $ f'(0) = 1 $.
  \end{solution}

	\item {\bfseries \boldmath Έστω η συνάρτηση $ f \colon \mathbb{R} \to \mathbb{R} $ 
      με $ f(0) \neq 0	$ τέτοια ώστε \inlineequation[eq:funcrel]{f(x+y) = f(x) \cdot
		f(y)}, $ \forall x,y \in \mathbb{R} $. Αν η $f$ είναι παραγωγίσιμη στο $0$, τότε να 
  δείξετε ότι είναι παραγωγίσιμη $ \forall x \in \mathbb{R} $ με παράγωγο 
  $ f'(x) = f'(0) \cdot f(x) $.}
  \begin{solution}
		Από τη σχέση~\eqref{eq:funcrel} με αντικατάσταση $ x=y=0 $ προκύπτει
		
		\begin{equation}\label{eq:funcrel2}
			f(0) = f^{2}(0) \Leftrightarrow f(0)(f(0) - 1) = 0 \overset{f(0)\neq
			0}{\Leftrightarrow} f(0) = 1	
		\end{equation}

		Από υπόθεση η $f$ είναι παραγωγίσιμη στο 0, οπότε

		\begin{equation}\label{eq:funcrel3}
			f'(0) = \lim_{x\to 0} \frac{f(x) - f(0)}{x - 0}
			\overset{\eqref{eq:funcrel2}}{=} \lim_{x\to 0}
			\frac{f(x) - 1}{x} 	
		\end{equation}

		Έστω $ x_{0} \in \mathbb{R} $ (τυχαίο). Τότε
		\[
			f'(x_{0}) = \lim_{h\to 0} \frac{f(x_{0} + h) -
			f(x_{0})}{h}	\overset{\eqref{eq:funcrel}}{=} \lim_{h\to 0} \frac{f(x_{0})\cdot
		(f(h) - 1)}{h} \overset{\eqref{eq:funcrel3}}{=} f(x_{0})\cdot f'(0)
		\]
		Επομένως $ f'(x) = f(x)\cdot f'(0)$, $ \forall x \in \mathbb{R} $.
  \end{solution}

	\item {\bfseries \boldmath Αν ο πραγματικός αριθμός $p$ είναι η ρίζα ενός πολυωνύμου 
      $ P(x) $ και της παραγώγου του $ P'(x) $ τότε να δείξετε ότι το $p$ είναι διπλή
	ρίζα του $ P(x) $ και αντίστροφα.}
  \begin{solution}
		Έστω $p$ ρίζα του $ P(x) $ και του $ P'(x) $. Άρα
		\[
			\left.
				\begin{tabular}{l}
			$P(x) = (x-p)f(x)$ \\
			$P'(x) = (x-p)g(x)$
				\end{tabular}
			\right\}\Rightarrow 
			\left.
				\begin{tabular}{l}
			$P'(x) = f(x) + (x-p)f'(x)$ \\
			$P'(x) = (x-p)g(x)$
				\end{tabular}
			\right\}\Rightarrow
			f(x) + (x-p)f'(x) = (x-p)g(x) 
			\]
			Άρα \inlineequation[eq:pol]{f(x) = (x-p)[g(x) - f'(x)]} 

			Οπότε $ P(x) = (x-p)f(x) \overset{\eqref{eq:pol}}{=} (x-p)^{2}(g(x)
			- f'(x)) $, από όπου προκύπτει το ζητούμενο.

			Αντίστροφα, τώρα, έστω ότι το $p$ είναι διπλή ρίζα του $ P(x) $. Άρα 

			$ P(x) = (x-p)^{2}\phi(x) $

			Παραγωγίζοντας έχουμε

			$ P'(x) = 2(x-p)\phi(x) + (x-p)^{2}\phi'(x) \Rightarrow P'(x) =
			(x-p)[2\phi(x) + (x-p)\phi'(x)] \Rightarrow P'(p) = 0 $.
  \end{solution}

		\item {\bfseries \boldmath Να δείξετε ότι για το πολυώνυμο $n$ βαθμού 
        $ P_{n}(x) =a_{n}x^{n} + a_{n-1}x^{n-1} + \cdots + a_{1}x + a_{0} $, 
        με  $ a_{n}\neq 0 $ ότι ισχύει $ P_{n}^{(n)}(x) = n! \cdot a_{n}$, 
        για  $n\geq 1 $. Στη συνέχεια να δείξετε ότι το πολυώνυμο $ P_{n}(x) $ 
        μπορεί να γραφεί στη μορφή $ P_{n}(x) = P(0) + \frac{P'(0)}{1!} x + 
        \frac{P''(0)}{2!} x^{2} + \cdots + \frac{P^{n}(0)}{n!} x^{n}$} 
  \begin{solution}
		Με μαθηματική επαγωγή.

		Για $ n=1 $ έχουμε: $ P_{1}(x) = a_{1}x + a_{0} \Rightarrow P_{1}'(x) =
		a_{1} \Rightarrow P_{1}^{(1)}(x)
		= a_{1} = 1!a_{1} $ 

		Έστω ότι η πρόταση ισχύει για $ n = k  $, δηλαδή

		\begin{equation}\label{eq:polyon}
			P_{k}(x) = a_{k}x^{k} + a_{k-1}x^{k-1}+ \cdots +a_{1}x + a_{0}
			\Rightarrow P_{k}^{(k)}(x) = k!\cdot a_{k}
		\end{equation}

		Θα δείξουμε ότι ισχύει για $ n = k+1 $.

		Πράγματι

		\begin{align*}
			P_{k+1}^{(k+1)}(x) = [P_{k+1}'(x)]^{(k)} 
			&= [(k+1)a_{k+1}x^{k} +	ka_{k}x^{k-1} + \cdots + 2a_{2}x + a_{1}]^{(k)} \\
			&\overset{\eqref{eq:polyon}}{=} k! (k+1) a_{k+1} = (k+1)!a_{k+1}
		\end{align*}	

		Έχουμε δείξει ότι $ P_{n}^{(n)}(x) = n! a_{n} $

		Για $ x = 0 $ έχουμε:

		\begin{align*}
			n &= 0 \Rightarrow P(0) = 0!a_{0} \Rightarrow a_{0} = P(0) \\
			n &= 1 \Rightarrow P'(0) = 1!a_{1} \Rightarrow a_{1} = \frac{P'(0)}{1!} \\ 
			n &= 2 \Rightarrow P''(0) = 2!a_{2} \Rightarrow a_{2} = \frac{P''(0)}{2!} \\
			  &\vdots \\
			n &= n \Rightarrow P^{n}(0) = n!a_{n} \Rightarrow a_{n} = \frac{P^{n}(0)}{n!} \\
		\end{align*}	

		Επομένως 
		\[
			P_{n}(x) = P(0) + \frac{P'(0)}{1!} x + \frac{P''(0)}{2!} x^{2} + \cdots
			+ \frac{P^{n}(0)}{n!} x^{n}
		\]
  \end{solution}
\end{enumerate}
    


\end{document}
