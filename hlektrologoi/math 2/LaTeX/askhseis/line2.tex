\documentclass[a4paper,12pt]{article}
\usepackage{etex}
%%%%%%%%%%%%%%%%%%%%%%%%%%%%%%%%%%%%%%
% Babel language package
\usepackage[english,greek]{babel}
% Inputenc font encoding
\usepackage[utf8]{inputenc}
%%%%%%%%%%%%%%%%%%%%%%%%%%%%%%%%%%%%%%

%%%%% math packages %%%%%%%%%%%%%%%%%%
\usepackage{amsmath}
\usepackage{amssymb}
\usepackage{amsfonts}
\usepackage{amsthm}
\usepackage{proof}

\usepackage{physics}

%%%%%%% symbols packages %%%%%%%%%%%%%%
\usepackage{bm} %for use \bm instead \boldsymbol in math mode 
\usepackage{dsfont}
\usepackage{stmaryrd}
%%%%%%%%%%%%%%%%%%%%%%%%%%%%%%%%%%%%%%%


%%%%%% graphicx %%%%%%%%%%%%%%%%%%%%%%%
\usepackage{graphicx}
\usepackage{color}
%\usepackage{xypic}
\usepackage[all]{xy}
\usepackage{calc}
\usepackage{booktabs}
\usepackage{minibox}
%%%%%%%%%%%%%%%%%%%%%%%%%%%%%%%%%%%%%%%

\usepackage{enumerate}

\usepackage{fancyhdr}
%%%%% header and footer rule %%%%%%%%%
\setlength{\headheight}{14pt}
\renewcommand{\headrulewidth}{0pt}
\renewcommand{\footrulewidth}{0pt}
\fancypagestyle{plain}{\fancyhf{}
\fancyhead{}
\lfoot{}
\rfoot{\small \thepage}}
\fancypagestyle{vangelis}{\fancyhf{}
\rhead{\small \leftmark}
\lhead{\small }
\lfoot{}
\rfoot{\small \thepage}}
%%%%%%%%%%%%%%%%%%%%%%%%%%%%%%%%%%%%%%%

\usepackage{hyperref}
\usepackage{url}
%%%%%%% hyperref settings %%%%%%%%%%%%
\hypersetup{pdfpagemode=UseOutlines,hidelinks,
bookmarksopen=true,
pdfdisplaydoctitle=true,
pdfstartview=Fit,
unicode=true,
pdfpagelayout=OneColumn,
}
%%%%%%%%%%%%%%%%%%%%%%%%%%%%%%%%%%%%%%

\usepackage[space]{grffile}

\usepackage{geometry}
\geometry{left=25.63mm,right=25.63mm,top=36.25mm,bottom=36.25mm,footskip=24.16mm,headsep=24.16mm}

%\usepackage[explicit]{titlesec}
%%%%%% titlesec settings %%%%%%%%%%%%%
%\titleformat{\chapter}[block]{\LARGE\sc\bfseries}{\thechapter.}{1ex}{#1}
%\titlespacing*{\chapter}{0cm}{0cm}{36pt}[0ex]
%\titleformat{\section}[block]{\Large\bfseries}{\thesection.}{1ex}{#1}
%\titlespacing*{\section}{0cm}{34.56pt}{17.28pt}[0ex]
%\titleformat{\subsection}[block]{\large\bfseries{\thesubsection.}{1ex}{#1}
%\titlespacing*{\subsection}{0pt}{28.80pt}{14.40pt}[0ex]
%%%%%%%%%%%%%%%%%%%%%%%%%%%%%%%%%%%%%%

%%%%%%%%% My Theorems %%%%%%%%%%%%%%%%%%
\newtheorem{thm}{Θεώρημα}[section]
\newtheorem{cor}[thm]{Πόρισμα}
\newtheorem{lem}[thm]{λήμμα}
\theoremstyle{definition}
\newtheorem{dfn}{Ορισμός}[section]
\newtheorem{dfns}[dfn]{Ορισμοί}
\theoremstyle{remark}
\newtheorem{remark}{Παρατήρηση}[section]
\newtheorem{remarks}[remark]{Παρατηρήσεις}
%%%%%%%%%%%%%%%%%%%%%%%%%%%%%%%%%%%%%%%




\newcommand{\vect}[2]{(#1_1,\ldots, #1_#2)}
%%%%%%% nesting newcommands $$$$$$$$$$$$$$$$$$$
\newcommand{\function}[1]{\newcommand{\nvec}[2]{#1(##1_1,\ldots, ##1_##2)}}

\newcommand{\linode}[2]{#1_n(x)#2^{(n)}+#1_{n-1}(x)#2^{(n-1)}+\cdots +#1_0(x)#2=g(x)}

\newcommand{\vecoffun}[3]{#1_0(#2),\ldots ,#1_#3(#2)}

\newcommand{\mysum}[1]{\sum_{n=#1}^{\infty}



\everymath{\displaystyle}
\thispagestyle{empty}

\begin{document}

\begin{center}
    \fbox{\large\bfseries  Ασκήσεις στο Επικαμπύλιο Ολοκλήρωμα 2ου είδους}
\end{center}


\vspace{\baselineskip}


\begin{enumerate}
    \item Να υπολογιστούν τα παρακάτω επικαμπύλια ολοκληρώματα.

        \begin{enumerate}[i)]
            \item $ \int_{c} (x^{2}-y)dx + (y^{2}-x)dy $, \quad όπου $ c:
                $ το ευθύγραμμο τμήμα με άκρα $ A(0,1) $ και $ B(1,2) $ \hfill Απ: $
                \frac{5}{3}$ 

            \item $ \int_{c} (3x^{2}-6yz)dx+(2y+3xz)dy+(1-4xyz^{2})dz $, \quad $ c:\;
                \mathbf{r}(t)=t \mathbf{i}+t^{2} \mathbf{j}+ t^{3} \mathbf{k} $, $ t \in [0,1]$
                \hfill Απ: $ 2 $  

            \item $ \int_{c} 2xy dx + x^{2} dy $, \quad όπου $ c: y=x^{2} $, με άκρα $ A(0,0) $ και
                $ B(1,1) $ \hfill Απ: $ 1 $ 

            \item $ \int_{c} 3x^{2}y dx + (x^{3}+1)dy $, \quad όπου $ c: x=1 $ με άκρα $ A(1,0)
                $ και $ B(1,1) $ \hfill Απ: $ 2 $ 

            \item $ \int_{c} x^{2}dx+xy dy $, \quad όπου $ c:\; \mathbf{r}(t) = \cos{t} \mathbf{i}
                + \sin{t} \mathbf{j} $, στο 1ο τεταρτημόριο. \hfill Απ: $ 0 $

            \item $ \int_{c} y^{2}dx - x^{2}dy $, \quad όπου $ c: x^{2}+y^{2}=1 $ \hfill Απ: $ 0 $ 

            \item $ \oint_{c} y^{2}dx + x^{2}dy $, \quad όπου $ c: $  περίμετρος τριγώνου  με 
                κορυφές $ O(0,0) $, $ A(1,0) $ και $ B(1,1) $ \hfill Απ: $ \frac{1}{3} $ 

            \item $ \int_{c} yzdx+xzdy+xydz  $, \quad όπου $ c: \mathbf{r}(t) = t \mathbf{i}+
                \sin{t} \mathbf{j} + \cos{t} \mathbf{k}$, $ t \in [0, 2 \pi] $ \hfill Απ: $ 0 $ 

        \end{enumerate}

    \item Να υπολογιστεί το επικαμπύλιο ολοκλήρωμα 
        \[ 
            \oint_{c} (x+y)dx + (2x-z)dy + (y+z)dz 
\]
όπου $ c: $ περίμετρος τριγώνου που σχηματίζουν τα σημεία τομής των επιφανειών $ 3x+2y+z=6 $ 

\hfill Απ: $ 21 $ 


\item Να υπολογιστεί το έργο που παράγεται όταν η δύναμη $ \mathbf{F}(x,y,z) = xy \mathbf{i} +
    y^{2}z \mathbf{j} + z^{3} \mathbf{k} $ μετατοπίζει το σημείο εφαρμογής της κατά μήκος του τόξου
    $ AB $ της καμπύλης $ \mathbf{r}(t) = \cos{t} \mathbf{i}+ \sin{t} \mathbf{j} + 2t \mathbf{k} $,
    $ t \in [0,  \pi] $

    \hfill Απ: $ 4 \pi ^{4} $ 

\item Να υπολογιστεί το εμβαδό του επίπεδου χωρίου που περικλείεται από την αστεροειδή καμπύλη $
    \mathbf{r}(t) = a \cos^{3}{t} \mathbf{i} + a \sin^{3}{t} \mathbf{j} $, $ t \in [0, 2 \pi] $

    \hfill Απ: $ \frac{3 \pi a^{2}}{8} $ 

\item Να υπολογιστεί το εμβαδό του επίπεδου χωρίου που περικλείεται από την έλλειψη $
    \frac{x^{2}}{a^{2}} + \frac{y^{2}}{b^{2}} =1 $

    \hfill Απ: $ \pi a b $

\end{enumerate}



\end{document}


