\documentclass[a4paper,table]{report}
\input{preamble_themata}
\newcommand{\vect}[2]{(#1_1,\ldots, #1_#2)}
%%%%%%% nesting newcommands $$$$$$$$$$$$$$$$$$$
\newcommand{\function}[1]{\newcommand{\nvec}[2]{#1(##1_1,\ldots, ##1_##2)}}

\newcommand{\linode}[2]{#1_n(x)#2^{(n)}+#1_{n-1}(x)#2^{(n-1)}+\cdots +#1_0(x)#2=g(x)}

\newcommand{\vecoffun}[3]{#1_0(#2),\ldots ,#1_#3(#2)}

\newcommand{\suma}{\sum_{n=0}^{\infty}a_n x^n}

\newcommand{\sumb}{\sum_{n=1}^{\infty}a_n n x^{n-1}}

\newcommand{\sumc}{\sum_{n=2}^{\infty}a_n n (n-1) x^{n-2}}

\newcommand{\varsum}[2]{\sum_{n=#1}^{#2}}
\input{myboxes}

\everymath{\displaystyle}
\pagestyle{vangelis}


\begin{document}

\begin{mybox3}
  \begin{thema}
    Δίνεται η ακόλουθη συνάρτηση: 
    \[
      f(x,y) = Ax^{2}+2Bxy+Cy^2+2Dx+2Ey+F 
    \] 
    όπου $ A>0 $ και $ B^{2}-AC <0 $. Να βρεθούν ακρότατα.
  \end{thema}
\end{mybox3}
\begin{solution}
  \item {}
  \begin{enumerate}
    \item Βρίσκουμε τις μερικές παραγώγους 1ης και 2ης τάξης, της συνάρτησης.

      \vspace{0.5\baselineskip}
      \twocolumnsidesl{
        \begin{myitemize}
          \item $ f_{x} = 2Ax+2By+2D $
          \item $ f_{y} = 2Bx+2Cy+2E $
        \end{myitemize}
      }{
        \begin{myitemize}
          \item $ f_{xx} = 2A $
          \item $ f_{yy} = 2C $
          \item $ f_{xy} = f_{yx} = 2B $
        \end{myitemize}
      }
    \item Βρίσκουμε τα κρίσιμα σημεία της συνάρτησης.

      Επειδή η συνάρτηση $ f(x,y) $ είναι συνεχής και διαφορίσιμη σε κάθε σημείο 
      του $ \mathbb{R}^{2} $, αναζητούμε τα κρίσιμα σημεία της στις λύσεις του
      συστήματος:
      \[
        \left.
          \begin{matrix}
            f_{x}=0 \\
            f_{y}=0
          \end{matrix} 
        \right\} \Leftrightarrow 
        \left.
          \begin{matrix}
            Ax+By=-D \\
            Bx+Cy=-E
          \end{matrix} 
        \right\} \Leftrightarrow 
        \begin{pmatrix*}[r] A & B \\ B & C \end{pmatrix*} \cdot 
        \begin{pmatrix*}[r] x \\ y \end{pmatrix*} = 
        \begin{pmatrix*}[r] -D \\ -E \end{pmatrix*}
      \] 
      με ορίζουσα 
      \[
        \begin{vmatrix*}[r]
          A & B \\
          B & C
        \end{vmatrix*} = AC-B^{2}>0
      \] 
      Άρα το σύστημα έχει μοναδική λύση. Δηλαδή η συνάρτηση $ f(x,y) $ έχει μοναδικό 
      κρίσιμο σημείο, έστω $ (x_{0}, y_{0}) $.

    \item Υπολογίζουμε τις Εσσιανές ορίζουσες στο σημείο $ (x_{0}, y_{0}) $ και έχουμε:
      \[
        \abs{H_{2}} = 
        \begin{vmatrix*}[r]
          f_{xx} & f_{xy} \\
          f_{yx} & f_{yy}
        \end{vmatrix*}_{(x_{0}, y_{0})} = 
        \begin{vmatrix*}[r]
          2A & 2B \\
          2B & 2C 
        \end{vmatrix*} = 
        4AC-4B^{2} = 4(AC-B^{2}) > 0
      \]
      \[
        \abs{H_{1}} = f_{xx}(x_{0}, y_{0}) = 2A >0
      \] 
      Άρα στο σημείο $ (x_{0}, y_{0}) $ η συνάρτηση $ f $ παρουσιάζει τοπικό
      \textbf{ελάχιστο}.



  \end{enumerate}
\end{solution}




\end{document}
