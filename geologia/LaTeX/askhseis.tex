\documentclass[a4paper,12pt]{article}


\usepackage[english,greek]{babel}
\usepackage[utf8]{inputenc}
\usepackage{amsmath}
\usepackage{amssymb}
\usepackage{amsfonts}
\usepackage{enumerate}

\usepackage{physics}
\usepackage{graphicx}
\usepackage{geometry}

\geometry{left=25.63mm,right=25.63mm,top=36.25mm,bottom=36.25mm,footskip=24.16mm,headsep=24.16mm}








\title{\textbf{Μαθηματικά ΙΙ} \\ Τμήμα Γεωλογίας}
\author{Ασκήσεις Επανάληψης}


\begin{document}
\maketitle


\begin{enumerate}

\item Δίνονται τα σημεία $Α(1,2,0), Β(0,3,1)$ και $C(2,3,1)$. Ζητούνται:
\begin{enumerate}[i)]
\item Να εξετασθεί αν αποτελούν κορυφές τριγώνου. Αν ναι:
\item Να βρεθούν τα μέσα των πλευρών του. 
\item Να βρεθούν οι γωνίες του τριγώνου και τα μήκη των πλευρών.
\end{enumerate}

\item Δίνονται τα διανύσματα: $\vec{a}=(2,1,3), \vec{b}=(0,-1,4)$. Ζητούνται:
\begin{enumerate}[i)]
\item Να εξετασθεί αν είναι συγγραμμικά. Αν όχι:
\item Να βρεθεί ένα διάνυσμα $\vec{c}$ κάθετο στα $\vec{a}, \vec{b}$.
\item Να βρεθεί ένα μοναδιαίο διάνυσμα κάθετο στα $\vec{a}, \vec{b}$.
\item Να εξετασθεί αν τα διανύσματα $\vec{a}, \vec{b}$ είναι μεταξύ τους κάθετα. Αν όχι, να βρεθεί η μεταξύ τους γωνία.
\end{enumerate}

\item Δίνονται τα επίπεδα: $(\Pi_1): x-2y+3z=1$ και $(\Pi_2): 2x+2y-z=0$. Να εξετασθεί αν τέμνονται και αν ναι να βρεθεί η εξίσωση της τομής σε διανυσματική παραμετρική μορφή.

\item Δίνεται το επίπεδο $(\Pi_1): 2x+y+3z=6$ και η ευθεία $(\varepsilon): x=y-1=\frac{z}{2}$. Αφού δειχθεί ότι η ευθεία τέμνει το επίπεδο, να βρεθεί η εξίσωση της ευθείας $(\varepsilon')$ που βρίσκεται πάνω στο επίπεδο και είναι κάθετη στην $(\varepsilon)$ καθώς και η εξίσωση του επιπέδου $(\Pi_1)$ που είναι κάθετο στο $(\Pi)$ και περιέχει την $(\varepsilon)$.

\item Δίνεται ευθεία $(\varepsilon)$: $x+1=\frac{y}{2}=\frac{z-1}{4}$ και το σημεία $A(1,2,3)$.
\begin{enumerate}[i)]
\item Να βρεθεί η προβολή του σημείου $A$ πάνω στην ευθεία $(\varepsilon)$.
\item Η απόσταση του σημείου $A$ από την ευθεία $(\varepsilon)$.
\item Η εξίσωση της ευθείας που περνά από το $A$ και έχει διεύθυνση παράλληλη στην $(\varepsilon)$.
\item Η εξίσωση του επιπέδου που ορίζει το σημείο $A$ και η ευθεία $(\varepsilon)$.
\item Η εξίσωση της ευθείας που περνά από το $A(1,2,3)$ και έχει διεύθυνση κάθετη στο επίπεδο $(\Pi)$.
\end{enumerate}

\item Δίνονται οι ευθείες $(\varepsilon_1):\frac{x-1}{2}=\frac{y-2}{3}=\frac{z-3}{4}$ και $(\varepsilon_2):\frac{x}{2}=y=\frac{z-1}{2}$.
\begin{enumerate}[i)]
\item Να δειχθεί ότι είναι ασύμβατες (ούτε τεμνόμενες, ούτε παράλληλες).
\item Να βρεθεί η εξίσωση του επιπέδου που περιέχει την $(\varepsilon_2)$ και είναι παράλληλο στην $(\varepsilon_1)$.
\item Η απόσταση των ασύμβατων ευθειών $(\varepsilon_1)$ και $(\varepsilon_2)$. 
\item Η εξίσωση της ευθείας $(\varepsilon)$ που περνά από το σημείο $A(1,2,0)$ και τέμνει τις $(\varepsilon_1)$ και $(\varepsilon_2)$. 
\end{enumerate}

\item Να βρεθούν οι τιμές των οριζουσών:

\begin{enumerate}[i)]
\item 
\(
D_1=\begin{vmatrix}
1 & -1 & 2 \\
2 & -3 & -1 \\
2 & 4 & 3
\end{vmatrix}\
\)  \hfill Απ: 31

\item 
\(
D_2=\begin{vmatrix}
-2 & 1 & 2 \\
3 & 5 & -1 \\
-3 & 2 & 1
\end{vmatrix}
\)  \hfill Απ: 28

\item 
\(
D_3=\begin{vmatrix}
1 & 2 & 0 & 1 \\
2 & -1 & 1 & 2 \\
6 & 3 & 0 & -2 \\
-1 & 2 & 0 & 2
\end{vmatrix}\
\) \hfill Απ: -5
\end{enumerate}

\item Να λυθεί το γραμμικό σύστημα:

\begin{enumerate}[i)]
\item 
\begin{align*}
x-2y+3z&=2\\
2x-3z&=3 \\
x+y+z&=6 
\end{align*} Απ: $x=3, y=2, z=1$


\item 
\begin{align*}
x+y+3z&=0\\
-x-2y+z&=0 \\
3x+2y+13z&=0 
\end{align*}

 Απ: $x=-7z, y=4z, z \in \mathbb{R}$
\end{enumerate}

\item Να βρεθεί ο βαθμός των πινάκων:
\begin{enumerate}[i)]
\item 
\(A=\begin{pmatrix}
3 & 1\\
-2 & 4
\end{pmatrix}
\)\hfill $\rank(A)=2$
\item 
\(
B=\begin{pmatrix}
1 & 2 & -2 \\
-3 & 2 & -1 \\
-2 & 1 & 1
\end{pmatrix}
\)\hfill $\rank(B)=3$
\item 
\(
C=\begin{pmatrix}
3 & 1 & -2 \\
4 & 1 & -3 \\
-2 & 1 & 3
\end{pmatrix}
\)\hfill $\rank(C)=2$
\end{enumerate}

\item Να βρεθεί η τιμή του $x$ ώστε ο βαθμός του παρακάτω πίνακα να είναι 3.

\[
Α=\begin{pmatrix}
2 & -1 & -2 & 0 \\
-1 & 0 & 2 & 1 \\
0 & 4 & x & 2 \\
1 & 0 & 1 & 1 
\end{pmatrix}
\]\hfill Απ: $x=7$

\item Να βρεθούν το $\lambda$ ώστε το σύστημα να έχει μία, καμία ή άπειρες λύσεις. Σε κάθε περίπτωση να βρίσκετε τη λύση.


\begin{align*}
x+y+\lambda z&=1\\
x+\lambda y+z&=\lambda \\
\lambda x+y+z&=\lambda^2 
\end{align*}

Απ: Για $\lambda\neq 1, \lambda\neq -2$ μία λύση, για $\lambda=1$, άπειρες λύσεις, και για $\lambda=-2$, αδύνατο.
\end{enumerate}


\end{document}