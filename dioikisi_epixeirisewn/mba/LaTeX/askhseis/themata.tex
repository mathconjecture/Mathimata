\input{preamble_ask.tex}
\input{definitions_ask.tex}

\everymath{\displaystyle}
\pagestyle{askhseis}



\begin{document}

\begin{center}
  \minibox{\large\bfseries \textcolor{Col1}{Θέματα}}
\end{center}

\vspace{\baselineskip}

\begin{enumerate}
  \item Έστω το γραμμικό υπόδειγμα ενός αγαθού σε μια μεμονωμένη αγορά της μορφής
    \[
      \left.
        \begin{matrix*}[l]
          Q_{d}= \alpha - \beta P \\
          Q_{s}=- \gamma + \delta P
        \end{matrix*} 
      \right\}
    \] 
    Ποια από τις παρακάτω εκφράσεις είναι σωστή;
    \begin{enumerate}[i)]
      \item Η μείωση του $\alpha$ προκαλεί μείωση της τιμής ισορροπίας επειδή $\beta$,
        $\delta > 0$ 
      \item Η μείωση του $\alpha$ προκαλεί αύξηση της τιμής ισορροπίας επειδή $\beta$,
        $\delta > 0$
    \end{enumerate}

  \item Έστω το γραμμικό υπόδειγμα ενός αγαθού σε μια μεμονωμένη αγορά της μορφής
    \[
      \left.
        \begin{matrix*}[l]
          Q_{d}= \alpha - \beta P \\
          Q_{s}=- \gamma + \delta P
        \end{matrix*} 
      \right\}
    \]
    Ποια από τις παρακάτω εκφράσεις είναι σωστή;
    \begin{enumerate}[i)]
      \item Η αύξηση  του $\beta$ προκαλεί αύξηση της τιμής ισορροπίας επειδή $\alpha$,
        $\gamma > 0$ 
      \item Η αύξηση  του $\beta$ προκαλεί μείωση της τιμής ισορροπίας επειδή $\alpha$,
        $\gamma > 0$
    \end{enumerate}

  \item Έστω η συνάρτηση $ f(x) = (x-3)^{3} $. Στο σημείο $ x_{0} = 3 $ η συνάρτηση 
    παρουσιάζει:

    \begin{enumerate*}[i)]
      \item τοπικό μέγιστο \quad
      \item τοπικό ελάχιστο \quad \item σημείο καμπής
    \end{enumerate*}

  \item Έστω η συνάρτηση $ f(x) = x^{3} - 9x^{2}+27x-27 $. Στο σημείο $ x_{0} = 3 $ η 
    συνάρτηση παρουσιάζει:

    \begin{enumerate*}[i)]
      \item τοπικό μέγιστο \quad
      \item τοπικό ελάχιστο \quad \item σημείο καμπής
    \end{enumerate*}

  \item Έστω η συνάρτηση $ f(x) = x^{3}+3px+5 $. Ποια από τις παρακάτω προτάσεις είναι 
    λάθος;

    \begin{enumerate}[i)]
      \item Αν $ p>0 $ η συνάρτηση δεν παρουσιάζει τοπικά ακρότατα
      \item Αν $ p>0 $ η συνάρτηση παρουσιάζει τοπικά ακρότατα
    \end{enumerate}

  \item Έστω η συνάρτηση $ f(x) = x^{n}-nx+k $, όπου $ n>0 $ και $ n \neq 1 $. 
    Ποια από τις παρακάτω προτάσεις είναι σωστή;

    \begin{enumerate}[i)]
      \item Αν $ n>1 $ τότε η συνάρτηση έχει τοπικό ελάχιστο στο 1
      \item Αν $ n>1 $ τότε η συνάρτηση έχει τοπικό ελάχιστο στο 1
    \end{enumerate}
\end{enumerate}



  \end{document}
