\chapter{Μερικές Παράγωγοι}

\section{Ορισμός}

\subsection*{Δύο Μεταβλητών}

Έστω $ f \colon A \subseteq \mathbb{R}^{2} \to \mathbb{R} $
και $ (x_{0}, y_{0}) \in A $. Τότε η \textcolor{Col1}{μερική παράγωγος της $f$ ως 
προς $x$} στο σημείο $ (x_{0}, y_{0}) $ είναι:
\begin{align*}
  \eval{\pdv{f}{x}}_{(x_{0}, y_{0})} = \lim_{x \to x_{0}} 
  \frac{f(x, y_{0}) - f(x_{0}, y_{0})}{x - x_{0}} \overset{h=x- x_{0}}{=} 
  \lim_{h \to 0} \frac{f(x_{0}+h, y_{0}) - f(x_{0}, y_{0})}{h}  
  \intertext{και η \textcolor{Col1}{μερική παράγωγος της $f$ ως προς $y$} στο σημείο 
  $ (x_{0}, y_{0}) $ είναι:}
  \eval{\pdv{f}{y}}_{(x_{0}, y_{0})} = \lim_{y \to y_{0}} 
  \frac{f(x_{0}, y) - f(x_{0}, y_{0})}{y - y_{0}} \overset{k=y- y_{0}}{=} 
  \lim_{k \to 0} \frac{f(x_{0}, y_{0}+k) - f(x_{0}, y_{0})}{k}  
\end{align*}

\subsection*{Τριών Μεταβλητών}
Έστω $ f \colon A \subseteq \mathbb{R}^{3} \to \mathbb{R} $ και 
$ (x_{0}, y_{0}, z_{0}) \in A $.
Τότε η \textcolor{Col1}{μερική παράγωγος της $f$ ως προς $x$} στο σημείο 
$ (x_{0}, y_{0}) $ είναι :
\begin{align*}
  \eval{\pdv{f}{x}}_{(x_{0}, y_{0}, z_{0})} = \lim_{x \to x_{0}} 
  \frac{f(x, y_{0}, z_{0}) - f(x_{0}, y_{0}, z_{0})}{x - x_{0}} 
  \overset{h=x- x_{0}}{=} \lim_{h \to 0}
  \frac{f(x_{0}+h, y_{0}, z_{0}) - f(x_{0}, y_{0}, z_{0})}{h}  
  \intertext{και η \textcolor{Col1}{μερική παράγωγος της $f$ ως προς $y$} στο σημείο 
  $ (x_{0}, y_{0}) $ είναι:}
  \eval{\pdv{f}{y}}_{(x_{0}, y_{0}, z_{0})} = \lim_{y \to y_{0}} 
  \frac{f(x_{0}, y, z_{0}) - f(x_{0}, y_{0}, z_{0})}{y - y_{0}} 
  \overset{k=y- y_{0}}{=} \lim_{k \to 0}
  \frac{f(x_{0}, y_{0}+k, z_{0}) - f(x_{0}, y_{0}, z_{0})}{k}  
  \intertext{και η \textcolor{Col1}{μερική παράγωγος της $f$ ως προς $z$} στο σημείο 
  $ (x_{0}, y_{0}) $ είναι:}
  \eval{\pdv{f}{z}}_{(x_{0}, y_{0}, z_{0})} = \lim_{z \to z_{0}} 
  \frac{f(x_{0}, y_{0}, z) - f(x_{0}, y_{0}, z_{0})}{z - z_{0}} 
  \overset{s=z- z_{0}}{=} \lim_{s \to 0}
  \frac{f(x_{0}, y_{0}, z_{0}+s) - f(x_{0}, y_{0}, z_{0})}{s}  
\end{align*}

\subsection*{Συμβολισμός}

Διάφοροι \textbf{συμβολισμοί} για τις μερικές παραγώγους της συνάρτησης $f(x,y)$ ως 
προς $x$ και ως προς $y$ είναι:
\begin{align*}
  \eval{\pdv{f}{x} }_{(x_{0}, y_{0})} = \pdv{f(x_{0}, y_{0})}{x} = 
  f_{x}(x_{0}, y_{0}) = f'_{x}(x_{0}, y_{0} ) \quad \text{και} \quad
  \eval{\pdv{f}{y} }_{(x_{0}, y_{0})} = \pdv{f(x_{0}, y_{0})}{y} = 
  f_{y}(x_{0}, y_{0}) = f'_{y}(x_{0}, y_{0} ) 
\end{align*} 

\begin{example}
  Δίνεται η $ f(x,y)=3xy^{2}-2x^{3} $. Να υπολογιστούν με τον \textbf{ορισμό} οι 
  μερικές παράγωγοι $ f_{x}(0,1) $ και $ f_{y}(2,1) $.
\end{example}
\begin{solution}
  \begin{align*}
    f_{x}(0,1) &= \lim_{x \to 0} \frac{f(x,1)-f(0,1)}{x-0} = 
    \lim_{x \to 0} \frac{3x1^{2}-2x^{3}-0}{x} = 
    \lim_{x \to 0} (3-2x^{2}) = 3
    \intertext{και}
    f_{y}(2,1) &= \lim_{y \to 1} \frac{f(2,y)-f(2,1)}{y-1} = 
    \lim_{y \to 0} \frac{3\cdot 2y^{2}-2\cdot 2^{3}-3\cdot 
    2\cdot 1^{2}+2\cdot 2^{3}}{y-1} = 
    \lim_{y \to 1} \frac{6y^{2}-16-6+16}{y-1} \\ 
               &= \lim_{y \to 1} \frac{6(y^{2}-1)}{y-1} = \lim_{y \to 1}
               \frac{6(y-1)(y+1)}{y-1} = \lim_{y \to 1}[6(y+1)] = 12
  \end{align*}          
  Εναλλακτικά μπορούμε να χρησιμοποιήσουμε τα όρια 
  \begin{align*}
    f_{x}(0,1) &= \lim_{h \to 0} \frac{f(0+h,1)-f(0,1)}{h} = 
    \lim_{h \to 0} \frac{3(0+h)1^{2}-2(0+h)^{3}-0}{h} = 
    \lim_{h \to 0} \frac{3h-2h^{3}}{h} = \lim_{h \to 0} (3-2h^{2}) = 3 
    \intertext{και}
    f_{y}(2,1) &= \lim_{k \to 0} \frac{f(2,1+k)-f(2,1)}{k} = 
    \lim_{k \to 0} \frac{3\cdot 2(1+k)^{2}-2\cdot 2^{3}+10}{k} = 
    \lim_{k \to 0} \frac{6(1+k)^{2}-6}{k} 
    \overset{(\frac{0}{0})}{\underset{\text{L H}}{=}} 
    \lim_{k \to 0} \frac{12(1+k)}{1} = 12\!\!\!
  \end{align*}
\end{solution}


\section{Συναρτήσεις Μερικών Παραγώγων}

Έστω $ f(x,y) $ συνάρτηση δύο μεταβλητών. 
\begin{myitemize}
  \item Η μερική παράγωγος της $f$ ως προς $x$ υπολογίζεται παραγωγίζοντας 
    την $ f(x,y) $ ως προς $x$, \textbf{θεωρώντας το $y$ σταθερό}. 
  \item Η μερική παράγωγος της $f$ ως προς $y$ υπολογίζεται παραγωγίζοντας 
    την $ f(x,y) $ ως προς $y$, \textbf{θεωρώντας το $x$ σταθερό}. 
\end{myitemize}

\begin{rem}
  Γενικότερα η \textcolor{Col1}{μερική παράγωγος της $f$ ως προς $ x_{i} $} 
  υπολογίζεται παραγωγίζοντας τη συνάρτηση $ f(x_{1}, \ldots, x_{n}) $ ως προς 
  $ x_{i} $, θεωρώντας \textbf{όλες} τις υπόλοιπες μεταβλητές σταθερές.
\end{rem}

\subsection{Κανόνες Παραγώγισης}

\twocolumnsides{
  \begin{myitemize}
    \item $ \pdv{x}(f+g) = \pdv{f}{x} + \pdv{g}{x} $
    \item $ \pdv{x}(af) = a \pdv{f}{x} $ 
  \end{myitemize}
}{
  \begin{myitemize}
    \item $ \pdv{x}(f\cdot g) = \pdv{f}{x} \cdot g + f \cdot \pdv{g}{x} $
    \item $ \pdv{x}(\frac{f}{g}) = \frac{\pdv{f}{x} \cdot g - f \cdot 
      \pdv{g}{x}}{g^{2}} $
\end{myitemize}
}

\subsection*{Παραδείγματα}

\begin{example}
  Έστω $ f(x,y)=x^{2}y^{3}+4xy^{2}+4y+5 $. Να 
  υπολογιστούν οι μερικές παράγωγοι $ f_{x} $ και 
  $ f_{y} $.
\end{example}
\begin{solution}
  \begin{align*}
    f_{x} &= (x^{2}y^{3}+4xy^{2}+4y+5)_{x} =
    (x^{2}y^{3})_{x}+(4xy^{2})_{x}+(4y)_{x}+(5)_{x} = 2xy^{3} + 4y^{2}
    \intertext{και}
    f_{y}&=(x^{2}y^{3}+4xy^{2}+4y+5)_{y} = 
    (x^{2}y^{3})_{y}+(4xy^{2})_{y}+(4y)_{y}+(5)_{y} = 3x^{2}y^{2} + 
    8xy + 4
  \end{align*} 
\end{solution}

\begin{example}
  Έστω $ f(x,y)=2x^{2}y+3 \cos{3y} +1 $. Να υπολογιστούν οι 
  μερικές παράγωγοι $ f_{x}$ και $ f_{y} $.
\end{example}
\begin{solution}
  \[
    f_{x}=4xy \quad \text{και} \quad f_{y}=2x^{2}-3 \sin{3y} (3y)_{y} 
    = 2x^{2}-9 \sin{3y}
  \] 
\end{solution}

\begin{example}
  Έστω $ f(x,y,z)=x^{2}yz - y \cos{(xy)} $. Να υπολογιστούν οι 
  μερικές παράγωγοι $ f_{x}, f_{y}, f_{z} $. 
\end{example}
\begin{solution}
\item {}
  \begin{align*}
    f_{x}&=2xyz- \cos{(xy)}(xy)_{x} = 2xyz-y \cos{xy} \\
    f_{y}&=x^{2}z- \cos{xy}(xy)_{y}=x^{2}z - x \cos{xy} \\
    f_{z}&=x^{2}z
  \end{align*}
\end{solution}

Αν η συνάρτηση της οποίας θέλουμε να υπολογίσουμε τις μερικές παραγώγους είναι 
δίκλαδη, τότε εργαζόμαστε όπως στα παρακάτω παραδείγματα.

\subsection*{Παραδείγματα}

\begin{example}
  Να υπολογίσετε τις μερικές παραγώγους 1ης τάξης της συνάρτησης.
  \[
    f(x,y) = 
    \begin{cases}
      x \frac{x^{2}-y^{2}}{x^{2}+y^{2}}, & (x,y) \neq (0,0) \\ 
      0, & (x,y) = (0,0) 
    \end{cases}
  \] 
\end{example}
\begin{solution}
\item {}
  \begin{myitemize}
    \item Αν $ (x,y) \neq (0,0) $ τότε 
      \[
        f_{x} = 
        \left( 
          x \frac{x^{2}-y^{2}}{x^{2}+y^{2}} 
        \right)_{x} = \cdots = 
        \frac{x^{4}+4x^{2}y^{2}-y^{4}}{(x^{2}+y^{2})^{2}} 
        \quad \text{και} \quad
        f_{y} = 
        \left(
          x \frac{x^{2}-y^{2}}{x^{2}+y^{2}} 
        \right)_{y} = \cdots - \frac{4x^{3}y}{(x^{2}+y^{2})^{2}}
      \]
    \item Αν $ (x,y) = (0,0) $ τότε εξετάζουμε τα όρια:
      \begin{align*}
        f_{x}(0,0) = \lim_{x \to 0} \frac{f(x,0)-f(0,0)}{x-0} = 
        \lim_{x \to 0} \frac{x}{x} = \lim_{x \to 0} 1 = 1
        \intertext{και}
        f_{y}(0,0) = \lim_{y \to 0} \frac{f(0,y)-f(0,0)}{y-0} = 
        \lim_{y \to 0} 0 = 0 
      \end{align*} 
      Άρα $ f_{x}= 
      \begin{cases}
        \frac{x^{4}+4x^{2}y^{2}-y^{4}}{(x^{2}+y^{2})^{2}}, &(x,y) 
        \neq (0,0) \\ 1 , &(x,y)=(0,0) 
      \end{cases}
      \quad \text{και} \quad f_{y} = 
      \begin{cases}
        - \frac{4x^{3}y}{(x^{2}+y^{2})^{2}}, &(x,y) \neq (0,0) \\ 
        0, &(x,y)=(0,0) 
      \end{cases} $  
  \end{myitemize}
\end{solution}

\begin{example}
\item     Να υπολογίσετε τις μερικές παραγώγους 1ης τάξης της συνάρτησης 
  \[
    f(x,y) = 
    \begin{cases}
      x^{2} \sin{\frac{y}{x}}, & x \neq 0 \\
      0, & x = 0 
    \end{cases}
  \] 
\end{example}
\begin{solution}
\item {}
  \begin{myitemize}
    \item Αν $ x \neq 0 $ τότε: 

      Υπολογίζουμε τις μερικές παραγώγους της συνάρτησης στα σημεία 
      $ (x,y) $ με $ x \neq 0 $.
      \begin{align*}
        f_{x}(x,y) &= \left(x^{2} \sin{\frac{y}{x}}\right)_{x} = 2x 
        \sin{\frac{y}{x}} + x^{2} \cos{\frac{y}{x}} 
        \left(\frac{y}{x}\right)_{x} = 
        2x \sin{\frac{y}{x}} - y \cos{\frac{y}{x}} 
        \intertext{και}
        f_{y}(x,y) &= \left(x^{2} \sin{\frac{y}{x}}\right)_{y} = 
        x^{2} \cos{\frac{y}{x}} \left(\frac{y}{x}\right)_{y} = x 
        \cos{\frac{y}{x}} 
      \end{align*} 

    \item Αν $ x = 0 $ τότε: 

      Υπολογίζουμε την μερική παράγωγο ως προς $x$ στο σημείο 
      $ (0, y_{0}) $.
      \[
        f_{x}(0, y_{0}) = \lim_{x \to 0} \frac{f(x, y_{0}) - 
          f(0, y_{0})}{x-0} = \lim_{x \to 0} \frac{x^{2} 
        \sin{\frac{y_{0}}{x} - 0}}{x} = \lim_{x \to 0} x 
        \sin{\frac{y_{0}}{x}} = 0 
      \] 
      Υπολογίζουμε την μερική παράγωγο ως προς $y$ στο σημείο 
      $ (0, y_{0}) $.
      \[
        f_{y}(0, y_{0}) = \lim_{y \to 0} \frac{f(0,y)-
        f(0, y_{0})}{y-0} = \lim_{y \to 0} \frac{0-0}{y} = 
        \lim_{y \to 0} = 0 
      \] 
      Άρα $ f_{x}= 
      \begin{cases}
        2x \sin{\frac{y}{x}} - y \cos{\frac{y}{x}}, & x \neq 0 \\
        0, & x = 0
      \end{cases}
      \quad \text{και} \quad f_{y} = 
      \begin{cases}
        x \cos{\frac{y}{x}}, & x \neq 0 \\
        0, & x = 0
      \end{cases} $  
  \end{myitemize}
\end{solution}


\subsection{Μερικές Παράγωγοι Ανώτερης Τάξης}

\begin{example}
\item {}
  Έστω $ f(x,y,z) = 3x^{2}y^{2} + xy^{3} + 3x +1 $. 
  Να υπολογιστούν οι μερικές παράγωγοι 1ης και 2ης τάξης.
\end{example}
\begin{solution}
  \begin{align*}
    f_{x} &= 6xy^{2}+y^{3}+3 \quad \text{και} \quad 
    f_{y} = 6x^{2}y+3xy^{2} \\
    f_{xx} &= (f_{x})_{x} = (6xy^{2}+y^{3}+3)_{x} =
    6y^{2} \\
    f_{yy} &= (f_{y})_{y} = (6x^{2}y+3xy^{2})_{y} = 
    6x^{2}+6xy \\
    f_{xy} &= (f_{x})_{y} = (6xy^{2}+y^{3}+3)_{y} = 
    12xy = 3y^{2} \tikzmark{a} \\
    f_{yx} &= (f_{y})_{x} = (6x^{2}y+3xy^{2})_{x} = 
    12xy+3y^{2} \; \; \, \tikzmark{b}
    \mybrace{a}{b}[\text{Μικτές Παράγωγοι}]
  \end{align*}
\end{solution}

\begin{rem}
\item {}
  Για τις μικτές παραγώγους $ f_{xy} $ και $ f_{yx} $ ισχύει:
  \begin{align*}
    \pdv[2]{f}{x}{y} = \pdv{}{x} \left(\pdv{f}{y}\right) = \pdv{}{x} \left(f_{y}\right) 
    = (f_{y})_{x} = f_{yx}
    \quad \text{και} \quad 
    \pdv[2]{f}{y}{x} = \pdv{}{y} \left(\pdv{f}{x}\right) = \pdv{}{y} \left(f_{x}\right) 
    = (f_{x})_{y} = f_{xy}
  \end{align*} 
\end{rem}

\begin{rem}
\item {}
  Οι πολυωνυμικές συναρτήσεις δύο (ή περισσότερων) μεταβλητών, 
  έχουν \textbf{συνεχείς} μερικές παραγώγους σε κάθε σημείο του $ \mathbb{R}^{2} $ 
  (ή $\mathbb{R}^{n}$).
  Οι λοιπές στοιχειώδεις συναρτήσεις $ \sin{f(x,y)}, \cos{f(x,y)}, a^{f(x,y)}, 
  \ln{f(x,y)} $ κ.λ.π.\ όπου $ f(x,y) $ \textbf{πολυωνυμική} συνάρτηση, έχουν 
  \textbf{συνεχείς} μερικές παραγώγους σε κάθε σημείο του πεδίου ορισμού τους.
  Για αυτές τις συναρτήσεις ισχύει $ f_{xy}=f_{yx} $.
\end{rem}

\begin{prop}
  Αν για μια συνάρτηση $ f(x,y)$ υπάρχουν οι μερικές παράγωγοι 1ης τάξης στο σημείο 
  $ (x_{0}, y_{0}) $ \textbf{και είναι συνεχείς}, τότε η συνάρτηση $f$ είναι συνεχής 
  στο σημείο $ (x_{0}, y_{0}) $.
\end{prop}

\begin{rem}
\item {}
  Η ύπαρξη και μόνο των μερικών παραγώγων 1ης τάξης μιας συνάρτησης σε ένα σημείο, 
  \textbf{δεν} σημαίνει ότι η συνάρτηση είναι συνεχής στο σημείο αυτό. Αυτό φαίνεται 
  στο επόμενο παράδειγμα.
\end{rem}

\begin{example}
\item {}
  Να εξετάσετε ως προς την ύπαρξη των μερικών παραγώγων και τη συνέχεια την συνάρτηση
  \[
    f(x,y) = 
    \begin{cases}
      \frac{xy}{x^{2}+y^{2}}, &(x,y) \neq (0,0) \\ 0, &(x,y) = (0,0) 
    \end{cases}
  \]
\end{example}
\begin{solution}
  Αρχικά, αποδεικνύουμε την ύπαρξη των μερικών παραγώγων, στο σημείο $ (0,0) $.
  \begin{align*}
    f_{x}(0,0) &= \lim_{x \to 0} \frac{f(x,0)-f(0,0)}{x-0} = \lim_{x \to 0}
    \frac{0}{x} = \lim_{x \to 0} 0 = 0 \quad \text{(υπάρχει)}
    \intertext{και}
    f_{y}(0,0) &= \lim_{y \to 0} \frac{f(0,y)-f(0,0)}{y-0} = \lim_{y \to 0}
    \frac{0}{y} = \lim_{y \to 0} = 0 \quad \text{(υπάρχει)}
    \intertext{όμως}
    \lim\limits_{\substack{x\to 0 \\y \to 0}} f(x,y) &= 
    \lim\limits_{\substack{x\to 0 \\y \to 0}} \frac{xy}{x^{2}+y^{2}} 
    \overset{y=\lambda x}{=} \lim_{x \to 0} 
    \frac{\lambda x^{2}}{x^{2}+ \lambda ^{2}x^{2}} = 
    \lim_{x \to 0} \frac{\lambda x^{2}}{x^{2}(1+ \lambda ^{2})} = 
    \lim_{x \to 0} \frac{\lambda}{1 + \lambda ^{2}} =
    \frac{\lambda}{1 + \lambda ^{2}} \quad \text{(δεν υπάρχει)}
  \end{align*} 
  Επομένως παρόλο που υπάρχουν οι μερικές παράγωγοι της συνάρτησης στο σημείο
  $ (0,0) $, η συνάρτηση δεν είναι συνεχής στο $ (0,0) $, γιατί δεν υπάρχει το 
  όριο της εκεί.
\end{solution}

\begin{thm}[Schwarz]
  Αν για τη συνάρτηση $ f(x,y) $ υπάρχουν οι μερικές παράγωγοι $ f_{xy} $ και 
  $ f_{yx} $ και είναι συνεχείς σε μια περιοχή του σημείου $ (x_{0}, y_{0}) $, τότε 
  $ f_{xy}=f_{yx} $ στην περιοχή αυτή.
\end{thm}

\begin{example}
  Να εξετάσετε αν ισχύει το θεώρημα Schwarz για την συνάρτηση 
  \[
    f(x,y) = 
    \begin{cases}
      xy \frac{x^{2}-y^{2}}{x^{2}+y^{2}}, &(x,y) \neq (0,0) \\
      0, & (x,y) = (0,0)
    \end{cases}
  \] 
\end{example}
\begin{solution}
\item {}
  Αν $ (x,y) \neq (0,0) $ τότε έχουμε
  \begin{equation*}
    f_{x}(x,y) = \left(xy\frac{x^{2}-y^{2}}{x^{2}+y^{2}}\right)_{x} = \cdots = 
    y\frac{x^{4}+4x^{2}y^{2}-y^{4}}{(x^{2}+y^{2})^{2}} 
    \quad \text{και} \quad
    f_{y}(x,y) = \left(xy\frac{x^{2}-y^{2}}{x^{2}+y^{2}}\right)_{y} = \cdots = 
    x\frac{x^{4}-4x^{2}y^{2}-y^{4}}{(x^{2}+y^{2})^{2}} 
  \end{equation*} 
  Αν $ (x,y) = (0,0) $ τότε έχουμε
  \[
    f_{x}(0,0) = \lim_{x \to 0} \frac{f(x,0)-f(0,0)}{x-0} = \lim_{x \to 0}
    \frac{0-0}{x} = 0 
    \quad \text{και} \quad
    f_{y}(0,0) = \lim_{y \to 0} \frac{f(0,y)-f(0,0)}{y-0} = \lim_{y \to 0} 
    \frac{0-0}{y} = 0 
  \]
  Επομένως οι μερικές παράγωγοι 1ης τάξης είναι 
  \[
    f_{x}(x,y) = 
    \begin{cases}
      \frac{x^{4}y+4x^{2}y^{3}-y^{5}}{(x^{2}+y^{2})^{2}}, & (x,y) \neq (0,0) \\
      0, & (x,y) = (0,0)
    \end{cases} \quad \text{και} \quad 
    f_{y}(x,y) = 
    \begin{cases}
      \frac{x^{5}-4x^{3}y^{2}-xy^{4}}{(x^{2}+y^{2})^{2}}, & (x,y) \neq (0,0) \\
      0, & (x,y) = (0,0)
    \end{cases}
  \] 
  Για τις μερικές παραγώγους 2ης τάξης, έχουμε:
  Αν $ (x,y) = (0,0) $ τότε
  \begin{align*}
    f_{xy}(0,0) = \lim_{y \to 0} \frac{f_{x}(0,y)-f_{x}(0,0)}{y-0} = 
    \lim_{y \to 0} \frac{\frac{-y^{5}}{y^{4}} - 0}{y} = \lim_{y \to 0}
    \frac{-y}{y} = \lim_{y \to 0} -1 = -1
    \intertext{και}
    f_{yx}(0,0) = \lim_{x \to 0} \frac{f_{y}(x,0)-f_{x}(0,0)}{x-0} = 
    \lim_{x \to 0} \frac{\frac{x^{5}}{x^{4}} - 0}{x} = \lim_{x \to 0}
    \frac{x}{x} = \lim_{x \to 0} 1 = 1 
  \end{align*} 

  Επομένως, $ f_{xy}(0,0) \neq f_{yx}(0,0) $, άρα για τη συνάρτηση 
  $ f(x,y) $ \textbf{δεν ισχύει} το θεώρημα Schwarz.
\end{solution}


\section{Μερική Ολοκλήρωση}

\begin{rem}
\item {}
  Αν $ f_{x}(x,y) = g(x,y)$ και $ f_{y}(x,y)=h(x,y) $ τότε ισχύει:
  \begin{align*}
    f(x,y) = \int g(x,y) \,{dx} + c(y) \quad \text{και} \quad f(x,y) = 
    \int h(x,y) \,{dy} + c(x) 
  \end{align*} 
\end{rem}

\begin{example}
  Έστω $ f(x,y)$ με $ f_{x}=e^{x+y} $ και $ f(0,y)=e^{y} $. Να βρεθεί ο τύπος της $f$.
\end{example}
\begin{solution}
  \begin{align*}
    f(x,y) = \int e^{x+y} \,{dx} = e^{x+y} + c(y) \; \tikzmark{a} \\ 
    f(0,y) = e^{y} \Leftrightarrow e^{y}+ c(y) = e^{y} \Rightarrow c(y) = 0 
    \; \tikzmark{b}
  \end{align*}
  \mybrace{a}{b}[ $f(x,y) = e^{x+y}$ ]
\end{solution}

\begin{rem}
  Όταν ολοκληρώνουμε μια συνάρτηση πολλών μεταβλητών, ως προς κάποια από τις 
  μεταβλητές της, τότε η σταθερά ολοκλήρωσης είναι \textbf{συνάρτηση} των υπολοίπων 
  μεταβλητών, οι οποίες θεωρούνται σταθερές κατά την ολοκλήρωση.
\end{rem}


