\chapter{Διαφορικά}

\section{Ολικό Διαφορικό}

\begin{dfn}
  Έστω η συνάρτηση $ f(x,y) $. Τα \textcolor{Col1}{ολικά διαφορικά} 1ης και 
  2ης τάξης της συνάρτηση $f$ συμβολίζονται με $ df $ και $ d^{2}f $, αντίστοιχα 
  και ισχύει:
  \[
    \boxed{df = f_{x}dx + f_{y}dy} \quad \text{και} \quad 
    \boxed{d^{2}f = f_{xx}dx^{2}+2f_{xy}dxdy+f_{yy}dy^{2}}
  \] 
  Στην περίπτωση όπου $ f= f(x_{1}, x_{2}, \ldots, x_{n}) $, το ολικό 
  διαφορικό 1ης τάξης γίνεται: 
  \[
    df = f_{x_{1}}d{x_{1}} + f_{x_{2}}d{x_{2}} + \cdots + f_{x_{n}} dx_{n}
  \]
\end{dfn}


\begin{example}
  Να βρείτε το ολικό διαφορικό της συνάρτησης $ f(x,y) = xye^{x+2y} $ 
\end{example}
\begin{solution}
\item {}
  Το ολικό διαφορικό δίνεται από τη σχέση $ df = f_{x} dx + f_{y} dy $.  
  Για τις μερικές παραγώγους έχουμε ότι: 
  \[
    f_{x} = ye^{x+2y}+xye^{x+2y} \quad \text{και} \quad f_{y} = xe^{x+2y} +
    2xye^{x+2y}
  \] 
  Επομένως
  \[
    df = y(1+x)e^{x+2y} dx + x(1+2y)e^{x+2y}dy
  \]
\end{solution}

\section{Εφαρμογές του Διαφορικού}

Αν $ \Delta x $ και $ \Delta y $ είναι \textit{μικρές} μεταβολές του $ x $ και του $y$ 
αντίστοιχα, τότε θέτουμε $ \Delta x = dx $ και $ \Delta y = dy $ και 
αποδεικνύεται ότι 
\[
  \Delta f(x,y) \approx df (x,y) \Leftrightarrow f(x+\Delta x, y+\Delta y) - f(x,y) 
  \approx df(x,y)
\] 
ή ισοδύναμα 
\begin{equation}\label{eq:efdiaf}
  \boxed{f(x+\Delta x, y+\Delta y) \approx f(x,y) + f_{x}(x,y)dx + f_{y}(x,y)dy}
\end{equation}

\begin{rem}
  Από τη σχέση σχέση~\eqref{eq:efdiaf}, καταλαβαίνουμε ότι, μπορούμε να υπολογίσουμε 
  κατά προσέγγιση, την τιμή της συνάρτησης $f$ σε κάποιο σημείο, αν γνωρίζουμε 
  την τιμή της και τις τιμές των μερικών παραγώγων της, σε κάποιο άλλο
  \textit{κοντινό} σημείο.
\end{rem}

\begin{example}
  Να υπολογίσετε το $ \Delta f $ και το $ df $ στο σημείο $ (x,y) = (1,1) $, 
  για τη συνάρτηση $ f(x,y) = x^{3}+y^{2} $ και για 
  \begin{enumerate}[i)]
    \item $ \Delta x = 0,1, \; \Delta y = 0,1 $
    \item $ \Delta x = 0,01, \; \Delta y = 0,01 $
  \end{enumerate}
  Τι παρατηρείτε?
\end{example}
\begin{solution}
\item {}
  \begin{enumerate}[i)]
    \item 
      \begin{align*} 
        \Delta f &= f(x+ \Delta x, y + \Delta y) - f(x,y) = f(1+0.1,1+0.1) - 
        f(1,1) = f(1.1,1.1) - f(1,1) \\ 
                 &= [(1.1)^{3}+(1.1)^{2}] - (1^{3}+1^{2}) = 0.541 \\
        df &= f_{x}dx + f_{y}dy  = 3x^{2} dx + 2y dy \Rightarrow df (1,1) = 
        3\cdot 1^{2} \cdot 0.1 + 2 \cdot 1 \cdot 0.1 = 0.5 
      \end{align*}
    \item 
      \begin{align*} 
        \Delta f &= f(x+ \Delta x, y + \Delta y) - f(x,y) = f(1+0.01,1+0.01) - 
        f(1,1) = f(1.01,1.01) - f(1,1) \\ 
                 &= [(1.01)^{3}+(1.01)^{2}] - (1^{3}+1^{2}) = 0.0504 \\
        df &= f_{x}dx + f_{y}dy  = 3x^{2} dx + 2y dy \Rightarrow df (1,1) = 
        3\cdot 1^{2} \cdot 0.01 + 2 \cdot 1 \cdot 0.01 = 0.05 
      \end{align*}
  \end{enumerate}
  Παρατηρούμε ότι στην 1η περίπτωση, έχουμε
  $ \Delta f - df \approx 0.04 $, ενώ στη 2η περίπτωση έχουμε $ \Delta f - df \approx 
  0,0004 $, δηλαδή στη 2η περίπτωση, όπου τα $ \Delta x $ και $ \Delta y $ επιλέχθηκαν 
  να είναι μικρότερα, το διαφορικό προσεγγίζει με \textbf{μεγαλύτερη ακρίβεια} τη 
  διαφορά $ \Delta f $.
\end{solution}

\begin{example}
  Κατά τη μέτρηση των στοιχείων μιας τριγωνικής επιφάνειας, βρέθηκαν οι πλευρές να 
  έχουν μήκος \SI{100}{cm} και \SI{125}{cm} αντίστοιχα, ενώ η περιεχόμενη γωνία βρέθηκε 
  να είναι \SI{60}{\degree}. Αν το πιθανό σφάλμα κατά την μέτρηση των πλευρών είναι 
  $ \SI{0.2}{cm} $ και για τις γωνίες είναι $ \SI{1}{\degree} $, να υπολογίσετε κατά 
  προσέγγιση το σφάλμα στο εμβαδό της επιφάνειας.
\end{example}

\begin{solution}
\item {}

  \twocolumnsidelc{
    \begin{tikzpicture}[scale=0.7]
      \node  (B) at (0, 0) {};
      \node  (A) at (6, 0) {};
      \node  (C) at (2, 2) {};
      \node  (3) at (2, 0) {};
      \draw (B.center) to node[left=3pt]{$x$} (C.center);
      \draw (C.center) to (A.center);
      \draw (B.center) to node[below]{$y$} (A.center) 
        pic[draw,fill=magenta!25,"$\phi$",angle eccentricity=1.5]{angle} ;
      \draw [dashed, in=90, out=-90] (C.center) to node[right]{$z$} (3.center);
    \end{tikzpicture}
  }{
    Έχουμε ότι $ E = \frac{y \cdot z}{2} $. Όμως από το ορθογώνιο τρίγωνο έχουμε ότι 
    $ z = x \sin{\phi} $, άρα 
    \[
      E(x,y,\phi) = \frac{y \cdot x \cdot \sin{\phi}}{2} 
    \] 
  }
  Παρατηρούμε ότι $ \Delta x = dx = 0.2, \; \Delta y = dy = 0.2 $ και 
  $ \Delta \phi = d\phi = \SI{1}{\degree} = \frac{\pi}{180} $. Οπότε,ισχύει ότι
  \begin{equation*}
    \Delta E \approx dE = E_{x} dx + E_{y} dy + E_{\phi} d\phi = \frac{y \cdot
    \sin{\phi}}{2} dx + \frac{x \cdot \sin{\phi}}{2} dy + 
    \frac{y \cdot x \cdot \cos{\phi}}{2} d\phi
  \end{equation*} 
  και πιο συγκεκριμένα ισχύει ότι 
  \begin{align*}
    \Delta E(100,125,60) \approx dE(100,125,60) 
  &= E_{x} dx + E_{y} dy + E_{\phi} d\phi \\
  &= E_{x}(100,125,60) dx + E_{y}(100,125,60) dy + E_{\phi}(100,125,60) d\phi \\
  &= \frac{1}{2} \left(125 \cdot \sin{\frac{\pi}{3}} \cdot 0.2 + 100 \cdot
    \sin{\frac{\pi}{3}} \cdot 0.2 + 100 \cdot 125 \cdot \cos{\frac{\pi}{3}} \cdot
  \frac{\pi}{180}\right) \\
  &= \SI{74.03}{cm^{2}}
  \end{align*}
  Άρα το σφάλμα μέτρησης του εμβαδού της τριγωνικής επιφάνειας είναι 
  $ \SI{74.03}{cm^{2}} $
\end{solution}

\begin{example}
  Κατά την εκτίμηση των συντεταγμένων ενός σημείου $ P(x,y) $ γίνονται σφάλματα 
  $ dx $ και $ dy $ αντίστοιχα. Να υπολογίσετε την επίδραση αυτών των σφαλμάτων 
  στη γωνία $ \theta $ και να γίνει αριθμητική εφαρμογή για $ x=1 $, $ y=2 $ και για 
  $ dx = 0.01 $ και $ dy =-0.02 $.
\end{example}
\begin{solution}
\item {}

  \twocolumnsidelc{
    \begin{tikzpicture}[scale=0.7]
      \node (0) at (0, 0) {};
      \node (1) at (4.5, 0) {};
      \node (2) at (0, 4) {};
      \node (3) at (3, 2) {};
      \draw [fill] (3,2) node[right] {$P(x,y)$} circle (2pt);
      \node (4) at (2, 3) {};
      \draw [fill] (2,3) node[right] {$P'(x+dx, y+dy)$} circle (2pt);
      \node (5) at (2, 0) {};
      \node (6) at (3, 0) {};
      \draw (0.center) to (3.center);
      \draw (0.center) to (4.center);
      \draw [dashed] (4.center) to (5.center);
      \draw [dashed] (3.center) to (6.center);
      \draw [-stealth] (0.center) to (1.center);
      \draw [-stealth] (0.center) to (2.center);
      \draw pic[draw,fill=magenta!25,"$\theta$",angle eccentricity=1.5]{angle=1--0--3};
  \end{tikzpicture}}{
  Έστω οι συντεταγμένες του σημείου $ P'(x+ dx, y + dy) $, όπως φαίνονται στο σχήμα.
  Έχουμε ότι 
  \[
    \tan{\theta} = \frac{y}{x} \Rightarrow \theta = \arctan{\frac{y}{x}}
  \] 
  Αν η γωνία μετά το σφάλμα της εκτίμησης είναι $ \theta'$, έχουμε ότι 
  \[
    \theta ' = \arctan{\frac{y+ dy}{x + dx}} 
\]} 
Παρατηρούμε ότι $ \Delta x = dx = 0.01 $, $ \Delta y = dy = -0.02 $. Οπότε ισχύει ότι 
\[
  \Delta \theta \approx d \theta = \theta _{x} dx + \theta _{y} dy =
  \frac{-y}{x^{2}+y^{2}} dx + \frac{x}{x^{2}+y^{2}} dy = \frac{xdy-ydx}{x^{2}+y^{2}}
\]
και μάλιστα υπολογίζουμε την απόλυτη τιμή και έχουμε
\[
  \abs{\Delta \theta} \approx \frac{\abs{xdy-ydx}}{x^{2}+y^{2}} \leq \frac{\abs{x dy}
  + \abs{y dx}}{x^{2}+y^{2}} 
\]
και πιο συγκεκριμένα, για το σημείο $ P(1,2) $ έχουμε
\[
  \abs{\Delta \theta(1,2)} \leq \frac{\abs{1(-0.02)}+\abs{2\cdot 0.01}}{1^{2}+2^{2}} =
  \frac{0.04}{5} = 0.008
\]
\end{solution}

\section{Τέλειο Διαφορικό}

\dfn{Θεωρούμε τις συναρτήσεις $ P(x,y) $ και $ Q(x,y) $ με πεδίο ορισμού $ A \subseteq
  \mathbb{R}^{2} $.
  Η παράσταση $ P(x,y) dx + Q(x,y) dy $ λέγεται \textcolor{Col2}{τέλειο 
  διαφορικό} αν υπάρχει συνάρτηση $ f(x,y) $ με πεδίο ορισμού το $A$, ώστε 
  \begin{gather*}
    df = P(x,y)dx + Q(x,y)dy \Leftrightarrow \pdv{f}{x} dx + \pdv{f}{y} dy = 
    P(x,y)dx + Q(x,y)dy \Leftrightarrow \\
    \boxed{\pdv{f}{x} = P(x,y) \quad \text{και} \quad \pdv{f}{y} = Q(x,y)}
\end{gather*}}

\begin{prop}
  Αν οι  $ P(x,y) $  και  $ Q(x,y) $  είναι συνεχείς συναρτήσεις και έχουν συνεχείς 
  παραγώγους πρώτης τάξης, σε μια ορθογώνια περιοχή $A$ του $ \mathbb{R}^{2} $,  
  τότε η  παράσταση  $ P(x,y)dx + Q(x,y)dy $ είναι τέλειο διαφορικό αν 
  \[
    \boxed{\pdv{p}{y} = \pdv{q}{x}} \quad \forall (x,y) \in A
  \]
\end{prop}

\begin{dfn}
  Η παράσταση  $ P(x,y,z)dx + Q(x,y,z)dy + R(x,y,z)dz $ είναι τέλειο διαφορικό 
  αν υπάρχει συνάρτηση  $ f(x,y,z) $  τέτοια ώστε  $ df = P(x,y,z)dx + Q(x,y,z)dy 
  + R(x,y,z)dz $.  Τότε ισχύουν οι παρακάτω σχέσεις:
  \[
    \boxed{\pdv{f}{x} = P(x,y,z) \quad \text{και} \quad \pdv{f}{y} = Q(x,y,z) 
    \quad \text{και} \quad \pdv{f}{z} = R(x,y,z)} 
  \] 
\end{dfn}

\begin{prop}
  Αν οι  $ P(x,y,z) $, $ Q(x,y,z) $  και  $ R(x,y,z) $ είναι συνεχείς συναρτήσεις 
  και έχουν συνεχείς παραγώγους πρώτης τάξης, σε μια ορθογώνια περιοχή Α του 
  $ \mathbb{R}^{3} $ τότε η  παράσταση 
  $ P(x,y,z)dx + Q(x,y,z)dy + R(x,y,z)dz $   είναι τέλειο διαφορικό αν 
  \[
    \boxed{\pdv{P}{y} = \pdv{Q}{x}} \quad \text{και} \quad \boxed{\pdv{Q}{z} = 
    \pdv{R}{y}} \quad \text{και} \quad  \boxed{\pdv{P}{z} = \pdv{R}{x}} 
    \quad \forall (x,y,z) \in A 
  \] 
\end{prop}

\begin{rem}\label{olokl}
  Οι συναρτήσεις  $ f(x,y) $  και  $ f(x,y,z) $ υπολογίζονται επίσης από τις 
  παρακάτω σχέσεις:
  \begin{align*}
    f(x,y) &= \int_{x_{0}}^{x} P(t,y) \,{dt} + \int_{y_{0}}^{y} Q(x_{0},t) \,{dt} \\
    f(x,y,z) &= \int_{x_{0}}^{x} P(t,y,z) \,{dt} + \int_{y_{0}}^{y} Q(x_{0},t,z) 
    \,{dt} + \int _{z_{0}}^{z} R(x_{0},y_{0},t) \,{dt}  
  \end{align*}
  όπου τα $ x_{0} $, $ y_{0} $  και  $ z_{0} $ επιλέγονται \textbf{αυθαίρετα} στο πεδίο 
  ορισμού των  $ P $, $ Q $  και  $ R $.
\end{rem}

\begin{rem}
  Η συνάρτηση $ f(x,y) $ ή η συνάρτηση $ f(x,y,z) $ του ορισμού του τέλειου διαφορικού
  λέγεται \textcolor{Col2}{συνάρτηση δυναμικού}.
\end{rem}

\begin{example}
  Να εξετάσετε αν η παράσταση $ \left(1+x- {2}/{y}\right)dx + 
  \left(1+ {2x}/{y^{2}} \right)dy $ είναι τέλειο διαφορικό και αν ναι, να 
  υπολογίσετε τη συνάρτηση δυναμικού.
\end{example}
\begin{solution}
  Ελέγχουμε με το κριτήριο:
  \[ 
    \pdv{P}{y} = \frac{2}{y^{2}} = \pdv{Q}{x} 
  \]
  Άρα η παράσταση είναι τέλειο διαφορικό. Επομένως υπάρχει 
  συνάρτηση, $ f(x,y) $ τέτοια ώστε: 
  \begin{align}
    \pdv{f}{x} &= 1 + x - \frac{2}{y} \label{fx1} \\
    \pdv{f}{y} &= 1+ \frac{2x}{y^{2}} \label{fy1}
  \end{align}
  Ολοκληρώνουμε μερικώς ως προς $x$ την~\eqref{fx1} και έχουμε
  \[
    f(x,y) = \int \left(1+x- \frac{2}{y}\right) \,{dx} = x + 
    \frac{x^{2}}{2} - \frac{2x}{y} + c(y) 
  \] 
  Άρα  
  \begin{equation}
    f(x,y) = x + \frac{x^{2}}{2} - \frac{2x}{y} + c(y) \label{fxy}
  \end{equation}
  Στη συνέχεια παραγωγίζουμε μερικώς ως προς $y$ τη συνάρτηση $ f(x,y) $ που μόλις 
  βρήκαμε και έχουμε:
  \begin{equation}
    \pdv{f}{y} = \frac{2x}{y^{2}} + c'(y) \label{ffy}
  \end{equation} 
  Στη συνέχεια εξισώνουμε την~\eqref{ffy} με την~\eqref{fy1} και προκύπτει
  \[
    c'(y) = 1 \Rightarrow c(y) = y + k 
  \] 
  Άρα, τελικά η συνάρτηση δυναμικού είναι 
  \[
    f(x,y) = x + \frac{x^{2}}{2} - \frac{2x}{y} + y + k 
  \] 
\end{solution}

\begin{example}
  Να εξετάσετε αν η παράσταση $ (3x^{2}+3y-1)dx + (z^{2}+3x)dy + (2yz+1)dz $ είναι 
  τέλειο διαφορικό και αν ναι, να υπολογίσετε τη συνάρτηση δυναμικού.
\end{example}
\begin{solution}
  Ελέγχουμε με το κριτήριο:
  \[
    \pdv{P}{y} = 3 = \pdv{Q}{x} \quad \text{και} \quad \pdv{Q}{z} = 2z = \pdv{R}{y}
    \quad \text{και} \quad \pdv{P}{z} = 0 = \pdv{R}{x}
  \] 
  Άρα η παράσταση είναι τέλειο διαφορικό. Για να βρούμε τη συνάρτηση δυναμικού, έχουμε
  \begin{description}
    \item [A᾽ Τρόπος: (Με Τύπο)]
      Θα χρησιμοποιήσουμε τον τύπο της παρατήρησης~\ref{olokl} για να υπολογίσουμε 
      τη συνάρτηση δυναμικού. 
      \begin{align*}
        f(x,y,z) &= \int _{x_{0}}^{x} P(t,y,z) \,{dt} + \int _{y_{0}}^{y} Q(x_{0},t,z) 
        \,{dt} + \int _{z_{0}}^{z} R(x_{0}, y_{0}, t) \,{dt} \\
                 &= \int _{0}^{x} (3t^{2}+3y-1) \,{dt} + \int _{0}^{y} (z^{2}+3\cdot 0) 
                 \,{dt} + \int _{0}^{z} (2\cdot 0\cdot t + 1) \,{dt} \\ 
                 &= \int _{0}^{x} (3t^{2}+3y-1) \,{dt} + \int _{0}^{y} z^{2} \,{dt} + 
                 \int _{0}^{z} 1 \,{dt} \\
                 &= \left[t^{3}+3yt-t\right]_{0}^{x} + \left[z^{2}t\right]_{0}^{y} + 
                 \bigl[t\bigr]_{0}^{z} \\
                 &= x^{3}+3xy-x + z^{2}y+z
      \end{align*}
    \item [B᾽ Τρόπος: (Με Ολοκλήρωση)] Αφού η παράσταση είναι τέλειο διαφορικό,  
      υπάρχει συνάρτηση, $ f(x,y,z) $ τέτοια ώστε: 
      \begin{align}
        \pdv{f}{x} &= 3x^{2}+3y-1 \label{fx11} \\
        \pdv{f}{y} &= z^{2}+3x \label{fy11} \\
        \pdv{f}{z} &= 2yz+1 \label{fz11}
      \end{align} 
      Ολοκληρώνουμε μερικώς την~\eqref{fx11} και έχουμε
      \begin{equation*}
        f(x,y,z) = \int (3x^{2}+3y-1) \,{dx} = x^{3} + 3xy -x + c(y,z) 
      \end{equation*} 
      Άρα 
      \begin{equation}
        f(x,y,z) =  x^{3} + 3xy -x + c(y,z) \label{fxyz}
      \end{equation}
      Στη συνέχεια παραγωγίζουμε μερικώς ως προς $y$ και ως προς $z$  
      τη συνάρτηση $f(x,y,z)$ που μόλις βρήκαμε και έχουμε:
      \begin{align}
        \pdv{f}{y} &= 3x + \pdv{c}{y} \label{cy} \\
        \pdv{f}{z} &= \pdv{c}{z} \label{cz}
      \end{align}
      Στη συνέχεια εξισώνοντας την~\eqref{cy} με την~\eqref{fy11} προκύπτει 
      \begin{equation}
        \pdv{c}{y} = z^{2} \label{last1}
      \end{equation}
      και εξισώνοντας την~\eqref{cz} με την~\eqref{fz11} προκύπτει
      \begin{equation}
        \pdv{c}{z} = 2yz+1 \label{last2}
      \end{equation}
      Από τις σχέσεις~\eqref{last1} και~\eqref{last2} μπορούμε να βρούμε τη συνάρτηση 
      $ c(y,z) $ ακολουθώντας τη διαδικασία του προηγούμενου παραδείγματος, αφού 
      ουσιαστικά έχουμε τις μερικές παραγώγους της συνάρτησης και ζητάμε 
      να βρούμε την ίδια τη συνάρτηση. Οπότε ολοκληρώνουμε μερικώς ως προς 
      $y$ την~\eqref{last1} και έχουμε
      \begin{equation}
        c(y,z) = \int z^{2} \,{dy} = z^{2}y + k(z) \label{kz} 
      \end{equation} 
      Στη συνέχεια παραγωγίζουμε ως προς $z$ τη συνάρτηση $ c(y,z) $ που μόλις βρήκαμε 
      και έχουμε
      \begin{equation}
        \pdv{c}{z} = 2yz + k'(z) \label{kz1}
      \end{equation} 
      Έπειτα εξισώνουμε τις σχέσεις~\eqref{last2} και~\eqref{kz1} και προκύπτει 
      \[
        k'(z) = 1 \Rightarrow k(z) = z + c_{1} \label{finally}
      \] 
      Οπότε με αντικατάσταση της~\eqref{finally} στην~\eqref{kz} βρίσκουμε 
      \begin{equation}
        c(y,z) = z^{2}y+z + c_{1} \label{cyz}
      \end{equation}
      Τέλος με αντικατάσταση της~\eqref{cyz} στη συνάρτηση $ f(x,y,z) $ από τη 
      σχέση~\eqref{fxyz} βρίσκουμε
      \[
        f(x,y,z) = x^{3}+3xy-x+z^{2}y+z+ c_{1} 
      \] 
  \end{description}
\end{solution}


