\input{preamble_ask.tex}
\input{definitions_ask.tex}


\pagestyle{askhseis}

\everymath{\displaystyle}




\begin{document}

\begin{center}
  {\color{Col1}\bfseries\large Ασκήσεις}
\end{center} 
\section*{Θεωρήματα Πεπλεγμένων Συναρτήσεων}

\subsection*{Εξίσωση της μορφής \ensuremath{F(x,y)=0}}

\begin{enumerate}
    \item Έστω ότι η εξίσωση $ x-1-2y+e^{xy} = 0 $ ορίζει πεπλεγμένη συνάρτηση της 
        μορφής $ y = y(x) $. Να υπολογιστούν οι μερικές παράγωγοι 
        1ης και 2ης τάξης, αυτής της συνάρτησης.

        \hfill Απ: $ \dv{y}{x} = \frac{1+ye^{xy}}{2-xe^{xy}} $ και $ \dv[2]{y}{x}
        = \frac{(y+xy')^{2}+2y'}{2-xe^{xy}} $ 

    \item Έστω ότι η εξίσωση $ y^{3} - yx $ ορίζει πεπλεγμένη συνάρτηση της 
        μορφής $ y = y(x) $. Να υπολογιστεί η μερική παράγωγος 2ης τάξης, αυτής 
        της συνάρτησης.

        \hfill Απ: $ \dv[2]{y}{x} = \frac{ -2xy }{ (3y^{2} - x)^{3} }  $ 

    \item Να δείξετε ότι η εξίσωση $ e^{x+y} + y- x = 0 $ ορίζει πεπλεγμένη συνάρτηση 
        $ y=y(x) $ σε μια περιοχή του σημείου $ \left(\frac{1}{2}, - 
        \frac{1}{2}\right) $, και να βρεθεί η παράγωγός $ \dv{y}{x} $.

        \hfill Απ: $ \dv{y}{x} = \frac{1-e^{x+y}}{1+e^{x+y}} $  
\end{enumerate}

\subsection*{Εξίσωση της μορφής \ensuremath{F(x,y,z)=0}}

\begin{enumerate}

    \item Να δείξετε ότι η εξίσωση $ y^{2} + xz + z^{2} - e^{z} - 4 = 0 $ ορίζει 
        πεπλεγμένη συνάρτηση $ z = z(x,y) $ σε μια περιοχή του σημείου $ M(0,e,2) $ και 
        να υπολογίσετε τις μερικές παραγώγους $ z_{x} $ και $ z_{y} $ στο σημείο $M$.

        \hfill Απ: $ z_{x} = - \frac{2}{4-e^{2}} $ και $ z_{y} = - \frac{2e}{4-e^{2}} $ 

    \item Να δείξετε ότι η εξίσωση $ 2x-y-z- \ln{x+y+z-2} e^{x+y} = 0 $ ορίζει 
        πεπλεγμένη συνάρτηση $ z = z(x,y) $ σε μια περιοχή του σημείου $ M(1,1,1) $ και 
        να υπολογίσετε τις μερικές παραγώγους $ z_{x} $ και $ z_{y} $ στο σημείο $M$.

        \hfill Απ: $ z_{x} =  \frac{2-e^{2}}{1+e^{2}} $ και $ z_{y} = -1 $ 

    \item Να δείξετε ότι η εξίσωση $ x \cos{y} + y \cos{z} + z \cos{x} -1 = 0  $ ορίζει 
        πεπλεγμένη συνάρτηση $ z=z(x,y) $ σε μια περιοχή του σημείου 
        $ (0,0) $, τέτοια ώστε $ z(0,0) = 1 $ και να υπολογιστούν οι μερικές παράγωγοι 
        $ z_{x} $ και $ z_{y} $.

        \hfill Απ: $ z_{x} = - \frac{\cos{y} - z \sin{x}}{\cos{x} - y \sin{z}} $ και 
        $ z_{y} = - \frac{\cos{z} - x \sin{y}}{\cos{x} - y \sin{z}} $ 

    \item Έστω η εξίσωση $ z^{3} - xz - y = 0 $. 
        \begin{enumerate}[i)]
            \item Να βρεθούν τα σημεία $ (x,y,z) \in \mathbb{R}^{3} $ για τα οποία 
                ορίζεται πεπλεγμένη συνάρτηση $ z=z(x,y) $. 
            \item Να υπολογιστούν οι $ z_{x}(0,1) $ και $ z_{yy}(0,1) $. 
            \item Να υπολογιστεί το ανάπτυγμα Taylor της συνάρτησης $ z(x,y) $ 
                γύρω από το σημείο $ (0,1) $, μέχρι όρους 2ης τάξης.
        \end{enumerate}

        \hfill Απ: \begin{tabular}{l}
            $ \mathrm{i)} \; 3{z_{0}}^{2} \neq x_{0} $ \\
            $ \mathrm{ii)} \; z_{x}(0,1) = \frac{1}{3} $ και 
            $ z_{y}(0,1) = -\frac{2}{9} $ \\
            $ \mathrm{iii)} \; z(x,y) = 1 + \frac{1}{3} x + \frac{1}{3}(y-1) 
            - \frac{1}{9} x(y-1) - \frac{2}{9} (y-1)^{2} + \cdots $
        \end{tabular}

    \item Έστω η εξίσωση $ y - 2xz + z^{3} = 0 $. Να αποδείξετε ότι η εξίσωση 
        ορίζει μια πεπλεγμένη συνάρτηση $ z = z(x,y) $ στην περιοχή του σημείου 
        $ P(1,1,1) $ και στη συνέχεια να υπολογίσετε το πολυώνυμο Taylor αυτής 
        της συνάρτησης γύρω από το σημείο $ (1,1) $, μέχρι και όρους 2ης τάξης. 

        \hfill Απ: $ z(x,y) =  1 + 2(x-1) - (y-1) - 8(x-1)^{2} + 10(x-1)(y-1) - 
        3(y-1)^{2} + \cdots $ 

\end{enumerate}

\end{document}
