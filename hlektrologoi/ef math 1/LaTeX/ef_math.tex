\input{preamble.tex}
\newcommand{\vect}[2]{(#1_1,\ldots, #1_#2)}
%%%%%%% nesting newcommands $$$$$$$$$$$$$$$$$$$
\newcommand{\function}[1]{\newcommand{\nvec}[2]{#1(##1_1,\ldots, ##1_##2)}}

\newcommand{\linode}[2]{#1_n(x)#2^{(n)}+#1_{n-1}(x)#2^{(n-1)}+\cdots +#1_0(x)#2=g(x)}

\newcommand{\vecoffun}[3]{#1_0(#2),\ldots ,#1_#3(#2)}

\newcommand{\suma}{\sum_{n=0}^{\infty}a_n x^n}

\newcommand{\sumb}{\sum_{n=1}^{\infty}a_n n x^{n-1}}

\newcommand{\sumc}{\sum_{n=2}^{\infty}a_n n (n-1) x^{n-2}}

\newcommand{\varsum}[2]{\sum_{n=#1}^{#2}}





\begin{document}



\begin{center}
\fbox{\bfseries\large Μέθοδος Δυναμοσειρών ($x_0$ Ομαλό Σημείο)}
\end{center}

\vspace{2\baselineskip}
%\everymath{\displaystyle}
%\pagestyle{empty}

\begin{enumerate}

\item Να λυθούν οι παρακάτω δ.ε. με τη μέθοδο των δυναμοσειρών.

\begin{enumerate}[i)]

\item \fbox{$y''+y=0,\quad x_0=0$}

\vspace{\baselineskip}
\underline{\textbf{Λύση}}

\vspace{\baselineskip}
Έχουμε $p(x)=0$ και $q(x)=1$ αναλυτικές $\forall x\in\mathbb{R}$ άρα και για $x_0=0$, οπότε το $x_0=0$ είναι \textbf{ομαλό} σημείο. 

Άρα θεωρούμε λύση της μορφής 
\[
y(x)=\suma
\]
οπότε,
\begin{align*}
y'(x) &= \suma \\
y''(x) &= \sumb
\end{align*}
και με αντικατάσταση στη δ.ε. έχουμε:

\begin{align*}
\sumc+\suma &=0 \Leftrightarrow \\
\varsum{0}{\infty}a_{n+2}(n+1)(n+2)x^{n}+\suma &=0 \Leftrightarrow \\
\varsum{0}{\infty}[a_{n+2}(n+1)(n+2)+a_n]x^n &=0 \Leftrightarrow \\
\end{align*}
Πρέπει:
\[
a_{n+2}=-\frac{1}{(n+1)(n+2)}a_n,\quad n\geq 0
\]
Άρα 
\[
a_2=-\frac{a_0}{2\cdot 1}=-\frac{a_0}{2},\quad a_4=-\frac{a_2}{4\cdot 3}=\frac{a_0}{4!},\quad a_6=-\frac{a_4}{6\cdot 5}=-\frac{a_0}{6!}, \ldots
\]

\end{enumerate}

\end{enumerate}





\end{document}