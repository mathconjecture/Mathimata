\input{preamble_ask.tex}
\input{definitions_ask.tex}


\pagestyle{askhseis}

\begin{document}


\begin{center}
  \minibox{\large \bfseries \textcolor{Col1}{Ασκήσεις στις Παραγώγους}}
\end{center}

\vspace{\baselineskip}

\begin{enumerate}
  \item Να υπολογιστούν οι παράγωγοι των παρακάτω συναρτήσεων
    \begin{enumerate}[(i)]
      \item $ f(x) = \ln{\sqrt[5]{1+3x^{2}}} $ \hfill Απ: $ \frac{6x}{5(1+3x^{2})} $
      \item $ f(x) = \ln({\sin({\cos{x}})}) $ \hfill Απ: $ \frac{1}{\sin{(\cos{x})}} [\cos{(\cos{x})}] (- \sin{x}) $ 
      \item $ f(x) = \arctan \frac{x}{\sqrt{1 + x^{2}}} $ \hfill Απ: $ \frac{1}{(1+2x^{2})\sqrt{1 + x^{2}}} $
      \item $ f(x) = \ln{(e^{\sin{x}})} + \sqrt{x^{2} - 25x} $ \hfill Απ: $ \cos{x} + \frac{2x - 25}{2 \sqrt{x^{2} - 25x}}  $  
    \end{enumerate}

  \item  Να υπολογιστούν οι παράγωγοι των παρακάτω συναρτήσεων
    \begin{enumerate}[(i)]
      \item $ f(x) = (\cos{x})^{\sin{2x}} $ \hfill Απ: $ (\cos{x})^{\sin{2x}} 2(\cos{2x} \ln{(\cos{x})} - \sin^{2}{x}) $
      \item $ f(x) = \left(1 + \frac{1}{x} \right)^{x} $ \hfill Απ: $ \left(1 + \frac{1}{x}\right)^{x}\left[\ln{(1 + \frac{1}{x})} - \frac{1}{x+1}\right] $
      \item $ f(x) = (\sin{x})^{x} $ \hfill Απ: $ (\sin{x})^{x}[\ln{(\sin{x})} + x \cot{x}] $ 
      \item $ f(x) =  \cos{x}^{x} $ \hfill Απ: $ (- \sin{x^{x}})x^{x} (1 + \ln{x}) $
    \end{enumerate}

  \item Να υπολογιστούν τα παρακάτω όρια.
    \begin{enumerate}[(i)]
      \twocolumnside{
      \item $ \lim_{x\to 1} \left(\frac{1}{\ln{x}} - \frac{1}{x-1}\right) $ \hfill Απ: $ \frac{1}{2} $
      \item $ \lim_{x\to 1} \left[(1-x) \tan{\frac{\pi x}{2}}\right] $ \hfill Απ: $ \frac{2}{\pi} $
      \item $ \lim_{x\to \frac{\pi}{4}} \frac{\sqrt{2} - \sin{x} - \cos{x}}{\ln{(\sin{2x})}} $ \hfill Απ: $ - \frac{\sqrt{2}}{4} $
        }{
      \item $ \lim_{x\to +\infty} \left(1 + \frac{1}{x} + \frac{2}{x^{2}}\right)^{x} $ \hfill Απ: $ e $ 
      \item $ \lim_{x\to 0^{+}} \left(\frac{1}{x}\right)^{\sin{x}} $ \hfill $ 1 $
      \item $ \lim_{x\to 0} \left(\cos{2x}\right)^{\frac{3}{x^{2}}}  $ \hfill Απ:
        $ e^{-6} $
      }
  \end{enumerate}

\item Να υπολογιστεί το προσεγγιστικό πολυώνυμο Maclaurin 3ης τάξης, των παρακάτω συναρτήσεων.
  \begin{enumerate}[i)]
    \item $ y= \mathrm{e}^{-x} $ 
      \hfill Απ: $ \mathrm{e}^{-x} \approx 1-x+ \frac{1}{2} x^{2} - \frac{1}{6} x^{3} $ 
    \item $ y= \ln{(x+2)} $ 
      \hfill Απ: $ \ln{(x+2)} \approx \ln{2} + \frac{1}{2} x - \frac{1}{8} x^{2} +
      \frac{1}{24} x^{3} $ 
    \item $ y= \frac{1}{x-1} $ \hfill Απ: $ \frac{1}{x-1} \approx -1 -x -x^{2} - x^{3} $ 
    \item $ y= \sqrt{x+1} $ \hfill Απ: $ \sqrt{x+1} \approx 1 + \frac{1}{2} x -
      \frac{1}{8} x^{2} + \frac{1}{16} x^{3} $ 
  \end{enumerate}

\item Να υπολογιστεί το προσεγγιστικό πολυώνυμο Taylor 3ης τάξης, γύρω από 
  το σημείο $ x_{0}=1 $, των παρακάτω συναρτήσεων.
  \begin{enumerate}[i)]
    \item $ y= \ln{(x+1)} $ 
      \hfill Απ: $ \ln{(x+1)} \approx \ln{2} + \frac{1}{2} (x-1) - \frac{1}{8}
      (x-1)^{2} + \frac{1}{24} (x-1)^{3} $ 
    \item $ y= \frac{1}{x+1} $ \hfill Απ: $ \frac{1}{x+1} \approx \frac{1}{2} -
      \frac{1}{4} (x-1) + \frac{1}{3} (x-1)^{2} - \frac{1}{16} (x-1)^{3} $ 
  \end{enumerate}
\end{enumerate}


\end{document}

