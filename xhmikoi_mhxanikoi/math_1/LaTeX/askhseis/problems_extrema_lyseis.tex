\input{preamble_ask.tex}
\input{definitions_ask.tex}
\input{tikz.tex}

\input{insbox}
\pagestyle{askhseis}
\everymath{\displaystyle}
\usepackage{wrapfig}

\begin{document}

\begin{center}
  \minibox{\large \bfseries \textcolor{Col1}{Προβλήματα Ακροτάτων}}
\end{center}

\vspace{\baselineskip}

\begin{exercise}
  {\bfseries \boldmath Να προσδιοριστούν οι διαστάσεις μιας πισίνας 
    $ \SI{32}{m^{3}} $ με τετραγωνική βάση έτσι ώστε η επιφάνεια των εσωτερικών 
  τοίχων και του πυθμένα να είναι ελάχιστη.}
\end{exercise}
\begin{solution}
\item {}
  \InsertBoxR{2}{\parbox[b][\baselineskip][c]{0.30\textwidth}
    {\begin{tikzpicture}[scale=0.7]
        \node (0) at (0, 0) {};
        \node (1) at (3.5, 0) {};
        \node (2) at (5.5, 1) {};
        \node (3) at (2, 1) {};
        \node (4) at (0, 1.75) {};
        \node (5) at (3.5, 1.75) {};
        \node (6) at (5.5, 2.75) {};
        \node (7) at (2, 2.75) {};
        \draw (0.center) to node[below]{$x$} (1.center);
        \draw (1.center) to node[below]{$x$} (2.center);
        \draw[dashed] (2.center) to (3.center);
        \draw[dashed] (3.center) to (0.center);
        \draw (5.center) to (6.center);
        \draw (6.center) to (7.center);
        \draw (7.center) to (4.center);
        \draw (0.center) to (4.center);
        \draw (1.center) to node[below left,yshift=3pt]{$y$} (5.center);
        \draw (2.center) to (6.center);
        \draw[name path=a] (4.center) to (5.center);
        \path[name path=b] (3.center) to (7.center);
        \draw[dashed] [name intersections={of=a and b,by={x}}] (3.center) to (x) ;
        \draw [name intersections={of=a and b,by={x}}] (x.center) to (7.center) ;
  \end{tikzpicture}}}

  Το εμβαδό της πισίνας δίνεται από τη σχέση
  \inlineequation[eq:sur1]{E = x^{2} + 4xy}

  Ο όγκος της πισίνας είναι $V = 32 \Leftrightarrow  x^{2}y = 32$
  δηλαδή $\inlineequation[eq:vol1]{y = {32}/{x^{2}}}$.

  Άρα με αντικατάσταση της σχέσης~\eqref{eq:vol1} στη
  σχέση~\eqref{eq:sur1} έχουμε
  \begin{align*}
    E(x) = x^{2} + 4x\cdot \frac{32}{x^{2}} = x^{2} + \frac{128}{x}
  \end{align*}

  Αναζητάμε ελάχιστο για αυτή τη συνάρτηση.
  \begin{myitemize}
    \item $ E'(x) = 2x - \frac{128}{x^{2}} $ και $ E''(x) = 2 + \frac{256}{x^{3}} $
    \item $ E'(x) = 0 \Leftrightarrow 2x - \frac{128}{x^{2}} = 0
      \Leftrightarrow \frac{2x^{3} - 128}{x^{2}} = 0 \Leftrightarrow
      2x^{3} = 128 \Leftrightarrow x^{3} = 64 \Leftrightarrow$
      \inlineequation[eq:sta1]{\boxed{x = 4}}
    \item $ E''(4) = 2 + \frac{256}{64} = 6 > 0 $, επομένως πρόκειται για
      ολικό ελάχιστο.
  \end{myitemize}
  Με αντικατάσταση της σχέσης~\eqref{eq:sta1} στη σχέση~\eqref{eq:vol1} έχουμε 
  $ y = 2 $.

  Επομένως οι διαστάσεις της πισίνας πρέπει να είναι $ x = 4 $ και $ y = 2 $.
\end{solution}

\begin{exercise}
  {\bfseries \boldmath Να βρεθεί η εξίσωση της ευθείας $ (\varepsilon)
    $ που περνάει από γνωστό σημείο $ P(a,b) $ και σχηματίζει με τους άξονες
  συντεταγμένων τρίγωνο ελάχιστου εμβαδού $ (a>0,\, b>0) $.}
\end{exercise}
\begin{solution}
\item {}
  \InsertBoxR{3}{\parbox[b][\baselineskip][c]{0.30\textwidth}
    {\begin{tikzpicture}[scale=0.8]
        \draw[-latex,thick] (0,0) -- (3,0) node[right] {x};
        \draw[-latex,thick] (0,0) -- (0,3) node[left] {y};
        \draw (0,2) node[label=left:{$b-a \lambda$}] (a) {} -- 
          node[above right,midway] (p) {$ P(a,b)$} 
          (2.4,0) node[label=below:{$a-b/ \lambda$}] (b) {};
        \fill (0,2) circle[radius=1.5pt];
        \fill (2.4,0) circle[radius=1.5pt];
        \fill ($ (a)!.5!(b)$) circle[radius=1.5pt];
  \end{tikzpicture}}}

  Το εμβαδό του τριγώνου δίνεται από τη σχέση 
  \inlineequation[eq:sur2]{E = {xy}/{2}}. 

  Η εξίσωση της ευθείας που διέρχεται από το σημείο $ P(a,b) $ 

  είναι \inlineequation[eq:line2]{\varepsilon: y-b = \lambda(x-a)}

  Βρίσκουμε τα σημεία τομής της ευθείας με τους άξονες:

  Για $ x = 0 $ η σχέση~\eqref{eq:line2} δίνει $ y = b - a\lambda $.

  Για $ y = 0 $ η ίδια σχέση δίνει $ x = a - {b}/{\lambda} $.

  Με αντικατάσταση των $x$ και $y$ που μόλις βρήκαμε στη
  σχέση~\eqref{eq:sur2} έχουμε
  \begin{equation}
    E = \frac{1}{2} \abs{a - \frac{b}{\lambda}} \cdot \abs{b -
    a\lambda} \Leftrightarrow
    2E = \abs{\frac{a\lambda - b}{\lambda}} \cdot \abs{b - a\lambda}
    \Leftrightarrow
    2E \abs{\lambda} = \abs{a\lambda - b}\cdot \abs{b - a\lambda} \Leftrightarrow
    2E \abs{\lambda} = (a\lambda - b)^{2} \label{eq:sur2a}
  \end{equation}
  Διακρίνουμε τις παρακάτω δύο περιπτώσεις.
  \begin{myitemize}
    \item Αν $ \lambda > 0 $ τότε από τη σχέση~\eqref{eq:sur2a} έχουμε
      \begin{align*}
          &2E\lambda  = (a\lambda - b)^{2} \Leftrightarrow \\
          &a^{2}\lambda^{2} - 2ab\lambda + b^{2} - 2\lambda E = 0
          \Leftrightarrow \\
          &a^{2}\lambda^{2} - 2(ab + E)\lambda + b^{2} = 0 \qq{(τριώνυμο
          ως προς $\lambda$)}
      \end{align*}
      Υπολογίζουμε τη Διακρίνουσα του τριωνύμου.
      \[
        \Delta = 4(ab+E)^{2} - 4a^{2}b^{2} = 4a^{2}{b}^{2} + 8abE +
        4E^{2} - 4a^{2}b^{2} = 4E(2ab + E) 
      \]
      Θέλουμε να έχουμε πραγματικές λύσεις για το $\lambda$, οπότε
      \[
        \Delta \geq 0 \Leftrightarrow 4E(2ab+E) \geq 0
        \Leftrightarrow E\leq -2ab \qq{(απορ.) ή} E\geq 0  	
      \]
      Σε αυτή την περίπτωση η μικρότερη τιμή για το εμβαδό είναι $
      E = 0 $ και είναι η τετριμμένη περίπτωση. Από τη
      σχέση~\eqref{eq:sur2a} προκύπτει ότι το
      $\lambda = b/a $ και με αντικατάσταση στη
      σχέση~\eqref{eq:line2} έχουμε ότι $ \varepsilon: y =
      (b/a)x $ που σημαίνει ότι η ευθεία περνά από την αρχή των
      αξόνων.

    \item Αν $ \lambda < 0 $ τότε από τη σχέση~\eqref{eq:sur2a} έχουμε
      \begin{align*}
          &2E(-\lambda)  = (a\lambda - b)^{2} \Leftrightarrow \\
          &a^{2}\lambda^{2} - 2ab\lambda + b^{2} + 2\lambda E = 0
          \Leftrightarrow \\
          &a^{2}\lambda^{2} - 2(ab - E)\lambda + b^{2} = 0 \qq{(τριώνυμο
          ως προς $\lambda$)}
      \end{align*}
      Υπολογίζουμε τη Διακρίνουσα του τριωνύμου.
      \[
        \Delta = 4(ab-E)^{2} - 4a^{2}b^{2} = 4a^{2}{b}^{2} - 8abE +
        4E^{2} - 4a^{2}b^{2} = 4E(2ab - E) 
      \]
      Θέλουμε να έχουμε πραγματικές λύσεις για το $\lambda$, οπότε
      \[
        \Delta \geq 0 \Leftrightarrow 4E(2ab-E) \geq 0
        \Leftrightarrow E\leq 0 \qq{(απορ.) ή} E\geq 2ab  	
      \]
      Σε αυτή την περίπτωση η μικρότερη τιμή για το εμβαδό είναι $
      E = 2ab $. Από τη σχέση~\eqref{eq:sur2a} προκύπτει ότι το
      $\lambda = -b/a $ και με αντικατάσταση στη
      σχέση~\eqref{eq:line2} προκύπτει ότι $ bx+ay-2ab=0 $ που σημαίνει ότι η
      ευθεία περνά από την αρχή των αξόνων.

      Επομένως η εξίσωση της ζητούμενης ευθείας είναι
      \[
        \boxed{y=-\frac{b}{a}x +2b}	
      \]
  \end{myitemize}
\end{solution}


%todo να γραψω τις υπολοιπες ασκησεις
\end{document}

