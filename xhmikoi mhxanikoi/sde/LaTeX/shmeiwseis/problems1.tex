\input{preamble.tex}
\input{definitions_ask.tex}


\pagestyle{vangelis}


\begin{document}

\begin{center}
  \minibox{\large\bfseries \textcolor{Col1}{Προβλήματα σδε 1ης τάξης}}
\end{center}


\section*{Μεταφορά Θερμότητας}

Η εξίσωση που περιγράφει το ρυθμό μεταβολής της θερμοκρασίας $\theta$ ενός σώματος, που 
βρίσκεται εντός μέσου σταθερής θερμοκρασίας $ \theta_{\pi} $ είναι ανάλογος της διαφοράς 
θερμοκρασίας μεταξύ του σώματος και του περιβάλλοντος μέσου
\[
  \dv{\theta(t)}{t} = k (\theta _{\pi} - \theta(t))
\] 
όπου $ \theta _{\pi} $ είναι η θερμοκρασία του περιβάλλοντος, το $k$ ονομάζεται 
σταθερά μεταφοράς θερμότητας και $\theta$ είναι η θερμοκρασία την τυχαία 
χρονική στιγμή.

\begin{problem}
  Βρίσκεστε στο σπίτι σας, χειμώνα, και έξω έχει κρύο, σταθερή θερμοκρασία 
  $ \SI{0}{\degree C} $. Στις 10 το βράδυ κλείνετε τη θέρμανση στο σπίτι σας 
  και το θερμόμετρο δείχνει $ \SI{20}{\degree C} $. Όταν πηγαίνετε για ύπνο 
  τα μεσάνυχτα το θερμόμετρο δείχνει $ \SI{18}{\degree C} $. Να υπολογιστεί η
  θερμοκρασία του δωματίου, το επόμενο πρωί στις 8, όταν θα ξυπνήσετε. Ποια θα είναι 
  η θερμοκρασία του σπιτιού μετά από πολύ χρόνο;
\end{problem}
\begin{solution}
  Η εξίσωση που περιγράφει τη μεταφορά θερμότητας, είναι:
  \[
    \dv{\theta}{t} = k (\theta _{\pi} - \theta)
  \] 
  Η θερμοκρασία του περιβάλλοντος δίνεται ότι είναι 
  $ \SI{0}{\degree C} $, επομένως η εξίσωση γίνεται: 
  \[
    \dv{\theta}{t} = - k \theta  
  \] 
  Πρόκειται για εξίσωση χωριζομένων μεταβλητών, επομένως έχουμε:
  \[
    \frac{d \theta}{\theta} = - k dt \Rightarrow \int \frac{1}{\theta} \,{d \theta } 
    = -k \int  \,{dt} \Rightarrow \ln{\abs{\theta}} = - kt + c \Rightarrow \abs{\theta} 
    = \mathrm{e}^{-kt+c} \Rightarrow \abs{\theta} = \mathrm{e}^{c} \cdot \mathrm{e}^{-kt}
    \Rightarrow \theta = \pm \mathrm{e}^{c} \cdot \mathrm{e}^{-kt} 
  \] 
  και αν θέσουμε $ Q = \pm \mathrm{e}^{c} $, τότε προκύπτει η γενική λύση της σδε
  \[
    \theta(t) = Q \mathrm{e}^{-kt} 
  \]
  Χρειάζεται να υπολογίσουμε τις σταθερές $ Q $ και $ k $. 
  Αρχικά (στις 10 το βράδυ), για $ t=0 $, η θερμοκρασία του σπιτιού είναι 
  $ \SI{20}{\degree C} $, επομένως $ \theta (0) = 20 $. 
  \[
    \theta (0) = 20 \Rightarrow Q \mathrm{e}^{-k\cdot 0} = 20 \Rightarrow Q = 20
  \] 
  Άρα 
  \[
    \theta (t) = 20 \mathrm{e}^{-kt} 
  \] 
  Στη συνέχεια (τα μεσάνυχτα), για $ t=2 $ η θερμοκρασία του σπιτιού είναι $
  \SI{18}{\degree C} $, επομένως $ \theta (2) = 18 $.
  \[
    \theta (2) = 18 \Rightarrow 20 \mathrm{e}^{-k\cdot 2} = 18 \Rightarrow
    \mathrm{e}^{-2k} = \frac{18}{20} \Rightarrow -2k = \ln{\frac{9}{10}} \Rightarrow 
    k \approx 0,053
  \] 
  Άρα 
  \[
    \theta (t) = 20 \mathrm{e}^{-0,053 t} 
  \] 
  Το επόμενο πρωί, στις 8 (μετά από 10 ώρες), για $ t=10 $, η θερμοκρασία του σπιτιού 
  θα είναι 
  \[
    \theta (10) = 20 \mathrm{e}^{-0,053\cdot 10} = 20 \mathrm{e}^{-0,53} \approx 
    \SI{11,8}{\degree C}
  \] 
  Μετά από πολύ χρόνο, για $ t \to \infty $, έχουμε 
  \[
    \lim_{t \to \infty} \theta (t) = \lim_{t \to \infty} 20 \cdot 0 = 0
  \] 
  Δηλαδή, μετά από πολύ χρόνο, η θερμοκρασία του σπιτιού θα γίνει ίση με τη 
  θερμοκρασία του περιβάλλοντος.
\end{solution}

\begin{problem}
  Μια μεταλλική ράβδος έχει θερμοκρασία $ \SI{100}{\degree C} $ και τοποθετείται 
  σε χώρο που έχει θερμοκρασία $ \SI{0}{\degree C} $. Αν μετά από $ 20 $ λεπτά η 
  θερμοκρασία της ράβδου είναι $ \SI{50}{\degree C} $, να υπολογίσετε:
  \begin{enumerate}[i)]
    \item Το χρόνο που χρειάζεται για να φτάσει η θερμοκρασία της ράβδου τους 
      $ \SI{25}{\degree C} $ 
    \item Τη θερμοκρασία της ράβδου μετά από 10 λεπτά.
  \end{enumerate}
  Δίνεται η εξίσωση μεταφοράς θερμότητας 
  \[
    \dv{T}{t} = -k (T-T_{0}) 
  \]
\end{problem}
\begin{solution}
  Η θερμοκρασία του περιβάλλοντος δίνεται ότι είναι $ \SI{0}{\degree C} $, επομένως 
  η εξίσωση γίνεται: 
  \[
    \dv{T}{t} = -kT 
  \]
  Πρόκειται για εξίσωση χωριζομένων μεταβλητών, επομένως έχουμε:
  \[
    \frac{d T}{T} = - k dt \Rightarrow \int \frac{1}{T} \,{d T} 
    = -k \int  \,{dt} \Rightarrow \ln{\abs{T}} = - kt + c \Rightarrow \abs{T} =
    \mathrm{e}^{-kt+c} \Rightarrow \abs{T} = \mathrm{e}^{c} \cdot \mathrm{e}^{-kt}
    \Rightarrow T = \pm \mathrm{e}^{c} \cdot \mathrm{e}^{-kt} 
  \] 
  και αν θέσουμε $ Q = \pm \mathrm{e}^{c} $, τότε προκύπτει η γενική λύση της σδε
  \[
    T(t) = Q \mathrm{e}^{-kt} 
  \]
  Αρχικά (κατά την τοποθέτηση της ράβδου στο χώρο), για $ t=0 $, η θερμοκρασία 
  της ράβδου είναι $ \SI{100}{\degree C} $. 
  \[
    T(0) = 100 \Rightarrow Q \mathrm{e}^{-k\cdot 0} = 100 \Rightarrow Q = 100
  \] 
  Άρα 
  \[
    T(t) = 100 \mathrm{e}^{-kt} 
  \] 
  Στη συνέχεια (μετά από 20 λεπτά), για $ t=20 $ η θερμοκρασία της ράβδου είναι 
  $ \SI{50}{\degree C} $ επομένως $ T(20) = 50 $.
  \[
    T(20) = 50 \Rightarrow 100 \mathrm{e}^{-k\cdot 20} = 50 \Rightarrow
    \mathrm{e}^{-20k} = \frac{50}{100} \Rightarrow -20k = \ln{\frac{1}{2}} \Rightarrow 
    k \approx 0,035
  \] 
  Άρα 
  \[
    T(t) = 100 \mathrm{e}^{-0,035 t} 
  \] 
  \begin{enumerate}[i)]
    \item Για να βρούμε το χρόνο, $t_{1}$,  που χρειάζεται για να φτάσει η 
      θερμοκρασία της ράβδου τους $ \SI{25}{\degree C} $, έχουμε:
      \[
        T(t_{1}) = 25 \Rightarrow 100 \mathrm{e}^{-0,035 t_{1}} = 25 \Rightarrow 
        \mathrm{e}^{-0,035 t_{1}} = \frac{25}{100} \Rightarrow -0,035 t_{1} =
        \ln{\frac{1}{4}} \Rightarrow t_{1} \approx 39,6 
      \]
    \item Για να βρούμε τη θερμοκρασία της ράβδου μετά από 10 λεπτά, έχουμε:
      \[
        T(10) = 100 \mathrm{e}^{-0,035 \cdot 10} = 100 \mathrm{e}^{0,35} = 70,5 
      \]
  \end{enumerate}
\end{solution}

\section*{Διάσπαση Ραδιενεργών Πυρήνων}

Ο ρυθμός διάσπασης των ραδιενεργών στοιχείων, κάθε χρονική στιγμή, είναι ανάλογος της 
υπάρχουσας ποσότητας του στοιχείου
\[
  \dv{N(t)}{t} = -k Nt)
\] 
όπου $k$ είναι θετική σταθερά και ονομάζεται σταθερά διάσπασης της ουσίας. Το αρνητικό 
πρόσημο δηλώνει ότι η ποσότητα της ραδιενεργούς ουσίας μειώνεται με την πάροδο του 
χρόνου.

\begin{problem}
  Σε ένα ραδιερνεργό υλικό η ποσότητα που διασπάται (μεταστοιχειώνεται) ανά μονάδα 
  χρόνου είναι ανάλογη της ολικής ποσότητας. Αν αρχικά έχουμε $ \SI{50}{mgr} $ και 
  μετά από 2 ώϱες έχει διασπαστεί το $ 10 \% $ της αρχικής μάζας, να υπολογίσετε:
  \begin{enumerate}[i)]
    \item Την ολική μάζα την τυχαία χρονική στιγμή $t$ 
    \item Την ολική μάζα μετά από 4 ώρες
    \item Το χρόνο που απαιτείται για τη διάσπαση της μισής μάζας του υλικού (χρόνος
      ημίσειας ζωής)
  \end{enumerate}
\end{problem}
\begin{solution}
  Η διάσπαση του ραδιενεργού υλικού περιγράφεται από την εξίσωση 
  \[
    \dv{N}{t} = kN 
  \] 
  που είναι χωριζομένων μεταβλητών, οπότε έχουμε
  \[ 
    \frac{dN}{N} = k dt \Rightarrow \int \frac{1}{N} \,{dN} = k \int \,{dt} \Rightarrow 
    \ln{\abs{N}} = kt + c \Rightarrow \abs{N} = \mathrm{e}^{kt+c} \Rightarrow \abs{N} = 
    \mathrm{e}^{c} \cdot \mathrm{e}^{kt} \Rightarrow N = \pm \mathrm{e}^{c} \cdot
    \mathrm{e}^{kt}
  \]
  και αν θέσουμε $ Q = \pm \mathrm{e}^{c} $, τότε προκύπτει η γενική λύση της σδε
  \[
    N(t) = Q \mathrm{e}^{kt} 
  \]
  \begin{enumerate}[i)]
    \item Αρχικά, για $ t=0 $, η ολική μάζα είναι $ \SI{50}{mgr} $, επομένως
      \[
        N(0) = 50 \Rightarrow Q \mathrm{e}^{k\cdot 0} = 50 \Rightarrow Q = 50 
      \] 
      Άρα 
      \[
        N(t) = 50 \mathrm{e}^{kt} 
      \] 
      Στη συνέχεια (μετά από 2 ώρες) έχει διασπαστεί το $ 10 \% $ της αρχικής μάζας, 
      δηλαδή το $ 10 \% $ των $ \SI{50}{mgr} $, άρα  
      \[ \frac{10}{100} \cdot 50 = \SI{5}{mgr} \]
      Άρα η μάζα του υλικού μετά από 2 ώρες θα είναι $ 50-5= \SI{45}{mgr} $. Άρα 
      \[
        N(2) = 45 \Rightarrow 50 \mathrm{e}^{k\cdot 2} = 45 \Rightarrow 
        \mathrm{e}^{2k} = \frac{45}{50} \Rightarrow 2k = \ln{\frac{9}{10}} 
        \Rightarrow k = -0,053
      \] 
      Άρα 
      \[
        N(t) = 50 \mathrm{e}^{-0,053 t} 
      \]
    \item Για να βρούμε την ολική μάζα του υλικού μετά από 4 ώρες, έχουμε
      \[
        N(4) = 50 \mathrm{e}^{-0,053\cdot 4} = 50 \mathrm{e}^{-0,212} = \SI{40,5}{mgr}
      \] 
    \item Για να βρούμε το χρόνο που απαιτείται για να διασπαστεί η μισή μάζα του 
      υλικού, δηλαδή $ \SI{25}{mgr} $ έχουμε
      \[
        25 = 50 \mathrm{e}^{-0,053t} \Rightarrow \mathrm{e}^{-0,053t} = \frac{25}{50} 
        \Rightarrow -0,053 t = \ln{\frac{1}{2}} \Rightarrow t \approx 13 \; \text{ώρες}
      \] 
  \end{enumerate}
\end{solution}

\begin{problem}
  Έστω $ Q(t) $ η ποσότητα άνθρακα-14, που βρίσκεται σε ένα σώμα κάθε χρονική στιγμή,
  και $ Q_{0} $ η αρχική ποσότητα. Αν υποθέσουμε ότι η ποσότητα $Q$ ικανοποιεί τη 
  διαφορική εξίσωση $ Q' = -k Q $, να υπολογίσετε την σταθερά διάσπασης $k$ του 
  άνθρακα-14, αν είναι γνωστό ότι ο χρόνος ημίσειας ζωής του είναι 5730 έτη. 
  Να βρεθεί η ποσότητα άνθρακα-14 του σώματος κάθε χρονική στιγμή. 
  Αν υποθέσουμε ότι σε ένα σώμα, η περιεκτικότητα σε άνθρακα-14 που έχει απομείνει 
  είναι $ 20 \% $ της αρχικής ποσότητας, να υπολογιστεί η ηλικία αυτού του σώματος.
\end{problem}
\begin{solution}
  Η διάσπαση του άνθρακα-14 περιγράφεται από την εξίσωση 
  \[
    \dv{Q}{t} = -kQ 
  \] 
  που είναι χωριζομένων μεταβλητών, οπότε έχουμε
  \[ 
    \frac{dQ}{Q} = -k dt \Rightarrow \int \frac{1}{Q} \,{dQ} = -k \int \,{dt} 
    \Rightarrow \ln{\abs{Q}} = -kt + c \Rightarrow \abs{Q} = \mathrm{e}^{-kt+c} 
    \Rightarrow \abs{Q} = \mathrm{e}^{c} \cdot \mathrm{e}^{-kt} 
    \Rightarrow Q = \pm \mathrm{e}^{c} \cdot \mathrm{e}^{-kt}
  \]
  και αν θέσουμε $ A = \pm \mathrm{e}^{c} $, τότε προκύπτει η γενική λύση 
  \[
    Q(t) = A \mathrm{e}^{-kt} 
  \]
  Αρχικά, για $ t=0 $, η ποσότητα άνθρακα-14 είναι $Q_{0}$, επομένως
  \[
    Q(0) = Q_{0} \Rightarrow A \mathrm{e}^{-k\cdot 0} = Q_{0} \Rightarrow A = Q_{0}
  \] 
  Άρα 
  \[
    Q(t) = Q_{0}\mathrm{e}^{-kt} 
  \] 
  Ο χρόνος ημίσειας ζωής, δηλαδή ο χρόνος που απαιτείται μέχρι να διασπαστεί η 
  μισή από την αρχική ποσότητα άνθρακα-14 σε ένα σώμα, είναι 5730 έτη, άρα 
  \[
    Q(5730) = \frac{Q_{0}}{2} \Rightarrow Q_{0} \mathrm{e}^{-k\cdot 5730} = 
    \frac{Q_{0}}{2} \Rightarrow \mathrm{e}^{-k\cdot 5730} = \frac{1}{2} \Rightarrow 
    -k \cdot 5730 = \ln{\frac{1}{2}} \Rightarrow k = 0,00012097 \; \text{έτη}^{-1}
  \] 
  \[
    Q(t) = Q_{0} \mathrm{e}^{-0,00012097t} 
  \]
  Για να βρούμε την ηλικία του σώματος, $t_{1} $, όταν θα έχει απομείνει το 
  $ 20 \% $ της αρχικής ποσότητας άνθρακα-14 σε αυτό, δηλαδή 
  $ 20 \% Q_{0} = \frac{20}{100} Q_{0} = Q_{0}/5 $ 
  έχουμε
  \[
    \frac{Q_{0}}{5} = Q_{0} \mathrm{e}^{-0,00012097t_{1}} \Rightarrow 
    \mathrm{e}^{-0,00012097t_{1}} = \frac{1}{5} \Rightarrow -0,00012097t_{1} =
    \ln{\frac{1}{5}} \Rightarrow t_{1} \approx 13305 \; \text{έτη}
  \] 
\end{solution}

\section*{Λογιστικό Μοντέλο}

\subsection*{Το μαθηματικό μοντέλο Malthus}


Σύμφωνα με το μαθηματικό μοντέλο του Malthus υποθέτουμε ότι σ᾽ έναν πληθυσμό, τόσο ο 
ρυθμός των γεννήσεων, όσο και ο αριθμός των θανάτων είναι ανάλογοι του μεγέθους του 
πληθυσμού. Ακόμη υποθέτουμε ότι \textbf{ο ρυθμός αύξησης του πληθυσμού είναι ανάλογος 
  της διαφοράς μεταξύ του ρυθμού των γεννήσεων και του ρυθμού των θανάτων}. 
  Αν τη χρονική στιγμή $t$ ο πληθυσμός είναι $ y(t) $, τότε 
\[
  \dv{y(t)}{dt} = ay(t)-by(t) = ky(t) 
 \] 
 όπου $a$ και $b$ είναι σταθερές αναλογίας που σχετίζονται με το ρυθμό των γεννήσεων και 
 το ρυθμό των θανάτων αντίστοιχα.


 \begin{problem}
   Υποθέτουμε ότι ο πληθυσμός των λαγών σε ένα μικρό νησί, αυξάνεται με ρυθμό ανάλογο 
   του ήδη υπάρχοντος πληθυσμού. Αν ο πληθυσμός διπλασιάζεται μετά από 100 ημέρες να
   βρεθεί μετά από πόσες ημέρες ο πληθυσμός θα τριπλασιαστεί.
 \end{problem}
 \begin{solution}
   Η μεταβολή του πληθυσμού των λαγών $ y(t) $ στο νησί, περιγράφεται από την 
   εξίσωση Malthus 
   \[
     \dv{y(t)}{t} = ky(t) 
   \]
   η οποία είναι χωριζομένων μεταβλητών, οπότε
   \[
     \frac{dy}{y} = k dt \Rightarrow \int \frac{1}{y} \,{dy} = k \int \,{dt} \Rightarrow
     \ln{y} = k t + c \Rightarrow y = \mathrm{e}^{kt+c} \Rightarrow y = \mathrm{e}^{c} 
     \cdot \mathrm{e}^{kt}  
   \] 
   και αν θέσουμε $ Q = \mathrm{e}^{c} $, τότε προκύπτει η γενική λύση 
   \[
     y(t) = Q \mathrm{e}^{kt} 
   \]
   Αν αρχικά, για $ t=0 $ ο πληθυσμός των λαγών στο νησί είναι $ y_{0} $, τότε 
   \[
     y(0)= y_{0} \Rightarrow Q \mathrm{e}^{k \cdot 0} = y_{0} \Rightarrow Q = y_{0} 
   \] 
   οπότε έχουμε
   \[
     y(t) = y_{0} \mathrm{e}^{kt} 
   \] 
   Ο χρόνος διπλασιασμού του αρχικού πληθυσμού των λαγών είναι 100 ημέρες, άρα 
   \[
     y(100) = 2 y_{0} \Rightarrow y_{0} \mathrm{e}^{k \cdot 100} = 2 y_{0} \Rightarrow 
     \mathrm{e}^{k \cdot 100} = 2 \Rightarrow 100 k = \ln{2} \Rightarrow k 
     \approx 0.00693 
   \] 
   άρα 
   \[
     y(t) = y_{0} \mathrm{e}^{0,00693t}  
   \] 
   Ο χρόνος, $ t_{1} $ που απαιτείται ώστε να τριπλασιαστεί ο αρχικός πληθυσμός 
   των λαγών θα είναι
   \[
     3 y_{0} = y_{0} \mathrm{e}^{0,000693 t_{1}} \Rightarrow  \mathrm{e}^{0,00693 t_{1}}
     = 3 \Rightarrow 0,00693 t_{1} = \ln{3} \Rightarrow t_{1} \approx 158,5 \;
     \text{ημέρες}
   \] 
 \end{solution}

 \begin{problem}
   Σε ένα νησί, όπου δεν υπήρχαν γάτες, τα ποντίκια αυξάνονταν με ρυθμό τέτοιο ώστε 
   να διπλασιάζονται κάθε δέκα χρόνια. Έτσι το χρονικό διάστημα 2000 με 2010 έφτασαν τα
   50000. Τότε οι κάτοικοι του νησιού αποφάσισαν να πάρουν δραστικά μέτρα και έτσι 
   έφεραν στο νησί γάτες οι οποίες άρχισαν να τα εξολοθρεύουν με ρυθμό 6000 ποντίκια το 
   χρόνο. Μετά από πόσα χρόνια θα εξολοθρευτούν όλα τα ποντίκια;
 \end{problem}
 \begin{solution}
   Ο πληθυσμός των ποντικιών, $y(t)$ περιγράφεται από την εξίσωση Malthus, άρα 
   \[
     \dv{y(t)}{t} = k y(t) 
   \]
   η οποία είναι χωριζομένων μεταβλητών, οπότε
   \[
     \frac{dy}{y} = k dt \Rightarrow \int \frac{1}{y} \,{dy} = k \int \,{dt} \Rightarrow
     \ln{y} = k t + c \Rightarrow y = \mathrm{e}^{kt+c} \Rightarrow y = \mathrm{e}^{c} 
     \cdot \mathrm{e}^{kt}  
   \] 
   και αν θέσουμε $ Q = \mathrm{e}^{c} $, τότε προκύπτει η γενική λύση 
   \[
     y(t) = Q \mathrm{e}^{kt} 
   \]
   Αν αρχικά, για $ t=0 $ ο πληθυσμός των ποντικιών στο νησί είναι $ y_{0} $, τότε 
   \[
     y(0)= y_{0} \Rightarrow Q \mathrm{e}^{k \cdot 0} = y_{0} \Rightarrow Q = y_{0} 
   \] 
   οπότε έχουμε
   \[
     y(t) = y_{0} \mathrm{e}^{kt} 
   \] 
   Αν ο χρόνος διπλασιασμού του πληθυσμού των ποντικιών είναι 10 χρόνια, τότε
   \[
     y(10) = 2 y_{0} \Rightarrow y_{0} \mathrm{e}^{k \cdot 10} = 2 y_{0} \Rightarrow 
     \mathrm{e}^{k \cdot 10} = 2 \Rightarrow 10 k = \ln{2} \Rightarrow k 
     = \frac{\ln 2}{10} \approx 0.0693 
   \] 
   άρα 
   \[
     y(t) = y_{0} \mathrm{e}^{\frac{\ln{2}}{10} t}  
   \] 
   Μετά την άφιξη των γατών στο νησί, το 2010, ο πληθυσμός των ποντικιών θα 
   περιγράφεται πλέον από την εξίσωση 
   \[
     \dv{y(t)}{t} = \frac{\ln{2}}{10}y(t) - 6000 \Leftrightarrow \dv{y(t)}{t} -
     \frac{\ln{2}}{10}y(t) = -6000
   \] 
   η οποία είναι γραμμική με $ f(t) = - \frac{\ln{2}}{10} $ και $ g(t) = -6000 $, 
   οπότε έχουμε
   \[
     \mu (t) = \mathrm{e}^{- \int \frac{\ln{2}}{10} \,{dt}} = 
     \mathrm{e}^{- \frac{\ln{2}}{10} t} 
   \]
   οπότε
   \[
     \left( \mathrm{e}^{- \frac{\ln{2}}{10} t} y \right)' = 
     -6000 \mathrm{e}^{- \frac{\ln{2}}{10} t} \Rightarrow 
     \mathrm{e}^{- \frac{\ln{2}}{10} t} y = - 6000 \int \mathrm{e}^{- \frac{\ln{2}}{10}t}
     \,{dt} \Rightarrow \mathrm{e}^{- \frac{\ln{2}}{10} t} y = -6000
     \frac{\mathrm{e}^{-\frac{\ln{2}}{10} t} }{- \frac{\ln{2}}{10}} + c
   \] 
   επομένως
   \[
     y = \frac{6000}{\frac{\ln{2}}{10}} + c \mathrm{e}^{\frac{\ln{2}}{10} t} \Rightarrow 
     y(t) = \frac{60000}{\ln 2} + c \mathrm{e}^{\frac{\ln{2}}{10} t} 
   \] 

   Αν λάβουμε υπόψιν μας ότι ο πληθυσμός των ποντικιών αρχικά, το 2010, ήταν $ 50000 $, 
   τότε έχουμε:
   \[
     y(0) = 50000 \Rightarrow \frac{60000}{\ln{2}} + c \mathrm{e}^{\frac{\ln{2}}{10} 
     \cdot 0} = 50000 \Rightarrow \frac{60000}{\ln{2}} + c = 50000 \Rightarrow c =
     \frac{50000 \ln{2}- 60000}{\ln{2}} 
   \] 
   Άρα ο πληθυσμός των ποντικιών μετά το 2010 περιγράφεται από την εξίσωση
   \[
     y(t) = \frac{50000 \ln{2} - 60000}{\ln{2}} \mathrm{e}^{\frac{\ln{2}}{10} t} 
     + \frac{6000}{\ln{2}} = \frac{(50000 \ln{2} - 60000) \mathrm{e}^{t \ln{2}
     /10}+60000}{\ln{2}} 
   \] 
   Ο χρόνος $ t_{1} $ που απαιτείται ώστε οι γάτες να εξολοθρεύσουν όλα τα 
   ποντίκια είναι
   \[
     y(t) = 0 \Rightarrow 
     \frac{(50000 \ln{2} - 60000) \mathrm{e}^{t \ln{2} /10}+60000}{\ln{2}} = 0 
     \Rightarrow (50000 \ln{2} - 60000) \mathrm{e}^{t \ln{2} /10}+60000 = 0
   \] 
   άρα 
   \[
     \mathrm{e}^{t \ln{2} /10} = \frac{-60000}{(50000 \ln{2} - 60000)} \Rightarrow 
     t = \frac{10}{\ln{2}} \ln{\frac{-60000}{50000 \ln{2} - 60000}} \approx 12 \;
     \text{έτη}
    \] 
 \end{solution}



% \section*{Προβλήματα Μίξης}

% Στα προβλήματα μίξης ουσιών, ένας βασικός νόμος που διέπει τη ροή τους, είναι η 
% \textit{αρχή διατήρησης της μάζας} σύμφωνα με την οποία, ο ρυθμός μεταβολής της
% συγκέντρωσης μιας ουσίας που εισέρχεται εντός ενός κλειστού χώρου, και αναμιγνύεται 
% καλώς με τα υπάρχοντα σε αυτόν υλικά, και εξέρχεται από αυτόν με κάποια δεδομένη 
% ταχύτητα, ισούται με τη διαφορά του ρυθμού εξόδου από το ρυθμό εισόδου της ουσίας.


% \begin{center}
% \begin{tabular}{l}
% ρυθμός εισόδου $=$ συγκέντρωση εισαγόμενης ουσίας $\cdot$ ταχύτητα εισόδου \\
% ρυθμός εξόδου $=$ συγκέντρωση εξαγόμενης ουσίας $\cdot$ ταχύτητα εξόδου 
% \end{tabular}
% \end{center}
% \[
%   \dv{Q(t)}{t} = \text{ρυθμός εισόδου} - \text{ρυθμός εξόδου} 
% \] 
\end{document}
