\documentclass[a4paper,12pt]{article}
\usepackage{etex}
%%%%%%%%%%%%%%%%%%%%%%%%%%%%%%%%%%%%%%
% Babel language package
\usepackage[english,greek]{babel}
% Inputenc font encoding
\usepackage[utf8]{inputenc}
%%%%%%%%%%%%%%%%%%%%%%%%%%%%%%%%%%%%%%

%%%%% math packages %%%%%%%%%%%%%%%%%%
\usepackage{amsmath}
\usepackage{amssymb}
\usepackage{amsfonts}
\usepackage{amsthm}
\usepackage{proof}

\usepackage{physics}

%%%%%%% symbols packages %%%%%%%%%%%%%%
\usepackage{bm} %for use \bm instead \boldsymbol in math mode 
\usepackage{dsfont}
\usepackage{stmaryrd}
%%%%%%%%%%%%%%%%%%%%%%%%%%%%%%%%%%%%%%%


%%%%%% graphicx %%%%%%%%%%%%%%%%%%%%%%%
\usepackage{graphicx}
\usepackage{color}
%\usepackage{xypic}
\usepackage[all]{xy}
\usepackage{calc}
\usepackage{booktabs}
\usepackage{minibox}
%%%%%%%%%%%%%%%%%%%%%%%%%%%%%%%%%%%%%%%

\usepackage{enumerate}

\usepackage{fancyhdr}
%%%%% header and footer rule %%%%%%%%%
\setlength{\headheight}{14pt}
\renewcommand{\headrulewidth}{0pt}
\renewcommand{\footrulewidth}{0pt}
\fancypagestyle{plain}{\fancyhf{}
\fancyhead{}
\lfoot{}
\rfoot{\small \thepage}}
\fancypagestyle{vangelis}{\fancyhf{}
\rhead{\small \leftmark}
\lhead{\small }
\lfoot{}
\rfoot{\small \thepage}}
%%%%%%%%%%%%%%%%%%%%%%%%%%%%%%%%%%%%%%%

\usepackage{hyperref}
\usepackage{url}
%%%%%%% hyperref settings %%%%%%%%%%%%
\hypersetup{pdfpagemode=UseOutlines,hidelinks,
bookmarksopen=true,
pdfdisplaydoctitle=true,
pdfstartview=Fit,
unicode=true,
pdfpagelayout=OneColumn,
}
%%%%%%%%%%%%%%%%%%%%%%%%%%%%%%%%%%%%%%

\usepackage[space]{grffile}

\usepackage{geometry}
\geometry{left=25.63mm,right=25.63mm,top=36.25mm,bottom=36.25mm,footskip=24.16mm,headsep=24.16mm}

%\usepackage[explicit]{titlesec}
%%%%%% titlesec settings %%%%%%%%%%%%%
%\titleformat{\chapter}[block]{\LARGE\sc\bfseries}{\thechapter.}{1ex}{#1}
%\titlespacing*{\chapter}{0cm}{0cm}{36pt}[0ex]
%\titleformat{\section}[block]{\Large\bfseries}{\thesection.}{1ex}{#1}
%\titlespacing*{\section}{0cm}{34.56pt}{17.28pt}[0ex]
%\titleformat{\subsection}[block]{\large\bfseries{\thesubsection.}{1ex}{#1}
%\titlespacing*{\subsection}{0pt}{28.80pt}{14.40pt}[0ex]
%%%%%%%%%%%%%%%%%%%%%%%%%%%%%%%%%%%%%%

%%%%%%%%% My Theorems %%%%%%%%%%%%%%%%%%
\newtheorem{thm}{Θεώρημα}[section]
\newtheorem{cor}[thm]{Πόρισμα}
\newtheorem{lem}[thm]{λήμμα}
\theoremstyle{definition}
\newtheorem{dfn}{Ορισμός}[section]
\newtheorem{dfns}[dfn]{Ορισμοί}
\theoremstyle{remark}
\newtheorem{remark}{Παρατήρηση}[section]
\newtheorem{remarks}[remark]{Παρατηρήσεις}
%%%%%%%%%%%%%%%%%%%%%%%%%%%%%%%%%%%%%%%




\newcommand{\vect}[2]{(#1_1,\ldots, #1_#2)}
%%%%%%% nesting newcommands $$$$$$$$$$$$$$$$$$$
\newcommand{\function}[1]{\newcommand{\nvec}[2]{#1(##1_1,\ldots, ##1_##2)}}

\newcommand{\linode}[2]{#1_n(x)#2^{(n)}+#1_{n-1}(x)#2^{(n-1)}+\cdots +#1_0(x)#2=g(x)}

\newcommand{\vecoffun}[3]{#1_0(#2),\ldots ,#1_#3(#2)}

\newcommand{\mysum}[1]{\sum_{n=#1}^{\infty}


\everymath{\displaystyle}

\thispagestyle{empty}



\begin{document}

\begin{center}
    \fbox{\large\bfseries Θέματα}
\end{center}

\vspace{\baselineskip}

\begin{enumerate}
    \item Έστω το γραμμικό υπόδειγμα ενός αγαθού σε μια μεμονωμένη αγορά της μορφής
        \[
        \sysdelim.\}\systeme*{Q_{d}= \alpha - \beta P,Q_{s}=- \gamma + \delta P} 
        \] 
        Ποια από τις παρακάτω εκφράσεις είναι σωστή;
        \begin{enumerate}[i)]
            \item Η μείωση του $\alpha$ προκαλεί μείωση της τιμής ισορροπίας επειδή $\beta$,
                $\delta > 0$ 
            \item Η μείωση του $\alpha$ προκαλεί αύξηση της τιμής ισορροπίας επειδή $\beta$,
                $\delta > 0$
        \end{enumerate}

    \item Έστω το γραμμικό υπόδειγμα ενός αγαθού σε μια μεμονωμένη αγορά της μορφής
        \[
        \sysdelim.\}\systeme*{Q_{d}= \alpha - \beta P,Q_{s}=- \gamma + \delta P} 
        \] 
        Ποια από τις παρακάτω εκφράσεις είναι σωστή;
        \begin{enumerate}[i)]
            \item Η αύξηση  του $\beta$ προκαλεί αύξηση της τιμής ισορροπίας επειδή $\alpha$,
                $\gamma > 0$ 
            \item Η αύξηση  του $\beta$ προκαλεί μείωση της τιμής ισορροπίας επειδή $\alpha$,
                $\gamma > 0$
        \end{enumerate}

    \item Έστω η συνάρτηση $ f(x) = (x-3)^{3} $. Στο σημείο $ x_{0} = 3 $ η συνάρτηση παρουσιάζει:

        \begin{inparaenum}[i)]
        \item τοπικό μέγιστο \quad
        \item τοπικό ελάχιστο \quad \item σημείο καμπής
        \end{inparaenum}

    \item Έστω η συνάρτηση $ f(x) = x^{3} - 9x^{2}+27x-27 $. Στο σημείο $ x_{0} = 3 $ η συνάρτηση
        παρουσιάζει:

        \begin{inparaenum}[i)]
        \item τοπικό μέγιστο \quad
        \item τοπικό ελάχιστο \quad \item σημείο καμπής
        \end{inparaenum}

    \item Έστω η συνάρτηση $ f(x) = x^{3}+3px+5 $. Ποια από τις παρακάτω προτάσεις είναι λάθος;

        \begin{enumerate}[i)]
            \item Αν $ p>0 $ η συνάρτηση δεν παρουσιάζει τοπικά ακρότατα
            \item Αν $ p>0 $ η συνάρτηση παρουσιάζει τοπικά ακρότατα
        \end{enumerate}

    \item Έστω η συνάρτηση $ f(x) = x^{n}-nx+k $, όπου $ n>0 $ και $ n \neq 1 $. Ποια από τις
        παρακάτω προτάσεις είναι σωστή;

        \begin{enumerate}[i)]
            \item Αν $ n>1 $ τότε η συνάρτηση έχει τοπικό ελάχιστο στο 1
            \item Αν $ n>1 $ τότε η συνάρτηση έχει τοπικό ελάχιστο στο 1
        \end{enumerate}


\end{enumerate}



\end{document}
