\documentclass[a4paper,table]{report}
\documentclass[a4paper,12pt]{article}
\usepackage{etex}
%%%%%%%%%%%%%%%%%%%%%%%%%%%%%%%%%%%%%%
% Babel language package
\usepackage[english,greek]{babel}
% Inputenc font encoding
\usepackage[utf8]{inputenc}
%%%%%%%%%%%%%%%%%%%%%%%%%%%%%%%%%%%%%%

%%%%% math packages %%%%%%%%%%%%%%%%%%
\usepackage{amsmath}
\usepackage{amssymb}
\usepackage{amsfonts}
\usepackage{amsthm}
\usepackage{proof}

\usepackage{physics}

%%%%%%% symbols packages %%%%%%%%%%%%%%
\usepackage{bm} %for use \bm instead \boldsymbol in math mode 
\usepackage{dsfont}
\usepackage{stmaryrd}
%%%%%%%%%%%%%%%%%%%%%%%%%%%%%%%%%%%%%%%


%%%%%% graphicx %%%%%%%%%%%%%%%%%%%%%%%
\usepackage{graphicx}
\usepackage{color}
%\usepackage{xypic}
\usepackage[all]{xy}
\usepackage{calc}
\usepackage{booktabs}
\usepackage{minibox}
%%%%%%%%%%%%%%%%%%%%%%%%%%%%%%%%%%%%%%%

\usepackage{enumerate}

\usepackage{fancyhdr}
%%%%% header and footer rule %%%%%%%%%
\setlength{\headheight}{14pt}
\renewcommand{\headrulewidth}{0pt}
\renewcommand{\footrulewidth}{0pt}
\fancypagestyle{plain}{\fancyhf{}
\fancyhead{}
\lfoot{}
\rfoot{\small \thepage}}
\fancypagestyle{vangelis}{\fancyhf{}
\rhead{\small \leftmark}
\lhead{\small }
\lfoot{}
\rfoot{\small \thepage}}
%%%%%%%%%%%%%%%%%%%%%%%%%%%%%%%%%%%%%%%

\usepackage{hyperref}
\usepackage{url}
%%%%%%% hyperref settings %%%%%%%%%%%%
\hypersetup{pdfpagemode=UseOutlines,hidelinks,
bookmarksopen=true,
pdfdisplaydoctitle=true,
pdfstartview=Fit,
unicode=true,
pdfpagelayout=OneColumn,
}
%%%%%%%%%%%%%%%%%%%%%%%%%%%%%%%%%%%%%%

\usepackage[space]{grffile}

\usepackage{geometry}
\geometry{left=25.63mm,right=25.63mm,top=36.25mm,bottom=36.25mm,footskip=24.16mm,headsep=24.16mm}

%\usepackage[explicit]{titlesec}
%%%%%% titlesec settings %%%%%%%%%%%%%
%\titleformat{\chapter}[block]{\LARGE\sc\bfseries}{\thechapter.}{1ex}{#1}
%\titlespacing*{\chapter}{0cm}{0cm}{36pt}[0ex]
%\titleformat{\section}[block]{\Large\bfseries}{\thesection.}{1ex}{#1}
%\titlespacing*{\section}{0cm}{34.56pt}{17.28pt}[0ex]
%\titleformat{\subsection}[block]{\large\bfseries{\thesubsection.}{1ex}{#1}
%\titlespacing*{\subsection}{0pt}{28.80pt}{14.40pt}[0ex]
%%%%%%%%%%%%%%%%%%%%%%%%%%%%%%%%%%%%%%

%%%%%%%%% My Theorems %%%%%%%%%%%%%%%%%%
\newtheorem{thm}{Θεώρημα}[section]
\newtheorem{cor}[thm]{Πόρισμα}
\newtheorem{lem}[thm]{λήμμα}
\theoremstyle{definition}
\newtheorem{dfn}{Ορισμός}[section]
\newtheorem{dfns}[dfn]{Ορισμοί}
\theoremstyle{remark}
\newtheorem{remark}{Παρατήρηση}[section]
\newtheorem{remarks}[remark]{Παρατηρήσεις}
%%%%%%%%%%%%%%%%%%%%%%%%%%%%%%%%%%%%%%%




\newcommand{\vect}[2]{(#1_1,\ldots, #1_#2)}
%%%%%%% nesting newcommands $$$$$$$$$$$$$$$$$$$
\newcommand{\function}[1]{\newcommand{\nvec}[2]{#1(##1_1,\ldots, ##1_##2)}}

\newcommand{\linode}[2]{#1_n(x)#2^{(n)}+#1_{n-1}(x)#2^{(n-1)}+\cdots +#1_0(x)#2=g(x)}

\newcommand{\vecoffun}[3]{#1_0(#2),\ldots ,#1_#3(#2)}

\newcommand{\mysum}[1]{\sum_{n=#1}^{\infty}



\pagestyle{askhseis}
\everymath{\displaystyle}
% \geometry{top=2cm}

\begin{document}

\begin{center}
  \minibox{\large\bfseries \textcolor{Col1}{Ασκήσεις Ολοκληρώματα}}
\end{center}

\vspace{\baselineskip} 


\section*{Αρμονικές Συναρτήσεις}

\begin{enumerate}

  \item Δίνεται η συνάρτηση $ u(x,y) = x^{2} +4x-y^{2}+2y $. 
    \begin{enumerate}[i)]
      \item Να δείξετε ότι η συνάρτηση $ u $ είναι αρμονική.
      \item Να βρεθεί η συνάρτηση $ v(x,y) $ τέτοια ώστε η μιγαδική συνάρτηση 
        $ f(z) = u(x,y) + iv(x,y) $ να είναι αναλυτική. 
      \item Να γραφεί η αντίστοιχη συνάρτηση $ f(z) $.
    \end{enumerate}

    \hfill Απ: $ f(z) = z^{2}+4z-2iz+a $ 

  \item Δίνεται η συνάρτηση $ u(x,y) = \ln{\sqrt{x^{2}+y^{2}}} $ που είναι ορισμένη 
    στο σύνολο $ \mathbb{R}^{2}- \{ 0,0 \} $. 
    \begin{enumerate}[i)]
      \item Να δείξετε ότι η συνάρτηση $ u $ είναι αρμονική.
      \item Να βρεθεί μια συζυγής αρμονική συνάρτηση της $u$.
      \item Να γραφεί η αντίστοιχη ολόμορφη συνάρτηση.
    \end{enumerate}

    \hfill Απ: $ f(z) = \ln{\sqrt{x^{2}+y^{2}}} + i(\arctan{\frac{y}{x}} + c)  $ 

  \item Δίνεται η συνάρτηση $ u(x,y) = 2x-x^{3}+axy^{2} $. 
    \begin{enumerate}[i)]
      \item Να βρεθεί για ποια τιμή του $a$, η συνάρτηση είναι αρμονική. 
      \item Για αυτήν την τιμή του $a$ να βρεθεί η συζυγής αρμονική συνάρτηση της $u$.
    \end{enumerate} 

    \hfill Απ: $ a=3 $, $ v(x,y) = 2y-3x^{2}y+y^{3}+c $ 

\end{enumerate}


\section*{Ολοκληρώματα}

\begin{enumerate}
\item Να υπολογιστούν τα παρακάτω ολοκληρώματα.

  \begin{enumerate}[i)]
    % spand ex 8 p.151
    \item $ \int \limits_c\frac{e^{z}}{z-2} \,{dz} $, \quad όπου 
      \begin{enumerate*}[i),itemjoin=\hspace{\baselineskip}]
        \item $ c: \abs{z} = 3 $
        \item $ c: \abs{z} = 1 $. 
      \end{enumerate*}
      \hfill Απ: 
      \begin{enumerate*}[i),itemjoin=\hspace{\baselineskip}]
        \item $ 2 \pi i e^{2} $ 
        \item $0$ 
      \end{enumerate*}
      % spand ex 9 p.153
    \item $ \int\limits_c \frac{1}{z^{2}+1} \,{dz} $, όπου $ c:\abs{z} = 2 $.
      \hfill Απ: 0 
      % spand ex 8 p.278
    \item $ \int \limits_{c}\frac{1}{z^{3}(z+4)} \,{dz} $, \quad όπου $c:\abs{z}=2 $ 
      \hfill Απ: $ \frac{2 \pi i}{4^{3}} $  
      % spand ex 1 p.145
    \item $\int\limits_c\frac{5z^2-2}{z(e^z-1)}\,dz$, \quad όπου $c:\abs{z}=2$ 
      \hfill Απ: $2\pi i$
      % spand ex 4 p.254 
    \item $\int\limits_c\tan z\,dz$, \quad όπου $c:\abs{z}=2$ \hfill Απ: $-4\pi i$
      % χρειάζεται να έχω πει ολοκληρωτικό τύπο Cauchy αλλιως παραγωγίσεις δυσκολες
      % \item $\int\limits_c\frac{2z-3}{z(z-1)^2(z^2+4)}\,dz$, \quad οπου $c:
      % \abs{z}=3$   \hfill Απ: 0
  \end{enumerate}

\pagebreak

\item Να υπολογιστούν τα παρακάτω ολοκληρώματα.

  \begin{enumerate}[i)]
    \item $\int\limits_0^{2\pi}\frac{1}{2+\cos x}\,dx$ \hfill Απ:$\frac{2
      \pi}{\sqrt{3}}$
    \item $\int\limits_0^{2\pi}\frac{1}{(5-3\sin x)^2}\,dx$ \hfill Απ: $\frac{5}{32}\pi$
    \item $\int\limits_0^{2\pi}\frac{\cos 3x}{5-4\cos x}\,dx$ 
      \hfill Απ: $\frac{\pi}{12}$
    \item $\int\limits_0^{2\pi}\frac{1}{3-2\cos x+\sin x}\,dx$ \hfill Απ: $\pi$
      % \item $\int\limits_0^{2\pi}\frac{1}{1-2a\cos x+a^2}\,dx$ 
      %   \hfill Απ: \begin{tabular}{c}$\abs{a}<1\Rightarrow I=\frac{2\pi}{1-a^2}$ \\ 
      %   $\abs{a}>1\Rightarrow I=\frac{2\pi}{a^2-1}$\end{tabular}
  \end{enumerate}

\item Να υπολογιστούν τα παρακάτω ολοκληρώματα.

  \begin{enumerate}[i)]
    \item $\int_{-\infty}^{+\infty}\frac{x\sin x}{1+x^2}\,dx$ 
      \hfill  Απ: $\frac{\pi}{e}$
    \item $\int_{-\infty}^{+\infty}\frac{\cos 2x}{(1+x^2)^2}\,dx$ 
      \hfill  Απ: $\frac{3\pi}{2e^2}$
  \end{enumerate}
\end{enumerate}

\end{document}


