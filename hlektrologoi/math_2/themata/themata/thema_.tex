\documentclass[a4paper,table]{report}
\input{preamble_themata}
\newcommand{\vect}[2]{(#1_1,\ldots, #1_#2)}
%%%%%%% nesting newcommands $$$$$$$$$$$$$$$$$$$
\newcommand{\function}[1]{\newcommand{\nvec}[2]{#1(##1_1,\ldots, ##1_##2)}}

\newcommand{\linode}[2]{#1_n(x)#2^{(n)}+#1_{n-1}(x)#2^{(n-1)}+\cdots +#1_0(x)#2=g(x)}

\newcommand{\vecoffun}[3]{#1_0(#2),\ldots ,#1_#3(#2)}

\newcommand{\suma}{\sum_{n=0}^{\infty}a_n x^n}

\newcommand{\sumb}{\sum_{n=1}^{\infty}a_n n x^{n-1}}

\newcommand{\sumc}{\sum_{n=2}^{\infty}a_n n (n-1) x^{n-2}}

\newcommand{\varsum}[2]{\sum_{n=#1}^{#2}}
\input{myboxes}

\everymath{\displaystyle}
\pagestyle{vangelis}


\begin{document}

\begin{mybox3}
  \begin{thema}
    Έστω $ F(x,y,z) = -36(x^{2}+z^{2}) + (x^{2}+y^{2}+z^{2}+5)^2 $.
    \begin{enumerate}[i)]
      \item Να βρεθούν τα τοπικά ακρότατα της συνάρτησης $ z=f(x,y) $ η οποία ορίζεται σε
        πεπλεγμένη μορφή από την εξίσωση $ F(x,y,z)=0 $.
      \item Θεωρούμε το σημείο $ A(1,0,2 \sqrt{6}) $.
        \begin{enumerate}
          \item Να δείξετε ότι υπάρχει μοναδική $ C^{1} $ τάξης συνάρτηση $ z=f(x,y) $ 
            κοντά στο $ (1,0) $, η οποία ορίζεται σε πεπλεγμένη μορφή από την εξίσωση $
            F(x,y,z)=0 $, τέτοια ώστε $ f(1,0) = 2 \sqrt{6} $.
          \item Εφαρμόζοντας το ανάπτυγμα Taylor, να προσεγγίσετε κοντά στο $ (1,0) $ 
            τη συνάρτηση $f$ με ένα πολυώνυμο πρώτου βαθμού.
        \end{enumerate}
      \item Θεωρούμε την καμπύλη του $ \mathbb{R}^{3} $ που δίνεται από τις παραμετρικές
        εξισώσεις:
        \[
          (c): \begin{rcases}
            x = \sqrt{t} \\
            y= \sqrt{4-t} \\
            z= \sqrt{9-t}
          \end{rcases}
        \] 
        Αν $B$ είναι το σημείο τομής της επιφάνειας $ F(x,y,z)=0 $ με την καμπύλη $ (c)
        $, να βρείτε το εφαπτόμενο επίπεδο αυτής της επιφάνειας στο σημείο $ B $.
    \end{enumerate}
  \end{thema}
\end{mybox3}

\end{document}
