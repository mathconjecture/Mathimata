\documentclass[a4paper,12pt]{article}
\usepackage{etex}
%%%%%%%%%%%%%%%%%%%%%%%%%%%%%%%%%%%%%%
% Babel language package
\usepackage[english,greek]{babel}
% Inputenc font encoding
\usepackage[utf8]{inputenc}
%%%%%%%%%%%%%%%%%%%%%%%%%%%%%%%%%%%%%%

%%%%% math packages %%%%%%%%%%%%%%%%%%
\usepackage{amsmath}
\usepackage{amssymb}
\usepackage{amsfonts}
\usepackage{amsthm}
\usepackage{proof}

\usepackage{physics}

%%%%%%% symbols packages %%%%%%%%%%%%%%
\usepackage{bm} %for use \bm instead \boldsymbol in math mode 
\usepackage{dsfont}
\usepackage{stmaryrd}
%%%%%%%%%%%%%%%%%%%%%%%%%%%%%%%%%%%%%%%


%%%%%% graphicx %%%%%%%%%%%%%%%%%%%%%%%
\usepackage{graphicx}
\usepackage{color}
%\usepackage{xypic}
\usepackage[all]{xy}
\usepackage{calc}
\usepackage{booktabs}
\usepackage{minibox}
%%%%%%%%%%%%%%%%%%%%%%%%%%%%%%%%%%%%%%%

\usepackage{enumerate}

\usepackage{fancyhdr}
%%%%% header and footer rule %%%%%%%%%
\setlength{\headheight}{14pt}
\renewcommand{\headrulewidth}{0pt}
\renewcommand{\footrulewidth}{0pt}
\fancypagestyle{plain}{\fancyhf{}
\fancyhead{}
\lfoot{}
\rfoot{\small \thepage}}
\fancypagestyle{vangelis}{\fancyhf{}
\rhead{\small \leftmark}
\lhead{\small }
\lfoot{}
\rfoot{\small \thepage}}
%%%%%%%%%%%%%%%%%%%%%%%%%%%%%%%%%%%%%%%

\usepackage{hyperref}
\usepackage{url}
%%%%%%% hyperref settings %%%%%%%%%%%%
\hypersetup{pdfpagemode=UseOutlines,hidelinks,
bookmarksopen=true,
pdfdisplaydoctitle=true,
pdfstartview=Fit,
unicode=true,
pdfpagelayout=OneColumn,
}
%%%%%%%%%%%%%%%%%%%%%%%%%%%%%%%%%%%%%%

\usepackage[space]{grffile}

\usepackage{geometry}
\geometry{left=25.63mm,right=25.63mm,top=36.25mm,bottom=36.25mm,footskip=24.16mm,headsep=24.16mm}

%\usepackage[explicit]{titlesec}
%%%%%% titlesec settings %%%%%%%%%%%%%
%\titleformat{\chapter}[block]{\LARGE\sc\bfseries}{\thechapter.}{1ex}{#1}
%\titlespacing*{\chapter}{0cm}{0cm}{36pt}[0ex]
%\titleformat{\section}[block]{\Large\bfseries}{\thesection.}{1ex}{#1}
%\titlespacing*{\section}{0cm}{34.56pt}{17.28pt}[0ex]
%\titleformat{\subsection}[block]{\large\bfseries{\thesubsection.}{1ex}{#1}
%\titlespacing*{\subsection}{0pt}{28.80pt}{14.40pt}[0ex]
%%%%%%%%%%%%%%%%%%%%%%%%%%%%%%%%%%%%%%

%%%%%%%%% My Theorems %%%%%%%%%%%%%%%%%%
\newtheorem{thm}{Θεώρημα}[section]
\newtheorem{cor}[thm]{Πόρισμα}
\newtheorem{lem}[thm]{λήμμα}
\theoremstyle{definition}
\newtheorem{dfn}{Ορισμός}[section]
\newtheorem{dfns}[dfn]{Ορισμοί}
\theoremstyle{remark}
\newtheorem{remark}{Παρατήρηση}[section]
\newtheorem{remarks}[remark]{Παρατηρήσεις}
%%%%%%%%%%%%%%%%%%%%%%%%%%%%%%%%%%%%%%%




\input{definitions_ask.tex}


\pagestyle{askhseis}
\everymath{\displaystyle}

\begin{document}

\begin{center}
  \minibox[c]{\large\bf \textcolor{Col1}{Ασκήσεις Διπλό Ολοκλήρωμα} \\
  \textcolor{Col1}{(Αλλαγή Μεταβλητών)}}
\end{center}

\vspace{\baselineskip}

\begin{enumerate}
  \item Να υπολογίσετε τα παρακάτω διπλά ολοκληρώματα κάνοντας χρήση των πολικών 
    και ελλειπτικών συντεταγμένων.
    \begin{enumerate}[i)]
      \item $I=\iint_{D}e^{x^2+y^2}\,dxdy$, 
        \quad $D=\{(x,y)\in\mathbb{R}^2 \mid x^2+y^2\leq a^2,\; a>0\}$ 
        \hfill Απ: $\pi(e^{a^2}-1)$ %spandagos p.73 ex.1
      \item $I=\iint_{D}x\,dxdy$, \quad $ D= \{(x,y)\in \mathbb{R}^{2} \mid x^{2}+y^{2} 
        \leq 4,\; x^{2}+y^{2} \geq 1,\; x \geq 0,\; y \geq 0 \} $ 
        \hfill Απ: $ \frac{7}{3} $ %spandagos p.73 ex.2
      \item $I=\iint_{D}\sin(x^2+y^2)\,dxdy$, \quad $D= \{(x,y)\in \mathbb{R}^{2} 
        \mid x^{2}+y^{2} \leq a^{2} \}$ 
        \hfill Απ: $\pi (1 - \cos{a^{2}}) $
      \item $I=\iint_{D}(x^2+y^2)\,dxdy$, \quad $ D= \{(x,y)\in \mathbb{R}^{2} 
        \mid (x-a)^{2}+y^{2} \leq a^{2},\; a>0 \}  $ 
        \hfill Απ: $\frac{3}{2}a^4\pi$
      \item $I=\iint_{D}\sqrt{x^2+y^2}\,dxdy$, \quad $ D= \{(x,y)\in \mathbb{R}^{2} 
        \mid 2x \leq x^{2}+y^{2} \leq 4,\, x \geq 0  \} $ 
        \hfill Απ: $\frac{8}{3}\Bigl(\pi-\frac{4}{3}\Bigr)$
      \item $ I=\iint_{D} \sqrt{ 1-x^{2}-y^{2} } \,dxdy $, 
        \quad $ D= \{(x,y)\in \mathbb{R}^{2} \mid x^{2}+y^{2} \leq x\} $ 
        \hfill Απ:  $ \frac{3\pi -4}{9} $ %spandagos p.141 ask.66
      \item $ I=\iint_{D} \sqrt{ 16-x^{2}-y^{2}}\,dxdy  $, \quad $ D= \{(x,y)\in
        \mathbb{R}^{2} \mid x^{2}+y^{2} \leq 4y \} $ \hfill Απ: $ \frac{64}{3} 
        \Bigl(\pi - \frac{4}{3}\Bigr) $  
      \item $ I=\iint_{D} \arctan(\frac{y}{x})\,dxdy $, 
        \quad $ D= \{(x,y)\in \mathbb{R}^{2} \mid x^{2}+y^{2} \leq 4,\, x \geq 1,\, y 
        \geq 0 \} $ 
        \hfill Απ: $ \frac{\pi ^{2}}{9} - \frac{\pi \sqrt{3}}{6} - 
        \frac{1}{2} \ln{\frac{1}{2} } $ %spandagos p.144 ask 69
    \end{enumerate}

  \item Να υπολογιστούν τα παρακάτω διπλά ολοκληρώματα κάνοντας χρήση 
    κατάλληλων μετασχηματισμών.
    \begin{enumerate}[i)]
      \item $I=\iint_{D}e^{\frac{y}{x+y}}\,dxdy$, \quad $ D= \{(x,y)\in \mathbb{R}^{2} 
        \mid x+y \leq 1,\; x \geq 0,\; y \geq 0 \}$ \hfill Απ: $\frac{1}{2}(e-1)$
        %spandagos p.157 ask.84
      \item $ I= \iint_{D}e^{(x+y)^{2}}\,dxdy $, \quad $ D = \{ (x,y) \in \mathbb{R}^{2}
        \mid x+y \leq 2,\; y \leq x,\; y \geq 0 \} $ \hfill Απ: $ \frac{1}{4} (e^{4}-1)$ 
      \item $ I=\iint_{D}(x^{2}+y^{2})\,dxdy $, \quad $ D= \{(x,y)\in \mathbb{R}^{2} 
        \mid 2 \leq xy \leq 4,\; 1 \leq x^{2}-y^{2} \leq 9 \} $ \hfill Απ: 8 
        %spandagos p.154 ask.81
      \item $I=\iint_{D}\frac{dxdy}{(x+y)^2(x-y)^2}$, \quad $D$ τρίγωνο με κορυφές τα
        σημεία $(1,0), (3,0), (2,1)$  \hfill Απ: $\frac{1}{9}$ %spand p.152 ask.79
      \item $ I=\iint_{D} \frac{dxdy}{1+(x-y)^{2}} $, \quad $D$ τρίγωνο με κορυφές τα 
        σημεία $(0,0), (1,0), (1,1)$ \hfill Απ: $ \frac{\pi}{4} - \ln{\sqrt{ 2 }} $ 
        %spandagos p.158 ask.86 (θετω u=y-x, v=x D^*=(0,0), (-1,1), (0,1))
      \item $ I=\iint_{D} \cos{\left(\frac{x-y}{x+y}\right)},dxdy $, \quad 
        $ D= \{(x,y)\in \mathbb{R}^{2} \mid x+y \leq 1,\; x \geq 0,\; y \geq 0 \} $ 
        \hfill Απ: $ \frac{\sin 1}{2} $ %spandagos p.158 ask.86
      \item $ I=\iint_{D}(x+y)^{2} \sin^{2}{(x-y)}\,dxdy $, \quad $ D $ τετράγωνο 
        με κορυφές $ (0,1) $, $ (1,2) $, $ (2,1) $, $ (1,0) $
        \hfill Απ: $ \frac{13}{6} (2 - \sin{2}) $ 
    \end{enumerate}

    \section*{Εφαρμογές του διπλού ολοκληρώματος} 

  \item Να υπολογιστεί ο \textbf{όγκος} του στερεού που περικλείεται από το 
    παραβολοειδές με εξίσωση $ x^{2}+y^{2}+z=4 $ και από το επίπεδο $ z=0 $.
    \hfill Απ: $ 8\pi $ 

  \item Να υπολογιστεί ο \textbf{όγκος} του στερεού που περικλείεται από τις 
    επιφάνειες $ x^{2}+y^{2}=1 $ και $ y+z=2 $.
    \hfill Απ: $ 2 \pi $ 

  \item Να υπολογιστεί ο \textbf{όγκος} του στερεού που περικλείεται από τις 
    επιφάνειες $ z=x^{2}+y^{2} $ και $ x^{2}+y^{2}=2x $.
    \hfill Απ: $ \frac{3 \pi}{2} $ 

  \item Να υπολογιστεί ο \textbf{όγκος} του στερεού που περικλείεται από το 
    παραβολοειδές $ z = x^{2}+y^{2} $ και τον κώνο με εξίσωση $ z^{2}=x^{2}+y^{2} $.
    \hfill Απ: $ \pi/6 $ %spandagos p.188 ask.125

  \item Να υπολογιστεί ο \textbf{όγκος} του στερεού που περικλείεται πάνω από τον 
    παραβολικό κύλινδρο $ z=4-y^{2} $ και κάτω από το ελλειπτικό παραβολοειδές 
    $ z=x^{2}+3y^{2} $.  
    \hfill Απ: $ 4 \pi $  

  \item Να υπολογιστεί ο \textbf{όγκος} της μπάλας $ x^{2}+y^{2}+z^{2}=\rho^{2} $.
    \hfill Απ: $ \frac{4}{3} \pi \rho^{3} $ 

  \item Να υπολογιστεί ο \textbf{όγκος} του ελλειψοειδούς 
    $ \frac{x^{2}}{a^{2}} + \frac{y^{2}}{b^{2}} + \frac{z^{2}}{c^{2}} =1 $, με  
    $ a,b,c > 0 $.
    \hfill Απ: $ \frac{4}{3} \pi abc $ 

  \item Να υπολογιστεί το \textbf{εμβαδό} του τμήματος της κωνικής επιφάνειας 
    $ 3z^{2}=x^{2}+y^{2} $ που βρίσκεται εντός του κυλίνδρου 
    $ 4y=x^{2}+y^{2} $, $ z \geq 0 $. \hfill Απ: $ 8 \pi \sqrt{3}/{3} $ 
    %spandagos p. 209 ask. 152

  \item Να υπολογιστεί το \textbf{εμβαδό} του τμήματος της κωνικής επιφάνειας 
    $ x^{2}=y^{2}+z^{2} $ που βρίσκεται εντός του κυλίνδρου 
    $ =x^{2}+y^{2} = a^{2} $, $ x \geq 0,\; a > 0 $. 
    \hfill Απ: $ \pi a^{2}$ %spandagos p. 209 ask. 152

  \item Να υπολογιστεί το \textbf{εμβαδό} του τμήματος της επιφάνειας της σφαίρας
    $x^{2} + y^{2}+z^{2}=25 $ που βρίσκεται μεταξύ των επιπέδων $ z=2 $ και $ z=4 $.
    \hfill Απ: $ 20 \pi $.  %spandagos p. 212 ask. 155
\end{enumerate}


\vspace{\baselineskip}

\begin{center}
  \minibox{\large\bf \textcolor{Col1}{Υποδείξεις}}
\end{center}

\vspace{\baselineskip}

\begin{enumerate}
  \item Ο τύπος του εμβαδού τμήματος επιφάνειας $S$ πάνω σε επιφάνεια $ z=f(x,y) $ 
    είναι: 
    \[
      E_{S}=\iint\limits_{D} \sqrt{1 + (f_{x})^{2}+(f_{y})^{2}} \,dxdy
    \]
\end{enumerate}



\end{document}
