\documentclass[a4paper,table]{report}
\input{preamble_ask.tex}
\input{definitions_ask.tex}

\geometry{top=2cm}

\pagestyle{vang}
\everymath{\displaystyle}


\begin{document}

\begin{center}
  \minibox{\large\bfseries \textcolor{Col1}{Ασκήσεις στα Όρια}}
\end{center}


\vspace{\baselineskip}

\setcounter{chapter}{1}

\begin{enumerate}
  \item Να υπολογιστούν τα όρια των παρακάτω συναρτήσεων.

    \twocolumnside{
      \begin{enumerate}[i)]
        \item $\lim_{x\to -\infty}x^2(x+1)(3-x)$ \hfill Απ: $-\infty$
        \item $\lim_{x\to \infty}\frac{x+1}{x^2+3}$ \hfill Απ: $0$
        \item $\lim_{x\to -\infty}\frac{x+7}{x-7}$ \hfill Απ: $1$
      \end{enumerate}
      }{
      \begin{enumerate}[i),start=5]
        \item $\lim_{x\to -\infty}\left(\frac{1-x^3}{x^2-7x}\right)^5$ 
          \hfill Απ: $\infty$
        \item $\lim_{x\to \infty}\sqrt{\frac{8x^2-3}{2x^2 +3}}$ \hfill Απ: $2$
        \item $\lim_{x\to \infty}\frac{2x^3+7}{x^3-x^2+x+7}$ \hfill Απ: $2$
        % \item $\lim_{x\to \infty}\cos x$ \hfill Απ: δεν υπάρχει
        % \item $\lim_{x\to \left(\frac{\pi}{2}\right)^-}\tan x$ \hfill Απ: $\infty$
      \end{enumerate}
    }

  \item Να υπολογιστούν τα παρακάτω όρια, εξετάζοντας το \textbf{πρόσημο του 
    παρονομαστή}, κοντά στο $ x_{0} $.

    \twocolumnside{
      \begin{enumerate}[i)]
        \item $\lim_{x\to 0^+}\frac{1}{3x}$ \hfill Απ: $\infty$
        \item $\lim_{x\to 0^-}\frac{5}{2x}$ \hfill Απ: $-\infty$
        \item $\lim_{x\to 2^-}\frac{3}{x-2}$ \hfill Απ: $-\infty$
        \item $\lim_{x\to -5^-}\frac{3x}{2x+10}$ \hfill Απ: $\infty$
        \item $\lim_{x\to -8^+}\frac{2x}{x+8}$ \hfill Απ: $-\infty$
        \item $\lim_{x\to 7}\frac{4}{(x-7)^2}$ \hfill Απ: $\infty$
        % \item $\lim_{x\to 0}\frac{-1}{x^2(x+1)}$ \hfill Απ: $-\infty$
        \item $\lim_{x\to 0^+}\frac{2}{3x^{\frac{1}{3}}}$ \hfill Απ: $\infty$
      \end{enumerate}
      }{
      \begin{enumerate}[i),start=8]
        \item $\lim_{x\to 0^-}\frac{2}{x^{\frac{1}{5}}}$ \hfill Απ: $-\infty$
        \item $\lim_{x\to 2^+}\frac{1}{x^2-4}$ \hfill Απ: $\infty$
        \item $\lim_{x\to 2^-}\frac{1}{x^2-4}$ \hfill Απ: $-\infty$
        \item $\lim_{x\to -2^+}\frac{1}{x^2-4}$ \hfill Απ: $-\infty$
        \item $\lim_{x\to -2^-}\frac{1}{x^2-4}$ \hfill Απ: $\infty$
        \item $\lim_{x\to -2^+}\frac{x^2-1}{2x+4}$ \hfill Απ: $\infty$
        \item $\lim_{x\to -2^-}\frac{x^2-1}{2x+4}$ \hfill Απ: $-\infty$
      \end{enumerate}
    }

  \item Να υπολογιστούν τα όρια των παρακάτω \textbf{ρητών} συναρτήσεων.

    \twocolumnside{
    \begin{enumerate}[i)]
      \item $\lim_{x\to \infty}\frac{1}{3x+4}$ \hfill Απ: $0$
      \item $\lim_{x\to -\infty}\frac{8x^4+3x-5}{x^2+1}$ \hfill Απ: $\infty$
    \end{enumerate}
    }{
    \begin{enumerate}[i),start=3]
      \item $\lim_{x\to \infty}\frac{3x+4}{x-7}$ \hfill Απ: $3$
      \item $\lim_{x\to \infty}\frac{5x^4+5}{(x^2-1)(2x^3+3)}$ \hfill Απ: $0$
      % \item $\lim_{x\to 0}\frac{x^{2}-4x+4}{x^3+5x^2-14x}$ \hfill Απ: δεν υπάρχει
    \end{enumerate}
  }

  \item Να υπολογιστούν τα παρακάτω όρια με \textbf{άρση της απροσδιοριστίας} 
    και χρήση του κανόνα L' Hospital.

    \twocolumnside{
      \begin{enumerate}[i)]
        \item $\lim_{x\to \infty}\frac{\ln x}{x^2}$ \hfill Απ: $0$
        \item $\lim_{x\to (\frac{\pi}{2})^-}\frac{\cos x}{1-\sin x}$ \hfill Απ: $\infty$
        \item $\lim_{x\to 0^+}\frac{\ln(e^x-1)}{\ln x}$ \hfill Απ: $1$
        \item $\lim_{x\to 0}\frac{1-\cos x }{\tan x}$ \hfill Απ: $0$
        \item $ \lim_{x \to 2^{+}} (x-2)^{\ln{\frac{x}{2}}} $ \hfill Απ: $1$ 
      \end{enumerate}
    }{
    \begin{enumerate}[i),start=6]
      \item $\lim_{x\to \infty}x^2e^{-x}$ \hfill Απ: $0$
      \item $\lim_{x\to 0^+}x\ln x$ \hfill Απ: $0$
      \item $\lim_{x\to \infty}x\sin(\frac{1}{x})$ \hfill Απ: $1$
      \item $\lim_{x\to \infty}(\sqrt{x} - x)$ \hfill Απ: $-\infty$
      \item $\lim_{x\to \infty}x^{\frac{1}{x^2}}$ \hfill Απ: $1$
      % \item $\lim_{x\to 0^+}(\cos x)^{\frac{2}{x}}$ \hfill Απ: $1$
      \item $ \lim_{t\to 1} \left(\frac{t}{t-1} - \frac{1}{\ln{t}}\right) $
        \hfill Απ: $ \frac{1}{2} $ 
    \end{enumerate}
  }
\end{enumerate}


\end{document}


