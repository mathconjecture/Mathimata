\documentclass[a4paper,12pt]{article}
\usepackage{etex}
%%%%%%%%%%%%%%%%%%%%%%%%%%%%%%%%%%%%%%
% Babel language package
\usepackage[english,greek]{babel}
% Inputenc font encoding
\usepackage[utf8]{inputenc}
%%%%%%%%%%%%%%%%%%%%%%%%%%%%%%%%%%%%%%

%%%%% math packages %%%%%%%%%%%%%%%%%%
\usepackage{amsmath}
\usepackage{amssymb}
\usepackage{amsfonts}
\usepackage{amsthm}
\usepackage{proof}

\usepackage{physics}

%%%%%%% symbols packages %%%%%%%%%%%%%%
\usepackage{bm} %for use \bm instead \boldsymbol in math mode 
\usepackage{dsfont}
\usepackage{stmaryrd}
%%%%%%%%%%%%%%%%%%%%%%%%%%%%%%%%%%%%%%%


%%%%%% graphicx %%%%%%%%%%%%%%%%%%%%%%%
\usepackage{graphicx}
\usepackage{color}
%\usepackage{xypic}
\usepackage[all]{xy}
\usepackage{calc}
\usepackage{booktabs}
\usepackage{minibox}
%%%%%%%%%%%%%%%%%%%%%%%%%%%%%%%%%%%%%%%

\usepackage{enumerate}

\usepackage{fancyhdr}
%%%%% header and footer rule %%%%%%%%%
\setlength{\headheight}{14pt}
\renewcommand{\headrulewidth}{0pt}
\renewcommand{\footrulewidth}{0pt}
\fancypagestyle{plain}{\fancyhf{}
\fancyhead{}
\lfoot{}
\rfoot{\small \thepage}}
\fancypagestyle{vangelis}{\fancyhf{}
\rhead{\small \leftmark}
\lhead{\small }
\lfoot{}
\rfoot{\small \thepage}}
%%%%%%%%%%%%%%%%%%%%%%%%%%%%%%%%%%%%%%%

\usepackage{hyperref}
\usepackage{url}
%%%%%%% hyperref settings %%%%%%%%%%%%
\hypersetup{pdfpagemode=UseOutlines,hidelinks,
bookmarksopen=true,
pdfdisplaydoctitle=true,
pdfstartview=Fit,
unicode=true,
pdfpagelayout=OneColumn,
}
%%%%%%%%%%%%%%%%%%%%%%%%%%%%%%%%%%%%%%

\usepackage[space]{grffile}

\usepackage{geometry}
\geometry{left=25.63mm,right=25.63mm,top=36.25mm,bottom=36.25mm,footskip=24.16mm,headsep=24.16mm}

%\usepackage[explicit]{titlesec}
%%%%%% titlesec settings %%%%%%%%%%%%%
%\titleformat{\chapter}[block]{\LARGE\sc\bfseries}{\thechapter.}{1ex}{#1}
%\titlespacing*{\chapter}{0cm}{0cm}{36pt}[0ex]
%\titleformat{\section}[block]{\Large\bfseries}{\thesection.}{1ex}{#1}
%\titlespacing*{\section}{0cm}{34.56pt}{17.28pt}[0ex]
%\titleformat{\subsection}[block]{\large\bfseries{\thesubsection.}{1ex}{#1}
%\titlespacing*{\subsection}{0pt}{28.80pt}{14.40pt}[0ex]
%%%%%%%%%%%%%%%%%%%%%%%%%%%%%%%%%%%%%%

%%%%%%%%% My Theorems %%%%%%%%%%%%%%%%%%
\newtheorem{thm}{Θεώρημα}[section]
\newtheorem{cor}[thm]{Πόρισμα}
\newtheorem{lem}[thm]{λήμμα}
\theoremstyle{definition}
\newtheorem{dfn}{Ορισμός}[section]
\newtheorem{dfns}[dfn]{Ορισμοί}
\theoremstyle{remark}
\newtheorem{remark}{Παρατήρηση}[section]
\newtheorem{remarks}[remark]{Παρατηρήσεις}
%%%%%%%%%%%%%%%%%%%%%%%%%%%%%%%%%%%%%%%




\newcommand{\vect}[2]{(#1_1,\ldots, #1_#2)}
%%%%%%% nesting newcommands $$$$$$$$$$$$$$$$$$$
\newcommand{\function}[1]{\newcommand{\nvec}[2]{#1(##1_1,\ldots, ##1_##2)}}

\newcommand{\linode}[2]{#1_n(x)#2^{(n)}+#1_{n-1}(x)#2^{(n-1)}+\cdots +#1_0(x)#2=g(x)}

\newcommand{\vecoffun}[3]{#1_0(#2),\ldots ,#1_#3(#2)}

\newcommand{\mysum}[1]{\sum_{n=#1}^{\infty}




\begin{document}

\begin{center}
	\fbox{\Large \bfseries Ασκήσεις στις Παραγώγους}
\end{center}

\vspace{\baselineskip}

\begin{enumerate}

	\item Να υπολογιστεί η ποσότητα $ \frac{1}{\sqrt[3]{1,01}} $ με τη βοήθεια
		του διωνυμικού αναπτύγματος με ακρίβεια έξι δεκαδικών ψηφίων.

		\hfill Απ: 0,996689

	\item Να εξεταστεί πλήρως ως προς τη συνέχεια και την παραγωγισιμότητα η
		συνάρτηση $ f(x) = e^{\abs{x}} $.

		\hfill Απ: συνεχής, όχι παραγωγίσιμη 

	\item Να βρεθούν τα $ a, b \in \mathbb{R} $ έτσι ώστε η συνάρτηση 
		\[
			f(x) = \begin{cases}
				x^{2}, & x\geq 2 \\
				ax+b , & x<2
			\end{cases}
		\]
		να είναι παραγωγίσιμη στο $ x_{0} = 2 $.

		\hfill Απ: $ a=4, b=-4 $

	\item Δίνεται η συνάρτηση $ f(x) \colon \mathbb{R} \to \mathbb{R} $ τέτοια
		ώστε $ x - x^{2} \leq f(x) \leq x + x^{3} $, $ \forall x \in \mathbb{R}
		$. Να υπολογίσετε την $ f'(0) $.

		\hfill Απ: $ f'(0) = 1 $

	\item Έστω η συνάρτηση $ f \colon \mathbb{R} \to \mathbb{R} $ με $ f(0)
		\neq 0	$ τέτοια ώστε $ f(x+y) = f(x) \cdot f(y) $, $ \forall x \in
		\mathbb{R} $. Αν η $f$ είναι παραγωγίσιμη στο $0$, τότε να βρεθεί η $
		f'(x) $, $ \forall x \in \mathbb{R} $.

		\hfill Απ: $ f'(x) = f'(0)\cdot f(x) $

	\item Αν ο πραγματικός αριθμός $p$ είναι η ρίζα ενός πολυωνύμου $ P(x)
		$ και της παραγώγου του $ P'(x) $ τότε να δείξετε ότι το $p$ είναι διπλή
		ρίζα του $ P(x) $ και αντίστροφα.

	\item Να δείξετε ότι για το πολυώνυμο $n$ βαθμού $ P_{n}(x) =a_{n}x^{n} +
		a_{n-1}x^{n-1} + \cdots + a_{1}x + a_{0} $, με  $ a_{n}\neq 0 $ ότι ισχύει $
		P_{n}^{(n)}(x) = n! \cdot a_{n}$, για  $n\geq 1 $.

	\item Να υπολογιστούν οι παράγωγοι των παρακάτω συναρτήσεων
		\begin{enumerate}[(i)]
			\item $ f(x) = \ln{\sqrt[5]{1+3x^{2}}} $ \hfill Απ: $
				\frac{6x}{5(1+3x^{2})} $
			\item $ f(x) = \ln({\sin({\cos{x}})}) $ \hfill Απ: $
				\frac{1}{\sin{(\cos{x})}} [\cos{(\cos{x})}] (- \sin{x}) $ 
			\item $ f(x) = \arctan \frac{x}{\sqrt{1 + x^{2}}} $ \hfill Απ: $
			\frac{1}{(1+2x^{2})\sqrt{1 + x^{2}}} $
			\item $ f(x) = \ln{(e^{\sin{x}})} + \sqrt{x^{2} - 25x} $ \hfill Απ: $
				\cos{x} + \frac{2x - 25}{2 \sqrt{x^{2} - 25x}}  $  
		\end{enumerate}

	\item  Να υπολογιστούν οι παράγωγοι των παρακάτω συναρτήσεων

	\begin{enumerate}[(i)]
		\item $ f(x) = (\cos{x})^{\sin{2x}} $ \hfill Απ: $
			(\cos{x})^{\sin{2x}} 2(\cos{2x} \ln{(\cos{x})} - \sin^{2}{x}) $
			\item $ f(x) = \left(1 + \frac{1}{x} \right)^{x} $ \hfill Απ: $
				\left(1 + \frac{1}{x}\right)^{x}\left[\ln{(1 + \frac{1}{x})} -
				\frac{1}{x+1}\right] $
			\item $ (\sin{x})^{x} $ \hfill Απ: $ (\sin{x})^{x}[\ln{(\sin{x}
				)} + x \cot{x}] $ 
			\item $ \cos{x}^{x} $ \hfill Απ: $ (- \sin{x^{x}})x^{x} (1 +
				\ln{x}) $
	\end{enumerate}

	\item Να βρεθούν οι παράγωγοι τών αντιστρόφων, των παρακάτω συναρτήσεων.
		\begin{enumerate}[(i)]
			\item $ y = \cos{x} $ \hfill Απ: $ \frac{-1}{\sqrt{1 - y^{2}}} $
			\item $ y = \tan{x} $ \hfill Απ: $ \frac{1}{1 + y^{2}} $
			\item $ y = \cosh{x} $  \hfill Απ: $ \frac{1}{\sqrt{y^{2} - 1}} $
			\item $ y = \tanh{x} $ \hfill Απ: $ \frac{1}{x^{2} - 1} $
		\end{enumerate}

	\item Να βρεθεί η παράγωγος της συνάρτησης $ y= \left[(\sin{x}) \cdot
		x^{2}\right]^{(25)}$, χρησιμοποιώντας τον τύπο \textlatin{Leinbiz}.

		\hfill Απ: $ y' = (x^{2} - 600) \cos{x} + 50 x \sin{x} $

	\item Να βρείτα τα $ a, b \in \mathbb{R} $ έτσι ώστε η ευθεία $ y = 2x + 5
		$ να είναι εφαπτομένη της συνάρτησης $ f(x) = x^{2} + ax + b $ στο
		σημείο $ x_{0} = -1 $. 

		\hfill Απ: $ a = 4, b = 6 $

	\item Δίνεται η σχέση $ x^{2} + xy + y^{3} -2x + 3y = 0 $, $ y=y(x) $. Να βρεθεί η
		2η παράγωγος της $y$ στο $ x=0 $. 
		
		\hfill Απ: $ y'(0) = \frac{2}{3} $, $ y''(0) = -\frac{10}{9} $ 

	\item Δίνεται η σχέση $ x^{2} - xy + y^{2} = 3 $, $ y=y(x) $. Να βρεθεί η 1η
		και η 2η παράγωγος της $y$ ως προς $x$ στο σημείο $ (1,-1) $.

		\hfill Απ: $ y' = 1$, $ y'' = \frac{2}{3} $

	\item Δίνεται η σχέση $ 4x^{3} - 3xy^{2} + 6x^{2} - 5xy - 8 y^{2} + 9x + 14
		= 0$. Να βρείτε τις εξισώσεις της εφαπτομένης και της κάθετης ευθείας
		της καμπύλης στο σημείο $ (-2,3) $.

		\hfill Απ: $\varepsilon\colon y = \frac{9}{2} x - 6 $, $\kappa\colon y = \frac{2}{9} x +
		\frac{31}{9} $.

	\item Δίνονται οι παραμετρικές εξισώσεις $ x = 3 \cos^{3}{t} $, $ y = 4
		\sin^{3}{t}	$. Να βρεθεί η 1η και η 2η παράγωγος της συνάρτησης $y$.

		\hfill Απ: $ y' = -\frac{4}{3} \tan{t} $, $ y'' = \frac{4}{27}
		\frac{1}{\cos^{4}t \sin{t}} $ 

	\item {\bfseries (Σεπ 2017)}
		\begin{enumerate}[i)]
			\item Να δοθεί ο ορισμός καθώς και η γεωμετρική
				ερμηνεία του διαφορικού πρώτης τάξης της συνάρτησης $ y = g(x) $ στο
				τυχαίο σημείο $x$. 
			\item Να βρεθεί το διαφορικό δεύτερης τάξης της σύνθετης συνάρτησης $ z(x) =
				f(u(x))	$.
		\end{enumerate}

	\item Να υπολογιστούν κατά προσέγγιση οι τιμές με τη βοήθεια του διαφορικού:
		\begin{enumerate}[i)]
			\item $\sqrt{50}$ \hfill Απ: $7+\frac{1}{14}$
			\item $\sqrt[4]{17}$ \hfill Απ: $\frac{1}{4}17^{-\frac{3}{4}}+2$
		\end{enumerate}

	\item Να υπολογιστούν τα παρακάτω όρια.
		\begin{enumerate}[(i)]
			\item $ \lim_{x\to 1} \left(\frac{1}{\ln{x}} - \frac{1}{x-1}\right) $ \hfill
				Απ: $ \frac{1}{2} $
			\item $ \lim_{x\to 1} \left[(1-x) \tan{\frac{\pi x}{2}}\right] $ \hfill Απ: $
				\frac{2}{\pi} $
			\item $ \lim_{x\to \frac{\pi}{4}} \frac{\sqrt{2} - \sin{x} -
				\cos{x}}{\ln{(\sin{2x})}} $ \hfill Απ: $ - \frac{\sqrt{2}}{4} $
			\item $ \lim_{x\to +\infty} \left(1 + \frac{1}{x} +
				\frac{2}{x^{2}}\right)^{x} $ \hfill Απ: $ e $ 
			\item $ \lim_{x\to 0^{+}} \left(\frac{1}{x}\right)^{\sin{x}} $ \hfill $ 1 $
			\item $ \lim_{x\to 0} \left(\cos{2x}\right)^{\frac{3}{x^{2}}}  $ \hfill Απ:
				$ e^{-6} $
		\end{enumerate}

	\item {\bfseries (Ιαν 2018)} Να αποδείξετε ότι η εξίσωση $ x^{2} = x \sin{x} + \cos{x} $ έχει δύο ακριβώς
		πραγματικές ρίζες $ x_{1} $, $ x_{2} $, με $ x_{1} \in (-\pi, 0) $, και
		$x_{2} \in (0, \pi) $.

	\item Να αποδείξετε την παρακάτω ανισότητα   
		\[
			\frac{a - b}{\cos^{2}{b}} \leq \tan{a} - \tan{b}\leq \frac{a -
			b}{\cos^{2}{a}}, \qq{με}  0 < b \leq a < \frac{\pi}{2}
		\]

	\item Να αποδείξετε την ανισότητα 
		\[
			\frac{a-b}{a} \leq \ln{\frac{a}{b}} \leq \frac{a-b}{b}, \qq{με}  0<b\leq a 
		\]

	\item{\bfseries (Ιαν 2016)} Να βρείτε μια πολυωνυμική προσέγγιση μέχρι και όρους 3ης τάξης της
		συνάρτησης που ορίζεται πεπλεγμένα από την εξίσωση $ x^{2} - xy + y^{2}
		= 3$ στο σημείο $ (1,-1) $.

		\hfill Απ: $f(x) \cong -1 + (x-1) + \frac{(x-1)^{2}}{3} +
		\frac{(x-1){3}}{9}$

	\item Να δειχθεί ότι η συνάρτηση $ f(x) = \frac{x^{2} + 6x + 12}{x^{2} - 6x
		+ 12} $ είναι καλή προσέγγιση της συνάρτησης $ e^{x} $ για μικρές τιμές
		του $x$ διότι τα αναπτύγματα των δύο συναρτήσεων συμπίπτουν στους 5
		πρώτους όρους. 

		\hfill Απ: $ f(x) = e^{x} \cong 1 + x + \frac{x^{2}}{2} +
		\frac{x^{3}}{6} + \frac{x^{4}}{24} $

	\item Να δείξετε ότι 
		\[
			\sin{x} \cong \sin{a} + \cos{a} (x-a) - \frac{\sin{a}}{2!} (x-a)^{2} -
			\frac{\cos{\xi} (x-a)^{3}}{3!}
		\]

		όπου $\xi$ μεταξύ $a$ και $x$. Στη συνέχεια να υπολογίσετε το $
		\sin{\ang{51}}$ καθώς και το διαπραττόμενο σφάλμα.
		\hfill Απ: $ \abs{R_{3}(\ang{51})} < 0,00019 $

	\item Έστω $ f(x) = \ln{(1+x)} $, $ x>-1 $. Να υπολογιστεί το ανάπτυγμα
		\textlatin{Maclaurin} μέχρι και όρους 3ης τάξης και στη συνέχεια να
		υπολογιστεί το αντίστοιχο σφάλμα για $ x = 0,5 $.

		\hfill Απ: \begin{tabular}{l}
			$ \ln(1+x) \cong x - \frac{1}{2} x^{2} + \frac{1}{3}x^{3} $ \\
			$ \abs{R_{4}(0,5)} < 0,15625 $	
		\end{tabular}

	\item Να βρεθεί η ακτίνα καμπυλότητας στο τυχαίο σημείο $\theta$ της
		καμπύλης που περιγράφεται από τις εξισώσεις 
		\begin{align*}
			x &= a(\theta - \sin{\theta}) \\
			y &= a(1 - \cos{\theta})
		\end{align*}		

		\hfill Απ: $ \rho = 4a \abs{\sin{\frac{\theta}{2}}} $
\end{enumerate}


\end{document}
