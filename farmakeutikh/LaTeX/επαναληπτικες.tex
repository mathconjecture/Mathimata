\documentclass[a4paper,12pt]{article}

\usepackage[english,greek]{babel}
\usepackage[utf8]{inputenc}

\usepackage{amsmath}
\usepackage{amssymb}
\usepackage{amsthm}
\usepackage{amsfonts}

\usepackage[margin=2cm]{geometry}


\begin{document}

\begin{center}
\fbox{\Large \bfseries Επαναληπτικές Ασκήσεις}
\end{center}

\vspace{\baselineskip}

Αν η περιοχή δεν είναι ορθογώνια και τα ολοκληρώματα δεν δίνονται έτοιμα, με τα άκρα τους αλλά θα πρέπει να τα φτιάξεις, όπως τα ολοκληρώματα $6$ και $7$ θα πρέπει να ξέρεις τους παρακάτω τύπους: (ο Μπουντουριδης τα χει κανει, δεν τα εβαλε βεβαια στις εξετασεις, αλλα ειναι στην υλη) 

\begin{enumerate}

\item 
Αν $D = \{(x,y)\mid a\leq x\leq b, g_1(x)\leq y \leq g_2(x)\}$, τότε

\[
\iint\limits_D f(x,y)dA = \int\limits_{a}^{b}\int\limits_{g_1(x)}^{g_2(x)}f(x,y)dydx 
\]

\item 
Αν $D = \{(x,y)\mid y\leq c\leq d, h_1(y)\leq x \leq h_2(y)\}$, τότε

\[
\iint\limits_D f(x,y)dA = \int\limits_{c}^{d}\int\limits_{h_1(y)}^{h_2(y)}f(x,y)dxdy 
\]
\end{enumerate}

\end{document}