\documentclass[a4paper,table]{report}
\input{preamble_themata}
\newcommand{\vect}[2]{(#1_1,\ldots, #1_#2)}
%%%%%%% nesting newcommands $$$$$$$$$$$$$$$$$$$
\newcommand{\function}[1]{\newcommand{\nvec}[2]{#1(##1_1,\ldots, ##1_##2)}}

\newcommand{\linode}[2]{#1_n(x)#2^{(n)}+#1_{n-1}(x)#2^{(n-1)}+\cdots +#1_0(x)#2=g(x)}

\newcommand{\vecoffun}[3]{#1_0(#2),\ldots ,#1_#3(#2)}

\newcommand{\mysum}[1]{\sum_{n=#1}^{\infty}

\input{myboxes}

\everymath{\displaystyle}
\pagestyle{vangelis}


\begin{document}
\begin{mybox3}
  \begin{thema}
    Θεωρούμε μεταλλικό κώνο ύψους $h$, ακτίνας βάσεως $a$ και γωνία κορυφής ίση με 
    $ 2 \theta $. Η κωνική επιφάνεια περιγράφεται από την εξίσωση 
    \[
      z = \frac{h}{a} \Bigl(a - \sqrt{x^{2}+y^{2}}\Bigr) 
    \] 
    Ο κώνος βρίσκεται μέσα σε ρεύμα θερμού αέρα και η θερμοκρασία $T$ στο περιβάλλον 
    του, ακολουθεί την κατανομή
    \[
      T(x,y,z) = \frac{T_{0}}{1+z^{3}} \, \mathrm{e}^{a^{2}-(x^{2}+y^{2})}
    \]
    όπου $ T_{0} $ είναι η σταθερή θερμοκρασία στην 
    περιφέρεια της βάσης. Να υπολογιστεί η παράγωγος της θερμοκρασίας ως προς τη 
    διεύθυνση του κάθετου διανύσματος στην επιφάνεια του κώνου.
  \end{thema}
\end{mybox3}
\begin{solution}
  Θεωρούμε την εξίσωση της επιφάνειας του κώνου:
  \[
    F(x,y,z) = az - h\Bigl(a- \sqrt{x^{2}+y^{2}}\Bigr) = 0
  \] 
  Ενα διάνυσμα κάθετο στην επιφάνεια του κώνου δίνεται απο διάνυσμα:
  \begin{align*}
    \grad F &= \dv{F}{x} \, \mathbf{i} + \dv{F}{y} \, \mathbf{j} + \dv{F}{z} \, \mathbf{k}
    = \frac{xh}{\sqrt{x^{2}+y^{2}}} \. \mathbf{i} + \frac{yh}{\sqrt{x^{2}+y^{2}}} \,
    \mathbf{j} + a \, \mathbf{k} \\
    \norm{\grad F} &= \sqrt{\frac{x^{2}h^{2}}{x^{2}+y^{2}} +
    \frac{y^{2}h^{2}}{x^{2}+y^{2}} + a^{2}} = \sqrt{h^{2}+a^{2}}
    \end{align*} 
    Επομένως 
    \[
      \widehat{\mathbf{n}} = \frac{\grad F}{\norm{\grad F}} = \frac{1}{\sqrt{a^{2}+h^{2}}}
      \biggl(\frac{xh}{\sqrt{x^{2}+y^{2}}} \, \mathbf{i} + \frac{yh}{\sqrt{x^{2}+y^{2}}} \,
      \mathbf{j} + a \, \mathbf{k}\biggr)
    \] 
    Βρίσκουμε επίσης την κλίση της θερμοκρασίας:
    \begin{align*}
      \grad T &= \dv{T}{x} \, \mathbf{i} + \dv{T}{y} \, \mathbf{j} + \dv{T}{z} \, \mathbf{k}
      = \frac{-2x T_{0}}{1+z^{3}} \, \mathrm{e}^{a^{2}-(x^{2}+y^{2})} \, \mathbf{i} + 
      \frac{-2y T_{0}}{1+z^{3}} \, \mathrm{e}^{a^{2}-(x^{2}+y^{2})} \, \mathbf{j} + 
      \frac{-3z^{2} T_{0}}{(1+z^{3})^{2}} \, \mathrm{e}^{a^{2}-(x^{2}+y^{2})} \,
      \mathbf{k} \\
              &= \frac{T_{0}}{1+z^{3}} \mathrm{e}^{a^{2}-(x^{2}+y^{2})}
              (-2x,-2y,\frac{-3z^{2}}{1+z^{3}})
    \end{align*}
    Άρα 
    \[
      \dv{T}{\mathbf{n}} = \grad T \cdot \widehat{\mathbf{n}} = 
    \] 
    %να το συνεχισω και ισως να το διορθωσω (ισως η παραγωγος του Τ ειναι συνθετη)
  \end{solution}
\end{document}
