\input{preamble_ask.tex}
\newcommand{\vect}[2]{(#1_1,\ldots, #1_#2)}
%%%%%%% nesting newcommands $$$$$$$$$$$$$$$$$$$
\newcommand{\function}[1]{\newcommand{\nvec}[2]{#1(##1_1,\ldots, ##1_##2)}}

\newcommand{\linode}[2]{#1_n(x)#2^{(n)}+#1_{n-1}(x)#2^{(n-1)}+\cdots +#1_0(x)#2=g(x)}

\newcommand{\vecoffun}[3]{#1_0(#2),\ldots ,#1_#3(#2)}

\newcommand{\mysum}[1]{\sum_{n=#1}^{\infty}



% \DeclareMathOperator{\r}{r}

\begin{document}

\chapter*{Ιδιοτιμές - Ιδιοδιανύσματα}

\begin{dfn}
  Έστω $ A \in \textbf{M}_{n}(\mathbb{K}) $. Ένας αριθμός $ \lambda \in \mathbb{K} $, 
  λέγεται \textcolor{Col1}{ιδιοτιμή} (ή χαρακτηριστική τιμή) του πίνκα $A$, 
  αν υπάρχει ένα \textbf{μη-μηδενικό} διάνυσμα $ \mathbf{u} \in \mathbb{K}^{n} $, 
  τέτοιο ώστε 
  \[
    A \mathbf{u} = \lambda \mathbf{u}  
  \] 
  Στην περίπτωση αυτή το διάνυσμα $ \mathbf{u} $ λέγεται \textcolor{Col1}{ιδιοδιάνυσμα} 
  (ή χαρακτηριστικό διάνυσμα) του $A$ που αντιστοιχεί στην ιδιοτιμή $\lambda$.
\end{dfn}

\begin{dfn}
  Το σύνολο $ V(\lambda) = \{ \mathbf{u} \in \mathbb{K}^{n} \; : \; A \mathbf{u}= \lambda
  \mathbf{u} \} $, των ιδιοδιανυσμάτων του $A$ που αντιστοιχούν στην ιδιοτιμή 
  $\lambda$, λέγεται \textcolor{Col1}{ιδιόχωρος} του $A$ που αντιστοιχεί στην ιδιοτιμή 
  $ \lambda $. 
\end{dfn}

\begin{thm}\label{thm:char}
  Έστω $ A \in \textbf{M}_{n}(\mathbb{K}) $ και $ \lambda \in \mathbb{K} $. Τότε το 
  $ \lambda $ είναι ιδιοτιμή του $A$ αν και μόνον αν $ \det(A- \lambda I_{n}) = 0 $.
\end{thm}

\begin{rem}
  Έστω $ A = (a_{ij}) \in \textbf{M}_{n}(\mathbb{K}) $. 
  \begin{myitemize}
    \item Από το θεώρημα~\ref{thm:char} βλέπουμε ότι οι ιδιοτιμές του $A$ είναι 
      ακριβώς οι λύσεις της εξίσωσης $ \det(A-xI_{n}) = 0 $, η οποία λέγεται 
      \textcolor{Col1}{χαρακτηριστική εξίσωση} του $A$, και πιο αναλυστικά είναι η
      \[
        \begin{vmatrix*}[c]
          a_{11}-x & a_{12} & \cdots & a_{1n} \\
          a_{21} & a_{22}-x & \cdots & a_{2n} \\
          \vdots & \vdots & \ddots & \vdots \\
          a_{n1} & a_{n2} & \cdots & a_{nn} - x
        \end{vmatrix*} = 0
      \] 
    \item Αν $\lambda$ είναι μια ιδιοτιμή του πίνακα $A$, τότε τα ιδιοδιασνύσματα 
      του $A$ που αντιστοιχούν στην ιδιοτιμή $\lambda$ είναι ακριβώς οι
      \textcolor{Col1}{μη-μηδενικές} λύσεις του ομογενούς συστήματος 
      $ (A- \lambda I_{n}) = \mathbf{0} $, δηλασή του συστήματος 
      \[
        \begin{pmatrix*}[c]
          a_{11}-x & a_{12} & \cdots & a_{1n} \\
          a_{21} & a_{22}-x & \cdots & a_{2n} \\
          \vdots & \vdots & \ddots & \vdots \\
          a_{n1} & a_{n2} & \cdots & a_{nn} - x
        \end{pmatrix*} \cdot 
        \begin{pmatrix*}[c]
          x_{1} \\
          x_{2} \\
          \vdots \\
          x_{n}
        \end{pmatrix*} = 
        \begin{pmatrix*}[r]
          0 \\ 0 \\ \vdots \\ 0
        \end{pmatrix*}
      \]
    \item Η ορίζουσα $ \det(A-xI_{n}) $ είναι ένα πολυώνυμο μιας μεταβλητής, της $x$, 
      βαθμού $n$, (με συντελεστές από $ \mathbb{K} $). Το πολυώνυμο αυτό λέγεται 
      \textcolor{Col1}{χαρακτηριστικό πολυώνυμο} του πίνακα $A$ και συμβολίζεται με 
      $ \mathcal{\chi}_{A} (x) $. 
      \[
        \mathcal{\chi}_{A} (x) = \det(A-xI_{n})
      \] 
      Η μορφή του χαρακτηριστικού πολυωνύμου του $A$ είναι η εξής:
      \[
        \mathcal{\chi}_{A} (x) = (-1)^{n}x^{n} + a_{n-1}x^{n-1} + \cdots + a_{1}x +
        a_{0}  
      \] 
      Από τα παραπάνω βλέπουμε ότι οι ιδιοτιμές του $A$ είναι ακριβώς οι ρίζες του 
      χαρακτηριστικού πολυωνύμου $ \mathcal{\chi}_{A} (x) $ του πίνακα $A$.
    \item Ένας πίνακας $ A \in \smash{\textbf{M}_{n}(\mathbb{K})} $, όπου $ \mathbb{K} =
      \mathbb{R} \; \text{ή} \; \mathbb{C} $  έχει πάντοτε ιδιοτιμές στο $ \mathbb{C} $.
      Αυτό προκύπτει από το θεμελιώδες θεώρημα της Άλγεβρας που λέει ότι κάθε πολυώνυμο 
      βαθμού $n$ με συντελεστές μιγαδικούς αριθμούς έχει ακριβώς $n$ ρίζες στο 
      $ \mathbb{C} $ (όχι απαραίτητα όλες διάφορες ανά δύο). Έτσι ένας πίνακας 
      $ A \in \textbf{M}_{n}(\mathbb{K}) $ όπου $ \mathbb{K}= \mathbb{R} $ ή 
      $ \mathbb{C} $, έχει $n$ ακριβώς ιδιοτιμές στο $ \mathbb{C} $.
  \end{myitemize}

\end{rem}

\end{document}
