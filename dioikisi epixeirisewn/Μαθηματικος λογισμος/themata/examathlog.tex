\documentclass[a4paper,12pt]{article}

\usepackage[english,greek]{babel}
\usepackage[utf8]{inputenc}

\usepackage{amsmath}
\usepackage{amssymb}
\usepackage{amsfonts}
\usepackage{amsthm}
\usepackage[top=2cm,bottom=2cm,left=2cm,right=2cm]{geometry}
\begin{document}
\thispagestyle{empty}
\begin{center}
{\large\bfseries \textsc{Τμημα Διοικησης Επιχειρησεων \\
Πανεπιστημιο Πατρων}} \\[0,5cm]
\textbf{Μαθηματικός Λογισμός} \\[0,5cm]
Εξεταστική Σεπτεμβρίου $2015$
\end{center}
\vspace{0,5cm}
\begin{description}

\item [{\bfseries Θέμα $1$ο:}] Έστω το γραμμικό υπόδειγμα:
\begin{align*}
(1+\lambda)x + y + z &= 1 \\
x + (1 + \lambda)y + z &=\lambda\\
x+y+(1+\lambda)z&=\lambda^2
\end{align*}

\begin{enumerate}

\item Να υπολογίσετε την τάξη του πίνακα των συντελεστών των αγνώστων. ($1,5$ μον.)
\item Για ποιες τιμές του $\lambda$ το γραμμικό σύστημα έχει μια λύση, για ποιες άπειρες και για ποιες είναι αδύνατο. ($1,5$ μον.)
\item Όταν το παραπάνω σύστημα έχει μια λύση, να την βρείτε με τη μέθοδο της αντιστροφής των πινάκων. ($1$ μον.)
\item Έχουμε ένα γραμμικό σύστημα $m$ εξισώσεων με $n$ αγνώστους. Αναλύστε τα είδη των λύσεων που προκύπτουν αναφορικά με την τάξη του πίνακα των συντελεστών των αγνώστων και τη σχέση μεταξύ $m$ και $n$. ($2$ μον.)

\end{enumerate}

\item [{\bfseries Θέμα $2$ο:}] 
\begin{enumerate}
\item Έστω ότι κάποιος αγόρασε ένα οικόπεδο στην τιμή των $10000$ ευρώ. Υποθέστε ότι ο ρυθμός αύξησης της τιμής του οικοπέδου είναι $\sqrt{0,2t}$, ενώ ο ρυθμός με τον οποίο χάνουν την αξία τους τα χρήματα που τοποθετήθηκαν σε αυτήν την επένδυση αντιστοιχούν σε συνεχή ανατοκισμό με ρυθμό προεξόφλησης $4\%$. Να βρεθεί ο άριστος χρόνος πώλησης του οικοπέδου και η αξία πώλησής του.

\item Έστω η παρακάτω συνάρτηση:
\begin{align*}
w(x,y,u,v) &= x+y+u+v+e^{x^2y^3u^4v^5}\\
x(u,v,t) &=\alpha u^t+\beta u^t\\
y(u,v,t) &=\delta\frac{uv}{t}+ \zeta u^2v^3t^2
\end{align*}

\begin{description}
\item [$\alpha$)] Να κατασκευαστεί ο χάρτης ροών της συνάρτησης $w$. ($0,5$ μον.)
\item [$\beta$)] Να υπολογιστεί το ολικό διαφορικό της $w$. ($0,5$ μον.)
\item [$\gamma$)] Να υπολογιστεί η μερική ολική παράγωγος του $w$ ως προς $u$.
\end{description}

\item Να διατυπώσετε το θεώρημα πεπλεγμένων συναρτήσεων για σύστημα. Δώστε ένα παράδειγμα εφαρμογής του. Προσοχή το παράδειγμα να είναι πρωτότυπο, δηλαδή να διαφέρει από τα αντίστοιχα που υπάρχουν στις σημειώσσεις και τα διδακτικά συγγράμματα. ($3$ μον.)

\end{enumerate}
\end{description}
\hfill Χρόνος: $2$ ώρες
\end{document}