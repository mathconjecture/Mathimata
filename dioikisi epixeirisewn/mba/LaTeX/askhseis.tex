\documentclass[a4paper,12pt]{article}
\usepackage{etex}
%%%%%%%%%%%%%%%%%%%%%%%%%%%%%%%%%%%%%%
% Babel language package
\usepackage[english,greek]{babel}
% Inputenc font encoding
\usepackage[utf8]{inputenc}
%%%%%%%%%%%%%%%%%%%%%%%%%%%%%%%%%%%%%%

%%%%% math packages %%%%%%%%%%%%%%%%%%
\usepackage{amsmath}
\usepackage{amssymb}
\usepackage{amsfonts}
\usepackage{amsthm}
\usepackage{proof}

\usepackage{physics}

%%%%%%% symbols packages %%%%%%%%%%%%%%
\usepackage{bm} %for use \bm instead \boldsymbol in math mode 
\usepackage{dsfont}
\usepackage{stmaryrd}
%%%%%%%%%%%%%%%%%%%%%%%%%%%%%%%%%%%%%%%


%%%%%% graphicx %%%%%%%%%%%%%%%%%%%%%%%
\usepackage{graphicx}
\usepackage{color}
%\usepackage{xypic}
\usepackage[all]{xy}
\usepackage{calc}
\usepackage{booktabs}
\usepackage{minibox}
%%%%%%%%%%%%%%%%%%%%%%%%%%%%%%%%%%%%%%%

\usepackage{enumerate}

\usepackage{fancyhdr}
%%%%% header and footer rule %%%%%%%%%
\setlength{\headheight}{14pt}
\renewcommand{\headrulewidth}{0pt}
\renewcommand{\footrulewidth}{0pt}
\fancypagestyle{plain}{\fancyhf{}
\fancyhead{}
\lfoot{}
\rfoot{\small \thepage}}
\fancypagestyle{vangelis}{\fancyhf{}
\rhead{\small \leftmark}
\lhead{\small }
\lfoot{}
\rfoot{\small \thepage}}
%%%%%%%%%%%%%%%%%%%%%%%%%%%%%%%%%%%%%%%

\usepackage{hyperref}
\usepackage{url}
%%%%%%% hyperref settings %%%%%%%%%%%%
\hypersetup{pdfpagemode=UseOutlines,hidelinks,
bookmarksopen=true,
pdfdisplaydoctitle=true,
pdfstartview=Fit,
unicode=true,
pdfpagelayout=OneColumn,
}
%%%%%%%%%%%%%%%%%%%%%%%%%%%%%%%%%%%%%%

\usepackage[space]{grffile}

\usepackage{geometry}
\geometry{left=25.63mm,right=25.63mm,top=36.25mm,bottom=36.25mm,footskip=24.16mm,headsep=24.16mm}

%\usepackage[explicit]{titlesec}
%%%%%% titlesec settings %%%%%%%%%%%%%
%\titleformat{\chapter}[block]{\LARGE\sc\bfseries}{\thechapter.}{1ex}{#1}
%\titlespacing*{\chapter}{0cm}{0cm}{36pt}[0ex]
%\titleformat{\section}[block]{\Large\bfseries}{\thesection.}{1ex}{#1}
%\titlespacing*{\section}{0cm}{34.56pt}{17.28pt}[0ex]
%\titleformat{\subsection}[block]{\large\bfseries{\thesubsection.}{1ex}{#1}
%\titlespacing*{\subsection}{0pt}{28.80pt}{14.40pt}[0ex]
%%%%%%%%%%%%%%%%%%%%%%%%%%%%%%%%%%%%%%

%%%%%%%%% My Theorems %%%%%%%%%%%%%%%%%%
\newtheorem{thm}{Θεώρημα}[section]
\newtheorem{cor}[thm]{Πόρισμα}
\newtheorem{lem}[thm]{λήμμα}
\theoremstyle{definition}
\newtheorem{dfn}{Ορισμός}[section]
\newtheorem{dfns}[dfn]{Ορισμοί}
\theoremstyle{remark}
\newtheorem{remark}{Παρατήρηση}[section]
\newtheorem{remarks}[remark]{Παρατηρήσεις}
%%%%%%%%%%%%%%%%%%%%%%%%%%%%%%%%%%%%%%%




\newcommand{\vect}[2]{(#1_1,\ldots, #1_#2)}
%%%%%%% nesting newcommands $$$$$$$$$$$$$$$$$$$
\newcommand{\function}[1]{\newcommand{\nvec}[2]{#1(##1_1,\ldots, ##1_##2)}}

\newcommand{\linode}[2]{#1_n(x)#2^{(n)}+#1_{n-1}(x)#2^{(n-1)}+\cdots +#1_0(x)#2=g(x)}

\newcommand{\vecoffun}[3]{#1_0(#2),\ldots ,#1_#3(#2)}

\newcommand{\mysum}[1]{\sum_{n=#1}^{\infty}




\everymath{\displaystyle}


\begin{document}

\begin{center}
    \fbox{\large\bfseries Ασκήσεις Επανάληψης}
\end{center}

\vspace{\baselineskip}


\begin{enumerate}

    \item Να λυθούν οι παρακάτω εξισώσεις.

        \begin{enumerate}[i)]
            \item $ (1-x)^{\frac{1}{2}} = 4x-1 $ \hfill Απ: $ x=0 $
            \item $ (1-x)^{\frac{1}{2}} = 4x $ \hfill Απ: $ x = \frac{-1+ \sqrt{65}}{2} $ 
        \end{enumerate}

    \item Να βρεθεί η τιμή και η ποσότητα ισορροπίας για τα παρακάτω υποδείγματα:

        \begin{enumerate}[i)]
            \item $\sysdelim.\}\systeme*{Q_{d} = 100 - 2P,Q_{s} = 10P-15,Q_{d}=Q_{s}} $
                \hfill Απ: $ \bar{P} = \frac{115}{12} $, $ \bar{Q} = \frac{485}{6} $

            \item  $\sysdelim.\}\systeme*{Q_{d} = \phantom{1}50 - 3P,Q_{s}=7P+5,Q_{d}=Q_{s}}$
                \hfill Απ: $ \bar{P} = \frac{45}{10} $, $ \bar{Q} = \frac{365}{10} $ 

            \item $\sysdelim.\}\systeme*{Q_{d}=10-P^{2},Q_{s}=8P-4,Q_{d}=Q_{s}} $
                \hfill Απ: $ \bar{P}=-4+2 \sqrt{30} $   

            \item $\sysdelim.\}\systeme*{Q_{d_{1}}=28-3P_{1}+P_{2},Q_{s_{1}}=-4+4P_{1},
                    Q_{d_{1}}=Q_{s_{1}},Q_{d_{2}}=10+P_{1}-P_{2},Q_{s_{2}}=-5+9P_{2},
                    Q_{d_{2}}=Q_{s_{2}}}$
                \hfill Απ: $ \bar{P}_{1} = \frac{335}{69} $, $ \bar{P}_{2} = \frac{137}{69} $  
        \end{enumerate}

    \item Να λυθεί το σύστημα 
    \[\sysdelim.\}\systeme*{Y=C+I_{0}+G_{0},C=a+bY}, \quad a>0,\; 0<b<1 \]
με τη μέθοδο των οριζουσών και με τη μέθοδο του αντίστροφου πίνακα.

        \hfill Απ: $ \bar{Y} = \frac{a + I_{0}+G_{0}}{1-b} $, 
            $ \bar{C} = \frac{a + b(I_{0}+G_{0})}{1-b} $  

    \item Να υπολογιστούν τα παρακάτω όρια.
        
        \begin{enumerate}[i)]
            \item $ \lim_{x\to 4^{+}} \frac{\abs{x-4}}{x^{2}-16} $ \hfill Απ: $ \frac{1}{8} $
            \item $ \lim_{x\to 8^{+}} \frac{3x+2}{9x^{2}-69x-24} $ \hfill Απ: $ +\infty $
            \item $ \lim_{x\to +\infty} (\sqrt{x^{2}-4x+1} - x) $ \hfill Απ: $ -2 $ 
            \item $ \lim_{x\to 4} \frac{x^{2}+13x+36}{x+4} $ \hfill Απ: $ 13 $
            \item $ \lim_{x\to 1} \frac{2x^{2}+3x-4}{x^{2}+5x-3} $ \hfill Απ: $ \frac{1}{3} $
            \item $ \lim_{x\to 2} (16+ \sqrt{x+2}) $ \hfill Απ: $ 16 $
            \item $ \lim_{x\to 1} \frac{2x^{2}+3x-5}{x^{2}+5x-6} $ \hfill Απ: $ 1 $
            \item $ \lim_{x\to 2} \frac{\sqrt{2x+5} - 3}{x-2} $ \hfill Απ: $ \frac{1}{3} $
            \item $ \lim_{x\to 4} \frac{x^{3}-7x^{2}+17x-20}{x^{2}-5x+4} $ \hfill Απ: $ 3 $ 
        \end{enumerate}

    \item Να βρεθεί η ολική παράγωγος των ακόλουθων συναρτήσεων

        \begin{enumerate}[i)]
            \item $ z = f(x,y) = 2x+xy-y^{2} $, όπου $ x = g(y) = 3y^{2} 
               $ \hfill Απ: $ \dv{f}{y} = 9y^{2}+10y $ 
            \item $ z = f(x,y) = 6x^{2}-3xy+2y^{2} $, όπου $ x = g(y) = \frac{1}{y} $
                \hfill Απ: $ \dv{f}{y} = - \frac{12}{y^{3}} + \frac{4}{y} $
            \item $ z = f(x,y) = (x+y)(x-2y) $, όπου $ x = g(y) = 2-7y $
                \hfill Απ: $ \dv{f}{y} = -30 + 108y $ 
        \end{enumerate}

    \item Δίνεται η συνάρτηση κατανάλωσης $ C=a+by $ με $ a>0 $ και $ 0<b<1 $.

        \begin{enumerate}[i)]
            \item Να βρεθεί η οριακή και η μέση συνάρτηση κατανάλωσης. 
            \item Να βρεθεί η εισοδηματική ελαστικότητα της κατανάλωσης και να προσδιοριστεί το
                πρόσημο υποθέτοντας ότι $ y>0 $. 
            \item Να δείξετε ότι αυτή η συνάρτηση κατανάλωσης είναι ανελαστική σε όλα τα θετικά
επίπεδα εισοδήματος. 

\hfill Απ:  \begin{inparaenum}[i)]
               \item  $ AC = \frac{a}{y} +b $, $ MC =b $  \;
               \item $ \varepsilon_{C,y} = \frac{by}{a+by} $ \;
               \item $ \abs{ \varepsilon_{C,y}}<1,\; \forall y>0 $ \;
               \end{inparaenum}
        \end{enumerate}

    \item Να βρεθούν τα στάσιμα σημεία των επόμενων συναρτήσεων αν υποθέσουμε ότι το πεδίο ορισμού
        τους είναι το σύνολο των πραγματικών αριθμών.

        \begin{enumerate}[i)]
            \item $ y = -2x^{2}+4x+9 $ \hfill Απ: $ (1,11) $
            \item $ y = 5x^{2}+x $ \hfill Απ: $ \left(- \frac{1}{10}, - \frac{1}{20}\right) $ 
        \end{enumerate}

    \item Να δείξετε ότι η συνάρτηση $ y = x + \frac{1}{x} $ με $ x \neq 0 $ έχει δύο σχετικά
        ακρότατα, ένα μέγιστο και ένα ελάχιστο. 

        \hfill Απ: $ x_{min} = 1$, $ x_{max}=-1  $ 

    \item Να βρεθούν η 2η και η 3η παράγωγος των συναρτήσεων.

        \begin{enumerate}[i)]
            \item $ f(x) = ax^{2}+bx+c $ \hfill Απ: $ f''(x)=2a $
            \item $ f(x) = 6x^{4}-3x-4 $ \hfill Απ: $ f''(x) = 72x^{2} $ 
        \end{enumerate}

    \item Να βρεθούν τα μέγιστα και τα ελάχιστα της συνάρτησης $ y $ με το κριτήριο της 2ης
        παραγώγου.

        \begin{enumerate}[i)]
            \item $ y = -2x^{2}+8x+25 $ \hfill Απ: $ f_{max}(2)=33 $
            \item $ y = x^{3}+6x^{2}+7 $ \hfill Απ: $ f_{max}(0)=7 $, $ f_{min}(-4)=39 $ 
        \end{enumerate}

    \item Να βρεθούν οι τιμές των παρακάτω παραγοντικών εκφράσεων.

        \begin{enumerate}[i)]
            \item $ 5! $ \hfill Απ: $ 120 $
            \item $ \frac{(n+2)!}{n!} $ \hfill Απ: $ (n+1)(n+2) $
        \end{enumerate}

    \item Θεωρώντας ως δεδομένο ότι η $ e^{t} $ είναι ίση με την παράγωγό της, χρησιμοποιείστε τον
        κανόνα αλυσίδας για να βρείτε την $ \dv{y}{t} $ για τις επόμενες συναρτήσεις.

        \begin{enumerate}[i)]
            \item $ y = e^{5t} $ \hfill Απ: $ \dv{y}{t} = 5e^{5t} $
            \item $ y = 4e^{3t} $ \hfill Απ: $ \dv{y}{t} = 12e^{3t} $ 
        \end{enumerate}

    \item Χρησιμοποιώντας την άπειρη σειρά του $ e^{x} $ να βρεθεί μια προσεγγιστική τιμή (εως 3
        δεκαδικά ψηφία) των 

        \begin{enumerate}[i)]
            \item $ y=e^{2} $ \hfill Απ: $ e^{2} \approx 7,388 $ 
            \item $ y=e^{\frac{1}{2}}= \sqrt{e} $ \hfill Απ: $ \sqrt{e} \approx 1,649 $ 
        \end{enumerate}

    \item Να υπολογιστούν με τη βοήθεια των ιδιοτήτων των λογαριθμικών συναρτήσεων οι παρακάτω
        παραστάσεις.

        \begin{enumerate}[i)]
            \item $ \ln{e^{2}} $ \hfill Απ: $ 2 $ 
            \item $ \ln{e^{x}} - e^{\ln{x}} $ \hfill Απ: $ 0  $ 
            \item $ \ln{\frac{1}{e^{3}}} $ \hfill Απ: $ -3 $ 
        \end{enumerate}

    \item Να βρεθεί η αντίστροφη της συνάρτησης $ y = ab^{ct}$ \hfill Απ: $ t = \frac{\log_{b}{y} -
        \log_{b}{a}}{c} $ 

    \item Να βρεθούν οι παράγωγοι των παρακάτω συναρτήσεων.

        \begin{enumerate}[i)]
            \item $ y = 5^{t} $ \hfill Απ: $ y'=5^{t} \ln{5} $
            \item $ y = 13^{2t+3} $ \hfill Απ: $ y' = 2\cdot 13^{2t+3} \ln{13}  $ 
            \item $ y = \ln{(t+9)} $ \hfill Απ: $y'= \frac{1}{t+9} $
            \item $ y = xe^{x} $ \hfill Απ: $ y'=e^{x}(1+x) $ 
        \end{enumerate}
\end{enumerate}


\end{document}
