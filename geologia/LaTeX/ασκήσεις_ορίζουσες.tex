\documentclass[a4paper,12pt]{article}


\usepackage[english,greek]{babel}
\usepackage[utf8]{inputenc}
\usepackage{amsmath}
\usepackage{amssymb}
\usepackage{amsfonts}
\usepackage{enumerate}

\usepackage{physics}
\usepackage{graphicx}
\usepackage{geometry}

\geometry{left=25.63mm,right=25.63mm,top=36.25mm,bottom=36.25mm,footskip=24.16mm,headsep=24.16mm}

\pagestyle{empty}


\begin{document}

\begin{center}
\fbox{\large \bfseries Ασκήσεις στις Ορίζουσες}
\end{center}

\vspace{2\baselineskip}

\begin{enumerate}


\item Να βρεθούν οι τιμές των οριζουσών:

\begin{enumerate}[i)]
\item 
\(
D_1=\begin{vmatrix}
1 & -1 & \phantom{-}2 \\
2 & -3 & -1 \\
2 & \phantom{-}4 & \phantom{-}3
\end{vmatrix}\
\)  \hfill Απ: 31

\item 
\(
D_2=\begin{vmatrix}
-2 & 1 & \phantom{-}2 \\
\phantom{-}3 & 5 & -1 \\
-3 & 2 & \phantom{-}1
\end{vmatrix}
\)  \hfill Απ: 28

\item 
\(
D_3=\begin{vmatrix}
\phantom{-}1 & \phantom{-}2 & 0 & \phantom{-}1 \\
\phantom{-}2 & -1 & 1 & \phantom{-}2 \\
\phantom{-}6 & \phantom{-}3 & 0 & -2 \\
-1 & \phantom{-}2 & 0 & \phantom{-}2
\end{vmatrix}\
\) \hfill Απ: -5
\end{enumerate}

\end{enumerate}

\end{document}