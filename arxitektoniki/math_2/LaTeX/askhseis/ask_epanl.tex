\input{preamble_ask.tex}
\input{definitions_ask.tex}

\pagestyle{askhseis}

\begin{document}

\begin{center}
\minibox{\large \bfseries \textcolor{Col1}{Ασκήσεις Επανάληψης}}
\end{center}

\vspace{\baselineskip}


\subsection*{Μερική Παράγωγος}

\begin{enumerate}

  \item Να υπολογιστεί η $ f_{x} $ της συνάρτησης $ f(x,y) =
    \cos^{-1}{(\frac{3y}{x})} $ 
    \hfill Απ: $ f_{x} = \frac{3y}{x \sqrt{x^{2}-9y^{2}}} $ 

  \item Να υπολογιστούν οι $ f_{x} $ και $ f_{xx} $ της συνάρτησης 
    $ f(x,y) = \arcsin(1-2x) $ 
    \hfill Απ: 
    \begin{tabular}{l}
      $ f_{x} = - \frac{1}{\sqrt{x(1-x)}} $ \\ 
      $ f_{xx} = \frac{1-2x}{2[x(1-x)]^{3/2}} $ 
    \end{tabular}

  \item Να υπολογιστούν οι $ f_{x} $ και $ f_{xx} $ της συνάρτησης 
    $ f(x,y) = \arctan{(y-2x)} $
      \hfill Απ: 
      \begin{tabular}{l}
        $ f_{x} = - \frac{2}{1 + (y-2x)^{2 }} $ \\ 
        $ f_{xx} = - \frac{8(y-2x)}{[1+(y-2x)^{2}]^{2}} $ 
      \end{tabular}

  \item Για τη συνάρτηση $ f(x,y) = \ln{(\frac{y}{x^{2} + y^2})} $ να δείξετε ότι 
    $ f_{xy}= f_{yx} $.
\end{enumerate}

\subsection*{Τέλεια Διαφορικά}

\begin{enumerate}
  \item Να εξετάσετε αν οι παρακάτω παραστάσεις είναι ολικά (τέλεια) διαφορικά και αν 
    ναι να βρεθεί η συνάρτηση (δυναμικού) $ f(x,y) $ της οποίας η παράσταση είναι το 
    διαφορικό.
    \begin{enumerate}[i)]
      \item $ (\frac{1}{x} + y^{3}) dx + (e^{y}+3xy^2) dy $ 
        \hfill Απ: $ f(x,y) = \ln{x} + \mathrm{e}^{y} +xy^3 + c $  

      \item $ (\cos{x} + \ln{y})dx + \left(\frac{x}{y} + e^{y}\right)dy $ 
        \hfill Απ: $ f(x,y) = \sin{x} + x \ln{y} + e^{y}+c $ 
    \end{enumerate}

  \item Να βρεθεί η συνάρτηση $ f(x,y) $ όταν ικανοποιεί τη σχέση $ f(1, \pi ) = -2 $ 
    και έχει μερικές παραγώγους 
    \[
      \pdv{f}{x}= 2 \ln{\Bigl(\frac{2y}{\pi}\Bigr)} + 2x \tan{y} \quad \text{και} \quad
      \pdv{f}{y} = \frac{2x}{y} + x^{2} \sec^{2}{y}
    \]
    \hfill Απ: $ f(x,y) = x^{2} \tan{y} + 2x \ln{(\frac{2y}{\pi})} - 
    2(1+ \ln{2}) $  

\end{enumerate}


\subsection*{Ακρότατα}

\begin{enumerate}

  \item Να βρεθούν οι διαστάσεις ενός ορθογωνίου παραλληλεπιπέδου με μέγιστο όγκο $ V $ 
    που χωράει μέσα σε σφαίρα με εξίσωση $ x^{2}+y^{2}+z^{2}= 4 $ 
    \hfill Απ: $ x = \frac{2}{\sqrt{3}} , y= \frac{2}{\sqrt{3}}, 
    z= \frac{2}{\sqrt{3} } ,  V_{max} = \frac{8}{3 \sqrt{3}} $  

  \item Να βρεθεί το ορθογώνιο παραλληλεπίπεδο με μέγιστο όγκο που χωράει μέσα σε 
    ελλειψοειδές με εξίσωση $ \frac{x^{2}}{a^{2}} + \frac{y^{2}}{b^{2}} +
    \frac{z^{2}}{c^{2}} =1 $.
    \hfill Απ: $ x = \frac{a}{\sqrt{3}} , y= \frac{b}{\sqrt{3}}, 
    z= \frac{c}{\sqrt{3} } ,  V_{max} = \frac{8abc}{3 \sqrt{3}} $  

  \item Να βρεθεί το ορθογώνιο παραλληλεπίπεδο με μέγιστο όγκο που βρίσκεται πάνω στο 
    επίπεδο $ xy $ με την μια κορυφή του στην αρχή των αξόνων ενώ η απέναντί του 
    κορυφή ακουμπά στο επίπεδο $ 2x+y+z=2 $.

    \hfill Απ: $ x = \frac{1}{3} , y= \frac{2}{3}, 
    z= \frac{2}{3} ,  V_{max} = \frac{4}{27} $  

  \item Να βρεθεί το σημείο του επιπέδου $ 3x-y+z=8 $ το οποίο απέχει ελάχιστη απόσταση 
    από την αρχή $ (0,0,0) $.

    \hfill Απ: $ (\frac{24}{11} , \frac{-8}{11} , \frac{8}{11}), d_{min} =
    \frac{64}{11} $ 
\end{enumerate}


\subsection*{Διπλό Ολοκλήρωμα}

\begin{enumerate}
  \item Να υπολογιστούν τα παρακάτω διπλά ολοκληρώματα (γενικά χωρία).
    \begin{enumerate}[i)]
      \item $\iint\limits_{D}(x-1)\,dxdy,\quad D$ περικλείεται από τις καμπύλες 
        $y=x$ και $y=x^3$. %spandagos p.63 ex 
        \hfill Απ: $-\frac{1}{2}$
      \item $ \iint\limits_{D} y\,dxdy, \quad D $ περικλείεται από τις παραβολές 
        $ y^2=4x $ και $ x^{2}=4y $ 
        \hfill Απ: $ \frac{48}{5} $ %spandagos p.109 ask.14
      \item $ \iint\limits_{D} xy\,dxdy, \quad D $ περικλείεται από τις καμπύλες 
        $ x=y^{2} $, $ x=4-y^{2} $ και $ y=0 $, 1ο τεταρτ.
        \hfill Απ: $4$ %spandagos p.135 ask.57

    \end{enumerate}

  \item Να βρείτε το \textbf{εμβαδόν} του χωρίου $D$ που περικλείεται από τις καμπύλες: 
    \begin{enumerate}[i)]
      \item $y=3-2x^2$ και $y=x^4$ \hfill Απ: $\frac{64}{15}$
      \item $x=\frac{1}{4}$ και $y^2=4x$ \hfill Απ: $\frac{1}{3}$ %spand p.168 ask.95
      \item $y^2=x$ και $y=x^2$ \hfill Απ: $\frac{1}{3}$ %spandagos p.170 ask.100
      \item $xy=2$, $4y=x^2$ και $y=4$ \hfill Απ: $\frac{28}{3}-2\ln 4$ 
        %spandagos p.77 ex.1
    \end{enumerate}

  \item Να βρεθεί ο όγκος του στερεού που βρίσκεται κάτω από την επιφάνεια 
    $ x+z=2 $ και έχει ως βάση την περιοχή που περικλείεται από τις καμπύλες 
    $ y=3-x^{2} $ και $ y=2x $. 

    \hfill Απ: $ 11/4 $ 

  \item Να βρεθεί ο όγκος του στερεού που έχει βάση τετράγωνο πλευράς 1 και βρίσκεται
    κάτω από την επιφάνεια $ y=x^{2}+z^{2}+1 $. 
    \textcolor{Col1}{Υπόδειξη:} Ολοκληρώστε ως προς $ xz $

    \hfill Απ: $5/3$ 

  \item Να βρεθεί το εμβαδό του τμήματος της επιφάνειας $ z= \frac{2}{3} x^{3/2} $ 
    όταν η προβολή του στο επίπεδο $ xy $ ειναι το ορθογώνιο $ [0,3] \times [0,2] $ 

    \hfill Απ: $ \frac{28}{ 3} $ 
\end{enumerate}

 

\end{document}
