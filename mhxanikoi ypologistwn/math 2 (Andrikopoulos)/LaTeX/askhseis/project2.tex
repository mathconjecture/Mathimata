\documentclass[a4paper,12pt]{article}
\usepackage{etex}
%%%%%%%%%%%%%%%%%%%%%%%%%%%%%%%%%%%%%%
% Babel language package
\usepackage[english,greek]{babel}
% Inputenc font encoding
\usepackage[utf8]{inputenc}
%%%%%%%%%%%%%%%%%%%%%%%%%%%%%%%%%%%%%%

%%%%% math packages %%%%%%%%%%%%%%%%%%
\usepackage{amsmath}
\usepackage{amssymb}
\usepackage{amsfonts}
\usepackage{amsthm}
\usepackage{proof}

\usepackage{physics}

%%%%%%% symbols packages %%%%%%%%%%%%%%
\usepackage{bm} %for use \bm instead \boldsymbol in math mode 
\usepackage{dsfont}
\usepackage{stmaryrd}
%%%%%%%%%%%%%%%%%%%%%%%%%%%%%%%%%%%%%%%


%%%%%% graphicx %%%%%%%%%%%%%%%%%%%%%%%
\usepackage{graphicx}
\usepackage{color}
%\usepackage{xypic}
\usepackage[all]{xy}
\usepackage{calc}
\usepackage{booktabs}
\usepackage{minibox}
%%%%%%%%%%%%%%%%%%%%%%%%%%%%%%%%%%%%%%%

\usepackage{enumerate}

\usepackage{fancyhdr}
%%%%% header and footer rule %%%%%%%%%
\setlength{\headheight}{14pt}
\renewcommand{\headrulewidth}{0pt}
\renewcommand{\footrulewidth}{0pt}
\fancypagestyle{plain}{\fancyhf{}
\fancyhead{}
\lfoot{}
\rfoot{\small \thepage}}
\fancypagestyle{vangelis}{\fancyhf{}
\rhead{\small \leftmark}
\lhead{\small }
\lfoot{}
\rfoot{\small \thepage}}
%%%%%%%%%%%%%%%%%%%%%%%%%%%%%%%%%%%%%%%

\usepackage{hyperref}
\usepackage{url}
%%%%%%% hyperref settings %%%%%%%%%%%%
\hypersetup{pdfpagemode=UseOutlines,hidelinks,
bookmarksopen=true,
pdfdisplaydoctitle=true,
pdfstartview=Fit,
unicode=true,
pdfpagelayout=OneColumn,
}
%%%%%%%%%%%%%%%%%%%%%%%%%%%%%%%%%%%%%%

\usepackage[space]{grffile}

\usepackage{geometry}
\geometry{left=25.63mm,right=25.63mm,top=36.25mm,bottom=36.25mm,footskip=24.16mm,headsep=24.16mm}

%\usepackage[explicit]{titlesec}
%%%%%% titlesec settings %%%%%%%%%%%%%
%\titleformat{\chapter}[block]{\LARGE\sc\bfseries}{\thechapter.}{1ex}{#1}
%\titlespacing*{\chapter}{0cm}{0cm}{36pt}[0ex]
%\titleformat{\section}[block]{\Large\bfseries}{\thesection.}{1ex}{#1}
%\titlespacing*{\section}{0cm}{34.56pt}{17.28pt}[0ex]
%\titleformat{\subsection}[block]{\large\bfseries{\thesubsection.}{1ex}{#1}
%\titlespacing*{\subsection}{0pt}{28.80pt}{14.40pt}[0ex]
%%%%%%%%%%%%%%%%%%%%%%%%%%%%%%%%%%%%%%

%%%%%%%%% My Theorems %%%%%%%%%%%%%%%%%%
\newtheorem{thm}{Θεώρημα}[section]
\newtheorem{cor}[thm]{Πόρισμα}
\newtheorem{lem}[thm]{λήμμα}
\theoremstyle{definition}
\newtheorem{dfn}{Ορισμός}[section]
\newtheorem{dfns}[dfn]{Ορισμοί}
\theoremstyle{remark}
\newtheorem{remark}{Παρατήρηση}[section]
\newtheorem{remarks}[remark]{Παρατηρήσεις}
%%%%%%%%%%%%%%%%%%%%%%%%%%%%%%%%%%%%%%%




\newcommand{\vect}[2]{(#1_1,\ldots, #1_#2)}
%%%%%%% nesting newcommands $$$$$$$$$$$$$$$$$$$
\newcommand{\function}[1]{\newcommand{\nvec}[2]{#1(##1_1,\ldots, ##1_##2)}}

\newcommand{\linode}[2]{#1_n(x)#2^{(n)}+#1_{n-1}(x)#2^{(n-1)}+\cdots +#1_0(x)#2=g(x)}

\newcommand{\vecoffun}[3]{#1_0(#2),\ldots ,#1_#3(#2)}

\newcommand{\mysum}[1]{\sum_{n=#1}^{\infty}



\everymath{\displaystyle}
\setlength{\tabcolsep}{1cm}


\begin{document}

\begin{center}
	\fbox{\large \bfseries  Λύσεις 2ου  \textlatin{project}}	
\end{center}

\vspace{\baselineskip}

\begin{enumerate}
	\setlength\itemsep{2em}
	\item  

		\begin{enumerate}[i)]
			\item $ \lim_{(x,y)\to(0,0)} \frac{\sin{(x^{2}+y^{2})}}{x^{2}+y^{2}} \underset{\lim u = 0}
				{\overset{u=x^{2}+y^{2}}{=}} \lim_{u\to 0} \frac{\sin{u}}{u} = 1 $

			\item $ \lim_{(x,y,z)\to (1,1,1)} \frac{xy+yz+zx}{x^{2}+y^{2}+z^{2}} = \frac{1 \cdot 1 + 1 \cdot 1 + 1 \cdot1}{1^{2}+1^{2}+1^{2}} $

			\item $ L_{1} = \lim_{x\to 0} \left(\lim_{y\to 0} \frac{x+y}{x-y}\right) = \lim_{x\to 0} \frac{x}{x} = \lim_{x\to 0} 1 = 1$

				$ L_{2} = \lim_{y\to 0} \left(\lim_{x\to 0} \frac{x+y}{x-y}\right) = \lim_{y\to 0} \frac{y}{-y}
				= \lim_{y\to 0} -1 = -1 $

				Άρα δεν υπάρχει το όριο καθώς  $ (x,y)\to (0,0) $ γιατί τα επάλληλα όρια δεν είναι ίσα.

				Επίσης η συνάρτηση φαίνεται πως δεν έχει όριο πάνω στην ευθεία $ y=x $ αφού η $f$ δεν είναι
				συνεχής. 

			\item Αν  $ \lim_{(x,y)\to (x_{0},y_{0})} f(x,y) = L $, τότε η $f$ δεν είναι απαραίτητο να ορίζεται
				στο σημείο $ (x_{0},y_{0}) $. 

				Το σημείο  $ (x_{0},y_{0}) $ πρέπει όμως να είναι σημείο \textbf{συσσώρευσης} του  πεδίου ορισμού
				της.
		\end{enumerate}

	\item 
		\begin{enumerate}[i)]
			\item Η συνάρτηση δίνεται από τον τύπο:
				\[
					f(x,y,z) = y^{(0 \mod 3 + 1)} \cdot \ln{(4+3)z} \cdot y^{(9 \mod
					2 + 1)x} = y \cdot \ln{7z} \cdot y^{2x} = y^{2x+1} \ln{7z} 
				\]  

				\renewcommand{\arraystretch}{2.5}
				\[
					\begin{tabular}{ll}
						$ \pdv{f}{x} = 2y^{2x+1} \ln{7z} $ & $ \pdv[2]{f}{x} = 4y^{2x+1} \ln{7z} $ \\
						$ \pdv{f}{y} = 2(2x+1)y^{2x} \ln{7z} $ & $ \pdv[2]{f}{y} = 4(2x+1)y^{2x-1} \ln{7z} $ \\
						$ \pdv{f}{z} = y^{2x+1} \frac{1}{z} $ & $ \pdv[2]{f}{z} = -y^{2x+1} \frac{1}{z^{2}} $ \\
															  & $ \pdv[2]{f}{x}{y} = \pdv[2]{f}{y}{x} =  4(2x+1)y^{2x} \ln{7z}  $  \\
															  & $ \pdv[2]{f}{y}{z} = \pdv[2]{f}{z}{y} = 2(2x+1)y^{2x} \frac{1}{z}  $ \\
															  & $ \pdv[2]{f}{x}{z} = \pdv[2]{f}{z}{x} = 2y^{2x+1} \frac{1}{z}  $\\
					\end{tabular}
				\]

			\item Ζητάμε την κλίση της συνάρτησης $f$.
				\[
					\grad f = \left(\pdv{f}{x}, \pdv{f}{y}, \pdv{f}{z}\right) =
					\left(\underbrace{2y^{2x+1} \ln{7z}}_{P}, \underbrace{2(2x+1)y^{2x}
					\ln{7z}}_{Q}, \underbrace{y^{2x+1} \frac{1}{z}}_{R}\right)
				\] 

			\item 
				\[
					\div \grad f = \pdv{P}{x} + \pdv{Q}{y} + \pdv{R}{z} = 4y^{2x+1} \ln{7z} +
					8x(2x+1)y^{2x-1} \ln{7z} - y^{2x+1} \frac{1}{z^{2}} 
				\] 

			\item 
				\[
					\curl \grad f = \begin{vmatrix*}[c]
						\mathbf{i} & \mathbf{j} & \mathbf{k} \\
						\pdv{}{x} & \pdv{}{y} & \pdv{}{z} \\
						2y^{2x+1} \ln{7z}  & 2(2x+1)y^{2x} \ln{7z} & y^{2x+1} \frac{1}{z} 
					\end{vmatrix*} = 0
				\]

		\end{enumerate}

	\item  Ζητάμε τοπικά ακρότατα για τη συνάρτηση
		\[
			f(x,y) = (x+y)^{4} + y^{2} 
		\] 
		\begin{itemize}
			\item Βρίσκουμε όλες τις μερικές παραγώγους της  $f$
				\[
					\renewcommand{\arraystretch}{2.5}
					\begin{tabular}{ll}
						$ f_{x} = 4(x+y)^{3} $ & $ f_{xx} = 12(x+y)^{2} $ \\
						$ f_{y} = 2y $ & $ f_{yy} = 2  $ \\
											& $ f_{xy} = f_{yx} = 12(x+y)^{2} $
					\end{tabular}
				\] 

			\item Βρίσκουμε τα κρίσιμα σημεία της  $f$
				\[
					\left.
						\begin{tabular}{l}
							$	f_{x} = 0$ \\
							$	f_{y} = 0$
						\end{tabular}
					\right\} \Leftrightarrow 
					\left.
						\begin{tabular}{l}
							$	4(x+y)^{3} = 0$ \\
							$	4(x+y)^{3} + 2y = 0$
						\end{tabular}
					\right\} \Leftrightarrow  
					\left.
						\begin{tabular}{l}
							$ x=-y	$ \\
							$	4(x+y)^{3}+2y = 0$
						\end{tabular}
					\right\} \Leftrightarrow  
					\left.
						\begin{tabular}{l}
							$ x=-y	$ \\
							$ y=0	$
						\end{tabular}
					\right\} \Leftrightarrow  
					\left.
						\begin{tabular}{l}
							$ x=0	$ \\
							$ y=0	$
						\end{tabular}
					\right\}   
				\] 
				Άρα η  $f$  έχει μοναδικό στάσιμο σημείο το  $ (0,0) $.

			\item Ελέγχουμε το στάσιμο σημείο  $ (0,0) $ με το κριτήριο της 2ης παραγώγου. 
				\[
					|H_{2}|=	\begin{vmatrix*}[c]
						f_{xx} & f_{xy} \\
						f_{yx} & f_{yy}
					\end{vmatrix*} = 
					\begin{vmatrix*}[c]
						0 & 0 \\
						0 & 2
					\end{vmatrix*} = 0
				\] 
				Επομένως το κριτήριο δεν δίνει συμπέρασμα και γι᾽ αυτό ελέγχουμε με τη βοήθεια του ορισμού:

				\[f(x,y) - f(0,0) = (x+y)^{4} + y^{2} > 0 \]  για κάθε  $ (x,y) $.  Επομένως
				\[
					f(x,y)>f(0,0) 
				\]  που σημαίνει ότι στο σημείο  $ (0,0) $ η  $f$  παρουσιάζει τοπικό \textbf{ελάχιστο}.

		\end{itemize}

	\item Ζητάμε τοπικά ακρότατα για τη συνάρτηση
		 \[
			 f(x,y,z) = 7 - x^{2} - 2y^{2} - 3z^{2} +2xz 
		 \] 
		 \begin{itemize}
			 \item  Βρίσκουμε όλες τις μερικές παραγώγους της  $f$
				 \[
					 \renewcommand{\arraystretch}{2.5}
					  \begin{tabular}{lll}
						  $\pdv{f}{x} = -2x+2z$ & $f_{xx} = -2$ &  $f_{xy}=f_{yx}=0$ \\
						  $\pdv{f}{y} = -4y $& $f_{yy} = -4$ & $f_{yz}=f_{zy}=0$ \\
						  $\pdv{f}{z} = -6z + 2x$ & $f_{zz} = -7$ &  $f_{xz}=f_{zx}=2$ \\
					  \end{tabular} 
				 \] 

	 \item  Βρίσκουμε τα κρίσιμα σημεία
		   \[
					\left.
						\begin{tabular}{l}
							$ f_{x} = 0	$ \\
							$ f_{y} = 0 	$\\
							$ f_{z} = 0 	$\\

						\end{tabular}
					\right\}   \Leftrightarrow 
					\left.
						\begin{tabular}{l}
							$-2x+2z=0	$ \\
							$ -4y=0 	$ \\
							$ -6z+2x=0  	$
						\end{tabular}
					\right\}   \Leftrightarrow 
					\left.
						\begin{tabular}{l}
							$-2x+2z=0	$ \\
							$ -4y=0 	$ \\
							$ -6z+2x=0  	$
						\end{tabular}
					\right\} \Leftrightarrow (x,y,z)=(0,0,0)
		   \] 
		   Επομένως η  $f$  έχει μοναδικό στάσιμο σημείο, το  $ (0,0,0) $.

	   \item  Ελέγχουμε το στάσιμο σημείο με τη βοήθεια του κριτηρίου 2ης παραγώγου
		   \[
			   |H_{1}|=f_{xx}=-2 
		   \]
		   \[
			   |H_{2}|= \begin{vmatrix*}[c]
				   f_{xx} & f_{xy} \\
				   f_{yx} & f_{yy}
			   \end{vmatrix*} = 
			   \begin{vmatrix*}[r]
				   -2 & 0 \\
				   0 & -4
			   \end{vmatrix*} = 8 > 0 
		   \]
		   \[
			   |H_{3}|=\begin{vmatrix*}[c]
				   f_{xx} & f_{xy} & f_{xz} \\
				   f_{yx} & f_{yy} & f_{yz} \\
				   f_{zx} & f_{zy} & f_{zz} 
			    \end{vmatrix*} = 
				\begin{vmatrix*}[r]
					-2 & 0 & 2 \\
					0 & -4 & 0 \\
					2 & 0 & -6
				\end{vmatrix*} = -32<0
		   \]
		   Επομένως η  $f$  στο σημείο  $ (0,0,0) $ παρουσιάζει τοπικό \textbf{μέγιστο}.

		\end{itemize}

	\item  Για το πρόβλημα, δείτε την υπόδειξη που δίνει.  Είναι αρκετά απλό.
		   
\end{enumerate}

\end{document}
