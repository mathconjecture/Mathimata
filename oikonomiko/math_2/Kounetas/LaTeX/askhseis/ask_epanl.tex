\documentclass[a4paper,table]{report}
\input{preamble_ask.tex}
\input{definitions_ask.tex}

\pagestyle{vangelis}



\begin{document}

\begin{center}
  \textcolor{Col1}{\minibox{\large\bfseries{Ασκήσεις Επανάληψης}}}
\end{center}

\vspace{\baselineskip}

\subsection*{Μερική Παράγωγος - Διαφορικό}

\begin{enumerate}
  \item Να υπολογιστούν οι μερικές παράγωγοι 1ης τάξης των παρακάτω συναρτήσεων.
    \begin{enumerate}[i)]
      \item $ f(x,y) = -3x^{3}+8xy^{2}+y^{3} $ 
        \hfill Απ: $ \pdv{f}{x} = -9x^{2}+8y^{2} $, $ \pdv{f}{y} = 16xy+3y^{2} $
      \item $ f(x,y) = \mathrm{e}^{5x^{3}-2y^{2}} $ \hfill Απ: $ \pdv{f}{x} = 15x^{2}
        \mathrm{e}^{5x^{3}-2y^{2}} $, $ \pdv{f}{y} = -4y \mathrm{e}^{5x^{3}-2y^{2}} $ 
      \item $ f(x,y) = \frac{2}{3} \frac{x^{2}}{y^{3}} $ \hfill Απ: $ \pdv{f}{x}
        = \frac{4}{3} \frac{x}{y^{3}} $, $ \pdv{f}{y} = -2 \frac{x^{2}}{y^{4}} $
    \end{enumerate}

  \item Δίνεται η συνάρτηση παραγωγής $ Q(K,L) = 96K^{0.3}L^{0.7} $.  
    \begin{enumerate}[i)]
      \item Να υπολογιστεί το \textbf{οριακό προϊόν κεφαλαίου} ($ MP_{K} $) και
        \textbf{εργασίας} ($ MP_{L} $).
      \item Να υπολογιστεί ο \textbf{οριακός λόγος τεχνικής υποκατάστασης} ($ MRTS $)
    \end{enumerate}
    \textcolor{Col1}{Υπόδειξη:} $ MP_{K} = \pdv{Q}{K}, 
    \; MP_{L} = \pdv{Q}{L} $  
    \hfill Απ: $ MP_{K} = 28.8 (\frac{L}{K})^{0.7} $, $ MP_{L} = 67.2
    (\frac{K}{L})^{0.3} $  

    \textcolor{Col1}{Υπόδειξη:} $ MRTS = \frac{MP_{L}}{MP_{k}} $
    \hfill Απ: $ MRTS = 2.33 \frac{K}{L} $  

  \item Να υπολογιστεί το \textbf{διάνυσμα βαθμίδας} ($ \grad f $) της συνάρτησης 
    $ f(x,y) = 2x^{3}+y^{2} $.

    \textcolor{Col1}{Υπόδειξη:} $ \grad f = \bigl(\pdv{f}{x} , \pdv{f}{y}\bigr) $ 
    \hfill Απ: $ \grad Q = (6x^{2},2y) $ 

  \item Για τη συνάρτηση παραγωγής Cobb-Douglas $ Q=60K^{3/4}L^{1/4} $ να
    υπολογίσετε τη \textbf{μεταβολή} στην παραγωγή από την αύξηση του κεφαλαίου 
    κατά 2 μονάδες και της εργασίας κατά 1 μονάδα αν αρχικά το κεφάλαιο είναι $ K=81 $ 
    και η εργασία $ L=16 $.

    \textcolor{Col1}{Υπόδειξη:} $ \Delta Q \approx dQ $
    \hfill Απ: $ \Delta Q \approx 110,625 $ 
\end{enumerate}

\subsection*{Βελτιστοποίηση - Ακρότατα}

\begin{enumerate}
  \item Να υπολογιστούν τα \textbf{ακρότατα} των παρακάτω συναρτήσεων.
    \begin{enumerate}[i)]
      \item $ f(x,y) = 2x^{2}+y^{2} -12x-8y+50 $ \hfill Απ: $ f_{min}(3,4) = 16 $ 
      \item $ f(x,y,z) = x^{2}+y^{2}+3z^{2}+xz-y+yz $ \hfill Απ: $ f_{min}(\frac{1}{20}
        , \frac{11}{20} , - \frac{2}{20}) = - \frac{11}{40} $ 
    \end{enumerate}

  \item Μια επιχείρηση δρα υπό συνθήκες τέλειου ανταγωνισμού και πουλάει δύο αγαθά, $A$
    και $B$ προς $ 70 $ και $ 50 $ ευρώ αντίστοιχα. Το συνολικό κόστος παραγωγής αυτών 
    των αγαθών είναι 
    \[
      TC = q_{1}^{2}+ q_{1}q_{2} + q_{2}^{2}
    \]
    όπου $ q_{1}, q_{2} $ συμβολίζουν τα επίπεδα παραγωγής των $A$ και $B$. Να 
    υπολογίσετε το \textbf{μέγιστο κέρδος} της επιχείρησης.

    {\flushleft{\textcolor{Col1}{Υπόδειξη:} Τέλειος Ανταγωνισμός $=$ σταθερές τιμές των
    προϊόντων}}
    \hfill Απ: $ \Pi_{max}(30,10) = 1300$ 

  \item Μια επιχείρηση που δρα υπό καθεστώς μονοπωλίου πουλάει το προϊόν της σε δύο 
    απομονωμένες αγορές με συναρτήσεις ζήτησης $ p_{1}=32- q_{1} $ και $ p_{2}=40-2 q_{2}
    $. Η συνάρτηση συνολικού κόστους είναι 
    \[
      TC = 4(q_{1}+q_{2}) 
    \] 
    Να υπολογίσετε το \textbf{μέγιστο κέρδος} της επιχείρησης.

    {\flushleft{\textcolor{Col1}{Υπόδειξη:} Μονοπώλιο $=$ Οι τιμές καθορίζονται από το
    επίπεδο παραγωγής με σκοπό το κέρδος}}

    \hfill Απ: $ \Pi_{max}(14,9) = 358$ 

  \item Χρησιμοποιείστε Πολλαπλασιαστές \textbf{Lagrange} για να βρείτε την 
    μέγιστη τιμή της συνάρτησης $ f(x,y) = x+2yx $ υπό τον περιορισμό $ x+2y=5 $.

    \hfill Απ: $ f_{max}(3,1)=9 $ 

  \item Η συνάρτηση παραγωγής μιας επιχείρησης είναι $ Q=KL $. Το κόστος για μία μονάδα 
    κεφαλαίου και μια μονάδα εργασίας είναι 2 και 1 ευρώ αντίστοιχα. Βρείτε το μέγιστο 
    επίπεδο παραγωγής αν το συνολικό κόστος κεφαλαίου και εργασίας είναι 6 ευρώ.

    \flushleft{\textcolor{Col1}{Υπόδειξη:} Περιορισμός $= 2K+L=6$  }
    \hfill Απ: $ Q_{max}(\frac{3}{2} , 3) = \frac{9}{2} $ 
\end{enumerate}


\subsection*{Πίνακες - Γραμμικά Συστήματα}

\begin{enumerate}
  \item Να υπολογιστεί ο \textbf{αντίστροφος} του πίνακα $A = \begin{pmatrix*}[r]
      1 & 3 & 3 \\
      1 & 4 & 3 \\
      1 & 3 & 4 
    \end{pmatrix*}$ \hfill Απ: $ A^{-1} = 
    \begin{pmatrix*}[r]
      7 & -3 & -3 \\
      -1 & 1 & 0 \\
      -1 & 0 & 1
    \end{pmatrix*}
    $ 

  \item Να λυθεί το παρακάτω σύστημα ως προς τιμές των 2 προϊόντων, με τη μέθοδο του 
    \textbf{αντίστροφου} πίνακα
    \[
      \left.
        \begin{matrix}
          9 p_1+p_2=43 \\
          2 p_1+7p_2=57
        \end{matrix} 
      \right\}
    \]
    \hfill Απ: $ |A|=61$, $ A^{-1} = \frac{1}{61} 
    \begin{pmatrix*}[r]
      7 & -1 \\
      -2 & 9
    \end{pmatrix*}
    $, $ p_1=4=, p_2=7 $

  \item Να λυθεί το παρακάτω σύστημα με τη μέθοδο \textbf{Crammer} και 
    με τη μέθοδο \textbf{Gauss.}
    \[
      \left.
        \begin{matrix}
          4x+y+3z=8 \\
          -2x+5y+z=4 \\
          3x+2y+4z=9
        \end{matrix} 
      \right\}
    \]
    \hfill Απ: $ |A|=|A_x|=|A_y|=|A_z|=26 $, $ x=y=z=1 $
\end{enumerate}

\subsection*{Ελαστικότητα}

\begin{enumerate}
  \item Δίνεται η παρακάτω συνάρτηση ζήτησης, όπου $P_{A}=20$ και $ P_{B}=30 $ 
    είναι οι τιμές 2 προϊόντων και $M=5000$ είναι το εισόδημα
    \[
      Q=500-3P_{A}-2P_{B}+0.01M
    \]
    \begin{enumerate}[i)]
      \item Να υπολογίσετε την \textbf{ελαστικότητα ζήτησης} του προϊόντος $A$. 
        \hfill Απ: $ \varepsilon_{P_A} = -0.14 $
      \item Να υπολογίσετε την \textbf{σταυροειδή ελαστικότητα ζήτησης} του προϊόντος 
        $A$.
        \hfill Απ: $ \varepsilon_{P_B} = -0.14 $
      \item Να υπολογίσετε την \textbf{εισοδηματική ελαστικότητα ζήτησης} του προϊόντος $A$.
        \hfill Απ: $ \varepsilon_{M} = 0.12 $
      \item Αν το εισόδημα αυξηθεί κατά $ 5\% $ υπολογίστε την ποσοστιαία μεταβολή 
        της ζήτησης. \hfill Απ: $ 0.6\% $ 
    \end{enumerate}

\end{enumerate}

\end{document}
