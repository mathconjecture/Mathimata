\documentclass[a4paper,12pt]{article}
\usepackage{etex}
%%%%%%%%%%%%%%%%%%%%%%%%%%%%%%%%%%%%%%
% Babel language package
\usepackage[english,greek]{babel}
% Inputenc font encoding
\usepackage[utf8]{inputenc}
%%%%%%%%%%%%%%%%%%%%%%%%%%%%%%%%%%%%%%

%%%%% math packages %%%%%%%%%%%%%%%%%%
\usepackage{amsmath}
\usepackage{amssymb}
\usepackage{amsfonts}
\usepackage{amsthm}
\usepackage{proof}

\usepackage{physics}

%%%%%%% symbols packages %%%%%%%%%%%%%%
\usepackage{bm} %for use \bm instead \boldsymbol in math mode 
\usepackage{dsfont}
\usepackage{stmaryrd}
%%%%%%%%%%%%%%%%%%%%%%%%%%%%%%%%%%%%%%%


%%%%%% graphicx %%%%%%%%%%%%%%%%%%%%%%%
\usepackage{graphicx}
\usepackage{color}
%\usepackage{xypic}
\usepackage[all]{xy}
\usepackage{calc}
\usepackage{booktabs}
\usepackage{minibox}
%%%%%%%%%%%%%%%%%%%%%%%%%%%%%%%%%%%%%%%

\usepackage{enumerate}

\usepackage{fancyhdr}
%%%%% header and footer rule %%%%%%%%%
\setlength{\headheight}{14pt}
\renewcommand{\headrulewidth}{0pt}
\renewcommand{\footrulewidth}{0pt}
\fancypagestyle{plain}{\fancyhf{}
\fancyhead{}
\lfoot{}
\rfoot{\small \thepage}}
\fancypagestyle{vangelis}{\fancyhf{}
\rhead{\small \leftmark}
\lhead{\small }
\lfoot{}
\rfoot{\small \thepage}}
%%%%%%%%%%%%%%%%%%%%%%%%%%%%%%%%%%%%%%%

\usepackage{hyperref}
\usepackage{url}
%%%%%%% hyperref settings %%%%%%%%%%%%
\hypersetup{pdfpagemode=UseOutlines,hidelinks,
bookmarksopen=true,
pdfdisplaydoctitle=true,
pdfstartview=Fit,
unicode=true,
pdfpagelayout=OneColumn,
}
%%%%%%%%%%%%%%%%%%%%%%%%%%%%%%%%%%%%%%

\usepackage[space]{grffile}

\usepackage{geometry}
\geometry{left=25.63mm,right=25.63mm,top=36.25mm,bottom=36.25mm,footskip=24.16mm,headsep=24.16mm}

%\usepackage[explicit]{titlesec}
%%%%%% titlesec settings %%%%%%%%%%%%%
%\titleformat{\chapter}[block]{\LARGE\sc\bfseries}{\thechapter.}{1ex}{#1}
%\titlespacing*{\chapter}{0cm}{0cm}{36pt}[0ex]
%\titleformat{\section}[block]{\Large\bfseries}{\thesection.}{1ex}{#1}
%\titlespacing*{\section}{0cm}{34.56pt}{17.28pt}[0ex]
%\titleformat{\subsection}[block]{\large\bfseries{\thesubsection.}{1ex}{#1}
%\titlespacing*{\subsection}{0pt}{28.80pt}{14.40pt}[0ex]
%%%%%%%%%%%%%%%%%%%%%%%%%%%%%%%%%%%%%%

%%%%%%%%% My Theorems %%%%%%%%%%%%%%%%%%
\newtheorem{thm}{Θεώρημα}[section]
\newtheorem{cor}[thm]{Πόρισμα}
\newtheorem{lem}[thm]{λήμμα}
\theoremstyle{definition}
\newtheorem{dfn}{Ορισμός}[section]
\newtheorem{dfns}[dfn]{Ορισμοί}
\theoremstyle{remark}
\newtheorem{remark}{Παρατήρηση}[section]
\newtheorem{remarks}[remark]{Παρατηρήσεις}
%%%%%%%%%%%%%%%%%%%%%%%%%%%%%%%%%%%%%%%




\input{definitions_ask.tex}

\usepackage{array}
\pagestyle{askhseis}
\everymath{\displaystyle}

\begin{document}

\begin{center}
  \minibox{\large \bfseries \textcolor{Col1}{Ασκήσεις Καμπύλες - Τρίεδρο Frenet}}
\end{center}

\vspace{\baselineskip}

\begin{enumerate}
  \item Θεωρούμε την καμπύλη $\mathbf{r}(t) = (-3\sin 2t) \mathbf{i} + (-3\cos 2t)
    \mathbf{j}$ , $t\in\mathbb{R}$. 
    Να βρεθούν τα μοναδιαία διανύσματα $\mathbf{T},\mathbf{N}$ και η καμπυλότητα.

    \hfill Απ: \begin{tabular}{l}
      $ \mathbf{T} = -\cos 2t \mathbf{i} + \sin 2t \mathbf{j} $, \quad 
      $ \mathbf{N} = \sin 2t \mathbf{i} + \cos 2t \mathbf{j} $, \quad
      $ \mathbf{\kappa} = {1}/{3} $
    \end{tabular}

  \item Θεωρούμε την καμπύλη $\mathbf{r}(t) = 2\cos t \mathbf{i} + 2\sin t \mathbf{j}+ 
    3t \mathbf{k}$ , $t\in\mathbb{R}$. Να βρεθούν τα μοναδιαία διανύσματα 
    $\mathbf{T},\mathbf{N},\mathbf{B}$.

    \hfill Απ: 
    \begin{tabular}{l}
      $\mathbf{T} = \frac{-2\sin t}{\sqrt{13}} \mathbf{i} + \frac{2\cos t}{\sqrt{13}} 
      \mathbf{j} + \frac{3}{\sqrt{13}} \mathbf{k}$, \quad
      $ \mathbf{N} = -\cos t \mathbf{i} + -\sin t \mathbf{j} + 0 \mathbf{k}$, \quad
      $\mathbf{B} = \frac{3\sin t}{\sqrt{13}} \mathbf{i} + \frac{-3\cos
      t}{\sqrt{13}} \mathbf{j} + \frac{2}{\sqrt{13}} \mathbf{k}$
    \end{tabular}

  \item Θεωρούμε την καμπύλη $ \mathbf{r}(t) = (3t-t^3)\mathbf{i}+(3t^2)\mathbf{j}+
    (3t+t^3)\mathbf{k} $ , $t\in\mathbb{R}$. 
    Να βρεθούν τα μοναδιαία διανύσματα $\mathbf{T},\mathbf{N}, \mathbf{B}$ 
    και η καμπυλότητα.

    \hfill Απ: 
    \begin{tabular}{l}
      $\mathbf{T}=\frac{(1-t^2)}{\sqrt{2}(1+t^2)}\mathbf{i}+\frac{2t}{\sqrt{2}(1+t^2)}\mathbf{j}+\frac{1}{\sqrt{2}} \mathbf{k}$, $\mathbf{N}=
      \frac{-2t}{1+t^2}\mathbf{i}+\frac{1-t^2}{1+t^2}\mathbf{j}$,
      $\mathbf{B}=\frac{t^2-1}{\sqrt{2}(1+t^2)}\mathbf{i}+\frac{-2t}{\sqrt{2}(1+t^2)}\mathbf{j}+\frac{1}{\sqrt{2}}
      \mathbf{k}$ \\
      $\kappa = \frac{1}{3(1+t^2)^2}$
    \end{tabular}

  \item Να δείξετε ότι ο κύκλος ακτίνας $ a $ έχει σταθερή καμπυλότητα και ίση με $
    {1}/{a} $.


  \item Δίνεται η καμπύλη $\mathbf{r} = b\cos\frac{z}{a} 
    \mathbf{i} + b\sin \frac{z}{a} \mathbf{j} + z \mathbf{k}$.
    \begin{enumerate}[i)]
      \item Να γραφεί με διανυσματική παραμετρική μορφή με παράμετρο το μήκος τόξου 
        $s$, όταν η αρχή είναι το σημείο $z=0$.
      \item Να βρεθεί το μοναδιαίο εφαπτόμενο και κάθετο διάνυσμα.
      \item Να βρεθεί η καμπυλότητα και η στρέψη.
    \end{enumerate}

    \hfill Απ: 
    \begin{enumerate*}[i)]
      \item $ \mathbf{r}(s) = b\cos\frac{s}{\sqrt{a^{2}+b^{2}}} \mathbf{i} +
        b\sin\frac{s}{\sqrt{a^{2}+b^{2}}} \mathbf{j} + \frac{as}{\sqrt{a^{2}+b^{2}}}
        \mathbf{k} $ 
      \item $ \mathbf{\kappa}= \frac{b}{\sqrt{a^{2}+b^{2}}} $
      \item $ \mathbf{\sigma}=\frac{a}{\sqrt{a^{2}+b^{2}}}$
    \end{enumerate*}

  \item  Θεωρούμε τις παρακάτω καμπύλες. Να γραφούν οι εξισώσεις με παράμετρο 
    το μήκος τόξου $s$.
    \begin{enumerate}[i)]
      \item  $\mathbf{r_{1}}(t) = \cos t \mathbf{i} + \sin t \mathbf{j} + t \mathbf{k}$, 
        $t\in [0,2\pi]$
      \item  $\mathbf{r_{2}}(t) = e^{t}\cos t \mathbf{i} + e^{t}\sin t \mathbf{j} + e^t
        \mathbf{k}$, $t\in [0,\pi]$
    \end{enumerate}
    Για την $ \mathbf{r_{2}}(t) $ να δείξετε ότι κάθε διάνυσμα του τριέδρου 
    Frenet της καμπύλης, στο τυχαίο σημείο, σχηματίζει σταθερή γωνία με τον 
    άξονα $ z $.

    \hfill  Απ: \begin{tabular}{l}
      $   \rm{i)}\quad \mathbf{r_{1}}(s) = \cos \frac{s}{\sqrt{2}} \mathbf{i} + \sin
      \frac{s}{\sqrt{2}} \mathbf{j} + \frac{s}{\sqrt{2}} \mathbf{k} $ \\
      $\rm{ii)}\quad \mathbf{r_{2}}(s) = {\color{red}q}\cos\ln({{\color{red}q}}) 
      \mathbf{i} + {\color{red}q}\sin\ln({{\color{red}q}}) \mathbf{j} + 
      {\color{red}q} \mathbf{k},\;\text{όπου}\; 
      {\color{red}{q}}=\left(\frac{s}{\sqrt{3}}+1\right)$
    \end{tabular}

  \item Θεωρούμε την καμπύλη $ \mathbf{r}(t) = \cos{t} \mathbf{i} + 
    \sin{t} \mathbf{j} + t \mathbf{k}$. 
    \begin{enumerate}[i)]
      \item Να βρεθούν τα μοναδιαία διανύσματα $\mathbf{T},\mathbf{N}, \mathbf{B}$.
      \item Να βρεθούν οι εξισώσεις της εφαπτόμενης, της 1ης και 2ης κάθετης της 
        καμπύλης στο τυχαίο σημείο $ t_{0} $.
      \item Να βρεθούν οι εξισώσεις των τριών επιπέδων του τριέδρου Frenet της καμπύλης 
        στο σημειο για $ t = \frac{\pi}{2} $.
    \end{enumerate}

    \hfill Απ: \begin{tabular}{l}
      $ \mathrm{i)} \quad \mathbf{T} = - \frac{\sin{t}}{\sqrt{2}} \mathbf{i} + 
      \frac{\cos{t}}{\sqrt{2}} \mathbf{j} + \frac{1}{\sqrt{2}} \mathbf{k} $, \quad
      $ \phantom{\mathrm{i)}} \quad \mathbf{N} = \phantom{-}\frac{\sin{t}}{\sqrt{2}} 
      \mathbf{i} - \frac{\cos{t}}{\sqrt{2}} \mathbf{j} + \frac{1}{\sqrt{2}} 
      \mathbf{k} $, \quad
      $ \phantom{\mathrm{i)}} \quad \mathbf{B} = - \cos{t} \mathbf{i} - \sin{t} 
      \mathbf{j} + 0 \mathbf{k} $ \\
      $ \mathrm{ii)} \quad \text{ε}: \; \mathbf{R}(t) = \left(\cos{t_{0}} - t 
        \frac{\sin{t_{0}}}{\sqrt{2}}\right) \mathbf{i} + \left(\sin{t_{0}} + 
      t\frac{\cos{t_{0}}}{\sqrt{2} }\right) \mathbf{j} + 
      \left(t_{0} + t\frac{1}{\sqrt{2}}\right) \mathbf{k}$ \\
      $ \phantom{\mathrm{ii)}} \quad \text{1η:} \; \mathbf{R}(t) = 
      (\cos{t_{0}} - t \cos{t_{0}} ) \mathbf{i} + (\sin{t_{0}} - t \sin{t_{0}} )
      \mathbf{j}+ t_{0} \mathbf{k}$ \\
      $ \phantom{\mathrm{ii)}} \quad \text{2η:} \; \mathbf{R}(t) = 
      (\cos{t_{0}} + t \frac{\sin{t_{0}}}{\sqrt{2}}) \mathbf{i} + (\sin{t_{0}} - t
      \frac{\cos{t_{0}}}{\sqrt{2}}) \mathbf{j}+ (t_{0} + t \frac{1}{\sqrt{2}}) 
      \mathbf{k}$ \\
      $\mathrm{iii)} \quad  \text{εγγ.:} \; x+z = \frac{\pi}{2}, \quad \text{καθ.:} \; 
      -x + z = \frac{\pi}{2}, \quad \text{ευθ.:} \; y=1 $ 
    \end{tabular}

\end{enumerate}



\end{document}
