\documentclass[a4paper,table]{report}
\input{preamble_ask.tex}
\input{definitions_ask.tex}
\input{tikz.tex}


\everymath{\displaystyle}
\thispagestyle{vangelis}

\begin{document}

\begin{center}
  \minibox{\large \bfseries \textcolor{Col1}{Προβλήματα Ακροτάτων}}
\end{center}

\vspace{\baselineskip}

\begin{enumerate}

	\item Να προσδιοριστούν οι διαστάσεις μιας πισίνας 
    $ \SI{32}{m^{3}} $ με τετραγωνική βάση έτσι ώστε η επιφάνεια των εσωτερικών 
    τοίχων και του πυθμένα να είναι ελάχιστη. 

		\hfill Απ: $x=4, y=2$

	\item Θεωρούμε τρίγωνο ΑΒΓ με  ΒΓ$=a $ και ύψος ΑΔ$=h$. Επίσης θεωρούμε τα
		εγγεγραμένα σε αυτό ορθογώνια των οποίων η μία πλευρά βρίσκεται πάνω στη
		ΒΓ. Να βρεθεί εκείνο το παραλληλόγραμμο που έχει μέγιστο εμβαδό.
		
		\hfill Απ: $ x = \frac{h}{2}, y= \frac{a}{2} $

	\item Να εγγραφεί ορθογώνιο παραλληλόγραμμο με μέγιστο εμβαδό στην έλλειψη $
		\frac{x^{2}}{a^{2}} + \frac{y^{2}}{b^{2}} = 1 $. 

		\hfill Απ: $ x = \frac{a\sqrt{2}}{2}, y = \frac{b \sqrt{2}}{2} $

	\item Ένα φύλλο ειδικού χαρτιού για μεγάλο μηχανολογικό σχέδιο, έχει
		επιφάνεια \SI{2}{m^{2}}. Στο σχέδιο που θα γίνει πρέπει να αφεθούν
		περιθώρια στην πάνω και στην κάτω πλευρά του σχεδίου, \SI{21}{cm}, ενώ
		στις πλαινές πλευρές \SI{14}{cm}. Ποιες πρέπει να είναι οι διαστάσεις
		τουτου φύλλου ώστε το καθαρό εμβαδό για σχεδίαση να είναι μέγιστο?

		\hfill $ x = \sqrt{3}, y = \frac{2 \sqrt{3}}{3} $

	\item Δυο πόλεις Α και Β βρίσκονται προς το ίδιο μέρος της όχθης ενός
		ποταμού και απέχουν απ᾽ αυτήν 10 και 15 χιλιόμετρα αντίστοιχα. Οι
		κάθετες προβολές των δύο πόλεων στην όχθη απέχουν 20 χιλιόμετρα. Οι δύο
		πόλεις πρέπει να εφοδιαστούν με νερό από ένα εργοστάσιο που θα
		κατασκευαστεί στην όχθη του ποταμού. Σε ποιο σημείο της όχθης πρέπει να
		κατασκευαστεί το εργοστάσιο ώστε για τους αγωγούς που θα συνδέσουν αυτό
		με τις πόλεις να έχουμε το ελάχιστο κόστος?

		\hfill Απ: $ x = \SI{8}{km} $

	\item Να εγγραφεί σε κύκλο διαμέτρου $d$ ορθογώνιο με βάση $b$ και ύψος $h$
		ώστε το γινόμενο $ b\cdot h^{2} $ να είναι μέγιστο.

		\hfill Απ: $b = \frac{\sqrt{3}}{3} d$, $ h = \sqrt{\frac{2}{3}
		} d $

	\item Να προσδιοριστεί ο λόγος της ακτίνας προς το ύψος ενός κλειστού
		κυκλικού ορθού κυλινδρικού δοχείου χωρητικότητας $c$ μονάδων όγκου αν
		θέλουμε η επιφάνεια του δοχείου να είναι ελάχιστη.

		\hfill Απ: $\frac{r}{h} = \frac{1}{2} $
		
\end{enumerate}

\end{document}
