\input{preamble_ask.tex}
\input{definitions_ask.tex}


\pagestyle{askhseis}

% \everymath{\displaystyle}




\begin{document}

\begin{center}
  {\color{Col1}\bfseries\large Μερικές Παράγωγοι}
\end{center} 

\section*{Κανόνες Παραγώγισης}

\begin{enumerate}
  \item Με τη βοήθεια των κανόνων παραγώγισης να υπολογιστούν οι μερικές 
    παράγωγοι των συναρτήσεων:

    \begin{enumerate}[i)]
      \item $f(x,y)=y\sin (xy)$ \hfill Απ: \begin{tabular}{l}
          $f_x=y^2\cos(xy)$ \\ 
          $f_y=\sin(xy)+yx\cos(xy)$
        \end{tabular}

      \item $f(x,y)=y\ln(x+y)$\hfill Απ: \begin{tabular}{l}
          $f_x=\frac{y}{x+y}$ \\ 
          $f_y=\ln(x+y)+\frac{y}{x+y}$
        \end{tabular}

      \item $f(x,y)=\arcsin(\frac{x}{y})$\hfill Απ: \begin{tabular}{l}
          $f_x=\frac{1}{y^2-x^2}$ \\ 
          $f_y=-\frac{x}{y\sqrt{y^2-x^2}}$
        \end{tabular}
      \item $ f(x,y,z) = (x+y^{2}) \sin{(xz)} $ \hfill Απ: \begin{tabular}{l}
          $ f_{x} = \sin{(xz)} + z(x+y^{2}) \sin{(xz)} $ \\
          $ f_{y} = 2y \sin{(xz)} $ \\
          $ f_{z} = x(x+y^{2}) \sin{(xz)} $
        \end{tabular} 
    \end{enumerate}

  \item Έστω η συνάρτηση $f(x,y)=\ln\left(\cos y+x\cos x\right)$.  Να υπολογισθεί 
    η $ f_{xy} $ στο σημείο $(\pi,-\pi/2)$.  \hfill Απ: $\frac{1}{\pi^2}$

    % \item Έστω η συνάρτηση $ f(x,y) = x^{y} $. Να δείξετε ότι ισχύει ότι το θεώρημα 
    % Schwartz, δηλαδή ότι $ f_{xy} = f_{yx} $.

    %hlektr
    %\item Να δείξετε ότι η συνάρτηση $ f(x,y) = (y+3x)^{1/2} - 
    %    (y-3x)^{2} $ ικανοποιεί τη σχέση $ f_{xx} - 9 f_{yy} = 0 $.

    %    %span
  \item Να δείξετε ότι η συνάρτηση $ f(x,y) = \cos{(x+y)} + \cos{(x-y)} $ 
    επαληθεύει τη διαφορική εξίσωση $ f_{xx} - f_{yy} = 0 $.

  \item Να δείξετε ότι η συνάρτηση $ f(x,y) = x \arctan{\frac{y}{x}} $ 
    επαληθεύει τη διαφορική εξίσωση $ x^{2} f_{xx} + 2xyf_{xy} + y^{2} f_{yy} = 0 $ 
\end{enumerate}


\section*{Διαφορικό}

\begin{enumerate}
  \item Να βρεθεί το ολικό διαφορικό 1ης τάξης, της συνάρτησης 
    $f(x,y)=\ln(xy)+\cos(y^2)$ 

    \hfill Απ: $df=\frac{dx}{x}+\left(\frac{1}{y}-2y\sin(y^2)\right)dy$

  \item Να βρεθεί το ολικό διαφορικό 1ης τάξης, της συνάρτησης 
    $ f(x,y) = \arctan(\frac{ x+y }{ x-y }) $, αν $ x>0 $ και $ y>0 $.

    \hfill Απ: $df = \frac{ -ydx + xdy }{ x^{2} + y^{2} } $ 

  % \item Να βρεθεί το ολικό διαφορικό της συνάρτησης $ f(x,y) = x^{y} \cdot y^{x} $, 
  %   αν $ x>0$ και $ y>0 $.

    % \hfill Απ: $df =  (x^{y-1}\cdot y^{x+1} + x^{y}\cdot y^{x} \ln{y} )dx 
    % + (x^{y}\cdot y^{x} \ln{x} + x^{y+1} \cdot y^{x-1})dy $ 

\end{enumerate}


\section*{Εφαρμογές του Διαφορικού}

\begin{enumerate}
  \item Να υπολογιστεί κατά προσέγγιση η τιμή των παραστάσεων
    \begin{enumerate}[i)]
      \item $A = (1,02)^{3,01} $ \hfill Απ: $ A \approx 1,06 $ 
      \item $B =  \sqrt{ 9(1,95)^{2} + (8,1)^{2} } $ 
        \hfill Απ: $ B \approx 9,99 $ 
    \end{enumerate}

  \item Θέλουμε να κατασκευάσουμε ένα ορθογώνιο παραλληλεπίπεδο με ακμές $ x = a $, 
    $ y=b $ και $ z=c $. Αν τα σφάλματα που έγιναν κατά την κατασκευή των ακμών 
    είναι $ \Delta x = 0,01a $, $ \Delta y = -0,02b $ και $ \Delta z = 0,03c $, 
    τότε να υπολογίσετε το σφάλμα που έγινε στον όγκο του παραλληλεπιπέδου.
    \hfill Απ: $ 0,02abc $ 
\end{enumerate}


\section*{Κλίση - Λαπλασιανή - Αρμονικές Συναρτήσεις}

\begin{enumerate}
  \item Να υπολογίσετε την κλίση των παρακάτω συναρτήσεων.
    \begin{enumerate}[i)]
      \item $ f(x,y) = x^{2} \mathrm{e}^{y+x^{3}} $ 
        \hfill Απ: $ \grad(f) = (2x\mathrm{e}^{y+x^{3}}+3x^{4}\mathrm{e}^{y+x^{3}}, 
        x^{2} \mathrm{e}^{y+x^{3}}) $ 
      \item $ f(x,y) = \frac{1}{y+x^{3}} $ 
        \hfill Απ: $\grad(f) =\left(\frac{-3x^{2}}{(y+x^{3})^{2}},
        \frac{-1}{(y+x^{3})^{2}}\right)$ 
      \item $ f(x,y) = x^{2} y \mathrm{e}^{zy}  $ 
        \hfill Απ: $ \grad(f) = (2xy\mathrm{e}^{zy},
        x^{2}\mathrm{e}^{zy}+x^{2}yz\mathrm{e}^{zy},x^{2}y^{2} \mathrm{e}^{zy})$ 
    \end{enumerate}

  \item Αν $ d\mathbf{r} = dx \mathbf{i}+ dy \mathbf{j}$, τότε να αποδείξετε ότι 
    $ \grad{f} \cdot d \mathbf{r} = df $

  \item Να υπολογίσετε τη Λαπλασιανή των παρακάτω συναρτήσεων.
    \begin{enumerate}[i)]
      \item $ f(x,y) = x^{2} \mathrm{e}^{y+x^{3}} $ 
        \hfill Απ: $ \grad^{2} f = 2 \mathrm{e}^{y+x^{3}} + 6x^{3} \mathrm{e}^{y+x^{3}} 
        + 12x^{3} \mathrm{e}^{y+x^{3}} + 9x^{6} \mathrm{e}^{y+x^{3}} + x^{2}
        \mathrm{e}^{y+x^{3}} $ 
      \item $ f(x,y) = \frac{1}{y+x^{3}} $
        \hfill Απ: $ \grad^{2}f = \frac{-6(y+x^{3})+18x^{4}+6x^{2}}{(y+x^{3})^{3}} $ 
    \end{enumerate}

  \item Να δείξετε ότι οι παρακάτω συναρτήσεις είναι αρμονικές:
    \begin{enumerate}[(i)]
      \item $ f(x,y) = x^{3}-3xy^{2} $
      \item $ f(x,y) = \ln(x^{2} + y^{2}) $
    \end{enumerate}

  \item Να αποδείξετε ότι αν μία συνάρτηση $f(x,y)$, που έχει συνεχείς 
    μερικές παραγώγους 2ης τάξης είναι αρμονική, τότε και οι συναρτήσεις 
    \begin{enumerate*}[i),itemjoin=\hspace{7pt}]
      \item $ f_{x} $ 
      \item $ f_{y} $
      \item $ xf_{x}+yf_{y} $
      \item $ xf_{x}-yf_{y} $
    \end{enumerate*}
    είναι αρμονικές.
\end{enumerate}


\section*{Παράγωγος Κατά Κατεύθυνση}

\begin{enumerate}
  \item Να υπολογίσετε την παράγωγο κατά κατεύθυνση, των παρακάτω συναρτήσεων στο σημείο 
    και προς της κατεύθυνση που σας δίνεται κάθε φορά.
    \begin{enumerate}[i)]
      % \item $ f(x,y) = y \mathrm{e}^{-x} $, στο σημείο $ P_{0}(0,4) $ και προς την
      %   κατεύθυνση της \textbf{γωνίας} $ \theta = \frac{2\pi}{3} $. 
      %   \hfill Απ: $ 2 + \sqrt{3} / 2 $ 
      \item $ f(x,y) = \sin{(2x+3y)} $, στο σημείο $ P_{0}(-6,4) $ και προς την
        κατεύθυνση  $ \mathbf{u} = \sqrt{3} /2 \mathbf{i} - 1/2 \mathbf{j}$. 
        \hfill Απ: $ \sqrt{3} - 3/2 $ 
      \item $ f(x,y) = \sqrt{xy} $, στο σημείο $ P_{0}(2,8) $ και προς την
        κατεύθυνση του \textbf{σημείου} $P(5,4) $. \hfill Απ: $ 2 / 5 $ 
      \item $ f(x,y,z) = x \mathrm{e}^{2yz} $, στο σημείο $ P_{0}(3,0,2) $ και προς την
        κατεύθυνση  $ \mathbf{u} = (2/3, -2/3 , 1/3) $. 
        \hfill Απ: $-22/3$ 
      \item $ f(x,y,z) = xy+yz+zx $, στο σημείο $ P_{0}(1,-1,3) $ και προς την
        κατεύθυνση  του \textbf{σημείου} $P(2,4,5)$. 

        \hfill Απ: $ 22/\sqrt{30} $
    \end{enumerate}

  \item Να υπολογίσετε τον μέγιστο ρυθμό μεταβολής των παρακάτω συναρτήσεων, στο σημείο 
    που σας δίνεται κάθε φορά, καθώς και την κατεύθυνση προς την οποία αυτός 
    παρατηρείται.
    \begin{enumerate}[i)]
      \item $ f(x,y) = y^{2}/x $, στο σημείο $ P_{0}(2,4) $.
        \hfill Απ: $4 \sqrt{2}$ 
      \item $ f(x,y,z) = \sqrt{x^{2}+y^{2}+z^{2}} $, στο σημείο $ P_{0}(3,6,-2) $.
        \hfill Απ: $ 1 $
    \end{enumerate}

  \item 
    \begin{enumerate}[i)]
      \item Να δείξετε ότι μια συνάρτηση $ f(x,y) $, μειώνεται με το γρηγορότερο ρυθμό
        στο τυχαίο σημείο $ P_{0}(x,y) $, προς την κατεύθυνση που είναι αντίθετη από το 
        διάνυσμα της κλίσης της $ - \grad f(P_{0}) $.
        \hfill Απ: $ \theta = \pi $  
      \item Χρησιμοποιείστε το $ \rm{i)} $ ερώτημα, για να προσδιορίσετε την κατεύθυνση,
        προς την οποία η συνάρτηση με τύπο $ f(x,y) = x^{4}y-x^{2}y^{3} $ μειώνεται με 
        το γρηγορότερο ρυθμό στο σημείο $ P_{0}(2,-3) $. 
        \hfill Απ: $ - \grad f_{P_{0}} = (-12,92) $
    \end{enumerate}

  \item Η παράγωγος της συνάρτησης $ f(x,y) $, στο σημείο $ P(1,2) $ και προς την 
    κατεύθυνση $ \mathbf{u} = \mathbf{i} + \mathbf{j} $ είναι $ 2 \sqrt{2} $, ενώ 
    προς την κατεύθυνση $ \mathbf{v} = -2 \mathbf{j} $ είναι $ -3 $. Ποια είναι 
    η παράγωγος της $f$ προς την κατεύθυνση $ \mathbf{w} = - \mathbf{i}- 2\mathbf{j} $, 
    στο ίδιο σημείο; \hfill Απ: $ -7/ \sqrt{5} $
\end{enumerate}

\section*{Πολυώνυμο Taylor}

\begin{enumerate}
  \item Να βρεθούν τα αναπτύγματα \textbf{Taylor}, μέχρι και όρους 
    \textbf{2ης τάξης}, των συναρτήσεων:

    \begin{enumerate}[i)]
      \item  $f(x,y)=y\cos{xy} $, γύρω από το σημείο 
        $ \left(1, \frac{ \pi }{ 2 }\right) $.

        \hfill Απ: $f(x,y)=-\frac{\pi^{2}}{4}(x-1) - \frac{ \pi }{ 2 } 
        \left(y - \frac{ \pi }{2 }\right) - \pi(x-1)
        \left(y-\frac{\pi}{2}\right)- \left(y- \frac{ \pi }{ 2} \right)^{2} $

      \item $ f(x,y)=e^{x}\tan{y} $ σε δυνάμεις των $ (x-1) $ και 
        $ \left(y - \frac{ \pi }{ 4 }\right) $

        \hfill Απ: $ f(x,y) = e + e(x-1) + 2e\left(y- \frac{ \pi }{ 4 }\right)
        + \frac{1}{ 2 } \left(e(x-1)^{2}+4e(x-1)\left(y- \frac{ \pi }{ 4 }
        \right) + 4e\left(y- \frac{ \pi }{ 4 } \right)^{2}\right) $
    \end{enumerate}

  \item Να βρεθεί το ανάπτυγμα \textbf{Maclaurin}, μέχρι όρους \textbf{2ης
    τάξης}, της συνάρτησης $ f(x,y) = e^{x}\ln(1+y)$.

    \hfill Απ: $ f(x,y)=y + xy - \frac{1}{ 2 } y^{2} + \frac{1}{ 2 } x^{2}y - 
    \frac{1}{ 2 } xy^{2} + \frac{1}{ 3 } y^{3} $
\end{enumerate}




\end{document}
