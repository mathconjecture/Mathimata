\documentclass[a4paper,table]{report}
\input{preamble.tex}
\input{definitions2.tex}
\input{tikz}
\input{myboxes}

\usepackage{geometry}
\geometry{left=18mm,right=18mm,top=20.00mm,bottom=32.00mm,footskip=24.16mm,headsep=24.16mm}

\newcommand{\twocolumnsidelcc}[2]{\begin{minipage}[c]{0.35\linewidth}
        #1
        \end{minipage}\hfill\begin{minipage}[c]{0.60\linewidth}
        #2
    \end{minipage}
}


\input{insbox}

\linespread{1.2}

\pagestyle{askhseis}
\usepackage{cutwin}
\everymath{\displaystyle}
\setcounter{chapter}{1}

\begin{document}

\chapter*{Ακρότατα Συναρτήσεων Πολλών Μεταβλητών}

\section{Τοπικά Ακρότατα}

\begin{mybox1}
\begin{dfn}
\item {}
  Έστω $ f \colon A \subseteq \mathbb{R}^{2} \to \mathbb{R} $, έστω 
  $ (x_{0}, y_{0}) \in A $ και έστω $R(x_{0}, y_{0}) $ περιοχή στοιχείων του $A$, 
  γύρω από το $ (x_{0}, y_{0}) $.
  \begin{enumerate}[i)]
    \item 
      Η $ f(x,y) $, έχει τοπικό ελάχιστο στο σημείο $ (x_{0}, y_{0}) $, αν 
      $ f(x_{0}, y_{0}) \leq f(x,y), \; \forall (x,y) \in R(x_{0}, y_{0}) $ 
    \item 
      Η $ f(x,y) $, έχει τοπικό μέγιστο στο σημείο $ (x_{0}, y_{0}) $, αν 
      $ f(x_{0}, y_{0}) \geq f(x,y), \; \forall (x,y) \in R(x_{0}, y_{0}) $ 
  \end{enumerate}
  Το τοπικά μέγιστο και το τοπικά ελάχιστο, ονομάζονται \textcolor{Col1}{τοπικά
  ακρότατα}. 
  Αν οι ανισότητες στον παραπάνω ορισμό, ισχύουν \textbf{για κάθε σημείο} στο πεδίο 
  ορισμού της συνάρτησης, τότε λέμε ότι η $f$ έχει \textcolor{Col1}{ολικό ακρότατο}.
\end{dfn}
\end{mybox1}

\begin{prop}\label{prop:fermat2}
\item {}
  Αν η συνάρτηση $ f(x,y) $ έχει τοπικό ακρότατο στο σημείο $ (x_{0}, y_{0}) $, 
  τότε:
  \begin{enumerate}[i)]
    \item ή υπάρχουν οι $ f_{x}(x_{0}, y_{0}) $ και $ f_{y}(x_{0}, y_{0}) $ 
      και ισχύει $ f_{x}(x_{0}, y_{0}) = f_{y}(x_{0}, y_{0} )=0 $
    \item ή μία τουλάχιστον από τις $ f_{x}(x_{0}, y_{0}) $ και 
      $ f_{y}(x_{0}, y_{0}) $ δεν υπάρχει.
  \end{enumerate}
\end{prop}

\begin{rem}
\item {}
  Το αντίστροφο της παραπάνω πρότασης δεν ισχύει. 
\end{rem}

\begin{dfn}
  Τα \textbf{εσωτερικά} σημεία του πεδίου ορισμού της $ f(x,y) $, για τα οποία 
  υπάρχουν οι μερικές παράγωγοι 1ης τάξης, και είναι ίσες με 0 ή που δεν υπάρχει 
  τουλάχιστον μία εξ αυτών, λέγονται \textcolor{Col1}{κρίσιμα} ή
  \textcolor{Col1}{στάσιμα} σημεία της $ f(x,y) $. 
\end{dfn}

\begin{rem}
  Σύμφωνα με την πρόταση~\ref{prop:fermat2} τα \textbf{κρίσιμα} σημεία της $ f(x,y) $, 
  μαζί με τα \textbf{συνοριακά} σημεία του πεδίου ορισμού της είναι θέσεις \textbf{πιθανών} ακροτάτων.
\end{rem}

Όπως γνωρίζουμε για τις συναρτήσεις μιας μεταβλητής, ότι κάθε κρίσιμο σημείο δεν είναι 
αναγκαία τοπικό ακρότατο, γιατί μπορεί να είναι σημείο καμπής, έτσι κ για τις
συναρτήσεις δύο μεταβλητών, ένα κρίσιμο σημείο, μπορεί να μην είναι τοπικό ακρότατο,
αλλά να είναι σαγματικό σημείο.


\subsection{Σαγματικά Σημεία}

Όπως γνωρίζουμε για τις συναρτήσεις μιας μεταβλητής, ότι κάθε κρίσιμο σημείο δεν είναι 
αναγκαία τοπικό ακρότατο, γιατί μπορεί να είναι σημείο καμπής, έτσι κ για τις
συναρτήσεις δύο μεταβλητών, ένα κρίσιμο σημείο, μπορεί να είναι σαγματικό σημείο.

\begin{mybox1}
\begin{dfn}
  Ένα κρίσιμο σημείο $ (x_{0}, y_{0}) $, μιας διαφορίσιμης συνάρτησης $ f(x,y) $, είναι
  \textcolor{Col1}{σαγματικό σημείο}, αν για κάθε ανοιχτό δίσκο με κέντρο το 
  $ (x_{0}, y_{0}) $, υπάρχουν σημεία $ (x,y) $ στο πεδίο ορισμού της $f$, για τα 
  οποία, άλλοτε $ f(x,y) > f(x_{0}, y_{0}) $ κι άλλοτε $ f(x,y) < f(x_{0}, y_{0}) $. 
\end{dfn}
\end{mybox1}

\begin{example}
\item {}
  Έστω η συνάρτηση $ f(x,y) = y^{2}-x^{2} $. Έχουμε ότι 
  $ f_{x}=-2x $ και $ f_{y}=2y $, άρα το $ (0,0) $ είναι το μοναδικό στάσιμο σημείο 
  της $f$. 

  \twocolumnsidelcc{
    \begin{tikzpicture}[scale=0.7]
      \begin{axis}[blue!50,samples=30,xlabel={$y$},ylabel={$x$},zlabel={$f(x,y)$}]
        \addplot3[surf,color=Col1,opacity=0.5,domain=-2:2,faceted color=black] {x^2-y^2};
      \end{axis}
  \end{tikzpicture}
  }
  {
  Παρατηρούμε, ότι για όλα τα σημεία του άξονα $x$, άρα και για κάθε 
  $ (x,0) $ κοντά στο σημείο $(0,0)$, έχουμε ότι $ f(x,y) = -x^{2} < 0 $. 
  Όμως, για όλα τα σημεία του άξονα $ y $, άρα και για κάθε 
  $ (0,y) $ κοντά στο σημείο $ (0,0) $, έχουμε ότι $ f(x,y)=y^{2} > 0 $. 
  Άρα, σε κάθε 
  περιοχή του σημείου $ (0,0) $, πάντα θα βρίσκουμε τιμές της $ f(x,y) $ που είναι
  θετικές και αρνητικές. Άρα το σημείο $ (0,0) $ δεν μπορεί να είναι τοπικό ακρότατο 
  της $f$. Σε αυτή την περίπτωση, το σημείο $ (0,0) $ λέγεται \textbf{σαγματικό
  σημείο}, και μοιάζει με το σημείο στο κέντρο μιας σέλας, 
  όπως φαίνεται και στο σχήμα, γι᾽ αυτό και πολλές φορές χαρακτηρίζεται και ως 
  σελλοειδές σημείο.
}
\end{example}


\subsection{Θεωρήματα Ακροτάτων}

\begin{dfn}
  Έστω συνάρτηση $ f(x,y) $ και έστω $ (x_{0}, y_{0}) $ κρίσιμο σημείο της $f$. 
  Ορίζουμε τις παρακάτω ορίζουσες:
  \[
    \abs{H_{1}} = f_{xx}(x_{0}, y_{0}) \quad \text{και} \quad 
    \abs{H_{2}} = 
    \begin{vmatrix*}[r]
      f_{xx}(x_{0}, y_{0}) & f_{xy}(x_{0}, y_{0}) \\
      f_{yx}(x_{0}, y_{0}) & f_{yy}(x_{0}, y_{0}) \\
    \end{vmatrix*} \overset{\text{ή}}{=}
    \begin{vmatrix}
      f_{xx} & f_{xy} \\
      f_{yx} & f_{yy}
    \end{vmatrix}_{(x_{0}, y_{0})} 
  \] 
\end{dfn}

\begin{mybox2}
\begin{thm}[Για συνάρτηση $ f(x,y) $ δύο μεταβλητών]
  \label{thm:2var}
\item {}
  Έστω $ f(x,y) $ συνάρτηση δύο μεταβλητών, ορισμένη σε ένα ανοιχτό 
  υποσύνολο $A$ του $ \mathbb{R}^{2} $, με μερικές παραγώγους 1ης και 2ης τάξης 
  ορισμένες σε μια περιοχή του κρίσιμου σημείου $ (x_{0}, y_{0}) \in A $ και 
  έστω ότι οι παράγωγοι 2ης τάξης είναι \textbf{συνεχείς} στο $ (x_{0}, y_{0}) $ και 
  $ f_{x}(x_{0}, y_{0}) = f_{y}(x_{0}, y_{0}) = 0 $. Τότε
\end{thm}

\begin{myitemize}
  \item Αν $ \abs{H_{1}} > 0 $ και $ \abs{H_{2}} > 0 $ τότε η $f$ παρουσιάζει στο 
    σημείο $ (x_{0}, y_{0}) $ \textbf{τοπικό ελάχιστο}.
  \item Αν $ \abs{H_{1}} < 0 $ και $ \abs{H_{2}} > 0 $ τότε η $f$ παρουσιάζει στο 
    σημείο $ (x_{0}, y_{0}) $ \textbf{τοπικό μέγιστο}.
  \item Αν $ \abs{H_{2}} < 0 $ τότε η $f$ δεν παρουσιάζει ακρότατο και σε αυτήν 
    την περίπτωση το σημείο λέγεται \textbf{σαγματικό}.
  \item Αν $ \abs{H_{2}} = 0 $, τότε δεν βγαίνει κάποιο συμπέρασμα σχετικά με το 
    σημείο $ (x_{0}, y_{0}) $.
\end{myitemize}
\end{mybox2}


Το προηγούμενο θεώρημα επεκτείνεται και για συναρτήσεις με τρεις μεταβλητές. Σε αυτή 
την περίπτωση, εκτός από τις ορίζουσες $ \abs{H_{1}} $ και $ \abs{H_{2}} $, ορίζεται 
ακόμη η
\[
  \abs{H_{3}} = 
  \begin{vmatrix}
    f_{xx} & f_{xy} & f_{xz} \\
    f_{yx} & f_{yy} & f_{yz} \\
    f_{zx} & f_{zy} & f_{zz}
  \end{vmatrix} 
  \hfill 
\] 

\begin{rem}
  Όλες οι παραπάνω ορίζουσες ονομάζονται \textcolor{Col1}{Εσσιανές} και περιέχουν τις 
  παραγώγους 2ης τάξης της συνάρτησης. Οι αντίστοιχοι πίνακες, ονομάζονται 
  \textcolor{Col1}{Εσσιανοί} 
  και είναι \textbf{συμμετρικοί}, αφού από τις προϋποθέσεις του θεωρήματος ισχύει το 
  θεώρημα Schwartz και επομένως $ f_{xy} = f_{yx} $, $ f_{xz} = f_{zx} $, κλπ.
\end{rem}


\section{Μεθοδολογία Εύρεσης τοπικών ακροτάτων για συνάρτηση δύο μεταβλητών}

\begin{enumerate}
  \item Βρίσκουμε όλες τις μερικές παραγώγους 1ης και 2ης τάξης.
  \item Βρίσκουμε τα κρίσιμα σημεία της συνάρτησης. Λύνουμε το σύστημα των εξισώσεων: 
    $ \begin{rcases}
      f_{x} = 0 \\
      f_{y} = 0  
    \end{rcases} $ 
  \item Για κάθε κρίσιμο σημείο, έστω $ (x_{0}, y_{0}), $ υπολογίζουμε τις Εσσιανές 
    ορίζουσες σε αυτό το σημείο και εφαρμόζουμε το θεώρημα~\ref{thm:2var}.
  \item Στην περίπτωση όπου $ \abs{H_{2}} = 0 $, τότε ακολουθούμε τον ορισμό των 
    τοπικών ακροτάτων.
    \begin{myitemize}
      \item Σχηματίζουμε τη διαφορά $ f(x,y) - f(x_{0}, y_{0}) $.
      \item Προσπαθούμε να προσδιορίσουμε το πρόσημό αυτής της διαφοράς σε 
        μια περιοχή του σημείου $ (x_{0}, y_{0}) $.
        \begin{myitemize}
          \item Αν $ f(x,y) - f(x_{0}, y_{0}) < 0, \; \forall (x,y) $ σε μια
            περιοχή του $ (x_{0}, y_{0}) $ τότε έχουμε ελάχιστο. 
          \item Αν $ f(x,y) - f(x_{0}, y_{0}) > 0, \; \forall (x,y) $ σε μια
            περιοχή του $ (x_{0}, y_{0}) $ τότε έχουμε μέγιστο. 
        \end{myitemize}
      \item Για παράδειγμα, για να εξετάσουμε το σημείο $ (0,0) $ επιλέγουμε μια 
        καμπύλη που περνά από αυτό το σημείο, όπως για παράδειγμα 
        $ y= \lambda x $ ή $ y= \lambda x^{2} $ κλπ, ώστε η διαφορά 
        $ f(x,y) - f(0, 0) $ να εξαρτάται μόνο από το $x$ και να 
        είναι πιο εύκολο να προσδιορίσουμε το πρόσημό της. Αν η διαφορά αυτή, δεν 
        διατηρεί σταθερό πρόσημο σε μια περιοχή του σημείου $ (0,0) $ τότε το σημείο 
        είναι σαγματικό.
    \end{myitemize}
\end{enumerate}


\section{Ολικά Ακρότατα}

Ένα υποσύνολο $D$ του $ \mathbb{R}^{2} $ ονομάζεται \textcolor{Col1}{κλειστό}, αν 
περιέχει και όλα τα συνοριακά του σημεία, ενώ ονομάζεται \textcolor{Col1}{φραγμένο}, αν 
υπάρχει δίσκος του $ \mathbb{R}^{2} $ που το περιέχει, δηλαδή, ουσιαστικά ότι είναι 
περιορισμένο σε έκταση.

\begin{thm}
  Αν $f(x,y)$ είναι συνεχής σε κάποιο \textbf{κλειστό} και \textbf{φραγμένο} υποσύνολο 
  $A$ του $ \mathbb{R}^{2} $, τότε η $f$ παίρνει μέγιστη και ελάχιστη τιμή στο $A$. 
  Δηλαδή, υπάρχουν σημεία $ (x_{1}, y_{1}) \in A $ και $ (x_{2}, y_{2}) \in A $ τέτοια 
  ώστε 
  \[
    f(x_{1}, y_{1}) \leq f(x,y) \leq f(x_{2}, y_{2}), \quad \forall (x,y) \in A
  \]
\end{thm}

Η μέγιστη και η ελάχιστη τιμή του προηγούμενου θεωρήματος, είναι το \textbf{ολικά
ακρότατα} της συνάρτησης, και για να τα προσδιορίσουμε ακολουθούμε τα παρακάτω βήματα.

\begin{myitemize}
  \item Βρίσκουμε τις τιμές των \textbf{κρίσιμων} σημείων της $f$ στο $A$.
  \item Βρίσκουμε τις τιμές των \textbf{ακροτάτων} της $f$ στο σύνορο του $A$. 
  \item Η μεγαλύτερη από τις τιμές που βρήκαμε στα δύο προηγούμενα βήματα 
    είναι το ολικό μέγιστο της συνάρτησης, ενώ η μικρότερη είναι το ολικό ελάχιστο.
\end{myitemize}




\chapter{Ακρότατα Υπό Συνθήκη}

\section{Πολλαπλασιαστές Lagrange}

\subsection{Ακρότατα υπό μία συνθήκη}

\enlargethispage{2\baselineskip}

\begin{thm}
  Έστω $ f(x,y) $ συνάρτηση δύο μεταβλητών, ορισμένη σε ένα ανοιχτό 
  υποσύνολο $A$ του $ \mathbb{R}^{2} $ και έστω ότι οι παράγωγοι 2ης τάξης είναι 
  συνεχείς στο $A$. Υποθέτουμε ότι για την συνάρτηση $ \phi $ οι μερικές παράγωγοι 
  1ης τάξης είναι συνεχείς στο $A$ και ικανοποιείται η συνθήκη (περιορισμός)
  \begin{equation}
    \label{eq:constr1}
    \phi (x,y) = 0
  \end{equation}
  Θεωρούμε τη συνάρτηση Lagrange που ορίζεται από τον τύπο
  \[
    L(x,y, \lambda) = f(x,y) + \lambda \phi (x,y) 
  \] 

  Τα ακρότατα $ P(x,y) $ της συνάρτησης $ f(x,y) $ υπό τον περιορισμό $ \phi (x,y)=0 $  
  θα αναζητηθούν από την λύση του συστήματος 
  \begin{equation}\label{eq:lag}
    \left.
      \begin{matrix}
        \grad L(x,y)=0 \\
        \phi (x,y)=0
      \end{matrix} 
    \right\} \Leftrightarrow 
    \left.
      \begin{matrix}
        L_{x}=0 \\
        L_{y}=0 \\
        \phi(x,y)=0
      \end{matrix} 
    \right\}
  \end{equation}
  Το σημείο $ P(x,y) $ θα είναι: 
  \begin{myitemize}
    \item \textbf{τοπικό ελάχιστο} αν $ [H(L(P)) \cdot \mathbf{u}] \cdot \mathbf{u} > 0 $
    \item \textbf{τοπικό μέγιστο} αν $ [H(L(P)) \cdot \mathbf{u}] \cdot \mathbf{u} < 0 $
  \end{myitemize}
όπου $ H(L(P)) $ είναι ο \textbf{Εσσιανός} πίνακας της συνάρτησης Lagrange στο 
σημείο $P$ και $ \mathbf{u} $ ένα οποιοδήποτε διάνυσμα \textbf{κάθετο} στο διάνυσμα της 
κλίσης της συνάρτησης $ \phi(x,y) $.
\end{thm}

\begin{rem}
\item {}
  \begin{myitemize}
    \item Η σταθερά $ \lambda $ στον ορισμό της συνάρτησης Lagrange ονομάζεται 
      \textcolor{Col1}{πολλαπλασιαστής Lagrange}.
    \item  Από τη σχέση~\eqref{eq:lag} έχουμε ότι
      \[
        \grad L = 0 \Leftrightarrow \grad f + \lambda \grad \phi = 0 \Leftrightarrow
        \grad f = - \lambda \grad \phi
      \] 
      που σημαίνει ότι τα σημεία ακροτάτων της $f$ υπό τον περιορισμό $\phi(x,y)=0$, 
      είναι ακριβώς εκείνα τα σημεία για τα οποία το διάνυσμα της κλίσης της $f$ είναι 
      \textbf{παράλληλο} με το διάνυσμα της κλίσης της $ \phi $.
  \end{myitemize}
\end{rem}


\subsection{Ακρότατα υπό δύο συνθήκες}

\begin{thm}
  Έστω $ f(x,y,z) $ συνάρτηση τριών μεταβλητών, ορισμένη σε ένα ανοιχτό 
  υποσύνολο $A$ του $ \mathbb{R}^{3} $ και έστω ότι οι παράγωγοι 2ης τάξης είναι 
  συνεχείς στο $A$. Υποθέτουμε ότι για τις συναρτήσεις $ g_{1}, g_{2} $ 
  οι μερικές παράγωγοι 1ης τάξης είναι συνεχείς στο $A$ και ικανοποιούνται οι συνθήκες 
  (περιορισμοί)
  \begin{equation} \label{eq:constr2}
    \begin{aligned}
      g_{1} (x,y,z) = 0 \\
      g_{2} (x,y,z) = 0
    \end{aligned}
  \end{equation}
  Θεωρούμε τη συνάρτηση Lagrange που ορίζεται από τον τύπο
  \[
    L(x,y,z, \lambda) = f(x,y,z) + \lambda g_{1} (x,y,z) + \mu g_{2}(x,y,z)
  \] 

  Τα ακρότατα $ P(x,y,z) $ της συνάρτησης $ f(x,y,z) $ υπό τους περιορισμούς 
  $ g_{1}(x,y,z)=0 $ και $ g_{2}(x,y,z)=0 $ θα αναζητηθούν από την λύση του συστήματος 
  \begin{equation}\label{eq:lag2}
    \left.
      \begin{matrix}
        \grad L(x,y,z)=0 \\
        g_{1} (x,y,z)=0 \\
        g_{2} (x,y,z)=0
      \end{matrix} 
    \right\} \Leftrightarrow 
    \left.
      \begin{matrix}
        L_{x}=0 \\
        L_{y}=0 \\
        L_{z}=0 \\
        g_{1}(x,y,z)=0 \\
        g_{2}(x,y,z)=0
      \end{matrix} 
    \right\}
  \end{equation}
\end{thm}

\begin{rem}
\item {}
  Από τη σχέση~\eqref{eq:lag2} έχουμε ότι 
  \[
    \grad L = 0 \Leftrightarrow \grad f + \lambda \grad g_{1} + \mu \grad g_{2} = 0 
    \Leftrightarrow \grad f = - \lambda \grad g_{1} - \mu \grad g_{2}
  \]
  που σημαίνει ότι τα σημεία ακροτάτων της $f$ υπό τους 
  περιορισμούς $g_{1}(x,y,z)=0$ και $ g_{2}(x,y,z)=0 $, είναι ακριβώς εκείνα τα σημεία 
  για τα οποία το διάνυσμα της κλίσης της $f$ ανήκει στο \textbf{επίπεδο} που ορίζουν τα
  διανύσματα $ \grad g_{1} $ και $ \grad g_{2} $, αφού $ \grad f $ είναι γραμμικός
  συνδυασμός των $ \grad g_{1} $ και $ \grad g_{2} $.
\end{rem}
\end{document}
