\documentclass[a4paper,12pt]{article}
\usepackage{etex}
%%%%%%%%%%%%%%%%%%%%%%%%%%%%%%%%%%%%%%
% Babel language package
\usepackage[english,greek]{babel}
% Inputenc font encoding
\usepackage[utf8]{inputenc}
%%%%%%%%%%%%%%%%%%%%%%%%%%%%%%%%%%%%%%

%%%%% math packages %%%%%%%%%%%%%%%%%%
\usepackage{amsmath}
\usepackage{amssymb}
\usepackage{amsfonts}
\usepackage{amsthm}
\usepackage{proof}

\usepackage{physics}

%%%%%%% symbols packages %%%%%%%%%%%%%%
\usepackage{bm} %for use \bm instead \boldsymbol in math mode 
\usepackage{dsfont}
\usepackage{stmaryrd}
%%%%%%%%%%%%%%%%%%%%%%%%%%%%%%%%%%%%%%%


%%%%%% graphicx %%%%%%%%%%%%%%%%%%%%%%%
\usepackage{graphicx}
\usepackage{color}
%\usepackage{xypic}
\usepackage[all]{xy}
\usepackage{calc}
\usepackage{booktabs}
\usepackage{minibox}
%%%%%%%%%%%%%%%%%%%%%%%%%%%%%%%%%%%%%%%

\usepackage{enumerate}

\usepackage{fancyhdr}
%%%%% header and footer rule %%%%%%%%%
\setlength{\headheight}{14pt}
\renewcommand{\headrulewidth}{0pt}
\renewcommand{\footrulewidth}{0pt}
\fancypagestyle{plain}{\fancyhf{}
\fancyhead{}
\lfoot{}
\rfoot{\small \thepage}}
\fancypagestyle{vangelis}{\fancyhf{}
\rhead{\small \leftmark}
\lhead{\small }
\lfoot{}
\rfoot{\small \thepage}}
%%%%%%%%%%%%%%%%%%%%%%%%%%%%%%%%%%%%%%%

\usepackage{hyperref}
\usepackage{url}
%%%%%%% hyperref settings %%%%%%%%%%%%
\hypersetup{pdfpagemode=UseOutlines,hidelinks,
bookmarksopen=true,
pdfdisplaydoctitle=true,
pdfstartview=Fit,
unicode=true,
pdfpagelayout=OneColumn,
}
%%%%%%%%%%%%%%%%%%%%%%%%%%%%%%%%%%%%%%

\usepackage[space]{grffile}

\usepackage{geometry}
\geometry{left=25.63mm,right=25.63mm,top=36.25mm,bottom=36.25mm,footskip=24.16mm,headsep=24.16mm}

%\usepackage[explicit]{titlesec}
%%%%%% titlesec settings %%%%%%%%%%%%%
%\titleformat{\chapter}[block]{\LARGE\sc\bfseries}{\thechapter.}{1ex}{#1}
%\titlespacing*{\chapter}{0cm}{0cm}{36pt}[0ex]
%\titleformat{\section}[block]{\Large\bfseries}{\thesection.}{1ex}{#1}
%\titlespacing*{\section}{0cm}{34.56pt}{17.28pt}[0ex]
%\titleformat{\subsection}[block]{\large\bfseries{\thesubsection.}{1ex}{#1}
%\titlespacing*{\subsection}{0pt}{28.80pt}{14.40pt}[0ex]
%%%%%%%%%%%%%%%%%%%%%%%%%%%%%%%%%%%%%%

%%%%%%%%% My Theorems %%%%%%%%%%%%%%%%%%
\newtheorem{thm}{Θεώρημα}[section]
\newtheorem{cor}[thm]{Πόρισμα}
\newtheorem{lem}[thm]{λήμμα}
\theoremstyle{definition}
\newtheorem{dfn}{Ορισμός}[section]
\newtheorem{dfns}[dfn]{Ορισμοί}
\theoremstyle{remark}
\newtheorem{remark}{Παρατήρηση}[section]
\newtheorem{remarks}[remark]{Παρατηρήσεις}
%%%%%%%%%%%%%%%%%%%%%%%%%%%%%%%%%%%%%%%




\newcommand{\vect}[2]{(#1_1,\ldots, #1_#2)}
%%%%%%% nesting newcommands $$$$$$$$$$$$$$$$$$$
\newcommand{\function}[1]{\newcommand{\nvec}[2]{#1(##1_1,\ldots, ##1_##2)}}

\newcommand{\linode}[2]{#1_n(x)#2^{(n)}+#1_{n-1}(x)#2^{(n-1)}+\cdots +#1_0(x)#2=g(x)}

\newcommand{\vecoffun}[3]{#1_0(#2),\ldots ,#1_#3(#2)}

\newcommand{\mysum}[1]{\sum_{n=#1}^{\infty}


\pagestyle{empty}

\begin{document}

\begin{center}
\fbox{\large\bfseries Ασκήσεις στα Σύνολα}
\end{center}

\vspace{\baselineskip}

\begin{enumerate}

\item {\bfseries(Θέμα Σεπ '15)} Δίνονται τα παρακάτω σύνολα. Να υπολογίσετε τα καρτεσιανά γινόμενα $Α_i\times B_i$, για $i=1,2,3$ και σχεδιάστε τα γινόμενα \textbf{σε ένα} γράφημα στο $\mathbb{R}^2$.

\begin{tabular}{l}
$A_1=\{x:x\in\{1\}\}$\\
$A_2=\{x:x\in[1,3]\}$ \\
$A_3=\{x:x\in \{3\}\}$ \\
$B_1=\{y:y\in[1,2]\}$\\
$B_2=\{y:y=2,5-0,5x\}$ \\
$B_3=\{y:y\in[1,3]$\}
\end{tabular}


\item Έστω τα σύνολα $A=\{1,2,3\}, B=\{1,2\}, C=\{1,3\}, D=\{2,3\}, E=\{1\}, F=\{2\}, G=\{3\}$. Να βρεθούν τα παρακάτω σύνολα.

\begin{enumerate}[i)]

\item $A\cup B$
\item $A\cap B$
\item $A\cap (B\cap C)$
\item $(C\cup A)\cap B$
\item $A\setminus B$
\item $C\setminus A$
\item $D\triangle F = (D\setminus F)\cup(F\setminus D)$
\end{enumerate}

\item Να βρεθεί η ένωση και η τομή των συνόλων $A$ και $B$ στις παρακάτω περιπτώσεις.

\begin{enumerate}[i)]

\item Αν είναι $A=\mathbb{R}\setminus\{1,2\}$ και $B=\mathbb{R}\setminus\{1,3\}$
\item Αν είναι $A=\mathbb{R}\setminus\{1,2\}$ και $B=[1,+\infty)/$
\item Αν είναι $A=(3,+\infty)$ και $B=(-\infty,5]$
\end{enumerate}

\item Δίνεται το γενικό σύνολο $S=\{x\in \mathbb{N} : 0<x<10\}$ και τα υποσύνολά του $A=\{x\in \mathbb{N} : 1\leq x\leq 5\}$ και $B=\{x\in \mathbb{N} : 3\leq x\leq 8\}$. Να παραστήσετε αναλυτικά τα παρακάτω σύνολα:

\begin{enumerate}[i)]

\item $A\cup B$
\item $A\cap B$
\item $A^c$
\item $B^c$
\item $(A\cup B)^c$
\item $(A\cap B)^c$
\item $A^c\cap B^c$
\item $A^c\cup B^c$
\end{enumerate}



\end{enumerate}



\end{document}