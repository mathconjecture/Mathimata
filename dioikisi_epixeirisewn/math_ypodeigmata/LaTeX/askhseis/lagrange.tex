\input{preamble.tex}
\input{definitions_ask.tex}

\geometry{top=2cm}
\pagestyle{askhseis}

\renewcommand{\vec}{\mathbf}

\begin{document}

\begin{center}
  \minibox{\large \bfseries \textcolor{Col1}{Ασκήσεις στα Ακρότατα Με Περιορισμό}}
\end{center}

\section*{Ασκήσεις}

\begin{enumerate}
  \item Να μελετηθούν τα ακρότατα της συνάρτησης $ f(x,y) = xy $ με περιορισμό 
    $ x+y=1 $. \hfill Απ: $ f_{\max}(1/2,1/2) $ 
    %pauln notes
  \item Να μελετηθούν τα ακρότατα της συνάρτησης $ f(x,y) = 5x-3y $ με περιορισμό 
    $x^{2}+y^{2}=136$. 

    \hfill Απ: $ f_{\max}(10,-6), \; f_{\min}(-10,6) $ 

  \item Να μελετηθούν τα ακρότατα της συνάρτησης $ f(x,y,z) = xyz $ με περιορισμό 
    $x^{2}+y^{2}+z^{2}=1 $, όπου $ x,y,z>0 $. 

    \hfill Απ: $ f_{\max}(\sqrt{3} /3,\sqrt{3} /3,\sqrt{3} /3) $ 

    %pauln notes
  \item Να μελετηθούν τα ακρότατα της συνάρτησης $ f(x,y,z) = xyz $ με περιορισμό 
    $x+y+z=1 $, όπου $ x,y,z \geq 0 $. 

    \hfill Απ: $ f_{\max}(1/3,1/3,1/3), f_{\min}(0,0,1), f_{min}(0,1,0), 
    f_{min}(1,0,0) $ 

  \item Να μελετηθούν τα ακρότατα της συνάρτησης $ f(x,y,z) = 4y-2z $ με περιορισμούς 
    $2x-y-z=2$, και $x^{2}+y^{2}=1$. 

    \hfill Απ: $ f_{\max}(-2/ \sqrt{13} , 3 / \sqrt{13}), 
    f_{\min}(2 / \sqrt{13}, - 3 / \sqrt{13})$ 

  \item Να μελετηθούν τα ακρότατα της συνάρτησης $ f(x,y,z) = x+y+z $, υπό 
    τους περιορισμούς $ x^{2}+y^{2}=2 $ και $ x+z=1 $.

\hfill Απ: $ f_{max}(0, \sqrt{2} , 1), \; f_{\min}(0, - \sqrt{2} , 1) $ 
\end{enumerate}


\section*{Προβλήματα}

\begin{enumerate}
  \item Έστω ότι μια επιχείρηση έχει συνάρτηση ολικών κερδών
    \[
      \pi (x,y) = -4x^{2}-5y^{2}+20xy
    \] 
    όπου $ x $ και $ y $ είναι οι παραγόμενες και πωλούμενες ποσότητες των προϊόντων 
    $ X $ και $ Y $. Για την παραγωγή κάθε μονάδας προϊόντος $X$ απαιτούνται 2 μονάδες 
    ενός συντελεστή παραγωγής και για την παραγωγή κάθε μονάδας προϊόντος $Y$ 
    απαιτούνται 5 μονάδες του συντελεστή αυτού που διατίθεται σε 40 μονάδες. Να 
    υπολογίσετε το μέγιστο κέρδος, υπό την παραπάνω συνθήκη.
    \hfill Απ: $ \pi_{\max}(15/2,5)=4000 $ 

  \item Το συνολικό κόστος παραγωγής $x$ μονάδων ενός προϊόντος $A$ και $y$ μονάδων ενός 
    προϊόντος $B$ δίνεται από τη σχέση $ TC = 22x^{2} + 8 y^{2} - 5xy $. Αν η επιχείρηση 
    δεσμεύεται να παράγει 20 μονάδες συνολικά, γράψτε τον περιορισμό που συνδέει το 
    $x$ και το $y$ και στη συνέχεια υπολογίστε τον αριθμό κάθε είδους προϊόντος που 
    πρέπει να παραχθεί ώστε να ελαχιστοποιηθεί το κόστος.
    \hfill Απ: $ x+y=20, \; x=6, \; y=14 $

  \item Υπολογίστε τη μέγιστη τιμή της συνάρτησης χρησιμότητας 
    $ u(x_{1}, x_{2}) = x_{1} x_{2} $ η οποία  υπόκειται στον εισοδηματικό περιορισμό 
    $ x_{1}+4 x_{2}=360 $.
    \hfill Απ: $ 8100 $

  \item Υπολογίστε τις τιμές των $ x_{1} $ και $ x_{2} $, που μεγιστοποιούν τη 
    συνάρτηση χρησιμότητας $ u(x_{1}, x_{2}) = \ln{x_{1}} + 2 \ln{x_{2}} $, η οποία  
    υπόκειται στον εισοδηματικό περιορισμό $ 2x_{1}+3 x_{2}=18 $.
    \hfill Απ: $x1 = 3, \; x_{2}=4$

  \item Μια επιχείρηση παράγει δύο αγαθά, το $A$ και το $B$. Το εβδομαδιαίο κόστος
    παραγωγής $x$ μονάδων $A$ και $y$ μονάδων $B$ είναι $ TC=0.2 x^{2}+0.05y^{2} +0.1xy
    +2x+5y+1000 $.
    \begin{enumerate}[i)]
      \item Υπολογίστε την ελάχιστη τιμή του $ TC $ στην περίπτωση που δεν υπάρχουν 
        περιορισμοί.
      \item Υπολογίστε στην ελάχιστη τιμή του $ TC $ όταν η επιχείρηση δεσμεύεται να 
        παράγει συνολικά 500 αγαθά και από τα 2 είδη.
        \hfill Απ: $1000, \; 15985$
    \end{enumerate}

  \item Η συνάρτηση παραγωγής μιας επιχείρησης δίνεται από τη σχέση 
    $ Q=2L^{1/2}+3K^{1/2} $, όπου τα $ Q, L $ και $K$ συμβολίζουν τον αριθμό των 
    μονάδων των παραγόμενων προϊόντων, της εργασίας και του κεφαλαίου αντίστοιχα. 
    Το κόστος εργασίας είναι 2 ευρώ ανά μονάδα και του κεφαλαίου 1 ευρώ ανά μονάδα, ενώ 
    τα παραγόμενα προϊόντα πωλούνται 8 ευρώ ανά μονάδα. Αν η επιχείρηση είναι πρόθυμη να 
    ξόδεψε 99 ευρώ στο κόστος εισροών, να υπολογίσετε το μέγιστο κέρδος και τις τιμές 
    των $K$ και $L$, για τις οποίες αυτό επιτυγχάνεται.
    \hfill Απ: $ K=81, \; L=9, \; \Pi = 165 $ 
\end{enumerate} 





\end{document}
