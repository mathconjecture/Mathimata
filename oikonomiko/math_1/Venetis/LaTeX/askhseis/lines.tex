\documentclass[a4paper,table]{report}
\input{preamble_ask.tex}
\input{definitions_ask.tex}

\everymath{\displaystyle}
\pagestyle{vangelis}

\begin{document}

\begin{center}
  \minibox{\large\bfseries \textcolor{Col1}{Ευθείες}}
\end{center}

\vspace{\baselineskip}

\begin{enumerate}
  \item  Δίνεται η ευθεία ε:  $ 3x - 4y = 12 $.
    \begin{enumerate}[i)]
      \item Να εξετάσετε αν η ευθεία διέρχεται από τα σημεία $\left(2,- \frac{3}{2}
        \right)$, $ (2,-2) $ \hfill Απ:  ναι, όχι 
      \item  Να υπολογίσετε τα σημεία τομής της ευθείας $ \varepsilon $  
        με τους άξονες  $ xx' $  και  $ yy' $ \hfill Απ: $ xx': 4, yy':-3 $ 
      \item Να γραφεί η $ \varepsilon $ στη μορφή $ y=ax+b $ και να 
        προσδιορίσετε την κλίση της. 
        \hfill Απ: $ \lambda_{\varepsilon} = \frac{3}{4} $ 
    \end{enumerate}

  \item  Να βρείτε την εξίσωση της ευθείας  $ \varepsilon $, η οποία: 
    \begin{enumerate}[i)]
      \item διέρχεται από το σημείο  $ A(2,-1) $ και έχει συντελεστή διεύθυνσης 
        $ -3 $.  \hfill Απ: $ y = -3x+5 $ 
      \item διέρχεται από το σημείο  $ A(1,2) $ και έχει συντελεστή διεύθυνσης 
        $ \frac{1}{3} $.  \hfill Απ: $ y = \frac{1}{3} x + \frac{5}{3} $ 
      \item διέρχεται από τα σημεία  $ A(1,1) $  και  $ B(3,5) $. 
        \hfill Απ: $ y=2x-1 $ 
      \item διέρχεται από τα σημεία  $ A(-1,2) $  και  $ B(-3,2) $. 
        \hfill Απ: $ y=2 $ 
    \end{enumerate}

  \item  Να βρείτε τα  σημεία τομής των ευθειών:
    \begin{enumerate}[i)]
      \item $ \varepsilon_{1}: x + 2y = 5 $   και  $\varepsilon_{2}: 4x + y = 6 $.
        \hfill Απ: $ (1,2) $ 
      \item $ \varepsilon_{1}: 4x+3y=11 $ και  $ \varepsilon_{2}: 5x+7y=17 $ 
        \hfill Απ: $ (2,1) $ 
    \end{enumerate}
    
    \textcolor{Col1}{Υπόδειξη:} Τα σημεία τομής ευθείων, είναι οι λύσεις του συστήματος 
    των εξισώσεών τους.

  \item Να βρείτε τα σημεία ισορροπίας των παρακάτω υποδειγμάτων αγοράς για ένα προϊόν.
    \begin{enumerate}[i),itemsep=\baselineskip]
      \item $ 
        \left.
          \begin{matrix}
            Q_{d} = 30-2P \\
            Q_{s} = -6+5P
          \end{matrix} 
        \right\}$ 
        \hfill Απ: $ P^{*} = \frac{36}{7} \approx 5.14, \; Q^{*}= \frac{138}{7} 
        \approx 19.71 $

      \item $ 
        \left.
          \begin{matrix}
            Q_{d} = 51 - 3P \\
            Q_{s} = 6P - 10
          \end{matrix} 
        \right\}$ 
        \hfill Απ: $ P^{*} = \frac{61}{9} \approx 6.78, \; Q^{*}= \frac{92}{3} \approx
        30.67 $

    \end{enumerate}
\end{enumerate}


\end{document}
