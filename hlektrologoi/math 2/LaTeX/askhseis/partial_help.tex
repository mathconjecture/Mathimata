\documentclass[a4paper,12pt]{article}
\usepackage{etex}
%%%%%%%%%%%%%%%%%%%%%%%%%%%%%%%%%%%%%%
% Babel language package
\usepackage[english,greek]{babel}
% Inputenc font encoding
\usepackage[utf8]{inputenc}
%%%%%%%%%%%%%%%%%%%%%%%%%%%%%%%%%%%%%%

%%%%% math packages %%%%%%%%%%%%%%%%%%
\usepackage{amsmath}
\usepackage{amssymb}
\usepackage{amsfonts}
\usepackage{amsthm}
\usepackage{proof}

\usepackage{physics}

%%%%%%% symbols packages %%%%%%%%%%%%%%
\usepackage{bm} %for use \bm instead \boldsymbol in math mode 
\usepackage{dsfont}
\usepackage{stmaryrd}
%%%%%%%%%%%%%%%%%%%%%%%%%%%%%%%%%%%%%%%


%%%%%% graphicx %%%%%%%%%%%%%%%%%%%%%%%
\usepackage{graphicx}
\usepackage{color}
%\usepackage{xypic}
\usepackage[all]{xy}
\usepackage{calc}
\usepackage{booktabs}
\usepackage{minibox}
%%%%%%%%%%%%%%%%%%%%%%%%%%%%%%%%%%%%%%%

\usepackage{enumerate}

\usepackage{fancyhdr}
%%%%% header and footer rule %%%%%%%%%
\setlength{\headheight}{14pt}
\renewcommand{\headrulewidth}{0pt}
\renewcommand{\footrulewidth}{0pt}
\fancypagestyle{plain}{\fancyhf{}
\fancyhead{}
\lfoot{}
\rfoot{\small \thepage}}
\fancypagestyle{vangelis}{\fancyhf{}
\rhead{\small \leftmark}
\lhead{\small }
\lfoot{}
\rfoot{\small \thepage}}
%%%%%%%%%%%%%%%%%%%%%%%%%%%%%%%%%%%%%%%

\usepackage{hyperref}
\usepackage{url}
%%%%%%% hyperref settings %%%%%%%%%%%%
\hypersetup{pdfpagemode=UseOutlines,hidelinks,
bookmarksopen=true,
pdfdisplaydoctitle=true,
pdfstartview=Fit,
unicode=true,
pdfpagelayout=OneColumn,
}
%%%%%%%%%%%%%%%%%%%%%%%%%%%%%%%%%%%%%%

\usepackage[space]{grffile}

\usepackage{geometry}
\geometry{left=25.63mm,right=25.63mm,top=36.25mm,bottom=36.25mm,footskip=24.16mm,headsep=24.16mm}

%\usepackage[explicit]{titlesec}
%%%%%% titlesec settings %%%%%%%%%%%%%
%\titleformat{\chapter}[block]{\LARGE\sc\bfseries}{\thechapter.}{1ex}{#1}
%\titlespacing*{\chapter}{0cm}{0cm}{36pt}[0ex]
%\titleformat{\section}[block]{\Large\bfseries}{\thesection.}{1ex}{#1}
%\titlespacing*{\section}{0cm}{34.56pt}{17.28pt}[0ex]
%\titleformat{\subsection}[block]{\large\bfseries{\thesubsection.}{1ex}{#1}
%\titlespacing*{\subsection}{0pt}{28.80pt}{14.40pt}[0ex]
%%%%%%%%%%%%%%%%%%%%%%%%%%%%%%%%%%%%%%

%%%%%%%%% My Theorems %%%%%%%%%%%%%%%%%%
\newtheorem{thm}{Θεώρημα}[section]
\newtheorem{cor}[thm]{Πόρισμα}
\newtheorem{lem}[thm]{λήμμα}
\theoremstyle{definition}
\newtheorem{dfn}{Ορισμός}[section]
\newtheorem{dfns}[dfn]{Ορισμοί}
\theoremstyle{remark}
\newtheorem{remark}{Παρατήρηση}[section]
\newtheorem{remarks}[remark]{Παρατηρήσεις}
%%%%%%%%%%%%%%%%%%%%%%%%%%%%%%%%%%%%%%%




\input{definitions_ask.tex}


%\renewcommand{\baselinestretch}{1.2}

\pagestyle{vangelis}
\everymath{\displaystyle}

\begin{document}

\begin{center}
\fbox{\large\bfseries Ασκήσεις στις Μερικές Παραγώγους}
\end{center}

\vspace{\baselineskip}

\begin{enumerate}




    \section{Απλες}
 
\item Να βρεθεί η παράγωγος 1ης τάξης $\dv{f}{t}$ της σύνθετης συνάρτησης $f(x,y)=\ln(y^2-x^2)$, όταν $x=\sin t, y=\cos t$, για $t=\frac{\pi}{8}$.

\hfill Απ: $\scriptstyle{-2}$

\item Να βρεθεί η παράγωγος 1ης τάξης $\dv{w}{t}$ της σύνθετης συνάρτησης $ w = xy+z $, όπου $ x =
	\cos{t}, y = \sin{t}$ και $ z = t $, για $ t = \frac{ \pi }{ 4 } $.

	\hfill Απ: 1

\item Να υπολογιστούν οι μερικές παράγωγοι 1ης τάξης, ως προς $u$ και $v$, της συνάρτησης $ f(x,y,z)
	= x + 2y + z^{2}$, όπου $ x = \frac{ u }{ v } $, $y = u^{2} + \ln{v} $ και $ z = 2u $.

	\hfill Απ: $ \pdv{f}{u} = \frac{1}{ v } + 12u $, $\pdv{f}{v} = -\frac{ u }{ v^{2} } + \frac{
	2 }{ v } $

 \item Έστω η συνάρτηση $ f : \mathbb{R}^{2} \to \mathbb{R} $ που ορίζεται με τον τύπο $ f(x,y) =
	 x^{2} + xy $, όπου $ x=r \cos{\theta} $ και $ y= r \sin{\theta} $. Να υπολογίσετε με τη χρήση
	 του κανόνα αλυσίδας τις μερικές παραγώγους $ \pdv{f}{r} $ και $ \pdv{f}{\theta} $.

	 \hfill Απ: \begin{tabular}{l}
		 $\scriptstyle{i) \pdv{f}{r} = 2r(\cos{\theta} )(\cos{\theta} + \sin{\theta})}
			 $ \\
			 $\scriptstyle{ii) \pdv{f}{\theta}=r^{2}(\cos{2\theta} - \sin{2 \theta})} $
	 \end{tabular}


 \item Έστω η συνάρτηση $ f : \mathbb{R}^{2} \to \mathbb{R} $ που ορίζεται με τον τύπο $ f(x,y,z) =
	 xyz^{2}$, όπου $ x = \sin{t}, y = \cos{t} $ και $ z = t^{2}+1 $. Να υπολογίσετε την μερική
	 παράγωγο 2ης τάξης $ \pdv[2]{f}{t} $.

	 \hfill Απ: $\scriptstyle{\pdv[2]{f}{t} = -2(t^{2}+1)^{2} \sin{2t} + 8t(t^{2}+1) \cos{2t} +
	 2(t^{2}+1) \sin{2t} + 4t^{2} \sin{2t}}$

 \item Έστω η συνάρτηση $ f : \mathbb{R}^{2} \to \mathbb{R} $ που ορίζεται με τον τύπο $ f(x,y) =
	 x+y$, όπου $ x = u^{2} - v^{2} $ και $ y = e^{uv} $. Να υπολογίσετε την μερική παράγωγο 2ης
	 τάξης $ \pdv[2]{f}{u} $.

	 \hfill Απ: $ \scriptstyle{\pdv[2]{f}{u} = 2 + v^{2}e^{uv}} $



     \section{καλες}

 % \item Αν $ z=f(x,y) $ είναι μια συνάρτηση με συνεχείς δεύτερες μερικές παραγώγους με $
	 % x=u^{2}+v^{2} $ και $ y = 2uv $, να υπολογίσετε τις παραγώγους
	 % \begin{inparaenum}[i)]
		 % \item $\pdv{z}{u}$
		 % \item $\pdv[2]{z}{u}$
	 % \end{inparaenum}

	 % \hfill Απ:  \begin{tabular}{l}
		 % $\scriptstyle{\textlatin\rm{i})  \pdv{z}{u} = 2u\pdv{z}{x}+2v\pdv{z}{y}}  $ \\
		 % $\scriptstyle{\textlatin\rm{ii}) \pdv[2]{z}{u} =
		 % 2\pdv{z}{x}+4u^{2}\pdv[2]{z}{x}+8uv\pdv[2]{z}{x}{y}+4v^{2}\pdv[2]{z}{y}} $
	 % \end{tabular}




	


 

 % \item Να μετασχηματιστεί η εξίσωση \textlatin{Laplace} σε 
	 % \begin{inparaenum}[(i)]
	 	% \item πολικές
		% \item κυλινδρικές 
		% \item σφαιρικές
	 % \end{inparaenum}
			% συντεταγμένες

			% \hfill Απ: \begin{tabular}{l} $\scriptstyle{\textlatin\rm{i}) 
		 % \pdv[2]f{x} + \pdv[2]{f}{y} = \pdv[2]{f}{r} + \frac{ 1 }{ r } \pdv{f}{r} + \frac{ 1 }{
		 % r^{2} } \pdv[2]{f}{\theta}}$  \\ $\scriptstyle{\textlatin\rm{ii})
		 % \pdv[2]{f}{x}+\pdv[2]{f}{y}+\pdv[2]{f}{z} = \pdv[2]{f}{r} + \frac{1}{r} \pdv{f}{r} +
		 % \frac{1}{r^{2}} \pdv[2]{f}{r} + \pdv[2]{f}{z}}$ \\ $\scriptstyle{\textlatin\rm{iii}) \pdv[2]{f}{x}+\pdv[2]{f}{y}+\pdv[2]{f}{z} = \frac{ 1 }{ r^{2} } \pdv{}{r}(r^{2}\pdv{f}{r}) + \frac{ 1 }{ r^{2} \sin^{2}{\phi} }
	 % \pdv[2]{f}{\theta} + \frac{ 1 }{ r^{2} \sin{\phi } } \pdv{}{\phi}(\sin{\phi} \pdv{f}{\phi})}$
% \end{tabular}



 % \item Αν $ z = f(x,y) $, με $ x = x(u,v) $ και $ y = y(u,v) $, τότε να υπολογιστούν οι $ f_{uu},
	 % f_{vv} $ και $ f_{uv} $.


     \section{Ομογενεις}

 \item Έστω η συνεχής συνάρτηση $ f : \mathbb{R}^{2} \to \mathbb{R} $ των ανεξάρτητων μεταβλητών $
	 x,y $ που είναι ομογενής βαθμού $ \mu $. Αν η $f$ έχει συνεχείς μερικές παραγώγους 2ης τάξης
	 στο $ \mathbb{R}^{2} $ να αποδείξετε ότι ισχύει 
	 \[
		 x^{2}\pdv[2]{f}{x} + 2xy\pdv[2]{f}{x}{y} + y^{2}\pdv[2]{f}{y} = \mu(\mu -1)f(x,y)
	 \] 

 \item Αν $ u = u(x,y) $ και $ v = v(x,y) $ είναι ομογενείς συναρτήσεις βαθμού ομογένειας $\mu$, να
	 αποδείξετε ότι για κάθε συνάρτηση $f$ με συχεχείς μερικές παραγώγους 1ης τάξης ως προς τις
	 μεταβλητές $u$, $v$, ισχύει
	 \[
		 x\pdv{f}{x} + y\pdv{f}{y} = \mu \left(u\pdv{f}{u} + v\pdv{f}{v}\right) 
	 \] 

 \item  Έστω $ u = u(x,y) $ και $ v = v(x,y) $ δύο ομογενείς συναρτήσεις βαθμού ομογένειας $\mu$, με
	 $u(x,y)\neq 0$ και $ v(x,y)\neq 0 $ για κάθε $(x,y) \in \mathbb{R}^{2} $. Να αποδείξετε ότι
	 \[
		 udv - vdu = \frac{ 1 }{ \mu } \cdot \pdv{(u,v)}{(x,y)}\cdot(xdy-ydx) 
	 \] 













\end{enumerate}


\end{document}
