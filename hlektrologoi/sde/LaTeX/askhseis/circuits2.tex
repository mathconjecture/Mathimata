\input{preamble_ask.tex}
\input{definitions_ask.tex}

\usepackage[RPvoltages]{circuitikz}


\pagestyle{vangelis}
% \everymath{\displaystyle}



\begin{document}

\begin{center}
  \minibox{\large\bfseries \textcolor{Col1}{Ηλεκτρικά Κυκλώματα}}
\end{center}

\vspace{\baselineskip}

%xatzikonstantinou p.214

% \begin{problem}
%   Να υπολογιστεί το ρεύμα $ i(t) $ σε ένα κύκλωμα $ RLC $, όπως στο παρακάτω σχήμα, 
%   με ωμική αντίσταση $ R= \SI{100}{\ohm} $, συντελεστή αυτεπαγωγής $ L= \SI{0,1}{\henry}
%   $ και πυκνωτή χωρητικότητας $ C= \SI{d-3}{\farad} $ τα οποία συνδέονται σε σειρά με 
%   μια πηγή τάσης $ E(t) = 155 \sin{377t} $ 
%   (ουσιαστικά συχνότητας $ \nu = \omega / 2 \pi = 377/2 \pi = \SI{60}{\hertz} $). 
%   Η αρχική συνθήκη είναι ότι τα φορτία και τα ρεύματα για $ t=0 $ είναι μηδέν, δηλαδή  
%   $ i(0)=0, \; i'(0)=0 $.
% \end{problem}
% \begin{center}
%   \begin{circuitikz}[american,cute inductors]
%     \draw (0,0) to[short,i=$i$,battery1,l=$E$] (0,3) to[L=$L$]  (4,3) to[C=$C$]
%     (4,0) to[R=$R$]  (0,0) ;
%   \end{circuitikz}
% \end{center}
% \hfill Απ: $ i(t) = -0,043 \mathrm{e}^{-10t} + 0,526 \mathrm{e}^{-990t} - 0,483
% \cos{(377t)} + 1,381 \sin{(377t)} $ 

\begin{problem}
  Έστω ένα κύκλωμα $ RLC $, όπως στο παρακάτω σχήμα, 
  με ωμική αντίσταση $ R= \SI{100}{\ohm} $, συντελεστή αυτεπαγωγής $ L= \SI{3}{\henry}
  $ και πυκνωτή χωρητικότητας $ C= \SI{1.5d-6}{\farad} $ τα οποία συνδέονται με 
  μια πηγή τάσης $ E(t) = \SI{100}{\volt} $. Να υπολογιστεί το ρεύμα που διαρρέει το 
  πηνίο καθώς και το ρεύμα που διαρέει την αντίσταση, αν τα φορτιά και τα ρεύματα 
  είναι μηδέν για $ t=0 $.
\end{problem} 
\begin{center}
  \begin{circuitikz}[american,cute inductors]
    \draw (0,0) to[short,i=$i$,battery1,l=$E$] (0,3)
    to[L=$L$] (3,3)
    to[R=$R$, i>_=$i_1$] (3,0) -- (0,0);
    \draw (3,3) -- (6,3)
    to[C=$C$, i>_=$i_2$]
    (6,0) -- (3,0);
  \end{circuitikz}
\end{center}

\hfill Απ: 
\renewcommand{\arraystretch}{1.5}
\begin{tabular}{l}
  $ i(t)= 1- \mathrm{e}^{- \frac{100}{3} t} \cos{\left(\frac{100}{3}t\right)} $ \\
  $ i_{1}(t)= 1- \mathrm{e}^{- \frac{100}{3} t} 
  \left[\cos{\left(\frac{100}{3}t\right)} + \sin{\left(\frac{100}{3}t\right)}\right] $
\end{tabular} 

\begin{problem}
  Έστω ένα κύκλωμα $ RLC $, όπως στο παρακάτω σχήμα, 
  με ωμική αντίσταση $ R= \SI{20}{\ohm} $, και συντελεστή αυτεπαγωγής 
  $ L_{1}= \SI{1}{\henry} $ και $ L_{2}= \SI{0.5}{\henry} $ τα οποία συνδέονται με 
  μια πηγή τάσης $ E(t) = \SI{50}{\volt} $. Να υπολογιστεί το ρεύμα που διαρρέει το 
  2ο πηνίο καθώς και το ρεύμα που διαρέει την αντίσταση, αν τα φορτιά και τα ρεύματα 
  είναι μηδέν για $ t=0 $.
\end{problem}
\begin{center}
  \begin{circuitikz}[american,cute inductors]
    \draw (0,0) to[battery1,l=$E$] (0,3)
    to[L=$L_1$] (3,3)
    to[R=$R$, i>_=$i_1$] (3,0) -- (0,0);
    \draw (3,3) -- (6,3)
    to[L=$L_2$, i>_=$i_2$]
    (6,0) -- (3,0);
  \end{circuitikz}
\end{center}

\hfill Απ: 
\renewcommand{\arraystretch}{1.5}
\begin{tabular}{l}
  $ i_{1}(t) = - \frac{5}{6} (1- \mathrm{e}^{-60t}) $ \\
  $ i_{2}(t) = \frac{100}{3} t - \frac{5}{9} (\mathrm{e}^{-60t} -1) $ 
\end{tabular} 


%todo να προσθέσω μερικές ασκήσεις με κυκλωματα RLC

\end{document}
