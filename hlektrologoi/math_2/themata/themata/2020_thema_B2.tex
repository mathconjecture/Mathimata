\documentclass[a4paper,table]{report}
\input{preamble_themata}
\newcommand{\vect}[2]{(#1_1,\ldots, #1_#2)}
%%%%%%% nesting newcommands $$$$$$$$$$$$$$$$$$$
\newcommand{\function}[1]{\newcommand{\nvec}[2]{#1(##1_1,\ldots, ##1_##2)}}

\newcommand{\linode}[2]{#1_n(x)#2^{(n)}+#1_{n-1}(x)#2^{(n-1)}+\cdots +#1_0(x)#2=g(x)}

\newcommand{\vecoffun}[3]{#1_0(#2),\ldots ,#1_#3(#2)}

\newcommand{\suma}{\sum_{n=0}^{\infty}a_n x^n}

\newcommand{\sumb}{\sum_{n=1}^{\infty}a_n n x^{n-1}}

\newcommand{\sumc}{\sum_{n=2}^{\infty}a_n n (n-1) x^{n-2}}

\newcommand{\varsum}[2]{\sum_{n=#1}^{#2}}
\input{myboxes}

\everymath{\displaystyle}
\pagestyle{vangelis}


\begin{document}

\begin{mybox3}
  \begin{thema}
    Δίνεται η καμπύλη $ c: \mathbf{r}(t)= \cos{t}\, \mathbf{i} + \sin{t}\, \mathbf{j} + t
    \, \mathbf{k}$. Να βρεθεί η εξίσωση του \textbf{κάθετου} επιπέδου (ορίζεται από τα 
    διανύσματα $ \mathbf{N} $ και $ \mathbf{B} $) στο σημείο της $ P_{0} $ που 
    αντιστοιχεί στην τιμή της παραμέτρου $ t= {\pi}/{2} $.
  \end{thema}
\end{mybox3}
\begin{solution}
  \item {}
    \begin{mybox1}
    \item {}
      \textcolor{Col1}{Θυμάμαι:} Η εξίσωση του επιπέδου $ \Pi $, που διέρχεται από ένα 
      σημείο $ P_{0}(x_{0}, y_{0}, z_{0}) $ και είναι κάθετο στο διάνυσμα 
      $ \mathbf{n} = (a,b,c) $ είναι:
      \[
        \Pi: a(x- x_{0}) + b(y- y_{0}) + c(z- z_{0}) = 0. 
      \] 
    \end{mybox1}
    Τα διανύσματα $ \mathbf{T}, \mathbf{N} $ και $ \mathbf{B} $ που σχηματίζουν το τρίεδρο
    Frenet είναι ανά δύο κάθετα. Επομένως ένα διάνυσμα κάθετο στο ζητούμενο επίπεδο που 
    ορίζουν τα διανύσματα $ \mathbf{N} $ και $ \mathbf{B} $ είναι το $ \mathbf{T} $. Άρα 
    υπολογίζουμε το διάνυσμα $ \mathbf{T} $ 
    \begin{myitemize}
      \item $ \mathbf{r'}(t)=- \sin{t}\, \mathbf{i} + \cos{t}\, \mathbf{j} + 1 \, \mathbf{k} $
      \item $ \norm{\mathbf{r'}(t)} = \sqrt{(- \sin{t} )^{2} + (\cos{t} )^{2} + 1^{2}} =
        \sqrt{\sin^{2}{t} + \cos^{2}{t} + 1} = \sqrt{2} $ 
      \item $ \mathbf{T}= \frac{\mathbf{r'}(t)}{\norm{\mathbf{r'}(t)}} =
        \frac{- \sin{t}}{\sqrt{2}}\,\mathbf{i}
        + \frac{\cos{t}}{\sqrt{2}}\,\mathbf{j} + \frac{1}{\sqrt{2}}\,\mathbf{k} $
    \end{myitemize}
    Υπολογίζουμε το διάνυσμα $ \mathbf{T} $ στο σημείο $ t = \pi /2 $ και έχουμε:
    \[
      \mathbf{T}\Bigl(\frac{\pi}{2}\Bigr) =  \frac{- \sin{\frac{\pi}{2}
      }}{\sqrt{2}}\,\mathbf{i} +
      \frac{\cos{\frac{\pi}{2}}}{\sqrt{2}}\,\mathbf{j} + \frac{1}{\sqrt{2}}\,\mathbf{k} = 
      - \frac{1}{\sqrt{2}}\,\mathbf{i} + 0\,\mathbf{j} + \frac{1}{\sqrt{2}}\,\mathbf{k}
    \]
    Για να βρούμε τις συντεταγμένες του σημείου $ P_{0} $ της καμπύλης, βρίσκουμε το 
    διάνυσμα θέσης στο σημείο $ t = \pi /2 $
    \[
      \mathbf{r}\Bigl(\frac{\pi}{2}\Bigr)= \cos{\frac{\pi}{2}}\, \mathbf{i} +
      \sin{\frac{\pi}{2}} \,
      \mathbf{j} + \frac{\pi}{2}\, \mathbf{k} =  0\,\mathbf{i} + 1\,\mathbf{j} +
      \pi /2\,\mathbf{k} = \Bigl(0,1, \frac{\pi}{2}\Bigr)
    \] 
    Άρα η εξίσωση του κάθετου επιπέδου, θα είναι:
    \begin{gather*}
      \Pi: - \frac{1}{\sqrt{2}} (x- 0)  + 0(y-1) + \frac{1}{\sqrt{2}} (z-
      \frac{\pi}{2}) = 0 \\ 
      \Pi: - \frac{1}{\sqrt{2}} x + \frac{1}{\sqrt{2}} z - \frac{\pi}{2
      \sqrt{2}} = \\ 
      \Pi: - x + z = \frac{\pi}{2}
    \end{gather*}  




  \end{solution}


\end{document}
