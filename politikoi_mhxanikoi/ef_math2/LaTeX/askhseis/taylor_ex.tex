\input{preamble_ask.tex}
\input{definitions_ask.tex}

\pagestyle{askhseis}

\everymath{\displaystyle}


\begin{document}

\begin{center}
  \large\bfseries \textcolor{Col1}{Τύπος Taylor (Λυμένο Παράδειγμα)}
\end{center}

\vspace{\baselineskip}

\begin{example}
  Να υπολογιστεί το ανάπτυγμα \textbf{Taylor} της συνάρτησης 
  $ f(x,y) = x^{3} + y^{3} + xy^{2} $. 
\end{example}

Για να υπολογίσουμε το ανάπτυγμα της συνάρτησης σε δυνάμεις του $(x-1)$ και $(y-2)$ 
αρκεί ισοδύναμα να υπολογίσουμε το ανάπτυγμα της συνάρτησης γύρω από το 
σημείο $(x_0,y_0)=(1,2)$

Παρατηρώ ότι δεν αναφέρεται στην εκφώνηση μέχρι τους όρους ποιας τάξης 
xρειάζεται να βρω το ανάπτυγμα.  Γι' αυτό, μια και η συνάρτηση είναι πολυωνυμική,
βρίσκω μέχρι την τάξη όπου μηδενίζονται οι μερικές παράγωγοι: 

(Δηλαδή στο συγκεκριμένο παράδειγμα μέχρι $3$ης τάξης, αφού όλες οι παράγωγοι 
$4$ης και ανώτερης τάξης, θα είναι όλες μηδέν)

\vspace{\baselineskip}

\twocolumnsides{
  \begin{myitemize}
    \item $f_x=3x^2+y^2\Rightarrow f_x(1,2)=7$
    \item $f_y=3y^2+2xy\Rightarrow f_y(1,2)=16$
    \item $f_{xx}=6x\Rightarrow f_{xx}(1,2)=6$
    \item $f_{xy}=2y\Rightarrow f_{xy}(1,2)=4$
    \end{myitemize}}{\begin{myitemize}
    \item $f_{yy}=6y+2x\Rightarrow f_{yy}(1,2)=14$
    \item $f_{xxx}=6$
    \item $f_{xxy}=f_{xyx}=0$
    \item $f_{xyy}=f_{yxy}=2$
    \item $f_{yyy}=6$ 
\end{myitemize}}

\vspace{\baselineskip}

Με αντικατάσταση των μερικών παραγώγων στον τύπο Taylor, έχουμε:
\begin{equation*}
  \begin{split}
    f(x,y)&=13+\Bigl(7(x-1)+16(y-2)\Bigr)+ \\ 
          &\quad +\frac{1}{2!}\Bigl(6(x-1)^2 +2\cdot 4(x-1)(y-2)+14(y-2)^2\Bigr)+ \\
          &\quad +\frac{1}{3!}\Bigl(6(x-1)^3+3\cdot 0(x-1)^2(y-2)+3\cdot 2(x-1)(y-2)^2+6(y-2)^3\Bigr).
  \end{split}
\end{equation*}
Και μετά τις πράξεις, έχουμε: 
\begin{equation*}
  \begin{split}
    f(x,y)&=13+7(x-1)(y-2)+16(y-2)+3(x-1)^2+4(x-1)(y-2) \\
          &\quad +7(y-2)^2+(x-1)^3+(x-1)(y-2)^2+(y-2)^3.
  \end{split}
\end{equation*}
Δεν κάνουμε άλλες πράξεις.

Έχουμε το ανάπτυγμα της $f(x,y)$ σε δυνάμεις του $(x-1)$ και $(y-2)$ όπως ζητήθηκε.

\end{document}
