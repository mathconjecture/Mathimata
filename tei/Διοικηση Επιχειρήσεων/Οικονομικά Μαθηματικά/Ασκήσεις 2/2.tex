\documentclass[a4paper,12pt]{article}


\usepackage[english,greek]{babel}
\usepackage[utf8]{inputenc}

\usepackage{amsmath}
\usepackage{amssymb}
\usepackage{amsthm}
\usepackage{amsfonts}

\usepackage[margin=2cm]{geometry}
\usepackage{outlines}

\usepackage[margin=2cm]{geometry}


\begin{document}


\begin{center}
\fbox{\Large\bfseries{Ασκήσεις στον Ανατοκισμό}}
\end{center}

\vspace{1cm}

\begin{enumerate}
\item Κεφάλαιο 200\euro\ τοκίζεται για 2 έτη με επιτόκιο 4\%. Ποια είναι η τελική αξία του κεφαλαίου?

\hspace{.15\textwidth}Λύση

\vspace{3cm}\hfill Απ: 216,32\euro

\item Κεφάλαιο 100\euro\ τοκίζεται με ετήσιο ανατοκισμό 3\% για 4 χρόνια και 5 μήνες. Ποια είναι η τελική αξία του κεφαλαίου?

\hspace{.15\textwidth}Λύση

\vspace{3cm}\hfill Απ: 113,95\euro

\item Κεφάλαιο 150\euro\ ανατοκίζεται με ετήσιο επιτόκιο 5\% για 3 χρόνια και 20 ημέρες. Ποια είναι η τελική αξία του κεφαλαίου?

\hspace{.15\textwidth}Λύση

\vspace{3cm}\hfill Απ: 174,12\euro

\item Κεφάλαιο 12000\euro\ τοκίστηκε στην τράπεζα Δ με ετήσιο ανατοκισμό και επιτόκιο 8\% για διάστημα 3 ετών. Ποιος είναι ο τόκος του κεφαλαίου και ποιο ποσό θα πάρει ο δικαιούχος (καταθέτης) αν η φορολογία είναι 15\%? 

\hspace{.15\textwidth}Λύση

\vspace{3cm}\hfill Απ: 2647,042\euro

\newpage

\item Κεφάλαιο 12000\euro\ τοκίστηκε στην τράπεζα Ζ με ετήσιο ανατοκισμό και επιτόκιο 8\% για διάστημα 3 ετών και 9 μηνών. Ποιος είναι ο τόκος του κεφαλαίου και ποιο ποσό θα καταβληθεί από την τράπεζα στον δικαιούχο αν η φορολογία καταθέσεων είναι 15\%?

\hspace{.15\textwidth}Λύση

\vspace{3cm}\hfill Απ: 15423,12\euro

\item Κεφάλαιο 12000\euro τοκίστηκε στην τράπεζα Ζ με ετήσιο ανατοκισμό και επιτόκιο 8\% για διάστημα 3 ετών, 9 μηνών και 18 ημερών. Ποιος είναι ο τόκος του κεφαλαίου και ποιο ποσό θα καταβληθεί από την τράπεζα στον δικαιούχο αν η φορολογία των καταθέσεων είναι 15\%?

\hspace{.15\textwidth}Λύση

\vspace{4cm}\hfill Απ: 16087,68\euro



\end{enumerate}

\end{document}