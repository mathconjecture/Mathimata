\documentclass[a4paper,table]{report} \input{preamble.tex}
\input{definitions2.tex}
\input{tikz}
\input{myboxes}

\input{insbox}

\linespread{1.2}

\newcommand{\twocolumnsidelcc}[2]{\begin{minipage}[c]{0.35\linewidth}
        #1
        \end{minipage}\hfill\begin{minipage}[c]{0.60\linewidth}
        #2
    \end{minipage}
}


\pagestyle{askhseis}
\usepackage{cutwin}
\everymath{\displaystyle}

\begin{document}

\chapter{Ακρότατα Συναρτήσεων}

\section{Τοπικά Ακρότατα}

\begin{mybox1}
\begin{dfn}
\item {}
  Έστω $ f \colon A \subseteq \mathbb{R}^{2} \to \mathbb{R} $, έστω 
  $ (x_{0}, y_{0}) \in A $ και έστω $R(x_{0}, y_{0}) $ περιοχή στοιχείων του $A$, 
  γύρω από το $ (x_{0}, y_{0}) $.
  \begin{enumerate}[i)]
    \item 
      Η $ f(x,y) $, έχει τοπικό ελάχιστο στο σημείο $ (x_{0}, y_{0}) $, αν 
      $ f(x_{0}, y_{0}) \leq f(x,y), \; \forall (x,y) \in R(x_{0}, y_{0}) $ 
    \item 
      Η $ f(x,y) $, έχει τοπικό μέγιστο στο σημείο $ (x_{0}, y_{0}) $, αν 
      $ f(x_{0}, y_{0}) \geq f(x,y), \; \forall (x,y) \in R(x_{0}, y_{0}) $ 
  \end{enumerate}
  Το τοπικά μέγιστο και το τοπικά ελάχιστο, ονομάζονται \textcolor{Col1}{τοπικά
  ακρότατα}. 
  Αν οι ανισότητες στον παραπάνω ορισμό, ισχύουν \textbf{για κάθε σημείο} στο πεδίο 
  ορισμού της συνάρτησης, τότε λέμε ότι η $f$ έχει \textcolor{Col1}{ολικό ακρότατο}.
\end{dfn}
\end{mybox1}

\begin{prop}\label{prop:fermat2}
\item {}
  Αν η συνάρτηση $ f(x,y) $ έχει τοπικό ακρότατο στο σημείο $ (x_{0}, y_{0}) $, 
  τότε:
  \begin{enumerate}[i)]
    \item ή υπάρχουν οι $ f_{x}(x_{0}, y_{0}) $ και $ f_{y}(x_{0}, y_{0}) $ 
      και ισχύει $ f_{x}(x_{0}, y_{0}) = f_{y}(x_{0}, y_{0} )=0 $
    \item ή μία τουλάχιστον από τις $ f_{x}(x_{0}, y_{0}) $ και 
      $ f_{y}(x_{0}, y_{0}) $ δεν υπάρχει.
  \end{enumerate}
\end{prop}

\begin{rem}
\item {}
  Το αντίστροφο της παραπάνω πρότασης δεν ισχύει. 
\end{rem}

\begin{dfn}
  Τα \textbf{εσωτερικά} σημεία του πεδίου ορισμού της $ f(x,y) $, για τα οποία 
  υπάρχουν οι μερικές παράγωγοι 1ης τάξης, και είναι ίσες με 0 ή που δεν υπάρχει 
  τουλάχιστον μία εξ αυτών, λέγονται \textcolor{Col1}{κρίσιμα} ή
  \textcolor{Col1}{στάσιμα} σημεία της $ f(x,y) $. 
\end{dfn}

\begin{rem}
  Σύμφωνα με την πρόταση~\ref{prop:fermat2} τα \textbf{κρίσιμα} σημεία της $ f(x,y) $, 
  μαζί με τα \textbf{συνοριακά} σημεία του πεδίου ορισμού της (τα οποία είναι επίσης  
  σημεία όπου δεν ορίζονται οι μερικές παράγωγοι 1ης τάξης) είναι θέσεις \textbf{πιθανών} ακροτάτων.
\end{rem}


\subsection{Σαγματικά Σημεία}

Όπως γνωρίζουμε για τις συναρτήσεις μιας μεταβλητής, ότι κάθε κρίσιμο σημείο δεν είναι 
αναγκαία τοπικό ακρότατο, γιατί μπορεί να είναι σημείο καμπής, έτσι κ για τις
συναρτήσεις δύο μεταβλητών, ένα κρίσιμο σημείο, μπορεί να είναι σαγματικό σημείο.

\begin{mybox1}
\begin{dfn}
  Ένα κρίσιμο σημείο $ (x_{0}, y_{0}) $, μιας διαφορίσιμης συνάρτησης $ f(x,y) $, είναι
  \textcolor{Col1}{σαγματικό σημείο}, αν για κάθε ανοιχτό δίσκο με κέντρο το 
  $ (x_{0}, y_{0}) $, υπάρχουν σημεία $ (x,y) $ στο πεδίο ορισμού της $f$, για τα 
  οποία, άλλοτε $ f(x,y) > f(x_{0}, y_{0}) $ κι άλλοτε $ f(x,y) < f(x_{0}, y_{0}) $. 
\end{dfn}
\end{mybox1}

\begin{example}
\item {}
  Έστω η συνάρτηση $ f(x,y) = y^{2}-x^{2} $. Έχουμε ότι 
  $ f_{x}=-2x $ και $ f_{y}=2y $, άρα το $ (0,0) $ είναι το μοναδικό στάσιμο σημείο 
  της $f$. 

  \twocolumnsidelcc{
    \begin{tikzpicture}[scale=0.6]
      \begin{axis}[blue!50,samples=30,xlabel={$y$},ylabel={$x$},zlabel={$f(x,y)$}]
        \addplot3[surf,color=Col1,opacity=0.5,domain=-2:2,faceted color=black] {x^2-y^2};
      \end{axis}
  \end{tikzpicture}
  }
  {
  Παρατηρούμε, ότι για όλα τα σημεία του άξονα $x$, άρα και για κάθε 
  $ (x,0) $ κοντά στο σημείο $(0,0)$, έχουμε ότι $ f(x,y) = -x^{2} < 0 $. 
  Όμως, για όλα τα σημεία του άξονα $ y $, άρα και για κάθε 
  $ (0,y) $ κοντά στο σημείο $ (0,0) $, έχουμε ότι $ f(x,y)=y^{2} > 0 $. 
  Άρα, σε κάθε 
  περιοχή του σημείου $ (0,0) $, πάντα θα βρίσκουμε τιμές της $ f(x,y) $ που είναι
  θετικές και αρνητικές. Άρα το σημείο $ (0,0) $ δεν μπορεί να είναι τοπικό ακρότατο 
  της $f$. Σε αυτή την περίπτωση, το σημείο $ (0,0) $ λέγεται \textbf{σαγματικό
  σημείο}, και μοιάζει με το σημείο στο κέντρο μιας σέλας, 
  όπως φαίνεται και στο σχήμα, γι᾽ αυτό και πολλές φορές χαρακτηρίζεται και ως 
  σελλοειδές σημείο.
}
\end{example}


\subsection{Θεωρήματα Ακροτάτων}


\begin{dfn}
  Έστω $ (x_{0}, y_{0}) $ κρίσιμο σημείο της $f$. Ορίζουμε τις παρακάτω ορίζουσες:
  \[
    \abs{H_{1}} = f_{xx}(x_{0}, y_{0}) \quad \text{και} \quad 
    \abs{H_{2}} = \begin{vmatrix}
      f_{xx} & f_{xy} \\
      f_{yx} & f_{yy}
    \end{vmatrix}_{(x_{0}, y_{0})}
  \] 
\end{dfn}

\begin{mybox2}
\begin{thm}
  \label{thm:2var}
\item {}
  Έστω $ f(x,y) $ συνάρτηση δύο μεταβλητών, ορισμένη σε ένα ανοιχτό 
  υποσύνολο $A$ του $ \mathbb{R}^{2} $, με μερικές παραγώγους 1ης και 2ης τάξης 
  ορισμένες σε μια  περιοχή του κρίσιμου σημείου $ (x_{0}, y_{0}) \in A $ και 
  έστω ότι οι παράγωγοι 2ης τάξης είναι συνεχείς στο $ (x_{0}, y_{0}) $. Τότε
\end{thm}

\begin{myitemize}
  \item Αν $ \abs{H_{1}} > 0 $ και $ \abs{H_{2}} > 0 $ τότε η $f$ παρουσιάζει στο 
    σημείο $ (x_{0}, y_{0}) $ \textbf{τοπικό ελάχιστο}.
  \item Αν $ \abs{H_{1}} < 0 $ και $ \abs{H_{2}} > 0 $ τότε η $f$ παρουσιάζει στο 
    σημείο $ (x_{0}, y_{0}) $ \textbf{τοπικό μέγιστο}.
  \item Αν $ \abs{H_{2}} < 0 $ τότε η $f$ δεν παρουσιάζει ακρότατο και σε αυτήν 
    την περίπτωση το σημείο λέγεται \textbf{σαγματικό}.
  \item Αν $ \abs{H_{2}} = 0 $, τότε δεν βγαίνει κάποιο συμπέρασμα σχετικά με το 
    σημείο $ (x_{0}, y_{0}) $.
\end{myitemize}
\end{mybox2}


\begin{rem}
  Όλες οι παραπάνω ορίζουσες ονομάζονται \textcolor{Col1}{Εσσιανές} και περιέχουν τις 
  παραγώγους 2ης τάξης της συνάρτησης. Οι αντίστοιχοι πίνακες, ονομάζονται 
  \textcolor{Col1}{Εσσιανοί} 
  και είναι \textbf{συμμετρικοί}, αφού από τις προϋποθέσεις του θεωρήματος ισχύει το 
  θεώρημα Schwartz και επομένως $ f_{xy} = f_{yx} $, $ f_{xz} = f_{zx} $, κλπ.
\end{rem}



\section{Μεθοδολογία Εύρεσης τοπικών ακροτάτων για συνάρτηση δύο μεταβλητών}

\begin{enumerate}
  \item Βρίσκουμε όλες τις μερικές παραγώγους 1ης και 2ης τάξης.
  \item Βρίσκουμε τα κρίσιμα σημεία της συνάρτησης λύνοντας το σύστημα των εξισώσεων: 
    $ \begin{rcases}
      f_{x} = 0 \\
      f_{y} = 0  
    \end{rcases} $
  \item Για κάθε κρίσιμο σημείο, έστω $ (x_{0}, y_{0}), $ υπολογίζουμε τις Εσσιανές 
    ορίζουσες σε αυτό το σημείο και εφαρμόζουμε το θεώρημα~\ref{thm:2var}.
\end{enumerate}



\section{Ολικά ή Απόλυτα Ακρότατα}

Ένα υποσύνολο $D$ του $ \mathbb{R}^{2} $ ονομάζεται \textcolor{Col1}{κλειστό}, αν 
περιέχει και όλα τα συνοριακά του σημεία, ενώ ονομάζεται \textcolor{Col1}{φραγμένο}, αν 
υπάρχει δίσκος του $ \mathbb{R}^{2} $ που το περιέχει, δηλαδή, ουσιαστικά ότι είναι 
περιορισμένο σε έκταση.

\begin{thm}
  Αν $f(x,y)$ είναι συνεχής σε κάποιο \textbf{κλειστό} και \textbf{φραγμένο} υποσύνολο 
  $A$ του $ \mathbb{R}^{2} $, τότε η $f$ παίρνει \textbf{μέγιστη} και \textbf{ελάχιστη} 
  τιμή στο $A$.  Δηλαδή, υπάρχουν σημεία $ (x_{1}, y_{1}) \in A $ και $ (x_{2}, y_{2}) 
  \in A $ τέτοια ώστε 
  \[
    f(x_{1}, y_{1}) \leq f(x,y) \leq f(x_{2}, y_{2}), \quad \forall (x,y) \in A
  \]
\end{thm}

Η μέγιστη και η ελάχιστη τιμή του προηγούμενου θεωρήματος, είναι τα \textbf{ολικά
ακρότατα} της συνάρτησης, και για να τα προσδιορίσουμε ακολουθούμε τα παρακάτω βήματα.

\begin{myitemize}
  \item Βρίσκουμε τις τιμές των \textbf{κρίσιμων} σημείων της $f$ στο $A$.
  \item Βρίσκουμε τις τιμές των \textbf{ακροτάτων} της $f$ στο σύνορο του $A$. 
  \item Η μεγαλύτερη από τις τιμές που βρήκαμε στα δύο προηγούμενα βήματα 
    είναι το ολικό μέγιστο της συνάρτησης, ενώ η μικρότερη είναι το ολικό ελάχιστο.
\end{myitemize}






\chapter{Ακρότατα Υπό Συνθήκη}

\section{Πολλαπλασιαστές Lagrange}

\subsection{Ακρότατα υπό μία συνθήκη}

\begin{thm}
  Έστω $ f(x,y,z) $ συνάρτηση τριών μεταβλητών, ορισμένη σε ένα ανοιχτό 
  υποσύνολο $A$ του $ \mathbb{R}^{3} $ και έστω ότι οι παράγωγοι 2ης τάξης είναι 
  συνεχείς στο $A$. Υποθέτουμε ότι για την συνάρτηση $ \phi $ οι μερικές παράγωγοι 
  1ης τάξης είναι συνεχείς στο $A$ και ικανοποιείται η συνθήκη 
  \begin{equation}
    \label{eq:constr1}
    \phi (x,y,z) = 0
  \end{equation}
  Θεωρούμε τη συνάρτηση Lagrange που ορίζεται από τον τύπο
  \[
    L(x,y,z, \lambda) = f(x,y,z) + \lambda \phi (x,y,z) 
  \] 
  και τις ορίζουσες
  \[
    \abs{\overline{H}_{1}} = 
    \begin{vmatrix}
      L_{xx} & L_{xy} & L_{xz} & {\phi}_{x} \\
      L_{yx} & L_{yy} & L_{yz} & {\phi}_{y} \\
      L_{zx} & L_{zy} & L_{zz} & {\phi}_{z} \\
      {\phi}_{x} & {\phi}_{y} & {\phi}_{z} & 0
    \end{vmatrix}, \quad 
    \abs{\overline{H}_{2}} = 
    \begin{vmatrix}
      L_{yy} & L_{yz} & {\phi}_{y} \\
      L_{zy} & L_{zz} & {\phi}_{z} \\
      {\phi}_{y} & {\phi}_{z} & 0
    \end{vmatrix}, \quad 
  \] 
  Υποθέτουμε ότι $ (x_{0}, y_{0}, z_{0}, \lambda) $ είναι μια 
  λύση του συστήματος 
  \[
    \left.
      \begin{matrix}
        L_{x} = 0 \\
        L_{y} = 0 \\
        L_{z} = 0 \\
        L_{\lambda} = 0 \\
      \end{matrix}
    \right\} 
  \]
  Τότε:
  \begin{myitemize}
    \item Αν $ \abs{\overline{H}_{1}} < 0 $ και $ \abs{\overline{H}_{2}} < 0 $ 
      στο σημείο $ (x_{0}, y_{0}, z_{0}, \lambda) $, τότε η $f$ 
      παρουσιάζει τοπικό ελάχιστο στο $ (x_{0}, y_{0}, z_{0}) $ υπό τη
      συνθήκη του περιορισμού~\eqref{eq:constr1}.
    \item Αν $ \abs{\overline{H}_{1}} < 0 $ και $ \abs{\overline{H}_{2}} > 0 $ 
      στο σημείο $ (x_{0}, y_{0}, z_{0}, \lambda) $, τότε η $f$ 
      παρουσιάζει τοπικό μέγιστο στο $ (x_{0}, y_{0}, z_{0}) $ υπό τη
      συνθήκη του περιορισμού~\eqref{eq:constr1}.
  \end{myitemize}
\end{thm}


\end{document}

