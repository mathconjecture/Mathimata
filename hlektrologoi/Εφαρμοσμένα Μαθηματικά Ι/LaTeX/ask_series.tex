\documentclass[a4paper,12pt]{article}
\usepackage{etex}
%%%%%%%%%%%%%%%%%%%%%%%%%%%%%%%%%%%%%%
% Babel language package
\usepackage[english,greek]{babel}
% Inputenc font encoding
\usepackage[utf8]{inputenc}
%%%%%%%%%%%%%%%%%%%%%%%%%%%%%%%%%%%%%%

%%%%% math packages %%%%%%%%%%%%%%%%%%
\usepackage{amsmath}
\usepackage{amssymb}
\usepackage{amsfonts}
\usepackage{amsthm}
\usepackage{proof}

\usepackage{physics}

%%%%%%% symbols packages %%%%%%%%%%%%%%
\usepackage{bm} %for use \bm instead \boldsymbol in math mode 
\usepackage{dsfont}
\usepackage{stmaryrd}
%%%%%%%%%%%%%%%%%%%%%%%%%%%%%%%%%%%%%%%


%%%%%% graphicx %%%%%%%%%%%%%%%%%%%%%%%
\usepackage{graphicx}
\usepackage{color}
%\usepackage{xypic}
\usepackage[all]{xy}
\usepackage{calc}
\usepackage{booktabs}
\usepackage{minibox}
%%%%%%%%%%%%%%%%%%%%%%%%%%%%%%%%%%%%%%%

\usepackage{enumerate}

\usepackage{fancyhdr}
%%%%% header and footer rule %%%%%%%%%
\setlength{\headheight}{14pt}
\renewcommand{\headrulewidth}{0pt}
\renewcommand{\footrulewidth}{0pt}
\fancypagestyle{plain}{\fancyhf{}
\fancyhead{}
\lfoot{}
\rfoot{\small \thepage}}
\fancypagestyle{vangelis}{\fancyhf{}
\rhead{\small \leftmark}
\lhead{\small }
\lfoot{}
\rfoot{\small \thepage}}
%%%%%%%%%%%%%%%%%%%%%%%%%%%%%%%%%%%%%%%

\usepackage{hyperref}
\usepackage{url}
%%%%%%% hyperref settings %%%%%%%%%%%%
\hypersetup{pdfpagemode=UseOutlines,hidelinks,
bookmarksopen=true,
pdfdisplaydoctitle=true,
pdfstartview=Fit,
unicode=true,
pdfpagelayout=OneColumn,
}
%%%%%%%%%%%%%%%%%%%%%%%%%%%%%%%%%%%%%%

\usepackage[space]{grffile}

\usepackage{geometry}
\geometry{left=25.63mm,right=25.63mm,top=36.25mm,bottom=36.25mm,footskip=24.16mm,headsep=24.16mm}

%\usepackage[explicit]{titlesec}
%%%%%% titlesec settings %%%%%%%%%%%%%
%\titleformat{\chapter}[block]{\LARGE\sc\bfseries}{\thechapter.}{1ex}{#1}
%\titlespacing*{\chapter}{0cm}{0cm}{36pt}[0ex]
%\titleformat{\section}[block]{\Large\bfseries}{\thesection.}{1ex}{#1}
%\titlespacing*{\section}{0cm}{34.56pt}{17.28pt}[0ex]
%\titleformat{\subsection}[block]{\large\bfseries{\thesubsection.}{1ex}{#1}
%\titlespacing*{\subsection}{0pt}{28.80pt}{14.40pt}[0ex]
%%%%%%%%%%%%%%%%%%%%%%%%%%%%%%%%%%%%%%

%%%%%%%%% My Theorems %%%%%%%%%%%%%%%%%%
\newtheorem{thm}{Θεώρημα}[section]
\newtheorem{cor}[thm]{Πόρισμα}
\newtheorem{lem}[thm]{λήμμα}
\theoremstyle{definition}
\newtheorem{dfn}{Ορισμός}[section]
\newtheorem{dfns}[dfn]{Ορισμοί}
\theoremstyle{remark}
\newtheorem{remark}{Παρατήρηση}[section]
\newtheorem{remarks}[remark]{Παρατηρήσεις}
%%%%%%%%%%%%%%%%%%%%%%%%%%%%%%%%%%%%%%%




\newcommand{\vect}[2]{(#1_1,\ldots, #1_#2)}
%%%%%%% nesting newcommands $$$$$$$$$$$$$$$$$$$
\newcommand{\function}[1]{\newcommand{\nvec}[2]{#1(##1_1,\ldots, ##1_##2)}}

\newcommand{\linode}[2]{#1_n(x)#2^{(n)}+#1_{n-1}(x)#2^{(n-1)}+\cdots +#1_0(x)#2=g(x)}

\newcommand{\vecoffun}[3]{#1_0(#2),\ldots ,#1_#3(#2)}

\newcommand{\mysum}[1]{\sum_{n=#1}^{\infty}


\pagestyle{empty}

\begin{document}

\begin{center}
\fbox{\large\bfseries Ασκήσεις Στη Μέθοδο Σειρών}
\end{center}

\vspace{\baselineskip}

\begin{enumerate}
\item Να εξετάσετε τι είδους σημείο είναι το $x_0=0$ για τις παρακάτω εξισώσεις:
\begin{enumerate}[i)]
\item $x^3y''+\sin(2x)y=0, y=y(x)$\hfill Απ: κανονικό ιδιάζον
\item $x^4y'' + (\cos x-1)y'=0, y=y(x)$\hfill Απ: μη-κανονικό ιδιάζον
\item $(x-x^2)y''-(1+2x)y'+2y=0, y=y(x)$\hfill Απ: κανονικό ιδιάζον
\item $x^{a}y''+\sin x y=0, y=y(x), a=1, a=4$\hfill Απ: \{\minibox[]{$a=1$, κανονικό ιδιάζον\\ 
$a=4$, μη-κανονικό ιδιάζον}
\end{enumerate}

\item Για τις παρακάτω εξισώσεις να βρεθούν δύο γραμμικώς ανεξάρτητες λύσεις, υπό μορφή σειράς γύρω από το σημείο $x_0=0$.
\begin{enumerate}[i)]
\item $y''-xy'+y=0$\hfill Απ: \begin{tabular}[t]{l}
$y_1=1-\frac{1}{2}x^2+\sum\limits_{n=2}^{\infty}(-1)\frac{(2n-3)!!}{(2n)!}x^{2n}$\\
$y_2=x$
\end{tabular}
\item $y''+x^2y'-4xy=0$\hfill Απ:\begin{tabular}[t]{l}$y_1=1+\frac{2}{3}x^3+\frac{1}{4\cdot 5}x^6-\frac{1}{16\cdot 20}x^9+\cdots$ \\[5pt]
$y_2=x+\frac{1}{4}x^4$
\end{tabular}
\item $y''-xy'+y=5, y(0)=5, y'(0)=3$\hfill Απ: \begin{tabular}[t]{l}
$y(x)=5+3x$
\end{tabular}
\item $(x-x^2)y''-(1+2x)y'+2y=0$\hfill Απ: \begin{tabular}[t]{l}
$y_1=1+2x$\\
$y_2=\sum\limits_{n=2}^{\infty}\frac{n+1}{3}x^n$
\end{tabular}
\item $xy''+y=0$\hfill Απ: \begin{tabular}[t]{l}
$y_1=-x+\sum\limits_{n=2}^{\infty}(-1)^n\frac{n}{(n!)^2}x^n$\\
$y_2=y_1\ln x+1+x+\sum\limits_{n=2}^{\infty}(-1)^{n+1}\frac{n}{(n!)^2}(2H_{n-1}+\frac{1}{n})x^n$
\end{tabular}
\item $xy''+2y'-xy=0$\hfill Απ: \begin{tabular}[t]{l}
$y_1=\frac{\cosh x}{x}$\\
$y_2=\frac{\sinh x}{x}$
\end{tabular}
\item $xy''-2y'+xy=0$\hfill Απ: \begin{tabular}[t]{l}
$y_1=1+\sum\limits_{n=1}^{\infty}\frac{(-1)^{n+1}}{2n(2n-2)!}x^{2n}$\\
$y_2=\sum\limits_{n=2}^{\infty}\frac{(-1)^n}{(2n-1)(2n-3)!}x^{2n-1}$
\end{tabular}


\end{enumerate}

\end{enumerate}


\end{document}