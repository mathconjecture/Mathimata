\documentclass[a4paper,12pt]{article}
\usepackage{etex}
%%%%%%%%%%%%%%%%%%%%%%%%%%%%%%%%%%%%%%
% Babel language package
\usepackage[english,greek]{babel}
% Inputenc font encoding
\usepackage[utf8]{inputenc}
%%%%%%%%%%%%%%%%%%%%%%%%%%%%%%%%%%%%%%

%%%%% math packages %%%%%%%%%%%%%%%%%%
\usepackage{amsmath}
\usepackage{amssymb}
\usepackage{amsfonts}
\usepackage{amsthm}
\usepackage{proof}

\usepackage{physics}

%%%%%%% symbols packages %%%%%%%%%%%%%%
\usepackage{bm} %for use \bm instead \boldsymbol in math mode 
\usepackage{dsfont}
\usepackage{stmaryrd}
%%%%%%%%%%%%%%%%%%%%%%%%%%%%%%%%%%%%%%%


%%%%%% graphicx %%%%%%%%%%%%%%%%%%%%%%%
\usepackage{graphicx}
\usepackage{color}
%\usepackage{xypic}
\usepackage[all]{xy}
\usepackage{calc}
\usepackage{booktabs}
\usepackage{minibox}
%%%%%%%%%%%%%%%%%%%%%%%%%%%%%%%%%%%%%%%

\usepackage{enumerate}

\usepackage{fancyhdr}
%%%%% header and footer rule %%%%%%%%%
\setlength{\headheight}{14pt}
\renewcommand{\headrulewidth}{0pt}
\renewcommand{\footrulewidth}{0pt}
\fancypagestyle{plain}{\fancyhf{}
\fancyhead{}
\lfoot{}
\rfoot{\small \thepage}}
\fancypagestyle{vangelis}{\fancyhf{}
\rhead{\small \leftmark}
\lhead{\small }
\lfoot{}
\rfoot{\small \thepage}}
%%%%%%%%%%%%%%%%%%%%%%%%%%%%%%%%%%%%%%%

\usepackage{hyperref}
\usepackage{url}
%%%%%%% hyperref settings %%%%%%%%%%%%
\hypersetup{pdfpagemode=UseOutlines,hidelinks,
bookmarksopen=true,
pdfdisplaydoctitle=true,
pdfstartview=Fit,
unicode=true,
pdfpagelayout=OneColumn,
}
%%%%%%%%%%%%%%%%%%%%%%%%%%%%%%%%%%%%%%

\usepackage[space]{grffile}

\usepackage{geometry}
\geometry{left=25.63mm,right=25.63mm,top=36.25mm,bottom=36.25mm,footskip=24.16mm,headsep=24.16mm}

%\usepackage[explicit]{titlesec}
%%%%%% titlesec settings %%%%%%%%%%%%%
%\titleformat{\chapter}[block]{\LARGE\sc\bfseries}{\thechapter.}{1ex}{#1}
%\titlespacing*{\chapter}{0cm}{0cm}{36pt}[0ex]
%\titleformat{\section}[block]{\Large\bfseries}{\thesection.}{1ex}{#1}
%\titlespacing*{\section}{0cm}{34.56pt}{17.28pt}[0ex]
%\titleformat{\subsection}[block]{\large\bfseries{\thesubsection.}{1ex}{#1}
%\titlespacing*{\subsection}{0pt}{28.80pt}{14.40pt}[0ex]
%%%%%%%%%%%%%%%%%%%%%%%%%%%%%%%%%%%%%%

%%%%%%%%% My Theorems %%%%%%%%%%%%%%%%%%
\newtheorem{thm}{Θεώρημα}[section]
\newtheorem{cor}[thm]{Πόρισμα}
\newtheorem{lem}[thm]{λήμμα}
\theoremstyle{definition}
\newtheorem{dfn}{Ορισμός}[section]
\newtheorem{dfns}[dfn]{Ορισμοί}
\theoremstyle{remark}
\newtheorem{remark}{Παρατήρηση}[section]
\newtheorem{remarks}[remark]{Παρατηρήσεις}
%%%%%%%%%%%%%%%%%%%%%%%%%%%%%%%%%%%%%%%




\newcommand{\vect}[2]{(#1_1,\ldots, #1_#2)}
%%%%%%% nesting newcommands $$$$$$$$$$$$$$$$$$$
\newcommand{\function}[1]{\newcommand{\nvec}[2]{#1(##1_1,\ldots, ##1_##2)}}

\newcommand{\linode}[2]{#1_n(x)#2^{(n)}+#1_{n-1}(x)#2^{(n-1)}+\cdots +#1_0(x)#2=g(x)}

\newcommand{\vecoffun}[3]{#1_0(#2),\ldots ,#1_#3(#2)}

\newcommand{\mysum}[1]{\sum_{n=#1}^{\infty}


\everymath{\displaystyle}
\thispagestyle{askhseis}

\begin{document}

\begin{center}
  \minibox{\large\bfseries \textcolor{Col1}{Μήκος Καμπύλης}}
\end{center}

\vspace{\baselineskip}



\section*{Συνάρτηση}


\begin{enumerate}
  \item Να υπολογιστεί το μήκος του τόξου της καμπύλης με εξίσωση $ y =
    \ln{\frac{e^{x} + 1}{e^{x} - 1}}$, από $ x_{1} = a $ εως $ x_{2} = b $.

    \hfill Απ: $ \ln{\frac{e^{b} - e^{-b}}{e^{a} - e^{-a}}} $

  \item Να υπολογιστεί το μήκος της καμπύλης $y=\ln(\cos x)$ όταν 
    $\frac{\pi}{6}\leq x \leq \frac{\pi}{3}$.

    \hfill Απ: $\ln\frac{2+\sqrt{3}}{\sqrt{3}}$

  \item Να υπολογιστεί το μήκος της καμπύλης $y=\frac{e^{x}+e^{-x}}{2}$, από $x=0$ έως 
    $x=1$.

    \hfill Απ: $\frac{e^{2}-1}{2e}$
\end{enumerate}


\section*{Πεπλεγμένες}


\begin{enumerate}
  \item Να υπολογιστεί το μήκος της αστροειδούς καμπύλης $ x^{\frac{2}{3}}
    + y^{\frac{2}{3}} = a^{\frac{2}{3}} $.

    \hfill Απ: 6a

    % \item Να υπολογιστεί το μήκος της καμπύλης που βρίσκεται στο 1ο και 4ο
    %   τεταρτημόριο και ορίζεται από τις καμπύλες $ y^{2} = 2x^{3} $ και $ x^{2} +
    %   y^{2} = 20 $.

    %   \hfill Απ: $ \frac{8}{27} \left(10 \sqrt{10} -1\right) + 2 \sqrt{20}
    %   \left(\frac{\pi}{2} - \arcsin\Bigl(\frac{\sqrt{5}}{5}\Bigr)\right) $ 

  \item Να υπολογιστεί το μήκος του τόξου της καμπύλης $ \frac{1}{4} y^{2} = x +
    \frac{1}{2} \ln{y} $ που περιορίζεται από τις ευθείες $ y = 1 $ και $ y =
    2$.

    \hfill Απ: $ \frac{3}{4} + \frac{\ln{2}}{2} $
\end{enumerate}


\section*{Παραμετρικές}


\begin{enumerate}
  \item Να υπολογιστεί το μήκος της καμπύλης $x=2-\ln(1+t^{2})$, $y=1+\arccos t$ για 
    $0\leq t\leq 1$.

    \hfill Απ: $\ln(\sqrt{2}+1)$

  \item Να υπολογιστεί το μήκος ενός τόξου της κυκλοειδούς καμπύλης με παραμετρικές 
    εξισώσεις $ x = a(t - \sin{t}) $ και $ y = a(1- \cos{t}) $.

    \hfill Απ: $ 8a $ 

  \item Να υπολογιστεί το μήκος του βρόγχου της καμπύλης με παραμετρικές εξισώσεις 
    $ x= \sqrt{3} t^{2} $ και $ y = t - t^{3} $.

    \hfill Απ: $ 4 $ 

  \item Να υπολογιστεί το μήκος της έλικα με παραμετρικές εξισώσεις 
    $ x=a \cos{(\omega t)} $, $ y=a \sin{(\omega t)} $ και $ z= bt $, με $ a>0 $ 
    και $ b, \omega \in \mathbb{R} $ στο διάστημα $ [0.2 \pi] $.

    \hfill Απ: $ \sqrt{a^{2} \omega ^{2} + b^{2}} $  
  \item Να υπολογιστεί το μήκος της καμπύλης $ x = e^{t} \sin{t} $, $ y = e^{t}
    \cos{t} $, για $ t \in [0, \pi] $.

    \hfill Απ: $ \sqrt{2} (e^{\pi} - 1)  $

  \item Να υπολογιστεί το μήκος της καμπύλης $x=1+\sin t$, $y=2+\cos t$, όταν 
    $0\leq t\leq \pi$.

    \hfill Απ: $4$

  \item Η θέση ενός κινητού τη χρονική στιγμή $t$ δίνεται από τις σχέσεις 
    $x=1+t^{3}$, και  $y=2-t^{2}$. Να βρεθεί η απόσταση που θα διανύσει αν ταξιδεύει 
    από $t=0$ έως $t=2$.

    \hfill Απ: $\frac{8}{27}(10\sqrt{10}-1)$
\end{enumerate}



\end{document}
